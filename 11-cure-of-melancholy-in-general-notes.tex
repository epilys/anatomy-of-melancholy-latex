\setauthornote{2789}{Consil. 234. pro Abbate Italo.}
\setauthornote{2790}{Consil. 23. aut curabitur, aut certe minus afficietur, si volet.}
\setauthornote{2791}{Vide Renatum Morey Animad. in scholam Salernit, c. 38. si ad 40. annos possent producere vitam, cur non ad centum? si ad centum, cur non ad mille?}
\setauthornote{2792}{Hist. Chinensum.}
\setauthornote{2793}{Alii dubitant an daemon possit morbus curare quos non fecit, alii negant, sed quotidiana experientia confirmat, magos magno multorum stupore morbos curare, singulas corporis parte citra impedimentum permeare, et mediis nobis ignotis curare.}
\setauthornote{2794}{Agentia cum patientibus conjugant.}
\setauthornote{2795}{Cap. 11. de Servat.}
\setauthornote{2796}{Haec alii rident, sed vereor ne dum nolumus esse creduli, vitium non efugiamus incredulitatis.}
\setauthornote{2797}{Refert Solomonem mentis morbos curasse, et daemones abegisse ipsos carminibus, quod et coram Vespasiano fecit Eleazar.}
\setauthornote{2798}{Spirituales morbi spiritualiter curari debent.}
\setauthornote{2799}{Sigillum ex auro peculiari ad Melancholiam, \&c.}
\setauthornote{2800}{Lib. 1. de occult. Philos. nihil refert an Deus an diabolus, angeli an immundi spiritus aegro opem ferant, morbus curetur.}
\setauthornote{2801}{Magus minister et Vicarius Dei.}
\setauthornote{2802}{Utere forti imaginatione et experieris effectum, dicant in adversum quicquid volunt Theologi.}
\setauthornote{2803}{Idem Plinius contendit quosdam esse morbos qui incantationibus solum curentur.}
\setauthornote{2804}{Qui talibus credunt, aut ad eorum domos euntes, aut suis domibus introducunt, aut interrogant, sciant se fidem Christianam et baptismum praevaricasse, et Apostatas esse. Austin de superstit. observ. hoc pacto a Deo deficitur ad diabolum, P. Mart.}
\setauthornote{2805}{Mori praestat quam superstitiose sanari, Disquis. mag. l. 2. c. 2. sect. 1. quaest. 1. Tom. 3.}
\setauthornote{2806}{P. Lumbard.}
\setauthornote{2807}{Suffitus, gladiorum ictus, \&c.}
\setauthornote{2808}{The Lord hath created medicines of the earth, and he that is wise will not abhor them, Ecclus. xxxviii. 4.}
\setauthornote{2809}{My son, fail not in thy sickness, but pray unto the Lord, and he will make thee whole, Ecclus. xxxviii. 9.}
\setauthornote{2810}{Huc omne principium, huc refer exitum. Hor. 3. carm. Od. 6.}
\setauthornote{2811}{Music and fine fare can do no good.}
\setauthornote{2812}{Hor. l. 1. ep. 2.}
\setauthornote{2813}{Sint Craesi et Crassi licet, non hos Pactolus aureas undas agens eripiet unquam e miseriis.}
\setauthornote{2814}{Scientia de Deo debet in medico infixa esse, Mesue Arabs. Sanat omnes languores Deus. For you shall pray to your Lord, that he would prosper that which is given for ease, and then use physic for the prolonging of life, Ecclus. xxxviii. 4.}
\setauthornote{2815}{27 Omnes optant quandam in medicina felicitatem, sed hanc non est quod expectent, nisi deum vera fide invocent, atque regros similiter ad ardentem vocationem excitent.}
\setauthornote{2816}{28 Lemnius e Gregor. exhor. ad vitam opt. instit. cap. 48. Quicquid meditaris aggredi aut perficere. Deum in consilium adhibeto.}
\setauthornote{2817}{Commentar. lib. 7. ob infelicem pugnam contristatus, in aegritudinem incidit, ita ut a medicis curari non posset.}
\setauthornote{2818}{In his animi malis princeps imprimis ad Deum precetur, et peccatis veniam exoret, inde ad medicinam, \&c.}
\setauthornote{2819}{Greg. Tholoss. To. 2. l. 28. c. 7. Syntax. In vestibule templi Solomon, liber remediorum cujusque morbi fuit, quem revulsit Ezechias, quod populus neglecto Deo nec invocato, sanitatem inde peteret.}
\setauthornote{2820}{Livius l. 23. Strepunt aures clamoribus plorantium sociorum, saepius nos quam deorum invocantium opem.}
\setauthornote{2821}{Rulandus adjungit optimam orationem ad finem Empyricorum. Mercurialis consil. 25. ita concludit. Montanus passim, \&c. et plures alii, \&c.}
\setauthornote{2822}{Lipsius.}
\setauthornote{2823}{Cap. 26.}
\setauthornote{2824}{Lib. 2. cap. 7. de Deo Morbisque in genera descriptis deos reperimus.}
\setauthornote{2825}{Selden prolog. cap. 3. de diis Syris. Rofinus.}
\setauthornote{2826}{See Lilii Giraldi syntagma de diis, \&c.}
\setauthornote{2827}{12 Cal. Januarii ferias celebrant, ut angores et animi solicitudines propitiata depellat.}
\setauthornote{2828}{Hanc divae pennam consecravi, Lipsius.}
\setauthornote{2829}{Jodocus Sincerus itin. Galliae. 1617. Huc mente captos deducunt, et statis orationibus, sacrisque peractis, in illum lectum dormitum ponunt, \&c.}
\setauthornote{2830}{In Gallia Narbonensi.}
\setauthornote{2831}{Lib. de orig. Festorum. Collo suspensa et pergameno inscripta, cum signo crucis, \&c.}
\setauthornote{2832}{Em. Acosta com. rerum in Oriente gest. a societat. Jesu, Anno 1568. Epist. Gonsalvi Fernandis, Anno 1560. e Japonia.}
\setauthornote{2833}{Spicel. de morbis daemoniacis, sic a sacrificulis parati unguentis Magicis corpori illitis, ut stultae plebeculae persuadeant tales curari a Sancto Antonio.}
\setauthornote{2834}{Printed at London 4'to by J. Roberts. 1605.}
\setauthornote{2835}{Greg. lib. 8. Cujus fanum aegrotantium multitudine refertum, undiquaque et tabellis pendentibus, in quibus sanati languores erant inscripti.}
\setauthornote{2836}{To offer the sailors' garments to the deity of the deep.}
\setauthornote{2837}{Mali angeli sumpserunt olim nomen Jovis, Junonis, Apollinis, \&c. quos Gentiles deos credebant, nunc S. Sebastiani, Barbarae, \&c. nomen habent, et aliorum.}
\setauthornote{2838}{Part. 2, cap. 9. de spect. Veneri substituunt Virginem Mariam.}
\setauthornote{2839}{Ad haec ludibria Deus connivet frequentur, ubi relicto verbo Dei, ad Satanam curritur, quales hi sunt, qui aquam lustralem, crucem, \&c. lubricae fidei hominibus offerunt.}
\setauthornote{2840}{Charior est ipsis homo quam sibi, Paul.}
\setauthornote{2841}{Bernard.}
\setauthornote{2842}{Austin.}
\setauthornote{2843}{Ecclus. xxxviii. In the sight of great men he shall be in admiration.}
\setauthornote{2844}{Tom. 4. Tract. 3. de morbis amentium, horum multi non nisi a Magis curandi et Astrologis, quoniam origo ejus a coelis petenda est.}
\setauthornote{2845}{Lib. de Podagra.}
\setauthornote{2846}{Sect. 5.}
\setauthornote{2847}{Langius. J. Caesar Claudinus consult.}
\setauthornote{2848}{Praedestinatum ad hunc curandum.}
\setauthornote{2849}{Helleborus curat, sed quod ab omni datus medico vanum est.}
\setauthornote{2850}{Antid. gen. lib. 3. cap. 2.}
\setauthornote{2851}{The leech never releases the skin until he is filled with blood.}
\setauthornote{2852}{Quod saepe evenit, lib. 3. cap. 2. cum non sit necessitas. Frustra fatigant remediis aegros, qui victus ratione curari possunt, Heurnius.}
\setauthornote{2853}{Modestus et sapiens medicus, nunquam properabit ad pharmacum, nisi cogente necessitate, 41 Aphor. prudens et pius medicus cibis prius medicinal, quam medicinis puris morbum expellere satagat.}
\setauthornote{2854}{Brev. 1. c. 18.}
\setauthornote{2855}{Similitudo saepe bonis modicis imponit.}
\setauthornote{2856}{Qui melancholicis praebent remedia non satis valida Longiores morbi imprimis solertiam medici postulant et fidelitatem, qui enim tumultuario hos tractant, vires absque ullo commodo laedunt et frangunt, \&c.}
\setauthornote{2857}{Naturae remissionem dare oportet.}
\setauthornote{2858}{Plerique hoc morbo medicina nihil profecisse visi sunt, et sibi demissi invaluerunt.}
\setauthornote{2859}{Abderitani ep. Hippoc.}
\setauthornote{2860}{Quicquid auri apud nos est, libenter persolvemus, etiamsi tota urbs nostra aurum esset.}
\setauthornote{2861}{Seneca.}
\setauthornote{2862}{Per. 3. Sat.}
\setauthornote{2863}{De anima. Barbara tamen immanitate, et deploranda inscitia contemnunt praecepta sanitatis mortem et morbos ultro accersunt.}
\setauthornote{2864}{Consul. 173. e Scoltzio Melanch. Aegrorum hoc fere proprium est, ut graviora dicant esse symptomata, quam revera sunt.}
\setauthornote{2865}{Melancholici plerumque medicis sunt molesti, ut alia aliis adjungant.}
\setauthornote{2866}{Oportet infirmo imprimere salutem, utcunque promittere, etsi ipse desperet. Nullum medicamentum efficax, nisi medicus etiam fuerit fortis imaginationis.}
\setauthornote{2867}{De promise, doct. cap. 15. Quoniam sanitatis formam animi medici continent.}
\setauthornote{2868}{Spes et confidentia, plus valent quam medicina.}
\setauthornote{2869}{Felicior in medicina ob fidem Ethnicorum.}
\setauthornote{2870}{Aphoris. 89. Aeger qui plurimos consulit medicos, plerumque in errorem singulorum cadit.}
\setauthornote{2871}{Nihil ita sanitatem impedit, ac remediorum crebra mutatio, nec venit vulnus ad cicatricem in quo diversa medicamenta tentantur.}
\setauthornote{2872}{Melancholicorum proprium, quum ex eorum arbitrio non fit subita mutatio in melius, alterare medicos qui quidvis, \&c.}
\setauthornote{2873}{Consil. 31. Dum ad varia se conferunt, nullo prosunt.}
\setauthornote{2874}{Imprimis hoc statuere oportet, requiri perseverantiam, et tolerantiam. Exiguo enim tempore nihil ex, \&c.}
\setauthornote{2875}{Si curari vult, opus est pertinaci perseverantia, fideli obedientia, et patientis singulari, si taedet aut desperet, nullum habebit effectum.}
\setauthornote{2876}{Aegritudine amittunt patientiam, et inde morbi incurabiles.}
\setauthornote{2877}{Non ad mensem aut annum, sed opportet toto vitae curriculo curationi operam dare.}
\setauthornote{2878}{Camerarius emb. 55. cent. 2.}
\setauthornote{2879}{Praefat. de nar. med. In libellis quae vulgo versantur apud literatos, incautiores multa legunt, a quibus decipiuntur, eximia illis, sed portentosum hauriunt venenum.}
\setauthornote{2880}{Operari ex libris, absque cognitione et solerti ingenio, periculosum est. Unde monemur, quam insipidum scriptis auctoribus credere, quod hic suo didicit periculo.}
\setauthornote{2881}{Consil. 23. haec omnia si quo ordine decet, egerit, vel curabitur, vel certe minus afficietur.}
\setauthornote{2882}{Fuchsius cap. 2. lib. 1.}
\setauthornote{2883}{In pract. med. haec affectio nostris temporibus frequentissima, ergo maxime pertinet ad nos hujus curationem intelligere.}
\setauthornote{2884}{Si aliquis horum morborum, summus sanatur, sanantur omnes inferiores.}
\setauthornote{2885}{Instit. cap. 8. sect. 1. Victus nomine non tam cibus et potus, sed aer, exercitatio, somnus, vigilia, et reliquae res sex non-naturales contineritur.}
\setauthornote{2886}{Sufficit plerumque regimen rerum sex non-naturalium.}
\setauthornote{2887}{Et in his potissima sanitas consistit.}
\setauthornote{2888}{Nihil hic agendum sine exquisita vivendi ratione, \&c.}
\setauthornote{2889}{Si recens malum sit ad pristinum habitum recuperandum, alia medela non est opus.}
\setauthornote{2890}{Consil. 99. lib. 2. si celsitudo tua, rectam victus rationem, \&c.}
\setauthornote{2891}{Moneo Domine, ut sis prudens ad victum, sine quo caetera remedia frustra adhibentur.}
\setauthornote{2892}{Omnia remedia irrita et vana sine his. Novistis me plerosque ita laborantes, victu potius quam medicamentis curasse.}
\setauthornote{2893}{When you are again lean, seek an exit through that hole by which lean you entered.}
\setauthornote{2894}{l. de finibus Tarentinis et Siculis.}
\setauthornote{2895}{Modo non multum elongentur.}
\setauthornote{2896}{Lib. 1. de melan. cap. 7. Calidus et humidus cibus concoctu, facilis, flatus exortes, elixi non assi, neque sibi frixi sint.}
\setauthornote{2897}{Si interna tantum pulpa devoretur, non superficies torrida ab igne.}
\setauthornote{2898}{Bene nutrientes cibi, tenella aetas multum valet, carnes non virosae, nec pingues.}
\setauthornote{2899}{Hoedoper. peregr. Hierosol.}
\setauthornote{2900}{Inimica stomacho.}
\setauthornote{2901}{Not fried or buttered, but poached.}
\setauthornote{2902}{Consil. 16. Non improbatur butyrum et oleum, si tamen plus quam par sit, non profundatur: sacchari et mellis usus, utiliter ad ciborum condimenta comprobatur.}
\setauthornote{2903}{Mercurialis consil. 88. acerba omnia evitantur.}
\setauthornote{2904}{Ovid. Met. lib. 15. Whoever has allayed his thirst with the water of the Clitorius, avoids wine, and abstemious delights in pure water only.}
\setauthornote{2905}{Pregr. Hier.}
\setauthornote{2906}{The Dukes of Venice were then permitted to marry.}
\setauthornote{2907}{De Legibus.}
\setauthornote{2908}{Lib. 4. cap. 10. Magna urbis utilitas cum perennes fontes muris includuntur, quod si natura non praestat, effondiendi, \&c.}
\setauthornote{2909}{Opera gigantum dicit aliquis.}
\setauthornote{2910}{De aquaeduct.}
\setauthornote{2911}{Curtius Fons a quadragesimo lapide in urbem opere arcuato perductus. Plin. 36. 15.}
\setauthornote{2912}{Quaeque domus Romae fistulas habebat et canales, \&c.}
\setauthornote{2913}{Lib. 2. ca. 20. Jod. a Meggen. cap. 15. pereg. Hier. Bellonius.}
\setauthornote{2914}{Cypr. Echovius delit. Hisp. Aqua profluens inde in omnes fere domos ducitur, in puteis quoque aestivo tempore frigidissima conservatur.}
\setauthornote{2915}{Sir Hugh Middleton, Baronet.}
\setauthornote{2916}{De quaesitis med. cent. fol. 354.}
\setauthornote{2917}{De piscibus lib. habent omnes in lautitiis, modo non sint e caenoso loco.}
\setauthornote{2918}{De pisc. c. 2. l. 7. Plurimum praestat ad utilitatem et jucunditatem. Idem Trallianus lib. 1. c. 16. pisces petrosi, et molles carne.}
\setauthornote{2919}{Etsi omnes putredini sunt obnoxii, ubi secundis mensis, incepto jam priore, devorentur, commodi succi prosunt, qui dulcedine sunt praediti. Ut dulcia cerasa, poma, \&c.}
\setauthornote{2920}{Lib. 2. cap. 1.}
\setauthornote{2921}{Montanus consil. 24.}
\setauthornote{2922}{Pyra quae grato sunt sapore, cocta mala, poma tosta, et saccliaro, vel anisi semine conspersa, utiliter statim a prandio vel a caena sumi possunt, eo quod ventriculum roborent et vapores caput petentes reprimant. Mont.}
\setauthornote{2923}{Punica mala aurantia commode permittuntur modo non sint austera et acida.}
\setauthornote{2924}{Olera omnia praeter boraginem, buglossum, intybum, feniculum, anisum, melissum vitari debent.}
\setauthornote{2925}{Mercurialis pract. Med.}
\setauthornote{2926}{Lib. 2. de com. Solus homo edit bibitque, \&c.}
\setauthornote{2927}{Consil. 21. 18. si plus ingerata quam par est, et ventriculus tolerare posset, nocet, et cruditates generat \&c.}
\setauthornote{2928}{Observat. lib. 1. Assuescat bis in die cibos, sumere, certa semper hora.}
\setauthornote{2929}{Ne plus ingerat cavendum quam ventriculus ferre potest, semperque surgat a mensa non satur.}
\setauthornote{2930}{Siquidem qui semimansum velociter ingerunt cibum, ventriculo laborem inferunt, et flatus maximos promovent, Crato.}
\setauthornote{2931}{Quidam maxime comedere nituntur, putantes ea ratione se vires refecturos; ignorantes, non ea quae ingerunt posse vires reficere, sed quae probe concoquunt.}
\setauthornote{2932}{Multa appetunt, pauca digerunt.}
\setauthornote{2933}{Saturnal. lib. 7. cap. 4.}
\setauthornote{2934}{Modicus et temperatus cibus et carni et animae utilis est.}
\setauthornote{2935}{Hygiasticon reg. 14. 16. unciae per diem sufficiant, computato pane, carne ovis, vel aliis obsoniis, et totidem vel paulo plures unciae protus.}
\setauthornote{2936}{Idem reg. 27. Plures in domibus suis brevi tempore pascentes extinguuntur, qui si triremibus vincti fuissent, aut gregario pane pasti, sani et incolumes in longam aetatem vitam prorogassent.}
\setauthornote{2937}{Nihil deterius quam diversa nutrientia simul adjungere, et comedendi tempus prorogare.}
\setauthornote{2938}{Lib. 1. hist.}
\setauthornote{2939}{Hor. ad lib. 5. ode ult.}
\setauthornote{2940}{Ciborum varietate et copia in eadem mensa nihil nocentius homini ad lutem, Fr. Valleriola, observ. l. 2. cap. 6.}
\setauthornote{2941}{Tul. orat. pro M. Marcel.}
\setauthornote{2942}{Nullus cibum sumere debet, nisi stomachus sit vacuus. Gordon, lib. med. l. 1. c. 11.}
\setauthornote{2943}{E multis eduliis unum elige, relictisque caeteris, ex eo comede.}
\setauthornote{2944}{L. de atra bile. Simplex sit cibus et non varius: quod licet dignitati tuae ob convivas difficile videatur, \&c.}
\setauthornote{2945}{Celsitudo tua prandeat sola, absque apparatu aulico, contentus sit illustrissimus princeps duobus tantum ferculis, vinoque Rhenano solum in mensa utatur.}
\setauthornote{2946}{Semper intra satietatem a mensa recedat, uno ferculo, contentus.}
\setauthornote{2947}{Lib. de Hel. et Jejunio. Multo melius in terram vina fudisses.}
\setauthornote{2948}{Crato. Multum refert non ignorare qui cibi priores, \&c. liquida precedant carnium jura, pisces, fructus, \&c. Coena brevior sit prandio.}
\setauthornote{2949}{Tract. 6. contradict. 1. Lib. 1.}
\setauthornote{2950}{Super omnia quotidianum leporem habuit, et pomis indulsit.}
\setauthornote{2951}{Annal. 6. Ridere solebat eos, qui post 30. aetatis annum, ad cognoscenda corpori suo noxia vel utilia, alicujus consilii indigerent.}
\setauthornote{2952}{A Lessio edit. 1614.}
\setauthornote{2953}{Aegyptii olim omnes morbos curabant vomitu et jejunio. Bohemus lib. 1. cap. 5.}
\setauthornote{2954}{He who lives medically lives miserably.}
\setauthornote{2955}{Cat. Major: Melior conditio senis viventis ex praescripto artis medicae, quam adolescentis luxuriosi.}
\setauthornote{2956}{Debet per amaena exerceri, et loca viridia, excretis prius arte vel natura alvi excrementis.}
\setauthornote{2957}{Hildesheim spicel, 2. de met. Primum omnium operam dabis ut singulis diebus habeas beneficium ventris, semper cavendo ne alvus sit diutis astricta.}
\setauthornote{2958}{Si non sponte, clisteribus purgetar.}
\setauthornote{2959}{Balneorum usus dulcium, siquid aliud, ipsis opitulatur. Credo haec dici cum aliqua jactantia, inquit Montanus consil. 26.}
\setauthornote{2960}{In quibus jejunus diu sedeat eo tempore, ne sudorem excitent aut manifestum teporem, sed quadam refrigeratione humectent.}
\setauthornote{2961}{Aqua non sit calida, sed tepida, ne sudor sequatur.}
\setauthornote{2962}{Lotiones capitis ex lixivio, in quo herbas capitales coxerint.}
\setauthornote{2963}{Cap. 8. de mel.}
\setauthornote{2964}{Aut axungia pulli, Piso.}
\setauthornote{2965}{Thermae. Nympheae.}
\setauthornote{2966}{Sandes lib. 1. saith, that women go twice a week to the baths at least.}
\setauthornote{2967}{Epist. 3.}
\setauthornote{2968}{Nec alvum excernunt, quin aquam secum portent qua partes obscaenas lavent. Busbequius ep. 3. Leg. Turciae.}
\setauthornote{2969}{Hildesheim speciel. 2. de mel. Hypocon. si non adesset jecoris caliditas, Thermas laudarem, et si non nimia humoris exsiccatio esset metuenda.}
\setauthornote{2970}{Fol. 141.}
\setauthornote{2971}{Thermas Lucenses adeat, ibique aquas ejus per 15. dies potet, et calidarum aquarum stillicidiis tum caput tum ventriculum de more subjiciat.}
\setauthornote{2972}{In panth.}
\setauthornote{2973}{Aquae Porrectanae.}
\setauthornote{2974}{Aquae Aquariae.}
\setauthornote{2975}{Ad aquas Aponenses velut ad sacram anchoram confugiat.}
\setauthornote{2976}{Joh. Baubinus li. 3. c. 14. hist. admir. Fontis Bollenses in ducat. Wittemberg laudat aquas Bollenses ad melancholicos morbos, maerorem, fascinationem, aliaque animi pathemata.}
\setauthornote{2977}{Balnea Chalderina.}
\setauthornote{2978}{Hepar externe ungatur ne calefiat.}
\setauthornote{2979}{Nocent calidis et siccis, cholericis, et omnibus morbis ex cholera, hepatis, splenisque affectionibus.}
\setauthornote{2980}{Lib. de aqua. Qui breve hoc vitae curriculum cupiunt sani transigere, frigidis aquis saepe lavare debent, nulli aetati cum sit incongrua, calidis imprimis utilis.}
\setauthornote{2981}{Solvit Venus rationis vim impeditam, ingentes iras remittit, \&c.}
\setauthornote{2982}{Multi comitiales, melancholici, insani, hujus usu solo sanati.}
\setauthornote{2983}{Si omittatur coitus, contristat, et plurimum gravat corpus et animum.}
\setauthornote{2984}{Nisi certo constet nimium semen aut sanguinem causam esse, aut amor praecesserit, aut, \&c.}
\setauthornote{2985}{Athletis, Arthriticis, podagricis nocet, nec opportuna prodest, nisi fortibus et qui multo sanguine abundant. Idem Scaliger exerc. 269. Turcis ideo luctatoribus prohibitum.}
\setauthornote{2986}{De sanit tuend. lib. 1.}
\setauthornote{2987}{Lib. 1. ca. 7. exhaurit enim spiritus animumque debilitat.}
\setauthornote{2988}{Frigidis et siccis corporibus inimicissima.}
\setauthornote{2989}{Vesci intra satietatem, impigrum esse ad laborem, vitale semen conservare.}
\setauthornote{2990}{Nequitia est quae te non sinit esse senem.}
\setauthornote{2991}{Vide Montanum, Pet. Godefridum, Amorum lib. 2. cap. 6. curiosum de his, nam et numerum de finite Talimudistis, unicuique sciatis assignari suum tempus, \&c.}
\setauthornote{2992}{Thespiadas genuit.}
\setauthornote{2993}{Vide Lampridium vit. ejus 4.}
\setauthornote{2994}{Et lassata viris, \&c.}
\setauthornote{2995}{Vid. Mizald. cent. 8. 11. Lemnium lib. 2. cap. 16. Catullum ad Ipsiphilam, \&c. Ovid. Eleg. lib. 3. et 6. \&c. quod itinera una nocte confecissent, tot coronas ludicro deo puta Triphallo, Marsiae, Hermae, Priapo donarent, Cin. gemus tibi mentulam coronis, \&c.}
\setauthornote{2996}{Pernobopcodid. Gasp. Barthii.}
\setauthornote{2997}{Nich. de Lynna, cited by Mercator in his map.}
\setauthornote{2998}{Mons Sloto. Some call it the highest hill in the world, next Teneriffe in the Canaries, Lat. 81.}
\setauthornote{2999}{Cap. 26. in his Treatise of Magnetic Bodies.}
\setauthornote{3000}{Lege lib. 1. cap. 23. et 24. de magnetica philosophia, et lib. 3. cap. 4.}
\setauthornote{3001}{1612.}
\setauthornote{3002}{M. Brigs, his map, and Northwest Fox.}
\setauthornote{3003}{Lib. 2. ca. 64. de nob. civitat. Quinsay, et cap. 10. de Cambalu.}
\setauthornote{3004}{Lib. 4. exped. ad Sinas, ca. 3. et lib. 5. c. 18.}
\setauthornote{3005}{M. Polus in Asia Presb. Joh, meminit lib. 2. cap. 30.}
\setauthornote{3006}{Alluaresius et alii.}
\setauthornote{3007}{Lat. 10. Gr. Aust.}
\setauthornote{3008}{Ferdinando de Quir. Anno 1612.}
\setauthornote{3009}{Alarum pennae continent in longitudine 12. passus, elephantem in sublime tollere potest. Polus l. 3. c. 40.}
\setauthornote{3010}{Lib. 2. Descript. terrae sanctae.}
\setauthornote{3011}{Natur. quaest. lib. 4. cap. 2.}
\setauthornote{3012}{Lib. de reg. Congo.}
\setauthornote{3013}{Exercit. 47.}
\setauthornote{3014}{See M. Carpenter's Geography, lib. 2. cap. 6. et Bern. Telesius lib. de mari.}
\setauthornote{3015}{Exercit. 52. de maris motu causae investigandae: prima reciprocationis, secunda varietatis, tertia celeritatis, quarta cessationis, quinta privationis, sexta contrarietatis. Patritius saith 52 miles in height.}
\setauthornote{3016}{Lib. de explicatione locoram Mathem. Aristot.}
\setauthornote{3017}{Laet. lib. 17. cap. 18. descrip. occid. Ind.}
\setauthornote{3018}{Luge alii vocant.}
\setauthornote{3019}{Geor. Wernerus, Aquae lanta celeritate erumpunt et absorbentur, ut expedito equiti aditum intereludant.}
\setauthornote{3020}{Boissardus de Magis cap. de Pilapiis.}
\setauthornote{3021}{In campis Lovicen, solum visuntur in nive, et ubinam vere, aestate, autumno se occultant. Hermes Polit. l. 1. Jul. Bellius.}
\setauthornote{3022}{Statim ineunte vere sylvae strepunt eorum cantilenis. Muscovit. comment.}
\setauthornote{3023}{Immergunt se fluminibus, lacubusque per hyemem totam, \&c.}
\setauthornote{3024}{Caeterasque volucres Pontum hyeme adveniente e nostris regionibus Europeis transvolantes.}
\setauthornote{3025}{Survey of Cornwall.}
\setauthornote{3026}{Porro ciconiae quonam a loco veniant, quo se conferant, incompertum adhuc, agmen venientium, descendentium, ut gruum venisse cernimus, nocturnis opinor temporibus. In patentibus Asiae campis certo die congregant se, eam quae novissime advenit lacerant, inde avolant. Cosmog. l. 4. c. 126.}
\setauthornote{3027}{Comment. Muscov.}
\setauthornote{3028}{Hist. Scot. l. 1.}
\setauthornote{3029}{Vertomannus l. 5. c. 16. mentioneth a tree that bears fruits to eat, wood to burn, bark to make ropes, wine and water to drink, oil and sugar, and leaves as tiles to cover houses, flowers, for clothes, \&c.}
\setauthornote{3030}{Animal infectum Cusino, ut quis legere vel scribere possit sine alterius ope luminis.}
\setauthornote{3031}{Cosmog. lib. 1. cap. 435 et lib. 3. cap. 1. habent ollas a natura formatas e terra extractas, similes illis a figulis factis, coronas, pisces, aves, et omnes animantium species.}
\setauthornote{3032}{Ut solent hirundines et ranae prae frigoris magnitudine mori, et postea redeunte vere 24. Aprilis reviviscere.}
\setauthornote{3033}{Vid. Pererium in Gen. Cor. a Lapide, et alios.}
\setauthornote{3034}{In Necyotnantia Tom. 2.}
\setauthornote{3035}{Pracastorius lib. de simp. Georgius Merula lib. de mem. Julius Billius, \&c.}
\setauthornote{3036}{Bimlerua, Ortelius, Brachiis centum subterra reperta est, in qua quadraginta octo cadavera inerant, Anchorae, \&c.}
\setauthornote{3037}{Pisces et conchae in montibus reperiuntur.}
\setauthornote{3038}{Lib. de locis Mathemat. Aristot.}
\setauthornote{3039}{Or plain, as Patricius holds, which Austin, Lactamius, and some others, held of old as round as a trencher.}
\setauthornote{3040}{Li. de Zilphia et Pigmeia, they penetrate the earth as we do the air.}
\setauthornote{3041}{Lib. 2. c. II2.}
\setauthornote{3042}{Commentar. ad annum 1537. Quicquid dicunt, Philosophi, quaedam sunt Tartari ostia, et loca puniendis animis destinata, ut Hecla mons, \&c. ubi mortuorum spiritus visuntur, \&c. voluit Deus extare talia loca, ut discant mortales.}
\setauthornote{3043}{Ubi miserabiles ejulantium voces audiuntur, qui auditoribus horrorem incutiunt hand vulgarem, \&c.}
\setauthornote{3044}{Ex sepulchris apparent mense Martio, et rursus sub terram se abscondunt, \&c.}
\setauthornote{3045}{Descript. Graec. lib. 6. de Pelop.}
\setauthornote{3046}{Conclave Ignatii.}
\setauthornote{3047}{Melius dubitare de occultis, quam litigare de incertis, ubi flamina inferni, \&c.}
\setauthornote{3048}{See Dr. Reynolds praelect. 55. in Apoc.}
\setauthornote{3049}{As they come from the sea, so they return to the sea again by secret passages, as in all likelihood the Caspian Sea vents itself into the Euxine or ocean.}
\setauthornote{3050}{Seneca quaest. lib. cap. 3, 4, 5, 6, 7, 8, 9, 10, 11, 12. de causis aquarum perpetuis.}
\setauthornote{3051}{In iis nec pullos hirundines excludunt, neque, \&c.}
\setauthornote{3052}{Th. Ravennas lib. de vit. hom. praerog. ca. ult.}
\setauthornote{3053}{At Quito in Peru. Plus auri quam terrae foditur in aurifodinis.}
\setauthornote{3054}{Ad Caput bonae spei incolae sunt nigerrimi: Si sol causa, cur non Hispani et Italiaeque nigri, in eadem latitudine, aeque distantes ab Aequatore, hi ad Austrum, illi ad Boream? qui sub Presbytero Johan. habitant subfusci sunt, in Zeilan et Malabar nigri, aeque distantes ab Aequatore, eodemque coeli parallelo: sed hoc magis mirari quis possit, in tota America nusquam nigros inveniri, praeter paucos in loco Quareno illis dicto: quae hujus coloris causa efficiens, coelive an terrae qualitas, an soli proprietas, aut ipsorum hominum innata ratio, aut omnia? Ortelius in Africa Theat.}
\setauthornote{3055}{Regio quocunque anni tempore temperatissima. Ortel. Multas Galliae et Italiae Regiones, molli tepore, et benigna quadam temperie prorsus antecellit, Jovi.}
\setauthornote{3056}{Lat. 45. Danubii.}
\setauthornote{3057}{Quevira lat. 40.}
\setauthornote{3058}{In Sir Fra. Drake's voyage.}
\setauthornote{3059}{Lansius orat. contra Hungaros.}
\setauthornote{3060}{Lisbon lat. 38.}
\setauthornote{3061}{Danzig lat. 54.}
\setauthornote{3062}{De nat. novi orbis lib. 1. cap. 9. Suavissimus omnium locus, \&c.}
\setauthornote{3063}{The same variety of weather Lod. Guicciardine observes betwixt Liege and Ajax not far distant, descript. Belg.}
\setauthornote{3064}{Magin. Quadus.}
\setauthornote{3065}{Hist. lib. 5.}
\setauthornote{3066}{Lib. 11. cap. 7.}
\setauthornote{3067}{Lib. 2. cap. 9. Cur. Potosi et Plata, urbes in tam tenui intervallo, utraque mont osa, \&c.}
\setauthornote{3068}{Terra malos homines nunc educat atque pusillos.}
\setauthornote{3069}{Nav. l. 1. c. 5.}
\setauthornote{3070}{Strabo.}
\setauthornote{3071}{As under the equator in many parts, showers here at such a time, winds at such a time, the Brise they call it.}
\setauthornote{3072}{Ferd. Cortesius. lib. Novus orbis inscript.}
\setauthornote{3073}{Lapidatum est. Livie.}
\setauthornote{3074}{Cosmog. lib. 4. cap. 22. Hae tempestatibus decidunt e nubibus faeculentis, depascunturque more locustorum omnia virentia.}
\setauthornote{3075}{Hort. Genial. An a terra sursum rapiuntur a solo iterumque cum pluviis praecipitantur? \&c.}
\setauthornote{3076}{Tam ominosus proventus in naturales causas referri vix potest.}
\setauthornote{3077}{Cosmog. c. 6.}
\setauthornote{3078}{Cardan saith vapours rise 288 miles from the earth, Eratosthenes 48 miles.}
\setauthornote{3079}{De Subtil. l. 2.}
\setauthornote{3080}{In progymnas.}
\setauthornote{3081}{Praefat. ad Euclid. Catop.}
\setauthornote{3082}{Manucodiatae, birds that live continually in the air, and are never seen on ground but dead: See Ulysses Alderovand. Ornithol. Scal. exerc. cap. 229.}
\setauthornote{3083}{Laet. descrip. Amer.}
\setauthornote{3084}{Epist. lib. 1 p. 83. Ex quibus constat nec diversa aeris et aetheris diaphana esse, nec refractiones aliunde quam a crasso aere causari-Non dura aut impervia, sed liquida, subtilis, motuique Planetarium facile cedens.}
\setauthornote{3085}{In Progymn. lib. 2. exempl. quinque.}
\setauthornote{3086}{In Theoria nova Met. caelestium 1578.}
\setauthornote{3087}{Epit. Astron. lib. 4.}
\setauthornote{3088}{Multa sane hinc consequuentur absurda, et si nihil aliud, tot Cometae in aethere animadversi, qui nullius orbus ductum comitantur, id ipsum sufficienter refellunt. Tycho astr. epist. page 107.}
\setauthornote{3089}{In Theoricis planetarum, three above the firmament, which all wise men reject.}
\setauthornote{3090}{Theor. nova coelest. Meteor.}
\setauthornote{3091}{Lib. de fabrica mundi.}
\setauthornote{3092}{Lib. de Cometis.}
\setauthornote{3093}{An sit crux et nubecula in coelis ad Polum Antarcticum, quod ex Corsalio refert Patritius.}
\setauthornote{3094}{Gilbertus Origanus.}
\setauthornote{3095}{See this discussed in Sir Walter Raleigh's history, in Zanch. ad Casman.}
\setauthornote{3096}{Vid. Fromundum de Meteoris, lib. 5. artic. 5. et Lansbergium.}
\setauthornote{3097}{Peculiari libello.}
\setauthornote{3098}{Comment. in mortum terrae Middlebergi 1630.}
\setauthornote{3099}{Peculiari libello.}
\setauthornote{3100}{See Mr. Carpenter's Geogr. cap. 4. lib. 1. Campanella et Origanus praef Ephemer. where Scripture places are answered.}
\setauthornote{3101}{De Magnete.}
\setauthornote{3102}{Comment, in 2 cap. sphaer. Jo. de Sacr. Bosc.}
\setauthornote{3103}{Dist. 3. gr. 1. a Polo.}
\setauthornote{3104}{Praef. Ephem.}
\setauthornote{3105}{Which may be full of planets, perhaps, to us unseen, as those about Jupiter, \&c.}
\setauthornote{3106}{Luna circumterrestris Planeta quum sit, consentaneum est esse in Luna viventes creaturas, et singulis Planetarum globis sui serviunt circulatores, ex qua consideratione, de eorum incolis summa probabilitate concludimus, quod et Tychoni Braheo, e sola consideratione vastitatis eorum visum fuit. Kepl. dissert, cum nun. sid. f. 29.}
\setauthornote{3107}{Temperare non possum quin ex inventis tuis hoc moneam, veri non absimile, non tam in Luna, sed etiam in Jove, et veliquis Planetis incolas esse. Kepl. fo. 26. Si non sint accolae in Jovis globo, qui notent admirandam hanc varietatem oculis, cui bono quatuor illi Planetae Jovem circumcursitant?}
\setauthornote{3108}{Some of those above Jupiter I have seen myself by the help of a glass eight feet long.}
\setauthornote{3109}{Rerum Angl. l. 1. c. 27 de viridibus pueris.}
\setauthornote{3110}{Infiniti alii mundi vel ut Brunus, terrae huic nostrae similes.}
\setauthornote{3111}{Libro Cont. philos. cap. 29.}
\setauthornote{3112}{Kepler fol. 2. dissert. Quid impedit quin credamus ex his initiis, plures alios mundos detegendos, vel (ut Democrito placuit) infinitos?}
\setauthornote{3113}{Lege somnium Kepler: edit. 1635.}
\setauthornote{3114}{Quid igitur inquies, si sint in coelo plures globi, similes nostrae telluris, an cum illis certabimus, quis meliorem mundi plagam teneat? Si nobiliores illorum globi, nos non sumus creaturarum rationalium nobilissimi: quomodo igitur omnia propter hominem? quomodo nos domini operum Dei? Kepler, fol. 29.}
\setauthornote{3115}{Franckfort. quarto 1620. ibid. 40. 1622.}
\setauthornote{3116}{Praefat. in Comment, in Genesin. Modo suadent Theologos, summa ignoratione versari, veras scientias admittere nolle, et tyrannidem exercere, ut eos falsis dogmatibus, superstitionibus, et religione Catholica, detineant.}
\setauthornote{3117}{Theat. Biblico.}
\setauthornote{3118}{His argumentis plane satisfecisti, de maculas in Luna esse maria, de lucidas partes esse terram. Kepler. fol. 16.}
\setauthornote{3119}{Anno. 1616.}
\setauthornote{3120}{In Hypothes. de mundo. Edit. 1597.}
\setauthornote{3121}{Lugduni 1633.}
\setauthornote{3122}{Whilst these blockheads avoid one fault, they fall into its opposite.}
\setauthornote{3123}{Jo. Fabritius de maculis in sole. Witeb. 1611.}
\setauthornote{3124}{In Burboniis sideribus.}
\setauthornote{3125}{Lib. de Burboniis sid. Stellae sunt erraticae, quae propriis orbibus feruntur, non longe a Sole dissitis, sed juxta Solem.}
\setauthornote{3126}{Braccini fol. 1630. lib. 4. cap. 52, 55. 59. \&c.}
\setauthornote{3127}{Lugdun. Bat. An. 1612.}
\setauthornote{3128}{Ne se subducant, et relicta statione decessum parent, ut curiositatis finem faciant.}
\setauthornote{3129}{Hercules tuam fidem Satyra Menip. edit. 1608.}
\setauthornote{3130}{I shall now enter upon a bold and memorable exploit; one never before attempted in this age. I shall explain this night's transactions in the kingdom of the moon, a place where no one has yet arrived, save in his dreams.}
\setauthornote{3131}{Sardi venales Satyr. Menip. An. 1612.}
\setauthornote{3132}{Puteani Comus sic incipit, or as Lipsius Satyre in a dream.}
\setauthornote{3133}{Tritemius. 1. de 7 secundis.}
\setauthornote{3134}{They have fetched Trajanus' soul out of hell, and canonise for saints whom they list.}
\setauthornote{3135}{In Minutius, sine delectu tempestates tangunt loca sacra et profana, bonorum et malorum fata, juxta, nullo ordine res fiunt, soluta legibus fortuna dominatur.}
\setauthornote{3136}{Vel malus vel impotens, qui peccatum permittit, \&c. unde haec superstitio?}
\setauthornote{3137}{Quid fecit Deus ante mundum creatum? ubi vixit otiosus a suo subjecto, \&c.}
\setauthornote{3138}{Lib. 3. recog. Pet. cap. 3. Peter answers by the simile of an eggshell, which is cunningly made, yet of necessity to be broken; so is the world, \&c. that the excellent state of heaven might be made manifest.}
\setauthornote{3139}{Ut me pluma levat, sic grave mergit onus.}
\setauthornote{3140}{Exercit. 184.}
\setauthornote{3141}{Laet. descrip. occid. Indiae.}
\setauthornote{3142}{Daniel principio historiae.}
\setauthornote{3143}{Veniant ad me audituri quo esculento, quo item poculento uti debeant, et praeter alimentum ipsum, potumque ventos ipsos docebo, item aeris ambientis temperiem, insuper regiones quas eligere, quas vitare ex usu sit.}
\setauthornote{3144}{Leo Afer, Maginus, \&c.}
\setauthornote{3145}{Lib. 1. Scot. hist.}
\setauthornote{3146}{Lib. 1. de rer. var.}
\setauthornote{3147}{Horat.}
\setauthornote{3148}{Maginus.}
\setauthornote{3149}{Haitonus de Tartaris.}
\setauthornote{3150}{Cyropaed li. 8. perpetuum inde ver.}
\setauthornote{3151}{The air so clear, it never breeds the plague.}
\setauthornote{3152}{Leander Albertus in Campania, e Plutarcho vita Luculli. Cum Cn. Pompeius, Marcus Cicero, multique nobiles viri L. Lucullum aestivo tempore convinessent, Pompeius inter coenam dum familiariter jocatus est, eam villam imprimis sibi sumptuosam, et elegantem videri, fenestris, porticibus, \&c.}
\setauthornote{3153}{Godwin vita Jo. Voysye al. Harman.}
\setauthornote{3154}{Descript. Brit.}
\setauthornote{3155}{In Oxfordshire.}
\setauthornote{3156}{Leander Albertus.}
\setauthornote{3157}{Cap. 21. de vit. hom. prorog.}
\setauthornote{3158}{The possession of Robert Bradshaw, Esq.}
\setauthornote{3159}{Of George Purefey, Esq.}
\setauthornote{3160}{The possession of William Purefey, Esq.}
\setauthornote{3161}{The seat of Sir John Reppington, Kt.}
\setauthornote{3162}{Sir Henry Goodieres, lately deceased.}
\setauthornote{3163}{The dwelling-house of Hum. Adderley, Esq.}
\setauthornote{3164}{Sir John Harpar's, lately deceased.}
\setauthornote{3165}{Sir George Greselies, Kt.}
\setauthornote{3166}{Lib. 1. cap. 2.}
\setauthornote{3167}{The seat of G. Purefey, Esq.}
\setauthornote{3168}{For I am now incumbent of that rectory, presented thereto by my right honourable patron, the Lord Berkley.}
\setauthornote{3169}{Sir Francis Willoughby.}
\setauthornote{3170}{Montani et Maritimi salubriores, acclives, et ad Boream ream vergentes.}
\setauthornote{3171}{The dwelling of Sir To. Burdet, Knight, Baronet.}
\setauthornote{3172}{In his Survey of Cornwall, book 2.}
\setauthornote{3173}{Prope paludes stagna, et loca concava, vel ad Austrum, vel ad Occidentem inclinatae, domus sunt morbosae.}
\setauthornote{3174}{Oportet igitur ad sanitatem domus in altioribus aedificare, et ad speculationem.}
\setauthornote{3175}{By John Bancroft, Dr. of Divinity, my quondam tutor in Christ Church, Oxon, now the Right Reverend Lord Bishop Oxon, who built this house for himself and his successors.}
\setauthornote{3176}{Hyeme erit vehementer frigida, et aestate non salubris: paludes enim faciunt crassum aerem, et difficiles morbos.}
\setauthornote{3177}{Vendas quot assibus possis, et si nequeas, relinquas.}
\setauthornote{3178}{Lib. 1. cap. 2. in Orco habita.}
\setauthornote{3179}{Aurora musis amica, Vitruv.}
\setauthornote{3180}{Aedes Orientem spectantes vir nobilissimus, inhabitet, et curet ut sit aer clarus, lucidus, odoriferus. Eligat habitationem optimo aere jucundam.}
\setauthornote{3181}{Quoniam angustiae itinerum et altitudo tectorum, non perinde Solis calorem admittit.}
\setauthornote{3182}{Consil. 21. li. 2. Frigidus aer, nubilosus, densus, vitandus, aeque ac venti septentrionales, \&c.}
\setauthornote{3183}{Consil. 24.}
\setauthornote{3184}{Fenestram non aperiat.}
\setauthornote{3185}{Discutit Sol horrorem crassi spiritus, mentem exhilarat, non enim tam corpora, quam et animi mutationem inde subeunt, pro coeli et ventorum ratione, et sani aliter affecti sini coelo nubilo, aliter sereno. De natura ventorum, see Pliny, lib. 2. cap. 26. 27. 28. Strabo, li. 7. \&c.}
\setauthornote{3186}{Fines Morison parr. 1. c. 4.}
\setauthornote{3187}{Altomarus car. 7. Bruel. Aer sit lucidus, bene olens, humidus. Montaltus idem ca. 26. Olfactus rerum suavium. Laurentius, c. 8.}
\setauthornote{3188}{Ant. Philos. cap. de melanc.}
\setauthornote{3189}{Tract. 15. c. 9. ex redolentibus herbis et foliis vitis viniferae, salicis, \&c.}
\setauthornote{3190}{Pavimentum aceto, et aqua rosacea irrorare, Laurent, c. 8.}
\setauthornote{3191}{Lib. 1. cap. de morb. Afrorum In Nigritarum regione tanta aeris temperis, ut siquis alibi morbosus eo advehatur, optimae statim sanitati restituatur, quod multis accidisse, ipse meis oculis vidi.}
\setauthornote{3192}{Lib. de peregrinat.}
\setauthornote{3193}{Epist. 2. cen. 1. Nec quisquam tam lapis aut frutex, quem non titillat amoena illa, variaque spectio locorum, urbium, gentium, \&c.}
\setauthornote{3194}{Epist. 86.}
\setauthornote{3195}{2. lib. de legibus.}
\setauthornote{3196}{Lib. 45.}
\setauthornote{3197}{Keckerman praefat, polit.}
\setauthornote{3198}{Fines Morison c. 3. part. 1.}
\setauthornote{3199}{Mutatio de loco in locum, Itinera, et voiagia longa et indeterminata, et hospitare in diversis diversoriis.}
\setauthornote{3200}{Modo ruri esse, modo in urbe, saepius in agro venari, \&c.}
\setauthornote{3201}{In Catalonia in Spain.}
\setauthornote{3202}{Laudaturque domos longos quae prospicit agros.}
\setauthornote{3203}{Many towns there are of that name, saith Adricomius, all high-sited.}
\setauthornote{3204}{Lately resigned for some special reasons.}
\setauthornote{3205}{At Lindley in Leicestershire, the possession and dwelling-place of Ralph Burton, Esquire, my late deceased father.}
\setauthornote{3206}{In Icon animorum.}
\setauthornote{3207}{Aegrotantes oves in alium locum transportandae sunt, ut alium aerem et aquam participantes, coalescant et corrobentur.}
\setauthornote{3208}{Alia utilia, sed ex mutatione aeris potissimum curatus.}
\setauthornote{3209}{Ne te daemon otiosum inveniat.}
\setauthornote{3210}{Praestat aliud agere quam nihil.}
\setauthornote{3211}{Lib. 3. de dictis Socratis, Qui tesseris et risui excitando vacant, aliquid faciunt, et si liceret his meliora agere.}
\setauthornote{3212}{Amasis compelled every man once a year to tell how he lived.}
\setauthornote{3213}{Nostra memoria Mahometes Othomannus qui Graeciae imperium subvertit, cum oratorum postulata audiret externarum gentium, cochlearia lignea assidue caelabat, aut aliquid in tabula affingebat.}
\setauthornote{3214}{Sands, fol. 37. of his voyage to Jerusalem.}
\setauthornote{3215}{Perkins, Cases of Conscience, l. 3. c. 4. q. 3.}
\setauthornote{3216}{Luscinius Grunnio. They seem to think they were born to idleness,-nay more, for the destruction of themselves and others.}
\setauthornote{3217}{Non est cura melior quam injungere iis necessaria, et opportuna; operum administratio illis magnum sanitatis incrementum, et quae repleant animos eorum et incutiant iis diversas cogitationes. Cont. 1. tract. 9.}
\setauthornote{3218}{Ante exercitum, leves toto corpore frictiones conveniunt. Ad hunc morbum exercitationes, quum recte et suo tempore fiunt, mirifice conducunt, et sanitatem tuentur, \&c.}
\setauthornote{3219}{Lib. 1. de san. tuend.}
\setauthornote{3220}{Exercitium naturae dormientis stimulatio, membrorum solatium, morborum medela, fuga vitiorum, medicina languorum, destructio omnium malorum, Crato.}
\setauthornote{3221}{Alimentis in ventriculo probe concotis.}
\setauthornote{3222}{Jejuno ventre vesica et alvo ab excrementis purgato, fricatis membris, lotis manibus et oculis, \&c. lib. de atra bile.}
\setauthornote{3223}{Quousque corpus universum intumescat, et floridum appareat, sudoreque, \&c.}
\setauthornote{3224}{Omnino sudorem vitent. cap. 7. lib. 1. Valescus de Tar.}
\setauthornote{3225}{Exercitium si excedat, valde periculosum. Salust. Salvianus de remed. lib. 2. cap. 1.}
\setauthornote{3226}{Camden in Staffordshire.}
\setauthornote{3227}{Fridevallius, lib. 1. cap. 2. optima omnium exercitationum multi ab hac solummodo morbis liberati.}
\setauthornote{3228}{Josephus Quercetanus dialect. polit. sect. 2. cap. 11. Inter omnia exercitia praestantiae laudem meretur.}
\setauthornote{3229}{Chyron in monte Pelio, praeceptor heroum eos a morbis animi venationibus et puris cibis tuebatur. M. Tyrius.}
\setauthornote{3230}{Nobilitas omnis fere urbes fastidit, castellis, et liberiore coelo gaudet, generisque dignitatem una maxime venatione, et falconum aucupiis tuetur.}
\setauthornote{3231}{Jos. Scaliger, commen. in Cir. in fol. 344. Salmuth. 23. de Novrepert. com. in Pancir.}
\setauthornote{3232}{Demetrius Constantinop. de re accipitraria, liber a P. Gillir latine redditus. Aelius. epist. Aquilae Symachi et Theodotionis ad Ptolomeum, \&c.}
\setauthornote{3233}{Lonicerus, Geffreus, jovius.}
\setauthornote{3234}{S. Antony Sherlie's relations.}
\setauthornote{3235}{Hacluit.}
\setauthornote{3236}{Coturnicum aucupio.}
\setauthornote{3237}{Fines Morison, part 3. c. 8.}
\setauthornote{3238}{Non majorem voluptatem animo capiunt, quam qui feras insectantur, aut missis canibus, comprehendunt, quum retia trahentes, squamosas pecudes in ripas adducunt.}
\setauthornote{3239}{More piscatorum cruribus ocreatus.}
\setauthornote{3240}{Si principibus venatio leporis non sit inhonesta, nescio quomodo piscatio cyprinorum videri debeat pudenda.}
\setauthornote{3241}{Omnino turpis piscatio, nullo studio digna, illiberalis credita est, quod nullum habet ingenium, nullam perspicaciam.}
\setauthornote{3242}{Praecipua hinc Anglis gloria, crebrae victoriae partae. Jovius.}
\setauthornote{3243}{Cap. 7.}
\setauthornote{3244}{Fracastorius.}
\setauthornote{3245}{Ambulationes subdiales, quas hortenses aurae ministrant, sub fornice viridi, pampinis virentibus concameratae.}
\setauthornote{3246}{Theophylact.}
\setauthornote{3247}{Itinerat. Ital.}
\setauthornote{3248}{Sedet aegrotus cespite viridi, et cum inclementia Canicularis terras excoquit, et siccat flumina, ipse securus sedet sub arborea fronde, et ad doloris sui solatium, naribus suis gramineas redolet species, pascit oculos herbarum amiena viriditas, aures suavi modulamine demulcet pictarum concentus avium, \&c. Deus bone, quanta pauperibus procures solatia!}
\setauthornote{3249}{Diod. Siculus, lib. 2.}
\setauthornote{3250}{Lib. 13 de animal. cap. 13.}
\setauthornote{3251}{Pet. Gillius. Paul. Hentzeus Itenerar. Italiae. 1617. Iod. Sincerus Itenerar. Galliae 1617. Simp. lib. 1. quest. 4.}
\setauthornote{3252}{Jucundissima deambulatio juxta mare, et navigatio prope terram. In utraque fluminis ripa.}
\setauthornote{3253}{Aurei panes, aurea obsonia, vis Margaritarum aceto subacta, \&c.}
\setauthornote{3254}{Lucan. The furniture glitters with brilliant gems, with yellow jasper, and the couches dazzle with their purple dye.}
\setauthornote{3255}{300 pellices, pecillatores et pincernae innumeri, pueri loti purpura induti, \&c. ex omnium pulchritudine delecti.}
\setauthornote{3256}{Ubi omnia cantu strepum.}
\setauthornote{3257}{Odyss.}
\setauthornote{3258}{Lucan. l. 8. The timbers were concealed by solid gold.}
\setauthornote{3259}{Iliad. 10. For neither was the contest for the hide of a bull, nor for a beeve, which are the usual prizes in the race, but for the life and soul of the great Hector.}
\setauthornote{3260}{Between Ardes and Guines, 1519.}
\setauthornote{3261}{Swertius in delitiis, fol. 487. veteri Horatiorum exemplo, virtute et successu admirabili, caesis hostibus 17. in conspectu patriae, \&c.}
\setauthornote{3262}{Paterculus, vol. post.}
\setauthornote{3263}{Quos antea audivi, inquit, hodie vidi deos.}
\setauthornote{3264}{Pandectae Triumph, fol.}
\setauthornote{3265}{Lib. 6. cap. 14. de bello Jud.}
\setauthornote{3266}{Procopius.}
\setauthornote{3267}{Laet. Lib. 10. Amer. descript.}
\setauthornote{3268}{Romulus Amaseus praefat. Pausan.}
\setauthornote{3269}{Virg. 1. Geor.}
\setauthornote{3270}{thirsting Tantalus gapes for the water that eludes his lips.}
\setauthornote{3271}{I may desire, but can't enjoy.}
\setauthornote{3272}{Roterus lib. 3. polit. cap. 1.}
\setauthornote{3273}{See Athenaeus dipnoso.}
\setauthornote{3274}{Ludi votivi, sacri, ludicri, Megalenses, Cereales, Florales, Martiales, \&c. Rosinus, 5. 12.}
\setauthornote{3275}{See Lipsius Amphitheatrum Rosinus lib. 5. Meursius de ludis Graecorum.}
\setauthornote{3276}{1500 men at once, tigers, lions, elephants, horses, dogs, bears, \&c.}
\setauthornote{3277}{Lib. ult. et l. 1. ad finem consuetudine non minus laudabili, quam veteri contubernia Rhetorum Rythmorum in urbibus et municipiis, certisque diebus exercebant se sagittarii, gladiatores, \&c. Alia ingenii, animique exercitia, quorum praecipuum studium, principem populum tragoediis, comoediis, fabulis scenicis, aliisque id genus ludis recreare.}
\setauthornote{3278}{Orbis terrae descript. part. 3.}
\setauthornote{3279}{What shall I say of their spectacles produced with the most magnificent decorations,-a degree of costliness never indulged in even by the Romans.}
\setauthornote{3280}{Lampridius.}
\setauthornote{3281}{Spartian.}
\setauthornote{3282}{Delectatus lusis catulorum, porcellorum, ut perdices inter se pugnarent, aut ut aves parvulae sursum et deorsum volitarent, his maxime delectatus, ut solitu dines publicas sublevaret.}
\setauthornote{3283}{Brumales laete ut possint producere noctes.}
\setauthornote{3284}{Miles. 4.}
\setauthornote{3285}{O dii similibus saepe conviviis date ut ipse videndo delectetur, et postmodum narrando delectet. Theod. prodromus Amorum dial. interpret. Gilberto Giaulinio.}
\setauthornote{3286}{Epist. lib. 8. Ruffino.}
\setauthornote{3287}{Hor.}
\setauthornote{3288}{Lib. 4. Gallicae consuetudinis est ut viatores etiam invitos consistere cogant, et quid quisque eorum audierit aut cognorit de qua re quaerunt.}
\setauthornote{3289}{Vitae ejus lib. ult.}
\setauthornote{3290}{Juven.}
\setauthornote{3291}{They account them unlawful because sortilegious.}
\setauthornote{3292}{Insist. c. 44. In his ludis plerumque non ars aut peritia viget, sed fraus, fallacia, dolus astutia, casus, fortuna, temeritas locum habent, non ratio consilium, spientia, \&c.}
\setauthornote{3293}{In a moment of fleeting time it changes masters and submits to new control.}
\setauthornote{3294}{Abusus tam frequens hodie in Europa ut plerique crebro harum usu patrimonium profundant, exhaustisque facultatibus, ad inopiam redigantur.}
\setauthornote{3295}{Ubi semel prurigo ista animum occupat aegre discuti potest, solicitantibus undique ejusdem farinae hominibus, damnosas illas voluptates repetunt, quod et scortatoribus insitum, \&c.}
\setauthornote{3296}{Instutitur ista exercitatio, non lucri, sed valetudinis et oblectamenti ratione, et quo animus defatigatus respiret, novasque vires ad subeundos labores denuo concipiat.}
\setauthornote{3297}{Latrunculorum ludus inventus est a duce, ut cum miles intolerabili fame laboraret, altero die edens altero ludens, famis oblivisceretur. Bellonius. See more of this game in Daniel Souter's Palamedes, vel de variis ludis, l. 3.}
\setauthornote{3298}{D. Hayward in vita ejus.}
\setauthornote{3299}{Muscovit. commentarium.}
\setauthornote{3300}{Inter cives Fessanos latrunculorum ludus est usitatissimus, lib. 3. de Africa.}
\setauthornote{3301}{It is better to dig than to dance.}
\setauthornote{3302}{Tullius. No sensible man dances.}
\setauthornote{3303}{De mor. gent.}
\setauthornote{3304}{Polycrat. l. 1. cap. 8.}
\setauthornote{3305}{Idem Salisburiensis.}
\setauthornote{3306}{Hist. lib. 1.}
\setauthornote{3307}{Nemo desidet otiosus, ita nemo asinino more ad seram noctem laborat; nam ea plusquam servilis aerumna, quae opificum vita eat, exceptis Utopiensibus qui diem in 24. horas dividunt, sex duntaxat operi deputant, reliquum a somno et cibo cujusque arbitrio permittitur.}
\setauthornote{3308}{Rerum Burgund. lib. 4.}
\setauthornote{3309}{Jussit hominem deferri ad palatium et lecto ducali collocari, \&c. mirari homo ubi se eo loci videt.}
\setauthornote{3310}{Quid interest, inquit Lodovicus Vives, (epist. ad Francisc. Barducem) interdiem illius et nostros aliquot annos? nihil penitus, nisi quod, \&c.}
\setauthornote{3311}{Hen. Stephan. praefat. Herodoti.}
\setauthornote{3312}{Study is the delight of old age, the support of youth, the ornament of prosperity, the solace and refuge of adversity, the comfort of domestic life, \&c.}
\setauthornote{3313}{Orat. 12. siquis animo fuerit afflictus aut aeger, nec somnum admittens, is mihi videtur e regione stans talis imaginis, oblivisci omnium posse, quae humanae vitae atrocia et difficilia accidere solent.}
\setauthornote{3314}{De anima.}
\setauthornote{3315}{Diad. 19.}
\setauthornote{3316}{Topogr. Rom. part. 1.}
\setauthornote{3317}{Quod heroum conviviis legi solitae.}
\setauthornote{3318}{Melancthon de Heliodoro.}
\setauthornote{3319}{I read a considerable part of your speech before dinner, but after I had dined I finished it completely. Oh what arguments, what eloquence!}
\setauthornote{3320}{Pluvines.}
\setauthornote{3321}{Thibault.}
\setauthornote{3322}{As in travelling the rest go forward and look before them, an antiquary alone looks round about him, seeing things past, \&c. hath a complete horizon. Janus Bifrons.}
\setauthornote{3323}{Cardan. What is more subtle than arithmetical conclusions; what more agreeable than musical harmonies; what more divine than astronomical, what more certain than geometrical demonstrations?}
\setauthornote{3324}{Hondius praefat. Mercatoris. It allures the mind by its agreeable attraction, on account of the incredible variety and pleasantness of the subjects, and excites to a further step in knowledge.}
\setauthornote{3325}{Atlas Geog.}
\setauthornote{3326}{Cardan. To learn the mysteries of the heavens, the secret workings of nature, the order of the universe, is a greater happiness and gratification than any mortal can think or expect to obtain.}
\setauthornote{3327}{Lib. de cupid. divitiarum.}
\setauthornote{3328}{Leon. Diggs. praefat. ad perpet. prognost.}
\setauthornote{3329}{Plus capio voluptatis, \&c.}
\setauthornote{3330}{In Hipperchen. divis. 3.}
\setauthornote{3331}{It is more honourable and glorious to understand these truths than to govern provinces, to be beautiful or to be young.}
\setauthornote{3332}{Cardan. praefat. rerum variet.}
\setauthornote{3333}{Poetices lib.}
\setauthornote{3334}{Lib. 3. Ode 9. Donec gratus eram tibi, \&c.}
\setauthornote{3335}{De Pelopones. lib. 6. descript. Graec.}
\setauthornote{3336}{Quos si integros haberemus, Dii boni, quas opes, quos thesauros teneremus.}
\setauthornote{3337}{Isaack Wake musae regnantes.}
\setauthornote{3338}{Si unquam mihi in fatis sit, ut captivus ducar, si mihi daretur optio, hoc cuperem carcere concludi, his catenia illigari, cum hisce captivis concatenatis aetatem agere.}
\setauthornote{3339}{Epist. Primiero. Plerunque in qua simul ac pedem posui, foribus pessulum abdo; ambitionem autem, amorem, libidinem, etc. excludo, quorum parens est ignavia, imperitia nutrix, et in ipso aeternitatis gremio, inter tot illustres animas sedem mihi sumo, cum ingenti quidem animo, ut subinde magnatum me misereat, qui felicitatem hanc ignorant.}
\setauthornote{3340}{Chil. 2. Cent. 1. Adag. 1.}
\setauthornote{3341}{Virg. eclog. 1.}
\setauthornote{3342}{Founder of our public library in Oxon.}
\setauthornote{3343}{Ours in Christ Church, Oxon.}
\setauthornote{3344}{Animus lavatur inde a curis multa quiete et tranquillitate fruens.}
\setauthornote{3345}{Ser. 38. ad Fratres Erem.}
\setauthornote{3346}{Hom. 4. de poenitentia. Nam neque arborum comae pro pecorum tuguriis factae meridie per aestatem, optabilem exhibentes umbram oves ita reficiunt, ac scripturarum lectio afflictas angore animas solatur et recreat.}
\setauthornote{3347}{Otium sine literis mors est, et vivi hominis sepultura, Seneca.}
\setauthornote{3348}{Cap. 99. l. 57. de rer. var.}
\setauthornote{3349}{Fortem reddunt animum et constantem; et pium colloquium non permittit animum absurda cogitatione torqueri.}
\setauthornote{3350}{Altercationibus utantur, quae non permittunt animum submergi profundis cogitationibus, de quibus otiose cogitat et tristatur in iis.}
\setauthornote{3351}{Bodin. prefat. ad meth. hist.}
\setauthornote{3352}{Operum subcis. cap. 15.}
\setauthornote{3353}{Hor.}
\setauthornote{3354}{Fatendum est cacumine Olympi constitutus supra ventos et procellas, et omnes res humanas.}
\setauthornote{3355}{Who explain what is fair, foul, useful, worthless, more fully and faithfully than Chrysippus and Crantor?}
\setauthornote{3356}{In Ps. xxxvi. omnis morbus animi in scriptura habet medicinam; tantum opus est ut qui sit seger, non recuset potionem quam Deus temperavit.}
\setauthornote{3357}{In moral. speculum quo nos intueri possimus.}
\setauthornote{3358}{Hom. 28. Ut incantatione viris fugatur, ita lectione malum.}
\setauthornote{3359}{Iterum atque, iterum moneo, ut animam sacrae scripturae lectione occupes. Masticat divinum pabulum meditatio.}
\setauthornote{3360}{Ad 2. definit. 2. elem. In disciplinis humanis nihil praestantius reperitur: quippe miracula quaedam numerorum eruit tam abstrusa et recondita, tanta nihilo minus facilitate et voluptate, ut, \&c.}
\setauthornote{3361}{Which contained 1,080,000 weights of brass.}
\setauthornote{3362}{Vide Clavium in com. de Sacrobosco.}
\setauthornote{3363}{Distantias caelorum sola Optica dijudicat.}
\setauthornote{3364}{Cap. 4. et 5.}
\setauthornote{3365}{If the lamp burn brightly, then the man is cheerful and healthy in mind and body; if, on the other hand, he from whom the blood is taken be melancholic or a spendthrift, then it will burn dimly, and flicker in the socket.}
\setauthornote{3366}{Printed at London, Anno 3620.}
\setauthornote{3367}{Once astronomy reader at Gresham College.}
\setauthornote{3368}{Printed at London by William Jones, 1623.}
\setauthornote{3369}{Praefat. Meth. Astrol.}
\setauthornote{3370}{Tot tibi sunt dotes virgo, quot sidera coelo.}
\setauthornote{3371}{Da pie Christe urbi bona sit pax tempore nostro.}
\setauthornote{3372}{Chalonerus, lib. 9. de Rep. Angel.}
\setauthornote{3373}{Hortus Coronarius medicus et culinarius, \&c.}
\setauthornote{3374}{Tom. 1. de sanit. tuend. Qui rationem corporis non habent, sed cogunt mortalem immortali, terrestrem aethereae aequalem praestare industriam: Caeterum ut Camelo usu venit, quod ei bos praedixerat, cum eidem servirent domino et parte oneris levare illum Camelus recusasset, paulo post et ipsius curem, et totum onus cogeretur gestare (quod mortuo bove impletum) Ita animo quoque contingit, dum defatigato corpori, \&c.}
\setauthornote{3375}{Ut pulchram illam et amabilem sanitatem praestemus.}
\setauthornote{3376}{Interdicendae Vigiliae, somni paulo longiores conciliandi. Altomarus cap. 7. Somnus supra modum prodest, quovismodo conciliandus, Piso.}
\setauthornote{3377}{Ovid.}
\setauthornote{3378}{In Hippoc. Aphoris.}
\setauthornote{3379}{Crato cons. 21. lib. 2. duabus aut tribus horis post caenam, quum jam cibus ad fundum ventriculi resederit, primum super latere dextro quiescendum, quod in tali decubito jecur sub ventriculo quiescat, non gravans sed cibum calfaciens, perinde ac ignis lebetem qui illi admovetur; post primum somnum quiescendum latere sinistro, \&c.}
\setauthornote{3380}{Saepius accidit melancholicis, ut nimium exsiccato cerebro vigiliis attenuentur. Ficinus, lib. 1. cap. 29.}
\setauthornote{3381}{Ter. That you may sleep calmly on either ear.}
\setauthornote{3382}{Ut sis nocte levis, sit tibi, caena brevis.}
\setauthornote{3383}{Juven. Sat. 3.}
\setauthornote{3384}{Hor. Scr. lib. 1. Sat. 5. The tipsy sailor and his travelling companion sing the praises of their absent sweethearts.}
\setauthornote{3385}{Sepositis curis omnibus quantum fieri potest, una cum vestibus, \&c. Kirkst.}
\setauthornote{3386}{Ad horam somni aures suavibus cantibus et sonis delinire.}
\setauthornote{3387}{Lectio jucunda, aut sermo, ad quem attentior animus convertitur, aut aqua ab alto in subjectam pelvim delabatur, \&c. Ovid.}
\setauthornote{3388}{Aceti sorbitio.}
\setauthornote{3389}{Attenuat melancholiam, et ad conciliandum somnum juvat.}
\setauthornote{3390}{Quod lieni acetum conveniat.}
\setauthornote{3391}{Cont. 1. tract. 9. meditandum de aceto.}
\setauthornote{3392}{Sect. 5. memb. 1. Subsect. 6.}
\setauthornote{3393}{Lib. de sanit. tuenda.}
\setauthornote{3394}{In Som. Scip. fit enim fere ut cogitationes nostrae et sermones pariant aliquid in somno, quale de Homero scribit Ennius, de quo videlicet saepissime vigilans solebat cogitare et loqui.}
\setauthornote{3395}{Aristae hist. Neither the shrines of the gods, nor the deities themselves, send down from the heavens those dreams which mock our minds with those flitting shadows,-we cause them to ourselves.}
\setauthornote{3396}{Optimum de coelestibus et honestis meditari, et ea facere.}
\setauthornote{3397}{Lib. 3. de causis corr. art. tam mira monstra quaestionum saepe nascuntur inter eos, ut mirer eos interdum in somniis non terreri, aut de illis in tenebris audere verba facere, adeo res sunt monstrosae.}
\setauthornote{3398}{Icon. lib. 1.}
\setauthornote{3399}{Sect. 5. Memb. 1. Subs. 6.}
\setauthornote{3400}{Animi perturbationes summe fugiendae, metus potissimum et tristitia: earumque loco animus demulcendus hilaritate, animi constantia, bona spe; removendi terrores, et earum consortium quos non probant.}
\setauthornote{3401}{Phantasiae eorum placide subvertendae, terrores ab animo removendi.}
\setauthornote{3402}{Ab omni fixa cogitatione quovismodo avertantur.}
\setauthornote{3403}{Cuncta mala corporis ab animo procedunt, quae nisi curentur, corpus curari minime potest, Charmid.}
\setauthornote{3404}{Disputat. An morbi graviores corporis an animi. Renoldo interpret. ut parum absit a furore, rapitur a Lyceo in concionem, a concione ad mare, a mari in Siciliam, \&c.}
\setauthornote{3405}{Ira bilem movet, sanguinem adurit, vitales spiritus accendit. moestitia universum corpus infrigidat, calorem innatum extinguit, appetituin destruit, concoctionem impedit, corpus exsiccat, intellectum pervertit. Quamobrem haec omnia prorsus vitanda sunt, et pro virili fugienda.}
\setauthornote{3406}{De mel. c. 26. ex illis solum remedium; multi ex visis, auditis, \&c. sanati sunt.}
\setauthornote{3407}{Pro viribus annitendum in praedictis, tum in aliis, a quibus malum velut a primaria causa occasionem nactum est, imaginationes absurdae falsaeque et moestitia quaecunque subierit propulsetur, aut aliud agendo, aut ratione persuadendo earum mutationem subito facere.}
\setauthornote{3408}{Lib. 2. c. 16. de occult. nat. Quisquis huic malo obnoxius est, acriter obsistat, et summa cura obluctetur, nec ullo modo foveat imaginationes tacite obrepentes animo, blandas ab initio et amabiles, sed quae adeo convalescunt, ut nulla ratione excuti queant.}
\setauthornote{3409}{3. Tusc. ad Apollonium.}
\setauthornote{3410}{Facastorius.}
\setauthornote{3411}{Epist. de secretis artis et naturae cap. 7. de retard. sen. Remedium esset contra corruptionem propriam, si quilibet exerceret regimen sanitatis, quod consistit in rebus sex non naturalibus.}
\setauthornote{3412}{Pro aliquo vituperio non indigneris, nec pro admissione alicujus rei, pro morte alicujus, nec pro carcere, nec pro exilio, nec pro alia re, nec irascaris, nec timeas, nec doleas, sed cum summa praesentia haec sustineas.}
\setauthornote{3413}{Quodsi incommoda adversitatis infortunia hoc malum invexerint, his infractum animum opponas, Dei verbo ejusque fiducia te suffulcias, \&c., Lemnius, lib. 1. c. 16.}
\setauthornote{3414}{Lib. 2. de ira.}
\setauthornote{3415}{Cap. 3. de affect. anim. Ut in civitatibus contumaces qui non cedunt politico imperio vi coercendi sunt; ita Deus nobis indidit alteram imperii formam; si cor non deponit vitiosum affectum, membra foras coercenda sunt, ne ruant in quod affectus impellant: et locomotiva, quae herili imperio obtemperat, alteri resistat.}
\setauthornote{3416}{Imaginatio impellit spiritus, et inde nervi moventur, \&c. Et obtemperant imaginationi et appetitui mirabili foedere, ad exequendum quod jubent.}
\setauthornote{3417}{Ovit Trist. lib. 5.}
\setauthornote{3418}{Participes inde calamitatis nostrae sunt, et velut exonerata in eos sarcina onere levamur. Arist. Eth. lib. 9.}
\setauthornote{3419}{Camerarius Embl. 26. Cen. 2.}
\setauthornote{3420}{Sympos. lib. 6. cap. 10.}
\setauthornote{3421}{Epist. 8. lib. 3. Adversa fortuna habet in querelis levamentum; et malorum relatio, \&c.}
\setauthornote{3422}{Alloquium chari juvat, et solamen amici. Emblem. 54. cent. 1.}
\setauthornote{3423}{As David did to Jonathan, 1 Sam. xx.}
\setauthornote{3424}{Seneca Epist. 67.}
\setauthornote{3425}{Hic in civitate magna et turba magna neminem reperire possumus quocum suspirare familiariter aut jocari libere possimus. Quare te expectamus, te desideramus, te arcessimus. Multa sunt enim quae me solicitant et angunt, quae mihi videor aurestuas nactus, unius ambulationis sermone exhaurire posse.}
\setauthornote{3426}{I have not a single friend this day, to whom I dare to disclose my secrets.}
\setauthornote{3427}{Ovid.}
\setauthornote{3428}{De amicitia.}
\setauthornote{3429}{De tranquil. c. 7. Optimum est amicum fidelem nancisci in quem secreta nostra infundamus; nihil aeque oblectat animum, quam ubi sint praeparata pectora, in quae tuto secreta descendant, quorum conscientia aeque ac tua: quorum sermo solitudinem leniat, sententia consilium expediat, hilaritas tristitiam dissipet, conspectusque ipse delectet.}
\setauthornote{3430}{Comment. l. 7. Ad Deum confugiamus, et peccatis veniam precemur, inde ad amicos, et cui plurimum tribuimus, nos patefaciamus totos, et animi vulnus quo affligimur: nihil ad reficiendum animum efficacius.}
\setauthornote{3431}{Ep. Q. frat.}
\setauthornote{3432}{Aphor. prim.}
\setauthornote{3433}{Epist. 10.}
\setauthornote{3434}{Observando motus, gestus, manus, pedes, oculos, phantasiam, Piso.}
\setauthornote{3435}{Mulier melancholia correpta ex longa viri peregrinatione, et iracunde omnibus respondens, quum maritus domum reversus, praeter spem, \&c.}
\setauthornote{3436}{Prae dolore moriturus quum nunciatum esset uxorem peperisse filium subito recuperavit.}
\setauthornote{3437}{Nisi affectus longo tempore infestaverit, tali artificio imaginationes curare oportet, praesertim ubi malum ab his velut a primaria causa occasionem habuerit.}
\setauthornote{3438}{Lib. 1. cap. 16. Si ex tristitia aut alio affectu caeperit, speciem considera, aut aliud qui eorum, quae subitam alterationem facere possunt.}
\setauthornote{3439}{Evitandi monstrifici aspectus, \&c.}
\setauthornote{3440}{Neque enim tam actio, aut recordatio rerum hujusmodi displicet, sed iis vel gestus alterius Imaginationi adumbrare, vehementer molestum. Galat. de mor. cap. 7.}
\setauthornote{3441}{Tranquil. Praecipue vitentur tristes, et omnia deplorantes; tranquillitati inimicus est comes perturbatus, omnia gemens.}
\setauthornote{3442}{Illorum quoque hominum, a quorum consortio abhorrent, praesentia amovenda, nec sermonibus ingratis obtudendi; si quis insaniam ab insania sic curari aestimet, et proterve utitur, magis quam aeger insanit. Crato consil. 184. Scoltzii.}
\setauthornote{3443}{Molliter ac suaviter aeger tractetur, nec ad ea adigatur quae non curat.}
\setauthornote{3444}{Ob suspiciones curas, aemulationem, ambitionem, iras, \&c. quas locus ille ministrat, et quae fecissent melancholicum.}
\setauthornote{3445}{Nisi prius animum turbatissimum curasset; oculi sine capite, nec corpus sine anima curari potest.}
\setauthornote{3446}{E graeco. You shall not cure the eye, unless you cure the whole head also; nor the head, unless the whole body; nor the whole body, unless the soul besides.}
\setauthornote{3447}{Et nos non paucos sanavimus, animi motibus ad debitum revocatis, lib. 1. de sanit. tuend.}
\setauthornote{3448}{Consol. ad Apollonium. Si quis sapienter et suo tempore adhibeat, Remedia morbis diversis diversa sunt; dolentem sermo benignus sublevat.}
\setauthornote{3449}{Lib. 12. Epist.}
\setauthornote{3450}{De nat. deorum consolatur afflictos, deducit perterritos a timore, cupiditates imprimis, et iracundias comprimit.}
\setauthornote{3451}{Heauton. Act. 1. Scen. 1. Ne metue, ne verere, crede inquam mihi, aut consolando, aut consilio, aut rejuvero.}
\setauthornote{3452}{Novi faeneratorem avarud apud meus sic curatum, qui multam pecuniam amiserat.}
\setauthornote{3453}{Lib. 1. consil. 12. Incredibile dictu quantum juvent.}
\setauthornote{3454}{Nemo istiusmodi conditionis hominibus insultet, aut in illos sit severior, verum miseriae potius indolescat, vicemque deploret. lib. 2. cap. 16.}
\setauthornote{3455}{Cap. 7. Idem Piso Laurentius cap. 8.}
\setauthornote{3456}{Quod timet nihil est, ubi cogitur et videt.}
\setauthornote{3457}{Una vice blandiantur, una vice iisdem terrorem incutiant.}
\setauthornote{3458}{Si vero fuerit ex novo malo audito, vel ex animi accidente, aut de amissione mercium, aut morte amici, introducantur nova contraria his quae ipsum ad gaudia moveant; de hoc semper niti debemus, \&c.}
\setauthornote{3459}{Lib. 3. cap. 14.}
\setauthornote{3460}{Cap. 3. Castratio olim a veteribus usa in morbis desperatis, \&c.}
\setauthornote{3461}{Lib. 1. cap. 5. sic morbum morbo, ut clavum clavo, retundimus, et malo nodo malum cuneum adhibemus. Novi ego qui ex subito hostium incursu et inopi nato timore quartanam depulerat.}
\setauthornote{3462}{Lib. 7. cap. 50. In acie pugnans febre quartana liberatus est.}
\setauthornote{3463}{Jacchinus, c. 15. in 9. Rhasis Mont. cap. 26.}
\setauthornote{3464}{Lib. 1. cap. 16. aversantur eos qui eorum affectus rident, contemnunt. Si ranas et viperas comedisse se putant, concedere debemus, et spem de cura facere.}
\setauthornote{3465}{Cap. 8. de mel.}
\setauthornote{3466}{Cistam posuit ex Medicorum consilio prope eum, in quem alium se mortuum fingentem pacuit; hic in cista jacens, \&c.}
\setauthornote{3467}{Serres. 1550.}
\setauthornote{3468}{In 9. Rhasis. Magnam vim habet musica.}
\setauthornote{3469}{Cap. de Mania. Admiranda profecto res est, et digna expensione, quod sonorum concinnitas mentem emolliat, sistatque procellosas ipsius affectiones.}
\setauthornote{3470}{Laguens animus inde erigitur et reviviscit, nec tam aures afficit, sed et sonitu per arterias undique diffuso, spiritus tum vitales tum animales excitat, mentem reddens aeilem, \&c.}
\setauthornote{3471}{Musica venustate sua mentes severiores capit, \&c.}
\setauthornote{3472}{Animos tristes subito exhilarat, nubilos vultus serenat, austeritatem reponit, jucunditatem exponit, barbariemque facit deponere gentes, mores instituit, iracundiam mitigat.}
\setauthornote{3473}{Cithara tristitiam jucundat, timidos furores attenuat, cruentam saevitiam blande reficit, languorem. \&c.}
\setauthornote{3474}{Pet. Aretine.}
\setauthornote{3475}{Castilio de aulic. lib 1. fol. 27.}
\setauthornote{3476}{Lib. de Natali. cap. 12.}
\setauthornote{3477}{Quod spiritus qui in corde agitant tremulem et subsaltantem recipiunt aerem in pectus, et inde excitantur, a spiritu musculi moventur, \&c.}
\setauthornote{3478}{Arbores radicibus avulsae, \&c.}
\setauthornote{3479}{M. Carew of Anthony, in descript. Cornwall, saith of whales, that they will come and show themselves dancing at the sound of a trumpet, fol. 35. 1. et fol. 154. 2 book.}
\setauthornote{3480}{De cervo, equo, cane, urso idem compertum; musica afficiuntur.}
\setauthornote{3481}{Numen inest numeris.}
\setauthornote{3482}{Saepe graves morbos modulatum carmen abegit. Et desperatis conciliavit opem.}
\setauthornote{3483}{Lib. 5. cap. 7. Moerentibus moerorem adimam, laetantem vero seipso reddam hilariorem, amantem calidiorem, religiosum divine numine correptum, et ad Deos colendos paratiorem.}
\setauthornote{3484}{Natalis Comes Myth. lib. 4. cap. 12.}
\setauthornote{3485}{Lib. 5. de rep. Curat. Musica furorem Sancti viti.}
\setauthornote{3486}{Exilire e convivio. Cardan, subtil, lib. 13.}
\setauthornote{3487}{Iliad. 1.}
\setauthornote{3488}{Libro 9. cap. 1. Psaltrias. Sambuciatrasque et convivalia ludorum oblectamenta addita epuliis ex Asia invexit in urbem.}
\setauthornote{3489}{Comineus.}
\setauthornote{3490}{Ista libenter et magna cum voluptate spectare soleo. Et scio te illecebris hisce captum iri et insuper tripudiaturum, haud dubie demulcebere.}
\setauthornote{3491}{In musicis supra omnem fidem capior et oblector; choreas libentissime aspicio, pulchraram foeminarum venustate detineor, otiari inter has solutus curis possum.}
\setauthornote{3492}{3. De legibus.}
\setauthornote{3493}{Sympos. quest. 5. Musica multos magis dementat quam vinum.}
\setauthornote{3494}{Animi morbi vel a musica curantur vel inferuntur.}
\setauthornote{3495}{Lib. 3. de anima Laetitia purgat sanguinem, valetudinem conservat, colorem inducit florentem, nitidum gratum.}
\setauthornote{3496}{Spiritus temperat, calorem excitat, naturalem virtutem corroborat, juvenile corpus diu servat, vitam prorogat, ingenium acuit, et hominum negotii quibuslibet aptiorem reddit. Schola Salern.}
\setauthornote{3497}{Dum contumelia vacant et festiva lenitate mordent, mediocres animi aegritudines sanari solent, \&c.}
\setauthornote{3498}{De mor. fol. 57. Amamusideo eos qui sunt faceti et jucundi.}
\setauthornote{3499}{Regim. sanit. part. 2. Nota quod arnicas bonus et dilectus socius, narrationibus suis jucundis superat omneni melodiam.}
\setauthornote{3500}{Lib. 21. cap. 27.}
\setauthornote{3501}{Comment. in 4 Odyss.}
\setauthornote{3502}{Lib. 26. c. 15.}
\setauthornote{3503}{Homericum illud Nepenthes quod moerorem tollit, et cuthimiam, et hilaritatem parit.}
\setauthornote{3504}{Plaut. Bacch.}
\setauthornote{3505}{De aegritud. capitis. Omni modo generet laetitiam in iis, de iis quae audiuntur et videntur, aut odorantur, aut gustantur, aut quocunque modo sentiri possunt, et aspectu formarum multi decoris et ornatus, et negotiatione; jucunda, et blandientibus ludis, et promissis distrahantur, eorum animi, de re aliqua quam timent et dolent.}
\setauthornote{3506}{Utantur ve nationibus ludis, jocis, amicorum consortiis, quae non sinunt animum turbari, vino et cantu et loci mutatione, et biberia, et gaudio, ex quibus praecipue delectantur.}
\setauthornote{3507}{Piso ex fabulis et ludis quaerenda delectatio. His versetur qui maxima grati, sunt, cantus et chorea ad laetitiam prosunt.}
\setauthornote{3508}{Praecipue valet ad expellendam melancholiam stare in cantibus, ludis, et sonis et habitare cum familiaribus, et praecipue cum puellis jucundis.}
\setauthornote{3509}{Par. 5. de avocamentis lib. de absolvendo luctu.}
\setauthornote{3510}{Corporum complexus, cantus, ludi, formae, \&c.}
\setauthornote{3511}{Circa hortos Epicuri frequenter.}
\setauthornote{3512}{Dypnosoph. lib. 10. Coronavit florido serto incendens odores, in culcitra plumea collocavit dulciculam potionem propinans psaltriam adduxit, \&c.}
\setauthornote{3513}{Ut reclinata suaviter in lectum puella, \&c.}
\setauthornote{3514}{Tom. 2. consult. 85.}
\setauthornote{3515}{Epist. fam. lib. 7. 22. epist. Heri demum bene potus, seroque redieram.}
\setauthornote{3516}{Valer. Max. cap. lib. 8. Interposita arundine cruribus suis, cum filiis ludens, ab Alcibiade risus est.}
\setauthornote{3517}{Hor.}
\setauthornote{3518}{Hominibus facetis et ludis puerilibus ultra modum deditus adeo ut si cui in eo tam gravitatem, quam levitatem considerare liberet, duas personas distinctas in eo esse diceret.}
\setauthornote{3519}{De nugis curial. lib. 1. cap. 4. Magistratus et viri graves, a ludis levioribus arcendi.}
\setauthornote{3520}{Machiavel vita ejus. Ab amico reprehensus, quod praeter dignitatem tripudiis operam daret, respondet, \&c.}
\setauthornote{3521}{There is a time for all things, to weep, laugh, mourn, dance, Eccles. iii. 4.}
\setauthornote{3522}{Hor.}
\setauthornote{3523}{John Harrington, Epigr. 50.}
\setauthornote{3524}{Lucretia toto sis licet usque die, Thaida nocte volo.}
\setauthornote{3525}{Lil. Giraldus hist. deor. Syntag. 1.}
\setauthornote{3526}{Lib. 2. de aur. as.}
\setauthornote{3527}{Eo quod risus esset laboris et modesti victus condimentum.}
\setauthornote{3528}{Calcag. epig.}
\setauthornote{3529}{Cap. 61. In deliciis habuit scurras et adulatores.}
\setauthornote{3530}{Universa gens supra mortales caeteros conviviorum studiosissima. Ea enim per varias et exquisitas dapes, interpositis musicis et joculatoribus, in multas saepius horas extrahunt, ac subinde productis choreis et amoribus foeminarum indulgent, \&c.}
\setauthornote{3531}{Syntag. de Musis.}
\setauthornote{3532}{Atheneus lib. 12 et 14. assiduis mulierum vocibus, cantuque symphoniae Palatium Persarum regis totum personabat. Jovius hist. lib. 18.}
\setauthornote{3533}{Eobanus Hessus.}
\setauthornote{3534}{Fracastorius.}
\setauthornote{3535}{Vivite ergo laeti, O amici, procul ab angustia, vivite laeti.}
\setauthornote{3536}{Iterum precor et obtestor, vivite laeti: illad quod cor urit, negligite.}
\setauthornote{3537}{Laetus in praesens animus quod ultra oderit curare. Hor. He was both Sacerdoa et Medicus.}
\setauthornote{3538}{Haec autem non tam ut Sacerdos, amici, mando vobis, quam ut medicus; nam absque hac una tanquam medicinarum vita, medicinae omnes ad vitam producendam. adhibitae moriuntur: vivite laeti.}
\setauthornote{3539}{Locheus Anacreon.}
\setauthornote{3540}{Lucian. Necyomantia. Tom. 2.}
\setauthornote{3541}{Omnia mundana nugas aestima. Hoc solum tota vita persequere, ut praesentibus bene compositis, minime curiosus, aut ulla in re solicitus, quam plurimum potes vitam hilarem traducas.}
\setauthornote{3542}{If the world think that nothing can be happy without love and mirth, then live in love and jollity.}
\setauthornote{3543}{Hildesheim spicel. 2. de Mania, fol. 161. Studia literarum et animi perturbationes fugiat, et quantum potest jucunde vivat.}
\setauthornote{3544}{Lib. de atra bile. Gravioribus curis ludos et facetias aliquando interpone, jocos, et quae solent animum relaxare.}
\setauthornote{3545}{Consil. 30. mala valetudo aucta et contracta est tristitia, ac proptera exhilaratione animi removenda.}
\setauthornote{3546}{Athen. dypnosoph. lib. 1.}
\setauthornote{3547}{Juven. sat. 8. You will find him beside some cutthroat, along with sailors, or thieves, or runaways.}
\setauthornote{3548}{Hor. What does it signify whether I perish by disease or by the sword!}
\setauthornote{3549}{Frossard. hist. lib. 1. Hispani cum Anglorum vires ferre non possent, in fugam se dederunt, \&c. Praecipites in fluvium se dederunt, ne in hostium manus venirent.}
\setauthornote{3550}{Ter.}
\setauthornote{3551}{Hor Although you swear that you dread the night air.}
\setauthornote{3552}{Ἠ πίθι ἠ ἄπιθι. Either drink or depart.}
\setauthornote{3553}{Lib. de lib. propriis. Hos libros, scio multos spernere, nam felices his se non indigere putant, infelices ad solationem miseriae non sufficere. Et tamen felicibus moderationem, dum inconstantiam humanae felicitatis docent, praestant; infelices si omnia recte aestimare velint, felices reddere possunt.}
\setauthornote{3554}{Nullum medicamentum omnes sanare potest; sunt affectus animi qui prorsus sunt insanabiles? non lamen artis opus sperni debet, aut medicinae, aut philosophae.}
\setauthornote{3555}{The insane consolations of a foolish mind.}
\setauthornote{3556}{Salust. Verba virtutem non addunt, nec imperatoris oratio facile timido fortem.}
\setauthornote{3557}{Job, cap. 16.}
\setauthornote{3558}{Epist. 13. lib. 1.}
\setauthornote{3559}{Hor.}
\setauthornote{3560}{Lib. 2. Essays, cap. 6.}
\setauthornote{3561}{Alium paupertas, alium orbitas, hunc morbi, illum timor, alium injuriae, hunc insidiae, illum uxor, filii distrahunt, Cardan.}
\setauthornote{3562}{Boethius l. 1. met. 5.}
\setauthornote{3563}{Apuleius 4. florid. Nihil homini tam prospere datum divinitus, quin ei admixtum sit aliquid difficultatis, in amplissima quaque laetitia subest quaedam querimonia, conjugatione quadam mellis et fellis.}
\setauthornote{3564}{Si omnes premantur, quis tu es qui solus evadere cupis ab ea lege quae neminem praeterit? cur te non mortalem factum et universi orbis regem fieri non doles?}
\setauthornote{3565}{Puteanus ep. 75. Neque cuiquam praecipue dolendum eo quod accidit universis.}
\setauthornote{3566}{Lorchan. Gallobelgicus lib. 3. Anno 1598. de Belgis. Sed eheu inquis euge quid agemus? ubi pro Epithalamio Bellonae flagellum, pro musica harmonia terribilum lituorum et tubarum audias clangorem, pro taedis nuptialibus, villarum, pagorum, urbium videas incendia; ubi pro jubilo lamenta, pro risu fletus aerem complent.}
\setauthornote{3567}{Ita est profecto, et quisquis haec videre abnuis, huic seculi parum aptus es, aut potius nostrorum omnium conditionem ignoras, quibus reciproco quodam nexu laeta tristibus, tristia laetis invicem succedunt.}
\setauthornote{3568}{In Tusc. e vetere poeta.}
\setauthornote{3569}{Cardan lib. 1. de consol. Est consolationis genus non leve, quod a necessitate fit; sive feras, sive non feras, ferendum est tamen.}
\setauthornote{3570}{Seneca.}
\setauthornote{3571}{Omni dolori tempus est medicina; ipsum luctum extinguit, injurias delet, omnis mali oblivionem adfert.}
\setauthornote{3572}{Habet hoc quoque commodum omnis infelicitas, suaviorem vitam cum abierit relinquit.}
\setauthornote{3573}{Virg.}
\setauthornote{3574}{Ovid. For there is no pleasure perfect, some anxiety always intervenes.}
\setauthornote{3575}{Lorchan. Sunt namque infera superis, humana terrenis longe disparia. Etenim beatae mentes feruntur libere, et sine ullo impedimento, stellae, aethereique orbes cursus et conversiones suas jam saeculis innumerabilibus constantissime conficiunt; verum homines magnis angustiis. Neque hac naturae lege est quisquam mortalium solutus.}
\setauthornote{3576}{Dionysius Halicar. lib. 8. non enim unquam contigit, nec post homines natos invenies quenquam, cui omnia ex animi sententia successerint, ita ut nulla in re fortuna sit ei adversata.}
\setauthornote{3577}{Vit. Gonsalvi lib. ult. ut ducibus fatale sit clarissimis a culpa sua, secus circumveniri cum malitia et invidia, imminutaque dignitate per contumeliam mori.}
\setauthornote{3578}{In terris purum illum aetherem non invenies, et ventos serenos; nimbos potius, procellas, calumnias. Lips. cent. misc. ep. 8.}
\setauthornote{3579}{Si omnes homines sua mala suasque curas in unum cumulum conferrent, aequis divisuri portionibus, \&c.}
\setauthornote{3580}{Hor. ser. lib. 1.}
\setauthornote{3581}{Quod unusquisque propria mala novit, aliorum nesciat, in causa est, ut se inter alios miserum putet. Cardan, lib. 3. de consol. Plutarch de consol, ad Apollonium.}
\setauthornote{3582}{Quam multos putas qui se coelo proximos putarent, totidem regulos, si de fortunae tuae reliquiis pars iis minima contingat. Boeth. de consol. lib. 2. pros. 4.}
\setauthornote{3583}{You know the value of a thing from wanting more than from enjoying it.}
\setauthornote{3584}{Hesiod. Esto quod es; quod sunt alii, sine quemlibet esse; Quod non es, nolis; quod potes esse, velis.}
\setauthornote{3585}{Aesopi fab.}
\setauthornote{3586}{Seneca.}
\setauthornote{3587}{Si dormirent semper omnes, nullus alio felicior esset. Card.}
\setauthornote{3588}{Seneca de ira.}
\setauthornote{3589}{Plato, Axiocho. An ignoras vitam hanc peregrinationem, \&c. quam sapiences cum gaudio percurrunt.}
\setauthornote{3590}{Sic expedit; medicus non dat quod patiens vult, sed quod ipse bonum scit.}
\setauthornote{3591}{Frumentum non egreditur nisi trituratum, \&c.}
\setauthornote{3592}{Non est poena damnantis sed flagellum corrigentis.}
\setauthornote{3593}{Ad haereditatem aeternam sic erudimur.}
\setauthornote{3594}{Confess. 6.}
\setauthornote{3595}{Nauclerum tempestas, athletam stadium, ducem pugna, magnanimum calamitas, Christianum vero tentatio probat et examinat.}
\setauthornote{3596}{Sen. Herc. fur. The way from the earth to the stars is not so downy.}
\setauthornote{3597}{Ideo Deus asperum fecit iter, ne dum delectantur in via, obliviscantur eorum quae sunt in patria.}
\setauthornote{3598}{Boethius l. 5. met. ult, Go now, brave fellows, whither the lofty path of a great example leads. Why do you stupidly expose your backs? The earth brings the stars to subjection.}
\setauthornote{3599}{Boeth. pro. ult. Manet spectator cunctorum desuper praescius deus, bonis proemia, malis supplicia dispensans.}
\setauthornote{3600}{Lib. de provid. voluptatem capiunt dii siquando magnos viros colluctantes cum calamitate vident.}
\setauthornote{3601}{Ecce spectaculum Deo dignum. Vir fortis mala fortuna compositus.}
\setauthornote{3602}{1 Pet. v. 7. Psal. lv. 22.}
\setauthornote{3603}{Raro sub eodem lare honestas et forma habitant.}
\setauthornote{3604}{Josephus Mussus vita ejus.}
\setauthornote{3605}{Homuncio brevis, macilentus, umbra hominis, \&c. Ad stuporem ejus eruditionem et eloquentiam admirati sunt.}
\setauthornote{3606}{Nox habet suas voluptates.}
\setauthornote{3607}{Lib. 5, ad finem, caecus potest esse sapiens et beatus, \&c.}
\setauthornote{3608}{In Convivio lib. 25.}
\setauthornote{3609}{Joachimus Camerarius vit. ejus.}
\setauthornote{3610}{Riber. vit. ejus.}
\setauthornote{3611}{Macrobius.}
\setauthornote{3612}{Sueton. c. 7. 9.}
\setauthornote{3613}{Lib. 1. Corpore exili et despecto, sed ingenio et prudentia longe aute se reges caeteros praeveniens.}
\setauthornote{3614}{Alexander Gaguinis hist. Polandiae. Corpore parvus eram, cubito vix altior uno, Sed tamen in parvo corpore magnus eram.}
\setauthornote{3615}{Ovid.}
\setauthornote{3616}{Vir. Aenei. 10.}
\setauthornote{3617}{If the fates give you large proportions, do you not require faculties?}
\setauthornote{3618}{Lib. 2. cap. 20. oneri est illis corporis moles, et spiritus minus vividi.}
\setauthornote{3619}{Corpore breves prudentiores quum coaretata sit anima. Ingenio pollet cui vim natura negavit.}
\setauthornote{3620}{Multis ad salutem animae profuit corporis aegritudo, Petrarch.}
\setauthornote{3621}{Lib. 7. Summa est totius Philosophiae, si tales, \&c.}
\setauthornote{3622}{When we are sick we are most amiable.}
\setauthornote{3623}{Plinius epist. 7. lib. Quem infirmum libido solicitat, aut avaritia, aut honores? nemini invidet, neminem miratur, neminem despicit, sermone maligno non alitur.}
\setauthornote{3624}{Non terret princeps, magister, parens, judex; at aegritudo superveniens, omnia correxit.}
\setauthornote{3625}{Nat. Chytraeus Europ. deliciis. Labor, dolor, aegritudo, luctus, servire superbis dominis, jugum ferre superstitionis, quos habet charos sepelire, \&c. condimenta vitae sunt.}
\setauthornote{3626}{Non tam mari quam proelio virtus, etiam lecto exhibetur: vincetur aut vincet; aut tu febrem relinques, aut ipsa te. Seneca.}
\setauthornote{3627}{Tullius lib. 7. fam. ep. Vesicae morbo laborans, et urinae mittendae difficultate tanta, ut vix incrementum caperet; repellebat haec omnia animi gaudium ob memoriam inventorum.}
\setauthornote{3628}{Boeth. lib. 2. pr. 4. Huic sensus exuperat, sed est pudori degener sanguis.}
\setauthornote{3629}{Gaspar Ens polit. thes.}
\setauthornote{3630}{Does such presumption in your origin possess you?}
\setauthornote{3631}{Alii pro pecunia emunt nobilitatem, alii illam lenocinio, alii veneficiis, alii parricidiis; multis perditio nobilitate conciliat, plerique adulatione, detractione, calumniis, \&c. Agrip. de vanit. scien.}
\setauthornote{3632}{Ex. homicidio saepe orta nobilitas et strenua carnificina.}
\setauthornote{3633}{Plures ob prostitutas filias, uxores, nobiles facti; multos venationes, rapinae, caedes, praestigia, \&c.}
\setauthornote{3634}{Sat. Menip.}
\setauthornote{3635}{Cum enim hos dici nobiles videmus, qui divitiis abundant, divitiae vero raro virtutis sunt comites, quis non videt ortum nobilitatis degenerem? hunc usurae ditarunt, illum spolia, proditiones; hic veneficiis ditatus, ille adulationibus, huic adulteria lucrum praebent, nonullis mendacia, quidam ex conjuge quaestum faciunt, plerique ex natis, \&c. Florent. hist. lib. 3.}
\setauthornote{3636}{Juven. A shepherd, or something that I should rather not tell.}
\setauthornote{3637}{Robusta improbitas a tyrannide incepta, \&c.}
\setauthornote{3638}{Gasper Ens thesauro polit.}
\setauthornote{3639}{Gresserus Itinerar. fol. 266.}
\setauthornote{3640}{Hor. Nobility without wealth is more worthless than seaweed.}
\setauthornote{3641}{Syl. nup. lib. 4. num. 111.}
\setauthornote{3642}{Exod. xxxii.}
\setauthornote{3643}{Omnium nobilium sufficientia in eo probatur si venatica noverint, si aleam, si corporis vires ingentibus poculis commonstrent, si naturae robur numerosa venere probent, \&c.}
\setauthornote{3644}{Difficile est, ut non sit superbus dives, Austin. ser. 24.}
\setauthornote{3645}{Nobilitas nihil aliud nisi improbitas, furor, rapina, latrocinium, homicidium, luxus, venatio, violentia, \&c.}
\setauthornote{3646}{The fool took away my lord in the mask, 'twas apposite.}
\setauthornote{3647}{De miser. curial. Miseri sunt, inepti sunt, turpes sunt, multi ut parietes aedium suarum speciosi.}
\setauthornote{3648}{Miraris aureos vestes, equos, canes, ordinem famulorum, lautas mensas, aedes, villas, praedia, piscinas, sylvas, \&c. haec omnia stultus assequi potest. Pandalus noster lenocinio nobilitatus est, Aeneas Sylvius.}
\setauthornote{3649}{Bellonius observ. lib. 2.}
\setauthornote{3650}{Mat. Riccius lib. 1. cap. 3. Ad regendam remp. soli doctores, aut licentiati adsciscuntur, \&c.}
\setauthornote{3651}{Lib. 1. hist, conditione servus, caeterum acer bello, et animi magnitudine maximorum regum nemini secundus: ob haec a Mameluchis in regem electus.}
\setauthornote{3652}{Olaus Magnus lib. 18. Saxo Grammaticus, a quo rex Sueno et caetera Danorum regum stemmata.}
\setauthornote{3653}{Seneca de Contro. Philos. epist.}
\setauthornote{3654}{Corpore sunt et animo fortiores spurii, plerumque ob amoris vehementiam, seminis crass. \&c.}
\setauthornote{3655}{Vita Castruccii. Nec praeter rationem mirum videri debet, si quis rem considerare velit, omnes eos vel saltem maximam partem, qui in hoc terrarum orbe res praestantiores aggressi sunt, atque inter caeteros aevi sui heroas excelluerunt, aut obscuro, aut abjecto loco editos, et prognatos fuisse abjectis parentibus. Eorum ego Catalogum infinitum recensere possem.}
\setauthornote{3656}{Exercit. 265.}
\setauthornote{3657}{It is a thing deserving of our notice, that most great men were born in obscurity, and of unchaste mothers.}
\setauthornote{3658}{Flor. hist. l. 3. Quod si nudos nos conspici contingat, omnium una eademque erit facies; nam si ipsi nostras, nos eorum vestes induamus, nos, \&c.}
\setauthornote{3659}{Ut merito dicam, quod simpliciter sentiam, Paulum Schalichium scriptorem, et doctorem, pluris facio quam comitem Hunnorum, et Baronem Skradinum; Encyclopaediam tuam, et orbem disciplinarum omnibus provinciis antefero. Balaeus epist. nuncupat. ad 5 cent, ultimam script. Brit.}
\setauthornote{3660}{Praefat hist. lib. 1. virtute tua major, quam aut Hetrusci imperii fortuna, aut numerosa et decora prolis felicitate beatior evadis.}
\setauthornote{3661}{Curtius.}
\setauthornote{3662}{Bodine de rep. lib. 3. cap. 8.}
\setauthornote{3663}{Aeneas Silvius, lib. 2. cap. 29.}
\setauthornote{3664}{If children be proud, haughty, foolish, they defile the nobility of their kindred, Eccl. xxii, 8.}
\setauthornote{3665}{Cujus possessio nec furto eripi, nec incendio absumi, nec aquarum voragine absorberi, vel vi morbi destrui potest.}
\setauthornote{3666}{Send them both to some strange place naked, ad ignotos, as Aristippus said, you shall see the difference. Bacon's Essays.}
\setauthornote{3667}{Familiae splendor nihil opis attulit, \&c.}
\setauthornote{3668}{Fluvius hic illustris, humanarum rerum imago, quae parvis ductae sub initiis, in immensum crescunt, et subito evanescunt. Exilis hic primo flavius, in admirandam magnitudinem excrescit, tandemque in mari Euxino evanescit. I. Stuckius pereg. mar. Euxini.}
\setauthornote{3669}{For fierce eagles do not procreate timid ring-doves.}
\setauthornote{3670}{Sabinus in 6. Ovid. Met. fab. 4.}
\setauthornote{3671}{Lib. 1. de 4. Complexionibus.}
\setauthornote{3672}{Hor. ep. Od. 2. And although he boast of his wealth, Fortune has not changed his nature.}
\setauthornote{3673}{Lib. 2. ep. 15. Natus sordido tuguriolo et paupere domo, qui vix milio rugientem ventrem, \&c.}
\setauthornote{3674}{Nihil fortunato insipiente intolerabilius.}
\setauthornote{3675}{Claud. l. 9. in Eutrop.}
\setauthornote{3676}{Lib. 1. de Rep. Gal. Quoniam et commodiore utuntur conditione, et honestiore loco nati, jam inde a parvulis ad morum civilitatem educati sunt, et assuefacti.}
\setauthornote{3677}{Nullum paupertate gravius onus.}
\setauthornote{3678}{Ne quis irae divinae judicium putaret, aut paupertas exosa foret. Gault. in cap. 2. ver. 18. Lucae.}
\setauthornote{3679}{Inter proceres Thebanos numeratus, lectum habuit genus, frequens famulitium, domus amplas, \&c. Apuleius Florid. l. 4.}
\setauthornote{3680}{P. Blesensis ep. 72. et 232. oblatos respui honores ex onere metiens; motus arabitiosos rogatus non ivi, \&c.}
\setauthornote{3681}{Sudat pauper foras in opere, dives in cogitatione; hic os aperit oscitatione, ille ructatione; gravius ille fastidio, quam hic inedia cruciatur. Ber. ser.}
\setauthornote{3682}{In Hysperchen. Natura aequa est, puerosque videmus mendicorum nulla ex parte regum filiis dissimiles, plerumque saniores.}
\setauthornote{3683}{Gallo Tom. 2.}
\setauthornote{3684}{Et e contubernio foedi atque olidi ventris mors tandem educit. Seneca ep. 103.}
\setauthornote{3685}{Divitiarum sequela, luxus, intemperies, arroganta, superbia, furor injustus, omnisque irrationibilis motus.}
\setauthornote{3686}{Juven. Sat. 6. Effeminate riches have destroyed the age by the introduction of shameful luxury.}
\setauthornote{3687}{Saturn. Epist.}
\setauthornote{3688}{Vos quidem divites putatis felices, sed nescitis eorum miserias.}
\setauthornote{3689}{Et quota pars haec eorum quae istos discruciant? si nossetis metus et curas, quibus obnoxii sunt, plane fugiendas vobis divitias existimaretis.}
\setauthornote{3690}{Seneca in Herc. Oeteo.}
\setauthornote{3691}{Et diis similes stulta cogitatio facit.}
\setauthornote{3692}{Flamma simul libidinis ingreditur; ira, furor et superbia, divitiarum sequela. Chrys.}
\setauthornote{3693}{Omnium oculis, odio, insidiis expositus, semper solicitus, fortunae ludibrium.}
\setauthornote{3694}{Hor. 2. 1. od. 10.}
\setauthornote{3695}{Quid me felicem toties jactastis amici? Qui cecidit, stabili non fuit ille loco. Boeth.}
\setauthornote{3696}{Ut postquam impinguati fuerint, devorentur.}
\setauthornote{3697}{Hor. Although a hundred thousand bushels of wheat may have been threshed in your granaries, your stomach will not contain more than mine.}
\setauthornote{3698}{Cap. 6. de curat. graec. affect. rap. de providentia; quotiescunque divitiis affluentem hominem videmus, cumque pessimum, ne quaeso hunc beatissimum putemus, sed infelicem, censeamus, \&c.}
\setauthornote{3699}{Hor. l. 2. Od. 9.}
\setauthornote{3700}{Hor. lib. 2.}
\setauthornote{3701}{Florid. lib. 4. Dives ille cibo interdicitur, et in omni copia sua cibum non accipit, cum interea totum ejus servitium hilare sit, atque epuletur.}
\setauthornote{3702}{Epist. 115.}
\setauthornote{3703}{Hor. et mihi curto Ire licet mulo vel si libet usque Tarentum.}
\setauthornote{3704}{Brisonius.}
\setauthornote{3705}{Si modum excesseris, suavissima sunt molesta.}
\setauthornote{3706}{Et in cupidiis gulae, coquus et pueri illotis manibus ab exoneratione ventris omnia tractant, \&c. Cardan. l. 8. cap. 46. de rerum varielate.}
\setauthornote{3707}{Epist.}
\setauthornote{3708}{Plin. lib. 57. cap. 6.}
\setauthornote{3709}{Zonaras 3. annal.}
\setauthornote{3710}{Plutarch. vit. ejus.}
\setauthornote{3711}{Hor Ser. lib. 1. Sat. 2.}
\setauthornote{3712}{Cap. 30. nullam vestem his induit.}
\setauthornote{3713}{Ad generum Cereris sine caede et sanguine pauci descendunt reges, et sicca morte tyranni.}
\setauthornote{3714}{God shall deliver his soul from the power of the grave, Psal. xlix. 15.}
\setauthornote{3715}{Contempl. Idiot. Cap. 37. divitiarum acquisitio magni laboris, possessio magni timoris, arnissio magni doloris.}
\setauthornote{3716}{Boethius de consol. phil. l. 3. How contemptible stolid minds! They covet riches and titles, and when they have obtained these commodities of false weight and measures, then, and not before, they understand what is truly valuable.}
\setauthornote{3717}{Austin in Ps. lxxvi. omnis Philosophiae magistra, ad coelum via.}
\setauthornote{3718}{Bonaae mentis soror paupertas.}
\setauthornote{3719}{Paedagoga pietatis sobria, pia mater, cultu simplex, habitu secura, consilio benesuada. Apul.}
\setauthornote{3720}{Cardan. Opprobrium non est paupertas: quod latro eripit, aut pater non reliquit, cur mihi vitio daretur, si fortuna divitias invidit? non aquilae, non, \&c.}
\setauthornote{3721}{Tully.}
\setauthornote{3722}{Epist. 74. servus summe homo; servus sum, immo contubernalis, servus sum, at humilis amicus, immo conservus si cogitaveris.}
\setauthornote{3723}{Epist. 66 et 90.}
\setauthornote{3724}{Panormitan. rebus gestis Alph.}
\setauthornote{3725}{Lib. 4. num. 218. quidam deprehensus quod sederet loco nobilium, mea nobilitas, ait, est circa caput, vestra declinat ad caudam.}
\setauthornote{3726}{Tanto beatior es, quanto collectior.}
\setauthornote{3727}{Non amoribus inservit, non appetit honores, et qualitercunque relictus satis habet, hominem se esse meminit, invidet nemini, neminem despicit, neminem miratur, sermonibus malignis non attendit aut alitur. Plinius.}
\setauthornote{3728}{Politianus in Rustico.}
\setauthornote{3729}{Gyges regno Lydiae inflatus sciscitatum misit Apollinem an quis mortalium se felicior esset. Aglaium Areadum pauperrimum Apollo praetulit, qui terminos agri sui nunquam excesserat, rure suo contentus. Val. lib. 1. c. 7.}
\setauthornote{3730}{Hor. haec est Vita solutorum misera ambitione, gravique.}
\setauthornote{3731}{Amos. 6.}
\setauthornote{3732}{Praefat. lib. 7. Odit naturam quod infra deos sit; irascitur diis quod quis illi antecedat.}
\setauthornote{3733}{De ira cap. 31. lib. 3. Et si multum acceperit, injuriam putat plura non accepisse; non agit pro tribunatu gratias, sed queritur quod non sit ad praeturam perductus; neque haec grata, si desit consulatus.}
\setauthornote{3734}{Lips. admir.}
\setauthornote{3735}{Of some 90,000 inhabitants now.}
\setauthornote{3736}{Read the story at large in John Fox, his Acts and Monuments.}
\setauthornote{3737}{Hor. Sat. 2. ser. lib. 2.}
\setauthornote{3738}{5 Florent. hist. virtus quietem parat, quies otium, otium porro luxum generat, luxus interitum, a quo iterum ad saluberrimas, \&c.}
\setauthornote{3739}{Guicciard. in Hiponest nulla infelicitas subjectum esse legi naturae \&c.}
\setauthornote{3740}{Persius.}
\setauthornote{3741}{Omnes divites qui coelo et terra frui possunt.}
\setauthornote{3742}{Hor. lib. 1. epis. 12.}
\setauthornote{3743}{Seneca epist. 15. panem et aquam natura desiderat, et haec qui habet, ipso cum Jove de felicitate contendat. Cibus simplex famem sedat, vestis tenuis frigius arcet. Senec. epist. 8.}
\setauthornote{3744}{Boethius.}
\setauthornote{3745}{Muffaes et alii.}
\setauthornote{3746}{Brissonius.}
\setauthornote{3747}{Psal. lxxxiv.}
\setauthornote{3748}{Si recte philosophemini, quicquid aptam moderationem supergreditur, oneri potius quam usui est.}
\setauthornote{3749}{Lib. 7. 16. Cereris munus et aquae poculum mortales quaerunt habere, et quorum saties nunquam est, luxus autem, sunt caetera, non epulae.}
\setauthornote{3750}{Satis est dives qui pane non indiget; nimium potens qui servire non cogitur. Ambitiosa non est fames, \&c.}
\setauthornote{3751}{Euripides menalip. O fili, mediocres divitiae hominibus conveniunt, nimia vero moles perniciosa.}
\setauthornote{3752}{Hor.}
\setauthornote{3753}{O noctes coenaeque deum.}
\setauthornote{3754}{Per mille fraudes doctosque dolos ejicitur, apud sociam paupertatem ejusque cultores divertens in eorum sinu et tutela deliciatur.}
\setauthornote{3755}{Lucan. O protecting quality of a poor man's life, frugal means, gifts scarce yet understood by the gods themselves.}
\setauthornote{3756}{Lip. miscell. ep. 40.}
\setauthornote{3757}{Sat. 6. lib. 2.}
\setauthornote{3758}{Hor. Sat. 4.}
\setauthornote{3759}{Apuleius.}
\setauthornote{3760}{Chytreus in Europae deliciis. Accipite cives Veneti quod est optimum in rebus humanis, res humans contemnere.}
\setauthornote{3761}{Vah, vivere etiam nunc lubet, as Demea said, Adelph. Act. 4. Quam multis non egeo, quam multa non desidero, ut Socrates in pompa, ille in nundinis.}
\setauthornote{3762}{Epictetus 77. cap. quo sum destinatus, et sequar alacriter.}
\setauthornote{3763}{Let whosoever covets it, occupy the highest pinnacle of fame, sweet tranquillity shall satisfy me.}
\setauthornote{3764}{Puteanus ep. 62.}
\setauthornote{3765}{Marullus. The immortal Muses confer imperishable pride of origin.}
\setauthornote{3766}{Hoc erit in votis, modus agri non ita parvus, Hortus ubi et tecto vicinus jugis aquae fons, et paulum sylvae, \&c. Hor. Sat. 6. lib. 2. Ser.}
\setauthornote{3767}{Hieronym.}
\setauthornote{3768}{Seneca consil. ad Albinum c. 11. qui continet se intra naturae limites, paupertatem non sentit; qui excedit, eum in opibus paupertas sequitur.}
\setauthornote{3769}{Hom. 12. pro his quae accepisti gratias age, noli indignare pro his quae non accepisti.}
\setauthornote{3770}{Nat. Chytreus deliciis Europ. Gustonii in aedibus Hubianis in coenaculo e regione mensae. If your table afford frugal fare with peace, seek not, in strife, to load it lavishly.}
\setauthornote{3771}{Quid non habet melius pauper quam dives? vitam, valetudinem, cibum, somnum, libertatem, \&c. Card.}
\setauthornote{3772}{Martial. l. 10. epig. 47. read it out thyself in the author.}
\setauthornote{3773}{Confess. lib. 6. Transiens per vicum quendam Mediolanensem, animadverti pauperem quendam mendicum, jam credo saturum, jocantem atque ridentem, et ingemui et locutus sum cum amicis qui mecum erant, \&c.}
\setauthornote{3774}{Et certe ille laetabatur, ego anxius; securus ille, ego trepidus. Et si percontaretur me quisquam an exultare mallem, an metuere, responderem, exultare: et si rursus interrogaret an ego talis essem, an qualis nunc sum, me ipsis curis confectum eligerem; sed perversitate, non veritate.}
\setauthornote{3775}{Hor.}
\setauthornote{3776}{Hor. ep. lib. 1.}
\setauthornote{3777}{O si nunc morerer, inquit, quanta et qualia mihi imperfecta manerent: sed si mensibus decem vel octo super vixero, omnia redigam ad libellum, ab omni debito creditoque me explicabo; praetereunt interim menses decem, et octo, et cum illis anni, et adhuc restant plura quam prius; quid igitur speras. O insane, finem quem rebus tuis non inveneras in juventa, in senecta impositurum? O dementiam, quum ob curas et negotia tuo judicio sis infelix, quid putas futuram quum plura supererint? Candan lib. 8. cap. 40. de rer. var.}
\setauthornote{3778}{Plutarch.}
\setauthornote{3779}{Lib. de natali. cap. 1.}
\setauthornote{3780}{Apud Stobeum ser. 17.}
\setauthornote{3781}{Hom. 12. in 2.}
\setauthornote{3782}{Non in paupertate, sed in paupere (Senec.) non re, sed opinione labores.}
\setauthornote{3783}{Vobiscus Aureliano, sed si populus famelicus inedia laboret, nec arma, leges, pudor, magistratus, coercere valent.}
\setauthornote{3784}{One of the richest men in Rome.}
\setauthornote{3785}{Serm. Quidam sunt qui pauperes esse volunt ita ut nihil illis desit, sic commendant ut nullam patiantur inopiam; sunt et alii mites, quamdiu dicitur et agitur ad eorum arbitrium, \&c.}
\setauthornote{3786}{Nemo paupertatem commendaret nisi pauper.}
\setauthornote{3787}{Petronius Catalec.}
\setauthornote{3788}{Ovid. There is no space left on our bodies for a fresh stripe.}
\setauthornote{3789}{Ovid.}
\setauthornote{3790}{Plutarch. vit. Crassi.}
\setauthornote{3791}{Lucan. lib. 9.}
\setauthornote{3792}{An quum super fimo sedit Job, an eum omnia abstulit diabolus, \&c. pecuniis privatus fiduciam deo habuit, omni thesauro preciosiorem.}
\setauthornote{3793}{Haec videntes sponte philosophemini, nec insipientum affectibus agitemur.}
\setauthornote{3794}{1 Sam. i. 8.}
\setauthornote{3795}{James i. 2. My brethren, count it an exceeding joy, when you fall into divers temptations.}
\setauthornote{3796}{Afflictio dat intellectum; quos Deus diligit castigat. Deus optimum quemque aut mala valetudine aut luctu afficit. Seneca.}
\setauthornote{3797}{Quam sordet mihi terra quum coelum intueor.}
\setauthornote{3798}{Senec. de providentia cap. 2. Diis ita visum, dii melius norunt quid sit in commodum meum.}
\setauthornote{3799}{Hom. Iliad. 4.}
\setauthornote{3800}{Hom. 9. voluit urbem tyrannus evertere, et Deus non prohibuit; voluit captivos ducere, non impedivit; voluit ligare, concessit, \&c.}
\setauthornote{3801}{Psal. cxiii. De terra inopem, de stercore erigit pauperem.}
\setauthornote{3802}{Micah. viii. 7.}
\setauthornote{3803}{Preme, preme, ego cum Pindaro, ἀβάπτιστος ὲιμι ως φελλος ὑπ' ἐλμα immersibillis sum sicut suber super maris septum. Lipsius.}
\setauthornote{3804}{Hic ure, hic seca, ut in aeternum parcas, Austin. Diis fruitur iratis, superat et crescit malis. Mutium ignis, Fabricium paupertas, Regulum tormenta, Socratem venenum superare non potuit.}
\setauthornote{3805}{Hor. epist. 16. lib. 1.}
\setauthornote{3806}{Hom. 5. Auferet pecunias? at habet in coelis: patria dejiciet? at in coelestem civitatem mittet: vincula injiciet? at habet solutam conscientiam: corpus interficiet, at iterum resurget; cum umbra pugnat qui cum justo pugnat.}
\setauthornote{3807}{Leonides.}
\setauthornote{3808}{Modo in pressura, in tentationibus, erit postea bonum tuum requies, aeternitas, immortalitas.}
\setauthornote{3809}{Dabit Deus his quoque finem.}
\setauthornote{3810}{Seneca.}
\setauthornote{3811}{Nemo desperet meliora lapsus.}
\setauthornote{3812}{Theocritus. Hope on, Battus, tomorrow may bring better luck; while there's life there's hope.}
\setauthornote{3813}{Ovid.}
\setauthornote{3814}{Ovid.}
\setauthornote{3815}{Thales.}
\setauthornote{3816}{Lib. 7. Flor. hist. Omnium felicissimus, et locupletissimus, \&c. incarceratus saepe adolescentiam periculo mortis habuit, solicitudinis et discriminis plenam, \&c.}
\setauthornote{3817}{Laetior successit securitas quae simul cum divitiis cohabitare nescit. Camden.}
\setauthornote{3818}{Pecuniam perdidisti, fortassis illa te perderet manens. Seneca.}
\setauthornote{3819}{Expeditior es ob pecuniarum jacturam. Fortuna opes auferre, non animum potest. Seneca.}
\setauthornote{3820}{Hor. Let us cast our jewels and gems, and useless gold, the cause of all vice, into the sea, since we truly repent of our sins.}
\setauthornote{3821}{Jubet me posthac fortuna expeditius Philosophari.}
\setauthornote{3822}{I do not desire riches, nor that a price should be set upon me.}
\setauthornote{3823}{In frag. Quirites, multa mihi pericula domi, militae multa adversa fuere, quorum alia toleravi, alia deorum auxilio repuli et virtute mea; nunquam animus negotio defuit, nec decretis labor; nullae res nec properae nec adversae ingenium mutabant.}
\setauthornote{3824}{Qualis mundi statis supra lunam semper serenus.}
\setauthornote{3825}{Bona meus nullum tristioris fortunae recipit incursum, Val. lib. 4. c. 1. Qui nil potest sperare, desperet nihil.}
\setauthornote{3826}{Hor.}
\setauthornote{3827}{Aequam. memento rebus in arduis servare mentem, lib. 2. Od. 3.}
\setauthornote{3828}{Epict. c. 18.}
\setauthornote{3829}{Ter. Adel. act. 4. Sc. 7.}
\setauthornote{3830}{Unaquaeque res duas habet ansas, alternam quae teneri, alteram quae non potest; in manu nostra quam volumus accipere.}
\setauthornote{3831}{Ter. And. Act. 4. sc. 6.}
\setauthornote{3832}{Epictetus. Invitatus ad convivium, quae apponuntur comedis, non quaeris ultra; in mundo multa rogitas quae dii negant.}
\setauthornote{3833}{Cap. 6. de providentia. Mortales cum sint rerum omnium indigi, ideo deus aliis divitias, aliis paupertatem distribuit, ut qui opibus pollent, materiam subministrent; qui vero inopes, exercitatas artibus manus admoveant.}
\setauthornote{3834}{Si sint omnes equales, necesse est ut omnes fame pereant; quis aratro terram sulcaret, quis sementem faceret, quis plantas sereret, quis vinum exprimeret?}
\setauthornote{3835}{Liv. lib. 1.}
\setauthornote{3836}{Lib. 3. de cons.}
\setauthornote{3837}{Seneca.}
\setauthornote{3838}{Vide Isaacum Pontanum descript. Amsterdam. lib. 2. c. 22.}
\setauthornote{3839}{Vide Ed. Pelham's book edit. 1630.}
\setauthornote{3840}{Heautontim. Act. 1. Sc. 2.}
\setauthornote{3841}{Epist. 98. Omni fortuna valentior ipse animus, in utramque partem res suas ducit, beataeque ac miserae vitae sibi causa est.}
\setauthornote{3842}{Fortuna quem nimium fovet stultum facil. Pub. Mimus.}
\setauthornote{3843}{Seneca de beat. vit. cap. 14. miseri si deserantur ab ea, miseriores si obruantur.}
\setauthornote{3844}{Plutarch, vit. ejus.}
\setauthornote{3845}{Hor. epist. l. 1. ep. 18.}
\setauthornote{3846}{Hor.}
\setauthornote{3847}{Boeth. 2.}
\setauthornote{3848}{Epist. lib. 3. vit. Paul. Ermit. Libet eos nunc interrogare qui domus marmoribus vestiunt, qui uno filo villarum ponunt precia, huic seni modo quid unquam defuit? vos gemma bibitis, ille concavis manibus naturae satisfecit; ille pauper paradisum capit, vos avaros gehenna suscipiet.}
\setauthornote{3849}{It matters little whether we are enslaved by men or things.}
\setauthornote{3850}{Satur. l. 11. Alius libidini servit, alius ambitioni, omnes spei, omnes timori.}
\setauthornote{3851}{Nat. lib. 3.}
\setauthornote{3852}{Consol. l. 5.}
\setauthornote{3853}{O generose, quid est vita nisi carcer animi!}
\setauthornote{3854}{Herbastein.}
\setauthornote{3855}{Vertomannus navig. l. 2. c. 4. Commercia in nundinis noctu hora secunda ob nimios qui saeviunt interdiu aestus exercent.}
\setauthornote{3856}{Ubi verior contemplatio quam in solitudine? ubi studium solidius quam in quiete?}
\setauthornote{3857}{Alex. ab Alex. gen. dier. lib. 1. cap. 2.}
\setauthornote{3858}{In Ps. lxxvi. non ita laudatur Joseph cum frumenta distribueret, ac quum carcerem habitaret.}
\setauthornote{3859}{Boethius.}
\setauthornote{3860}{Philostratus in deliciis. Peregrini sunt imbres in terra et fluvii in mari Jupiter apud Aegyptos, sol apud omnes; hospes anima in corpore, luscinia in aere, hirundo in domo, Ganymedes coelo, \&c.}
\setauthornote{3861}{Lib. 16. cap. 1. Nullam frugem habent potus ex imbre: Et hae gentes si vincantur, \&c.}
\setauthornote{3862}{Lib. 5. de legibus. Cumque cognatis careat et amicis, majorem apud deos et apud homines misericordiam meretur.}
\setauthornote{3863}{Cardan, de consol. lib. 2.}
\setauthornote{3864}{Seneca.}
\setauthornote{3865}{Benzo.}
\setauthornote{3866}{Summo mane ululatum oriuntur, pectora percutientes, \&c. miserabile spectaculum exhibentes. Ortelius in Graecia.}
\setauthornote{3867}{Catullus.}
\setauthornote{3868}{Virgil. I live now, nor as yet relinquish society and life, but I shall resign them.}
\setauthornote{3869}{Lucan. Overcome by grief, and unable to endure it, she exclaimed, 'Not to be able to die through sorrow for thee were base.'}
\setauthornote{3870}{3 Annal.}
\setauthornote{3871}{The colour suddenly fled her cheek, the distaff forsook her hand, the reel revolved, and with dishevelled locks she broke away, wailing as a woman.}
\setauthornote{3872}{Virg. Aen. 10. Transfix me, O Rutuli, if you have any piety: pierce me with your thousand arrows.}
\setauthornote{3873}{Confess. l. 1.}
\setauthornote{3874}{Juvenalis.}
\setauthornote{3875}{Amator scortum vitae praeponit, iracundus vindictam, parasitus gulam, ambitiosus honores, avarus opes, miles rapinam, fur praedam; morbos odimus et accersimus. Card.}
\setauthornote{3876}{Seneca; quum nos sumus, mors non adest; cum vero mors adest, tum nos non sumus.}
\setauthornote{3877}{Bernard. c. 3. med. nasci miserum, vivere poena, angustia mori.}
\setauthornote{3878}{Plato Apol. Socratis. Sed jam hora est hinc abire, \&c.}
\setauthornote{3879}{Comedi ad satietatem, gravitas me offendit; parcius edi, non est expletum desiderium; venereas delicias sequor, hinc morbus, lassitudo, \&c.}
\setauthornote{3880}{Bern. c. 3. med. de tantilla laetitia, quanta tristitia; post tantam voluptatem quam gravis miseria?}
\setauthornote{3881}{Est enim mors piorum felix transitus de labore ad refrigerium, de expectatione ad praemium, de agone ad bravium.}
\setauthornote{3882}{Vaticanus vita ejus.}
\setauthornote{3883}{Luc.}
\setauthornote{3884}{Il. 9 Homer. It is proper that, having indulged in becoming grief for one whole day, you should commit the dead to the sepulchre.}
\setauthornote{3885}{Ovid.}
\setauthornote{3886}{Consol. ad Apolon. non est libertate nostra positum non dolere, misericordiam abolet, \&c.}
\setauthornote{3887}{Ovid, 4 Trist.}
\setauthornote{3888}{Tacitus lib. 4.}
\setauthornote{3889}{Lib. 9. cap. 9. de civitate Dei. Non quaero cum irascatur sed cur, nor utrum sit tristis sed unde, non utrum timeat sed quid timeat.}
\setauthornote{3890}{Festus verbo minuitur. Luctui dies indicebatur cum liberi nascantur, cum frater abit, amicus ab hospite captivus domum redeat, puella desponsetur.}
\setauthornote{3891}{Ob hanc causam mulieres ablegaram ne talia facerent; nos haec audientes erubuimus et destitimus a lachrymis.}
\setauthornote{3892}{Lib. 1. class. 8. de Claris. Jurisconsultis Patavinis.}
\setauthornote{3893}{12. Innuptae puellae amictae viridibus pannis, \&c.}
\setauthornote{3894}{Lib. de consol.}
\setauthornote{3895}{Praeceptis philosophiae confirmatus adversus omnem fortunae vim, et te consecrata in coelumque recepta, tanta affectus laetitia sum ac voluptate, quantam animo capere possum, ac exultare plane mihi videor, victorque de omni dolore et fortuna triumphare.}
\setauthornote{3896}{Ut lignum uri natum, arista secari, sic homines mori.}
\setauthornote{3897}{Boeth. lib. 2. met. 3.}
\setauthornote{3898}{Boeth.}
\setauthornote{3899}{Nic. Hensel. Breslagr. fol. 47.}
\setauthornote{3900}{Twenty then present.}
\setauthornote{3901}{To Magdalen, the daughter of Charles the Seventh of France. Obeunt noctesque diesque, \&c.}
\setauthornote{3902}{Assyriorum regio funditus deleta.}
\setauthornote{3903}{Omnium quot unquam Sol aspexit urbium maxima.}
\setauthornote{3904}{Ovid. What of ancient Athens but the name remains?}
\setauthornote{3905}{Arcad. lib. 8.}
\setauthornote{3906}{Praefat. Topogr. Constantinop.}
\setauthornote{3907}{Nor can its own structure preserve the solid globe.}
\setauthornote{3908}{Epist. Tull. lib. 3.}
\setauthornote{3909}{Quum tot oppidorum cadavera ante oculus projecta jacent.}
\setauthornote{3910}{Hor. lib. 1. Od. 24.}
\setauthornote{3911}{De remed. fortuit.}
\setauthornote{3912}{Erubesce tanta tempestate quod ad unam anchoram stabas.}
\setauthornote{3913}{Vis aegrum, et morbidum, fitibundum-gaude potius quod his malis liberatus sit.}
\setauthornote{3914}{Uxorem bonam aut invenisti, aut sic fecisti; si inveneris, aliam habere te posse ex hoc intelligamus: si feceris, bene speres, salvus est artifex.}
\setauthornote{3915}{Stulti est compedes licet aureas amare.}
\setauthornote{3916}{Hor.}
\setauthornote{3917}{Hor. lib. 1. Od. 24.}
\setauthornote{3918}{Virg. 4. Aen.}
\setauthornote{3919}{Cap. 19. Si id studes ut uxor, amici, liberi perpetuo vivant, stultus es.}
\setauthornote{3920}{Deos quos diligit juvenes rapit, Menan.}
\setauthornote{3921}{Consol. ad Apol. Apollonius filius tuus in flore decessit, ante nos ad aeternitatem digressus, tanquam e convivio abiens, priusquam in errorem aliquem e temulentia incideret, quales in longa senecta accidere solent.}
\setauthornote{3922}{Tom. 1. Tract. de luctu. Quid me mortuum miserum vocas, qui te sum multo felicior? aut quid acerbi mihi putas contigisse? an quia non sum malus senex, ut tu facie rugosus, incurvus, \&c. O demens, quid tibi videtur in vita boni? nimirum amicitias, caenas, \&c. Longe melius non esurire quam edere; non sitire, \&c. Gaude potius quod morbos et febres effugerim, angorem animi, \&c. Ejulatus quid prodest quid lachryimae, \&c.}
\setauthornote{3923}{Virgil.}
\setauthornote{3924}{Hor.}
\setauthornote{3925}{Chytreus deliciis Europae.}
\setauthornote{3926}{Epist. 85.}
\setauthornote{3927}{Sardus de mor. gen.}
\setauthornote{3928}{Praemeditatione facilem reddere quemque casum. Plutarchus consolatione ad Apollonium. Assuefacere non casibus debemus. Tull. lib. 3. Tusculan. quaest.}
\setauthornote{3929}{Cap. 8. Si ollam diligas, memento te ollam diligere, non perturbaberis ea confracta; si filium aut uxorem, memento hominem. a te diligi, \&c.}
\setauthornote{3930}{Seneca.}
\setauthornote{3931}{Boeth, lib. 1. pros. 4.}
\setauthornote{3932}{Qui invidiam ferre non potest, ferre contemptum cogitur.}
\setauthornote{3933}{Ter. Heautont.}
\setauthornote{3934}{Epictetus c. 14. Si labor objectus fuerit tolerantiae, convicium patientiae, \&c. si ita consueveris, vitiis non obtemperabis.}
\setauthornote{3935}{Ter. Phor.}
\setauthornote{3936}{Alciat Embl.}
\setauthornote{3937}{Virg. Aen.}
\setauthornote{3938}{My breast was not conscious of this first wound, for I have endured still greater.}
\setauthornote{3939}{Nat. Chytreus deliciis Europae, Felix civitas quae tempore pacis de bello cogitat.}
\setauthornote{3940}{Occupat extremum scabies; mihi turpe relinqui est. Hor.}
\setauthornote{3941}{Lipsius epist. quaest. l. 1. ep. 7.}
\setauthornote{3942}{Lipsius epist. lib. I. epist. 7.}
\setauthornote{3943}{Gloria comitem habet invidiam, pari onere premitur retinendo ac acquirendo.}
\setauthornote{3944}{Quid aliud ambitiosus sibi parat quam ut probra ejus pateant? nemo vivens qui non habet in vita plura vitoperatione quam laude digna; his malis non melius occurritur, quam si bene latueris.}
\setauthornote{3945}{Et omnes fama per urbes garrula laudet.}
\setauthornote{3946}{Sen. Her. fur.}
\setauthornote{3947}{Hor. I live like a king without any of these acquisitions.}
\setauthornote{3948}{But all my labour was unprofitable; for while death took off some of my friends, to others I remain unknown, or little liked, and these deceive me with false promises. Whilst I am canvassing one party, captivating another, making myself known to a third, my age increases, years glide away, I am put off, and now tired of the world, and surfeited with human worthlessness. I rest content.}
\setauthornote{3949}{The right honourable Lady Francis Countess Dowager of Exeter. The Lord Berkley.}
\setauthornote{3950}{Distichon ejus in militem Christianum e Graeco. Engraven on the tomb of Fr. Puccius the Florentine in Rome. Chytreus in deliciis.}
\setauthornote{3951}{Paederatus in 300 Lacedaemoniorum numerum non electus risit, gratulari se dicens civitatem habere 300 cives se meliores.}
\setauthornote{3952}{Kissing goes by favour.}
\setauthornote{3953}{Aeneas Syl. de miser. curial. Dantur honores in curiis non secundum honores et virtutes, sed ut quisque ditior est atque potentior, eo magis honoratur.}
\setauthornote{3954}{Sesellius lib. 2. de repub. Gallorum. Favore apud nos et gratia plerumque res agitur; et qui commodum aliquem nacti sunt intercessorem, aditum fere habent ad omnes praefecturas.}
\setauthornote{3955}{Slaves govern; asses are decked with trappings; horses are deprived of them.}
\setauthornote{3956}{Imperitus periti munus occupat, et sic apud vulgus habetur. Ille profitetur mille coronatus, cum nec decem mercatur; alius e diverso mille dignus, vix decem consequi potest.}
\setauthornote{3957}{Epist. dedict. disput. Zeubbeo Bondemontio, et Cosmo Rucelaio.}
\setauthornote{3958}{Quum is qui regnat, et regnandi sit imperitus.}
\setauthornote{3959}{Lib. 22. hist.}
\setauthornote{3960}{Ministri locupletiores sunt iis quibus ministratur.}
\setauthornote{3961}{Hor. lib. 2. Sat. 5. Learn how to grow rich.}
\setauthornote{3962}{Solomon Eccles. ix. 11.}
\setauthornote{3963}{Sat. Menip.}
\setauthornote{3964}{O wretched virtue! you are therefore nothing but words, and I have all this time been looking upon you as a reality, while you are yourself the slave of fortune.}
\setauthornote{3965}{Tale quid est apud Valent. Andream Apolog. manip. 5. apol. 39.}
\setauthornote{3966}{Stella Fomahant immortalitatem dabit.}
\setauthornote{3967}{Lib. de lib. propiis.}
\setauthornote{3968}{Hor. The muse forbids the praiseworthy man to die.}
\setauthornote{3969}{Qui induit thoracem aut galeam, \&c.}
\setauthornote{3970}{Lib. 4. de guber. Dei. Quid est dignitas indigno nisi circulus aureus in naribus suis.}
\setauthornote{3971}{In Lysandro.}
\setauthornote{3972}{Ovid. Met.}
\setauthornote{3973}{Magistratus virum indicat.}
\setauthornote{3974}{Ideo boni viri aliquando gratiam non accipiunt, ne in superbiam eleventur venositate jactantiae, ne altitudo muneris neglentiores efficiat.}
\setauthornote{3975}{Aelian.}
\setauthornote{3976}{Injuriarum remedium est oblivio.}
\setauthornote{3977}{Mat. xviii. 22. Mat. v. 39.}
\setauthornote{3978}{Rom. xii. 17.}
\setauthornote{3979}{Si toleras injuriam, victor evadis; qui enim pecuniis privatus est, non est privatus victoria in hac philosophia.}
\setauthornote{3980}{Dispeream nisi te ultus fuero: dispeream nisi ut me deinceps ames effecero.}
\setauthornote{3981}{Joach. Camerarius Embl. 21. cent. 1.}
\setauthornote{3982}{Heliodorus.}
\setauthornote{3983}{Reipsa reperi nihil esse homini melius facilitate et clementia. Ter. Adelph.}
\setauthornote{3984}{Ovid.}
\setauthornote{3985}{Camden in Glouc.}
\setauthornote{3986}{Usque ad pectus ingressus est, aquam, \&c. cymbam amplectens, sapientissime, rex ait, tua humilitas meam vicit superbiam, et sapientia triumphavit ineptiam; collum ascende quod contra te fatuus erexi, intrabis terram quam hodie fecit tuam benignitas, \&c.}
\setauthornote{3987}{Chrysostom, contumeliis affectus est et eas pertulit; opprobriis, nec ultus est; verberibus caesus, nec vicem reddidit.}
\setauthornote{3988}{Rom. xii. 14.}
\setauthornote{3989}{Pro.}
\setauthornote{3990}{Contend not with a greater man, Pro.}
\setauthornote{3991}{Occidere possunt.}
\setauthornote{3992}{Non facile aut tutum in eum scribere qui potest proscribere.}
\setauthornote{3993}{Arcana tacere, otium recte collocare, injuriam posse ferre, difficillimum.}
\setauthornote{3994}{Psal. xlv.}
\setauthornote{3995}{Rom. xii.}
\setauthornote{3996}{Psa. xiii. 12.}
\setauthornote{3997}{Nullus tam severe inimicum suum ulcisci potest, quam Deus solet miserorum oppressores.}
\setauthornote{3998}{Arcturus in Plaut. He adjudicates judgment again, and punishes with a still greater penalty.}
\setauthornote{3999}{Hor. 3. od. 2.}
\setauthornote{4000}{Wisd. xi. 6.}
\setauthornote{4001}{Juvenal.}
\setauthornote{4002}{Apud Christianos non qui patitur, sed qui facit injuriam miser est. Leo ser.}
\setauthornote{4003}{Neque praecepisset Deus si grave fuisset; sed qua ratione potero? facile si coelum suspexeris; et ejus pulchritudine, et quod pollicetur Deus, \&c.}
\setauthornote{4004}{Valer. lib. 4. cap. 1.}
\setauthornote{4005}{Ep. Q. frat.}
\setauthornote{4006}{Camerarius, emb. 75. cen. 2.}
\setauthornote{4007}{Pape, inquit: nullum animal tam pusillum quod non cupiat ulcisci.}
\setauthornote{4008}{Quod tibi fieri non vis, alteri ne feceris.}
\setauthornote{4009}{1 Pet. ii.}
\setauthornote{4010}{Siquidem malorum proprium est inferre damna, et bonorum pedissequa est injuria.}
\setauthornote{4011}{Alciat. emb.}
\setauthornote{4012}{Naturam expellas furca licet usque recurret.}
\setauthornote{4013}{By many indignities we come to dignities. Tibi subjicito quae fiunt aliis, furtum convitia, \&c. Et in iis in te admissis non excandesces. Epictetus.}
\setauthornote{4014}{Plutarch. quinquagies Catoni dies dicta ab inimicis.}
\setauthornote{4015}{Lib. 18.}
\setauthornote{4016}{Hoc scio pro certo quod si cum stercore certo, vinco seu vincor, semper ego maculor.}
\setauthornote{4017}{Lib. 8. cap. 2.}
\setauthornote{4018}{Obloquutus est, probrumque tibi intulit quispiam, sive vera is dixerit, sive falsa, maximam tibi coronam texueris si mansuete convitium tuleris. Chrys. in 6. cap. ad Rom. ser. 10.}
\setauthornote{4019}{Tullius epist. Dolabella, tu forti sis animo; et tua moderatio, constantia, eorum infamet injuriam.}
\setauthornote{4020}{Boethius consol. lib. 4. pros. 3.}
\setauthornote{4021}{Amongst people in every climate.}
\setauthornote{4022}{Ter. Phor.}
\setauthornote{4023}{Camerar. emb. 61. cent. 3. Why should you regard the harmless shafts of a vain-speaking tongue-does the exalted Diana care for the barking of a dog?}
\setauthornote{4024}{Lipsius elect. lib. 3. ult. Latrant me jaceo, ac taceo, \&c.}
\setauthornote{4025}{Catullus.}
\setauthornote{4026}{The symbol of I. Kevenheder, a Carinthian baron, saith Sambucus.}
\setauthornote{4027}{The symbol of Gonzaga, Duke of Mantua.}
\setauthornote{4028}{Pers. sat. 1.}
\setauthornote{4029}{Magni animi est injurias despicere, Seneca de ira, cap. 31.}
\setauthornote{4030}{Quid turpius quam sapientis vitam ex insipientis sermone pendere? Tullius 2. de finibus.}
\setauthornote{4031}{Tua te conscientia salvare, in cubiculum ingredere, ubi secure requiescas. Minuit se quodammodo proba bonitas conscientiae secretum, Boethius, l. 1. pros. 4.}
\setauthornote{4032}{Ringantur licet et maledicant; Palladium illud pectori oppono, non moveri: consisto modestiae veluti sudi innitens, excipio et frango stultissimum impetum livoris. Putean. lib. 2. epist. 53.}
\setauthornote{4033}{Mil. glor. Act. 3. Plautus.}
\setauthornote{4034}{Bion said his father was a rogue, his mother a whore, to prevent obloquy, and to show that nought belonged to him but goods of the mind.}
\setauthornote{4035}{Lib. 2. ep. 25.}
\setauthornote{4036}{Nosce teipsum.}
\setauthornote{4037}{Contentus abi.}
\setauthornote{4038}{Ne fidas opibus, neque parasitis, trahunt in praecipitium.}
\setauthornote{4039}{Pace cum hominibus habe, bellum cum vitiis. Otho. 2. imperat. symb.}
\setauthornote{4040}{Daemon te nunquam otiosum inveniat. Hieron.}
\setauthornote{4041}{Diu deliberandum quod statuendum est semel.}
\setauthornote{4042}{Insipientis est dicere non putaram.}
\setauthornote{4043}{Ames parentem, si equum, aliter feras; praestes parentibus pietatem, amicis dilectionem.}
\setauthornote{4044}{Comprime linguam. Quid de quoque viro et cui dicas saepe caveto. Libentius audias quam loquaris; vive ut vivas.}
\setauthornote{4045}{Epictetus: optime feceris si ea fugeris quae in alio reprehendis. Nemini dixeris quae nolis efferri.}
\setauthornote{4046}{Fuge sussurones. Percontatorem fugito, \&c.}
\setauthornote{4047}{Sint sales sine vilitate. Sen.}
\setauthornote{4048}{Sponde, presto noxa.}
\setauthornote{4049}{Camerar. emb. 55. cent. 2. cave cui credas, vel nemini fidas Epicarmus.}
\setauthornote{4050}{Tecum habita.}
\setauthornote{4051}{Bis dat qui cito dat.}
\setauthornote{4052}{Post est occasio calva.}
\setauthornote{4053}{Nimia familiaritas parit contemptum.}
\setauthornote{4054}{Mendacium servile vitium.}
\setauthornote{4055}{Arcanum neque inscrutaberis ullius unquam, commissumque teges, Hor. lib. 1, ep. 19. Nec tua laudabis studia aut aliena reprendes. Hor. ep. lib. 18.}
\setauthornote{4056}{Ne te quaesiveris extra.}
\setauthornote{4057}{Stultum est timere, quod vitari non potest.}
\setauthornote{4058}{De re amissa irreparabili ne doleas.}
\setauthornote{4059}{Tant eris aliis quanti tibi fueris.}
\setauthornote{4060}{Neminem esto laudes vel accuses.}
\setauthornote{4061}{Nullius hospitis grata est mora longa.}
\setauthornote{4062}{Solonis lex apud. Aristotelem Gellius lib. 2. cap. 12.}
\setauthornote{4063}{Nullum locum putes sine teste, semper adesse Deum cogita.}
\setauthornote{4064}{Secreto amicos admone, lauda palam.}
\setauthornote{4065}{Ut ameris amabilis esto. Eros et anteros gemelli Veneris, amatio et redamatio. Plat.}
\setauthornote{4066}{Dum fata sinunt vivite laeti, Seneca.}
\setauthornote{4067}{Id apprime in vita utile, ex aliis observare sibi quod ex usu siet. Ter.}
\setauthornote{4068}{Dum furor in cursu currenti cede furori. Cretizandum cum Crete. Temporibus servi, nec contra flamina flato.}
\setauthornote{4069}{Nulla certior custodia innocentia: inexpugnabile munimentum munimento non egere.}
\setauthornote{4070}{Unicuique suum onus intolerabile videtur.}
\setauthornote{4071}{Livius.}
\setauthornote{4072}{Ter. scen. 2. Adelphus.}
\setauthornote{4073}{'Twas not the will but the way that was wanting.}
\setauthornote{4074}{Plautus.}
\setauthornote{4075}{Petronius Catul.}
\setauthornote{4076}{Parmeno Caelestinae, Act. 8. Si stultita dolor esset, in nulla non domo ejulatus audires.}
\setauthornote{4077}{Busbequius. Sands. lib. 1. fol. 89.}
\setauthornote{4078}{Quis hodie beatior, quam cui licet stultum esse, et eorundam immunitatibus frui. Sat. Menip.}
\setauthornote{4079}{Lib. Hist.}
\setauthornote{4080}{Parvo viventes laboriosi, longaevi, suo contenti, ad centum annos vivunt.}
\setauthornote{4081}{Lib. 6. de Nup. Philol. Ultra humanam fragilitatem prolixi, ut immature pereat qui centenarius moriatur, \&c.}
\setauthornote{4082}{Victus eorum caseo et laete consistit, potus aqua et serum; pisces loco panis habent; ita multos annos saepe 250 absque medico et medicina vivunt.}
\setauthornote{4083}{Lib. de 4. complex.}
\setauthornote{4084}{Per mortes agunt experimenta et animas nostras negotiantur; et quod aliis exitiale hominem occidere iis impunitas summa. Plinius.}
\setauthornote{4085}{Juven.}
\setauthornote{4086}{Omnis morbus lethalis aut curabilis, in vitam definit aut in mortem. Utroque igitur modo medicina inutilis; si lethalis, curari non potest; si curabilis, non requirit medicum: natura expellet.}
\setauthornote{4087}{In interpretationes politico-morales in 7 Aphorism. Hippoc. libros.}
\setauthornote{4088}{Praefat. de contrad. med.}
\setauthornote{4089}{Opinio facit medicos: a fair gown, a velvet cap, the name of a doctor is all in all.}
\setauthornote{4090}{Morbus alius pro alio curatur; aliud remedium pro alio.}
\setauthornote{4091}{Contrarias proferunt sententias. Card.}
\setauthornote{4092}{Lib. 3. de sap. Omnes artes fraudem admittunt, sola medicina sponte eam accersit.}
\setauthornote{4093}{Omnis aegrotus, propria culpa perit, sed nemo nisi medici beneficio restituitur. Agrippa.}
\setauthornote{4094}{How does the surgeon differ from the doctor? In this respect: one kills by drugs, the other by the hand; both only differ from the hangman in this way, they do slowly what he does in an instant.}
\setauthornote{4095}{Medicine cannot cure the knotty gout.}
\setauthornote{4096}{Lib. 3. Crat. ep. Winceslao Raphaeno. Ausim dicere, tot pulsuum differentias, quae describuntur a Galeno, nec a quoquam intelligi, nec observari posse.}
\setauthornote{4097}{Lib. 28. cap. 7. syntax, art. mirab. Mallem ego expertis credere solum, quam mere ratiocinantibus: neque satis laudare possum institutum Babylonicum, \&c.}
\setauthornote{4098}{Herod. Euterpe de Egyptiis. Apud eos singulorum morborum sunt singuli medici; alius curat oculos, alius dentes, alius caput, partes occultas alius.}
\setauthornote{4099}{Cyrip. lib. 1. Velut vestium fractarum resarcinatores, \&c.}
\setauthornote{4100}{Chrys. hom.}
\setauthornote{4101}{Prudens et pius medicus, morbum ante expellere satagit, cibis medicinalibus, quam puris medicinis.}
\setauthornote{4102}{Cuicunque potest per alimenta restitui sanitas, frugiendus est penitus usus medicamentorum.}
\setauthornote{4103}{Modestus et sapiens medicus, nunquam properabit ad pharmaciam, nisi cogente necessitate.}
\setauthornote{4104}{Quicunque pharmacatur in juventute, deflebit in senectute.}
\setauthornote{4105}{Hildesh. spic. 2. de mel. fol. 276. Nulla est firme medicina purgans, quae non aliquam de viribus et partibus corporis depraedatur.}
\setauthornote{4106}{Lib. 1. et Bart. lib. 8. cap. 12.}
\setauthornote{4107}{De vict. acut. Omne purgans medicamentum, corpori purgato contrarium, \&c. succos et spiritus abducit, substantiam corporis aufert.}
\setauthornote{4108}{Hesiod. op.}
\setauthornote{4109}{Heurnius praef. pra. med. Quot morborum sunt ideae, tot remediorum genera variis potentiis decorata.}
\setauthornote{4110}{Penottus denar. med. Quaecunque regio producit simplicia, pro morbis regionis; crescit raro absynthium in Italia, quod ibi plerumque morbi calidi, sed cicuta, papaver, et herbae frigidae; apud nos Germanos et Polonos ubique provenit absynthium.}
\setauthornote{4111}{Quum in villam venit, consideravit quae ibi crescebant medicamenta, simplicia frequentiora, et iis plerunque usus distillatis, et aliter, alimbacum ideo argenteum circumferens.}
\setauthornote{4112}{Herbae medicis utiles omnium in Apulia feracissimae.}
\setauthornote{4113}{Geog. ad quos magnus herbariorum numerus undique confluit. Sincerus Itiner. Gallia.}
\setauthornote{4114}{Baldus mons prope Benacum herbilegis maxime notus.}
\setauthornote{4115}{Qui se nihil effecisse arbitrantur, nisi Indiam, Aethiopiam, Arabiam, et ultra Garamantas a tribus mundi partibus exquisita remedia corradunt. Tutius saepe medetur rustica anus una, \&c.}
\setauthornote{4116}{Ep. lib. 8. Proximorum incuriosi longinqua sectamur, et ad ea cognoscenda iter ingredi et mare transmittere solemus; at quae sub oculis posita negligimus.}
\setauthornote{4117}{Exotica rejecit, domesticis solum nos contentos esse voluit. Melch. Adamus vit. ejus.}
\setauthornote{4118}{Instit, l. 1. cap. 8. sec. 1. ad exquisitam curandi rationem, quorum cognitio imprimis necessaria est.}
\setauthornote{4119}{Quae caeca vi ac specifica qualitate morbos futuros arcent. lib. 1. cap. 10. Instit. Phar.}
\setauthornote{4120}{Galen. lib. epar lupi epaticos curat.}
\setauthornote{4121}{Stercus pecoris ad Epilepsiam, \&c.}
\setauthornote{4122}{Priestpintle, rocket.}
\setauthornote{4123}{Sabina faetum educit.}
\setauthornote{4124}{Wecker. Vide Oswaldum Crollium, lib. de internis rerum signaturis, de herbis particularibus parti cuique convenientibus.}
\setauthornote{4125}{Idem Laurentius, c. 9.}
\setauthornote{4126}{Dicor borago gaudia semper ago.}
\setauthornote{4127}{Vino infusam hilaritatem facit.}
\setauthornote{4128}{Odyss. A.}
\setauthornote{4129}{Lib. 2. cap. 2. prax. med. mira vi laetitiam praebet et cor confirmat, vapores melancholicos purgat a spiritibus.}
\setauthornote{4130}{Proprium est ejus animum hilarem reddere, concoctionem juvare, ccrebri obstructiones resecare, sollicitudines fugare, sollicitas imaginationes tollere. Scorzonera.}
\setauthornote{4131}{Non solum ad viperarum morsus, comitiales, vertiginosos; sed per se accommodata radix tristitiam discutit, hilaritatemque conciliat.}
\setauthornote{4132}{Bilem utramque detrahit, sanguinem purgat.}
\setauthornote{4133}{Lib. 7. cap. 5. Laiet. occit. Indiae descrip. lib. 10. cap. 2.}
\setauthornote{4134}{Heurnius, l. 2. consil. 185. Scoltzii consil. 77.}
\setauthornote{4135}{Praef. denar. med. Omnes capitis dolores et phantasmata tollit; scias nullam herbam in terris huic comparandam viribus et bonitate nasci.}
\setauthornote{4136}{Optimum medicamentum in ceteri cordis confortatione, et ad omnes qui tristantur, \&c.}
\setauthornote{4137}{Rondoletius. Elenum quod vim habet miram ad hilaritatem et multi pro secreto habent. Sckenkius observ. med. cen. 5. observ. 86.}
\setauthornote{4138}{Afflictas mentes relevat, animi imaginationes et daemones expellit.}
\setauthornote{4139}{Sckenkius, Mizaldus, Rhasis.}
\setauthornote{4140}{Cratonis ep. vol. 1. Credat qui vult gemmas mirabilia efficere; mihi qui et ratione et experientia didici aliter rem habere, nullus facile persuadebit falsum esse verum.}
\setauthornote{4141}{L. de gemmis.}
\setauthornote{4142}{Margaritae et corallum ad melancholiam praecipue valent.}
\setauthornote{4143}{Margaritae et gemmae spiritus confortant et cor, melancholiam fugant.}
\setauthornote{4144}{Praefat. ad lap. prec. lib. 2. sect. 2. de mat. med. Regum coronas ornant, digitos illustrant, supellectilem ditant, e fascino tuentur, morbis medentur, sanitatem conservant, mentem exhilarant, tristitiam pellunt.}
\setauthornote{4145}{Encelius, l. 3. c. 4. Suspensus vel ebibitus tristitiae multum resistit, et cor recreat.}
\setauthornote{4146}{Idem. cap. 5. et cap. 6. de Hyacintho et Topazio. Iram sedat et animi tristitiam pellit.}
\setauthornote{4147}{Lapis hic gestatus aut ebibitus prudentiam auget, nocturnos timores pellit; insanos hac sanavi, et quum lapidem abjecerint, erupit iterum stultitia.}
\setauthornote{4148}{Inducit sapientiam, fugat stultitiam. Idem Cardanus, lunaticos juvat.}
\setauthornote{4149}{Confert ad bonum intellectum, comprimit malas cogitationes, \&c. Alacres reddit.}
\setauthornote{4150}{Albertus, Encelius, cap. 44. lib. 3. Plin. lib. 37. cap. 10. Jacobus de Dondis: dextro brachio alligatus sanat lunaticos, insanos, facit amabiles, jucundos.}
\setauthornote{4151}{Valet contra phantasticas illusiones ex melancholia.}
\setauthornote{4152}{Amentes sanat, tristitiam pellit, iram, \&c.}
\setauthornote{4153}{Valet ad fugandos timores et daemones, turbulenta somnia abigit, et nocturnos puerorum timores compescit.}
\setauthornote{4154}{Somnia laeta facit argenteo annulo gestatus.}
\setauthornote{4155}{Atrae bili adversatur, omnium gemmarum pulcherrima, coeli colorem refert, animum ab errore liberat, mores in melius mutat.}
\setauthornote{4156}{Longis moeroribus feliciter medetur, deliquiis, \&c.}
\setauthornote{4157}{Sec. 5. Memb. 1. Subs. 5.}
\setauthornote{4158}{Gestamen lapidum et gemmarum maximum fert auxilium et juvamen; unde qui dites sunt gemmas secum ferre student.}
\setauthornote{4159}{Margaritae et uniones quae a conchis et piscibus apud Persas et Indos, valde cordiales sunt, \&c.}
\setauthornote{4160}{Aurum laetitiam general, non in corde, sed in arca virorum.}
\setauthornote{4161}{Chaucer.}
\setauthornote{4162}{Aurum non aurum. Noxium ob aquas rodentes.}
\setauthornote{4163}{Ep. ad Monavium. Metallica omnia in universum quovismodo parata, nec tuto nec commode intra corpus sumi.}
\setauthornote{4164}{In parag. Stultissimus pilus occipitis mei plus scit, quam omnes vestri doctores, et calceorum meorum annuli doctiores sunt quam vester Galenus et Avicenna, barba mea plus experta est quam vestrae omnes Academiae.}
\setauthornote{4165}{Vide Ernestum Burgratium, edit. Franaker. 8vo. 1611. Crollius and others.}
\setauthornote{4166}{Plus proficiet gutta mea, quam tot eorum drachmae et unciae.}
\setauthornote{4167}{Nonnulli huic supra modum indulgent, usum etsi non adeo magnum, non tamen abjiciendum censeo.}
\setauthornote{4168}{Ausim dicere neminem medicum excellentem qui non in hac distillatione chymica sit versatus. Morbi chronici devinci citra metallica vix possint, aut ubi sanguis corrumpitur.}
\setauthornote{4169}{Fraudes hominum et ingeniorum capturae, officinas invenere istas, in quibus sua cuique venalis promittitur vita; statim compositiones et mixturae inexplicabiles ex Arabia et India, ulceri parvo medicina a rubro mari importatur.}
\setauthornote{4170}{Arnoldus Aphor. 15. Fallax medicus qui potens mederi simplicibus, composita dolose aut frustra quaerit.}
\setauthornote{4171}{Lib. 1. sect. 1. cap. 8. Dum infinita medicamenta miscent, laudem sibi comparare student, et in hoc studio alter alterum superare conatur, dum quisque quo plura miscuerit, eo se doctiorem putet, inde fit ut suam prodant inscitiam, dum ostentant peritiam, et se ridiculos exhibeant, \&c.}
\setauthornote{4172}{Multo plus periculi a medicamento, quam a morbo, \&c.}
\setauthornote{4173}{Expedit. in Sinas, lib. 1. c. 5. Praecepta medici dant nostris diversa, in medendo non infelices, pharmacis utuntur simplicibus, herbis, radicibus, \&c. tota eorum medicina nostrae herbariae praeceptis continetur, nullus ludus hujus artis, quisque privatus a quolibet magistro eruditur.}
\setauthornote{4174}{Lib. de Aqua.}
\setauthornote{4175}{Opusc. de Dos.}
\setauthornote{4176}{Subtil. cap. de scientiis.}
\setauthornote{4177}{Quaercetan. pharmacop. restitut. cap. 2. Nobilissimum et utilissimum inventum summa cum necessitate adinventum et introductum.}
\setauthornote{4178}{Cap. 25. Tetrabib. 4. ser. 2. Necessitas nunc cogit aliquando noxia quaerere remedia, et ex simplicibus compositas facere, tum ad saporem, odorem, palati gratiam, ad correctionem simplicium, tum ad futuros usus, conservationem, \&c.}
\setauthornote{4179}{Cum simplicia non possunt neccessitas cogit ad composita.}
\setauthornote{4180}{Lips. Epist.}
\setauthornote{4181}{Theod. Podromus Amor. lib. 9.}
\setauthornote{4182}{Sanguinem corruptum emaculat, scabiem abolet, lepram curat, spiritus recreat, et animum exhilarat. Melancholicos humores per urinam educit, et cerebrum a crassis, aerumnosis melancholiae fumis purgat, quibus addo dementes et furiosos vinculis retinendos plurimum juvat, et ad rationis usum ducit. Testis est mihi conscientia, quod viderim matronam quandam hinc liberatam, quae frequentius ex iracundia demens, et impos animi dicenda tacenda loquebatur, adeo furens ut ligari cogeretur. Fuit ei praestantissimo remedio, vini istius usus, indicatus a peregrino homine mendico, eleemosynam prae foribus dictae matronae implorante.}
\setauthornote{4183}{Iis qui tristautur sine causa, et vitant amicorum societatem et tremunt corde.}
\setauthornote{4184}{Modo non inflammetur melancholia, aut calidiore temperamento sint.}
\setauthornote{4185}{Heurnius: datur in sero lactis, aut vino.}
\setauthornote{4186}{Veratri modo expurgat cerebrum, roborat memoriam. Fuchsias.}
\setauthornote{4187}{Crassos et biliosos humores per vomitum educit.}
\setauthornote{4188}{Vomitum et menses cit. valet ad hydrop. \&c.}
\setauthornote{4189}{Materias atras educit.}
\setauthornote{4190}{Ab arte ideo rejiciendum, ob periculum suffocationis.}
\setauthornote{4191}{Cap. 16. magna vi educit, et molestia cum summa.}
\setauthornote{4192}{Quondam terribile.}
\setauthornote{4193}{Multi studiorum gratia ad providenda acrius quae commentabantur.}
\setauthornote{4194}{Medetur comitialibus, melancholicis, podagricis; vetatur senibus, pueris, mollibus et effaeminatis.}
\setauthornote{4195}{Collect. lib. 8. cap. 3. in affectionibus iis quae difficulter curantur, Helleborum damus.}
\setauthornote{4196}{Non sine summa cautio ne hoc remedio utemur; est enim validissimum, et quum vires Antimonii contemnit morbus, in auxilium evocatur, modo valide vires efflorescant.}
\setauthornote{4197}{Aetias tetrab. cap. 1. ser. 2. Iis solum dari vult Helleborum album, qui secus spem non habent, non iis qui Syncopem timent, \&c.}
\setauthornote{4198}{Cum salute multorum.}
\setauthornote{4199}{Cap. 12 de morbis cap.}
\setauthornote{4200}{Nos facillime utimur nostro prepaerato Helleboro albo.}
\setauthornote{4201}{In lib. 5. Dioscor. cap. 3. Omnibus opitulator morbis, quos atrabilis excitavit comitialibus iisque presertim qui Hypocondriacas obtinent passiones.}
\setauthornote{4202}{Andreas Gallus, Tridentinus medicus, salutem huic medicamento post Deum debet.}
\setauthornote{4203}{Integrae sanitati brevi restitutus. Id quod aliis accidisse scio, qui hoc mirabili medicamento usi sunt.}
\setauthornote{4204}{Qui melancholicus factus plane desipiebat, multaque stulte loquebaturr, huic exhibitum 12. gr. stibium, quod paulo post atram bilem ex alvo eduxit (ut ego vidi, qui vocatus tanquam ad miraculum adfui testari possum,) et ramenta tunquam carnis dissecta in partes totum excrementum tanquam sanguinem nigerrimum repraesentabat.}
\setauthornote{4205}{Antimonium venenum, non medicamentum.}
\setauthornote{4206}{Cratonis ep. sect. vel ad Monavium ep. In utramque partem dignissimum medicamentum, si recte utentur, secus venenum.}
\setauthornote{4207}{Maerores fugant; utilissime dantur melancholicis et quaternariis.}
\setauthornote{4208}{Millies horum vires expertus sum.}
\setauthornote{4209}{Sal nitrium, sal ammoniaeum, Dracontii radix, doctamnum.}
\setauthornote{4210}{Calet ordine secundo, siccat primo, adversus omnia vitia atrae bilis valet, sanguinem mundat, spiritus illustrat, maerorem discutit herba mirifica.}
\setauthornote{4211}{Cap. 4. lib. 2.}
\setauthornote{4212}{Recentiores negant ora venarum resecare.}
\setauthornote{4213}{An aloe aperiat ora venarum. lib. 9. cont. 3.}
\setauthornote{4214}{Vapores abstergit a vitalibus partibus.}
\setauthornote{4215}{Tract. 15. c. 6. Bonus Alexander, tantam lapide Arnteno confidentiam habuit, ut omnes melancholicas passiones ab eo curari posse crederet, et ego inde saepissime usus sum, et in ejus exhibitione nunquam fraudatus fui.}
\setauthornote{4216}{Maurorum medici hoc lapide plerumque purgant melancholiam, \&c.}
\setauthornote{4217}{Quo ego saepe feliciter usus sum, et magno cum auxilio.}
\setauthornote{4218}{Si non hoc, nihil restat nisi Helleborus, et lapis Armenus. Consil. 184. Scoltzii.}
\setauthornote{4219}{Multa corpora vidi gravissime hinc agitata, et stomacho multum obfuisse.}
\setauthornote{4220}{Cum vidissit ab eo curari capras furentes, \&c.}
\setauthornote{4221}{Lib. 6. simpl. med.}
\setauthornote{4222}{Pseudolo act. 4. scen. ult. helleboro hisce hominibus opus est.}
\setauthornote{4223}{Hor.}
\setauthornote{4224}{In Satyr.}
\setauthornote{4225}{Crato consil. 16. l. 2. Etsi multi magni viri probent, in bonam partem accipiant medici, non probem.}
\setauthornote{4226}{Vescuntur veratro coturnices quod hominibus toxicum est.}
\setauthornote{4227}{Lib. 23. c. 7. 12. 14.}
\setauthornote{4228}{De var. hist.}
\setauthornote{4229}{Corpus incolume reddit, et juvenile efficit.}
\setauthornote{4230}{Veteres non sine causa usi sunt: Difficilis ex Helleboro purgatio, et terroris plena, sed robustis datur tamen, \&c.}
\setauthornote{4231}{Innocens medicamentum, modo rite paretur.}
\setauthornote{4232}{Absit jactantia, ego primus praebere caepi, \&c.}
\setauthornote{4233}{In Catart. Ex una sola evacuatione furor cessavit et quietus inde vixit. Tale exemplum apud Sckenkium et apud Scoltzium, ep. 231. P. Monavius se stolidum curasse jactat hoc epoto tribus aut quatuor vicibus.}
\setauthornote{4234}{Ultimum refugium, extremum medicamentum, quod caetera omnia claudit, quaecunque caeteris laxativis pelli non possunt ad hunc pertinent; si non huic, nulli cedunt.}
\setauthornote{4235}{Testari possum me sexcentis hominibus Helleborum nigrum exhibuisse, nullo prorsus incommodo, \&c.}
\setauthornote{4236}{Pharmacop. Optimum est ad maniam et omnes melancholicos affactus, tum intra assumptum, tum extra, secus capiti cum linteolis in eo madefactis tepide admotutm.}
\setauthornote{4237}{Epist. Math. lib. 3. Tales Syrupi nocentissimi et omnibus modis extirpandi.}
\setauthornote{4238}{Purgantia censebant medicamenta, non unum humorem attrahere, sed quemcunque attigerint in suam naturam convertere.}
\setauthornote{4239}{Religantur omnes exsiccantes medicinae, ut Aloe, Hiera, pilulae quaecunque.}
\setauthornote{4240}{Contra eos qui lingua vulgari et vernacula remedia et medicamenta praescribunt, et quibusvis communia faciunt.}
\setauthornote{4241}{Quis, quantum, quando.}
\setauthornote{4242}{Fernelius, lib. 2. cap. 19.}
\setauthornote{4243}{Renodeus, lib. 5. cap. 21. de his Mercurialis lib. 3. de composit. med. cap. 24. Heurnius, lib. 1. prax. med. Wecker, \&c.}
