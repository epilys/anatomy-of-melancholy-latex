\setmarginnote{4244}{Cont. lib. 1. c. 9, festines ad impinguationem, et cum impinguantur, removetur malum.}
\setmarginnote{4245}{Beneficium ventris.}
\setmarginnote{4246}{Si ex primario cerebri affectu melancholici evaserint, sanguinis detractione non indigent, nisi ob alias causas sanguis mittatur, si multus in vasis, \&c. frustra enim fatigatur corpus, \&c.}
\setmarginnote{4247}{Competit iis phlebotomia frontis.}
\setmarginnote{4248}{Si sanguis abundet, quod scitur ex venarum repletione, victus ratione praecedente, risu aegri, aetate et aliis. Tundatur mediana; et si sanguis apparet clarus et ruber, supprimatur; aut si yere, si niger aut crassus permittatur fluere pro viribus aegri, dein post 8. vel. 12. diem aperiatur cephalica partis magis affectae, et vena frontis, aut sanguis provocetur setis per nares, \&c.}
\setmarginnote{4249}{Si quibus consuetae suae suppressae sunt menses, \&c. talo secare oportet, aut vena frontis si sanguis peccet cerebro.}
\setmarginnote{4250}{Nisi ortum ducat a sanguine, ne morbus inde augeatur; phlebotomia refrigerat et exiceat, nisi corpus sit valde sanguineum, rubicundum.}
\setmarginnote{4251}{Cum sanguinem detrahere oportet, deliberatione indiget. Areteus, lib. 7. c. 5.}
\setmarginnote{4252}{A lenioribus auspicandum. (Valescus, Fiso, Bruel) rariusque medicamentis purgantibus utendum, ni sit opus.}
\setmarginnote{4253}{Quia corpus exiccant, morbum augent.}
\setmarginnote{4254}{Guianerius Tract. 15. c. 6.}
\setmarginnote{4255}{Piso.}
\setmarginnote{4256}{Rhasis, saepe valent ex Helleboro.}
\setmarginnote{4257}{Lib. 7. Exigius medicamentis morbus non obsequitur.}
\setmarginnote{4258}{Modo caute detur et robustis.}
\setmarginnote{4259}{Consil. 10. l. 1.}
\setmarginnote{4260}{Plin. l. 31. c. 6. Navigationes ob vomitionem prosunt plurimis morbis capitis, et omnibus ob quae Helleborum bibitur. Idem Dioscorides, lib. 5. cap. 13. Avicenna tertia imprimis.}
\setmarginnote{4261}{Nunquam dedimus, quin ex una aut altera assumptione, Deo juvante, fuerint ad salutem restituti.}
\setmarginnote{4262}{Lib. 2. Inter composita purgantia melancholiam.}
\setmarginnote{4263}{Longo experimento a se observatum esse, melancholicos sine offensa egregie curandos valere. Idem responsione ad Aubertum, veratrum nigrum, alias timidum et periculosum vini spiritu etiam et olco commodum sic usui redditur ut etiam pueris tuto administrari possit.}
\setmarginnote{4264}{Certum est hujus herbae virtutem maximam et mirabilem esse, parumque distare a balsamo. Et qui norit eo recte uti, plus habet artis quam tota scribentium cohors aut omnes doctores in Germania.}
\setmarginnote{4265}{Quo feliciter usus sum.}
\setmarginnote{4266}{Hoc posito quod aliae medicina non valeant, ista tune Dei misericordia valebit, et est medicina coronata, quae secretissime tenentur.}
\setmarginnote{4267}{Lib. de artif. med.}
\setmarginnote{4268}{Sect. 3. Optimum remedium aqua composita Savanarolae.}
\setmarginnote{4269}{Sckenkius, observ. 31.}
\setmarginnote{4270}{Donatus ab Altomari, cap. 7. Tester Deum, me multos melancholicos hujus solius syrupi usu curasse, facta prius purgatione.}
\setmarginnote{4271}{Centum ova et unum, quolibet mane sumant ova sorbilia, cum sequenti pulvere supra ovum aspersa, et contineant quousque assumpserint centum et unum, maniacis et melancholicis utilissimum remedium.}
\setmarginnote{4272}{Quercetan, cap. 4. Phar. Oswaldus Crollius.}
\setmarginnote{4273}{Cap. 1. Licet tota Galenistarum schola, mineralia non sine impio et ingrato fastu a sua practica detestentur; tamen in gravioribus morbis omni vegetabilium derelicto subsidio, ad mineralia confugiunt, licet ea temere, ignaviter, et inutiliter usurpent. Ad finem libri.}
\setmarginnote{4274}{Veteres maledictis incessit, vincit, et contra omnem antiquitatem coronatur, ipseque a se victor declaratur. Gal. lib. 1. meth. c. 2.}
\setmarginnote{4275}{Codronchus de sale absynthii.}
\setmarginnote{4276}{Idem Paracelsus in medicina, quod Lutherus in Theologia.}
\setmarginnote{4277}{Disput. in eundem, parte 1. Magus ebrius, illiteratus, daemonem praeceptorem habuit, daemones familiares, \& c.}
\setmarginnote{4278}{Master D. Lapworth.}
\setmarginnote{4279}{Ant. Philos. cap. de melan. frictio vertice, \&c.}
\setmarginnote{4280}{Aqua fortissima purgans os, nares, quam non vult auro vendere.}
\setmarginnote{4281}{Mercurialis consil. 6. et 30. haemorroidum et mensium provocatio juvat, modo ex eorum suppressione ortum habuerit.}
\setmarginnote{4282}{Laurentius, Bruel, \&c.}
\setmarginnote{4283}{P. Bayerus, l. 2. cap. 13. naribus, \&c.}
\setmarginnote{4284}{Cucurbitulae siccae, et fontanellae crure sinistro.}
\setmarginnote{4285}{Hildesheim spicel. 2. Vapores a cerebro trahendi sunt frictionibus universi, cucurbitulis siccis, humeris ac dorso affixis, circa pedes et crura.}
\setmarginnote{4286}{Fontanellam aperi juxta occipitum, aut brachium.}
\setmarginnote{4287}{Baleni, ligaturae, frictiones, \&c.}
\setmarginnote{4288}{Canterium fiat sutura coronali, diu fluere permittantur loca ulcerosa. Trepano etiam cranii densitas imminui poterit, ut vaporibus fuliginosis exitus pateat.}
\setmarginnote{4289}{Quoniam difficulter cedit aliis medicamentis, ideo fiat in vertice cauterium, aut crure sinistro infra genu.}
\setmarginnote{4290}{Fiant duo aut tria cauteria, cum ossis perforatione.}
\setmarginnote{4291}{Vidi Romae melancholicum qui adhibitis multis remediis, sanari non poterat; sed cum cranium gladio fractum esset, optime sanatus est.}
\setmarginnote{4292}{Et alterum vidi melancholicum, qui ex alto cadens non sine astantium admiratione, liberatus est.}
\setmarginnote{4293}{Radatur caput et fiat cauterium in capite; procul dubio ista faciunt ad fumorum exhalationem; vidi melancholicum a fortuna gladio vulneratum, et cranium fractum, quam diu vulnus apertum, curatus optime; at cum vulnus sanatum, reversa est mania.}
\setmarginnote{4294}{Usque ad duram matrem trepanari feci, et per mensam aperte stetit.}
\setmarginnote{4295}{Cordis ratio semper habenda quod cerebro compatitur, et sese invicem officiunt.}
\setmarginnote{4296}{Aphor. 38. Medicina Theriacalis praecaeteris eligenda.}
\setmarginnote{4297}{Galen, de temp. lib. 3. c. 3. moderate vinum sumptum, acuit ingenium.}
\setmarginnote{4298}{Tardos aliter et tristes thuris in modum exhalare facit.}
\setmarginnote{4299}{Hilaritatem ut oleum flammam excitat.}
\setmarginnote{4300}{Viribus retinendis cardiacum eximium, nutriendo corpori ailimentum optimum, aetatem floridam facit, calorem innatum fovet, concoctionem juvat, stomachum roborat, excrementis viam parat, urinam movet, somnum conciliat, venena frigidos flatus dissipat, crassos humores attenuat, co quit, discutit, \&c.}
\setmarginnote{4301}{Hor. lib. 2. od. 11. Bacchus dissipates corroding cares.}
\setmarginnote{4302}{Odyss. A.}
\setmarginnote{4303}{Pausanias.}
\setmarginnote{4304}{Siracides, 31. 28.}
\setmarginnote{4305}{Legitur et prisci Catonis. Saepe mero caluisse virtus.}
\setmarginnote{4306}{In pocula et aleam se praecipitavit, et iis fere tempus traduxit, ut aegram crapula mentem levaret, et conditionis praesentis cogitationes quibus agitabatur sobrius vitaret.}
\setmarginnote{4307}{So did the Athenians of old, as Suidas relates, and so do the Germans at this day.}
\setmarginnote{4308}{Lib. 6. cap. 23. et 24. de rerum proprietat.}
\setmarginnote{4309}{Esther, i. 8.}
\setmarginnote{4310}{Tract. 1. cont. l. 1. Non est res laudabilior eo, vel cura melior; qui melancholicus, utatur societate hominum et biberia; et qui potest sustinere usum vini, non indiget alia medicina, quod eo sunt omnia ad usum necessaria hujus passionis.}
\setmarginnote{4311}{Tum quod sequatur inde sudor, vomitio, urina, a quibus superfluitates a corpore removentur et remanet corpus mundum.}
\setmarginnote{4312}{Hor.}
\setmarginnote{4313}{Lib. 15. 2. noct. Alt. Vigorem animi moderate vini usu tueamur, et calefacto simul, refotoque animo si quid in eo vel frigidae tristitiae, vel torpentis verecundiae fuerit, diluamus.}
\setmarginnote{4314}{Hor. l. 1. od. 27.}
\setmarginnote{4315}{Od. 7. lib. 1. 26. Nam praestat ebrium me quam mortuum jacere.}
\setmarginnote{4316}{Ephes. v. 18. ser. 19. in cap. 5.}
\setmarginnote{4317}{Lib. 14. 5. Nihil perniciosus viribus si modus absit, venenum.}
\setmarginnote{4318}{Theocritus idyl. 13. vino dari laetitiam et dolorem.}
\setmarginnote{4319}{Renodeus.}
\setmarginnote{4320}{Mercurialis consil. 25. Vinum frigidis optimum, et pessimum ferina melancholia.}
\setmarginnote{4321}{Fernelius consil. 44 et 45, vinum prohibet assiduum, et aromata.}
\setmarginnote{4322}{Modo jecur non incendatur.}
\setmarginnote{4323}{Per 24 horas sensum doloris omnem tollit, et ridere facit.}
\setmarginnote{4324}{Hildesheim, spicel. 2.}
\setmarginnote{4325}{Alkermes, omnia vitalia viscera mire confortat.}
\setmarginnote{4326}{Contra omnes melancholicos affectus confert, ac certum est ipsius usu omnes cordis et corporis vires mirum in modum refici.}
\setmarginnote{4327}{Succinum vero albissimum confortat ventriculum, statum discutit, urinam movet, \&c.}
\setmarginnote{4328}{Gartias ab Horto aromatum lib. 1. cap. 15. adversus omnes morbos melancholicos conducit, et venenum. Ego (inquit) utor in morbis melancholicis, \&c. et deploratos hujus usu ad pristinam sanitatem restitui. See more in Bauhinas' book de lap. Bezoar c. 45.}
\setmarginnote{4329}{Edit. 1617. Monspelii electuarium fit preciocissimum Alcherm. \&c.}
\setmarginnote{4330}{Nihil morbum hunc aeque exasperat, ac alimentorum vel calidiorum usus. Alchermes ideo suspectus, et quod semel moneam, caute adhibenda calida medicamenta.}
\setmarginnote{4331}{Sckenkius I. I. Observat. de Mania, ad mentis alienationem, et desipientiam vitio cerebri obortam, in manuscripto codice Germanico, tale medicamentum reperi.}
\setmarginnote{4332}{Caput arietis nondum experti venerem, uno ictu amputatum, cornibus tantum demotis, integrum cum lana et pelle bene elixabis, tum aperto cerebrum eximes, et addens aromata, \&c.}
\setmarginnote{4333}{Cinis testudinis ustus, et vino potus melancholiam curat, et rasura cornu Rhinocerotis, \&c. Sckenkius.}
\setmarginnote{4334}{Instat in matrice, quod sursum et deorsum ad odoris sensum praecipitatur.}
\setmarginnote{4335}{Viscount St. Alban's.}
\setmarginnote{4336}{Ex decocto florum nympheae, lactuae, violarum, chamomilae, alibeae, capitis vervecum, \&c.}
\setmarginnote{4337}{Inter auxilia multa adhibita, duo visa sunt remedium adferre, usus seri caprini cum extracto Hellebori, et irrigatio ex lacte Nympheae, violarum, \&c. suturae coronali adhibita; his remediis sanitate pristinam adeptus est.}
\setmarginnote{4338}{Confert et pulmo arietis, calidus agnus per dorsum divisus, exenteratus, admotus sincipiti.}
\setmarginnote{4339}{Semina cumini, rutae, dauci anethi cocta.}
\setmarginnote{4340}{Lib. 3. de locis affect.}
\setmarginnote{4341}{Tetrab. 2. ser. 1. cap. 10.}
\setmarginnote{4342}{Cap. de mel. collectum die vener. hora Jovis cum ad Energiam venit c. 1. ad plenilunium Julii, inde gesta et collo appensa hunc affectum apprime juvat et fanaticos spiritus expellit.}
\setmarginnote{4343}{L. de proprietat. animal. ovis a lupo correptae pellem non esse pro indumenta corporis usurpandam, cordis enim palpitationem excitat, \&c.}
\setmarginnote{4344}{Mart.}
\setmarginnote{4345}{Phar. lib. 1. cap. 12.}
\setmarginnote{4346}{Aetius cap. 31. Tet. 3. ser. 4.}
\setmarginnote{4347}{Dioscorides, Ulysses Alderovandus de aranea.}
\setmarginnote{4348}{Mistress Dorothy Burton, she died, 1629.}
\setmarginnote{4349}{Solo somno curata est citra medici auxilium, fol. 154.}
\setmarginnote{4350}{Bellonius observat. l. 3. c. 15. lassitudinem et labores animi tollunt; inde Garcias ab Horto, lib. 1. cap. 4. simp. med.}
\setmarginnote{4351}{Absynthium somnos allicit olfactu.}
\setmarginnote{4352}{Read Lemnius lib. her. bib. cap. 2. of Mandrake.}
\setmarginnote{4353}{Hyoscyamus sub cervicali viridis.}
\setmarginnote{4354}{Plantum pedis inungere pinguedine gliris dicunt efficacissimum, et quod vix credi potest, dentes inunctos ex sorditie aurium canis somnum profundum conciliare, \&c. Cardan de rerum varietat.}
\setmarginnote{4355}{Veni mecum lib.}
\setmarginnote{4356}{Aut si quid incautius exciderit aut, \&c.}
\setmarginnote{4357}{Nam qua parte pavor simul est pudor additus illi. Statius.}
\setmarginnote{4358}{Olysipponensis medicus; pudor aut juvat aut laedit.}
\setmarginnote{4359}{De mentis alienat.}
\setmarginnote{4360}{M. Doctor Ashworth.}
\setmarginnote{4361}{Facies nonnullis maxime calet rubetque si se paululum exercuerint; nonnullis quiescentibus idem accidit, faeminis praesertim; causa quicquid fervidum aut halituosum sanguinem facit.}
\setmarginnote{4362}{Interim faciei prospiciendum ut ipsa refrigeretur; utrumque praestabit frequens potio ex aqua rosarum, violarum, nenupharis, \&c.}
\setmarginnote{4363}{Ad faciei ruborem aqua spermatis ranarum.}
\setmarginnote{4364}{Recta utantur in aestate floribus Cichorii sacchoro conditis vel saccharo rosaceo, \&c.}
\setmarginnote{4365}{Solo usu decocti Cichorii.}
\setmarginnote{4366}{Utile imprimis noctu faciem illinire sanguine leporino, et mane aqua fragrorum vel aqua floribus verbasci cum succo limonum distillato abluere.}
\setmarginnote{4367}{Utile rubenti faciei caseum recentem imponere.}
\setmarginnote{4368}{Consil. 22 lib. unico vini haustu sit contentus.}
\setmarginnote{4369}{Idem consil. 283. Scoltzii laudatur conditus rosae caninae fructus ante prandium et caenem ad magnitudinem castaneae. Decoctum radium Sonchi, si ante cibum sumatur, valet plurimum.}
\setmarginnote{4370}{Cucurbit, ad scapulas apposite.}
\setmarginnote{4371}{Piso.}
\setmarginnote{4372}{Mediana prae caeteris.}
\setmarginnote{4373}{Succi melancholici malitia a sanguinis bonitate corrigitur.}
\setmarginnote{4374}{Perseverante malo ex quacunque parto sanguinis detrahi debet.}
\setmarginnote{4375}{Observat. fol. 154. curarus ex vulnere in crure ob cruorem arnissum.}
\setmarginnote{4376}{Studium sit omne ut melancholicus impinguetur: ex quo enim pingues et carnosi, illico sani sunt.}
\setmarginnote{4377}{Hildesheim spicel. 2. Inter calida radix petrofelini, apii, feniculi; Inter frigida emulsio seminis melonum cum sero caprino quod est commune vehiculum.}
\setmarginnote{4378}{Hoc unum praemoneo domine ut sis diligens circa victum, sine quo cetera remedia frustra adhibentur.}
\setmarginnote{4379}{Laurentius cap. 15. evulsionis gratia venam internam alterius brachii secamus.}
\setmarginnote{4380}{Si pertinax morbus, venam fronte secabis. Bruell.}
\setmarginnote{4381}{Ego maximam curam stomacho delegabo. Octa. Horatianus lib. 2. c. 7.}
\setmarginnote{4382}{Citius et efficacius suas vires exercet quam solent decocta ac diluta in quantitate multa, et magna cum assumentium molestia desumpta. Flatus hic sal efficaciter dissipat, urinam movet, humores crassos abstergit, stomachum egregie confortat, cruditatem, nauseam, appetentiam mirum in modum renovat, \&c.}
\setmarginnote{4383}{Piso, Altomarus, Laurentius c. 15.}
\setmarginnote{4384}{His utendum saepius iteratis: a vehementioribus semper abstinendum ne ventrem exasperent.}
\setmarginnote{4385}{Lib. 2. cap. 1. Quoniam caliditate conjuncta est siccitas quae malum auget.}
\setmarginnote{4386}{quisquis frigidis auxiliis hoc morbo usus fuerit, is obstructionem aliaque symptomata augebit.}
\setmarginnote{4387}{Ventriculus plerumque frigidus, epar calidum; quomodo ergo ventriculum calefaciet, vel refrigerabit hepar sine alterius maximo detrimento?}
\setmarginnote{4388}{Significatum per literas, incredibilem utilitatem ex decocto Chinae, et Sassafras percepisse.}
\setmarginnote{4389}{Tumorem splenis incurabilem sola cappari curavit, cibo tali aegritudine aptissimo: Soloque usu aquae, in qua faber ferrarius saepe candens ferrum extinxerat, \&c.}
\setmarginnote{4390}{Animalia quae apud hos fabros educantur, exiguos habent lienes.}
\setmarginnote{4391}{L. 1. cap 17.}
\setmarginnote{4392}{Continuum ejus usus semper felicem in aegris finem est assequutus.}
\setmarginnote{4393}{Si Hemorroides fluxerint, nullum praestantius esset remedium, quaesanguifugis admotis provocari poterunt. observat. lib. 1. pro hypoc. legulcio.}
\setmarginnote{4394}{Aliis apertio haec in hoc morbo videtur utilissima; mihi non admodum probatur, quia sanguinem tenuem attrahit et crassum relinquit.}
\setmarginnote{4395}{Lib. 2. cap. 13. omnes melancholici debent omittere urinam provocantia, quoniam per ea educitur subtile, et remanet crassum.}
\setmarginnote{4396}{Ego experientia probavi, multos Hypocondriacos solo usu Clysterum fuisse sanatos.}
\setmarginnote{4397}{In eradicate optimum, ventriculum aretius alligari.}
\setmarginnote{4398}{ℨj. Theriacae, Vere praesertim et aestate.}
\setmarginnote{4399}{Cons. 12. l. 1.}
\setmarginnote{4400}{Cap. 33.}
\setmarginnote{4401}{Trincavellius consil. 15. cerotum pro sene melancholico ad jecur optimum.}
\setmarginnote{4402}{Emplastra pro splene. Fernel. consil. 45.}
\setmarginnote{4403}{Dropax e pice navali, et oleo rutuceo affigatur ventriculo, et toti metaphreni.}
\setmarginnote{4404}{Cauteria cruribus inusta.}
\setmarginnote{4405}{Fontanellae sint in utroque crure.}
\setmarginnote{4406}{Lib. 1. c. 17.}
\setmarginnote{4407}{De mentis alienat. c. 3. flatus egregie discutiunt materiamque evocant.}
\setmarginnote{4408}{Gavendum hic diligenter a, multum, calefacientibus, atque exsiccantibus, sive alimenta fuerint haec, sive medicamenta: nonnulli enim ut ventositates et rugitus conpescant, hujusmodi utentes medicamentis, plurimum peccant, morbum sit augentes: debent enim medicamenta declinare ad calidum vel frigidum secundum exigentiam circumstantiarum, vel ut patiens inclinat ad cal. et frigid.}
\setmarginnote{4409}{Cap. 5 lib. 7.}
\setmarginnote{4410}{Piso Bruel. mire flatus resolvit.}
\setmarginnote{4411}{Lib. 1. c. 17. nonnullos praetensione ventris deploratos illico restitutos bis videmus.}
\setmarginnote{4412}{Velut incantamentum quoddam ex flatuoso spiritu, dolorem ortum levant.}
\setmarginnote{4413}{Terebinthinam Cypriam habeant familiarem, ad quantitatem deglutiant nucis parvae, tribus horis ante prandium vel coenam, ter singulis septimanis prout expedire videbitur; nam praeterquam quod alvum mollem efficit, obstructiones aperit, ventriculum purgat, urinam provocat hepar mundificat.}
