\chapter[Symptoms of Melancholy]{Symptoms, or Signs of Melancholy in the Body}\label{ch:symptoms}
{
%SECT. III. MEMB. I.
\lettrine{P}{arrhasius}, a painter of Athens, amongst those Olynthian captives
Philip of Macedon brought home to sell, \authorfootnote{2452}bought one very old man;
and when he had him at Athens, put him to extreme torture and torment,
the better by his example to express the pains and passions of his
Prometheus, whom he was then about to paint. I need not be so
barbarous, inhuman, curious, or cruel, for this purpose to torture any
poor melancholy man, their symptoms are plain, obvious and familiar,
there needs no such accurate observation or far-fetched object, they
delineate themselves, they voluntarily betray themselves, they are too
frequent in all places, I meet them still as I go, they cannot conceal
it, their grievances are too well known, I need not seek far to
describe them.

Symptoms therefore are either \authorfootnote{2453}universal or particular, saith
Gordonius, lib. med. cap. 19, part. 2, to persons, to species; some
signs are secret, some manifest, some in the body, some in the mind,
and diversely vary, according to the inward or outward causes,
Capivaccius: or from stars, according to Jovianus Pontanus, de reb.
caelest. lib. 10, cap. 13, and celestial influences, or from the
humours diversely mixed, Ficinus, lib. 1, cap. 4, de sanit. tuenda: as
they are hot, cold, natural, unnatural, intended, or remitted, so will
Aetius have melancholica deliria multiformia, diversity of melancholy
signs. Laurentius ascribes them to their several temperatures,
delights, natures, inclinations, continuance of time, as they are
simple or mixed with other diseases, as the causes are diverse, so must
the signs be, almost infinite, Altomarus cap. 7, art. med. And as wine
produceth diverse effects, or that herb Tortocolla in \authorfootnote{2454}Laurentius,
which makes some laugh, some weep, some sleep, some dance, some sing,
some howl, some drink, \&c. so doth this our melancholy humour work
several signs in several parties.

But to confine them, these general symptoms may be reduced to those of
the body or the mind. Those usual signs appearing in the bodies of such
as are melancholy, be these cold and dry, or they are hot and dry, as
the humour is more or less adust. From \authorfootnote{2455}these first qualities
arise many other second, as that of \authorfootnote{2456}colour, black, swarthy, pale,
ruddy, \&c., some are impense rubri, as Montaltus cap. 16 observes out
of Galen, lib. 3, de locis affectis, very red and high coloured.

Hippocrates in his book \authorfootnote{2457}de insania et melan. reckons up these
signs, that they are \authorfootnote{2458} lean, withered, hollow-eyed, look old,
wrinkled, harsh, much troubled with wind, and a griping in their
bellies, or bellyache, belch often, dry bellies and hard, dejected
looks, flaggy beards, singing of the ears, vertigo, light-headed,
little or no sleep, and that interrupt, terrible and fearful dreams,
\authorfootnote{2459}Anna soror, quae, me suspensam insomnia terrent? The same
symptoms are repeated by Melanelius in his book of melancholy collected
out of Galen, Ruffus, Aetius, by Rhasis, Gordonius, and all the
juniors, \authorfootnote{2460}continual, sharp, and stinking belchings, as if their
meat in their stomachs were putrefied, or that they had eaten fish, dry
bellies, absurd and interrupt dreams, and many fantastical visions
about their eyes, vertiginous, apt to tremble, and prone to venery.

\authorfootnote{2461}Some add palpitation of the heart, cold sweat, as usual symptoms,
and a leaping in many parts of the body, saltum in multis corporis
partibus, a kind of itching, saith Laurentius, on the superficies of
the skin, like a flea-biting sometimes. \authorfootnote{2462}Montaltus cap. 21. puts
fixed eyes and much twinkling of their eyes for a sign, and so doth
Avicenna, oculos habentes palpitantes, trauli, vehementer rubicundi,
\&c., lib. 3. Fen. 1. Tract. 4. cap. 18. They stut most part, which he
took out of Hippocrates' aphorisms. \authorfootnote{2463}Rhasis makes headache and a
binding heaviness for a principal token, much leaping of wind about the
skin, as well as stutting, or tripping in speech, \&c., hollow eyes,
gross veins, and broad lips. To some too, if they be far gone, mimical
gestures are too familiar, laughing, grinning, fleering, murmuring,
talking to themselves, with strange mouths and faces, inarticulate
voices, exclamations, \&c. And although they be commonly lean, hirsute,
uncheerful in countenance, withered, and not so pleasant to behold, by
reason of those continual fears, griefs, and vexations, dull, heavy,
lazy, restless, unapt to go about any business; yet their memories are
most part good, they have happy wits, and excellent apprehensions.

Their hot and dry brains make them they cannot sleep, Ingentes habent
et crebras vigilias (Arteus) mighty and often watchings, sometimes
waking for a month, a year together. \authorfootnote{2464}Hercules de Saxonia
faithfully averreth, that he hath heard his mother swear, she slept not
for seven months together: Trincavelius, Tom. 2. cons. 16. speaks of
one that waked 50 days, and Skenkius hath examples of two years, and
all without offence. In natural actions their appetite is greater than
their concoction, multa appetunt pauca digerunt as Rhasis hath it, they
covet to eat, but cannot digest. And although they \authorfootnote{2465}do eat much,
yet they are lean, ill-liking, saith Areteus, withered and hard, much
troubled with costiveness, crudities, oppilations, spitting, belching,
\&c. Their pulse is rare and slow, except it be of the \authorfootnote{2466}Carotides,
which is very strong; but that varies according to their intended
passions or perturbations, as Struthius hath proved at large,
Spigmaticae. artis l. 4. c. 13. To say truth, in such chronic diseases
the pulse is not much to be respected, there being so much superstition
in it, as Crato notes\authorfootnote{2467}, and so many differences in Galen, that he
dares say they may not be observed, or understood of any man.

Their urine is most part pale, and low coloured, urina pauca acris,
biliosa (Areteus), not much in quantity; but this, in my judgment, is
all out as uncertain as the other, varying so often according to
several persons, habits, and other occasions not to be respected in
chronic diseases. \authorfootnote{2468}Their melancholy excrements in some very much,
in others little, as the spleen plays his part, and thence proceeds
wind, palpitation of the heart, short breath, plenty of humidity in the
stomach, heaviness of heart and heartache, and intolerable stupidity
and dullness of spirits. Their excrements or stool hard, black to some
and little. If the heart, brain, liver, spleen, be misaffected, as
usually they are, many inconveniences proceed from them, many diseases
accompany, as incubus, \authorfootnote{2469}apoplexy, epilepsy, vertigo, those
frequent wakings and terrible dreams, \authorfootnote{2470}intempestive laughing,
weeping, sighing, sobbing, bashfulness, blushing, trembling, sweating,
swooning, \&c. \authorfootnote{2471}All their senses are troubled, they think they see,
hear, smell, and touch that which they do not, as shall be proved in
the following discourse.

\section{Symptoms or Signs in the Mind.}

\subsection{Fear.}
\lettrine{A}{rculanus} in 9. Rhasis ad Almansor. cap. 16. will have these
symptoms to be infinite, as indeed they are, varying according to the
parties, for scarce is there one of a thousand that dotes alike, \authorfootnote{2472}
Laurentius c. 16. Some few of greater note I will point at; and amongst
the rest, fear and sorrow, which as they are frequent causes, so if
they persevere long, according to Hippocrates \authorfootnote{2473}and Galen's
aphorisms, they are most assured signs, inseparable companions, and
characters of melancholy; of present melancholy and habituated, saith
Montaltus cap. 11. and common to them all, as the said Hippocrates,
Galen, Avicenna, and all Neoterics hold. But as hounds many times run
away with a false cry, never perceiving themselves to be at a fault, so
do they. For Diocles of old, (whom Galen confutes,) and amongst the
juniors, \authorfootnote{2474}Hercules de Saxonia, with Lod. Mercatus cap. 17. l. 1.
de melan., takes just exceptions, at this aphorism of Hippocrates, 'tis
not always true, or so generally to be understood, fear and sorrow are
no common symptoms to all melancholy; upon more serious consideration,
I find some (saith he) that are not so at all. Some indeed are sad, and
not fearful; some fearful and not sad; some neither fearful nor sad;
some both. Four kinds he excepts, fanatical persons, such as were
Cassandra, Nanto, Nicostrata, Mopsus, Proteus, the sibyls, whom
\authorfootnote{2475}Aristotle confesseth to have been deeply melancholy. Baptista
Porta seconds him, Physiog. lib. 1, cap. 8, they were atra bile
perciti: demoniacal persons, and such as speak strange languages, are
of this rank: some poets, such as laugh always, and think themselves
kings, cardinals, \&c., sanguine they are, pleasantly disposed most
part, and so continue. \authorfootnote{2476}Baptista Portia confines fear and sorrow
to them that are cold; but lovers, Sibyls, enthusiasts, he wholly
excludes. So that I think I may truly conclude, they are not always sad
and fearful, but usually so: and that \authorfootnote{2477}without a cause, timent de
non timendis, (Gordonius,) quaeque momenti non sunt, although not all
alike (saith Altomarus), \authorfootnote{2478}yet all likely fear, \authorfootnote{2479}some with an
extraordinary and a mighty fear, Areteus. \authorfootnote{2480}Many fear death, and
yet in a contrary humour, make away themselves, Galen, lib. 3. de loc.
affec. cap. 7. Some are afraid that heaven will fall on their heads:
some they are damned, or shall be. \authorfootnote{2481}They are troubled with
scruples of consciences, distrusting God's mercies, think they shall go
certainly to hell, the devil will have them, and make great
lamentation, Jason Pratensis. Fear of devils, death, that they shall be
so sick, of some such or such disease, ready to tremble at every
object, they shall die themselves forthwith, or that some of their dear
friends or near allies are certainly dead; imminent danger, loss,
disgrace still torment others, \&c.; that they are all glass, and
therefore will suffer no man to come near them: that they are all cork,
as light as feathers; others as heavy as lead; some are afraid their
heads will fall off their shoulders, that they have frogs in their
bellies, \&c. \authorfootnote{2482}Montanus consil. 23, speaks of one that durst not
walk alone from home, for fear he should swoon or die. A second
\authorfootnote{2483}fears every man he meets will rob him, quarrel with him, or kill
him. A third dares not venture to walk alone, for fear he should meet
the devil, a thief, be sick; fears all old women as witches, and every
black dog or cat he sees he suspecteth to be a devil, every person
comes near him is maleficiated, every creature, all intend to hurt him,
seek his ruin; another dares not go over a bridge, come near a pool,
rock, steep hill, lie in a chamber where cross beams are, for fear he
be tempted to hang, drown, or precipitate himself. If he be in a silent
auditory, as at a sermon, he is afraid he shall speak aloud at
unawares, something indecent, unfit to be said. If he be locked in a
close room, he is afraid of being stifled for want of air, and still
carries biscuit, aquavitae, or some strong waters about him, for fear
of deliquiums, or being sick; or if he be in a throng, middle of a
church, multitude, where he may not well get out, though he sit at
ease, he is so misaffected. He will freely promise, undertake any
business beforehand, but when it comes to be performed, he dare not
adventure, but fears an infinite number of dangers, disasters, \&c. Some
are \authorfootnote{2484} afraid to be burned, or that the \authorfootnote{2485}ground will sink
under them, or \authorfootnote{2486}swallow them quick, or that the king will call
them in question for some fact they never did (Rhasis cont.) and that
they shall surely be executed. The terror of such a death troubles
them, and they fear as much and are equally tormented in mind, \authorfootnote{2487}as
they that have committed a murder, and are pensive without a cause, as
if they were now presently to be put to death. Plater, cap. 3. de
mentis alienat. They are afraid of some loss, danger, that they shall
surely lose their lives, goods, and all they have, but why they know
not. Trincavelius, consil. 13. lib. 1. had a patient that would needs
make away himself, for fear of being hanged, and could not be persuaded
for three years together, but that he had killed a man. Plater,
observat. lib. 1. hath two other examples of such as feared to be
executed without a cause. If they come in a place where a robbery,
theft, or any such offence hath been done, they presently fear they are
suspected, and many times betray themselves without a cause. Lewis XI.,
the French king, suspected every man a traitor that came about him,
durst trust no officer. Alii formidolosi omnium, alii quorundam
(Fracatorius lib. 2. de Intellect.) \authorfootnote{2488}some fear all alike, some
certain men, and cannot endure their companies, are sick in them, or if
they be from home. Some suspect \authorfootnote{2489}treason still, others are afraid
of their \authorfootnote{2490}dearest and nearest friends. (Melanelius e Galeno,
Ruffo, Aetio,) and dare not be alone in the dark for fear of hobgoblins
and devils: he suspects everything he hears or sees to be a devil, or
enchanted, and imagineth a thousand chimeras and visions, which to his
thinking he certainly sees, bugbears, talks with black men, ghosts,
goblins, \&c., \authorfootnote{2491}Omnes se terrent aurae, sonus excitat omnis.

Another through bashfulness, suspicion, and timorousness will not be
seen abroad, \authorfootnote{2492}loves darkness as life, and cannot endure the light,
or to sit in lightsome places, his hat still in his eyes, he will
neither see nor be seen by his goodwill, Hippocrates, lib. de Insania
et Melancholia. He dare not come in company for fear he should be
misused, disgraced, overshoot himself in gesture or speeches, or be
sick; he thinks every man observes him, aims at him, derides him, owes
him malice. Most part \authorfootnote{2493}they are afraid they are bewitched,
possessed, or poisoned by their enemies, and sometimes they suspect
their nearest friends: he thinks something speaks or talks within him,
and he belcheth of the poison. Christophorus a Vega, lib. 2. cap. 1.
had a patient so troubled, that by no persuasion or physic he could be
reclaimed. Some are afraid that they shall have every fearful disease
they see others have, hear of, or read, and dare not therefore hear or
read of any such subject, no not of melancholy itself, lest by applying
to themselves that which they hear or read, they should aggravate and
increase it. If they see one possessed, bewitched, an epileptic
paroxysm, a man shaking with the palsy, or giddy-headed, reeling or
standing in a dangerous place, \&c., for many days after it runs in
their minds, they are afraid they shall be so too, they are in like
danger, as Perkins c. 12. sc. 12. well observes in his Cases of
Conscience and many times by violence of imagination they produce it.

They cannot endure to see any terrible object, as a monster, a man
executed, a carcase, hear the devil named, or any tragical relation
seen, but they quake for fear, Hecatas somniare sibi videntur (Lucian)
they dream of hobgoblins, and may not get it out of their minds a long
time after: they apply (as I have said) all they hear, see, read, to
themselves; as Felix Plater notes\authorfootnote{2494} of some young physicians, that
study to cure diseases, catch them themselves, will be sick, and
appropriate all symptoms they find related of others, to their own
persons. And therefore (quod iterum moneo, licet nauseam paret lectori,
malo decem potius verba, decies repetita licet abundare, quam unum
desiderari) I would advise him that is actually melancholy not to read
this tract of Symptoms, lest he disquiet or make himself for a time
worse, and more melancholy than he was before. Generally of them all
take this, de inanibus semper conqueruntur et timent, saith Aretius;
they complain of toys, and fear \authorfootnote{2495}without a cause, and still think
their melancholy to be most grievous, none so bad as they are, though
it be nothing in respect, yet never any man sure was so troubled, or in
this sort. As really tormented and perplexed, in as great an agony for
toys and trifles (such things as they will after laugh at themselves)
as if they were most material and essential matters indeed, worthy to
be feared, and will not be satisfied. Pacify them for one, they are
instantly troubled with some other fear; always afraid of something
which they foolishly imagine or conceive to themselves, which never
peradventure was, never can be, never likely will be; troubled in mind
upon every small occasion, unquiet, still complaining, grieving,
vexing, suspecting, grudging, discontent, and cannot be freed so long
as melancholy continues. Or if their minds be more quiet for the
present, and they free from foreign fears, outward accidents, yet their
bodies are out of tune, they suspect some part or other to be amiss,
now their head aches, heart, stomach, spleen, \&c. is misaffected, they
shall surely have this or that disease; still troubled in body, mind,
or both, and through wind, corrupt fantasy, some accidental distemper,
continually molested. Yet for all this, as Jacchinus notes\authorfootnote{2496}, in
all other things they are wise, staid, discreet, and do nothing
unbeseeming their dignity, person, or place, this foolish, ridiculous,
and childish fear excepted; which so much, so continually tortures and
crucifies their souls, like a barking dog that always bawls, but seldom
bites, this fear ever molesteth, and so long as melancholy lasteth,
cannot be avoided.

Sorrow is that other character, and inseparable companion, as
individual as Saint Cosmus and Damian, fidus Achates, as all writers
witness, a common symptom, a continual, and still without any evident
cause, \authorfootnote{2497}moerent omnes, et si roges eos reddere causam, non
possunt: grieving still, but why they cannot tell: Agelasti, moesti,
cogitabundi, they look as if they had newly come forth of Trophonius'
den. And though they laugh many times, and seem to be extraordinary
merry (as they will by fits), yet extreme lumpish again in an instant,
dull and heavy, semel et simul, merry and sad, but most part sad:
\authorfootnote{2498}Si qua placent, abeunt; inimica tenacius haerent: sorrow sticks
by them still continually, gnawing as the vulture did \authorfootnote{2499}Titius'
bowels, and they cannot avoid it. No sooner are their eyes open, but
after terrible and troublesome dreams their heavy hearts begin to sigh:
they are still fretting, chafing, sighing, grieving, complaining,
finding faults, repining, grudging, weeping, Heautontimorumenoi, vexing
themselves, \authorfootnote{2500}disquieted in mind, with restless, unquiet thoughts,
discontent, either for their own, other men's or public affairs, such
as concern them not; things past, present, or to come, the remembrance
of some disgrace, loss, injury, abuses, \&c. troubles them now being
idle afresh, as if it were new done; they are afflicted otherwise for
some danger, loss, want, shame, misery, that will certainly come, as
they suspect and mistrust. Lugubris Ate frowns upon them, insomuch that
Areteus well calls it angorem animi, a vexation of the mind, a
perpetual agony. They can hardly be pleased, or eased, though in other
men's opinion most happy, go, tarry, run, ride, \authorfootnote{2501}-post equitem
sedet atra cura: they cannot avoid this feral plague, let them come in
what company they will, \authorfootnote{2502}haeret leteri lethalis arundo, as to a
deer that is struck, whether he run, go, rest with the herd, or alone,
this grief remains: irresolution, inconstancy, vanity of mind, their
fear, torture, care, jealousy, suspicion, \&c., continues, and they
cannot be relieved. So \authorfootnote{2503}he complained in the poet,
Domum revertor moestus, atque animo fere
Perturbato, atque incerto prae aegritudine,
Assido, accurrunt servi: succos detrahunt,
Video alios festinare, lectos sternere,
Coenam apparare, pro se quisque sedulo
Faciebant, quo illam mihi lenirent miseriam.

He came home sorrowful, and troubled in his mind, his servants did all
they possibly could to please him; one pulled off his socks, another
made ready his bed, a third his supper, all did their utmost endeavours
to ease his grief, and exhilarate his person, he was profoundly
melancholy, he had lost his son, illud angebat, that was his Cordolium,
his pain, his agony which could not be removed.

\subsection{Taedium vitae.}
Hence it proceeds many times, that they are weary of
their lives, and feral thoughts to offer violence to their own persons
come into their minds, taedium vitae is a common symptom, tarda fluunt,
ingrataque tempora, they are soon tired with all things; they will now
tarry, now be gone; now in bed they will rise, now up, then go to bed,
now pleased, then again displeased; now they like, by and by dislike
all, weary of all, sequitur nunc vivendi, nunc moriendi cupido, saith
Aurelianus, lib. 1. cap. 6, but most part \authorfootnote{2504}vitam damnant,
discontent, disquieted, perplexed upon every light, or no occasion,
object: often tempted, I say, to make away themselves: \authorfootnote{2505}Vivere
nolunt, mori nesciunt: they cannot die, they will not live: they
complain, weep, lament, and think they lead a most miserable life,
never was any man so bad, or so before, every poor man they see is most
fortunate in respect of them, every beggar that comes to the door is
happier than they are, they could be contented to change lives with
them, especially if they be alone, idle, and parted from their ordinary
company, molested, displeased, or provoked: grief, fear, agony,
discontent, wearisomeness, laziness, suspicion, or some such passion
forcibly seizeth on them. Yet by and by when they come in company
again, which they like, or be pleased, suam sententiam rursus damnant,
et vitae solatia delectantur, as Octavius Horatianus observes, lib. 2.
cap. 5, they condemn their former mislike, and are well pleased to
live. And so they continue, till with some fresh discontent they be
molested again, and then they are weary of their lives, weary of all,
they will die, and show rather a necessity to live, than a desire.

Claudius the emperor, as Sueton describes him\authorfootnote{2506}, had a spice of
this disease, for when he was tormented with the pain of his stomach,
he had a conceit to make away himself. Julius Caesar Claudinus, consil.
84. had a Polonian to his patient, so affected, that through \authorfootnote{2507}fear
and sorrow, with which he was still disquieted, hated his own life,
wished for death every moment, and to be freed of his misery.
Mercurialis another, and another that was often minded to despatch
himself, and so continued for many years.

\subsection{Suspicion, Jealousy.}
Suspicion, and jealousy, are general symptoms:
they are commonly distrustful, apt to mistake, and amplify, facile
irascibiles, \authorfootnote{2508}testy, pettish, peevish, and ready to snarl upon
every \authorfootnote{2509}small occasion, cum amicissimis, and without a cause, datum
vel non datum, it will be scandalum acceptum. If they speak in jest, he
takes it in good earnest. If they be not saluted, invited, consulted
with, called to counsel, \&c., or that any respect, small compliment, or
ceremony be omitted, they think themselves neglected, and contemned;
for a time that tortures them. If two talk together, discourse,
whisper, jest, or tell a tale in general, he thinks presently they mean
him, applies all to himself, de se putat omnia dici. Or if they talk
with him, he is ready to misconstrue every word they speak, and
interpret it to the worst; he cannot endure any man to look steadily on
him, speak to him almost, laugh, jest, or be familiar, or hem, or
point, cough, or spit, or make a noise sometimes, \&c. \authorfootnote{2510}He thinks
they laugh or point at him, or do it in disgrace of him, circumvent
him, contemn him; every man looks at him, he is pale, red, sweats for
fear and anger, lest somebody should observe him. He works upon it, and
long after this false conceit of an abuse troubles him. Montanus
consil. 22. gives instance in a melancholy Jew, that was Iracundior
Adria, so waspish and suspicious, tam facile iratus, that no man could
tell how to carry himself in his company.

\subsection{Inconstancy.}
Inconstant they are in all their actions, vertiginous,
restless, unapt to resolve of any business, they will and will not,
persuaded to and fro upon every small occasion, or word spoken: and yet
if once they be resolved, obstinate, hard to be reconciled. If they
abhor, dislike, or distaste, once settled, though to the better by
odds, by no counsel, or persuasion, to be removed. Yet in most things
wavering, irresolute, unable to deliberate, through fear, faciunt, et
mox facti poenitent (Areteus) avari, et paulo post prodigi. Now
prodigal, and then covetous, they do, and by-and-by repent them of that
which they have done, so that both ways they are troubled, whether they
do or do not, want or have, hit or miss, disquieted of all hands, soon
weary, and still seeking change, restless, I say, fickle, fugitive,
they may not abide to tarry in one place long.

\authorfootnote{2511}Romae rus optans, absentem rusticus urbem
Tollit ad astra---

no company long, or to persevere in any action or business.
\authorfootnote{2512}Et similis regum pueris, pappare minutum
Poscit, et iratus mammae lallare recusat,

eftsoons pleased, and anon displeased, as a man that's bitten with
fleas, or that cannot sleep turns to and fro in his bed, their restless
minds are tossed and vary, they have no patience to read out a book, to
play out a game or two, walk a mile, sit an hour, \&c., erected and
dejected in an instant; animated to undertake, and upon a word spoken
again discouraged.

\subsection{Passionate.}
Extreme passionate, Quicquid volunt valde volunt; and
what they desire, they do most furiously seek; anxious ever, and very
solicitous, distrustful, and timorous, envious, malicious, profuse one
while, sparing another, but most part covetous, muttering, repining,
discontent, and still complaining, grudging, peevish, injuriarum
tenaces, prone to revenge, soon troubled, and most violent in all their
imaginations, not affable in speech, or apt to vulgar compliment, but
surly, dull, sad, austere; cogitabundi still, very intent, and as
\authorfootnote{2513} Albertus Durer paints melancholy, like a sad woman leaning on
her arm with fixed looks, neglected habit, \&c., held therefore by some
proud, soft, sottish, or half-mad, as the Abderites esteemed of
Democritus: and yet of a deep reach, excellent apprehension, judicious,
wise, and witty: for I am of that \authorfootnote{2514}nobleman's mind, Melancholy
advanceth men's conceits, more than any humour whatsoever, improves
their meditations more than any strong drink or sack. They are of
profound judgment in some things, although in others non recte judicant
inquieti, saith Fracastorius, lib. 2. de Intell. And as Arculanus, c.
16. in 9. Rhasis, terms it, Judicium plerumque perversum, corrupti, cum
judicant honesta inhonesta, et amicitiam habent pro inimicitia: they
count honesty dishonesty, friends as enemies, they will abuse their
best friends, and dare not offend their enemies. Cowards most part et
ad inferendam injuriam timidissimi, saith Cardan, lib. 8. cap. 4. de
rerum varietate: loath to offend, and if they chance to overshoot
themselves in word or deed: or any small business or circumstance be
omitted, forgotten, they are miserably tormented, and frame a thousand
dangers and inconveniences to themselves, ex musca elephantem, if once
they conceit it: overjoyed with every good rumour, tale, or prosperous
event, transported beyond themselves: with every small cross again, bad
news, misconceived injury, loss, danger, afflicted beyond measure, in
great agony, perplexed, dejected, astonished, impatient, utterly
undone: fearful, suspicious of all. Yet again, many of them desperate
harebrains, rash, careless, fit to be assassinates, as being void of
all fear and sorrow, according to \authorfootnote{2515}Hercules de Saxonia, Most
audacious, and such as dare walk alone in the night, through deserts
and dangerous places, fearing none.

\subsection{Amorous.}
They are prone to love, and \authorfootnote{2516}easy to be taken;
Propensi ad amorem et excandescentiam (Montaltus cap. 21.) quickly
enamoured, and dote upon all, love one dearly, till they see another,
and then dote on her, Et hanc, et hanc, et illam, et omnes, the present
moves most, and the last commonly they love best. Yet some again
Anterotes, cannot endure the sight of a woman, abhor the sex, as that
same melancholy \authorfootnote{2517}duke of Muscovy, that was instantly sick, if he
came but in sight of them; and that \authorfootnote{2518}Anchorite, that fell into a
cold palsy, when a woman was brought before him.

\subsection{Humorous.}
Humorous they are beyond all measure, sometimes profusely
laughing, extraordinarily merry, and then again weeping without a
cause, (which is familiar with many gentlewomen,) groaning, sighing,
pensive, sad, almost distracted, multa absurda fingunt, et a ratione
aliena (saith \authorfootnote{2519}Frambesarius), they feign many absurdities, vain,
void of reason: one supposeth himself to be a dog, cock, bear, horse,
glass, butter, \&c. He is a giant, a dwarf, as strong as an hundred men,
a lord, duke, prince, \&c. And if he be told he hath a stinking breath,
a great nose, that he is sick, or inclined to such or such a disease,
he believes it eftsoons, and peradventure by force of imagination will
work it out. Many of them are immovable, and fixed in their conceits,
others vary upon every object, heard or seen. If they see a stage-play,
they run upon that a week after; if they hear music, or see dancing,
they have nought but bagpipes in their brain: if they see a combat,
they are all for arms. \authorfootnote{2520}If abused, an abuse troubles them long
after; if crossed, that cross, \&c. Restless in their thoughts and
actions, continually meditating, Velut aegri somnia, vanae finguntur
species; more like dreams, than men awake, they fain a company of
antic, fantastical conceits, they have most frivolous thoughts,
impossible to be effected; and sometimes think verily they hear and see
present before their eyes such phantasms or goblins, they fear,
suspect, or conceive, they still talk with, and follow them. In fine,
cogitationes somniantibus similes, id vigilant, quod alii somniant
cogitabundi, still, saith Avicenna, they wake, as others dream, and
such for the most part are their imaginations and conceits,
\authorfootnote{2521}absurd, vain, foolish toys, yet they are \authorfootnote{2522}most curious and
solicitous, continual, et supra modum, Rhasis cont. lib. 1. cap. 9.
praemeditantur de aliqua re. As serious in a toy, as if it were a most
necessary business, of great moment, importance, and still, still,
still thinking of it: saeviunt in se, macerating themselves. Though
they do talk with you, and seem to be otherwise employed, and to your
thinking very intent and busy, still that toy runs in their mind, that
fear, that suspicion, that abuse, that jealousy, that agony, that
vexation, that cross, that castle in the air, that crotchet, that
whimsy, that fiction, that pleasant waking dream, whatsoever it is. Nec
interrogant (saith \authorfootnote{2523}Fracastorius) nec interrogatis recte
respondent. They do not much heed what you say, their mind is on
another matter; ask what you will, they do not attend, or much intend
that business they are about, but forget themselves what they are
saying, doing, or should otherwise say or do, whither they are going,
distracted with their own melancholy thoughts. One laughs upon a
sudden, another smiles to himself, a third frowns, calls, his lips go
still, he acts with his hand as he walks, \&c. 'Tis proper to all
melancholy men, saith \authorfootnote{2524}Mercurialis, con. 11. What conceit they
have once entertained, to be most intent, violent, and continually
about it. Invitas occurrit, do what they may they cannot be rid of it,
against their wills they must think of it a thousand times over,
Perpetuo molestantur nec oblivisci possunt, they are continually
troubled with it, in company, out of company; at meat, at exercise, at
all times and places, \authorfootnote{2525}non desinunt ea, quae, minime volunt,
cogitare, if it be offensive especially, they cannot forget it, they
may not rest or sleep for it, but still tormenting themselves, Sysiphi
saxum volvunt sibi ipsis, as Brunner observes\authorfootnote{2526}, Perpetua calamitas
et miserabile flagellum.

\subsection{Bashfulness.}
\authorfootnote{2527}Crato, \authorfootnote{2528}Laurentius, and Fernelius, put
bashfulness for an ordinary symptom, sabrusticus pudor, or vitiosus
pudor, is a thing which much haunts and torments them. If they have
been misused, derided, disgraced, chidden, \&c., or by any perturbation
of mind, misaffected, it so far troubles them, that they become quite
moped many times, and so disheartened, dejected, they dare not come
abroad, into strange companies especially, or manage their ordinary
affairs, so childish, timorous, and bashful, they can look no man in
the face; some are more disquieted in this kind, some less, longer
some, others shorter, by fits, \&c., though some on the other side
(according to \authorfootnote{2529}Fracastorius) be inverecundi et pertinaces,
impudent and peevish. But most part they are very shamefaced, and that
makes them with Pet. Blesensis, Christopher Urswick, and many such, to
refuse honours, offices, and preferments, which sometimes fall into
their mouths, they cannot speak, or put forth themselves as others can,
timor hos, pudor impedit illos, timorousness and bashfulness hinder
their proceedings, they are contented with their present estate,
unwilling to undertake any office, and therefore never likely to rise.

For that cause they seldom visit their friends, except some familiars:
pauciloqui, of few words, and oftentimes wholly silent. \authorfootnote{2530}
Frambeserius, a Frenchman, had two such patients, omnino taciturnos,
their friends could not get them to speak: Rodericus a Fonseca consult.
tom. 2. 85. consil. gives instance in a young man, of twenty-seven
years of age, that was frequently silent, bashful, moped, solitary,
that would not eat his meat, or sleep, and yet again by fits apt to be
angry, \&c.

\subsection{Solitariness.}
Most part they are, as Plater notes, desides,
taciturni, aegre impulsi, nec nisi coacti procedunt, \&c. they will
scarce be compelled to do that which concerns them, though it be for
their good, so diffident, so dull, of small or no compliment,
unsociable, hard to be acquainted with, especially of strangers; they
had rather write their minds than speak, and above all things love
solitariness. Ob voluptatem, an ob timorem soli sunt? Are they so
solitary for pleasure (one asks,) or pain? for both; yet I rather think
for fear and sorrow, \&c.

\authorfootnote{2531}Hinc metuunt cupiuntque, dolent fugiuntque, nec auras
Respiciunt, clausi tenebris, et carcere caeco.

Hence 'tis they grieve and fear, avoiding light,
And shut themselves in prison dark from sight.

As Bellerophon in \authorfootnote{2532}Homer,
Qui miser in sylvis moerens errabat opacis,
Ipse suum cor edens, hominum vestigia vitans.

That wandered in the woods sad all alone,
Forsaking men's society, making great moan.

They delight in floods and waters, desert places, to walk alone in
orchards, gardens, private walks, back lanes, averse from company, as
Diogenes in his tub, or Timon Misanthropus \authorfootnote{2533}, they abhor all
companions at last, even their nearest acquaintances and most familiar
friends, for they have a conceit (I say) every man observes them, will
deride, laugh to scorn, or misuse them, confining themselves therefore
wholly to their private houses or chambers, fugiunt homines sine causa
(saith Rhasis) et odio habent, cont. l. 1. c. 9. they will diet
themselves, feed and live alone. It was one of the chiefest reasons why
the citizens of Abdera suspected Democritus to be melancholy and mad,
because that, as Hippocrates related in his Epistle to Philopaemenes,
\authorfootnote{2534}he forsook the city, lived in groves and hollow trees, upon a
green bank by a brook side, or confluence of waters all day long, and
all night. Quae quidem (saith he) plurimum atra bile vexatis et
melancholicis eveniunt, deserta frequentant, hominumque congressum
aversantur; \authorfootnote{2535}which is an ordinary thing with melancholy men. The
Egyptians therefore in their hieroglyphics expressed a melancholy man
by a hare sitting in her form, as being a most timorous and solitary
creature, Pierius Hieroglyph. l. 12. But this, and all precedent
symptoms, are more or less apparent, as the humour is intended or
remitted, hardly perceived in some, or not all, most manifest in
others. Childish in some, terrible in others; to be derided in one,
pitied or admired in another; to him by fits, to a second continuate:
and howsoever these symptoms be common and incident to all persons, yet
they are the more remarkable, frequent, furious and violent in
melancholy men. To speak in a word, there is nothing so vain, absurd,
ridiculous, extravagant, impossible, incredible, so monstrous a
chimera, so prodigious and strange, \authorfootnote{2536}such as painters and poets
durst not attempt, which they will not really fear, feign, suspect and
imagine unto themselves: and that which \authorfootnote{2537}Lod. Vives said in a jest
of a silly country fellow, that killed his ass for drinking up the
moon, ut lunam mundo redderet, you may truly say of them in earnest;
they will act, conceive all extremes, contrarieties, and
contradictions, and that in infinite varieties. Melancholici plane
incredibilia sibi persuadent, ut vix omnibus saeculis duo reperti sint,
qui idem imaginati sint (Erastus de Lamiis), scarce two of two thousand
that concur in the same symptoms. The tower of Babel never yielded such
confusion of tongues, as the chaos of melancholy doth variety of
symptoms. There is in all melancholy similitudo dissimilis, like men's
faces, a disagreeing likeness still; and as in a river we swim in the
same place, though not in the same numerical water; as the same
instrument affords several lessons, so the same disease yields
diversity of symptoms. Which howsoever they be diverse, intricate, and
hard to be confined, I will adventure yet in such a vast confusion and
generality to bring them into some order; and so descend to
particulars.

%SECT. III. MEMB. I. SUBSECT. III.-_Particular Symptoms from the influence of Stars, parts
\section[Symptoms from the influence of Stars]{Particular Symptoms from the influence of Stars, parts of the Body, and Humours.}

\lettrine{S}{ome} men have peculiar symptoms, according to their temperament and
crisis, which they had from the stars and those celestial influences,
variety of wits and dispositions, as Anthony Zara contends, Anat.
ingen. sect. 1. memb. 11, 12, 13, 14. plurimum irritant influentiae,
caelestes, unde cientur animi aegritudines et morbi corporum. \authorfootnote{2538}One
saith, diverse diseases of the body and mind proceed from their
influences, \authorfootnote{2539}as I have already proved out of Ptolemy, Pontanus,
Lemnius, Cardan, and others as they are principal significators of
manners, diseases, mutually irradiated, or lords of the geniture, \&c.

Ptolomeus in his centiloquy, Hermes, or whosoever else the author of
that tract, attributes all these symptoms, which are in melancholy men,
to celestial influences: which opinion Mercurialis de affect, lib. cap.
10. rejects; but, as I say, \authorfootnote{2540}Jovianus Pontanus and others stiffly
defend. That some are solitary, dull, heavy, churlish; some again
blithe, buxom, light, and merry, they ascribe wholly to the stars. As
if Saturn be predominant in his nativity, and cause melancholy in his
temperature, then \authorfootnote{2541}he shall be very austere, sullen, churlish,
black of colour, profound in his cogitations, full of cares, miseries,
and discontents, sad and fearful, always silent, solitary, still
delighting in husbandry, in woods, orchards, gardens, rivers, ponds,
pools, dark walks and close: Cogitationes sunt velle aedificare, velle
arbores plantare, agros colere, \&c. To catch birds, fishes, \&c. still
contriving and musing of such matters. If Jupiter domineers, they are
more ambitious, still meditating of kingdoms, magistracies, offices,
honours, or that they are princes, potentates, and how they would carry
themselves, \&c. If Mars, they are all for wars, brave combats,
monomachies, testy, choleric, harebrain, rash, furious, and violent in
their actions. They will feign themselves victors, commanders, are
passionate and satirical in their speeches, great braggers, ruddy of
colour. And though they be poor in show, vile and base, yet like
Telephus and Peleus in the \authorfootnote{2542}poet, Ampullas jactant et
sesquipedalia verba, forget their swelling and gigantic words, their
mouths are full of myriads, and tetrarchs at their tongues' end. If the
sun, they will be lords, emperors, in conceit at least, and monarchs,
give offices, honours, \&c. If Venus, they are still courting of their
mistresses, and most apt to love, amorously given, they seem to hear
music, plays, see fine pictures, dancers, merriments, and the like.

Ever in love, and dote on all they see. Mercurialists are solitary,
much in contemplation, subtle, poets, philosophers, and musing most
part about such matters. If the moon have a hand, they are all for
peregrinations, sea voyages, much affected with travels, to discourse,
read, meditate of such things; wandering in their thoughts, diverse,
much delighting in waters, to fish, fowl, \&c.

But the most immediate symptoms proceed from the temperature itself,
and the organical parts, as head, liver, spleen, mesaraic veins, heart,
womb, stomach, \&c., and most especially from distemperature of spirits
(which, as Hercules de Saxonia contends\authorfootnote{2543}, are wholly immaterial),
or from the four humours in those seats, whether they be hot or cold,
natural, unnatural, innate or adventitious, intended or remitted,
simple or mixed, their diverse mixtures, and several adustions,
combinations, which may be as diversely varied, as those \authorfootnote{2544}four
first qualities in \authorfootnote{2545} Clavius, and produce as many several symptoms
and monstrous fictions as wine doth effect, which as Andreas Bachius
observes, lib. 3. de vino, cap. 20. are infinite. Of greater note be
these.

If it be natural melancholy, as Lod. Mercatus, lib. 1. cap. 17. de
melan. T. Bright. c. 16. hath largely described, either of the spleen,
or of the veins, faulty by excess of quantity, or thickness of
substance, it is a cold and dry humour, as Montanus affirms, consil. 26
the parties are sad, timorous and fearful. Prosper Calenus, in his book
de atra bile, will have them to be more stupid than ordinary, cold,
heavy, solitary, sluggish. Si multam atram bilem et frigidam habent.

Hercules de Saxonia, c. 19. l. 7. \authorfootnote{2546}holds these that are naturally
melancholy, to be of a leaden colour or black, and so doth Guianerius,
c. 3. tract. 15. and such as think themselves dead many times, or that
they see, talk with black men, dead men, spirits and goblins
frequently, if it be in excess. These symptoms vary according to the
mixture of those four humours adust, which is unnatural melancholy. For
as Trallianus hath written, cap. 16. l. 7. \authorfootnote{2547}There is not one cause
of this melancholy, nor one humour which begets, but diverse diversely
intermixed, from whence proceeds this variety of symptoms: and those
varying again as they are hot or cold. \authorfootnote{2548}Cold melancholy (saith
Benedic. Vittorius Faventinus pract. mag.) is a cause of dotage, and
more mild symptoms, if hot or more adust, of more violent passions, and
furies. Fracastorius, l. 2. de intellect. will have us to consider well
of it, \authorfootnote{2549}with what kind of melancholy every one is troubled, for it
much avails to know it; one is enraged by fervent heat, another is
possessed by sad and cold; one is fearful, shamefaced; the other
impudent and bold; as Ajax, Arma rapit superosque furens inpraelia
poscit: quite mad or tending to madness. Nunc hos, nunc impetit illos.

Bellerophon on the other side, solis errat male sanus in agris, wanders
alone in the woods; one despairs, weeps, and is weary of his life,
another laughs, \&c. All which variety is produced from the several
degrees of heat and cold, which \authorfootnote{2550}Hercules de Saxonia will have
wholly proceed from the distemperature of spirits alone, animal
especially, and those immaterial, the next and immediate causes of
melancholy, as they are hot, cold, dry, moist, and from their agitation
proceeds that diversity of symptoms, which he reckons up, in the
\authorfootnote{2551}thirteenth chap. of his Tract of Melancholy, and that largely
through every part. Others will have them come from the diverse
adustion of the four humours, which in this unnatural melancholy, by
corruption of blood, adust choler, or melancholy natural, \authorfootnote{2552}by
excessive distemper of heat turned, in comparison of the natural, into
a sharp lye by force of adustion, cause, according to the diversity of
their matter, diverse and strange symptoms, which T. Bright reckons up
in his following chapter. So doth \authorfootnote{2553}Arculanus, according to the
four principal humours adust, and many others.

For example, if it proceed from phlegm, (which is seldom and not so
frequently as the rest) \authorfootnote{2554}it stirs up dull symptoms, and a kind of
stupidity, or impassionate hurt: they are sleepy, saith
\authorfootnote{2555}Savanarola, dull, slow, cold, blockish, ass-like, Asininam
melancholiam, \authorfootnote{2556} Melancthon calls it, they are much given to
weeping, and delight in waters, ponds, pools, rivers, fishing, fowling,
\&c. (Arnoldus breviar. 1. cap. 18.) They are \authorfootnote{2557}pale of colour,
slothful, apt to sleep, heavy; \authorfootnote{2558}much troubled with headache,
continual meditation, and muttering to themselves; they dream of
waters, \authorfootnote{2559}that they are in danger of drowning, and fear such
things, Rhasis. They are fatter than others that are melancholy, of a
muddy complexion, apter to spit, \authorfootnote{2560} sleep, more troubled with rheum
than the rest, and have their eyes still fixed on the ground. Such a
patient had Hercules de Saxonia, a widow in Venice, that was fat and
very sleepy still; Christophorus a Vega another affected in the same
sort. If it be inveterate or violent, the symptoms are more evident,
they plainly denote and are ridiculous to others, in all their
gestures, actions, speeches; imagining impossibilities, as he in
Christophorus a Vega, that thought he was a tun of wine, \authorfootnote{2561}and that
Siennois, that resolved within himself not to piss, for fear he should
drown all the town.

If it proceed from blood adust, or that there be a mixture of blood in
it, \authorfootnote{2562}such are commonly ruddy of complexion, and high-coloured,
according to Salust. Salvianus, and Hercules de Saxonia. And as
Savanarola, Vittorius Faventinus Emper. farther adds, \authorfootnote{2563}the veins
of their eyes be red, as well as their faces. They are much inclined to
laughter, witty and merry, conceited in discourse, pleasant, if they be
not far gone, much given to music, dancing, and to be in women's
company. They meditate wholly on such things, and think \authorfootnote{2564}they see
or hear plays, dancing, and suchlike sports (free from all fear and
sorrow, as Hercules de Saxonia supposeth\authorfootnote{2565}) If they be more
strongly possessed with this kind of melancholy, Arnoldus adds,
Breviar. lib. 1. cap. 18. Like him of Argos in the Poet, that sate
laughing \authorfootnote{2566}all day long, as if he had been at a theatre. Such
another is mentioned by \authorfootnote{2567}Aristotle, living at Abydos, a town of
Asia Minor, that would sit after the same fashion, as if he had been
upon a stage, and sometimes act himself; now clap his hands, and laugh,
as if he had been well pleased with the sight. Wolfius relates of a
country fellow called Brunsellius, subject to this humour, \authorfootnote{2568}that
being by chance at a sermon, saw a woman fall off from a form half
asleep, at which object most of the company laughed, but he for his
part was so much moved, that for three whole days after he did nothing
but laugh, by which means he was much weakened, and worse a long time
following. Such a one was old Sophocles, and Democritus himself had
hilare delirium, much in this vein. Laurentius cap. 3. de melan. thinks
this kind of melancholy, which is a little adust with some mixture of
blood, to be that which Aristotle meant, when he said melancholy men of
all others are most witty, which causeth many times a divine
ravishment, and a kind of enthusiasmus, which stirreth them up to be
excellent philosophers, poets, prophets, \&c. Mercurialis, consil. 110.
gives instance in a young man his patient, sanguine melancholy,
\authorfootnote{2569}of a great wit, and excellently learned.

If it arise from choler adust, they are bold and impudent, and of a
more harebrain disposition, apt to quarrel, and think of such things,
battles, combats, and their manhood, furious; impatient in discourse,
stiff, irrefragable and prodigious in their tenets; and if they be
moved, most violent, outrageous, \authorfootnote{2570}ready to disgrace, provoke any,
to kill themselves and others; Arnoldus adds, stark mad by fits,
\authorfootnote{2571}they sleep little, their urine is subtle and fiery. (Guianerius.)
In their fits you shall hear them speak all manner of languages,
Hebrew, Greek, and Latin, that never were taught or knew them before.

Apponensis in com. in Pro. sec. 30. speaks of a mad woman that spake
excellent good Latin: and Rhasis knew another, that could prophecy in
her fit, and foretell things truly to come. \authorfootnote{2572}Guianerius had a
patient could make Latin verses when the moon was combust, otherwise
illiterate. Avicenna and some of his adherents will have these
symptoms, when they happen, to proceed from the devil, and that they
are rather demoniaci, possessed, than mad or melancholy, or both
together, as Jason Pratensis thinks, Immiscent se mali genii, \&c. but
most ascribe it to the humour, which opinion Montaltus cap. 21. stiffly
maintains, confuting Avicenna and the rest, referring it wholly to the
quality and disposition of the humour and subject. Cardan de rerum var.
lib. 8. cap. 10. holds these men of all others fit to be assassins,
bold, hardy, fierce, and adventurous, to undertake anything by reason
of their choler adust. \authorfootnote{2573}This humour, saith he, prepares them to
endure death itself, and all manner of torments with invincible
courage, and 'tis a wonder to see with what alacrity they will undergo
such tortures, ut supra naturam res videatur: he ascribes this
generosity, fury, or rather stupidity, to this adustion of choler and
melancholy: but I take these rather to be mad or desperate, than
properly melancholy; for commonly this humour so adust and hot,
degenerates into madness.

If it come from melancholy itself adust, those men, saith Avicenna,
\authorfootnote{2574} are usually sad and solitary, and that continually, and in
excess, more than ordinarily suspicious more fearful, and have long,
sore, and most corrupt imaginations; cold and black, bashful, and so
solitary, that as Arnoldus writes\authorfootnote{2575}, they will endure no company,
they dream of graves still, and dead men, and think themselves
bewitched or dead: if it be extreme, they think they hear hideous
noises, see and talk \authorfootnote{2576}with black men, and converse familiarly with
devils, and such strange chimeras and visions, (Gordonius) or that they
are possessed by them, that somebody talks to them, or within them.
Tales melancholici plerumque daemoniaci, Montaltus consil. 26. ex
Avicenna. Valescus de Taranta had such a woman in cure, \authorfootnote{2577}that
thought she had to do with the devil: and Gentilis Fulgosus quaest. 55.
writes that he had a melancholy friend, that \authorfootnote{2578} had a black man in
the likeness of a soldier still following him wheresoever he was.

Laurentius cap. 7. hath many stories of such as have thought themselves
bewitched by their enemies; and some that would eat no meat as being
dead. \authorfootnote{2579}\emph{Anno} 1550 an advocate of Paris fell into such a
melancholy fit, that he believed verily he was dead, he could not be
persuaded otherwise, or to eat or drink, till a kinsman of his, a
scholar of Bourges, did eat before him dressed like a corse. The story,
saith Serres, was acted in a comedy before Charles the Ninth. Some
think they are beasts, wolves, hogs, and cry like dogs, foxes, bray
like asses, and low like kine, as King Praetus' daughters.

\authorfootnote{2580}Hildesheim spicel. 2. de mania, hath an example of a Dutch baron
so affected, and Trincavelius lib. 1. consil. 11. another of a nobleman
in his country, \authorfootnote{2581}that thought he was certainly a beast, and would
imitate most of their voices, with many such symptoms, which may
properly be reduced to this kind.

If it proceed from the several combinations of these four humours, or
spirits, Herc. de Saxon. adds hot, cold, dry, moist, dark, confused,
settled, constringed, as it participates of matter, or is without
matter, the symptoms are likewise mixed. One thinks himself a giant,
another a dwarf. One is heavy as lead, another is as light as a
feather. Marcellus Donatus l. 2. cap. 41. makes mention out of Seneca,
of one Seneccio, a rich man, \authorfootnote{2582}that thought himself and everything
else he had, great: great wife, great horses, could not abide little
things, but would have great pots to drink in, great hose, and great
shoes bigger than his feet. Like her in \authorfootnote{2583}Trallianus, that supposed
she could shake all the world with her finger, and was afraid to clinch
her hand together, lest she should crush the world like an apple in
pieces: or him in Galen, that thought he was Atlas\authorfootnote{2584}, and sustained
heaven with his shoulders. Another thinks himself so little, that he
can creep into a mouse-hole: one fears heaven will fall on his head: a
second is a cock; and such a one, \authorfootnote{2585}Guianerius saith he saw at
Padua, that would clap his hands together and crow. \authorfootnote{2586}Another
thinks he is a nightingale, and therefore sings all the night long;
another he is all glass, a pitcher, and will therefore let nobody come
near him, and such a one \authorfootnote{2587}Laurentius gives out upon his credit,
that he knew in France. Christophorus a Vega cap. 3. lib. 14. Skenkius
and Marcellus Donatus l. 2. cap. 1. have many such examples, and one
amongst the rest of a baker in Ferrara that thought he was composed of
butter, and durst not sit in the sun, or come near the fire for fear of
being melted: of another that thought he was a case of leather, stuffed
with wind. Some laugh, weep; some are mad, some dejected, moped, in
much agony, some by fits, others continuate, \&c. Some have a corrupt
ear, they think they hear music, or some hideous noise as their
phantasy conceives, corrupt eyes, some smelling, some one sense, some
another. \authorfootnote{2588}Lewis the Eleventh had a conceit everything did stink
about him, all the odoriferous perfumes they could get, would not ease
him, but still he smelled a filthy stink. A melancholy French poet in
\authorfootnote{2589}Laurentius, being sick of a fever, and troubled with waking, by
his physicians was appointed to use unguentum populeum to anoint his
temples; but he so distasted the smell of it, that for many years
after, all that came near him he imagined to scent of it, and would let
no man talk with him but aloof off, or wear any new clothes, because he
thought still they smelled of it; in all other things wise and
discreet, he would talk sensibly, save only in this. A gentleman in
Limousin, saith Anthony Verdeur, was persuaded he had but one leg,
affrighted by a wild boar, that by chance struck him on the leg; he
could not be satisfied his leg was sound (in all other things well)
until two Franciscans by chance coming that way, fully removed him from
the conceit. Sed abunde fabularum audivimus,-enough of story-telling.

%SECT. III. MEMB. I. SUBSECT. IV.-_Symptoms from Education, Custom, continuance of Time, our Condition, mixed with other Diseases, by Fits, Inclination, \&c._
\section[Symptoms from Education and Custom]{Symptoms from Education, Custom, continuance of Time, our Condition, mixed with other Diseases, by Fits, Inclination, \&c.}

\lettrine{A}{nother} great occasion of the variety of these symptoms proceeds from
custom, discipline, education, and several inclinations, \authorfootnote{2590}this
humour will imprint in melancholy men the objects most answerable to
their condition of life, and ordinary actions, and dispose men
according to their several studies and callings. If an ambitious man
become melancholy, he forthwith thinks he is a king, an emperor, a
monarch, and walks alone, pleasing himself with a vain hope of some
future preferment, or present as he supposeth, and withal acts a lord's
part, takes upon him to be some statesman or magnifico, makes conges,
gives entertainment, looks big, \&c. Francisco Sansovino records of a
melancholy man in Cremona, that would not be induced to believe but
that he was pope, gave pardons, made cardinals, \&c. \authorfootnote{2591}Christophorus
a Vega makes mention of another of his acquaintance, that thought he
was a king, driven from his kingdom, and was very anxious to recover
his estate. A covetous person is still conversant about purchasing of
lands and tenements, plotting in his mind how to compass such and such
manors, as if he were already lord of, and able to go through with it;
all he sees is his, re or spe, he hath devoured it in hope, or else in
conceit esteems it his own: like him in \authorfootnote{2592}Athenaeus, that thought
all the ships in the haven to be his own. A lascivious inamorato plots
all the day long to please his mistress, acts and struts, and carries
himself as if she were in presence, still dreaming of her, as Pamphilus
of his Glycerium, or as some do in their morning sleep. \authorfootnote{2593}
Marcellus Donatus knew such a gentlewoman in Mantua, called Elionora
Meliorina, that constantly believed she was married to a king, and
\authorfootnote{2594} would kneel down and talk with him, as if he had been there
present with his associates; and if she had found by chance a piece of
glass in a muck-hill or in the street, she would say that it was a
jewel sent from her lord and husband. If devout and religious, he is
all for fasting, prayer, ceremonies, alms, interpretations, visions,
prophecies, revelations, \authorfootnote{2595} he is inspired by the Holy Ghost, full
of the spirit: one while he is saved, another while damned, or still
troubled in mind for his sins, the devil will surely have him, \&c. more
of these in the third partition of love-melancholy. \authorfootnote{2596}A scholar's
mind is busied about his studies, he applauds himself for that he hath
done, or hopes to do, one while fearing to be out in his next exercise,
another while contemning all censures; envies one, emulates another; or
else with indefatigable pains and meditation, consumes himself. So of
the rest, all which vary according to the more remiss and violent
impression of the object, or as the humour itself is intended or
remitted. For some are so gently melancholy, that in all their
carriage, and to the outward apprehension of others it can hardly be
discerned, yet to them an intolerable burden, and not to be endured.

\authorfootnote{2597}Quaedam occulta quaedam manifesta, some signs are manifest and
obvious to all at all times, some to few, or seldom, or hardly
perceived; let them keep their own council, none will take notice or
suspect them. They do not express in outward show their depraved
imaginations, as Hercules de Saxonia observes\authorfootnote{2598}, but conceal them
wholly to themselves, and are very wise men, as I have often seen; some
fear, some do not fear at all, as such as think themselves kings or
dead, some have more signs, some fewer, some great, some less, some
vex, fret, still fear, grieve, lament, suspect, laugh, sing, weep,
chafe, \&c. by fits (as I have said) or more during and permanent. Some
dote in one thing, are most childish, and ridiculous, and to be
wondered at in that, and yet for all other matters most discreet and
wise. To some it is in disposition, to another in habit; and as they
write of heat and cold, we may say of this humour, one is melancholicus
ad octo, a second two degrees less, a third halfway. 'Tis
superparticular, sesquialtera, sesquitertia, and superbipartiens
tertias, quintas Melancholiae, \&c. all those geometrical proportions
are too little to express it. \authorfootnote{2599}It comes to many by fits, and goes;
to others it is continuate: many (saith \authorfootnote{2600}Faventinus) in spring and
fall only are molested, some once a year, as that Roman \authorfootnote{2601} Galen
speaks of: \authorfootnote{2602}one, at the conjunction of the moon alone, or some
unfortunate aspects, at such and such set hours and times, like the
sea-tides, to some women when they be with child, as Plater
notes\authorfootnote{2603}, never otherwise: to others 'tis settled and fixed; to one led
about and variable still by that ignis fatuus of phantasy, like an
arthritis or running gout, 'tis here and there, and in every joint,
always molesting some part or other; or if the body be free, in a
myriad of forms exercising the mind. A second once peradventure in his
life hath a most grievous fit, once in seven years, once in five years,
even to the extremity of madness, death, or dotage, and that upon, some
feral accident or perturbation, terrible object, and for a time, never
perhaps so before, never after. A third is moved upon all such
troublesome objects, cross fortune, disaster, and violent passions,
otherwise free, once troubled in three or four years. A fourth, if
things be to his mind, or he in action, well pleased, in good company,
is most jocund, and of a good complexion: if idle, or alone, a la mort,
or carried away wholly with pleasant dreams and phantasies, but if once
crossed and displeased,
Pectore concipiet nil nisi triste suo;

He will imagine naught save sadness in his heart;

his countenance is altered on a sudden, his heart heavy, irksome
thoughts crucify his soul, and in an instant he is moped or weary of
his life, he will kill himself. A fifth complains in his youth, a sixth
in his middle age, the last in his old age.

Generally thus much we may conclude of melancholy; that it is
\authorfootnote{2604}most pleasant at first, I say, mentis gratissimus error, \authorfootnote{2605}a
most delightsome humour, to be alone, dwell alone, walk alone,
meditate, lie in bed whole days, dreaming awake as it were, and frame a
thousand fantastical imaginations unto themselves. They are never
better pleased than when they are so doing, they are in paradise for
the time, and cannot well endure to be interrupt; with him in the poet,
\authorfootnote{2606}pol me occidistis amici, non servastis ait? you have undone him,
he complains, if you trouble him: tell him what inconvenience will
follow, what will be the event, all is one, canis ad vomitum,
\authorfootnote{2607}'tis so pleasant he cannot refrain. He may thus continue
peradventure many years by reason of a strong temperature, or some
mixture of business, which may divert his cogitations: but at the last
laesa imaginatio, his phantasy is crazed, and now habituated to such
toys, cannot but work still like a fate, the scene alters upon a
sudden, fear and sorrow supplant those pleasing thoughts, suspicion,
discontent, and perpetual anxiety succeed in their places; so by little
and little, by that shoeing-horn of idleness, and voluntary
solitariness, melancholy this feral fiend is drawn on, \authorfootnote{2608}et quantum
vertice ad auras Aethereas, tantum radice in Tartara tendit, extending
up, by its branches, so far towards Heaven, as, by its roots, it does
down towards Tartarus; it was not so delicious at first, as now it is
bitter and harsh; a cankered soul macerated with cares and discontents,
taedium vitae, impatience, agony, inconstancy, irresolution,
precipitate them unto unspeakable miseries. They cannot endure company,
light, or life itself, some unfit for action, and the like. \authorfootnote{2609}Their
bodies are lean and dried up, withered, ugly, their looks harsh, very
dull, and their souls tormented, as they are more or less entangled, as
the humour hath been intended, or according to the continuance of time
they have been troubled.

To discern all which symptoms the better, \authorfootnote{2610}Rhasis the Arabian
makes three degrees of them. The first is, falsa cogitatio, false
conceits and idle thoughts: to misconstrue and amplify, aggravating
everything they conceive or fear; the second is, falso cogitata loqui,
to talk to themselves, or to use inarticulate incondite voices,
speeches, obsolete gestures, and plainly to utter their minds and
conceits of their hearts, by their words and actions, as to laugh,
weep, to be silent, not to sleep, eat their meat, \&c.: the third is to
put in practice \authorfootnote{2611}that which they think or speak. Savanarola, Rub.
11. tract. 8. cap. 1. de aegritudine, confirms as much, \authorfootnote{2612}when he
begins to express that in words, which he conceives in his heart, or
talks idly, or goes from one thing to another, which \authorfootnote{2613}Gordonius
calls nec caput habentia, nec caudam, (having neither head nor tail,)
he is in the middle way: \authorfootnote{2614} but when he begins to act it likewise,
and to put his fopperies in execution, he is then in the extent of
melancholy, or madness itself. This progress of melancholy you shall
easily observe in them that have been so affected, they go smiling to
themselves at first, at length they laugh out; at first solitary, at
last they can endure no company: or if they do, they are now dizzards,
past sense and shame, quite moped, they care not what they say or do,
all their actions, words, gestures, are furious or ridiculous. At first
his mind is troubled, he doth not attend what is said, if you tell him
a tale, he cries at last, what said you? but in the end he mutters to
himself, as old women do many times, or old men when they sit alone,
upon a sudden they laugh, whoop, halloo, or run away, and swear they
see or hear players, \authorfootnote{2615}devils, hobgoblins, ghosts, strike, or
strut, \&c., grow humorous in the end; like him in the poet, saepe
ducentos, saepe decem servos, (at one time followed by two hundred
servants, at another only by ten) he will dress himself, and undress,
careless at last, grows insensible, stupid, or mad. \authorfootnote{2616}He howls like
a wolf, barks like a dog, and raves like Ajax and Orestes, hears music
and outcries, which no man else hears. As he did\authorfootnote{2617} whom Amatus
Lusitanus mentioneth cent. 3, cura. 55, or that woman in
\authorfootnote{2618}Springer, that spake many languages, and said she was possessed:
that farmer in \authorfootnote{2619}Prosper Calenius, that disputed and discoursed
learnedly in philosophy and astronomy, with Alexander Achilles his
master, at Bologna, in Italy. But of these I have already spoken.

Who can sufficiently speak of these symptoms, or prescribe rules to
comprehend them? as Echo to the painter in Ausonius, vane quid
affectas, \&c., foolish fellow; what wilt? if you must needs paint me,
paint a voice, et similem si vis pingere, pinge sonum; if you will
describe melancholy, describe a fantastical conceit, a corrupt
imagination, vain thoughts and different, which who can do? The four
and twenty letters make no more variety of words in diverse languages,
than melancholy conceits produce diversity of symptoms in several
persons. They are irregular, obscure, various, so infinite, Proteus
himself is not so diverse, you may as well make the moon a new coat, as
a true character of a melancholy man; as soon find the motion of a bird
in the air, as the heart of man, a melancholy man. They are so
confused, I say, diverse, intermixed with other diseases. As the
species be confounded (which \authorfootnote{2620}I have showed) so are the symptoms;
sometimes with headache, cachexia, dropsy, stone; as you may perceive
by those several examples and illustrations, collected by \authorfootnote{2621}
Hildesheim spicel. 2. Mercurialis consil. 118. cap. 6 and 11. with
headache, epilepsy, priapismus. Trincavelius consil. 12. lib. 1.
consil. 49. with gout: caninus appetitus. Montanus consil. 26, \&c. 23,
234, 249, with falling-sickness, headache, vertigo, lycanthropia, \&c.
J. Caesar Claudinus consult. 4. consult. 89 and 116. with gout, agues,
haemorrhoids, stone, \&c., who can distinguish these melancholy symptoms
so intermixed with others, or apply them to their several kinds,
confine them into method? 'Tis hard I confess, yet I have disposed of
them as I could, and will descend to particularise them according to
their species. For hitherto I have expatiated in more general lists or
terms, speaking promiscuously of such ordinary signs, which occur
amongst writers. Not that they are all to be found in one man, for that
were to paint a monster or chimera, not a man: but some in one, some in
another, and that successively or at several times.

Which I have been the more curious to express and report; not to
upbraid any miserable man, or by way of derision, (I rather pity them,)
but the better to discern, to apply remedies unto them; and to show
that the best and soundest of us all is in great danger; how much we
ought to fear our own fickle estates, remember our miseries and
vanities, examine and humiliate ourselves, seek to God, and call to Him
for mercy, that needs not look for any rods to scourge ourselves, since
we carry them in our bowels, and that our souls are in a miserable
captivity, if the light of grace and heavenly truth doth not shine
continually upon us: and by our discretion to moderate ourselves, to be
more circumspect and wary in the midst of these dangers.

%SECT. III. MEMB. II. SUBSECT. I.-_Symptoms of Head-Melancholy_.
\section{Symptoms of Head-Melancholy.}

\lettrine{I}{f} \authorfootnote{2622}no symptoms appear about the stomach, nor the blood be
misaffected, and fear and sorrow continue, it is to be thought the
brain itself is troubled, by reason of a melancholy juice bred in it,
or otherwise conveyed into it, and that evil juice is from the
distemperature of the part, or left after some inflammation, thus far
Piso. But this is not always true, for blood and hypochondries both are
often affected even in head-melancholy. \authorfootnote{2623}Hercules de Saxonia
differs here from the common current of writers, putting peculiar signs
of head-melancholy, from the sole distemperature of spirits in the
brain, as they are hot, cold, dry, moist, all without matter from the
motion alone, and tenebrosity of spirits; of melancholy which proceeds
from humours by adustion, he treats apart, with their several symptoms
and cures. The common signs, if it be by essence in the head, are
ruddiness of face, high sanguine complexion, most part rubore saturato,
\authorfootnote{2624}one calls it, a bluish, and sometimes full of pimples, with red
eyes. Avicenna l. 3, Fen. 2, Tract. 4, c. 18. Duretus and others out of
Galen, de affect. l. 3, c. 6. \authorfootnote{2625}Hercules de Saxonia to this of
redness of face, adds heaviness of the head, fixed and hollow eyes.

\authorfootnote{2626}If it proceed from dryness of the brain, then their heads will be
light, vertiginous, and they most apt to wake, and to continue whole
months together without sleep. Few excrements in their eyes and
nostrils, and often bald by reason of excess of dryness, Montaltus
adds, c. 17. If it proceed from moisture: dullness, drowsiness,
headache follows; and as Salust. Salvianus, c. 1, l. 2, out of his own
experience found, epileptical, with a multitude of humours in the head.

They are very bashful, if ruddy, apt to blush, and to be red upon all
occasions, praesertim si metus accesserit. But the chiefest symptom to
discern this species, as I have said, is this, that there be no notable
signs in the stomach, hypochondries, or elsewhere, digna, as 
Montaltus terms them\authorfootnote{2627}, or of greater note, because oftentimes the
passions of the stomach concur with them. Wind is common to all three
species, and is not excluded, only that of the hypochondries is
\authorfootnote{2628}more windy than the rest, saith Hollerius. Aetius tetrab. l. 2,
sc. 2, c. 9 and 10, maintains the same, \authorfootnote{2629}if there be more signs,
and more evident in the head than elsewhere, the brain is primarily
affected, and prescribes head-melancholy to be cured by meats amongst
the rest, void of wind, and good juice, not excluding wind, or corrupt
blood, even in head-melancholy itself: but these species are often
confounded, and so are their symptoms, as I have already proved. The
symptoms of the mind are superfluous and continual cogitations;
\authorfootnote{2630}for when the head is heated, it scorcheth the blood, and from
thence proceed melancholy fumes, which trouble the mind, Avicenna. They
are very choleric, and soon hot, solitary, sad, often silent, watchful,
discontent, Montaltus, cap. 24. If anything trouble them, they cannot
sleep, but fret themselves still, till another object mitigate, or time
wear it out. They have grievous passions, and immoderate perturbations
of the mind, fear, sorrow, \&c., yet not so continuate, but that they
are sometimes merry, apt to profuse laughter, which is more to be
wondered at, and that by the authority of \authorfootnote{2631}Galen himself, by
reason of mixture of blood, praerubri jocosis delectantur, et irrisores
plerumque sunt, if they be ruddy, they are delighted in jests, and
oftentimes scoffers themselves, conceited: and as Rodericus a Vega
comments on that place of Galen, merry, witty, of a pleasant
disposition, and yet grievously melancholy anon after: omnia discunt
sine doctore, saith Aretus, they learn without a teacher: and as
\authorfootnote{2632}Laurentius supposeth, those feral passions and symptoms of such
as think themselves glass, pitchers, feathers, \&c., speak strange
languages, a colore cerebri (if it be in excess) from the brain's
distempered heat.

%SECT. III. MEMB. II. SUBSECT. II.-_Symptoms of windy Hypochondriacal Melancholy_.
\section{Symptoms of windy Hypochondriacal Melancholy.}

\lettrine{I}{n} this hypochondriacal or flatuous melancholy, the symptoms are so
ambiguous, saith \authorfootnote{2633}Crato in a counsel of his for a noblewoman, that
the most exquisite physicians cannot determine of the part affected.

Matthew Flaccius, consulted about a noble matron, confessed as much,
that in this malady he with Hollerius, Fracastorius, Falopius, and
others, being to give their sentence of a party labouring of
hypochondriacal melancholy, could not find out by the symptoms which
part was most especially affected; some said the womb, some heart, some
stomach, \&c., and therefore Crato, consil. 24. lib. 1. boldly avers,
that in this diversity of symptoms, which commonly accompany this
disease, \authorfootnote{2634}no physician can truly say what part is affected. Galen
lib. 3. de loc. affect., reckons up these ordinary symptoms, which all
the Neoterics repeat of Diocles; only this fault he finds with him,
that he puts not fear and sorrow amongst the other signs. Trincavelius
excuseth Diocles, lib. 3. consil. 35. because that oftentimes in a
strong head and constitution, a generous spirit, and a valiant, these
symptoms appear not, by reason of his valour and courage.

\authorfootnote{2635}Hercules de Saxonia (to whom I subscribe) is of the same mind
(which I have before touched) that fear and sorrow are not general
symptoms; some fear and are not sad; some be sad and fear not; some
neither fear nor grieve. The rest are these, beside fear and sorrow,
\authorfootnote{2636}sharp belchings, fulsome crudities, heat in the bowels, wind and
rumbling in the guts, vehement gripings, pain in the belly and stomach
sometimes, after meat that is hard of concoction, much watering of the
stomach, and moist spittle, cold sweat, importunus sudor, unseasonable
sweat all over the body, as Octavius Horatianus lib. 2. cap. 5. calls
it; cold joints, indigestion, \authorfootnote{2637}they cannot endure their own
fulsome belchings, continual wind about their hypochondries, heat and
griping in their bowels, praecordia sursum convelluntur, midriff and
bowels are pulled up, the veins about their eyes look red, and swell
from vapours and wind. Their ears sing now and then, vertigo and
giddiness come by fits, turbulent dreams, dryness, leanness, apt they
are to sweat upon all occasions, of all colours and complexions. Many
of them are high-coloured especially after meals, which symptom
Cardinal Caecius was much troubled with, and of which he complained to
Prosper Calenus his physician, he could not eat, or drink a cup of
wine, but he was as red in the face as if he had been at a mayor's
feast. That symptom alone vexeth many. \authorfootnote{2638}Some again are black,
pale, ruddy, sometimes their shoulders and shoulder blades ache, there
is a leaping all over their bodies, sudden trembling, a palpitation of
the heart, and that cardiaca passio, grief in the mouth of the stomach,
which maketh the patient think his heart itself acheth, and sometimes
suffocation, difficultas anhelitus, short breath, hard wind, strong
pulse, swooning. Montanus consil. 55. Trincavelius lib. 3. consil. 36.
et 37. Fernelius cons. 43. Frambesarius consult. lib. 1. consil. 17.
Hildesheim, Claudinus, \&c., give instance of every particular. The
peculiar symptoms which properly belong to each part be these. If it
proceed from the stomach, saith \authorfootnote{2639}Savanarola, 'tis full of pain
wind. Guianerius adds, vertigo, nausea, much spitting, \&c. If from the
mirach, a swelling and wind in the hypochondries, a loathing, and
appetite to vomit, pulling upward. If from the heart, aching and
trembling of it, much heaviness. If from the liver, there is usually a
pain in the right hypochondry. If from the spleen, hardness and grief
in the left hypochondry, a rumbling, much appetite and small digestion,
Avicenna. If from the mesaraic veins and liver on the other side,
little or no appetite, Herc. de Saxonia. If from the hypochondries, a
rumbling inflation, concoction is hindered, often belching, \&c. And
from these crudities, windy vapours ascend up to the brain which
trouble the imagination, and cause fear, sorrow, dullness, heaviness,
many terrible conceits and chimeras, as Lemnius well observes, l. 1. c.
16. as a black and thick cloud covers the sun\authorfootnote{2640}, and intercepts his
beams and light, so doth this melancholy vapour obnubilate the mind,
enforce it to many absurd thoughts and imaginations, and compel good,
wise, honest, discreet men (arising to the brain from the \authorfootnote{2641} lower
parts, as smoke out of a chimney) to dote, speak, and do that which
becomes them not, their persons, callings, wisdoms. One by reason of
those ascending vapours and gripings, rumbling beneath, will not be
persuaded but that he hath a serpent in his guts, a viper, another
frogs. Trallianus relates a story of a woman, that imagined she had
swallowed an eel, or a serpent, and Felix Platerus, observat. lib. 1.
hath a most memorable example of a countryman of his, that by chance,
falling into a pit where frogs and frogs' spawn was, and a little of
that water swallowed, began to suspect that he had likewise swallowed
frogs' spawn, and with that conceit and fear, his phantasy wrought so
far, that he verily thought he had young live frogs in his belly, qui
vivebant ex alimento suo, that lived by his nourishment, and was so
certainly persuaded of it, that for many years afterwards he could not
be rectified in his conceit: He studied physic seven years together to
cure himself, travelled into Italy, France and Germany to confer with
the best physicians about it, and A.D. 1609, asked his counsel amongst
the rest; he told him it was wind, his conceit, \&c., but mordicus
contradicere, et ore, et scriptis probare nitebatur: no saying would
serve, it was no wind, but real frogs: and do you not hear them croak?
Platerus would have deceived him, by putting live frog's into his
excrements; but he, being a physician himself, would not be deceived,
vir prudens alias, et doctus a wise and learned man otherwise, a doctor
of physic, and after seven years' dotage in this kind, a phantasia
liberatus est, he was cured. Laurentius and Goulart have many such
examples, if you be desirous to read them. One commodity above the rest
which are melancholy, these windy flatuous have, lucidia intervalla,
their symptoms and pains are not usually so continuate as the rest, but
come by fits, fear and sorrow, and the rest: yet in another they exceed
all others; and that is, \authorfootnote{2642}they are luxurious, incontinent, and
prone to venery, by reason of wind, et facile amant, et quamlibet fere
amant. (Jason Pratensis) \authorfootnote{2643}Rhasis is of opinion, that Venus doth
many of them much good; the other symptoms of the mind be common with
the rest.

%SECT. III. MEMB. II. SUBSECT. III.-_Symptoms of Melancholy abounding in the whole body_.
\section[Symptoms In the Whole Body]{Symptoms of Melancholy abounding in the whole body.}

\lettrine{T}{heir} bodies that are affected with this universal melancholy are most
part black, \authorfootnote{2644}the melancholy juice is redundant all over, hirsute
they are, and lean, they have broad veins, their blood is gross and
thick \authorfootnote{2645} Their spleen is weak, and a liver apt to engender the
humour; they have kept bad diet, or have had some evacuation stopped,
as haemorrhoids, or months in women, which \authorfootnote{2646}Trallianus, in the
cure, would have carefully to be inquired, and withal to observe of
what complexion the party is of, black or red. For as Forrestus and
Hollerius contend, if \authorfootnote{2647}they be black, it proceeds from abundance
of natural melancholy; if it proceed from cares, agony, discontents,
diet, exercise, \&c., they may be as well of any other colour: red,
yellow, pale, as black, and yet their whole blood corrupt: praerubri
colore saepe sunt tales, saepe flavi, (saith \authorfootnote{2648} Montaltus cap. 22.)
The best way to discern this species, is to let them bleed, if the
blood be corrupt, thick and black, and they withal free from those
hypochondriacal symptoms, and not so grievously troubled with them, or
those of the head, it argues they are melancholy, a toto corpore. The
fumes which arise from this corrupt blood, disturb the mind, and make
them fearful and sorrowful, heavy hearted, as the rest, dejected,
discontented, solitary, silent, weary of their lives, dull and heavy,
or merry, \&c., and if far gone, that which Apuleius wished to his
enemy, by way of imprecation, is true in them; \authorfootnote{2649}Dead men's bones,
hobgoblins, ghosts are ever in their minds, and meet them still in
every turn: all the bugbears of the night, and terrors, fairy-babes of
tombs, and graves are before their eyes, and in their thoughts, as to
women and children, if they be in the dark alone. If they hear, or
read, or see any tragical object, it sticks by them, they are afraid of
death, and yet weary of their lives, in their discontented humours they
quarrel with all the world, bitterly inveigh, tax satirically, and
because they cannot otherwise vent their passions or redress what is
amiss, as they mean, they will by violent death at last be revenged on
themselves.

%SECT. III. MEMB. II. SUBSECT. IV.-_Symptoms of Maids, Nuns, and Widows' Melancholy_.
\section{Symptoms of Maids, Nuns, and Widows' Melancholy.}

\lettrine{B}{ecause} Lodovicus Mercatus in his second book de mulier. affect. cap.
4. and Rodericus a Castro de morb. mulier. cap. 3. lib. 2. two famous
physicians in Spain, Daniel Sennertus of Wittenberg lib. 1. part 2.
cap. 13. with others, have vouchsafed in their works not long since
published, to write two just treatises de Melancholia virginum,
Monialium et Viduarum, as a particular species of melancholy (which I
have already specified) distinct from the rest; \authorfootnote{2650}(for it much
differs from that which commonly befalls men and other women, as having
one only cause proper to women alone) I may not omit in this general
survey of melancholy symptoms, to set down the particular signs of such
parties so misaffected.

The causes are assigned out of Hippocrates, Cleopatra, Moschion, and
those old Gynaeciorum Scriptores, of this feral malady, in more ancient
maids, widows, and barren women, ob septum transversum violatum, saith
Mercatus, by reason of the midriff or Diaphragma, heart and brain
offended with those vicious vapours which come from menstruous blood,
inflammationem arteriae circa dorsum, Rodericus adds, an inflammation
of the back, which with the rest is offended by \authorfootnote{2651}that fuliginous
exhalation of corrupt seed, troubling the brain, heart and mind; the
brain, I say, not in essence, but by consent, Universa enim hujus
affectus causa ab utero pendet, et a sanguinis menstrui malitia, for in
a word, the whole malady proceeds from that inflammation, putridity,
black smoky vapours, \&c., from thence comes care, sorrow, and anxiety,
obfuscation of spirits, agony, desperation, and the like, which are
intended or remitted; si amatorius accesserit ardor, or any other
violent object or perturbation of mind. This melancholy may happen to
widows, with much care and sorrow, as frequently it doth, by reason of
a sudden alteration of their accustomed course of life, \&c. To such as
lie in childbed ob suppressam purgationem; but to nuns and more ancient
maids, and some barren women for the causes abovesaid, 'tis more
familiar, crebrius his quam reliquis accidit, inquit Rodericus, the
rest are not altogether excluded.

Out of these causes Rodericus defines it with Areteus, to be angorem
animi, a vexation of the mind, a sudden sorrow from a small, light, or
no occasion, \authorfootnote{2652}with a kind of still dotage and grief of some part
or other, head, heart, breasts, sides, back, belly, \&c., with much
solitariness, weeping, distraction, \&c., from which they are sometimes
suddenly delivered, because it comes and goes by fits, and is not so
permanent as other melancholy.

But to leave this brief description, the most ordinary symptoms be
these, pulsatio juxta dorsum, a beating about the back, which is almost
perpetual, the skin is many times rough, squalid, especially, as
Areteus observes, about the arms, knees, and knuckles. The midriff and
heart-strings do burn and beat very fearfully, and when this vapour or
fume is stirred, flieth upward, the heart itself beats, is sore
grieved, and faints, fauces siccitate praecluduntur, ut difficulter
possit ab uteri strangulatione decerni, like fits of the mother, Alvus
plerisque nil reddit, aliis exiguum, acre, biliosum, lotium flavum.

They complain many times, saith Mercatus, of a great pain in their
heads, about their hearts, and hypochondries, and so likewise in their
breasts, which are often sore, sometimes ready to swoon, their faces
are inflamed, and red, they are dry, thirsty, suddenly hot, much
troubled with wind, cannot sleep, \&c. And from hence proceed ferina
deliramenta, a brutish kind of dotage, troublesome sleep, terrible
dreams in the night, subrusticus pudor et verecundia ignava, a foolish
kind of bashfulness to some, perverse conceits and opinions,
\authorfootnote{2653}dejection of mind, much discontent, preposterous judgment. They
are apt to loath, dislike, disdain, to be weary of every object, \&c.,
each thing almost is tedious to them, they pine away, void of counsel,
apt to weep, and tremble, timorous, fearful, sad, and out of all hope
of better fortunes. They take delight in nothing for the time, but love
to be alone and solitary, though that do them more harm: and thus they
are affected so long as this vapour lasteth; but by-and-by, as pleasant
and merry as ever they were in their lives, they sing, discourse, and
laugh in any good company, upon all occasions, and so by fits it takes
them now and then, except the malady be inveterate, and then 'tis more
frequent, vehement, and continuate. Many of them cannot tell how to
express themselves in words, or how it holds them, what ails them, you
cannot understand them, or well tell what to make of their sayings; so
far gone sometimes, so stupefied and distracted, they think themselves
bewitched, they are in despair, aptae ad fletum, desperationem, dolores
mammis et hypocondriis. Mercatus therefore adds, now their breasts, now
their hypochondries, belly and sides, then their heart and head aches,
now heat, then wind, now this, now that offends, they are weary of all;
\authorfootnote{2654}and yet will not, cannot again tell how, where or what offends
them, though they be in great pain, agony, and frequently complain,
grieving, sighing, weeping, and discontented still, sine causa
manifesta, most part, yet I say they will complain, grudge, lament, and
not be persuaded, but that they are troubled with an evil spirit, which
is frequent in Germany, saith Rodericus, amongst the common sort: and
to such as are most grievously affected, (for he makes three degrees of
this disease in women,) they are in despair, surely forespoken or
bewitched, and in extremity of their dotage, (weary of their lives,)
some of them will attempt to make away themselves. Some think they see
visions, confer with spirits and devils, they shall surely be damned,
are afraid of some treachery, imminent danger, and the like, they will
not speak, make answer to any question, but are almost distracted, mad,
or stupid for the time, and by fits: and thus it holds them, as they
are more or less affected, and as the inner humour is intended or
remitted, or by outward objects and perturbations aggravated,
solitariness, idleness, \&c.

Many other maladies there are incident to young women, out of that one
and only cause above specified, many feral diseases. I will not so much
as mention their names, melancholy alone is the subject of my present
discourse, from which I will not swerve. The several cures of this
infirmity, concerning diet, which must be very sparing, phlebotomy,
physic, internal, external remedies, are at large in great variety in
\authorfootnote{2655} Rodericus a Castro, Sennertus, and Mercatus, which whoso will,
as occasion serves, may make use of. But the best and surest remedy of
all, is to see them well placed, and married to good husbands in due
time, hinc illae, lachrymae, that is the primary cause, and this the
ready cure, to give them content to their desires. I write not this to
patronise any wanton, idle flirt, lascivious or light housewives, which
are too forward many times, unruly, and apt to cast away themselves on
him that comes next, without all care, counsel, circumspection, and
judgment. If religion, good discipline, honest education, wholesome
exhortation, fair promises, fame and loss of good name cannot inhibit
and deter such, (which to chaste and sober maids cannot choose but
avail much,) labour and exercise, strict diet, rigour and threats may
more opportunely be used, and are able of themselves to qualify and
divert an ill-disposed temperament. For seldom should you see an hired
servant, a poor handmaid, though ancient, that is kept hard to her
work, and bodily labour, a coarse country wench troubled in this kind,
but noble virgins, nice gentlewomen, such as are solitary and idle,
live at ease, lead a life out of action and employment, that fare well,
in great houses and jovial companies, ill-disposed peradventure of
themselves, and not willing to make any resistance, discontented
otherwise, of weak judgment, able bodies, and subject to passions,
(grandiores virgines, saith Mercatus, steriles et viduae plerumque
melancholicae,) such for the most part are misaffected, and prone to
this disease. I do not so much pity them that may otherwise be eased,
but those alone that out of a strong temperament, innate constitution,
are violently carried away with this torrent of inward humours, and
though very modest of themselves, sober, religious, virtuous, and well
given, (as many so distressed maids are,) yet cannot make resistance,
these grievances will appear, this malady will take place, and now
manifestly show itself, and may not otherwise be helped. But where am
I? Into what subject have I rushed? What have I to do with nuns, maids,
virgins, widows? I am a bachelor myself, and lead a monastic life in a
college, nae ego sane ineptus qui haec dixerim,) I confess 'tis an
indecorum, and as Pallas a virgin blushed, when Jupiter by chance spake
of love matters in her presence, and turned away her face; me reprimam
though my subject necessarily require it, I will say no more.

And yet I must and will say something more, add a word or two in
gratiam virginum et viduarum, in favour of all such distressed parties,
in commiseration of their present estate. And as I cannot choose but
condole their mishap that labour of this infirmity, and are destitute
of help in this case, so must I needs inveigh against them that are in
fault, more than manifest causes, and as bitterly tax those tyrannising
pseudopoliticians, superstitious orders, rash vows, hard-hearted
parents, guardians, unnatural friends, allies, (call them how you
will,) those careless and stupid overseers, that out of worldly
respects, covetousness, supine negligence, their own private ends (cum
sibi sit interim bene) can so severely reject, stubbornly neglect, and
impiously contemn, without all remorse and pity, the tears, sighs,
groans, and grievous miseries of such poor souls committed to their
charge. How odious and abominable are those superstitious and rash vows
of Popish monasteries, so to bind and enforce men and women to vow
virginity, to lead a single life, against the laws of nature, opposite
to religion, policy, and humanity, so to starve, to offer violence, to
suppress the vigour of youth, by rigorous statutes, severe laws, vain
persuasions, to debar them of that to which by their innate temperature
they are so furiously inclined, urgently carried, and sometimes
precipitated, even irresistibly led, to the prejudice of their soul's
health, and good estate of body and mind: and all for base and private
respects, to maintain their gross superstition, to enrich themselves
and their territories as they falsely suppose, by hindering some
marriages, that the world be not full of beggars, and their parishes
pestered with orphans; stupid politicians; haeccine fieri flagilia?
ought these things so to be carried? better marry than burn, saith the
Apostle, but they are otherwise persuaded. They will by all means
quench their neighbour's house if it be on fire, but that fire of lust
which breaks out into such lamentable flames, they will not take notice
of, their own bowels oftentimes, flesh and blood shall so rage and
burn, and they will not see it: miserum est, saith Austin, seipsum non
miserescere, and they are miserable in the meantime that cannot pity
themselves, the common good of all, and per consequens their own
estates. For let them but consider what fearful maladies, feral
diseases, gross inconveniences, come to both sexes by this enforced
temperance, it troubles me to think of, much more to relate those
frequent abortions and murdering of infants in their nunneries (read
\authorfootnote{2656}Kemnitius and others), and notorious fornications, those
Spintrias, Tribadas, Ambubeias, \&c., those rapes, incests, adulteries,
mastuprations, sodomies, buggeries of monks and friars. See Bale's
visitation of abbeys, \authorfootnote{2657}Mercurialis, Rodericus a Castro, Peter
Forestus, and diverse physicians; I know their ordinary apologies and
excuses for these things, sed viderint Politici, Medici, Theologi, I
shall more opportunely meet with them \authorfootnote{2658}elsewhere.
\authorfootnote{2659}Illius viduae, aut patronum Virginis hujus,
Ne me forte putes, verbum non amplius addam.


%MEMB. III.

\section{Immediate cause of these precedent Symptoms.}

\lettrine{T}{o} give some satisfaction to melancholy men that are troubled with
these symptoms, a better means in my judgment cannot be taken, than to
show them the causes whence they proceed; not from devils as they
suppose, or that they are bewitched or forsaken of God, hear or see,
\&c. as many of them think, but from natural and inward causes, that so
knowing them, they may better avoid the effects, or at least endure
them with more patience. The most grievous and common symptoms are fear
and sorrow, and that without a cause to the wisest and discreetest men,
in this malady not to be avoided. The reason why they are so, Aetius
discusseth at large, Tetrabib. 2. 2. in his first problem out of Galen,
lib. 2. de causis sympt. 1. For Galen imputeth all to the cold that is
black, and thinks that the spirits being darkened, and the substance of
the brain cloudy and dark, all the objects thereof appear terrible, and
the \authorfootnote{2660}mind itself, by those dark, obscure, gross fumes, ascending
from black humours, is in continual darkness, fear, and sorrow; divers
terrible monstrous fictions in a thousand shapes and apparitions occur,
with violent passions, by which the brain and fantasy are troubled and
eclipsed. \authorfootnote{2661}Fracastorius, lib. 2. de intellect, will have cold to
be the cause of fear and sorrow; for such as are cold are ill-disposed
to mirth, dull, and heavy, by nature solitary, silent; and not for any
inward darkness (as physicians think) for many melancholy men dare
boldly be, continue, and walk in the dark, and delight in it: solum
frigidi timidi: if they be hot, they are merry; and the more hot, the
more furious, and void of fear, as we see in madmen; but this reason
holds not, for then no melancholy, proceeding from choler adust, should
fear. \authorfootnote{2662}Averroes scoffs at Galen for his reasons, and brings five
arguments to repel them: so doth Herc. de Saxonia, Tract. de Melanch.
cap. 3. assigning other causes, which are copiously censured and
confuted by Aelianus Montaltus, cap. 5 and 6. Lod. Mercatus de Inter.
morb. cur. lib. 1. cap. 17. Altomarus, cap. 7. de mel. Guianerius,
tract. 15. c. 1. Bright cap. 37. Laurentius, cap. 5. Valesius, med.
cont. lib. 5, con. 1. \authorfootnote{2663}Distemperature, they conclude, makes black
juice, blackness obscures the spirits, the spirits obscured, cause fear
and sorrow. Laurentius, cap. 13. supposeth these black fumes offend
specially the diaphragma or midriff, and so per consequens the mind,
which is obscured as the sun by a cloud\authorfootnote{2664}. To this opinion of
Galen, almost all the Greeks and Arabians subscribe, the Latins new and
old, internae, tenebrae offuscant animum, ut externae nocent pueris, as
children are affrighted in the dark, so are melancholy men at all
times, \authorfootnote{2665}as having the inward cause with them, and still carrying
it about. Which black vapours, whether they proceed from the black
blood about the heart, as T. W. Jes. thinks in his treatise of the
passions of the mind, or stomach, spleen, midriff, or all the
misaffected parts together, it boots not, they keep the mind in a
perpetual dungeon, and oppress it with continual fears, anxieties,
sorrows, \&c. It is an ordinary thing for such as are sound to laugh at
this dejected pusillanimity, and those other symptoms of melancholy, to
make themselves merry with them, and to wonder at such, as toys and
trifles, which may be resisted and withstood, if they will themselves:
but let him that so wonders, consider with himself, that if a man
should tell him on a sudden, some of his especial friends were dead,
could he choose but grieve? Or set him upon a steep rock, where he
should be in danger to be precipitated, could he be secure? His heart
would tremble for fear, and his head be giddy. P. Byaras, Tract. de
pest. gives instance (as I have said) \authorfootnote{2666}and put case (saith he) in
one that walks upon a plank, if it lie on the ground, he can safely do
it: but if the same plank be laid over some deep water, instead of a
bridge, he is vehemently moved, and 'tis nothing but his imagination,
forma cadendi impressa, to which his other members and faculties obey.

Yea, but you infer, that such men have a just cause to fear, a true
object of fear; so have melancholy men an inward cause, a perpetual
fume and darkness, causing fear, grief, suspicion, which they carry
with them, an object which cannot be removed; but sticks as close, and
is as inseparable as a shadow to a body, and who can expel or overrun
his shadow? Remove heat of the liver, a cold stomach, weak spleen:
remove those adust humours and vapours arising from them, black blood
from the heart, all outward perturbations, take away the cause, and
then bid them not grieve nor fear, or be heavy, dull, lumpish,
otherwise counsel can do little good; you may as well bid him that is
sick of an ague not to be a dry; or him that is wounded not to feel
pain.

Suspicion follows fear and sorrow at heels, arising out of the same
fountain, so thinks \authorfootnote{2667}Fracastorius, that fear is the cause of
suspicion, and still they suspect some treachery, or some secret
machination to be framed against them, still they distrust.

Restlessness proceeds from the same spring, variety of fumes make them
like and dislike. Solitariness, avoiding of light, that they are weary
of their lives, hate the world, arise from the same causes, for their
spirits and humours are opposite to light, fear makes them avoid
company, and absent themselves, lest they should be misused, hissed at,
or overshoot themselves, which still they suspect. They are prone to
venery by reason of wind. Angry, waspish, and fretting still, out of
abundance of choler, which causeth fearful dreams and violent
perturbations to them, both sleeping and waking: That they suppose they
have no heads, fly, sink, they are pots, glasses, \&c. is wind in their
heads. \authorfootnote{2668}Herc. de Saxonia doth ascribe this to the several motions
in the animal spirits, their dilation, contraction, confusion,
alteration, tenebrosity, hot or cold distemperature, excluding all
material humours. \authorfootnote{2669}Fracastorius accounts it a thing worthy of
inquisition, why they should entertain such false conceits, as that
they have horns, great noses, that they are birds, beasts, \&c., why
they should think themselves kings, lords, cardinals. For the first,
\authorfootnote{2670} Fracastorius gives two reasons: One is the disposition of the
body; the other, the occasion of the fantasy, as if their eyes be
purblind, their ears sing, by reason of some cold and rheum, \&c. To the
second, Laurentius answers, the imagination inwardly or outwardly
moved, represents to the understanding, not enticements only, to favour
the passion or dislike, but a very intensive pleasure follows the
passion or displeasure, and the will and reason are captivated by
delighting in it.

Why students and lovers are so often melancholy and mad, the
philosopher of \authorfootnote{2671}Conimbra assigns this reason, because by a
vehement and continual meditation of that wherewith they are affected,
they fetch up the spirits into the brain, and with the heat brought
with them, they incend it beyond measure: and the cells of the inner
senses dissolve their temperature, which being dissolved, they cannot
perform their offices as they ought.

Why melancholy men are witty, which Aristotle hath long since
maintained in his problems; and that \authorfootnote{2672}all learned men, famous
philosophers, and lawgivers, ad unum fere omnes melancholici, have
still been melancholy, is a problem much controverted. Jason Pratensis
will have it understood of natural melancholy, which opinion Melancthon
inclines to, in his book de Anima, and Marcilius Ficinus de san. tuend.
lib. 1. cap. 5. but not simple, for that makes men stupid, heavy, dull,
being cold and dry, fearful, fools, and solitary, but mixed with the
other humours, phlegm only excepted; and they not adust, \authorfootnote{2673}but so
mixed as that blood he half, with little or no adustion, that they be
neither too hot nor too cold. Aponensis, cited by Melancthon, thinks it
proceeds from melancholy adust, excluding all natural melancholy as too
cold. Laurentius condemns his tenet, because adustion of humours makes
men mad, as lime burns when water is cast on it. It must be mixed with
blood, and somewhat adust, and so that old aphorism of Aristotle may be
verified, Nullum magnum ingenium sine mixtura dementiae, no excellent
wit without a mixture of madness. Fracastorius shall decide the
controversy, \authorfootnote{2674}phlegmatic are dull: sanguine lively, pleasant,
acceptable, and merry, but not witty; choleric are too swift in motion,
and furious, impatient of contemplation, deceitful wits: melancholy men
have the most excellent wits, but not all; this humour may be hot or
cold, thick, or thin; if too hot, they are furious and mad: if too
cold, dull, stupid, timorous, and sad: if temperate, excellent, rather
inclining to that extreme of heat, than cold. This sentence of his will
agree with that of Heraclitus, a dry light makes a wise mind, temperate
heat and dryness are the chief causes of a good wit; therefore, saith
Aelian, an elephant is the wisest of all brute beasts, because his
brain is driest, et ob atrae, bilis capiam: this reason Cardan
approves, subtil. l. 12. Jo. Baptista Silvaticus, a physician of Milan,
in his first controversy, hath copiously handled this question:
Rulandus in his problems, Caelius Rhodiginus, lib. 17. Valleriola 6to.
narrat. med. Herc. de Saxonia, Tract. posth. de mel. cap. 3. Lodovicus
Mercatus, de inter. morb. cur. lib. cap. 17. Baptista Porta, Physiog.
lib. 1. c. 13. and many others.

Weeping, sighing, laughing, itching, trembling, sweating, blushing,
hearing and seeing strange noises, visions, wind, crudity, are motions
of the body, depending upon these precedent motions of the mind:
neither are tears, affections, but actions (as Scaliger holds)
\authorfootnote{2675}the voice of such as are afraid, trembles, because the heart is
shaken (Conimb. prob. 6. sec. 3. de som.) why they stutter or falter in
their speech, Mercurialis and Montaltus, cap. 17. give like reasons out
of Hippocrates, \authorfootnote{2676}dryness, which makes the nerves of the tongue
torpid. Fast speaking (which is a symptom of some few) Aetius will have
caused \authorfootnote{2677} from abundance of wind, and swiftness of imagination:
\authorfootnote{2678}baldness comes from excess of dryness, hirsuteness from a dry
temperature. The cause of much waking in a dry brain, continual
meditation, discontent, fears and cares, that suffer not the mind to be
at rest, incontinency is from wind, and a hot liver, Montanus, cons.
26. Rumbling in the guts is caused from wind, and wind from ill
concoction, weakness of natural heat, or a distempered heat and cold;
\authorfootnote{2679}Palpitation of the heart from vapours, heaviness and aching from
the same cause. That the belly is hard, wind is a cause, and of that
leaping in many parts. Redness of the face, and itching, as if they
were flea-bitten, or stung with pismires, from a sharp subtle wind.
\authorfootnote{2680}Cold sweat from vapours arising from the hypochondries, which
pitch upon the skin; leanness for want of good nourishment. Why their
appetite is so great, \authorfootnote{2681}Aetius answers: Os ventris frigescit, cold
in those inner parts, cold belly, and hot liver, causeth crudity, and
intention proceeds from perturbations, \authorfootnote{2682}our souls for want of
spirits cannot attend exactly to so many intentive operations, being
exhaust, and overswayed by passion, she cannot consider the reasons
which may dissuade her from such affections.

\authorfootnote{2683}Bashfulness and blushing, is a passion proper to men alone, and
is not only caused for \authorfootnote{2684}some shame and ignominy, or that they are
guilty unto themselves of some foul fact committed, but as
\authorfootnote{2685}Fracastorius well determines, ob defectum proprium, et timorem,
from fear, and a conceit of our defects; the face labours and is
troubled at his presence that sees our defects, and nature willing to
help, sends thither heat, heat draws the subtlest blood, and so we
blush. They that are bold, arrogant, and careless, seldom or never
blush, but such as are fearful. Anthonius Lodovicus, in his book de
pudore, will have this subtle blood to arise in the face, not so much
for the reverence of our betters in presence, \authorfootnote{2686}but for joy and
pleasure, or if anything at unawares shall pass from us, a sudden
accident, occurse, or meeting: (which Disarius in \authorfootnote{2687} Macrobius
confirms) any object heard or seen, for blind men never blush, as
Dandinus observes, the night and darkness make men impudent. Or that we
be staid before our betters, or in company we like not, or if anything
molest and offend us, erubescentia turns to rubor, blushing to a
continuate redness. \authorfootnote{2688}Sometimes the extremity of the ears tingle,
and are red, sometimes the whole face, Etsi nihil vitiosum commiseris,
as Lodovicus holds: though Aristotle is of opinion, omnis pudor ex
vitio commisso, all shame for some offence. But we find otherwise, it
may as well proceed \authorfootnote{2689}from fear, from force and inexperience, (so
\authorfootnote{2690}Dandinus holds) as vice; a hot liver, saith Duretus (notis in
Hollerium:) from a hot brain, from wind, the lungs heated, or after
drinking of wine, strong drink, perturbations, \&c.

Laughter what it is, saith \authorfootnote{2691}Tully, how caused, where, and so
suddenly breaks out, that desirous to stay it, we cannot, how it comes
to possess and stir our face, veins, eyes, countenance, mouth, sides,
let Democritus determine. The cause that it often affects melancholy
men so much, is given by Gomesius, lib. 3. de sale genial. cap. 18.
abundance of pleasant vapours, which, in sanguine melancholy
especially, break from the heart, \authorfootnote{2692}and tickle the midriff, because
it is transverse and full of nerves: by which titillation the sense
being moved, and arteries distended, or pulled, the spirits from thence
move and possess the sides, veins, countenance, eyes. See more in
Jossius de risu et fletu, Vives 3 de Anima. Tears, as Scaliger defines,
proceed from grief and pity, \authorfootnote{2693}or from the heating of a moist
brain, for a dry cannot weep.

That they see and hear so many phantasms, chimeras, noises, visions,
\&c. as Fienus hath discoursed at large in his book of imagination, and
\authorfootnote{2694} Lavater de spectris, part. 1. cap. 2. 3. 4. their corrupt
phantasy makes them see and hear that which indeed is neither heard nor
seen, Qui multum jejunant, aut noctes ducunt insomnes, they that much
fast, or want sleep, as melancholy or sick men commonly do, see
visions, or such as are weak-sighted, very timorous by nature, mad,
distracted, or earnestly seek. Sabini quod volunt somniant, as the
saying is, they dream of that they desire. Like Sarmiento the Spaniard,
who when he was sent to discover the straits of Magellan, and confine
places, by the Prorex of Peru, standing on the top of a hill,
Amaenissimam planitiem despicere sibi visus fuit, aedificia magnifica,
quamplurimos Pagos, alias Turres, splendida Templa, and brave cities,
built like ours in Europe, not, saith mine \authorfootnote{2695}author, that there was
any such thing, but that he was vanissimus et nimis credulus, and would
fain have had it so. Or as Lod. Mercatus proves\authorfootnote{2696}, by reason of
inward vapours, and humours from blood, choler, \&c. diversely mixed,
they apprehend and see outwardly, as they suppose, diverse images, which
indeed are not. As they that drink wine think all runs round, when it
is in their own brain; so is it with these men, the fault and cause is
inward, as Galen affirms, \authorfootnote{2697}mad men and such as are near death,
quas extra se videre putant Imagines, intra oculos habent, 'tis in
their brain, which seems to be before them; the brain as a concave
glass reflects solid bodies. Senes etiam decrepiti cerebrum habent
concavum et aridum, ut imaginentur se videre (saith \authorfootnote{2698}Boissardus)
quae non sunt, old men are too frequently mistaken and dote in like
case: or as he that looketh through a piece of red glass, judgeth
everything he sees to be red; corrupt vapours mounting from the body to
the head, and distilling again from thence to the eyes, when they have
mingled themselves with the watery crystal which receiveth the shadows
of things to be seen, make all things appear of the same colour, which
remains in the humour that overspreads our sight, as to melancholy men
all is black, to phlegmatic all white, \&c. Or else as before the organs
corrupt by a corrupt phantasy, as Lemnius, lib. 1. cap. 16. well
quotes, \authorfootnote{2699}cause a great agitation of spirits, and humours, which
wander to and fro in all the creeks of the brain, and cause such
apparitions before their eyes. One thinks he reads something written in
the moon, as Pythagoras is said to have done of old, another smells
brimstone, hears Cerberus bark: Orestes now mad supposed he saw the
furies tormenting him, and his mother still ready to run upon him,
\authorfootnote{2700}O mater obsecro noli me persequi
His furiis, aspectu anguineis, horribilibus,
Ecce ecce me invadunt, in me jam ruunt;

but Electra told him thus raving in his mad fit, he saw no such sights
at all, it was but his crazed imagination.
\authorfootnote{2701}Quiesce, quiesce miser in linteis tuis,
Non cernis etenim quae videre te putas.

So Pentheus (in Bacchis Euripidis) saw two suns, two Thebes, his brain
alone was troubled. Sickness is an ordinary cause of such sights.
Cardan, subtil. 8. Mens aegra laboribus et jejuniis fracta, facit eos
videre, audire, \&c. And, Osiander beheld strange visions, and Alexander
ab Alexandro both, in their sickness, which he relates de rerum
varietat. lib. 8. cap. 44. Albategnius that noble Arabian, on his
death-bed, saw a ship ascending and descending, which Fracastorius
records of his friend Baptista Tirrianus. Weak sight and a vain
persuasion withal, may effect as much, and second causes concurring, as
an oar in water makes a refraction, and seems bigger, bended double,
\&c. The thickness of the air may cause such effects, or any object not
well-discerned in the dark, fear and phantasy will suspect to be a
ghost, a devil, \&c. \authorfootnote{2702}Quod nimis miseri timent, hoc facile credunt,
we are apt to believe, and mistake in such cases. Marcellus Donatus,
lib. 2. cap. 1. brings in a story out of Aristotle, of one Antepharon
which likely saw, wheresoever he was, his own image in the air, as in a
glass. Vitellio, lib. 10. perspect. hath such another instance of a
familiar acquaintance of his, that after the want of three or four
nights sleep, as he was riding by a river side, saw another riding with
him, and using all such gestures as he did, but when more light
appeared, it vanished. Eremites and anchorites have frequently such
absurd visions, revelations by reason of much fasting, and bad diet,
many are deceived by legerdemain, as Scot hath well showed in his book
of the discovery of witchcraft, and Cardan, subtil. 18. suffites,
perfumes, suffumigations, mixed candles, perspective glasses, and such
natural causes, make men look as if they were dead, or with
horse-heads, bull's-horns, and such like brutish shapes, the room full
of snakes, adders, dark, light, green, red, of all colours, as you may
perceive in Baptista Porta, Alexis, Albertus, and others, glow-worms,
fire-drakes, meteors, Ignis fatuus, which Plinius, lib. 2. cap. 37.
calls Castor and Pollux, with many such that appear in moorish grounds,
about churchyards, moist valleys, or where battles have been fought,
the causes of which read in Goclenius, Velouris, Fickius, \&c. such
fears are often done, to frighten children with squibs, rotten wood,
\&c. to make folks look as if they were dead, \authorfootnote{2703}solito majores,
bigger, lesser, fairer, fouler, ut astantes sine capitibus videantur;
aut toti igniti, aut forma daemonum, accipe pilos canis nigri, \&c.
saith Albertus; and so 'tis ordinary to see strange uncouth sights by
catoptrics: who knows not that if in a dark room, the light be admitted
at one only little hole, and a paper or glass put upon it, the sun
shining, will represent on the opposite wall all such objects as are
illuminated by his rays? with concave and cylinder glasses, we may
reflect any shape of men, devils, antics, (as magicians most part do,
to gull a silly spectator in a dark room), we will ourselves, and that
hanging in the air, when 'tis nothing but such an horrible image as
\authorfootnote{2704}Agrippa demonstrates, placed in another room. Roger Bacon of old
is said to have represented his own image walking in the air by this
art, though no such thing appear in his perspectives. But most part it
is in the brain that deceives them, although I may not deny, but that
oftentimes the devil deludes them, takes his opportunity to suggest,
and represent vain objects to melancholy men, and such as are ill
affected. To these you may add the knavish impostures of jugglers,
exorcists, mass-priests, and mountebanks, of whom Roger Bacon speaks,
\&c. de miraculis naturae et artis. cap. 1. \authorfootnote{2705}they can counterfeit
the voices of all birds and brute beasts almost, all tones and tunes of
men, and speak within their throats, as if they spoke afar off, that
they make their auditors believe they hear spirits, and are thence much
astonished and affrighted with it. Besides, those artificial devices to
overhear their confessions, like that whispering place of Gloucester
\authorfootnote{2706}with us, or like the duke's place at Mantua in Italy, where the
sound is reverberated by a concave wall; a reason of which Blancanus in
his Echometria gives, and mathematically demonstrates.

So that the hearing is as frequently deluded as the sight, from the
same causes almost, as he that hears bells, will make them sound what
he list. As the fool thinketh, so the bell clinketh. Theophilus in
Galen thought he heard music, from vapours which made his ears sound,
\&c. Some are deceived by echoes, some by roaring of waters, or concaves
and reverberation of air in the ground, hollow places and walls.

\authorfootnote{2707}At Cadurcum, in Aquitaine, words and sentences are repeated by a
strange echo to the full, or whatsoever you shall play upon a musical
instrument, more distinctly and louder, than they are spoken at first.

Some echoes repeat a thing spoken seven times, as at Olympus, in
Macedonia, as Pliny relates, lib. 36. cap. 15. Some twelve times, as at
Charenton, a village near Paris, in France. At Delphos, in Greece,
heretofore was a miraculous echo, and so in many other places. Cardan,
subtil. l. 18, hath wonderful stories of such as have been deluded by
these echoes. Blancanus the Jesuit, in his Echometria, hath variety of
examples, and gives his reader full satisfaction of all such sounds by
way of demonstration. \authorfootnote{2708}At Barrey, an isle in the Severn mouth,
they seem to hear a smith's forge; so at Lipari, and those sulphureous
isles, and many such like, which Olaus speaks of in the continent of
Scandia, and those northern countries. Cardan de rerum var. l. 15, c.
84, mentioneth a woman, that still supposed she heard the devil call
her, and speaking to her, she was a painter's wife in Milan: and many
such illusions and voices, which proceed most part from a corrupt
imagination.

Whence it comes to pass, that they prophesy, speak several languages, talk of
astronomy, and other unknown sciences to them (of which they have been ever
ignorant): \authorfootnote{2709}I have in brief touched, only this I will here
add, that Arculanus, Bodin. lib. 3, cap. 6, daemon. and some others,
\authorfootnote{2710} hold as a manifest token that such persons are possessed
with the devil; so doth \authorfootnote{2711}Hercules de Saxonia, and
Apponensis, and fit only to be cured by a priest. But Guianerius\authorfootnote{2712}, Montaltus\authorfootnote{2713}, Pomporiatius
of Padua, and Lemnius lib. 2. cap. 2, refer it wholly to the ill-disposition of
the \authorfootnote{2714}humour, and that out of the authority of Aristotle
prob. 30. 1, because such symptoms are cured by purging; and as by the striking
of a flint fire is enforced, so by the vehement motion of spirits, they do
elicere voces inauditas, compel strange speeches to be spoken: another argument
he hath from Plato's \li{reminiscentia}, which all out as likely as that which
Marsilius Ficinus speaks of his friend Pierleonus\authorfootnote{2715}; by a
divine kind of infusion he understood the secrets of nature, and tenets of
Grecian and barbarian philosophers, before ever he heard of, saw, or read their
works: but in this I should rather hold with Avicenna and his associates, that
such symptoms proceed from evil spirits, which take all opportunities of
humours decayed, or otherwise to pervert the soul of man: and besides, the
humour itself is \li{balneum diaboli}, the devil's bath; and as Agrippa proves, doth
entice him to seize upon them.
}
