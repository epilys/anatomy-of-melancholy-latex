\cleartoleftpage{}
\begin{figure}[p]
  \begingroup
  \centering
  \includegraphics[keepaspectratio,width=\textwidth]{v7cfk63b-small.jpg}
  \captionart{DeathLooms}
  \label{fig:deathlooms}
\end{figure}
% Force float here
\clearpage{}
%\part{THE FIRST PARTITION.}
%\chapter{THE FIRST SECTION, MEMBER, SUBSECTION.}
\chapter[Of Diseases and Melancholy]{Of Diseases in General, and of Melancholy; with a Digression of Anatomy}
\section[Man's Excellency, Fall, Miseries]{Man's Excellency, Fall, Miseries, Infirmities; The causes of them.}

\subsection{Man's Excellency.}

\lettrine[lines=4,findent=5pt,nindent=0pt]{M}{an} the most excellent and noble
creature of the world, "the principal and mighty work of God, wonder of
Nature," as Zoroaster calls him; \li{audacis naturae miraculum}, "the
\authorfootnote{820}marvel of marvels," as Plato; "the
\authorfootnote{821}abridgment and epitome of the world," as Pliny;
\li{microcosmus}, a little world, a model of the world,
\authorfootnote{822}sovereign lord of the earth, viceroy of the world, sole
commander and governor of all the creatures in it; to whose empire they are
subject in particular, and yield obedience; far surpassing all the rest, not in
body only, but in soul; \authorfootnote{823}\li{imaginis imago},
\authorfootnote{824}created to God's own \authorfootnote{825}image, to that
immortal and incorporeal substance, with all the faculties and powers belonging
unto it; was at first pure, divine, perfect, happy,
\authorfootnote{826}"created after God in true holiness and righteousness;"
\li{Deo congruens}, free from all manner of infirmities, and put in Paradise,
to know God, to praise and glorify him, to do his will, \li{Ut diis consimiles
parturiat deos} (as an old poet saith) to propagate the church.

\subsection{Man's Fall and Misery.}

But this most noble creature, \li{Heu tristis, et lachrymosa commutatio}
(\authorfootnote{827}one exclaims) O pitiful change! is fallen from that he
was, and forfeited his estate, become \li{miserabilis homuncio}, a castaway, a
caitiff, one of the most miserable creatures of the world, if he be considered
in his own nature, an unregenerate man, and so much obscured by his fall that
(some few relics excepted) he is inferior to a beast, \authorfootnote{828}"Man
in honour that understandeth not, is like unto beasts that perish," so David
esteems him: a monster by stupend metamorphoses, \authorfootnote{829}a fox, a
dog, a hog, what not? \li{Quantum mutatus ab illo}? How much altered from that
he was; before blessed and happy, now miserable and accursed;
\authorfootnote{830}"He must eat his meat in sorrow," subject to death and all
manner of infirmities, all kind of calamities.

\subsection{A Description of Melancholy.}

"Great travail is created for all men, and an heavy yoke on the sons of Adam,
from the day that they go out of their mother's womb, unto that day they return
to the mother of all things. Namely, their thoughts, and fear of their hearts,
and their imagination of things they wait for, and the day of death. From him
that sitteth in the glorious throne, to him that sitteth beneath in the earth
and ashes; from him that is clothed in blue silk and weareth a crown, to him
that is clothed in simple linen. Wrath, envy, trouble, and unquietness, and
fear of death, and rigour, and strife, and such things come to both man and
beast, but sevenfold to the ungodly."\authorfootnote{831} All this befalls him
in this life, and peradventure eternal misery in the life to come.

\subsection[The Impulsive Cause]{Impulsive Cause of Man's Misery and Infirmities.}

The impulsive cause of these miseries in man, this privation or destruction of
God's image, the cause of death and diseases, of all temporal and eternal
punishments, was the sin of our first parent Adam, \authorfootnote{832}in
eating of the forbidden fruit, by the devil's instigation and allurement. His
disobedience, pride, ambition, intemperance, incredulity, curiosity; from
whence proceeded original sin, and that general corruption of mankind, as from
a fountain, flowed all bad inclinations and actual transgressions which cause
our several calamities inflicted upon us for our sins. And this belike is that
which our fabulous poets have shadowed unto us in the tale of
\authorfootnote{833}Pandora's box, which being opened through her curiosity,
filled the world full of all manner of diseases. It is not curiosity alone, but
those other crying sins of ours, which pull these several plagues and miseries
upon our heads. For \li{Ubi peccatum, ibi procella}, as
\authorfootnote{834}\Chrysostom{} well observes. \authorfootnote{835}"Fools by
reason of their transgression, and because of their iniquities, are afflicted."
\authorfootnote{836}"Fear cometh like sudden desolation, and destruction like a
whirlwind, affliction and anguish," because they did not fear God.
\authorfootnote{837}"Are you shaken with wars?" as Cyprian well urgeth to
Demetrius, "are you molested with dearth and famine? is your health crushed
with raging diseases? is mankind generally tormented with epidemical maladies?
'tis all for your sins," \biblecite{Hag. \rn{i.} 9, 10}; \biblecite{Amos \rn{i.}};
\biblecite{Jer. \rn{vii.}} God is angry, punisheth and threateneth, because of
their obstinacy and stubbornness, they will not turn unto him.
\authorfootnote{838}"If the earth be barren then for want of rain, if dry and
squalid, it yield no fruit, if your fountains be dried up, your wine, corn, and
oil blasted, if the air be corrupted, and men troubled with diseases, 'tis by
reason of their sins:" which like the blood of Abel cry loud to heaven for
vengeance, \biblecite{Lam. \rn{v.} 15}. "That we have sinned, therefore our
hearts are heavy," \biblecite{Isa. \rn{lix.} 11, 12}. "We roar like bears, and
mourn like doves, and want health, \etc{} for our sins and trespasses." But
this we cannot endure to hear or to take notice of, \biblecite{Jer. \rn{ii.} 30}.
"We are smitten in vain and receive no correction;" and \biblecite{cap. \rn{v.}
3}. "Thou hast stricken them, but they have not sorrowed; they have refused to
receive correction; they have not returned. Pestilence he hath sent, but they
have not turned to him," \biblecite{Amos \rn{iv.}} \authorfootnote{839}Herod
could not abide John Baptist, nor \authorfootnote{840}Domitian endure
\Apollonius{} to tell the causes of the plague at Ephesus, his injustice, incest,
adultery, and the like.

To punish therefore this blindness and obstinacy of ours as a concomitant cause
and principal agent, is God's just judgment in bringing these calamities upon
us, to chastise us, I say, for our sins, and to satisfy God's wrath. For the
law requires obedience or punishment, as you may read at large, \biblecite{Deut.
\rn{xxviii.} 15}. "If they will not obey the Lord, and keep his commandments
and ordinances, then all these curses shall come upon them."
\authorfootnote{841}"Cursed in the town and in the field, \etc{}"
\authorfootnote{842}"Cursed in the fruit of the body, \etc{}"
\authorfootnote{843}"The Lord shall send thee trouble and shame, because of thy
wickedness." And a little after, \authorfootnote{844}"The Lord shall smite thee
with the botch of Egypt, and with emerods, and scab, and itch, and thou canst
not be healed; \authorfootnote{845}with madness, blindness, and astonishing of
heart." This Paul seconds, \biblecite{Rom. \rn{ii.} 9}. "Tribulation and anguish
on the soul of every man that doeth evil." Or else these chastisements are
inflicted upon us for our humiliation, to exercise and try our patience here in
this life to bring us home, to make us to know God ourselves, to inform and
teach us wisdom. \authorfootnote{846}"Therefore is my people gone into
captivity, because they had no knowledge; therefore is the wrath of the Lord
kindled against his people, and he hath stretched out his hand upon them." He
is desirous of our salvation. \authorfootnote{847}\li{Nostrae salutis avidus},
saith Lemnius, and for that cause pulls us by the ear many times, to put us in
mind of our duties: "That they which erred might have understanding, (as Isaiah
speaks xxix. 24) and so to be reformed." \authorfootnote{848}"I am afflicted,
and at the point of death," so David confesseth of himself, \biblecite{Psal.
\rn{lxxxviii.} \rn{v.} 15, \rn{v.} 9}. "Mine eyes are sorrowful through mine
affliction:" and that made him turn unto God. Great Alexander in the midst of
all his prosperity, by a company of parasites deified, and now made a god, when
he saw one of his wounds bleed, remembered that he was but a man, and remitted
of his pride. \li{In morbo recolligit se animus}, \authorfootnote{849}as
\authorfootnote{850}Pliny well perceived; "In sickness the mind reflects upon
itself, with judgment surveys itself, and abhors its former courses;" insomuch
that he concludes to his friend Marius, \authorfootnote{851}"that it were the
period of all philosophy, if we could so continue sound, or perform but a part
of that which we promised to do, being sick. Whoso is wise then, will consider
these things," as David did (\biblecite{Psal. \rn{cxliv.}, verse last}); and
whatsoever fortune befall him, make use of it. If he be in sorrow, need,
sickness, or any other adversity, seriously to recount with himself, why this
or that malady, misery, this or that incurable disease is inflicted upon him;
it may be for his good, \authorfootnote{852}\li{sic expedit} as Peter said of
his daughter's ague. Bodily sickness is for his soul's health, \li{periisset
nisi periisset}, had he not been visited, he had utterly perished; for
\authorfootnote{853}"the Lord correcteth him whom he loveth, even as a father
doth his child in whom he delighteth." If he be safe and sound on the other
side, and free from all manner of infirmity; \authorfootnote{854}\li{et cui}

\translatedverse{%
\begin{latin}
\begin{verse}%
Gratia, forma, valetudo contingat abunde\\*
Et mundus victus, non deficiente crumena.\\!
\end{verse}%
\end{latin}}{%
\begin{verse}%
And that he have grace, beauty, favour, health,\\*
A cleanly diet, and abound in wealth.\\!
\end{verse}}{}

Yet in the midst of his prosperity, let him remember that caveat of Moses,
\authorfootnote{855}"Beware that he do not forget the Lord his God;" that he be
not puffed up, but acknowledge them to be his good gifts and benefits, and
\authorfootnote{856}"the more he hath, to be more thankful," (as Agapetianus
adviseth) and use them aright.

\cleartoleftpage{}
\newgeometry{noheadfoot=true}
\begin{figure}[p]
  \begingroup
  \centering
  \includegraphics[keepaspectratio,width=0.95\textwidth]{caron-dionysius-converts-pagans-small.jpg}
  \captionart{DionysiusConvertsPagans}
  \label{fig:dionysiusconvertspagans}
\end{figure}
\restoregeometry

% Force float here
\clearpage{}
\subsection{Instrumental Causes of our Infirmities.}

Now the instrumental causes of these our infirmities, are as diverse as the
infirmities themselves; stars, heavens, elements, \etc{} And all those
creatures which God hath made, are armed against sinners. They were indeed once
good in themselves, and that they are now many of them pernicious unto us, is
not in their nature, but our corruption, which hath caused it. For from the
fall of our first parent Adam, they have been changed, the earth accursed, the
influence of stars, altered, the four elements, beasts, birds, plants, are now
ready to offend us. "The principal things for the use of man, are water, fire,
iron, salt, meal, wheat, honey, milk, oil, wine, clothing, good to the godly,
to the sinners turned to evil," \biblecite{Ecclus. \rn{xxxix.} 26}. "Fire, and
hail, and famine, and dearth, all these are created for vengeance,"
\biblecite{Ecclus. \rn{xxxix.} 29}. The heavens threaten us with their comets,
stars, planets, with their great conjunctions, eclipses, oppositions,
quartiles, and such unfriendly aspects. The air with his meteors, thunder and
lightning, intemperate heat and cold, mighty winds, tempests, unseasonable
weather; from which proceed dearth, famine, plague, and all sorts of epidemical
diseases, consuming infinite myriads of men. At Cairo in Egypt, every third
year, (as it is related by \authorfootnote{857}Boterus, and others)
300\thinspace{}000 die of the plague; and 200\thinspace{}000, in
Constantinople, every fifth or seventh at the utmost. How doth the earth
terrify and oppress us with terrible earthquakes, which are most frequent in
\authorfootnote{858}China, Japan, and those eastern climes, swallowing up
sometimes six cities at once? How doth the water rage with his inundations,
irruptions, flinging down towns, cities, villages, bridges, \etc{} besides
shipwrecks; whole islands are sometimes suddenly overwhelmed with all their
inhabitants in \authorfootnote{859}Zealand, Holland, and many parts of the
continent drowned, as the \authorfootnote{860}lake Erne in Ireland?
\li{Nihilque praeter arcium cadavera patenti cernimus
freto}\authorlatintrans{861.5}.\authorfootnote{861} In the fens of Friesland
1230, by reason of tempests, \authorfootnote{862}the sea drowned \li{multa
hominum millia, et jumenta sine numero}, all the country almost, men and cattle
in it. How doth the fire rage, that merciless element, consuming in an instant
whole cities? What town of any antiquity or note hath not been once, again and
again, by the fury of this merciless element, defaced, ruinated, and left
desolate? In a word,

\translatedverse{%
\begin{latin}
\begin{verse}
Ignis pepercit, unda mergit, aeris\\*
Vis pestilentis aequori ereptum necat,\\*
Bello superstes, tabidus morbo perit.\\!
\end{verse}
\end{latin}}{%
\begin{verse}
Whom fire spares, sea doth drown; whom sea,\\*
Pestilent air doth send to clay;\\*
Whom war 'scapes, sickness takes away.\\!
\end{verse}}{%
\attrib{\getauthornote{863}}}


To descend to more particulars, how many creatures are at deadly feud with men?
Lions, wolves, bears, \etc{} Some with hoofs, horns, tusks, teeth, nails: How
many noxious serpents and venomous creatures, ready to offend us with stings,
breath, sight, or quite kill us? How many pernicious fishes, plants, gums,
fruits, seeds, flowers, \etc{} could I reckon up on a sudden, which by their
very smell many of them, touch, taste, cause some grievous malady, if not death
itself? Some make mention of a thousand several poisons: but these are but
trifles in respect. The greatest enemy to man, is man, who by the devil's
instigation is still ready to do mischief, his own executioner, a wolf, a devil
to himself, and others. \authorfootnote{864}We are all brethren in Christ, or
at least should be, members of one body, servants of one lord, and yet no fiend
can so torment, insult over, tyrannise, vex, as one man doth another. Let me
not fall therefore (saith David, when wars, plague, famine were offered) into
the hands of men, merciless and wicked men:

\begin{latin}
\begin{verse}%
------Vix sunt homines hoc nomine digni,\\*
Quamque lupi, saevae plus feritatis habent.\\!
\end{verse}%
\end{latin}
\attrib{\getauthornote{865}}

We can most part foresee these epidemical diseases, and likely avoid them;
Dearths, tempests, plagues, our astrologers foretell us; Earthquakes,
inundations, ruins of houses, consuming fires, come by little and little, or
make some noise beforehand; but the knaveries, impostures, injuries and
villainies of men no art can avoid. We can keep our professed enemies from our
cities, by gates, walls and towers, defend ourselves from thieves and robbers
by watchfulness and weapons; but this malice of men, and their pernicious
endeavours, no caution can divert, no vigilancy foresee, we have so many secret
plots and devices to mischief one another.

Sometimes by the devil's help as magicians, \authorfootnote{866}witches:
sometimes by impostures, mixtures, poisons, stratagems, single combats, wars,
we hack and hew, as if we were \li{ad internecionem nati}, like Cadmus'
soldiers born to consume one another. 'Tis an ordinary thing to read of a
hundred and two hundred thousand men slain in a battle. Besides all manner of
tortures, brazen bulls, racks, wheels, strappadoes, guns, engines, \etc{}
\authorfootnote{867}\li{Ad unum corpus humanum supplicia plura, quam membra}:
We have invented more torturing instruments, than there be several members in a
man's body, as Cyprian well observes. To come nearer yet, our own parents by
their offences, indiscretion and intemperance, are our mortal enemies.
\authorfootnote{868}"The fathers have eaten sour grapes, and the children's
teeth are set on edge." They cause our grief many times, and put upon us
hereditary diseases, inevitable infirmities: they torment us, and we are ready
to injure our posterity;

\translatedverse{%
\begin{latin}
\begin{verse}
------mox daturi progeniem vitiosiorem.\\!
\end{verse}
\end{latin}}{%
\begin{verse}%
And yet with crimes to us unknown,\\*
Our sons shall mark the coming age their own;\\!
\end{verse}}{%
\attrib{\getauthornote{869}}}

and the latter end of the world, as \authorfootnote{870}Paul foretold, is still
like to be the worst. We are thus bad by nature, bad by kind, but far worse by
art, every man the greatest enemy unto himself. We study many times to undo
ourselves, abusing those good gifts which God hath bestowed upon us, health,
wealth, strength, wit, learning, art, memory to our own destruction,
\li{Perditio tua ex te}\authorlatintrans{871.5}.\authorfootnote{871} As
\authorfootnote{872}Judas Maccabeus killed \Apollonius{} with his own weapons, we
arm ourselves to our own overthrows; and use reason, art, judgment, all that
should help us, as so many instruments to undo us. Hector gave Ajax a sword,
which so long as he fought against enemies, served for his help and defence;
but after he began to hurt harmless creatures with it, turned to his own
hurtless bowels. Those excellent means God hath bestowed on us, well employed,
cannot but much avail us; but if otherwise perverted, they ruin and confound
us: and so by reason of our indiscretion and weakness they commonly do, we have
too many instances. This St. \Austin{} acknowledgeth of himself in his humble
confessions, "promptness of wit, memory, eloquence, they were God's good gifts,
but he did not use them to his glory." If you will particularly know how, and
by what means, consult physicians, and they will tell you, that it is in
offending in some of those six non-natural things, of which I shall
\authorfootnote{873}dilate more at large; they are the causes of our
infirmities, our surfeiting, and drunkenness, our immoderate insatiable lust,
and prodigious riot. \li{Plures crapula, quam gladius}, is a true saying, the
board consumes more than the sword. Our intemperance it is, that pulls so many
several incurable diseases upon our heads, that hastens \authorfootnote{874}old
age, perverts our temperature, and brings upon us sudden death. And last of
all, that which crucifies us most, is our own folly, madness (\li{quos Jupiter
perdit, dementat}; by subtraction of his assisting grace God permits it)
weakness, want of government, our facility and proneness in yielding to several
lusts, in giving way to every passion and perturbation of the mind: by which
means we metamorphose ourselves and degenerate into beasts. All which that
prince of \authorfootnote{875}poets observed of Agamemnon, that when he was
well pleased, and could moderate his passion, he was-- \li{os oculosque Jovi
par}: like Jupiter in feature, Mars in valour, Pallas in wisdom, another god;
but when he became angry, he was a lion, a tiger, a dog, \etc{}, there appeared
no sign or likeness of Jupiter in him; so we, as long as we are ruled by
reason, correct our inordinate appetite, and conform ourselves to God's word,
are as so many saints: but if we give reins to lust, anger, ambition, pride,
and follow our own ways, we degenerate into beasts, transform ourselves,
overthrow our constitutions, \authorfootnote{876}provoke God to anger, and heap
upon us this of melancholy, and all kinds of incurable diseases, as a just and
deserved punishment of our sins.

% SUBSECT. II.
\section{The Definition, Number, Division of Diseases.}

What a disease is, almost every physician defines.
\authorfootnote{877}Fernelius calleth it an "affection of the body contrary to
nature." \authorfootnote{878}Fuschius and Crato, "an hindrance, hurt, or
alteration of any action of the body, or part of it."
\authorfootnote{879}Tholosanus, "a dissolution of that league which is between
body and soul, and a perturbation of it; as health the perfection, and makes to
the preservation of it." \authorfootnote{880}Labeo in Agellius, "an ill habit
of the body, opposite to nature, hindering the use of it." Others otherwise,
all to this effect.

\subsection{Number of Diseases.}

How many diseases there are, is a question not yet determined;
\authorfootnote{881}Pliny reckons up 300 from the crown of the head to the sole
of the foot: elsewhere he saith, \lit{morborum infinita multitudo}{their number
is infinite}. Howsoever it was in those times, it boots not; in our days I am
sure the number is much augmented:

\translatedverse{%
\begin{latin}
\begin{verse}%
------macies, et nova febrium\\*
Terris incubit cohors.\\!
\end{verse}%
\end{latin}}{%\authorlatintrans{882.5}%
\begin{verse}%
Emaciation, and a new cohort of fevers\\*
broods over the earth.\\!
\end{verse}}{%
\attrib{\getauthornote{882}}}

For besides many epidemical diseases unheard of, and altogether unknown to
Galen and Hippocrates, as scorbutum, small-pox, plica, sweating sickness,
morbus Gallicus, \etc{}, we have many proper and peculiar almost to every part.

\subsection{No man free from some Disease or other.}

No man amongst us so sound, of so good a constitution, that hath not some
impediment of body or mind. \li{Quisque suos patimur manes}, we have all our
infirmities, first or last, more or less. There will be peradventure in an age,
or one of a thousand, like Zenophilus the musician in
\authorfootnote{883}Pliny, that may happily live 105 years without any manner
of impediment; a Pollio Romulus, that can preserve himself
\authorfootnote{884}"with wine and oil;" a man as fortunate as Q. Metellus, of
whom Valerius so much brags; a man as healthy as Otto Herwardus, a senator of
Augsburg in Germany, whom \authorfootnote{885}Leovitius the astrologer brings
in for an example and instance of certainty in his art; who because he had the
significators in his geniture fortunate, and free from the hostile aspects of
Saturn and Mars, being a very cold man, \authorfootnote{886}"could not remember
that ever he was sick." \authorfootnote{887}Paracelsus may brag that he could
make a man live 400 years or more, if he might bring him up from his infancy,
and diet him as he list; and some physicians hold, that there is no certain
period of man's life; but it may still by temperance and physic be prolonged.
We find in the meantime, by common experience, that no man can escape, but that
of \authorfootnote{888}\Hesiod{} is true:

\translatedverse{%
\begin{greek}
\begin{verse}%
Πλείη μὲν γὰρ γαῖα κακῶν, πλειη δὲ θάλασσα,\\*
Νοῦσοιδ' ἄνθρωποι ἐιν ἐφ' ἡμέρη, ἠδ' ἐπὶ νυκτὶ\\*
Ἁυτοματοι φοιτῶσι.------\\!
\end{verse}%
\end{greek}}{%
\begin{verse}%
Th' earth's full of maladies, and full the sea,\\*
Which set upon us both by night and day.\\!
\end{verse}}{}

\subsection{Division of Diseases.}

If you require a more exact division of these ordinary diseases which are
incident to men, I refer you to physicians; \authorfootnote{889}they will tell
you of acute and chronic, first and secondary, lethals, salutares, errant,
fixed, simple, compound, connexed, or consequent, belonging to parts or the
whole, in habit, or in disposition, \etc{} My division at this time (as most
befitting my purpose) shall be into those of the body and mind. For them of the
body, a brief catalogue of which Fuschius hath made,
\bookcite{\textlatin{Institut. lib. 3, sect. 1, cap. 11.}} I refer you to the
voluminous tomes of Galen, Areteus, Rhasis, \Avicenna{}, Alexander, Paulus Aetius,
Gordonerius: and those exact Neoterics, Savanarola, Capivaccius, Donatus
Altomarus, Hercules de Saxonia, Mercurialis, Victorius Faventinus, Wecker,
Piso, \etc{}, that have methodically and elaborately written of them all. Those
of the mind and head I will briefly handle, and apart.

%SUBSECT. III.
\section{Division of the Diseases of the Head.}


\lettrine{T}{hese} diseases of the mind, forasmuch as they have their chief seat and organs
in the head, which are commonly repeated amongst the diseases of the head which
are divers, and vary much according to their site. For in the head, as there be
several parts, so there be divers grievances, which according to that division
of \authorfootnote{890}Heurnius, (which he takes out of Arculanus,) are inward
or outward (to omit all others which pertain to eyes and ears, nostrils, gums,
teeth, mouth, palate, tongue, weezle, chops, face, \etc{}) belonging properly
to the brain, as baldness, falling of hair, furfur, lice, \etc{}
\authorfootnote{891}Inward belonging to the skins next to the brain, called
\emph{dura} and \emph{pia mater}, as all headaches, \etc{}, or to the
ventricles, caules, kells, tunicles, creeks, and parts of it, and their
passions, as caro, vertigo, incubus, apoplexy, falling sickness.

The diseases of the nerves, cramps, stupor, convulsion, tremor, palsy: or
belonging to the excrements of the brain, catarrhs, sneezing, rheums,
distillations: or else those that pertain to the substance of the brain itself,
in which are conceived frenzy, lethargy, melancholy, madness, weak memory,
sopor, or \li{Coma Vigilia et vigil Coma}. Out of these again I will single
such as properly belong to the phantasy, or imagination, or reason itself,
which \authorfootnote{892}Laurentius calls the disease of the mind; and
Hildesheim, \lit{morbos imaginationis, aut rationis laesae}{diseases of the
imagination, or of injured reason}, which are three or four in number, frenzy,
madness, melancholy, dotage, and their kinds: as hydrophobia, lycanthropia,
\lit{Chorus sancti viti, morbi daemoniaci}{St. Vitus's dance, possession of
devils} which I will briefly touch and point at, insisting especially in this
of melancholy, as more eminent than the rest, and that through all his kinds,
causes, symptoms, prognostics, cures: as Lonicerus hath done
\bookcite{\textlatin{de apoplexia}}, and many other of such particular
diseases. Not that I find fault with those which have written of this subject
before, as Jason Pratensis, Laurentius, Montaltus, T. Bright, \etc{}, they have
done very well in their several kinds and methods; yet that which one omits,
another may haply see; that which one contracts, another may enlarge. To
conclude with \authorfootnote{893}Scribanius, "that which they had neglected,
or perfunctorily handled, we may more thoroughly examine; that which is
obscurely delivered in them, may be perspicuously dilated and amplified by us:"
and so made more familiar and easy for every man's capacity, and the common
good, which is the chief end of my discourse.

%SUBSECT. IV.
\section[Madness]{Dotage, Frenzy, Madness, Hydrophobia, Lycanthropia, Chorus sancti Viti, Extasis.}
\begin{figure}[H]
  \begingroup
  \centering
  \includegraphics[keepaspectratio,width=\textwidth]{tmczmfhz-small.jpg}
  \captionart{Madness}
  \label{fig:madness}
\end{figure}

\subsection{Delirium, Dotage.}

\lettrine{D}{otage}, fatuity, or folly, is a common name to all the following
species, as some will have it. \authorfootnote{894}Laurentius and
\authorfootnote{895}Altomarus comprehended madness, melancholy, and the rest
under this name, and call it the \li{summum genus} of them all. If it be
distinguished from them, it is natural or ingenite, which comes by some defect
of the organs, and overmuch brain, as we see in our common fools; and is for
the most part intended or remitted in particular men, and thereupon some are
wiser than others: or else it is acquisite, an appendix or symptom of some
other disease, which comes or goes; or if it continue, a sign of melancholy
itself.

\subsection{Frenzy.}

\emph{Phrenitis}, which the Greeks derive from the word \textgreek{φρην}, is a
disease of the mind, with a continual madness or dotage, which hath an acute
fever annexed, or else an inflammation of the brain, or the membranes or kells
of it, with an acute fever, which causeth madness and dotage. It differs from
melancholy and madness, because their dotage is without an ague: this
continual, with waking, or memory decayed, \etc{} Melancholy is most part
silent, this clamorous; and many such like differences are assigned by
physicians.

\subsection{Madness.}

Madness, frenzy, and melancholy are confounded by Celsus, and many writers;
others leave out frenzy, and make madness and melancholy but one disease, which
\authorfootnote{896}Jason Pratensis especially labours, and that they differ
only \li{secundam majus} or \li{minus}, in quantity alone, the one being a
degree to the other, and both proceeding from one cause. They differ
\li{intenso et remisso gradu}, saith \authorfootnote{897}Gordonius, as the
humour is intended or remitted. Of the same mind is
\authorfootnote{898}Areteus, Alexander Tertullianus, Guianerius, Savanarola,
Heurnius; and Galen himself writes promiscuously of them both by reason of
their affinity: but most of our neoterics do handle them apart, whom I will
follow in this treatise. Madness is therefore defined to be a vehement dotage;
or raving without a fever, far more violent than melancholy, full of anger and
clamour, horrible looks, actions, gestures, troubling the patients with far
greater vehemency both of body and mind, without all fear and sorrow, with such
impetuous force and boldness, that sometimes three or four men cannot hold
them. Differing only in this from frenzy, that it is without a fever, and their
memory is most part better. It hath the same causes as the other, as choler
adust, and blood incensed, brains inflamed, \etc{}
\authorfootnote{899}Fracastorius adds, "a due time, and full age" to this
definition, to distinguish it from children, and will have it confirmed
impotency, to separate it from such as accidentally come and go again, as by
taking henbane, nightshade, wine, \etc{} Of this fury there be divers kinds;
\authorfootnote{900}ecstasy, which is familiar with some persons, as Cardan
saith of himself, he could be in one when he list; in which the Indian priests
deliver their oracles, and the witches in Lapland, as Olaus Magnus writeth,
\bookcite{\textlatin{l. 3, cap. 18.}} \li{Extasi omnia praedicere}, answer all
questions in an ecstasis you will ask; what your friends do, where they are,
how they fare, \etc{} The other species of this fury are enthusiasms,
revelations, and visions, so often mentioned by Gregory and Bede in their
works; obsession or possession of devils, sibylline prophets, and poetical
furies; such as come by eating noxious herbs, tarantulas stinging, \etc{},
which some reduce to this. The most known are these, lycanthropia, hydrophobia,
chorus sancti Viti.

\cleartoleftpage{}
\begin{figure}[p]
  \begingroup
  \centering
  \includegraphics[keepaspectratio,width=\textwidth]{Werewolf-with-bodies-small.jpg}
  \captionart{Werewolf}
  \label{fig:werewolf}
\end{figure}

% Force float here
\clearpage{}

\subsection{Lycanthropia.}

Lycanthropia, which \Avicenna{} calls \li{cucubuth}, others \li{lupinam insaniam},
or wolf-madness, when men run howling about graves and fields in the night, and
will not be persuaded but that they are wolves, or some such beasts.
\authorfootnote{901}Aetius and \authorfootnote{902}Paulus call it a kind of
melancholy; but I should rather refer it to madness, as most do.

Some make a doubt of it whether there be any such disease.
\authorfootnote{903}Donat ab Altomari saith, that he saw two of them in his
time: \authorfootnote{904}Wierus tells a story of such a one at Padua 1541,
that would not believe to the contrary, but that he was a wolf. He hath another
instance of a Spaniard, who thought himself a bear;
\authorfootnote{905}Forrestus confirms as much by many examples; one amongst
the rest of which he was an eyewitness, at Alcmaer in Holland, a poor
husbandman that still hunted about graves, and kept in churchyards, of a pale,
black, ugly, and fearful look. Such belike, or little better, were king
Praetus' \authorfootnote{906}daughters, that thought themselves kine. And
Nebuchadnezzar in Daniel, as some interpreters hold, was only troubled with
this kind of madness. This disease perhaps gave occasion to that bold assertion
of \authorfootnote{907}Pliny, "some men were turned into wolves in his time,
and from wolves to men again:" and to that fable of Pausanias, of a man that
was ten years a wolf, and afterwards turned to his former shape: to
\authorfootnote{908}\Ovid{}'s tale of Lycaon, \etc{} He that is desirous to hear
of this disease, or more examples, let him read Austin in his 18th book
\bookcite{\textlatin{de Civitate Dei, cap. 5.}} Mizaldus,
\bookcite{\textlatin{cent. 5. 77.}} Sckenkius, \bookcite{\textlatin{lib. 1.}}
Hildesheim, \bookcite{\textlatin{spicel. 2. de Mania}}. Forrestus
\bookcite{\textlatin{lib. 10. de morbis cerebri.}} Olaus Magnus, Vincentius
Bellavicensis, \bookcite{\textlatin{spec. met. lib. 31. c. 122.}} Pierius,
Bodine, Zuinger, Zeilger, Peucer, Wierus, Spranger, \etc{} This malady, saith
\Avicenna{}, troubleth men most in February, and is nowadays frequent in Bohemia
and Hungary, according to \authorfootnote{909}Heurnius. Scheretzius will have
it common in Livonia. They lie hid most part all day, and go abroad in the
night, barking, howling, at graves and deserts; \authorfootnote{910}"they have
usually hollow eyes, scabbed legs and thighs, very dry and pale,"
\authorfootnote{911}saith Altomarus; he gives a reason there of all the
symptoms, and sets down a brief cure of them.

\emph{Hydrophobia} is a kind of madness, well known in every village, which
comes by the biting of a mad dog, or scratching, saith
\authorfootnote{912}Aurelianus; touching, or smelling alone sometimes as
\authorfootnote{913}Sckenkius proves, and is incident to many other creatures
as well as men: so called because the parties affected cannot endure the sight
of water, or any liquor, supposing still they see a mad dog in it. And which is
more wonderful; though they be very dry, (as in this malady they are) they will
rather die than drink: \authorfootnote{914}de Venenis Caelius Aurelianus, an
ancient writer, makes a doubt whether this Hydrophobia be a passion of the body
or the mind. The part affected is the brain: the cause, poison that comes from
the mad dog, which is so hot and dry, that it consumes all the moisture in the
body. \authorfootnote{915}Hildesheim relates of some that died so mad; and
being cut up, had no water, scarce blood, or any moisture left in them. To such
as are so affected, the fear of water begins at fourteen days after they are
bitten, to some again not till forty or sixty days after: commonly saith
Heurnius, they begin to rave, fly water and glasses, to look red, and swell in
the face, about twenty days after (if some remedy be not taken in the meantime)
to lie awake, to be pensive, sad, to see strange visions, to bark and howl, to
fall into a swoon, and oftentimes fits of the falling sickness.
\authorfootnote{916}Some say, little things like whelps will be seen in their
urine. If any of these signs appear, they are past recovery. Many times these
symptoms will not appear till six or seven months after, saith
\authorfootnote{917}Codronchus; and sometimes not till seven or eight years, as
Guianerius; twelve as Albertus; six or eight months after, as Galen holds.
Baldus the great lawyer died of it: an Augustine friar, and a woman in Delft,
that were \authorfootnote{918}Forrestus' patients, were miserably consumed with
it. The common cure in the country (for such at least as dwell near the
seaside) is to duck them over head and ears in sea water; some use charms:
every good wife can prescribe medicines. But the best cure to be had in such
cases, is from the most approved physicians; they that will read of them, may
consult with Dioscorides, \bookcite{\textlatin{lib. 6. c. 37}}, Heurnius,
Hildesheim, Capivaccius, Forrestus, Sckenkius and before all others Codronchus
an Italian, who hath lately written two exquisite books on the subject.

\li{Chorus sancti Viti}, or St. Vitus's dance; the lascivious dance,
\authorfootnote{919}Paracelsus calls it, because they that are taken from it,
can do nothing but dance till they be dead, or cured. It is so called, for that
the parties so troubled were wont to go to St. Vitus for help, and after they
had danced there awhile, they were \authorfootnote{920}certainly freed. 'Tis
strange to hear how long they will dance, and in what manner, over stools,
forms, tables; even great bellied women sometimes (and yet never hurt their
children) will dance so long that they can stir neither hand nor foot, but seem
to be quite dead. One in red clothes they cannot abide. Music above all things
they love, and therefore magistrates in Germany will hire musicians to play to
them, and some lusty sturdy companions to dance with them. This disease hath
been very common in Germany, as appears by those relations of
\authorfootnote{921}Sckenkius, and Paracelsus in his book of Madness, who brags
how many several persons he hath cured of it. Felix Plateras
\bookcite{\textlatin{de mentis alienat. cap. 3}}, reports of a woman in Basil
whom he saw, that danced a whole month together. The Arabians call it a kind of
palsy. Bodine in his 5th book \bookcite{\textlatin{de Repub. cap. 1}}, speaks
of this infirmity; Monavius in his last epistle to Scoltizius, and in another
to Dudithus, where you may read more of it.

The last kind of madness or melancholy, is that demoniacal (if I may so call
it) obsession or possession of devils, which Platerus and others would have to
be preternatural: stupend things are said of them, their actions, gestures,
contortions, fasting, prophesying, speaking languages they were never taught,
\etc{} Many strange stories are related of them, which because some will not
allow, (for Deacon and Darrel have written large volumes on this subject pro
and con.) I voluntarily omit.

\authorfootnote{922}Fuschius, \bookcite{\textlatin{Institut. lib. 3. sec. 1.
cap. 11}}, Felix Plater, \authorfootnote{923}Laurentius, add to these another
fury that proceeds from love, and another from study, another divine or
religious fury; but these more properly belong to melancholy; of all which I
will speak \authorfootnote{924}apart, intending to write a whole book of them.

\cleartoleftpage{}
\begin{figure}[p]
  \begingroup
  \centering
  \includegraphics[keepaspectratio,width=\textwidth]{MelancholicTemperament-small.jpg}
  \captionart{MelancholicTemperament}
  \label{fig:melancholictemperament}
\end{figure}

% Force float here
\clearpage{}
\thispagestyle{titleontop}

%SUBSECT. V.
\section[Melancholic Disposition]{Melancholy in Disposition, improperly so called, Equivocations.}

Melancholy, the subject of our present discourse, is either in disposition or
habit. In disposition, is that transitory melancholy which goes and comes upon
every small occasion of sorrow, need, sickness, trouble, fear, grief, passion,
or perturbation of the mind, any manner of care, discontent, or thought, which
causeth anguish, dullness, heaviness and vexation of spirit, any ways opposite
to pleasure, mirth, joy, delight, causing frowardness in us, or a dislike. In
which equivocal and improper sense, we call him melancholy that is dull, sad,
sour, lumpish, ill disposed, solitary, any way moved, or displeased. And from
these melancholy dispositions, \authorfootnote{925}no man living is free, no
stoic, none so wise, none so happy, none so patient, so generous, so godly, so
divine, that can vindicate himself; so well composed, but more or less, some
time or other he feels the smart of it. Melancholy in this sense is the
character of mortality. \authorfootnote{926}"Man that is born of a woman, is of
short continuance, and full of trouble." Zeno, Cato, Socrates himself, whom
\authorfootnote{927}Aelian so highly commends for a moderate temper, that
"nothing could disturb him, but going out, and coming in, still Socrates kept
the same serenity of countenance, what misery soever befell him," (if we may
believe Plato his disciple) was much tormented with it. Q. Metellus, in whom
\authorfootnote{928}Valerius gives instance of all happiness, "the most
fortunate man then living, born in that most flourishing city of Rome, of noble
parentage, a proper man of person, well qualified, healthful, rich, honourable,
a senator, a consul, happy in his wife, happy in his children," \etc{} yet this
man was not void of melancholy, he had his share of sorrow.
\authorfootnote{929}Polycrates Samius, that flung his ring into the sea,
because he would participate of discontent with others, and had it miraculously
restored to him again shortly after, by a fish taken as he angled, was not free
from melancholy dispositions. No man can cure himself; the very gods had bitter
pangs, and frequent passions, as their own \authorfootnote{930}poets put upon
them. In general, \authorfootnote{931}"as the heaven, so is our life, sometimes
fair, sometimes overcast, tempestuous, and serene; as in a rose, flowers and
prickles; in the year itself, a temperate summer sometimes, a hard winter, a
drought, and then again pleasant showers: so is our life intermixed with joys,
hopes, fears, sorrows, calumnies: \lit{Invicem cedunt dolor et voluptas}{there
is a succession of pleasure and pain},".

\begin{latin}
\begin{verse}%
------medio de fonte leporum\\*
Surgit amari aliquid, in ipsis floribus angat.\\!
\end{verse}%
\end{latin}
\attrib{\getauthornote{932}}

"Even in the midst of laughing there is sorrow," (as
\authorfootnote{933}Solomon holds): even in the midst of all our feasting and
jollity, as \authorfootnote{934}\Austin infers in his \bookcite{\textlatin{Com.
on the 41st Psalm}}, there is grief and discontent. \li{Inter delicias semper
aliquid saevi nos strangulat}, for a pint of honey thou shalt here likely find
a gallon of gall, for a dram of pleasure a pound of pain, for an inch of mirth
an ell of moan; as ivy doth an oak, these miseries encompass our life. And it
is most absurd and ridiculous for any mortal man to look for a perpetual tenure
of happiness in his life. Nothing so prosperous and pleasant, but it hath
\authorfootnote{935}some bitterness in it, some complaining, some grudging; it
is all \textgreek{γλυκύπικρον}\inlinetrans{bittersweet}, a mixed passion, and
like a chequer table black and white: men, families, cities, have their falls
and wanes; now trines, sextiles, then quartiles and oppositions. We are not
here as those angels, celestial powers and bodies, sun and moon, to finish our
course without all offence, with such constancy, to continue for so many ages:
but subject to infirmities, miseries, interrupted, tossed and tumbled up and
down, carried about with every small blast, often molested and disquieted upon
each slender occasion, \authorfootnote{936}uncertain, brittle, and so is all
that we trust unto. \authorfootnote{937}"And he that knows not this is not
armed to endure it, is not fit to live in this world (as one condoles our
time), he knows not the condition of it, where with a reciprocalty, pleasure
and pain are still united, and succeed one another in a ring." \li{Exi e
mundo}, get thee gone hence if thou canst not brook it; there is no way to
avoid it, but to arm thyself with patience, with magnanimity, to
\authorfootnote{938}oppose thyself unto it, to suffer affliction as a good
soldier of Christ; as \authorfootnote{939}Paul adviseth constantly to bear it.
But forasmuch as so few can embrace this good council of his, or use it aright,
but rather as so many brute beasts give away to their passion, voluntary
subject and precipitate themselves into a labyrinth of cares, woes, miseries,
and suffer their souls to be overcome by them, cannot arm themselves with that
patience as they ought to do, it falleth out oftentimes that these dispositions
become habits, and "many affects contemned" (as \authorfootnote{940}\Seneca{}
notes) "make a disease. Even as one distillation, not yet grown to custom,
makes a cough; but continual and inveterate causeth a consumption of the
lungs;" so do these our melancholy provocations: and according as the humour
itself is intended, or remitted in men, as their temperature of body, or
rational soul is better able to make resistance; so are they more or less
affected. For that which is but a flea-biting to one, causeth insufferable
torment to another; and which one by his singular moderation, and well-composed
carriage can happily overcome, a second is no whit able to sustain, but upon
every small occasion of misconceived abuse, injury, grief, disgrace, loss,
cross, humour, \etc{} (if solitary, or idle) yields so far to passion, that his
complexion is altered, his digestion hindered, his sleep gone, his spirits
obscured, and his heart heavy, his hypochondries misaffected; wind, crudity, on
a sudden overtake him, and he himself overcome with melancholy. As it is with a
man imprisoned for debt, if once in the gaol, every creditor will bring his
action against him, and there likely hold him. If any discontent seize upon a
patient, in an instant all other perturbations (for-- \li{qua data porta
ruunt}) will set upon him, and then like a lame dog or broken-winged goose he
droops and pines away, and is brought at last to that ill habit or malady of
melancholy itself. So that as the philosophers make \authorfootnote{941}eight
degrees of heat and cold, we may make eighty-eight of melancholy, as the parts
affected are diversely seized with it, or have been plunged more or less into
this infernal gulf, or waded deeper into it. But all these melancholy fits,
howsoever pleasing at first, or displeasing, violent and tyrannizing over those
whom they seize on for the time; yet these fits I say, or men affected, are but
improperly so called, because they continue not, but come and go, as by some
objects they aye moved. This melancholy of which we are to treat, is a habit,
\li{mosbus sonticus}, or \li{chronicus}, a chronic or continuate disease, a
settled humour, as \authorfootnote{942}Aurelianus and
\authorfootnote{943}others call it, not errant, but fixed; and as it was long
increasing, so now being (pleasant, or painful) grown to an habit, it will
hardly be removed.

%SECT. I. MEMB. II.

%SECT. I. MEMB. II. SUBSECT. I.-_Digression of Anatomy_.
\section{Digression of Anatomy.}

\lettrine{B}{efore} I proceed to define the disease of melancholy, what it is,
or to discourse farther of it, I hold it not impertinent to make a brief
digression of the anatomy of the body and faculties of the soul, for the better
understanding of that which is to follow; because many hard words will often
occur, as mirach, hypocondries, emerods, \etc{}, imagination, reason, humours,
spirits, vital, natural, animal, nerves, veins, arteries, chylus, pituita;
which by the vulgar will not so easily be perceived, what they are, how cited,
and to what end they serve. And besides, it may peradventure give occasion to
some men to examine more accurately, search further into this most excellent
subject, and thereupon with that royal \authorfootnote{944}prophet to praise
God, ("for a man is fearfully and wonderfully made, and curiously wrought")
that have time and leisure enough, and are sufficiently informed in all other
worldly businesses, as to make a good bargain, buy and sell, to keep and make
choice of a fair hawk, hound, horse, \etc{} But for such matters as concern the
knowledge of themselves, they are wholly ignorant and careless; they know not
what this body and soul are, how combined, of what parts and faculties they
consist, or how a man differs from a dog. And what can be more ignominious and
filthy (as \authorfootnote{945}Melancthon well inveighs) "than for a man not to
know the structure and composition of his own body, especially since the
knowledge of it tends so much to the preservation, of his health, and
information of his manners?" To stir them up therefore to this study, to peruse
those elaborate works of \authorfootnote{946}Galen, Bauhines, Plater, Vesalius,
Falopius, Laurentius, Remelinus, \etc{}, which have written copiously in Latin;
or that which some of our industrious countrymen have done in our mother
tongue, not long since, as that translation of \authorfootnote{947}Columbus and
\authorfootnote{948}Microcosmographia, in thirteen books, I have made this
brief digression. Also because \authorfootnote{949}Wecker,
\authorfootnote{950}Melancthon, \authorfootnote{951}Fernelius,
\authorfootnote{952}Fuschius, and those tedious Tracts \bookcite{\textlatin{de
Anima}} (which have more compendiously handled and written of this matter,) are
not at all times ready to be had, to give them some small taste, or notice of
the rest, let this epitome suffice.

%SECT. I. MEMB. II. SUBSECT. II.-_Division of the Body, Humours, Spirits_.
\section[Division of the Body]{Division of the Body, Humours, Spirits.}

\lettrine{O}{f} the parts of the body there may be many divisions: the most
approved is that of \authorfootnote{953}Laurentius, out of Hippocrates: which
is, into parts contained, or containing. Contained, are either humours or
spirits.

\subsection{Humours.}

A humour is a liquid or fluent part of the body, comprehended in it, for the
preservation of it; and is either innate or born with us, or adventitious and
acquisite. The radical or innate, is daily supplied by nourishment, which some
call cambium, and make those secondary humours of ros and gluten to maintain
it: or acquisite, to maintain these four first primary humours, coming and
proceeding from the first concoction in the liver, by which means chylus is
excluded. Some divide them into profitable and excrementitious. But
\authorfootnote{954}Crato out of Hippocrates will have all four to be juice,
and not excrements, without which no living creature can be sustained: which
four, though they be comprehended in the mass of blood, yet they have their
several affections, by which they are distinguished from one another, and from
those adventitious, peccant, or \authorfootnote{955}diseased humours, as
Melancthon calls them.

\subsection{Blood.}

Blood is a hot, sweet, temperate, red humour, prepared in the mesaraic veins,
and made of the most temperate parts of the chylus in the liver, whose office
is to nourish the whole body, to give it strength and colour, being dispersed
by the veins through every part of it. And from it spirits are first begotten
in the heart, which afterwards by the arteries are communicated to the other
parts.

Pituita, or phlegm, is a cold and moist humour, begotten of the colder part of
the chylus (or white juice coming out of the meat digested in the stomach,) in
the liver; his office is to nourish and moisten the members of the body, which
as the tongue are moved, that they be not over dry.

Choler, is hot and dry, bitter, begotten of the hotter parts of the chylus, and
gathered to the gall: it helps the natural heat and senses, and serves to the
expelling of excrements.

\subsection{Melancholy.}

Melancholy, cold and dry, thick, black, and sour, begotten of the more feculent
part of nourishment, and purged from the spleen, is a bridle to the other two
hot humours, blood and choler, preserving them in the blood, and nourishing the
bones. These four humours have some analogy with the four elements, and to the
four ages in man.

\subsection{Serum, Sweat, Tears.}

To these humours you may add serum, which is the matter of urine, and those
excrementitious humours of the third concoction, sweat and tears.

\subsection{Spirits.}

Spirit is a most subtle vapour, which is expressed from the blood, and the
instrument of the soul, to perform all his actions; a common tie or medium
between the body and the soul, as some will have it; or as
\authorfootnote{956}Paracelsus, a fourth soul of itself. Melancthon holds the
fountain of those spirits to be the heart, begotten there; and afterward
conveyed to the brain, they take another nature to them. Of these spirits there
be three kinds, according to the three principal parts, brain, heart, liver;
natural, vital, animal. The natural are begotten in the liver, and thence
dispersed through the veins, to perform those natural actions. The vital
spirits are made in the heart of the natural, which by the arteries are
transported to all the other parts: if the spirits cease, then life ceaseth, as
in a syncope or swooning. The animal spirits formed of the vital, brought up to
the brain, and diffused by the nerves, to the subordinate members, give sense
and motion to them all.

%SECT. I. MEMB. II. SUBSECT. III.-_Similar Parts_.
\section{Similar Parts.}

\subsection{Similar Parts}

\lettrine{C}{ontaining} parts, by reason of their more solid substance, are
either homogeneal or heterogeneal, similar or dissimilar; so \Aristotle{} divides
them, \bookcite{\textlatin{lib. 1, cap. 1, de Hist. Animal.}}; Laurentius,
\bookcite{\textlatin{cap. 20, lib. 1.}} Similar, or homogeneal, are such as, if
they be divided, are still severed into parts of the same nature, as water into
water. Of these some be spermatical, some fleshy or carnal.
\authorfootnote{957}Spermatical are such as are immediately begotten of the
seed, which are bones, gristles, ligaments, membranes, nerves, arteries, veins,
skins, fibres or strings, fat.

\subsection{Bones.}

The bones are dry and hard, begotten of the thickest of the
seed, to strengthen and sustain other parts: some say there be 304, some 307,
or 313 in man's body. They have no nerves in them, and are therefore without
sense.

A gristle is a substance softer than bone, and harder than the rest, flexible,
and serves to maintain the parts of motion.

Ligaments are they that tie the bones together, and other parts to the bones,
with their subserving tendons: membranes' office is to cover the rest.

Nerves, or sinews, are membranes without, and full of marrow within; they
proceed from the brain, and carry the animal spirits for sense and motion. Of
these some be harder, some softer; the softer serve the senses, and there be
seven pair of them. The first be the optic nerves, by which we see; the second
move the eyes; the third pair serve for the tongue to taste; the fourth pair
for the taste in the palate; the fifth belong to the ears; the sixth pair is
most ample, and runs almost over all the bowels; the seventh pair moves the
tongue. The harder sinews serve for the motion of the inner parts, proceeding
from the marrow in the back, of whom there be thirty combinations, seven of the
neck, twelve of the breast, \etc{}

\subsection{Arteries.}

Arteries are long and hollow, with a double skin to convey the vital spirit; to
discern which the better, they say that Vesalius the anatomist was wont to cut
up men alive. \authorfootnote{958}They arise in the left side of the heart, and
are principally two, from which the rest are derived, aorta and venosa: aorta
is the root of all the other, which serve the whole body; the other goes to the
lungs, to fetch air to refrigerate the heart.

\subsection{Veins.}

Veins are hollow and round, like pipes, arising from the liver, carrying blood
and natural spirits; they feed all the parts. Of these there be two chief,
\emph{Vena porta} and \emph{Vena cava}, from which the rest are corrivated.
That \emph{Vena porta} is a vein coming from the concave of the liver, and
receiving those mesaraical veins, by whom he takes the chylus from the stomach
and guts, and conveys it to the liver. The other derives blood from the liver
to nourish all the other dispersed members. The branches of that \emph{Vena
porta} are the mesaraical and haemorrhoids. The branches of the \emph{cava} are
inward or outward. Inward, seminal or emulgent. Outward, in the head, arms,
feet, \etc{}, and have several names.

\subsection{Fibrae, Fat, Flesh.}

Fibrae are strings, white and solid, dispersed through the whole member, and
right, oblique, transverse, all which have their several uses. Fat is a similar
part, moist, without blood, composed of the most thick and unctuous matter of
the blood. The \authorfootnote{959}skin covers the rest, and hath
\li{cuticulum}, or a little skin tinder it. Flesh is soft and ruddy, composed
of the congealing of blood, \etc{}

%SECT. I. MEMB. II. SUBSECT. IV.-_Dissimilar Parts_.
\section{Dissimilar Parts.}

\lettrine{D}{issimilar} parts are those which we call organical, or
instrumental, and they be inward or outward. The chiefest outward parts are
situate forward or backward:--forward, the crown and foretop of the head,
skull, face, forehead, temples, chin, eyes, ears, nose, \etc{}, neck, breast,
chest, upper and lower part of the belly, hypocondries, navel, groin, flank,
\etc{}; backward, the hinder part of the head, back, shoulders, sides, loins,
hipbones, \li{os sacrum}, buttocks, \etc{} Or joints, arms, hands, feet, legs,
thighs, knees, \etc{} Or common to both, which, because they are obvious and
well known, I have carelessly repeated, \li{eaque praecipua et grandiora
tantum; quod reliquum ex libris de anima qui volet, accipiat}.

Inward organical parts, which cannot be seen, are divers in number, and have
several names, functions, and divisions; but that of
\authorfootnote{960}Laurentius is most notable, into noble or ignoble parts. Of
the noble there be three principal parts, to which all the rest belong, and
whom they serve--brain, heart, liver; according to whose site, three regions,
or a threefold division, is made of the whole body. As first of the head, in
which the animal organs are contained, and brain itself, which by his nerves
give sense and motion to the rest, and is, as it were, a privy counsellor and
chancellor to the heart. The second region is the chest, or middle belly, in
which the heart as king keeps his court, and by his arteries communicates life
to the whole body. The third region is the lower belly, in which the liver
resides as a \li{Legat a latere}, with the rest of those natural organs,
serving for concoction, nourishment, expelling of excrements. This lower region
is distinguished from the upper by the midriff, or diaphragma, and is
subdivided again by \authorfootnote{961}some into three concavities or regions,
upper, middle, and lower. The upper of the hypocondries, in whose right side is
the liver, the left the spleen; from which is denominated hypochondriacal
melancholy. The second of the navel and flanks, divided from the first by the
rim. The last of the water course, which is again subdivided into three other
parts. The Arabians make two parts of this region, \li{Epigastrium} and
\li{Hypogastrium}, upper or lower. \li{Epigastrium} they call \emph{Mirach},
from whence comes \li{Mirachialis Melancholia}, sometimes mentioned of them. Of
these several regions I will treat in brief apart; and first of the third
region, in which the natural organs are contained.

\subsection[The Lower Region]{De Anima.-The Lower Region, Natural Organs.}

But you that are readers in the meantime, "Suppose you were now brought into
some sacred temple, or majestical palace" (as \authorfootnote{962}Melancthon
saith), "to behold not the matter only, but the singular art, workmanship, and
counsel of this our great Creator. And it is a pleasant and profitable
speculation, if it be considered aright." The parts of this region, which
present themselves to your consideration and view, are such as serve to
nutrition or generation. Those of nutrition serve to the first or second
concoction; as the oesophagus or gullet, which brings meat and drink into the
stomach. The ventricle or stomach, which is seated in the midst of that part of
the belly beneath the midriff, the kitchen, as it were, of the first
concoction, and which turns our meat into chylus. It hath two mouths, one
above, another beneath. The upper is sometimes taken for the stomach itself;
the lower and nether door (as Wecker calls it) is named Pylorus. This stomach
is sustained by a large kell or caul, called omentum; which some will have the
same with peritoneum, or rim of the belly. From the stomach to the very
fundament are produced the guts, or intestina, which serve a little to alter
and distribute the chylus, and convey away the excrements. They are divided
into small and great, by reason of their site and substance, slender or
thicker: the slender is duodenum, or whole gut, which is next to the stomach,
some twelve inches long, saith \authorfootnote{963}Fuschius. Jejunum, or empty
gut, continuate to the other, which hath many mesaraic veins annexed to it,
which take part of the chylus to the liver from it. Ilion the third, which
consists of many crinkles, which serves with the rest to receive, keep, and
distribute the chylus from the stomach. The thick guts are three, the blind
gut, colon, and right gut. The blind is a thick and short gut, having one
mouth, in which the ilium and colon meet: it receives the excrements, and
conveys them to the colon. This colon hath many windings, that the excrements
pass not away too fast: the right gut is straight, and conveys the excrements
to the fundament, whose lower part is bound up with certain muscles called
sphincters, that the excrements may be the better contained, until such time as
a man be willing to go to the stool. In the midst of these guts is situated the
mesenterium or midriff, composed of many veins, arteries, and much fat, serving
chiefly to sustain the guts. All these parts serve the first concoction. To the
second, which is busied either in refining the good nourishment or expelling
the bad, is chiefly belonging the liver, like in colour to congealed blood, the
shop of blood, situate in the right hypochondry, in figure like to a half-moon,
\li{generosum membrum} Melancthon styles it, a generous part; it serves to turn
the chylus to blood, for the nourishment of the body. The excrements of it are
either choleric or watery, which the other subordinate parts convey. The gall
placed in the concave of the liver, extracts choler to it: the spleen,
melancholy; which is situate on the left side, over against the liver, a spongy
matter, that draws this black choler to it by a secret virtue, and feeds upon
it, conveying the rest to the bottom of the stomach, to stir up appetite, or
else to the guts as an excrement. That watery matter the two kidneys expurgate
by those emulgent veins and ureters. The emulgent draw this superfluous
moisture from the blood; the two ureters convey it to the bladder, which, by
reason of his site in the lower belly, is apt to receive it, having two parts,
neck and bottom: the bottom holds the water, the neck is constringed with a
muscle, which, as a porter, keeps the water from running out against our will.

Members of generation are common to both sexes, or peculiar to one; which,
because they are impertinent to my purpose, I do voluntarily omit.

\subsection{Middle Region.}

Next in order is the middle region, or chest, which comprehends the vital
faculties and parts; which (as I have said) is separated from the lower belly
by the diaphragma or midriff, which is a skin consisting of many nerves,
membranes; and amongst other uses it hath, is the instrument of laughing. There
is also a certain thin membrane, full of sinews, which covereth the whole chest
within, and is called pleura, the seat of the disease called pleurisy, when it
is inflamed; some add a third skin, which is termed mediastinus, which divides
the chest into two parts, right and left; of this region the principal part is
the heart, which is the seat and fountain of life, of heat, of spirits, of
pulse and respiration--the sun of our body, the king and sole commander of
it--the seat and organ of all passions and affections. \li{Primum vivens,
ultimum moriens}, it lives first, dies last in all creatures. Of a pyramidical
form, and not much unlike to a pineapple; a part worthy of
\authorfootnote{964}admiration, that can yield such variety of affections, by
whose motion it is dilated or contracted, to stir and command the humours in
the body. As in sorrow, melancholy; in anger, choler; in joy, to send the blood
outwardly; in sorrow, to call it in; moving the humours, as horses do a
chariot. This heart, though it be one sole member, yet it may be divided into
two creeks right and left. The right is like the moon increasing, bigger than
the other part, and receives blood from \emph{vena cava}, distributing some of
it to the lungs to nourish them; the rest to the left side, to engender
spirits. The left creek hath the form of a cone, and is the seat of life,
which, as a torch doth oil, draws blood unto it, begetting of it spirits and
fire; and as fire in a torch, so are spirits in the blood; and by that great
artery called aorta, it sends vital spirits over the body, and takes air from
the lungs by that artery which is called \emph{venosa}; so that both creeks
have their vessels, the right two veins, the left two arteries, besides those
two common anfractuous ears, which serve them both; the one to hold blood, the
other air, for several uses. The lungs is a thin spongy part, like an ox hoof,
(saith \authorfootnote{965}Fernelius) the town-clerk or crier,
(\authorfootnote{966}one terms it) the instrument of voice, as an orator to a
king; annexed to the heart, to express their thoughts by voice. That it is the
instrument of voice, is manifest, in that no creature can speak, or utter any
voice, which wanteth these lights. It is, besides, the instrument of
respiration, or breathing; and its office is to cool the heart, by sending air
unto it, by the venosal artery, which vein comes to the lungs by that
\emph{aspera arteria} which consists of many gristles, membranes, nerves,
taking in air at the nose and mouth, and by it likewise exhales the fumes of
the heart.

In the upper region serving the animal faculties, the chief organ is the brain,
which is a soft, marrowish, and white substance, engendered of the purest part
of seed and spirits, included by many skins, and seated within the skull or
brain pan; and it is the most noble organ under heaven, the dwelling-house and
seat of the soul, the habitation of wisdom, memory, judgment, reason, and in
which man is most like unto God; and therefore nature hath covered it with a
skull of hard bone, and two skins or membranes, whereof the one is called
\emph{dura mater}, or meninx, the other \emph{pia mater}. The dura mater is
next to the skull, above the other, which includes and protects the brain. When
this is taken away, the pia mater is to be seen, a thin membrane, the next and
immediate cover of the brain, and not covering only, but entering into it. The
brain itself is divided into two parts, the fore and hinder part; the fore part
is much bigger than the other, which is called the little brain in respect of
it. This fore part hath many concavities distinguished by certain ventricles,
which are the receptacles of the spirits, brought hither by the arteries from
the heart, and are there refined to a more heavenly nature, to perform the
actions of the soul. Of these ventricles there are three--right, left, and
middle. The right and left answer to their site, and beget animal spirits; if
they be any way hurt, sense and motion ceaseth. These ventricles, moreover, are
held to be the seat of the common sense. The middle ventricle is a common
concourse and cavity of them both, and hath two passages--the one to receive
pituita, and the other extends itself to the fourth creek; in this they place
imagination and cogitation, and so the three ventricles of the fore part of the
brain are used. The fourth creek behind the head is common to the cerebel or
little brain, and marrow of the backbone, the last and most solid of all the
rest, which receives the animal spirits from the other ventricles, and conveys
them to the marrow in the back, and is the place where they say the memory is
seated.

%SUBSECT. V.
\section{Of the Soul and her Faculties.}


\lettrine{A}{ccording} to \Aristotle\authorfootnote{967}, the soul is defined
to be \textgreek{ἐντελέχεια}, \li{perfectio et actus primus corporis organici,
vitam habentis in potentia}: the perfection or first act of an organical body,
having power of life, which most \authorfootnote{968}philosophers approve. But
many doubts arise about the essence, subject, seat, distinction, and
subordinate faculties of it. For the essence and particular knowledge, of all
other things it is most hard (be it of man or beast) to discern, as
\authorfootnote{969}\Aristotle{} himself, \authorfootnote{970}\Tully{},
\authorfootnote{971}Picus Mirandula, \authorfootnote{972}Tolet, and other
neoteric philosophers confess:-- \authorfootnote{973}"We can understand all
things by her, but what she is we cannot apprehend." Some therefore make one
soul, divided into three principal faculties; others, three distinct souls.
Which question of late hath been much controverted by Picolomineus and Zabarel.
\authorfootnote{974}Paracelsus will have four souls, adding to the three grand
faculties a spiritual soul: which opinion of his,
\idxname{campanella}[Campanella][\textlatin{De sensu rerum et magia}], in his
book \bookcite{\textlatin{de sensu rerum}} \authorfootnote{975}much labours to
demonstrate and prove, because carcasses bleed at the sight of the murderer;
with many such arguments And \authorfootnote{976}some again, one soul of all
creatures whatsoever, differing only in organs; and that beasts have reason as
well as men, though, for some defect of organs, not in such measure. Others
make a doubt whether it be all in all, and all in every part; which is amply
discussed in Zabarel amongst the rest. The \authorfootnote{977}common division
of the soul is into three principal faculties--vegetal, sensitive, and
rational, which make three distinct kinds of living creatures--vegetal plants,
sensible beasts, rational men. How these three principal faculties are
distinguished and connected, \li{Humano ingenio inaccessum videtur}, is beyond
human capacity, as \authorfootnote{978}Taurellus, Philip, Flavins, and others
suppose. The inferior may be alone, but the superior cannot subsist without the
other; so sensible includes vegetal, rational both; which are contained in it
(saith \Aristotle{}) \li{ut trigonus in tetragono} as a triangle in a quadrangle.

\subsection{Vegetal Soul.}

Vegetal, the first of the three distinct faculties, is defined to be "a
substantial act of an organical body, by which it is nourished, augmented, and
begets another like unto itself." In which definition, three several operations
are specified--altrix, auctrix, procreatrix; the first is
\authorfootnote{979}nutrition, whose object is nourishment, meat, drink, and
the like; his organ the liver in sensible creatures; in plants, the root or
sap. His office is to turn the nutriment into the substance of the body
nourished, which he performs by natural heat. This nutritive operation hath
four other subordinate functions or powers belonging to it--attraction,
retention, digestion, expulsion.

\subsection{Attraction.}

\authorfootnote{980}Attraction is a ministering faculty, which, as a loadstone
doth iron, draws meat into the stomach, or as a lamp doth oil; and this
attractive power is very necessary in plants, which suck up moisture by the
root, as, another mouth, into the sap, as a like stomach.

\subsection{Retention.}

Retention keeps it, being attracted unto the stomach, until such time it be
concocted; for if it should pass away straight, the body could not be
nourished.

\subsection{Digestion.}

Digestion is performed by natural heat; for as the flame of a torch consumes
oil, wax, tallow, so doth it alter and digest the nutritive matter. Indigestion
is opposite unto it, for want of natural heat. Of this digestion there be three
differences--maturation, elixation, assation.

\subsection{Maturation.}

Maturation is especially observed in the fruits of trees; which are then said
to be ripe, when the seeds are fit to be sown again. Crudity is opposed to it,
which gluttons, epicures, and idle persons are most subject unto, that use no
exercise to stir natural heat, or else choke it, as too much wood puts out a
fire.

\subsection{Elixation.}

Elixation is the seething of meat in the stomach, by
the said natural heat, as meat is boiled in a pot; to which corruption or
putrefaction is opposite.

\subsection{Assation.}

\worddef{roasting}{Assation} is a concoction of the inward moisture by heat; his opposite is
\lit{semiustulation}{being half-burnt}.

\subsection{Order of Concoction fourfold.}

Besides these three several operations of digestion, there is a fourfold order
of concoction:--mastication, or chewing in the mouth; chilification of this so
chewed meat in the stomach; the third is in the liver, to turn this chylus into
blood, called sanguification; the last is assimilation, which is in every part.

\subsection{Expulsion.}

Expulsion is a power of nutrition, by which it expels all superfluous
excrements, and relics of meat and drink, by the guts, bladder, pores; as by
purging, vomiting, spitting, sweating, urine, hairs, nails, \etc{}

\subsection{Augmentation.}

As this nutritive faculty serves to nourish the body, so doth the augmenting
faculty (the second operation or power of the vegetal faculty) to the
increasing of it in quantity, according to all dimensions, long, broad, thick,
and to make it grow till it come to his due proportion and perfect shape; which
hath his period of augmentation, as of consumption; and that most certain, as
the poet observes:--

\translatedverse{%
\begin{latin}
\begin{verse}
Stat sua cuique dies, breve et irreparabile tempus\\*
Omnibus est vitae.------\\!
\end{verse}
\end{latin}}{%
\begin{verse}%
A term of life is set to every man,\\*
Which is but short, and pass it no one can.\\!
\end{verse}}{}

\subsection{Generation.}

The last of these vegetal faculties is generation, which begets another by
means of seed, like unto itself, to the perpetual preservation of the species.
To this faculty they ascribe three subordinate operations:--the first to turn
nourishment into seed, \etc{}

\subsection{Life and Death concomitants of the Vegetal Faculties.}

Necessary concomitants or affections of this vegetal faculty are life and his
privation, death. To the preservation of life the natural heat is most
requisite, though \worddef{dryness}{siccity} and humidity, and those first qualities, be not
excluded. This heat is likewise in plants, as appears by their increasing,
fructifying, \etc{}, though not so easily perceived. In all bodies it must have
radical \authorfootnote{981}moisture to preserve it, that it be not consumed;
to which preservation our clime, country, temperature, and the good or bad use
of those six non-natural things avail much. For as this natural heat and
moisture decays, so doth our life itself; and if not prevented before by some
violent accident, or interrupted through our own default, is in the end dried
up by old age, and extinguished by death for want of matter, as a lamp for
defect of oil to maintain it.

%SECT. I. MEMB. II. SUBSECT. VI.-_Of the sensible Soul_.
\section{Of the sensible Soul.}

\lettrine{N}{ext} in order is the sensible faculty, which is as far beyond the
other in dignity, as a beast is preferred to a plant, having those vegetal
powers included in it. 'Tis defined an "Act of an organical body by which it
lives, hath sense, appetite, judgment, breath, and motion." His object in
general is a sensible or passible quality, because the sense is affected with
it. The general organ is the brain, from which principally the sensible
operations are derived. This sensible soul is divided into two parts,
apprehending or moving. By the apprehensive power we perceive the species of
sensible things present, or absent, and retain them as wax doth the print of a
seal. By the moving, the body is outwardly carried from one place to another;
or inwardly moved by spirits and pulse. The apprehensive faculty is subdivided
into two parts, inward or outward. Outward, as the five senses, of touching,
hearing, seeing, smelling, tasting, to which you may add \Scaliger{}'s sixth sense
of titillation, if you please; or that of speech, which is the sixth external
sense, according to Lullius. Inward are three--common sense, phantasy, memory.
Those five outward senses have their object in outward things only, and such as
are present, as the eye sees no colour except it be at hand, the ear sound.
Three of these senses are of commodity, hearing, sight, and smell; two of
necessity, touch, and taste, without which we cannot live. Besides, the
sensitive power is active or passive. Active in sight, the eye sees the colour;
passive when it is hurt by his object, as the eye by the sunbeams. According to
that axiom, \li{visibile forte destruit sensum}\authorlatintrans{982}. Or if
the object be not pleasing, as a bad sound to the ear, a stinking smell to the
nose, \etc{}

\subsection{Sight.}

Of these five senses, sight is held to be most precious, and the best, and that
by reason of his object, it sees the whole body at once. By it we learn, and
discern all things, a sense most excellent for use: to the sight three things
are required; the object, the organ, and the medium. The object in general is
visible, or that which is to be seen, as colours, and all shining bodies. The
medium is the illumination of the air, which comes from
\authorfootnote{983}light, commonly called diaphanum; for in dark we cannot
see. The organ is the eye, and chiefly the apple of it, which by those optic
nerves, concurring both in one, conveys the sight to the common sense. Between
the organ and object a true distance is required, that it be not too near, or
too far off! Many excellent questions appertain to this sense, discussed by
philosophers: as whether this sight be caused \li{intra mittendo, vel extra
mittendo}, \etc{}, by receiving in the visible species, or sending of them out,
which \authorfootnote{984}Plato, \authorfootnote{985}\Plutarch{},
\authorfootnote{986}Macrobius, \authorfootnote{987}Lactantius and others
dispute. And, besides, it is the subject of the perspectives, of which Alhazen
the Arabian, Vitellio, Roger Bacon, Baptista Porta, Guidus Ubaldus, Aquilonius,
\etc{}, have written whole volumes.

\subsection{Hearing.}

Hearing, a most excellent outward sense, "by which we learn and get knowledge."
His object is sound, or that which is heard; the medium, air; organ, the ear.
To the sound, which is a collision of the air, three things are required; a
body to strike, as the hand of a musician; the body struck, which must be solid
and able to resist; as a bell, lute-string, not wool, or sponge; the medium,
the air; which is inward, or outward; the outward being struck or collided by a
solid body, still strikes the next air, until it come to that inward natural
air, which as an exquisite organ is contained in a little skin formed like a
drum-head, and struck upon by certain small instruments like drum-sticks,
conveys the sound by a pair of nerves, appropriated to that use, to the common
sense, as to a judge of sounds. There is great variety and much delight in
them; for the knowledge of which, consult with Boethius and other musicians.

\subsection{Smelling.}

Smelling is an "outward sense, which apprehends by the nostrils drawing in
air;" and of all the rest it is the weakest sense in men. The organ in the
nose, or two small hollow pieces of flesh a little above it: the medium the air
to men, as water to fish: the object, smell, arising from a mixed body
resolved, which, whether it be a quality, fume, vapour, or exhalation, I will
not now dispute, or of their differences, and how they are caused. This sense
is an organ of health, as sight and hearing, saith
\authorfootnote{988}Agellius, are of discipline; and that by avoiding bad
smells, as by choosing good, which do as much alter and affect the body many
times, as diet itself.

\subsection{Taste.}

Taste, a necessary sense, "which perceives all savours by the tongue and
palate, and that by means of a thin spittle, or watery juice." His organ is the
tongue with his tasting nerves; the medium, a watery juice; the object, taste,
or savour, which is a quality in the juice, arising from the mixture of things
tasted. Some make eight species or kinds of savour, bitter, sweet, sharp, salt,
\etc{}, all which sick men (as in an ague) cannot discern, by reason of their
organs misaffected.

\subsection{Touching.}

Touch, the last of the senses, and most ignoble, yet of as great necessity as
the other, and of as much pleasure. This sense is exquisite in men, and by his
nerves dispersed all over the body, perceives any tactile quality. His organ
the nerves; his object those first qualities, hot, dry, moist, cold; and those
that follow them, hard, soft, thick, thin, \etc{} Many delightsome questions
are moved by philosophers about these five senses; their organs, objects,
mediums, which for brevity I omit.

%SECT. I. MEMB. II. SUBSECT. VII.-_Of the Inward Senses._
\section{Of the Inward Senses.}

\subsection{Common Sense.}

\lettrine{I}{nner} senses are three in number, so called, because they be
within the brainpan, as common sense, phantasy, memory. Their objects are not
only things present, but they perceive the sensible species of things to come,
past, absent, such as were before in the sense. This common sense is the judge
or moderator of the rest, by whom we discern all differences of objects; for by
mine eye I do not know that I see, or by mine ear that I hear, but by my common
sense, who judgeth of sounds and colours: they are but the organs to bring the
species to be censured; so that all their objects are his, and all their
offices are his. The fore part of the brain is his organ or seat.

\subsection{Phantasy.}

Phantasy, or imagination, which some call estimative, or cogitative,
(confirmed, saith \authorfootnote{989}Fernelius, by frequent meditation,) is an
inner sense which doth more fully examine the species perceived by common
sense, of things present or absent, and keeps them longer, recalling them to
mind again, or making new of his own. In time of sleep this faculty is free,
and many times conceive strange, stupend, absurd shapes, as in sick men we
commonly observe. His organ is the middle cell of the brain; his objects all
the species communicated to him by the common sense, by comparison of which he
feigns infinite other unto himself. In melancholy men this faculty is most
powerful and strong, and often hurts, producing many monstrous and prodigious
things, especially if it be stirred up by some terrible object, presented to it
from common sense or memory. In poets and painters imagination forcibly works,
as appears by their several fictions, antics, images: as \Ovid{}'s house of sleep,
Psyche's palace in \Apuleius{}, \etc{} In men it is subject and governed by
reason, or at least should be; but in brutes it hath no superior, and is
\li{ratio brutorum}, all the reason they have.

\subsection{Memory.}

Memory lays up all the species which the senses have
brought in, and records them as a good register, that they may be forthcoming
when they are called for by phantasy and reason. His object is the same with
phantasy, his seat and organ the back part of the brain.

\subsection{Affections of the Senses, sleep and waking.}

The affections of these senses are sleep and waking, common to all sensible
creatures. "Sleep is a rest or binding of the outward senses, and of the common
sense, for the preservation of body and soul" (as \Scaliger{}
\authorfootnote{990}defines it); for when the common sense resteth, the outward
senses rest also. The phantasy alone is free, and his commander reason: as
appears by those imaginary dreams, which are of divers kinds, natural, divine,
demoniacal, \etc{}, which vary according to humours, diet, actions, objects,
\etc{}, of which Artemidorus, Cardanus, and Sambucus, with their several
interpretators, have written great volumes. This litigation of senses proceeds
from an inhibition of spirits, the way being stopped by which they should come;
this stopping is caused of vapours arising out of the stomach, filling the
nerves, by which the spirits should be conveyed. When these vapours are spent,
the passage is open, and the spirits perform their accustomed duties: so that
"waking is the action and motion of the senses, which the spirits dispersed
over all parts cause."

%SECT. I. MEMB. II. SUBSECT. VIII.-_Of the Moving Faculty_.
\section{Of the Moving Faculty.}

\subsection{Appetite}

\lettrine{T}{his} moving faculty is the other power of the sensitive soul,
which causeth all those inward and outward animal motions in the body. It is
divided into two faculties, the power of appetite, and of moving from place to
place. This of appetite is threefold, so some will have it; natural, as it
signifies any such inclination, as of a stone to fall downward, and such
actions as retention, expulsion, which depend not on sense, but are vegetal, as
the appetite of meat and drink; hunger and thirst. Sensitive is common to men
and brutes. Voluntary, the third, or intellective, which commands the other two
in men, and is a curb unto them, or at least should be, but for the most part
is captivated and overruled by them; and men are led like beasts by sense,
giving reins to their concupiscence and several lusts. For by this appetite the
soul is led or inclined to follow that good which the senses shall approve, or
avoid that which they hold evil: his object being good or evil, the one he
embraceth, the other he rejecteth; according to that aphorism, \li{Omnia
appetunt bonum}, all things seek their own good, or at least seeming good. This
power is inseparable from sense, for where sense is, there are likewise
pleasure and pain. His organ is the same with the common sense, and is divided
into two powers, or inclinations, concupiscible or irascible: or (as one
\authorfootnote{991}translates it) coveting, anger invading, or impugning.
Concupiscible covets always pleasant and delightsome things, and abhors that
which is distasteful, harsh, and unpleasant. \li{Irascible, quasi
\authorfootnote{992}aversans per iram et odium}, as avoiding it with anger and
indignation. All affections and perturbations arise out of these two fountains,
which, although the stoics make light of, we hold natural, and not to be
resisted. The good affections are caused by some object of the same nature; and
if present, they procure joy, which dilates the heart, and preserves the body:
if absent, they cause hope, love, desire, and concupiscence. The bad are simple
or mixed: simple for some bad object present, as sorrow, which contracts the
heart, macerates the soul, subverts the good estate of the body, hindering all
the operations of it, causing melancholy, and many times death itself; or
future, as fear. Out of these two arise those mixed affections and passions of
anger, which is a desire of revenge; hatred, which is inveterate anger; zeal,
which is offended with him who hurts that he loves; and
\textgreek{ἐπικαιρεκακία}, a compound affection of joy and hate, when we
rejoice at other men's mischief, and are grieved at their prosperity; pride,
self-love, emulation, envy, shame, \etc{}, of which elsewhere.

\emph{Moving from place to place}, is a faculty necessarily following the
other. For in vain were it otherwise to desire and to abhor, if we had not
likewise power to prosecute or eschew, by moving the body from place to place:
by this faculty therefore we locally move the body, or any part of it, and go
from one place to another. To the better performance of which, three things are
requisite: that which moves; by what it moves; that which is moved. That which
moves, is either the efficient cause, or end. The end is the object, which is
desired or eschewed; as in a dog to catch a hare, \etc{} The efficient cause in
man is reason, or his subordinate phantasy, which apprehends good or bad
objects: in brutes imagination alone, which moves the appetite, the appetite
this faculty, which by an admirable league of nature, and by meditation of the
spirit, commands the organ by which it moves: and that consists of nerves,
muscles, cords, dispersed through the whole body, contracted and relaxed as the
spirits will, which move the muscles, or \authorfootnote{993}nerves in the
midst of them, and draw the cord, and so \li{per consequens} the joint, to the
place intended. That which is moved, is the body or some member apt to move.
The motion of the body is divers, as going, running, leaping, dancing, sitting,
and such like, referred to the predicament of \li{situs}. Worms creep, birds
fly, fishes swim; and so of parts, the chief of which is respiration or
breathing, and is thus performed. The outward air is drawn in by the vocal
artery, and sent by mediation of the midriff to the lungs, which, dilating
themselves as a pair of bellows, reciprocally fetch it in, and send it out to
the heart to cool it; and from thence now being hot, convey it again, still
taking in fresh. Such a like motion is that of the pulse, of which, because
many have written whole books, I will say nothing.

%SECT. I. MEMB. II. SUBSECT. IX.-_Of the Rational Soul._
\section{Of the Rational Soul.}

\lettrine{I}{n} the precedent subsections I have anatomised those inferior
faculties of the soul; the rational remaineth, "a pleasant, but a doubtful
subject" (as \authorfootnote{994}one terms it), and with the like brevity to be
discussed. Many erroneous opinions are about the essence and original of it;
whether it be fire, as Zeno held; harmony, as Aristoxenus; number, as
Xenocrates; whether it be organical, or inorganical; seated in the brain, heart
or blood; mortal or immortal; how it comes into the body. Some hold that it is
\li{ex traduce}, as \bookcite{\textlatin{Phil. 1. de Anima}}, Tertullian,
Lactantius \bookcite{\textlatin{de opific. Dei, cap. 19.}} Hugo,
\bookcite{\textlatin{lib. de Spiritu et Anima}}, Vincentius Bellavic.
\bookcite{\textlatin{spec. natural. lib. 23. cap. 2. et 11.}} Hippocrates,
\Avicenna{}, and many \authorfootnote{995}late writers; that one man begets
another, body and soul; or as a candle from a candle, to be produced from the
seed: otherwise, say they, a man begets but half a man, and is worse than a
beast that begets both matter and form; and, besides, the three faculties of
the soul must be together infused, which is most absurd as they hold, because
in beasts they are begot, the two inferior I mean, and may not be well
separated in men. \authorfootnote{996}Galen supposeth the soul \li{crasin
esse}, to be the temperature itself; Trismegistus, Musaeus, Orpheus, \Homer{},
Pindarus, Phaerecides Syrus, Epictetus, with the Chaldees and Egyptians,
affirmed the soul to be immortal, as did those British
\authorfootnote{997}Druids of old. The \authorfootnote{998}Pythagoreans defend
Metempsychosis; and Palingenesia, that souls go from one body to another,
\li{epota prius Lethes unda}, as men into wolves, bears, dogs, hogs, as they
were inclined in their lives, or participated in conditions:

\translatedverse{%
\begin{latin}
\begin{verse}
------inque ferinas\\*
Possumus ire domus, pecudumque in corpora condi.\\!
\end{verse}
\end{latin}}{%\authorlatintrans{999.5}
\begin{verse}%
We, who may take up our abode in wild beasts,\\*
or be lodged in the breasts of cattle\\!
\end{verse}}{%
\attrib{\getauthornote{999}}}

Lucian's cock was first Euphorbus, a captain:\authorfootnote{1000}

\begin{latin}
\begin{verse}%
Ille ego (nam memini) Trojani tempore belli,\\*
Panthoides Euphorbus eram,\\!
\end{verse}%
\end{latin}

a horse, a man, a sponge. \authorfootnote{1001}Julian the Apostate thought
Alexander's soul was descended into his body: Plato in Timaeo, and in his
Phaedon, (for aught I can perceive,) differs not much from this opinion, that
it was from God at first, and knew all, but being enclosed in the body, it
forgets, and learns anew, which he calls \li{reminiscentia}, or recalling, and
that it was put into the body for a punishment; and thence it goes into a
beast's, or man's, as appears by his pleasant fiction \bookcite{\textlatin{de
sortitione animarum, lib. 10. de rep.}} and after \authorfootnote{1002}ten
thousand years is to return into the former body again,

\begin{latin}
\begin{verse}%
------post varios annos, per mille figuras,\\*
Rursus ad humanae fertur primordia vitae.\\!
\end{verse}%
\end{latin}
\attrib{\getauthornote{1003}}

Others deny the immortality of it, which Pomponatus of Padua decided out of \Aristotle{} not long since, Plinias Avunculus, \bookcite{\textlatin{cap. 1. lib. 2, et lib. 7. cap. 55}}; \Seneca{}, \bookcite{\textlatin{lib. 7. epist. ad Lucilium, epist. 55}}; Dicearchus \bookcite{\textlatin{in Tull. Tusc.}} Epicurus, Aratus, Hippocrates, Galen, Lucretius, \bookcite{\textlatin{lib. 1.}}

\translatedverse{%
\begin{latin}
\begin{verse}
(Praeterea gigni pariter cum corpore,\\*
et una cresere sentimus, pariterque senescere mentem.)\\!
\end{verse}
\end{latin}}{%\authorlatintrans{1004}
\begin{verse}%
(Besides, we observe that the mind is born with the body,\\*
grows with it, and decays with it)\\!
\end{verse}}{}%

Averroes, and I know not how many Neoterics. \authorfootnote{1005}"This
question of the immortality of the soul, is diversely and wonderfully impugned
and disputed, especially among the Italians of late," saith Jab. Colerus,
\bookcite{\textlatin{lib. de immort. animae, cap. 1.}} The popes themselves
have doubted of it: Leo Decimus, that Epicurean pope, as
\authorfootnote{1006}some record of him, caused this question to be discussed
pro and con before him, and concluded at last, as a profane and atheistical
moderator, with that verse of Cornelius Gallus,

\translatedverse{%
\begin{latin}
\begin{verse}
Et redit in nihilum, quod fuit ante nihil.\\!
\end{verse}
\end{latin}}{%
\begin{verse}%
It began of nothing, and in nothing it ends.\\!
\end{verse}}{}%

It began of nothing, and in nothing it ends. Zeno and his Stoics, as
\authorfootnote{1007}\Austin{} quotes him, supposed the soul so long to continue,
till the body was fully putrified, and resolved into \li{materia prima}: but
after that, \li{in fumos evanescere}, to be extinguished and vanished; and in
the meantime, whilst the body was consuming, it wandered all abroad, \li{et e
longinquo multa annunciare}, and (as that Clazomenian Hermotimus averred) saw
pretty visions, and suffered I know not what.

\translatedverse{%
\begin{latin}
\begin{verse}
Errant exangues sine corpore et ossibus umbrae.\\!
\end{verse}
\end{latin}}{%\authorlatintrans{1008.5}
\begin{verse}%
The bloodless shades without either body or bones wanter
\end{verse}}{%
\attrib{\getauthornote{1008}}}

Others grant the immortality thereof, but they make many fabulous fictions in
the meantime of it, after the departure from the body: like Plato's Elysian
fields, and that Turkey paradise. The souls of good men they deified; the bad
(saith \authorfootnote{1009}\Austin{}) became devils, as they supposed; with many
such absurd tenets, which he hath confuted. Hierome, \Austin{}, and other Fathers
of the church, hold that the soul is immortal, created of nothing, and so
infused into the child or embryo in his mother's womb, six months after the
\authorfootnote{1010}conception; not as those of brutes, which are \li{ex
traduce}, and dying with them vanish into nothing. To whose divine treatises,
and to the Scriptures themselves, I rejourn all such atheistical spirits, as
\Tully{} did Atticus, doubting of this point, to Plato's Phaedon. Or if they
desire philosophical proofs and demonstrations, I refer them to Niphus, Nic.
Faventinus' tracts of this subject. To Fran. and John Picus
\bookcite{\textlatin{in digress: sup. 3. de Anima}}, Tholosanus, Eugubinus, To.
Soto, Canas, Thomas, Peresius, Dandinus, Colerus, to that elaborate tract in
Zanchius, to Tolet's Sixty Reasons, and Lessius' Twenty-two Arguments, to prove
the immortality of the soul. \idxname{campanella}[Campanella][\textlatin{De
sensu rerum et magia}], \bookcite{\textlatin{lib. de sensu rerum}}, is large in
the same discourse, Albertinus the Schoolman, Jacob. Nactantus,
\bookcite{\textlatin{tom. 2. op.}} handleth it in four questions, Antony
Brunus, Aonius Palearius, \idxname{mersenne}[Marinus Marcennus], with many
others. This reasonable soul, which \Austin{} calls a spiritual substance moving
itself, is defined by philosophers to be "the first substantial act of a
natural, humane, organical body, by which a man lives, perceives, and
understands, freely doing all things, and with election." Out of which
definition we may gather, that this rational soul includes the powers, and
performs the duties of the two other, which are contained in it, and all three
faculties make one soul, which is inorganical of itself, although it be in all
parts, and incorporeal, using their organs, and working by them. It is divided
into two chief parts, differing in office only, not in essence. The
understanding, which is the rational power apprehending; the will, which is the
rational power moving: to which two, all the other rational powers are subject
and reduced.

%SECT. I. MEMB. II. SUBSECT. X.-_Of the Understanding_.
\section{Of the Understanding.}

\lettrine[ante={\large{}"}]{U}{nderstanding} is a power of the soul,
\authorfootnote{1011}by which we perceive, know, remember, and judge as well
singulars, as universals, having certain innate notices or beginnings of arts,
a reflecting action, by which it judgeth of his own doings, and examines them."
Out of this definition (besides his chief office, which is to apprehend, judge
all that he performs, without the help of any instruments or organs) three
differences appear betwixt a man and a beast. As first, the sense only
comprehends singularities, the understanding universalities. Secondly, the
sense hath no innate notions. Thirdly, brutes cannot reflect upon themselves.
Bees indeed make neat and curious works, and many other creatures besides; but
when they have done, they cannot judge of them. His object is God,
\lit{ens}{being}, all nature, and whatsoever is to be understood: which
successively it apprehends. The object first moving the understanding, is some
sensible thing; after by discoursing, the mind finds out the corporeal
substance, and from thence the spiritual. His actions (some say) are
apprehension, composition, division, discoursing, reasoning, memory, which some
include in invention, and judgment. The common divisions are of the
understanding, agent, and patient; speculative, and practical; in habit, or in
act; simple, or compound. The agent is that which is called the wit of man,
acumen or subtlety, sharpness of invention, when he doth invent of himself
without a teacher, or learns anew, which abstracts those intelligible species
from the phantasy, and transfers them to the passive understanding,
\authorfootnote{1012}"because there is nothing in the understanding, which was
not first in the sense." That which the imagination hath taken from the sense,
this agent judgeth of, whether it be true or false; and being so judged he
commits it to the passible to be kept. The agent is a doctor or teacher, the
passive a scholar; and his office is to keep and further judge of such things
as are committed to his charge; as a bare and rased table at first, capable of
all forms and notions. Now these notions are twofold, actions or habits:
actions, by which we take notions of, and perceive things; habits, which are
durable lights and notions, which we may use when we will. Some reckon up eight
kinds of them, sense, experience, intelligence, faith, suspicion, error,
opinion, science; to which are added art, prudency, wisdom: as also
\authorfootnote{1013}synteresis, \li{dictamen rationis}, conscience; so that in
all there be fourteen species of the understanding, of which some are innate,
as the three last mentioned; the other are gotten by doctrine, learning, and
use. Plato will have all to be innate: \Aristotle{} reckons up but five
intellectual habits; two practical, as prudency, whose end is to practise; to
fabricate; wisdom to comprehend the use and experiments of all notions and
habits whatsoever. Which division of \Aristotle{} (if it be considered aright) is
all one with the precedent; for three being innate, and five acquisite, the
rest are improper, imperfect, and in a more strict examination excluded. Of all
these I should more amply dilate, but my subject will not permit. Three of them
I will only point at, as more necessary to my following discourse.

Synteresis, or the purer part of the conscience, is an innate habit, and doth
signify "a conversation of the knowledge of the law of God and Nature, to know
good or evil." And (as our divines hold) it is rather in the understanding than
in the will. This makes the major proposition in a practical syllogism. The
\li{dictamen rationis} is that which doth admonish us to do good or evil, and
is the minor in the syllogism. The conscience is that which approves good or
evil, justifying or condemning our actions, and is the conclusion of the
syllogism: as in that familiar example of Regulus the Roman, taken prisoner by
the Carthaginians, and suffered to go to Rome, on that condition he should
return again, or pay so much for his ransom. The synteresis proposeth the
question; his word, oath, promise, is to be religiously kept, although to his
enemy, and that by the law of nature. \authorfootnote{1014}"Do not that to
another which thou wouldst not have done to thyself." Dictamen applies it to
him, and dictates this or the like: Regulus, thou wouldst not another man
should falsify his oath, or break promise with thee: conscience concludes,
therefore, Regulus, thou dost well to perform thy promise, and oughtest to keep
thine oath. More of this in \hyperref[ch:religious-melancholy]{Religious
Melancholy}.

%SECT. I. MEMB. II. SUBSECT. XI.-_Of the Will_.
\section{Of the Will.}

\lettrine{W}{ill} is the other power of the rational soul,
\authorfootnote{1015}"which covets or avoids such things as have been before
judged and apprehended by the understanding." If good, it approves; if evil, it
abhors it: so that his object is either good or evil. \Aristotle{} calls this our
rational appetite; for as, in the sensitive, we are moved to good or bad by our
appetite, ruled and directed by sense; so in this we are carried by reason.
Besides, the sensitive appetite hath a particular object, good or bad; this an
universal, immaterial: that respects only things delectable and pleasant; this
honest. Again, they differ in liberty. The sensual appetite seeing an object,
if it be a convenient good, cannot but desire it; if evil, avoid it: but this
is free in his essence, \authorfootnote{1016}"much now depraved, obscured, and
fallen from his first perfection; yet in some of his operations still free," as
to go, walk, move at his pleasure, and to choose whether it will do or not do,
steal or not steal. Otherwise, in vain were laws, deliberations, exhortations,
counsels, precepts, rewards, promises, threats and punishments: and God should
be the author of sin. But in \authorfootnote{1017}spiritual things we will no
good, prone to evil (except we be regenerate, and led by the Spirit), we are
egged on by our natural concupiscence, and there is \textgreek{ἀταξία}, a
confusion in our powers, \authorfootnote{1018}"our whole will is averse from
God and his law," not in natural things only, as to eat and drink, lust, to
which we are led headlong by our temperature and inordinate appetite,

\translatedverse{%
\begin{latin}
\begin{verse}
Nec nos obniti contra, nec tendere tantum\\*
Sufficimus,--\\!
\end{verse}
\end{latin}}{% \authorlatintrans{1019.5}
\begin{verse}%
We are neither able to contend against them,\\*
nor only to make way\\!
\end{verse}}{%
\attrib{\getauthornote{1019}}}

we cannot resist, our concupiscence is originally bad, our heart evil, the seat
of our affections captivates and enforceth our will. So that in voluntary
things we are averse from God and goodness, bad by nature, by
\authorfootnote{1020}ignorance worse, by art, discipline, custom, we get many
bad habits: suffering them to domineer and tyrannise over us; and the devil is
still ready at hand with his evil suggestions, to tempt our depraved will to
some ill-disposed action, to precipitate us to destruction, except our will be
swayed and counterpoised again with some divine precepts, and good motions of
the spirit, which many times restrain, hinder and check us, when we are in the
full career of our dissolute courses. So David corrected himself, when he had
Saul at a vantage. Revenge and malice were as two violent oppugners on the one
side; but honesty, religion, fear of God, withheld him on the other.

The actions of the will are \li{velle} and \li{nolle}, to will and nill: which
two words comprehend all, and they are good or bad, accordingly as they are
directed, and some of them freely performed by himself; although the stoics
absolutely deny it, and will have all things inevitably done by destiny,
imposing a fatal necessity upon us, which we may not resist; yet we say that
our will is free in respect of us, and things contingent, howsoever in respect
of God's determinate counsel, they are inevitable and necessary. Some other
actions of the will are performed by the inferior powers, which obey him, as
the sensitive and moving appetite; as to open our eyes, to go hither and
thither, not to touch a book, to speak fair or foul: but this appetite is many
times rebellious in us, and will not be contained within the lists of sobriety
and temperance. It was (as I said) once well agreeing with reason, and there
was an excellent consent and harmony between them, but that is now dissolved,
they often jar, reason is overborne by passion: \li{Fertur equis auriga, nec
audit currus habenas,} as so many wild horses run away with a chariot, and will
not be curbed. We know many times what is good, but will not do it, as she
said,

\begin{latin}
\begin{verse}
Trahit invitum nova vis, aliudque cupido,\\*
Mens aliud suadet,------\\!
\end{verse}
\end{latin}
\attrib{\getauthornote{1021}}

Lust counsels one thing, reason another, there is a new reluctancy in men.
\authorfootnote{1022}\li{Odi, nec possum, cupiens non esse, quod odi}. We
cannot resist, but as Phaedra confessed to her nurse,
\authorfootnote{1023}\li{quae loqueris, vera sunt, sed furor suggerit sequi
pejora}: she said well and true, she did acknowledge it, but headstrong passion
and fury made her to do that which was opposite. So David knew the filthiness
of his fact, what a loathsome, foul, crying sin adultery was, yet
notwithstanding he would commit murder, and take away another man's wife,
enforced against reason, religion, to follow his appetite.

Those natural and vegetal powers are not commanded by will at all; for "who can
add one cubit to his stature?" These other may, but are not: and thence come
all those headstrong passions, violent perturbations of the mind; and many
times vicious habits, customs, feral diseases; because we give so much way to
our appetite, and follow our inclination, like so many beasts. The principal
habits are two in number, virtue and vice, whose peculiar definitions,
descriptions, differences, and kinds, are handled at large in the ethics, and
are, indeed, the subject of moral philosophy.

%\chapter{ MEMB. III.}
%SECT. I. MEMB. III.
%SECT. I. MEMB. III. SUBSECT. I.-_Definition of Melancholy, Name, Difference_.
\section{Definition of Melancholy, Name, Difference.}\label{sec:definition}
\lettrine{H}{aving} thus briefly anatomised the body and soul of man, as a
preparative to the rest; I may now freely proceed to treat of my intended
object, to most men's capacity; and after many ambages, perspicuously define
what this melancholy is, show his name and differences. The name is imposed
from the matter, and disease denominated from the material cause: as Bruel
observes, \textgreek{Μελανχολία} quasi \textgreek{Μελαιναχόλη}, from black
choler. And whether it be a cause or an effect, a disease or symptom, let
Donatus Altomarus and Salvianus decide; I will not contend about it. It hath
several descriptions, notations, and definitions.
\authorfootnote{1024}Fracastorius, in his second book of intellect, calls those
melancholy, "whom abundance of that same depraved humour of black choler hath
so misaffected, that they become mad thence, and dote in most things, or in
all, belonging to election, will, or other manifest operations of the
understanding." \authorfootnote{1025}Melanelius out of Galen, Ruffus, Aetius,
describe it to be "a bad and peevish disease, which makes men degenerate into
beasts:" Galen, "a privation or infection of the middle cell of the head,
\etc{}" defining it from the part affected, which \authorfootnote{1026}Hercules
de Saxonia approves, \bookcite{\textlatin{lib. 1. cap. 16.}} calling it "a
depravation of the principal function:" Fuschius, \bookcite{\textlatin{lib. 1.
cap. 23.}} Arnoldus \bookcite{\textlatin{Breviar. lib. 1. cap. 18.}}
Guianerius, and others: "By reason of black choler," Paulus adds. Halyabbas
simply calls it a "commotion of the mind." Aretaeus, \authorfootnote{1027}"a
perpetual anguish of the soul, fastened on one thing, without an ague;" which
definition of his, Mercurialis \bookcite{\textlatin{de affect. cap. lib. 1.
cap. 10.}} taxeth: but Aelianus Montaltus defends, \bookcite{\textlatin{lib. de
morb. cap. 1. de Melan.}} for sufficient and good. The common sort define it to
be "a kind of dotage without a fever, having for his ordinary companions, fear
and sadness, without any apparent occasion." So doth Laurentius,
\bookcite{\textlatin{cap. 4.}} Piso. \bookcite{\textlatin{lib. 1. cap. 43.}}
Donatus Altomarus, \bookcite{\textlatin{cap. 7. art. medic}}. Jacchinus,
\bookcite{\textlatin{in com. in lib. 9. Rhasis ad Almansor, cap. 15.}}
Valesius, \bookcite{\textlatin{exerc. 17.}} Fuschius,
\bookcite{\textlatin{institut. 3. sec. 1. c. 11.}} \etc{} which common
definition, howsoever approved by most, \authorfootnote{1028}Hercules de
Saxonia will not allow of, nor David Crucius, \bookcite{\textlatin{Theat. morb.
Herm. lib. 2. cap. 6.}} he holds it insufficient: as
\authorfootnote{1029}rather showing what it is not, than what it is: as
omitting the specific difference, the phantasy and brain: but I descend to
particulars. The \li{summum genus} is "dotage, or anguish of the mind," saith
Aretaeus; "of the principal parts," Hercules de Saxonia adds, to distinguish it
from cramp and palsy, and such diseases as belong to the outward sense and
motions [depraved] \authorfootnote{1030}to distinguish it from folly and
madness (which Montaltus makes \li{angor animi}, to separate) in which those
functions are not depraved, but rather abolished; [without an ague] is added by
all, to sever it from frenzy, and that melancholy which is in a pestilent
fever. (Fear and sorrow) make it differ from madness: [without a cause] is
lastly inserted, to specify it from all other ordinary passions of [fear and
sorrow.] We properly call that dotage, as \authorfootnote{1031}Laurentius
interprets it, "when some one principal faculty of the mind, as imagination, or
reason, is corrupted, as all melancholy persons have." It is without a fever,
because the humour is most part cold and dry, contrary to putrefaction. Fear
and sorrow are the true characters and inseparable companions of most
melancholy, not all, as Her. de Saxonia, \bookcite{\textlatin{Tract. de
posthumo de Melancholia, cap. 2.}} well excepts; for to some it is most
pleasant, as to such as laugh most part; some are bold again, and free from all
manner of fear and grief, as hereafter shall be declared.

%SECT. I. MEMB. III. SUBSECT. II.-_Of the part affected. Affection. Parties affected_.
\section{Of the part affected. Affection. Parties affected.}\label{sec:parts-affected}

\lettrine{S}{ome} difference I find amongst writers, about the principal part
affected in this disease, whether it be the brain, or heart, or some other
member. Most are of opinion that it is the brain: for being a kind of dotage,
it cannot otherwise be but that the brain must be affected, as a similar part,
be it by \authorfootnote{1032}consent or essence, not in his ventricles, or any
obstructions in them, for then it would be an apoplexy, or epilepsy, as
\authorfootnote{1033}Laurentius well observes, but in a cold, dry
distemperature of it in his substance, which is corrupt and become too cold, or
too dry, or else too hot, as in madmen, and such as are inclined to it: and
this \authorfootnote{1034}Hippocrates confirms, Galen, the Arabians, and most
of our new writers. Marcus de Oddis (in a consultation of his, quoted by
\authorfootnote{1035}Hildesheim) and five others there cited are of the
contrary part; because fear and sorrow, which are passions, be seated in the
heart. But this objection is sufficiently answered by
\authorfootnote{1036}Montaltus, who doth not deny that the heart is affected
(as \authorfootnote{1037}Melanelius proves out of Galen) by reason of his
vicinity, and so is the midriff and many other parts. They do \li{compati}, and
have a fellow feeling by the law of nature: but forasmuch as this malady is
caused by precedent imagination, with the appetite, to whom spirits obey, and
are subject to those principal parts, the brain must needs primarily be
misaffected, as the seat of reason; and then the heart, as the seat of
affection. \authorfootnote{1038}Capivaccius and Mercurialis have copiously
discussed this question, and both conclude the subject is the inner brain, and
from thence it is communicated to the heart and other inferior parts, which
sympathise and are much troubled, especially when it comes by consent, and is
caused by reason of the stomach, or \li{mirach}, as the Arabians term it, whole
body, liver, or \authorfootnote{1039}spleen, which are seldom free, pylorus,
mesaraic veins, \etc{} For our body is like a clock, if one wheel be amiss, all
the rest are disordered; the whole fabric suffers: with such admirable art and
harmony is a man composed, such excellent proportion, as Ludovicus Vives in his
Fable of Man hath elegantly declared.

As many doubts almost arise about the \authorfootnote{1040}affection, whether
it be imagination or reason alone, or both, Hercules de Saxonia proves it out
of Galen, Aetius, and Altomarus, that the sole fault is in
\authorfootnote{1041}imagination. Bruel is of the same mind: Montaltus in his
\bookcite{\textlatin{2 cap.}} of Melancholy confutes this tenet of theirs, and
illustrates the contrary by many examples: as of him that thought himself a
shellfish, of a nun, and of a desperate monk that would not be persuaded but
that he was damned; reason was in fault as well as imagination, which did not
correct this error: they make away themselves oftentimes, and suppose many
absurd and ridiculous things. Why doth not reason detect the fallacy, settle
and persuade, if she be free? \authorfootnote{1042}\Avicenna{} therefore holds
both corrupt, to whom most Arabians subscribe. The same is maintained by
\authorfootnote{1043}Areteus, \authorfootnote{1044}Gorgonius, Guianerius,
\etc{} To end the controversy, no man doubts of imagination, but that it is
hurt and misaffected here; for the other I determine with
\authorfootnote{1045}Albertinus Bottonus, a doctor of Padua, that it is first
in "imagination, and afterwards in reason; if the disease be inveterate, or as
it is more or less of continuance;" but by accident, as
\authorfootnote{1046}Herc. de Saxonia adds; "faith, opinion, discourse,
ratiocination, are all accidentally depraved by the default of imagination."

\subsection{Parties affected.}
To the part affected, I may here add the parties, which shall be more
opportunely spoken of elsewhere, now only signified. Such as have the moon,
Saturn, Mercury misaffected in their genitures, such as live in over cold or
over hot climes: such as are born of melancholy parents; as offend in those six
non-natural things, are black, or of a high sanguine complexion,
\authorfootnote{1047}that have little heads, that have a hot heart, moist
brain, hot liver and cold stomach, have been long sick: such as are solitary by
nature, great students, given to much contemplation, lead a life out of action,
are most subject to melancholy. Of sexes both, but men more often; yet
\authorfootnote{1048}women misaffected are far more violent, and grievously
troubled. Of seasons of the year, the autumn is most melancholy. Of peculiar
times: old age, from which natural melancholy is almost an inseparable
accident; but this artificial malady is more frequent in such as are of a
\authorfootnote{1049}middle age. Some assign 40 years, Gariopontus 30. Jubertus
excepts neither young nor old from this adventitious. Daniel Sennertus involves
all of all sorts, out of common experience, \authorfootnote{1050}\li{in omnibus
omnino corporibus cujuscunque constitutionis dominatar}. Aetius and Aretius
\authorfootnote{1051}ascribe into the number "not only
\authorfootnote{1052}discontented, passionate, and miserable persons, swarthy,
black; but such as are most merry and pleasant, scoffers, and high coloured."
"Generally," saith Rhasis, \authorfootnote{1053}"the finest wits and most
generous spirits, are before other obnoxious to it;" I cannot except any
complexion, any condition, sex, or age, but \authorfootnote{1054}fools and
stoics, which, according to \authorfootnote{1055}Synesius, are never troubled
with any manner of passion, but as Anacreon's \li{cicada, sine sanguine et
dolore; similes fere diis sunt}. Erasmus vindicates fools from this melancholy
catalogue, because they have most part moist brains and light hearts;
\authorfootnote{1056}they are free from ambition, envy, shame and fear; they
are neither troubled in conscience, nor macerated with cares, to which our
whole life is most subject.

%SECT. I. MEMB. III. SUBSECT. III.-_Of the Matter of Melancholy_.
\section{Of the Matter of Melancholy.}\label{sec:matter-of-melancholy}

\lettrine{O}{f} the matter of melancholy, there is much question betwixt Avicen
and Galen, as you may read in \authorfootnote{1057}Cardan's Contradictions,
\authorfootnote{1058}Valesius' Controversies, Montanus, Prosper Calenus,
Capivaccius, \authorfootnote{1059}Bright, \authorfootnote{1060}Ficinus, that
have written either whole tracts, or copiously of it, in their several
treatises of this subject. \authorfootnote{1061}"What this humour is, or whence
it proceeds, how it is engendered in the body, neither Galen, nor any old
writer hath sufficiently discussed," as Jacchinus thinks: the Neoterics cannot
agree. Montanus, in his Consultations, holds melancholy to be material or
immaterial: and so doth Arculanus: the material is one of the four humours
before mentioned, and natural. The immaterial or adventitious, acquisite,
redundant, unnatural, artificial; which \authorfootnote{1062}Hercules de
Saxonia will have reside in the spirits alone, and to proceed from a "hot,
cold, dry, moist distemperature, which, without matter, alter the brain and
functions of it." Paracelsus wholly rejects and derides this division of four
humours and complexions, but our Galenists generally approve of it, subscribing
to this opinion of Montanus.

This material melancholy is either simple or mixed; offending in quantity or
quality, varying according to his place, where it settleth, as brain, spleen,
mesaraic veins, heart, womb, and stomach; or differing according to the mixture
of those natural humours amongst themselves, or four unnatural adust humours,
as they are diversely tempered and mingled. If natural melancholy abound in the
body, which is cold and dry, "so that it be more \authorfootnote{1063}than the
body is well able to bear, it must needs be distempered," saith Faventius, "and
diseased;" and so the other, if it be depraved, whether it arise from that
other melancholy of choler adust, or from blood, produceth the like effects,
and is, as Montaltus contends, if it come by adustion of humours, most part hot
and dry. Some difference I find, whether this melancholy matter may be
engendered of all four humours, about the colour and temper of it. Galen holds
it may be engendered of three alone, excluding phlegm, or pituita, whose true
assertion \authorfootnote{1064}Valesius and Menardus stiffly maintain, and so
doth \authorfootnote{1065}Fuschius, Montaltus, \authorfootnote{1066}Montanus.
How (say they) can white become black? But Hercules de Saxonia,
\bookcite{\textlatin{lib. post. de mela. c. 8}}, and
\authorfootnote{1067}Cardan are of the opposite part (it may be engendered of
phlegm, \li{etsi raro contingat}, though it seldom come to pass), so is
\authorfootnote{1068}Guianerius and Laurentius, \bookcite{\textlatin{c. 1.}}
with Melanct. in his book \bookcite{\textlatin{de Anima}}, and Chap. of
Humours; he calls it \li{asininam}, dull, swinish melancholy, and saith that he
was an eyewitness of it: so is \authorfootnote{1069}Wecker. From melancholy
adust ariseth one kind; from choler another, which is most brutish; another
from phlegm, which is dull; and the last from blood, which is best. Of these
some are cold and dry, others hot and dry, \authorfootnote{1070}varying
according to their mixtures, as they are intended, and remitted. And indeed as
Rodericus a Fons. \bookcite{\textlatin{cons. 12. l. 1.}} determines, ichors,
and those serous matters being thickened become phlegm, and phlegm degenerates
into choler, choler adust becomes \li{aeruginosa melancholia}, as vinegar out
of purest wine putrified or by exhalation of purer spirits is so made, and
becomes sour and sharp; and from the sharpness of this humour proceeds much
waking, troublesome thoughts and dreams, \etc{} so that I conclude as before.
If the humour be cold, it is, saith \authorfootnote{1071}Faventinus, "a cause
of dotage, and produceth milder symptoms: if hot, they are rash, raving mad, or
inclining to it." If the brain be hot, the animal spirits are hot; much madness
follows, with violent actions: if cold, fatuity and sottishness,
\authorfootnote{1072}Capivaccius. \authorfootnote{1073}"The colour of this
mixture varies likewise according to the mixture, be it hot or cold; 'tis
sometimes black, sometimes not," Altomarus. The same
\authorfootnote{1074}Melanelius proves out of Galen; and Hippocrates in his
Book of Melancholy (if at least it be his), giving instance in a burning coal,
"which when it is hot, shines; when it is cold, looks black; and so doth the
humour." This diversity of melancholy matter produceth diversity of effects. If
it be within the \authorfootnote{1075}body, and not putrified, it causeth black
jaundice; if putrified, a quartan ague; if it break out to the skin, leprosy;
if to parts, several maladies, as scurvy, \etc{} If it trouble the mind; as it
is diversely mixed, it produceth several kinds of madness and dotage: of which
in their place.

%SECT. I. MEMB. III. SUBSECT. IV.-_Of the species or kinds of Melancholy_.
\section{Of the species or kinds of Melancholy.}

\lettrine{W}{hen} the matter is divers and confused, how should it otherwise
be, but that the species should be divers and confused? Many new and old
writers have spoken confusedly of it, confounding melancholy and madness, as
\authorfootnote{1076}Heurnius, Guianerius, Gordonius, Salustius Salvianus,
Jason Pratensis, Savanarola, that will have madness no other than melancholy in
extent, differing (as I have said) in degrees. Some make two distinct species,
as Ruffus Ephesius, an old writer, Constantinus Africanus, Aretaeus,
\authorfootnote{1077}Aurelianus, \authorfootnote{1078}Paulus Aegineta: others
acknowledge a multitude of kinds, and leave them indefinite, as Aetius in his
\bookcite{\textlatin{Tetrabiblos}}, \authorfootnote{1079}\Avicenna{},
\bookcite{\textlatin{lib. 3. Fen. 1. Tract. 4. cap. 18.}} Arculanus,
\bookcite{\textlatin{cap. 16. in 9. Rasis}}. Montanus,
\bookcite{\textlatin{med. part. 1.}} \authorfootnote{1080}"If natural
melancholy be adust, it maketh one kind; if blood, another; if choler, a third,
differing from the first; and so many several opinions there are about the
kinds, as there be men themselves." \authorfootnote{1081}Hercules de Saxonia
sets down two kinds, "material and immaterial; one from spirits alone, the
other from humours and spirits." Savanarola, \bookcite{\textlatin{Rub. 11.
Tract. 6. cap. 1. de aegritud. capitis}}, will have the kinds to be infinite;
one from the mirach, called \li{myrachialis} of the Arabians; another
\li{stomachalis}, from the stomach; another from the liver, heart, womb,
haemorrhoids, \authorfootnote{1082}"one beginning, another consummate."
Melancthon seconds him, \authorfootnote{1083}"as the humour is diversely adust
and mixed, so are the species divers;" but what these men speak of species I
think ought to be understood of symptoms; and so doth
\authorfootnote{1084}Arculanus interpret himself: infinite species, \li{id
est}, symptoms; and in that sense, as Jo. Gorrheus acknowledgeth in his
medicinal definitions, the species are infinite, but they may be reduced to
three kinds by reason of their seat; head, body, and hypochrondries. This
threefold division is approved by Hippocrates in his Book of Melancholy, (if it
be his, which some suspect) by Galen, \bookcite{\textlatin{lib. 3. de loc.
affectis, cap. 6.}} by Alexander, \bookcite{\textlatin{lib. 1. cap. 16.}}
Rasis, \bookcite{\textlatin{lib. 1. Continent. Tract. 9. lib. 1. cap. 16.}}
\Avicenna{} and most of our new writers. Th. Erastus makes two kinds; one
perpetual, which is head melancholy; the other interrupt, which comes and goes
by fits, which he subdivides into the other two kinds, so that all comes to the
same pass. Some again make four or five kinds, with Rodericus a Castro,
\bookcite{\textlatin{de morbis mulier. lib. 2. cap. 3.}} and Lod. Mercatus, who
in his second book \bookcite{\textlatin{de mulier. affect. cap. 4.}} will have
that melancholy of nuns, widows, and more ancient maids, to be a peculiar
species of melancholy differing from the rest: some will reduce enthusiasts,
ecstatical and demoniacal persons to this rank, adding
\authorfootnote{1085}love melancholy to the first, and lycanthropia. The most
received division is into three kinds. The first proceeds from the sole fault
of the brain, and is called head melancholy; the second sympathetically
proceeds from the whole body, when the whole temperature is melancholy: the
third ariseth from the bowels, liver, spleen, or membrane, called
\li{mesenterium}, named hypochondriacal or windy melancholy, which
\authorfootnote{1086}Laurentius subdivides into three parts, from those three
members, hepatic, splenetic, mesaraic. Love melancholy, which \Avicenna{} calls
\li{ilishi}: and Lycanthropia, which he calls \li{cucubuthe}, are commonly
included in head melancholy; but of this last, which Gerardus de Solo calls
\li{amoreus}, and most knight melancholy, with that of religious melancholy,
\li{virginum et viduarum}, maintained by Rod. a Castro and Mercatus, and the
other kinds of love melancholy, I will speak of apart by themselves in my third
partition. The three precedent species are the subject of my present discourse,
which I will anatomise and treat of through all their causes, symptoms, cures,
together and apart; that every man that is in any measure affected with this
malady, may know how to examine it in himself, and apply remedies unto it.

It is a hard matter, I confess, to distinguish these three species one from the
other, to express their several causes, symptoms, cures, being that they are so
often confounded amongst themselves, having such affinity, that they can scarce
be discerned by the most accurate physicians; and so often intermixed with
other diseases, that the best experienced have been plunged. Montanus
\bookcite{\textlatin{consil. 26}}, names a patient that had this disease of
melancholy and caninus appetitus both together; and
\bookcite{\textlatin{consil. 23}}, with vertigo, \authorfootnote{1087}Julius
Caesar Claudinus with stone, gout, jaundice. Trincavellius with an ague,
jaundice, caninus appetitus, \etc{} \authorfootnote{1088}Paulus Regoline, a
great doctor in his time, consulted in this case, was so confounded with a
confusion of symptoms, that he knew not to what kind of melancholy to refer it.
\authorfootnote{1089}Trincavellius, Fallopius, and Francanzanus, famous doctors
in Italy, all three conferred with about one party, at the same time, gave
three different opinions. And in another place, Trincavellius being demanded
what he thought of a melancholy young man to whom he was sent for, ingenuously
confessed that he was indeed melancholy, but he knew not to what kind to reduce
it. In his seventeenth consultation there is the like disagreement about a
melancholy monk. Those symptoms, which others ascribe to misaffected parts and
humours, \authorfootnote{1090}Herc. de Saxonia attributes wholly to distempered
spirits, and those immaterial, as I have said. Sometimes they cannot well
discern this disease from others. In Reinerus Solenander's counsels,
(\bookcite{\textlatin{Sect, consil. 5}},) he and Dr. Brande both agreed, that
the patient's disease was hypochondriacal melancholy. Dr. Matholdus said it was
asthma, and nothing else. \authorfootnote{1091}Solenander and Guarionius,
lately sent for to the melancholy Duke of Cleve, with others, could not define
what species it was, or agree amongst themselves. The species are so
confounded, as in Caesar Claudinus his forty-fourth consultation for a Polonian
Count, in his judgment \authorfootnote{1092}"he laboured of head melancholy,
and that which proceeds from the whole temperature both at once." I could give
instance of some that have had all three kinds \li{semel et simul}, and some
successively. So that I conclude of our melancholy species, as
\authorfootnote{1093}many politicians do of their pure forms of commonwealths,
monarchies, aristocracies, democracies, are most famous in contemplation, but
in practice they are temperate and usually mixed, (so
\authorfootnote{1094}Polybius informeth us) as the Lacedaemonian, the Roman of
old, German now, and many others. What physicians say of distinct species in
their books it much matters not, since that in their patients' bodies they are
commonly mixed. In such obscurity, therefore, variety and confused mixture of
symptoms, causes, how difficult a thing is it to treat of several kinds apart;
to make any certainty or distinction among so many casualties, distractions,
when seldom two men shall be like effected \li{per omnia}? 'Tis hard, I
confess, yet nevertheless I will adventure through the midst of these
perplexities, and, led by the clue or thread of the best writers, extricate
myself out of a labyrinth of doubts and errors, and so proceed to the causes.


%SECT. II. MEMB. I.

%SECT. II. MEMB. I. SUBSECT. I.-_Causes of Melancholy. God a cause._
\section{Causes of Melancholy. God a cause.}\label{sec:causes-of-melancholy}

\lettrine[ante={\large{}"}]{I}{t} is in vain to speak of cures, or think of remedies,
until such time as we have considered of the causes," so
\authorfootnote{1095}Galen prescribes Glauco: and the common experience of
others confirms that those cures must be imperfect, lame, and to no purpose,
wherein the causes have not first been searched, as
\authorfootnote{1096}Prosper Calenius well observes in his tract
\bookcite{\textlatin{de atra bile}} to Cardinal Caesius. Insomuch that
\authorfootnote{1097}"Fernelius puts a kind of necessity in the knowledge of
the causes, and without which it is impossible to cure or prevent any manner of
disease." Empirics may ease, and sometimes help, but not thoroughly root out;
\lit{sublata causa tollitur effectus}{if the cause be removed, the effect is
likewise vanquished} as the saying is. It is a most difficult thing (I confess)
to be able to discern these causes whence they are, and in such
\authorfootnote{1098}variety to say what the beginning was.
\authorfootnote{1099}He is happy that can perform it aright. I will adventure
to guess as near as I can, and rip them all up, from the first to the last,
general and particular, to every species, that so they may the better be
described.

General causes, are either supernatural, or natural. "Supernatural are from God
and his angels, or by God's permission from the devil" and his ministers. That
God himself is a cause for the punishment of sin, and satisfaction of his
justice, many examples and testimonies of holy Scriptures make evident unto us,
\biblecite{Ps. cvii, 17}. "Foolish men are plagued for their offence, and by
reason of their wickedness." Gehazi was stricken with leprosy, \biblecite{2 Reg.
\rn{v.} 27}. Jehoram with dysentery and flux, and great diseases of the bowels,
\biblecite{2 Chron. \rn{xxi.} 15}. David plagued for numbering his people,
\biblecite{1 Par. 21}. Sodom and Gomorrah swallowed up. And this disease is
peculiarly specified, \biblecite{Psalm \rn{cxxvii.} 12}. "He brought down their
heart through heaviness." \biblecite{Deut. \rn{xxviii.} 28}. "He struck them with
madness, blindness, and astonishment of heart." \authorfootnote{1100}"An evil
spirit was sent by the Lord upon Saul, to vex him."
\authorfootnote{1101}Nebuchadnezzar did eat grass like an ox, and his "heart
was made like the beasts of the field." Heathen stories are full of such
punishments. Lycurgus, because he cut down the vines in the country, was by
Bacchus driven into madness: so was Pentheus and his mother Agave for
neglecting their sacrifice. \authorfootnote{1102}Censor Fulvius ran mad for
untiling Juno's temple, to cover a new one of his own, which he had dedicated
to Fortune, \authorfootnote{1103}"and was confounded to death with grief and
sorrow of heart." When Xerxes would have spoiled \authorfootnote{1104}Apollo's
temple at Delphos of those infinite riches it possessed, a terrible thunder
came from heaven and struck four thousand men dead, the rest ran mad.
\authorfootnote{1105}A little after, the like happened to Brennus, lightning,
thunder, earthquakes, upon such a sacrilegious occasion. If we may believe our
pontifical writers, they will relate unto us many strange and prodigious
punishments in this kind, inflicted by their saints. How
\authorfootnote{1106}Clodoveus, sometime king of France, the son of Dagobert,
lost his wits for uncovering the body of St. Denis: and how a
\authorfootnote{1107}sacrilegious Frenchman, that would have stolen a silver
image of St. John, at Birgburge, became frantic on a sudden, raging, and
tyrannising over his own flesh: of a \authorfootnote{1108}Lord of Rhadnor, that
coming from hunting late at night, put his dogs into St. Avan's church, (Llan
Avan they called it) and rising betimes next morning, as hunters use to do,
found all his dogs mad, himself being suddenly strucken blind. Of Tyridates an
\authorfootnote{1109}Armenian king, for violating some holy nuns, that was
punished in like sort, with loss of his wits. But poets and papists may go
together for fabulous tales; let them free their own credits: howsoever they
feign of their Nemesis, and of their saints, or by the devil's means may be
deluded; we find it true, that \lit{ultor a tergo Deus}{He is God the
avenger},\authorfootnote{1110} as David styles him; and that it is our crying
sins that pull this and many other maladies on our own heads. That he can by
his angels, which are his ministers, strike and heal (saith
\authorfootnote{1111}Dionysius) whom he will; that he can plague us by his
creatures, sun, moon, and stars, which he useth as his instruments, as a
husbandman (saith Zanchius) doth a hatchet: hail, snow, winds, \etc{}
\authorfootnote{1112}\li{Et conjurati veniunt in classica venti}: as in
Joshua's time, as in Pharaoh's reign in Egypt; they are but as so many
executioners of his justice. He can make the proudest spirits stoop, and cry
out with Julian the Apostate, \li{Vicisti Galilaee}: or with Apollo's priest in
\authorfootnote{1113}\Chrysostom{}, \li{O coelum! o terra!} \lit{unde hostis
hic?}{What an enemy is this?} And pray with David, acknowledging his power, "I
am weakened and sore broken, I roar for the grief of mine heart, mine heart
panteth," \etc{} \biblecite{Psalm \rn{xxxviii.} 8}. "O Lord, rebuke me not in
thine anger, neither chastise me in thy wrath," \biblecite{Psalm \rn{xxxviii.}
1}. "Make me to hear joy and gladness, that the bones which thou hast broken,
may rejoice," \biblecite{Psalm \rn{li.} 8. and verse 12}. "Restore to me the joy
of thy salvation, and stablish me with thy free spirit." For these causes
belike \authorfootnote{1114}Hippocrates would have a physician take special
notice whether the disease come not from a divine supernatural cause, or
whether it follow the course of nature. But this is farther discussed by Fran.
Valesius, \bookcite{\textlatin{de sacr. philos. cap. 8.}}
\authorfootnote{1115}Fernelius, and \authorfootnote{1116}J. Caesar Claudinus,
to whom I refer you, how this place of Hippocrates is to be understood.
Paracelsus is of opinion, that such spiritual diseases (for so he calls them)
are spiritually to be cured, and not otherwise. Ordinary means in such cases
will not avail: \li{Non est reluctandum cum Deo} (we must not struggle with
God.) When that monster-taming Hercules overcame all in the Olympics, Jupiter
at last in an unknown shape wrestled with him; the victory was uncertain, till
at length Jupiter descried himself, and Hercules yielded. No striving with
supreme powers. \li{Nil juvat immensos Cratero promittere montes}, physicians
and physic can do no good, \authorfootnote{1117}"we must submit ourselves unto
the mighty hand of God," acknowledge our offences, call to him for mercy. If he
strike us \li{una eademque manus vulnus opemque feret}, as it is with them that
are wounded with the spear of Achilles, he alone must help; otherwise our
diseases are incurable, and we not to be relieved.

\cleartoleftpage{}
\begin{figure}[p!]
  \begingroup
  \centering
  \includegraphics[keepaspectratio,width=\textwidth]{The-Right-Hand-of-God-Protecting-the-Faithful-against-the-Demons-small.jpg}
  \captionart{TheRightHandOfGodAgainstDemons}
  \label{fig:therighthandofgod}
\end{figure}

% Force float here
\clearpage{}
\thispagestyle{titleontop}

%SECT. II. MEMB. I. SUBSECT. II.-_A Digression of the nature of Spirits, bad Angels, or Devils, and how they cause Melancholy_.
\section[Nature of bad Angels, or Devils]{A Digression of the nature of Spirits, bad Angels, or Devils, and how they cause Melancholy.}

\lettrine{H}{ow} How far the power of spirits and devils doth extend, and
whether they can cause this, or any other disease, is a serious question, and
worthy to be considered: for the better understanding of which, I will make a
brief digression of the nature of spirits. And although the question be very
obscure, according to \authorfootnote{1118}Postellus, "full of controversy and
ambiguity," beyond the reach of human capacity, \lit{fateor excedere vires
intentionis meae}{I confess I am not able to understand it}, saith
\authorfootnote{1119}\Austin{}, \li{finitum de infinito non potest statuere}, we
can sooner determine with \Tully{}, \bookcite{\textlatin{de nat. deorum}},
\li{quid non sint, quam quid sint}, our subtle schoolmen, Cardans, Scaligers,
profound Thomists, Fracastoriana and Ferneliana \li{acies}, are weak, dry,
obscure, defective in these mysteries, and all our quickest wits, as an owl's
eyes at the sun's light, wax dull, and are not sufficient to apprehend them;
yet, as in the rest, I will adventure to say something to this point. In former
times, as we read, \biblecite{Acts \rn{xxiii.}}, the Sadducees denied that there
were any such spirits, devils, or angels. So did Galen the physician, the
Peripatetics, even \Aristotle{} himself, as Pomponatius stoutly maintains, and
\Scaliger{} in some sort grants. Though Dandinus the Jesuit,
\bookcite{\textlatin{com. in lib. 2. de anima}}, stiffly denies it;
\li{substantiae separatae} and intelligences, are the same which Christians
call angels, and Platonists devils, for they name all the spirits,
\li{daemones}, be they good or bad angels, as Julius Pollux
\bookcite{\textlatin{Onomasticon, lib. 1. cap. 1.}} observes. Epicures and
atheists are of the same mind in general, because they never saw them. Plato,
Plotinus, Porphyrius, Jamblichus, Proclus, insisting in the steps of
Trismegistus, Pythagoras and Socrates, make no doubt of it: nor Stoics, but
that there are such spirits, though much erring from the truth. Concerning the
first beginning of them, the \authorfootnote{1120}Talmudists say that Adam had
a wife called Lilis, before he married Eve, and of her he begat nothing but
devils. The Turks' \authorfootnote{1121}Alcoran is altogether as absurd and
ridiculous in this point: but the Scripture informs us Christians, how Lucifer,
the chief of them, with his associates, \authorfootnote{1122}fell from heaven
for his pride and ambition; created of God, placed in heaven, and sometimes an
angel of light, now cast down into the lower aerial sublunary parts, or into
hell, "and delivered into chains of darkness (\biblecite{2 Pet. \rn{ii.} 4.}) to
be kept unto damnation."

\subsection{Nature of Devils.}

There is a foolish opinion which some hold, that they are the souls of men
departed, good and more noble were deified, the baser grovelled on the ground,
or in the lower parts, and were devils, the which with Tertullian, Porphyrius
the philosopher, M. Tyrius, \bookcite{\textlatin{ser. 27}} maintains. "These
spirits," he \authorfootnote{1123}saith, "which we call angels and devils, are
nought but souls of men departed, which either through love and pity of their
friends yet living, help and assist them, or else persecute their enemies, whom
they hated," as Dido threatened to persecute Aeneas:

\begin{latin}
\begin{verse}%
Omnibus umbra locis adero: dabis improbe poenas.\\!
\end{verse}%
\end{latin}

\begin{verse}%
My angry ghost arising from the deep,\\*
Shall haunt thee waking, and disturb thy sleep;\\*
At least my shade thy punishment shall know,\\*
And Fame shall spread the pleasing news below.\\!
\end{verse}%

They are (as others suppose) appointed by those higher powers to keep men from
their nativity, and to protect or punish them as they see cause: and are called
\li{boni et mali Genii} by the Romans. Heroes, lares, if good, lemures or
larvae if bad, by the stoics, governors of countries, men, cities, saith
\Apuleius,\authorfootnote{1124} \li{Deos appellant qui ex hominum numero juste
ac prudenter vitae curriculo gubernato, pro numine, postea ab hominibus
praediti fanis et ceremoniis vulgo admittuntur, ut in Aegypto
Osyris}\authorlatintrans{1124.5}, \etc{} \li{Praestites}, Capella calls them,
"which protected particular men as well as princes," Socrates had his
\li{Daemonium Saturninum et ignium}, which of all spirits is best, \li{ad
sublimes cogitationes animum erigentem}, as the Platonists supposed; Plotinus
his, and we Christians our assisting angel, as Andreas Victorellus, a copious
writer of this subject, Lodovicus de La-Cerda, the Jesuit, in his voluminous
tract \bookcite{\textlatin{de Angelo Custode}}, Zanchius, and some divines
think. But this absurd tenet of Tyreus, Proclus confutes at large in his book
\bookcite{\textlatin{de Anima et daemone}}.

Psellus \authorfootnote{1125}, a Christian, and sometimes tutor (saith
Cuspinian) to Michael Parapinatius, Emperor of Greece, a great observer of the
nature of devils, holds they are corporeal \authorfootnote{1126}, and have
"aerial bodies, that they are mortal, live and die," (which Martianus Capella
likewise maintains, but our Christian philosophers explode) "that they
\authorfootnote{1127}are nourished and have excrements, they feel pain if they
be hurt" (which Cardan confirms, and \Scaliger{} justly laughs him to scorn for;
\li{Si pascantur aere, cur non pugnant ob puriorem aera}? \etc{}) "or stroken:"
and if their bodies be cut, with admirable celerity they come together again.
\Austin{}, \bookcite{\textlatin{in Gen. lib. iii. lib. arbit.}}, approves as much,
\li{mutata casu corpora in deteriorem qualitatem aeris spissioris}, so doth
Hierome. \bookcite{\textlatin{Comment. in epist. ad Ephes. cap. 3}}, Origen,
Tertullian, Lactantius, and many ancient Fathers of the Church: that in their
fall their bodies were changed into a more aerial and gross substance. Bodine,
\bookcite{\textlatin{lib. 4, Theatri Naturae}} and David Crusius,
\bookcite{\textlatin{Hermeticae Philosophiae, lib. 1. cap. 4}}, by several
arguments proves angels and spirits to be corporeal: \li{quicquid continetur in
loco corporeum est; At spiritus continetur in loco, ergo. Si spiritus sunt
quanti, erunt corporei: At sunt quanti, ergo. sunt finiti, ergo.
quanti}\authorlatintrans{1128}, \etc{} Bodine \authorfootnote{1129}goes farther
yet, and will have these, \li{Animae separatae genii}, spirits, angels, devils,
and so likewise souls of men departed, if corporeal (which he most eagerly
contends) to be of some shape, and that absolutely round, like Sun and Moon,
because that is the most perfect form, \li{quae nihil habet asperitatis, nihil
angulis incisum, nihil anfractibus involutem, nihil eminens, sed inter corpora
perfecta est perfectissimum}\authorfootnote{1130}; therefore all spirits are
corporeal he concludes, and in their proper shapes round. That they can assume
other aerial bodies, all manner of shapes at their pleasures, appear in what
likeness they will themselves, that they are most swift in motion, can pass
many miles in an instant, and so likewise \authorfootnote{1131}transform bodies
of others into what shape they please, and with admirable celerity remove them
from place to place; (as the Angel did Habakkuk to Daniel, and as Philip the
deacon was carried away by the Spirit, when he had baptised the eunuch; so did
Pythagoras and \Apollonius{} remove themselves and others, with many such feats)
that they can represent castles in the air, palaces, armies, spectrums,
prodigies, and such strange objects to mortal men's eyes,
\authorfootnote{1132}cause smells, savours, \etc{}, deceive all the senses;
most writers of this subject credibly believe; and that they can foretell
future events, and do many strange miracles. Juno's image spake to Camillus,
and Fortune's statue to the Roman matrons, with many such. Zanchius, Bodine,
Spondanus, and others, are of opinion that they cause a true metamorphosis, as
Nebuchadnezzar was really translated into a beast, Lot's wife into a pillar of
salt; Ulysses' companions into hogs and dogs, by Circe's charms; turn
themselves and others, as they do witches into cats, dogs, hares, crows, \etc{}
Strozzius Cicogna hath many examples, \bookcite{\textlatin{lib. iii. omnif.
mag. cap. 4 and 5}}, which he there confutes, as \Austin{} likewise doth,
\bookcite{\textlatin{de civ. Dei lib. xviii}}. That they can be seen when and
in what shape, and to whom they will, saith Psellus, \li{Tametsi nil tale
viderim, nec optem videre}, though he himself never saw them nor desired it;
and use sometimes carnal copulation (as elsewhere I shall
\authorfootnote{1133}prove more at large) with women and men. Many will not
believe they can be seen, and if any man shall say, swear, and stiffly
maintain, though he be discreet and wise, judicious and learned, that he hath
seen them, they account him a timorous fool, a melancholy dizzard, a weak
fellow, a dreamer, a sick or a mad man, they contemn him, laugh him to scorn,
and yet Marcus of his credit told Psellus that he had often seen them. And Leo
Suavius, a Frenchman, \bookcite{\textlatin{c. 8, in Commentar. l. 1. Paracelsi
de vita longa}}, out of some Platonists, will have the air to be as full of
them as snow falling in the skies, and that they may be seen, and withal sets
down the means how men may see them; \li{Si irreverberatus oculis sole
splendente versus caelum continuaverint obtutus}\authorlatintrans{1134},
\etc{}, and saith moreover he tried it, \li{praemissorum feci experimentum},
and it was true, that the Platonists said. Paracelsus confesseth that he saw
them divers times, and conferred with them, and so doth Alexander ab
\authorfootnote{1135}Alexandro, "that he so found it by experience, when as
before he doubted of it." Many deny it, saith Lavater, \bookcite{\textlatin{de
spectris, part 1. c. 2}}, and \bookcite{\textlatin{part 2. c. 11}}, "because
they never saw them themselves;" but as he reports at large all over his book,
especially \bookcite{\textlatin{c. 19. part 1}}, they are often seen and heard,
and familiarly converse with men, as Lod. Vives assureth us, innumerable
records, histories, and testimonies evince in all ages, times, places, and
\authorfootnote{1136}all travellers besides; in the West Indies and our
northern climes, \li{Nihil familiarius quam in agris et urbibus spiritus
videre, audire qui vetent, jubeant}, \etc{} Hieronymus
\bookcite{\textlatin{vita Pauli}}, Basil \bookcite{\textlatin{ser. 40}},
Nicephorus, Eusebius, Socrates, Sozomenus, \authorfootnote{1137}Jacobus
Boissardus in his tract \bookcite{\textlatin{de spirituum apparitionibus}},
Petrus Loyerus \bookcite{\textlatin{l. de spectris}}, Wierus
\bookcite{\textlatin{l. 1.}} have infinite variety of such examples of
apparitions of spirits, for him to read that farther doubts, to his ample
satisfaction. One alone I will briefly insert. A nobleman in Germany was sent
ambassador to the King of Sweden (for his name, the time, and such
circumstances, I refer you to Boissardus, mine \authorfootnote{1138}Author).
After he had done his business, he sailed to Livonia, on set purpose to see
those familiar spirits, which are there said to be conversant with men, and do
their drudgery works. Amongst other matters, one of them told him where his
wife was, in what room, in what clothes, what doing, and brought him a ring
from her, which at his return, \li{non sine omnium admiratione}, he found to be
true; and so believed that ever after, which before he doubted of. Cardan,
\bookcite{\textlatin{l. 19. de subtil}}, relates of his father, Facius Cardan,
that after the accustomed solemnities, \emph{An.} 1491, 13 August, he conjured
up seven devils, in Greek apparel, about forty years of age, some ruddy of
complexion, and some pale, as he thought; he asked them many questions, and
they made ready answer, that they were aerial devils, that they lived and died
as men did, save that they were far longer lived (700 or 800
\authorfootnote{1139}years); they did as much excel men in dignity as we do
juments, and were as far excelled again of those that were above them; our
\authorfootnote{1140}governors and keepers they are moreover, which
\authorfootnote{1141}Plato in Critias delivered of old, and subordinate to one
another, \li{Ut enim homo homini sic daemon daemoni dominatur}, they rule
themselves as well as us, and the spirits of the meaner sort had commonly such
offices, as we make horse-keepers, neat-herds, and the basest of us, overseers
of our cattle; and that we can no more apprehend their natures and functions,
than a horse a man's. They knew all things, but might not reveal them to men;
and ruled and domineered over us, as we do over our horses; the best kings
amongst us, and the most generous spirits, were not comparable to the basest of
them. Sometimes they did instruct men, and communicate their skill, reward and
cherish, and sometimes, again, terrify and punish, to keep them in awe, as they
thought fit, \li{Nihil magis cupientes} (saith Lysius,
\bookcite{\textlatin{Phis. Stoicorum}}) \li{quam adorationem
hominum}\authorlatintrans{1142}. The same Author, Cardan, in his
\bookcite{\textlatin{Hyperchen}}, out of the doctrine of Stoics, will have some
of these \li{genii} (for so he calls them) to be \authorfootnote{1143}desirous
of men's company, very affable and familiar with them, as dogs are; others,
again, to abhor as serpents, and care not for them. The same belike Tritemius
calls \li{Ignios et sublunares, qui nunquam demergunt ad inferiora, aut vix
ullum habent in terris commercium}: \authorfootnote{1144}"Generally they far
excel men in worth, as a man the meanest worm; though some of them are inferior
to those of their own rank in worth, as the blackguard in a prince's court, and
to men again, as some degenerate, base, rational creatures, are excelled of
brute beasts."

That they are mortal, besides these testimonies of Cardan, Martianus, \etc{},
many other divines and philosophers hold, \li{post prolixum tempus moriuntur
omnes}; The \authorfootnote{1145}Platonists, and some Rabbins, Porphyrius and
\Plutarch{}, as appears by that relation of Thamus: \authorfootnote{1146}"The
great God Pan is dead; Apollo Pythius ceased; and so the rest." St. Hierome, in
the life of Paul the Hermit, tells a story how one of them appeared to St.
Anthony in the wilderness, and told him as much.
\authorfootnote{1147}Paracelsus of our late writers stiffly maintains that they
are mortal, live and die as other creatures do. Zozimus,
\bookcite{\textlatin{l. 2}}, farther adds, that religion and policy dies and
alters with them. The \authorfootnote{1148}Gentiles' gods, he saith, were
expelled by Constantine, and together with them. \li{Imperii Romani majestas,
et fortuna interiit, et profligata est}{The fortune and majesty of the Roman
Empire decayed and vanished}, as that heathen in \authorfootnote{1149}Minutius
formerly bragged, when the Jews were overcome by the Romans, the Jew's God was
likewise captivated by that of Rome; and Rabsakeh to the Israelites, no God
should deliver them out of the hands of the Assyrians. But these paradoxes of
their power, corporeity, mortality, taking of shapes, transposing bodies, and
carnal copulations, are sufficiently confuted by Zanch. \bookcite{\textlatin{c.
10, l. 4.}} Pererius in his comment, and Tostatus questions on the 6th of Gen.
Th. Aquin., St. \Austin{}, Wierus, Th. Erastus, Delrio, \bookcite{\textlatin{tom.
2, l. 2, quaest. 29}}; Sebastian Michaelis, \bookcite{\textlatin{c. 2, de
spiritibus}}, D. Reinolds \bookcite{\textlatin{Lect. 47.}} They may deceive the
eyes of men, yet not take true bodies, or make a real metamorphosis; but as
Cicogna proves at large, they are \authorfootnote{1150}\li{Illusoriae, et
praestigiatrices transformationes}, \bookcite{\textlatin{omnif. mag. lib. 4.
cap. 4}}, mere illusions and cozenings, like that tale of \li{Pasetis obulus}
in Suidas, or that of Autolicus, Mercury's son, that dwelt in Parnassus, who
got so much treasure by cozenage and stealth. His father Mercury, because he
could leave him no wealth, taught him many fine tricks to get means,
\authorfootnote{1151}for he could drive away men's cattle, and if any pursued
him, turn them into what shapes he would, and so did mightily enrich himself,
\li{hoc astu maximam praedam est adsecutus}. This, no doubt, is as true as the
rest; yet thus much in general. Thomas, Durand, and others, grant that they
have understanding far beyond men, can probably conjecture and
\authorfootnote{1152}foretell many things; they can cause and cure most
diseases, deceive our senses; they have excellent skill in all Arts and
Sciences; and that the most illiterate devil is \li{Quovis homine
scientior}{more knowing than any man}, as \authorfootnote{1153}Cicogna
maintains out of others. They know the virtues of herbs, plants, stones,
minerals, \etc{}; of all creatures, birds, beasts, the four elements, stars,
planets, can aptly apply and make use of them as they see good; perceiving the
causes of all meteors, and the like: \li{Dant se coloribus} (as
\authorfootnote{1154}\Austin{} hath it) \li{accommodant se figuris, adhaerent
sonis, subjiciunt se odoribus, infundunt se saporibus, omnes sensus etiam ipsam
intelligentiam daemones fallunt}, they deceive all our senses, even our
understanding itself at once. \authorfootnote{1155}They can produce miraculous
alterations in the air, and most wonderful effects, conquer armies, give
victories, help, further, hurt, cross and alter human attempts and projects
(\li{Dei permissu}) as they see good themselves. \authorfootnote{1156}When
Charles the Great intended to make a channel betwixt the Rhine and the Danube,
look what his workmen did in the day, these spirits flung down in the night,
\li{Ut conatu Rex desisteret, pervicere}. Such feats can they do. But that
which Bodine, \bookcite{\textlatin{l. 4, Theat. nat.}} thinks (following Tyrius
belike, and the Platonists,) they can tell the secrets of a man's heart,
\li{aut cogitationes hominum}, is most false; his reasons are weak, and
sufficiently confuted by Zanch. \bookcite{\textlatin{lib. 4, cap. 9.}} Hierom.
\bookcite{\textlatin{lib. 2, com. in Mat. ad cap. 15}}, Athanasius
\bookcite{\textlatin{quaest. 27, ad Antiochum Principem}}, and others.

\subsection{Orders.}
As for those orders of good and bad devils, which the Platonists hold, is
altogether erroneous, and those Ethnics \li{boni et mali Genii}, are to be
exploded: these heathen writers agree not in this point among themselves, as
Dandinus notes, \li{An sint \authorfootnote{1157}mali non conveniunt}, some
will have all spirits good or bad to us by a mistake, as if an Ox or Horse
could discourse, he would say the Butcher was his enemy because he killed him,
the grazier his friend because he fed him; a hunter preserves and yet kills his
game, and is hated nevertheless of his game; \li{nec piscatorem piscis amare
potest}, \etc{} But Jamblichus, Psellus, \Plutarch{}, and most Platonists
acknowledge bad, \li{et ab eorum maleficiis cavendum}, and we should beware of
their wickedness, for they are enemies of mankind, and this Plato learned in
Egypt, that they quarrelled with Jupiter, and were driven by him down to hell.
\authorfootnote{1158}That which \authorfootnote{1159}\Apuleius, Xenophon, and
Plato contend of Socrates Daemonium, is most absurd: That which Plotinus of
his, that he had likewise \li{Deum pro Daemonio}; and that which Porphyry
concludes of them all in general, if they be neglected in their sacrifice they
are angry; nay more, as Cardan in his \bookcite{\textlatin{Hipperchen}} will,
they feed on men's souls, \li{Elementa sunt plantis elementum, animalibus
plantae, hominibus animalia, erunt et homines aliis, non autem diis, nimis enim
remota est eorum natura a nostra, quapropter daemonibus}: and so belike that we
have so many battles fought in all ages, countries, is to make them a feast,
and their sole delight: but to return to that I said before, if displeased they
fret and chafe, (for they feed belike on the souls of beasts, as we do on their
bodies) and send many plagues amongst us; but if pleased, then they do much
good; is as vain as the rest and confuted by \Austin{}, \bookcite{\textlatin{l. 9.
c. 8. de Civ. Dei}}. Euseb. \bookcite{\textlatin{l. 4. praepar. Evang. c. 6.}}
and others. Yet thus much I find, that our schoolmen and other
\authorfootnote{1160}divines make nine kinds of bad spirits, as Dionysius hath
done of angels. In the first rank are those false gods of the gentiles, which
were adored heretofore in several idols, and gave oracles at Delphos, and
elsewhere; whose prince is Beelzebub. The second rank is of liars and
equivocators, as Apollo, Pythius, and the like. The third are those vessels of
anger, inventors of all mischief; as that Theutus in Plato; Esay calls them
\authorfootnote{1161}vessels of fury; their prince is Belial. The fourth are
malicious revenging devils; and their prince is Asmodaeus. The fifth kind are
cozeners, such as belong to magicians and witches; their prince is Satan. The
sixth are those aerial devils that \authorfootnote{1162}corrupt the air and
cause plagues, thunders, fires, \etc{}; spoken of in the Apocalypse, and Paul
to the Ephesians names them the princes of the air; Meresin is their prince.
The seventh is a destroyer, captain of the furies, causing wars, tumults,
combustions, uproars, mentioned in the Apocalypse; and called Abaddon. The
eighth is that accusing or calumniating devil, whom the Greeks call
\textgreek{Διαβολος}, that drives men to despair. The ninth are those tempters
in several kinds, and their prince is Mammon. Psellus makes six kinds, yet none
above the Moon: Wierus in his \bookcite{\textlatin{Pseudo-monarchia Daemonis}},
out of an old book, makes many more divisions and subordinations, with their
several names, numbers, offices, \etc{}, but Gazaeus cited by
\authorfootnote{1163}Lipsius will have all places full of angels, spirits, and
devils, above and beneath the Moon, \authorfootnote{1164}ethereal and aerial,
which \Austin{} cites out of Varro \bookcite{\textlatin{l. 7. de Civ. Dei, c. 6.}}
"The celestial devils above, and aerial beneath," or, as some will, gods above,
Semi-dei or half gods beneath, Lares, Heroes, Genii, which climb higher, if
they lived well, as the Stoics held; but grovel on the ground as they were
baser in their lives, nearer to the earth: and are Manes, Lemures, Lamiae,
\etc{} \authorfootnote{1165}They will have no place but all full of spirits,
devils, or some other inhabitants; \li{Plenum Caelum, aer, aqua terra, et omnia
sub terra}, saith \authorfootnote{1166}Gazaeus; though Anthony Rusca in his
book \bookcite{\textlatin{de Inferno, lib. v. cap. 7.}} would confine them to
the middle region, yet they will have them everywhere. "Not so much as a
hair-breadth empty in heaven, earth, or waters, above or under the earth." The
air is not so full of flies in summer, as it is at all times of invisible
devils: this \authorfootnote{1167}Paracelsus stiffly maintains, and that they
have every one their several chaos, others will have infinite worlds, and each
world his peculiar spirits, gods, angels, and devils to govern and punish it.

\translatedverse{%
\begin{latin}
\begin{verse}
Singula nonnulli credunt quoque sidera posse\\*
Dici orbes, terramque appellant sidus opacum,\\*
Cui minimus divum praesit.------\\!
\end{verse}
\end{latin}}{%
\begin{verse}%
Some persons believe each star to be a world,\\*
and this earth an opaque star,\\*
over which the least of the gods presides.
\end{verse}}{%
\attrib{\getauthornote{1168}}}

\authorfootnote{1169}Gregorius Tholsanus makes seven kinds of ethereal spirits
or angels, according to the number of the seven planets, Saturnine, Jovial,
Martial, of which Cardan discourseth \bookcite{\textlatin{lib. 20. de subtil.}}
he calls them \li{substantias primas, Olympicos daemones Tritemius, qui
praesunt Zodiaco}, \etc{}, and will have them to be good angels above, devils
beneath the Moon, their several names and offices he there sets down, and which
Dionysius of Angels, will have several spirits for several countries, men,
offices, \etc{}, which live about them, and as so many assisting powers cause
their operations, will have in a word, innumerable, as many of them as there be
stars in the skies. \authorfootnote{1170}Marcilius Ficinus seems to second this
opinion, out of Plato, or from himself, I know not, (still ruling their
inferiors, as they do those under them again, all subordinate, and the nearest
to the earth rule us, whom we subdivide into good and bad angels, call gods or
devils, as they help or hurt us, and so adore, love or hate) but it is most
likely from Plato, for he relying wholly on Socrates, \li{quem mori potius quam
mentiri voluisse scribit}, whom he says would rather die than tell a falsehood,
out of Socrates' authority alone, made nine kinds of them: which opinion belike
Socrates took from Pythagoras, and he from Trismegistus, he from Zoroastes,
first God, second idea, 3. Intelligences, 4. Arch-Angels, 5. Angels, 6. Devils,
7. Heroes, 8. Principalities, 9. Princes: of which some were absolutely good,
as gods, some bad, some indifferent \li{inter deos et homines}, as heroes and
daemons, which ruled men, and were called genii, or as
\authorfootnote{1171}Proclus and Jamblichus will, the middle betwixt God and
men. Principalities and princes, which commanded and swayed kings and
countries; and had several places in the spheres perhaps, for as every sphere
is higher, so hath it more excellent inhabitants: which belike is that
Galilaeus a Galileo and Kepler aims at in his nuncio Syderio, when he will have
\authorfootnote{1172}Saturnine and Jovial inhabitants: and which Tycho Brahe
doth in some sort touch or insinuate in one of his epistles: but these things
\authorfootnote{1173}Zanchius justly explodes, \bookcite{\textlatin{cap. 3.
lib. 4.}} P. Martyr, \bookcite{\textlatin{in 4. Sam. 28.}}

So that according to these men the number of ethereal spirits must needs be
infinite: for if that be true that some of our mathematicians say: if a stone
could fall from the starry heaven, or eighth sphere, and should pass every hour
an hundred miles, it would be 65 years, or more, before it would come to
ground, by reason of the great distance of heaven from earth, which contains as
some say 170 millions 800 miles, besides those other heavens, whether they be
crystalline or watery which Maginus adds, which peradventure holds as much
more, how many such spirits may it contain? And yet for all this
\authorfootnote{1174}Thomas Albertus, and most hold that there be far more
angels than devils.

\subsection{Sublunary devils, and their kinds.}

But be they more or less, \lit{Quod supra nos nihil ad nos}{what is beyond our
comprehension does not concern us}. Howsoever as Martianus foolishly supposeth,
\li{Aetherii Daemones non curant res humanas}, they care not for us, do not
attend our actions, or look for us, those ethereal spirits have other worlds to
reign in belike or business to follow. We are only now to speak in brief of
these sublunary spirits or devils: for the rest, our divines determine that the
devil had no power over stars, or heavens; \authorfootnote{1175}\li{Carminibus
coelo possunt deducere lunam}{by their charms (verses) they can seduce the moon
from the heavens}. Those are poetical fictions, and that they can
\authorfootnote{1176}\lit{sistere aquam fluviis, et vertere sidera retro}{stop
rivers and turn the stars backward in their courses} as Canadia in \Horace{}, 'tis
all false. \authorfootnote{1177}They are confined until the day of judgment to
this sublunary world, and can work no farther than the four elements, and as
God permits them. Wherefore of these sublunary devils, though others divide
them otherwise according to their several places and offices, Psellus makes six
kinds, fiery, aerial, terrestrial, watery, and subterranean devils, besides
those fairies, satyrs, nymphs, \etc{}

Fiery spirits or devils are such as commonly work by blazing stars,
fire-drakes, or \li{ignes fatui}; which lead men often \li{in flumina aut
praecipitia}, saith Bodine, \bookcite{\textlatin{lib. 2. Theat. Naturae, fol.
221.}} \li{Quos inquit arcere si volunt viatores, clara voce Deum appellare aut
pronam facie terram contingente adorare oportet, et hoc amuletum majoribus
nostris acceptum ferre debemus}{whom if travellers wish to keep off they must
pronounce the name of God with a clear voice, or adore him with their faces in
contact with the ground, \etc{}}; likewise they counterfeit suns and moons,
stars oftentimes, and sit on ship masts: \li{In navigiorum summitatibus
visuntur}; and are called \li{dioscuri}, as Eusebius \bookcite{\textlatin{l.
contra Philosophos, c. xlviii}}. informeth us, out of the authority of
Zenophanes; or little clouds, \li{ad motum nescio quem volantes}; which never
appear, saith Cardan, but they signify some mischief or other to come unto men,
though some again will have them to pretend good, and victory to that side they
come towards in sea fights, St. Elmo's fires they commonly call them, and they
do likely appear after a sea storm; Radzivilius, the Polonian duke, calls this
apparition, \li{Sancti Germani sidus}; and saith moreover that he saw the same
after in a storm, as he was sailing, 1582, from Alexandria to Rhodes.
\authorfootnote{1178}Our stories are full of such apparitions in all kinds.
Some think they keep their residence in that Hecla, a mountain in Iceland,
Aetna in Sicily, Lipari, Vesuvius, \etc{} These devils were worshipped
heretofore by that superstitious Pyromanteia \authorfootnote{1179}and the like.

Aerial spirits or devils, are such as keep quarter most part in the
\authorfootnote{1180}air, cause many tempests, thunder, and lightnings, tear
oaks, fire steeples, houses, strike men and beasts, make it rain stones, as in
Livy's time, wool, frogs, \etc{} Counterfeit armies in the air, strange noises,
swords, \etc{}, as at Vienna before the coming of the Turks, and many times in
Rome, as Scheretzius \bookcite{\textlatin{l. de spect. c. 1. part 1.}} Lavater
\bookcite{\textlatin{de spect. part. 1. c. 17.}} Julius Obsequens, an old
Roman, in his book of prodigies, \emph{ab urb. cond.} 505.
\authorfootnote{1181}Machiavel hath illustrated by many examples, and Josephus,
in his book \bookcite{\textlatin{de bello Judaico}}, before the destruction of
Jerusalem. All which Guil. Postellus, in his first book,
\bookcite{\textlatin{c. 7, de orbis concordia}}, useth as an effectual argument
(as indeed it is) to persuade them that will not believe there be spirits or
devils. They cause whirlwinds on a sudden, and tempestuous storms; which though
our meteorologists generally refer to natural causes, yet I am of Bodine's
mind, \bookcite{\textlatin{Theat. Nat. l. 2.}} they are more often caused by
those aerial devils, in their several quarters; for \li{Tempestatibus se
ingerunt}, saith \authorfootnote{1182}Rich. Argentine; as when a desperate man
makes away with himself, which by hanging or drowning they frequently do, as
Kommanus observes, \bookcite{\textlatin{de mirac. mort. part. 7, c. 76.}}
\li{tripudium agentes}, dancing and rejoicing at the death of a sinner. These
can corrupt the air, and cause plagues, sickness, storms, shipwrecks, fires,
inundations. At Mons Draconis in Italy, there is a most memorable example in
\authorfootnote{1183}\idxname{Jovianus}[Jovianus Pontanus]: and nothing so familiar (if we may
believe those relations of Saxo Grammaticus, Olaus Magnus, Damianus A. Goes) as
for witches and sorcerers, in Lapland, Lithuania, and all over Scandia, to sell
winds to mariners, and cause tempests, which Marcus Paulus the Venetian relates
likewise of the Tartars. These kind of devils are much
\authorfootnote{1184}delighted in sacrifices (saith Porphyry), held all the
world in awe, and had several names, idols, sacrifices, in Rome, Greece, Egypt,
and at this day tyrannise over, and deceive those Ethnics and Indians, being
adored and worshipped for \authorfootnote{1185}gods. For the Gentiles' gods
were devils (as \authorfootnote{1186}Trismegistus confesseth in his Asclepius),
and he himself could make them come to their images by magic spells: and are
now as much "respected by our papists" (saith \authorfootnote{1187}Pictorius)
"under the name of saints." These are they which Cardan thinks desire so much
carnal copulation with witches (Incubi and Succubi), transform bodies, and are
so very cold, if they be touched; and that serve magicians. His father had one
of them (as he is not ashamed to relate), \authorfootnote{1188}an aerial devil,
bound to him for twenty and eight years. As Agrippa's dog had a devil tied to
his collar; some think that Paracelsus (or else Erastus belies him) had one
confined to his sword pummel; others wear them in rings, \etc{} Jannes and
Jambres did many things of old by their help; Simon Magus, Cinops, \Apollonius{}
Tianeus, Jamblichus, and Tritemius of late, that showed Maximilian the emperor
his wife, after she was dead; \li{Et verrucam in collo ejus} (saith
\authorfootnote{1189}Godolman) so much as the wart in her neck. Delrio,
\bookcite{\textlatin{lib. 2.}} hath divers examples of their feats: Cicogna,
\bookcite{\textlatin{lib. 3. cap. 3.}} and Wierus in his book
\bookcite{\textlatin{de praestig. daemonum}}. Boissardus
\bookcite{\textlatin{de magis et veneficis}}.

Water-devils are those Naiads or water nymphs which have been heretofore
conversant about waters and rivers. The water (as Paracelsus thinks) is their
chaos, wherein they live; some call them fairies, and say that Habundia is
their queen; these cause inundations, many times shipwrecks, and deceive men
divers ways, as Succuba, or otherwise, appearing most part (saith Tritemius) in
women's shapes. \authorfootnote{1190}Paracelsus hath several stories of them
that have lived and been married to mortal men, and so continued for certain
years with them, and after, upon some dislike, have forsaken them. Such a one
as Aegeria, with whom Numa was so familiar, Diana, Ceres, \etc{}
\authorfootnote{1191}Olaus Magnus hath a long narration of one Hotherus, a king
of Sweden, that having lost his company, as he was hunting one day, met with
these water nymphs or fairies, and was feasted by them; and Hector Boethius, or
Macbeth, and Banquo, two Scottish lords, that as they were wandering in the
woods, had their fortunes told them by three strange women. To these,
heretofore, they did use to sacrifice, by that \textgreek{ὑδρομαντέια}, or
divination by waters.

Terrestrial devils are those \authorfootnote{1192}Lares, genii, fauns, satyrs,
\authorfootnote{1193}wood-nymphs, foliots, fairies, Robin Goodfellows, trulli,
\etc{}, which as they are most conversant with men, so they do them most harm.
Some think it was they alone that kept the heathen people in awe of old, and
had so many idols and temples erected to them. Of this range was Dagon amongst
the Philistines, Bel amongst the Babylonians, Astartes amongst the Sidonians,
Baal amongst the Samaritans, Isis and Osiris amongst the Egyptians, \etc{};
some put our \authorfootnote{1194}fairies into this rank, which have been in
former times adored with much superstition, with sweeping their houses, and
setting of a pail of clean water, good victuals, and the like, and then they
should not be pinched, but find money in their shoes, and be fortunate in their
enterprises. These are they that dance on heaths and greens, as
\authorfootnote{1195}Lavater thinks with Tritemius, and as
\authorfootnote{1196}Olaus Magnus adds, leave that green circle, which we
commonly find in plain fields, which others hold to proceed from a meteor
falling, or some accidental rankness of the ground, so nature sports herself;
they are sometimes seen by old women and children. Hierom. Pauli, in his
description of the city of Bercino in Spain, relates how they have been
familiarly seen near that town, about fountains and hills; \li{Nonnunquam}
(saith Tritemius) \li{in sua latibula montium simpliciores homines ducant,
stupenda mirantibus ostentes miracula, nolarum sonitus,
spectacula}\authorfootnote{1197}, \etc{} Giraldus Cambrensis gives instance in
a monk of Wales that was so deluded. \authorfootnote{1198}Paracelsus reckons up
many places in Germany, where they do usually walk in little coats, some two
feet long. A bigger kind there is of them called with us hobgoblins, and Robin
Goodfellows, that would in those superstitious times grind corn for a mess of
milk, cut wood, or do any manner of drudgery work. They would mend old irons in
those Aeolian isles of Lipari, in former ages, and have been often seen and
heard. \authorfootnote{1199}Tholosanus calls them \li{trullos} and Getulos, and
saith, that in his days they were common in many places of France. Dithmarus
Bleskenius, in his description of Iceland, reports for a certainty, that almost
in every family they have yet some such familiar spirits; and Felix Malleolus,
in his book \bookcite{\textlatin{de crudel. daemon.}} affirms as much, that
these trolli or telchines are very common in Norway, "and
\authorfootnote{1200}seen to do drudgery work;" to draw water, saith Wierus,
\bookcite{\textlatin{lib. 1. cap. 22}}, dress meat, or any such thing. Another
sort of these there are, which frequent forlorn \authorfootnote{1201}houses,
which the Italians call foliots, most part innoxious,
\authorfootnote{1202}Cardan holds; "They will make strange noises in the night,
howl sometimes pitifully, and then laugh again, cause great flame and sudden
lights, fling stones, rattle chains, shave men, open doors and shut them, fling
down platters, stools, chests, sometimes appear in the likeness of hares,
crows, black dogs," \etc{} of which read \authorfootnote{1203}Pet Thyraeus the
Jesuit, in his Tract, \bookcite{\textlatin{de locis infestis, part. 1. et cap.
4}}, who will have them to be devils or the souls of damned men that seek
revenge, or else souls out of purgatory that seek ease; for such examples
peruse \authorfootnote{1204}Sigismundus Scheretzius, \bookcite{\textlatin{lib.
de spectris, part 1. c. 1.}} which he saith he took out of Luther most part;
there be many instances. \authorfootnote{1205}Plinius Secundus remembers such a
house at Athens, which Athenodorus the philosopher hired, which no man durst
inhabit for fear of devils. \Austin{}, \bookcite{\textlatin{de Civ. Dei. lib. 22,
cap. 1.}} relates as much of Hesperius the Tribune's house, at Zubeda, near
their city of Hippos, vexed with evil spirits, to his great hindrance, \li{Cum
afflictione animalium et servorum suorum}. Many such instances are to be read
in Niderius Formicar, \bookcite{\textlatin{lib. 5. cap. xii. 3.}} \etc{}
Whether I may call these Zim and Ochim, which Isaiah, \biblecite{cap. \rn{xiii.}
21.} speaks of, I make a doubt. See more of these in the said Scheretz.
\bookcite{\textlatin{lib. 1. de spect. cap. 4.}} he is full of examples. These
kind of devils many times appear to men, and affright them out of their wits,
sometimes walking at \authorfootnote{1206}noonday, sometimes at nights,
counterfeiting dead men's ghosts, as that of Caligula, which (saith Suetonius)
was seen to walk in Lavinia's garden, where his body was buried, spirits
haunted, and the house where he died, \authorfootnote{1207}\li{Nulla nox sine
terrore transacta, donec incendio consumpta}; every night this happened, there
was no quietness, till the house was burned. About Hecla, in Iceland, ghosts
commonly walk, \li{animas mortuorum simulantes}, saith Joh. Anan,
\bookcite{\textlatin{lib. 3. de nat. daem.}} Olaus. \bookcite{\textlatin{lib.
2. cap. 2.}} Natal Tallopid. \bookcite{\textlatin{lib. de apparit. spir.}}
Kornmannus \bookcite{\textlatin{de mirac. mort. part. 1. cap. 44.}} such sights
are frequently seen \bookcite{\textlatin{circa sepulchra et monasteria}}, saith
Lavat. \bookcite{\textlatin{lib. 1. cap. 19.}} in monasteries and about
churchyards, \li{loca paludinosa, ampla aedificia, solitaria, et caede hominum
notata}{marshes, great buildings, solitary places, or remarkable as
the scene of some murder}. Thyreus adds, \li{ubi gravius peccatum est
commissum, impii, pauperum oppressores et nequiter insignes habitant}{where
some very heinous crime was committed, there the impious and infamous generally
dwell}. These spirits often foretell men's deaths by several signs, as
knocking, groanings, \etc{} \authorfootnote{1208}though Rich. Argentine,
\bookcite{\textlatin{c. 18. de praestigiis daemonum}}, will ascribe these
predictions to good angels, out of the authority of Ficinus and others;
\lit{prodigia in obitu principum saepius contingunt}{prodigies
frequently occur at the deaths of illustrious men}, as in the Lateran church in
\authorfootnote{1209}Rome, the popes' deaths are foretold by Sylvester's tomb.
Near Rupes Nova in Finland, in the kingdom of Sweden, there is a lake, in
which, before the governor of the castle dies, a spectrum, in the habit of
Arion with his harp, appears, and makes excellent music, like those blocks in
Cheshire, which (they say) presage death to the master of the family; or that
\authorfootnote{1210}oak in Lanthadran park in Cornwall, which foreshows as
much. Many families in Europe are so put in mind of their last by such
predictions, and many men are forewarned (if we may believe Paracelsus) by
familiar spirits in divers shapes, as cocks, crows, owls, which often hover
about sick men's chambers, \li{vel quia morientium foeditatem sentiunt}, as
\authorfootnote{1211}Baracellus conjectures, \li{et ideo super tectum
infirmorum crocitant}, because they smell a corse; or for that (as
\authorfootnote{1212}Bernardinus de Bustis thinketh) God permits the devil to
appear in the form of crows, and such like creatures, to scare such as live
wickedly here on earth. A little before \Tully{}'s death (saith \Plutarch{}) the
crows made a mighty noise about him, \li{tumultuose perstrepentes}, they pulled
the pillow from under his head. Rob. Gaguinus, \bookcite{\textlatin{hist.
Franc. lib. 8}}, telleth such another wonderful story at the death of Johannes
de Monteforti, a French lord, \emph{anno} 1345, \li{tanta corvorum multitudo
aedibus morientis insedit, quantam esse in Gallia nemo judicasset}{a multitude
of crows alighted on the house of the dying man, such as no one imagined
existed in France}. Such prodigies are very frequent in authors. See more of
these in the said Lavater, Thyreus \bookcite{\textlatin{de locis infestis, part
3, cap. 58.}} Pictorius, Delrio, Cicogna, \bookcite{\textlatin{lib. 3, cap.
9.}} Necromancers take upon them to raise and lay them at their pleasures: and
so likewise, those which Mizaldus calls \li{ambulones}, that walk about
midnight on great heaths and desert places, which (saith
\authorfootnote{1213}Lavater) "draw men out of the way, and lead them all night
a byway, or quite bar them of their way;" these have several names in several
places; we commonly call them Pucks. In the deserts of Lop, in Asia, such
illusions of walking spirits are often perceived, as you may read in M. Paulus
the Venetian his travels; if one lose his company by chance, these devils will
call him by his name, and counterfeit voices of his companions to seduce him.
Hieronym. Pauli, in his book of the hills of Spain, relates of a great
\authorfootnote{1214}mount in Cantabria, where such spectrums are to be seen;
Lavater and Cicogna have variety of examples of spirits and walking devils in
this kind. Sometimes they sit by the highway side, to give men falls, and make
their horses stumble and start as they ride (if you will believe the relation
of that holy man Ketellus in \authorfootnote{1215}Nubrigensis), that had an
especial grace to see devils, \li{Gratiam divinitus collatam}, and talk with
them, \li{Et impavidus cum spiritibus sermonem miscere}, without offence, and
if a man curse or spur his horse for stumbling, they do heartily rejoice at it;
with many such pretty feats.

Subterranean devils are as common as the rest, and do as much harm. Olaus
Magnus, \bookcite{\textlatin{lib. 6, cap. 19}}, make six kinds of them; some
bigger, some less. These (saith
\idxname{munster}[Munster][Cosmographia]\authorfootnote{1216}) are commonly
seen about mines of metals, and are some of them noxious; some again do no
harm. The metal-men in many places account it good luck, a sign of treasure and
rich ore when they see them. Georgius Agricola, in his book
\bookcite{\textlatin{de subterraneis animantibus, cap. 37}}, reckons two more
notable kinds of them, which he calls \authorfootnote{1217}\li{getuli} and
\li{cobali}, both "are clothed after the manner of metal-men, and will many
times imitate their works." Their office, as Pictorius and Paracelsus think, is
to keep treasure in the earth, that it be not all at once revealed; and
besides, \authorfootnote{1218}Cicogna avers that they are the frequent causes
of those horrible earthquakes "which often swallow up, not only houses, but
whole islands and cities;" in his third book, \bookcite{\textlatin{cap. 11}},
he gives many instances.

The last are conversant about the centre of the earth to torture the souls of
damned men to the day of judgment; their egress and regress some suppose to be
about Etna, Lipari, Mons Hecla in Iceland, Vesuvius, Terra del Fuego, \etc{},
because many shrieks and fearful cries are continually heard thereabouts, and
familiar apparitions of dead men, ghosts and goblins.

\subsection{Their Offices, Operations, Study.}
Thus the devil reigns, and in a thousand several shapes, "as a roaring lion
still seeks whom he may devour," \biblecite{1 Pet. \rn{v.}}, by sea, land, air,
as yet unconfined, though \authorfootnote{1219}some will have his proper place
the air; all that space between us and the moon for them that transgressed
least, and hell for the wickedest of them, \li{Hic velut in carcere ad finem
mundi, tunc in locum funestiorum trudendi}, as \Austin{} holds
\bookcite{\textlatin{de Civit. Dei, c. 22, lib. 14, cap. 3 et 23}}; but be
where he will, he rageth while he may to comfort himself, as
\authorfootnote{1220}Lactantius thinks, with other men's falls, he labours all
he can to bring them into the same pit of perdition with him. For
\authorfootnote{1221}"men's miseries, calamities, and ruins are the devil's
banqueting dishes." By many temptations and several engines, he seeks to
captivate our souls. The Lord of Lies, saith \authorfootnote{1222}\Austin{}, "as
he was deceived himself, he seeks to deceive others," the ringleader to all
naughtiness, as he did by Eve and Cain, Sodom and Gomorrah, so would he do by
all the world. Sometimes he tempts by covetousness, drunkenness, pleasure,
pride, \etc{}, errs, dejects, saves, kills, protects, and rides some men, as
they do their horses. He studies our overthrow, and generally seeks our
destruction; and although he pretend many times human good, and vindicate
himself for a god by curing of several diseases, \li{aegris sanitatem, et
caecis luminis usum restituendo}, as \Austin{} declares, \bookcite{\textlatin{lib.
10, de civit Dei, cap. 6}}, as Apollo, Aesculapius, Isis, of old have done;
divert plagues, assist them in wars, pretend their happiness, yet \li{nihil his
impurius, scelestius, nihil humano generi infestius}, nothing so impure,
nothing so pernicious, as may well appear by their tyrannical and bloody
sacrifices of men to Saturn and Moloch, which are still in use among those
barbarous Indians, their several deceits and cozenings to keep men in
obedience, their false oracles, sacrifices, their superstitious impositions of
fasts, penury, \etc{} Heresies, superstitious observations of meats, times,
\etc{}, by which they \authorfootnote{1223}crucify the souls of mortal men, as
shall be showed in \hyperref[ch:religious-melancholy]{our Treatise of
Religious Melancholy}. \li{Modico adhuc tempore sinitur malignari}, as
\authorfootnote{1224}Bernard expresseth it, by God's permission he rageth a
while, hereafter to be confined to hell and darkness, "which is prepared for
him and his angels," \biblecite{Mat. xxv}.

How far their power doth extend it is hard to determine; what the ancients held
of their effects, force and operations, I will briefly show you: Plato in
Critias, and after him his followers, gave out that these spirits or devils,
"were men's governors and keepers, our lords and masters, as we are of our
cattle." \authorfootnote{1225}"They govern provinces and kingdoms by oracles,
auguries," dreams, rewards and punishments, prophecies, inspirations,
sacrifices, and religious superstitions, varied in as many forms as there be
diversity of spirits; they send wars, plagues, peace, sickness, health, dearth,
plenty, \authorfootnote{1226}\li{Adstantes hic jam nobis, spectantes, et
arbitrantes}, \etc{} as appears by those histories of Thucydides, Livius,
Dionysius Halicarnassus, with many others that are full of their wonderful
stratagems, and were therefore by those Roman and Greek commonwealths adored
and worshipped for gods with prayers and sacrifices, \etc{}
\authorfootnote{1227}In a word, \li{Nihil magis quaerunt quam metum et
admirationem hominum}\authorlatintrans{1228}; and as another hath it, \li{Dici
non potest, quam impotenti ardore in homines dominium, et Divinos cultus
maligni spiritus affectent}\authorlatintrans{1229}. Tritemius in his book
\bookcite{\textlatin{de septem secundis}}, assigns names to such angels as are
governors of particular provinces, by what authority I know not, and gives them
several jurisdictions. Asclepiades a Grecian, Rabbi Achiba the Jew, Abraham
Avenezra, and Rabbi Azariel, Arabians, (as I find them cited by
\authorfootnote{1230}Cicogna) farther add, that they are not our governors
only, \li{Sed ex eorum concordia et discordia, boni et mali affectus
promanant}, but as they agree, so do we and our princes, or disagree; stand or
fall. Juno was a bitter enemy to Troy, Apollo a good friend, Jupiter
indifferent, \li{Aequa Venus Teucris, Pallas iniqua fuit}; some are for us
still, some against us, \li{Premente Deo, fert Deus alter opem}. Religion,
policy, public and private quarrels, wars are procured by them, and they are
\authorfootnote{1231}delighted perhaps to see men fight, as men are with cocks,
bulls and dogs, bears, \etc{}, plagues, dearths depend on them, our \li{bene}
and \li{male esse}, and almost all our other peculiar actions, (for as Anthony
Rusea contends, \bookcite{\textlatin{lib. 5, cap. 18}}, every man hath a good
and a bad angel attending on him in particular, all his life long, which
Jamblichus calls \li{daemonem},) preferments, losses, weddings, deaths, rewards
and punishments, and as \authorfootnote{1232}Proclus will, all offices
whatsoever, \li{alii genetricem, alii opificem potestatem habent}, \etc{} and
several names they give them according to their offices, as Lares, Indegites,
Praestites, \etc{} When the Arcades in that battle at Cheronae, which was
fought against King Philip for the liberty of Greece, had deceitfully carried
themselves, long after, in the very same place, \li{Diis Graeciae, ultoribus}
(saith mine author) they were miserably slain by Metellus the Roman: so
likewise, in smaller matters, they will have things fall out, as these
\li{boni} and \li{mali genii} favour or dislike us: \li{Saturni non conveniunt
Jovialibus}, \etc{} He that is Saturninus shall never likely be preferred.
\authorfootnote{1233}That base fellows are often advanced, undeserving
Gnathoes, and vicious parasites, whereas discreet, wise, virtuous and worthy
men are neglected and unrewarded; they refer to those domineering spirits, or
subordinate Genii; as they are inclined, or favour men, so they thrive, are
ruled and overcome; for as \authorfootnote{1234}Libanius supposeth in our
ordinary conflicts and contentions, \lit{Genius Genio cedit et obtemperat}{one
genius yields and is overcome by another}. All particular events almost they
refer to these private spirits; and (as Paracelsus adds) they direct, teach,
inspire, and instruct men. Never was any man extraordinary famous in any art,
action, or great commander, that had not \li{familiarem daemonem} to inform
him, as Numa, Socrates, and many such, as Cardan illustrates,
\bookcite{\textlatin{cap. 128}}, \li{Arcanis prudentiae civilis},
\authorfootnote{1235}\li{Speciali siquidem gratia, se a Deo donari asserunt
magi, a Geniis caelestibus instrui, ab iis doceri}. But these are most
erroneous paradoxes, \li{ineptae et fabulosae nugae}, rejected by our divines
and Christian churches. 'Tis true they have, by God's permission, power over
us, and we find by experience, that they can \authorfootnote{1236}hurt not our
fields only, cattle, goods, but our bodies and minds. At Hammel in Saxony,
\emph{An.} 1484. 20 \emph{Junii}, the devil, in likeness of a pied piper,
carried away 130 children that were never after seen. Many times men are
\authorfootnote{1237}affrighted out of their wits, carried away quite, as
Scheretzius illustrates, \bookcite{\textlatin{lib. 1, c. iv.}}, and severally
molested by his means, Plotinus the Platonist, \bookcite{\textlatin{lib. 14,
advers. Gnos.}} laughs them to scorn, that hold the devil or spirits can cause
any such diseases. Many think he can work upon the body, but not upon the mind.
But experience pronounceth otherwise, that he can work both upon body and mind.
Tertullian is of this opinion, \bookcite{\textlatin{c. 22.}}
\authorfootnote{1238}"That he can cause both sickness and health," and that
secretly. \authorfootnote{1239}Taurellus adds "by clancular poisons he can
infect the bodies, and hinder the operations of the bowels, though we perceive
it not, closely creeping into them," saith \authorfootnote{1240}Lipsius, and so
crucify our souls: \li{Et nociva melancholia furiosos efficit}. For being a
spiritual body, he struggles with our spirits, saith Rogers, and suggests
(according to \authorfootnote{1241}Cardan, \li{verba sine voce, species sine
visu}, envy, lust, anger, \etc{}) as he sees men inclined.

The manner how he performs it, Biarmannus in his Oration against Bodine,
sufficiently declares. \authorfootnote{1242}"He begins first with the phantasy,
and moves that so strongly, that no reason is able to resist." Now the phantasy
he moves by mediation of humours; although many physicians are of opinion, that
the devil can alter the mind, and produce this disease of himself.
\li{Quibusdam medicorum visum}, saith \authorfootnote{1243}\Avicenna{}, \li{quod
Melancholia contingat a daemonio}. Of the same mind is Psellus and Rhasis the
Arab. \bookcite{\textlatin{lib. 1. Tract. 9. Cont}}. \authorfootnote{1244}"That
this disease proceeds especially from the devil, and from him alone."
Arculanus, \bookcite{\textlatin{cap. 6. in 9. Rhasis}}, Aelianus Montaltus, in
his \bookcite{\textlatin{9. cap}}. Daniel Sennertus, \bookcite{\textlatin{lib.
1. part. 2. cap. 11.}} confirm as much, that the devil can cause this disease;
by reason many times that the parties affected prophesy, speak strange
language, but \li{non sine interventu humoris}, not without the humour, as he
interprets himself; no more doth \Avicenna{}, \li{si contingat a daemonio,
sufficit nobis ut convertat complexionem ad choleram nigram, et sit causa ejus
propinqua cholera nigra}; the immediate cause is choler adust, which
\authorfootnote{1245}Pomponatius likewise labours to make good: Galgerandus of
Mantua, a famous physician, so cured a demoniacal woman in his time, that spake
all languages, by purging black choler, and thereupon belike this humour of
melancholy is called \li{balneum diaboli}, the devil's bath; the devil spying
his opportunity of such humours drives them many times to despair, fury, rage,
\etc{}, mingling himself among these humours. This is that which Tertullian
avers, \li{Corporibus infligunt acerbos casus, animaeque repentinos, membra
distorquent, occulte repentes}, \etc{} and which Lemnius goes about to prove,
\li{Immiscent se mali Genii pravis humoribus, atque atrae, bili}, \etc{} And
\authorfootnote{1246}Jason Pratensis, "that the devil, being a slender
incomprehensible spirit, can easily insinuate and wind himself into human
bodies, and cunningly couched in our bowels vitiate our healths, terrify our
souls with fearful dreams, and shake our minds with furies." And in another
place, "These unclean spirits settled in our bodies, and now mixed with our
melancholy humours, do triumph as it were, and sport themselves as in another
heaven." Thus he argues, and that they go in and out of our bodies, as bees do
in a hive, and so provoke and tempt us as they perceive our temperature
inclined of itself, and most apt to be deluded. \authorfootnote{1247}Agrippa
and \authorfootnote{1248}Lavater are persuaded, that this humour invites the
devil to it, wheresoever it is in extremity, and of all other, melancholy
persons are most subject to diabolical temptations and illusions, and most apt
to entertain them, and the Devil best able to work upon them. But whether by
obsession, or possession, or otherwise, I will not determine; 'tis a difficult
question. Delrio the Jesuit, \bookcite{\textlatin{Tom. 3. lib. 6.}} Springer
and his colleague, \bookcite{\textlatin{mall. malef}}. Pet. Thyreus the Jesuit,
\bookcite{\textlatin{lib. de daemoniacis, de locis infestis, de
Terrificationibus nocturnis}}, Hieronymus Mengus \bookcite{\textlatin{Flagel.
daem}}. and others of that rank of pontifical writers, it seems, by their
exorcisms and conjurations approve of it, having forged many stories to that
purpose. A nun did eat a lettuce \authorfootnote{1249}without grace, or signing
it with the sign of the cross, and was instantly possessed. Durand.
\bookcite{\textlatin{lib. 6. Rationall. c. 86. numb. 8.}} relates that he saw a
wench possessed in Bononia with two devils, by eating an unhallowed
pomegranate, as she did afterwards confess, when she was cured by exorcisms.
And therefore our Papists do sign themselves so often with the sign of the
cross, \li{Ne daemon ingredi ausit}, and exorcise all manner of meats, as being
unclean or accursed otherwise, as Bellarmine defends. Many such stories I find
amongst pontifical writers, to prove their assertions, let them free their own
credits; some few I will recite in this kind out of most approved physicians.
Cornelius Gemma, \bookcite{\textlatin{lib. 2. de nat. mirac. c. 4.}} relates of
a young maid, called Katherine Gualter, a cooper's daughter, \emph{an.} 1571.
that had such strange passions and convulsions, three men could not sometimes
hold her; she purged a live eel, which he saw, a foot and a half long, and
touched it himself; but the eel afterwards vanished; she vomited some
twenty-four pounds of fulsome stuff of all colours, twice a day for fourteen
days; and after that she voided great balls of hair, pieces of wood, pigeon's
dung, parchment, goose dung, coals; and after them two pounds of pure blood,
and then again coals and stones, or which some had inscriptions bigger than a
walnut, some of them pieces of glass, brass, \etc{} besides paroxysms of
laughing, weeping and ecstasies, \etc{} \lit{Et hoc (inquit) cum horore vidi}{
this I saw with horror}. They could do no good on her by physic, but left her to
the clergy. Marcellus Donatus, \bookcite{\textlatin{lib. 2. c. 1. de med.
mirab.}} hath such another story of a country fellow, that had four knives in
his belly, \li{Instar serrae dentatos}, indented like a saw, every one a span
long, and a wreath of hair like a globe, with much baggage of like sort,
wonderful to behold: how it should come into his guts, he concludes, \li{Certe
non alio quam daemonis astutia et dolo}{could assuredly only have been
through the artifice of the devil}. Langius, \bookcite{\textlatin{Epist. med.
lib. 1. Epist. 38.}} hath many relations to this effect, and so hath
Christophorus a Vega: Wierus, Skenkius, Scribanius, all agree that they are
done by the subtlety and illusion of the devil. If you shall ask a reason of
this, 'tis to exercise our patience; for as \authorfootnote{1250}Tertullian
holds, \li{Virtus non est virtus, nisi comparem habet aliquem, in quo superando
vim suam ostendat} 'tis to try us and our faith, 'tis for our offences, and for
the punishment of our sins, by God's permission they do it, \li{Carnifices
vindictae justae Dei}, as \authorfootnote{1251}Tolosanus styles them,
Executioners of his will; or rather as David, \biblecite{Ps. 78. ver. 49}. "He
cast upon them the fierceness of his anger, indignation, wrath, and vexation,
by sending out of evil angels:" so did he afflict Job, Saul, the Lunatics and
demoniacal persons whom Christ cured, \biblecite{Mat. \rn{iv.} 8. Luke \rn{iv.}
11. Luke \rn{xiii.} Mark \rn{ix.} Tobit. \rn{viii.} 3}. \etc{} This, I say,
happeneth for a punishment of sin, for their want of faith, incredulity,
weakness, distrust, \etc{}

\cleartoleftpage{}
\begin{figure}[p]
  \begingroup
  \centering
  \includegraphics[keepaspectratio,width=\textwidth]{DeHorlende-small.jpg}
  \captionart{DeHorlendeKollendans}
  \label{fig:dehorlende}
\end{figure}

% Force float here
\clearpage{}
\thispagestyle{titleontop}

%SECT. II. MEMB. I. SUBSECT. III.-_Of Witches and Magicians, how they cause Melancholy_.
\section[Witches and Magicians]{Of Witches and Magicians, how they cause Melancholy.}

\lettrine{Y}{ou} have heard what the devil can do of himself, now you shall
hear what he can perform by his instruments, who are many times worse (if it be
possible) than he himself, and to satisfy their revenge and lust cause more
mischief, \li{Multa enim mala non egisset daemon, nisi provocatus a sagis}, as
\authorfootnote{1252}Erastus thinks; much harm had never been done, had he not
been provoked by witches to it. He had not appeared in Samuel's shape, if the
Witch of Endor had let him alone; or represented those serpents in Pharaoh's
presence, had not the magicians urged him unto it; \li{Nec morbos vel
hominibus, vel brutis infligeret} (Erastus maintains) \li{si sagae
quiescerent}; men and cattle might go free, if the witches would let him alone.
Many deny witches at all, or if there be any they can do no harm; of this
opinion is Wierus, \bookcite{\textlatin{lib. 3. cap. 53. de praestig. daem}}.
Austin Lerchemer a Dutch writer, Biarmanus, Ewichius, Euwaldus, our countryman
Scot; with him in \Horace{},

\begin{latin}
\begin{verse}
Somnia, terrores Magicos, miracula, sagas,\\*
Nocturnos Lemures, portentaque Thessala risu\\*
Excipiunt.------\\!
\end{verse}
\end{latin}

\begin{verse}
Say, can you laugh indignant at the schemes\\*
Of magic terrors, visionary dreams,\\*
Portentous wonders, witching imps of Hell,\\*
The nightly goblin, and enchanting spell?\\!
\end{verse}

They laugh at all such stories; but on the contrary are most lawyers, divines,
physicians, philosophers, Austin, Hemingius, Danaeus, Chytraeus, Zanchius,
Aretius, \etc{} Delrio, Springer, \authorfootnote{1253}Niderius,
\bookcite{\textlatin{lib. 5.}} Fornicar. Guiatius, Bartolus,
\bookcite{\textlatin{consil. 6. tom. 1. Bodine, daemoniant. lib 2. cap. 8.}}
Godelman, Damhoderius, \etc{} Paracelsus, Erastus, Scribanius, Camerarius,
\etc{} The parties by whom the devil deals, may be reduced to these two, such
as command him in show at least, as conjurors, and magicians, whose detestable
and horrid mysteries are contained in their book called
\authorfootnote{1254}Arbatell; \li{daemonis enim advocati praesto sunt, seque
exorcismis et conjurationibus quasi cogi patiuntur, ut miserum magorum genus,
in impietate detineant}. Or such as are commanded, as witches, that deal \li{ex
parte implicite}, or \li{explicite}, as the \authorfootnote{1255}king hath well
defined; many subdivisions there are, and many several species of sorcerers,
witches, enchanters, charmers, \etc{} They have been tolerated heretofore some
of them; and magic hath been publicly professed in former times, in
\authorfootnote{1256}Salamanca, \authorfootnote{1257}Krakow, and other places,
though after censured by several \authorfootnote{1258}Universities, and now
generally contradicted, though practised by some still, maintained and excused,
\li{Tanquam res secreta quae non nisi viris magnis et peculiari beneficio de
Coelo instructis communicatur} (I use \authorfootnote{1259}Boesartus his words)
and so far approved by some princes, \li{Ut nihil ausi aggredi in politicis, in
sacris, in consiliis, sine eorum arbitrio}; they consult still with them, and
dare indeed do nothing without their advice. Nero and Heliogabalus, Maxentius,
and Julianus Apostata, were never so much addicted to magic of old, as some of
our modern princes and popes themselves are nowadays. Erricus, King of Sweden,
had an \authorfootnote{1260}enchanted cap, by virtue of which, and some magical
murmur or whispering terms, he could command spirits, trouble the air, and make
the wind stand which way he would, insomuch that when there was any great wind
or storm, the common people were wont to say, the king now had on his conjuring
cap. But such examples are infinite. That which they can do, is as much almost
as the devil himself, who is still ready to satisfy their desires, to oblige
them the more unto him. They can cause tempests, storms, which is familiarly
practised by witches in Norway, Iceland, as I have proved. They can make
friends enemies, and enemies friends by philters;
\authorfootnote{1261}\li{Turpes amores conciliare}, enforce love, tell any man
where his friends are, about what employed, though in the most remote places;
and if they will, \authorfootnote{1262}"bring their sweethearts to them by
night, upon a goat's back flying in the air." Sigismund Scheretzius,
\bookcite{\textlatin{part. 1. cap. 9. de spect.}} reports confidently, that he
conferred with sundry such, that had been so carried many miles, and that he
heard witches themselves confess as much; hurt and infect men and beasts,
vines, corn, cattle, plants, make women abortive, not to conceive,
\authorfootnote{1263}barren, men and women unapt and unable, married and
unmarried, fifty several ways, saith Bodine, \bookcite{\textlatin{lib. 2. c.
2.}} fly in the air, meet when and where they will, as Cicogna proves, and
Lavat. \bookcite{\textlatin{de spec. part. 2. c. 17.}} "steal young children
out of their cradles, \li{ministerio daemonum}, and put deformed in their
rooms, which we call changelings," saith \authorfootnote{1264}Scheretzius,
\bookcite{\textlatin{part. 1. c. 6.}} make men victorious, fortunate, eloquent;
and therefore in those ancient monomachies and combats they were searched of
old, \authorfootnote{1265}they had no magical charms; they can make
\authorfootnote{1266}stick frees, such as shall endure a rapier's point, musket
shot, and never be wounded: of which read more in Boissardus,
\bookcite{\textlatin{cap. 6. de Magia}}, the manner of the adjuration, and by
whom 'tis made, where and how to be used \li{in expeditionibus bellicis,
praeliis, duellis}, \etc{}, with many peculiar instances and examples; they can
walk in fiery furnaces, make men feel no pain on the rack, \li{aut alias
torturas sentire}; they can stanch blood, \authorfootnote{1267}represent dead
men's shapes, alter and turn themselves and others into several forms, at their
pleasures. \authorfootnote{1268}Agaberta, a famous witch in Lapland, would do
as much publicly to all spectators, \li{Modo Pusilla, modo anus, modo procera
ut quercus, modo vacca, avis, coluber}, \etc{} Now young, now old, high, low,
like a cow, like a bird, a snake, and what not? She could represent to others
what forms they most desired to see, show them friends absent, reveal secrets,
\li{maxima omnium admiratione}, \etc{} And yet for all this subtlety of theirs,
as Lipsius well observes, \bookcite{\textlatin{Physiolog. Stoicor. lib. 1. cap.
17.}} neither these magicians nor devils themselves can take away gold or
letters out of mine or Crassus' chest, \li{et Clientelis suis largiri}, for
they are base, poor, contemptible fellows most part; as
\authorfootnote{1269}Bodine notes, they can do nothing \li{in Judicum decreta
aut poenas, in regum concilia vel arcana, nihil in rem nummariam aut
thesauros}, they cannot give money to their clients, alter judges' decrees, or
councils of kings, these \li{minuti Genii} cannot do it, \li{altiores Genii hoc
sibi adservarunt}, the higher powers reserve these things to themselves. Now
and then peradventure there may be some more famous magicians like Simon Magus,
\authorfootnote{1270}\Apollonius{} Tyaneus, Pasetes, Jamblichus,
\authorfootnote{1271}Odo de Stellis, that for a time can build castles in the
air, represent armies, \etc{}, as they are \authorfootnote{1272}said to have
done, command wealth and treasure, feed thousands with all variety of meats
upon a sudden, protect themselves and their followers from all princes'
persecutions, by removing from place to place in an instant, reveal secrets,
future events, tell what is done in far countries, make them appear that died
long since, and do many such miracles, to the world's terror, admiration and
opinion of deity to themselves, yet the devil forsakes them at last, they come
to wicked ends, and \li{raro aut nunquam} such impostors are to be found. The
vulgar sort of them can work no such feats. But to my purpose, they can, last
of all, cure and cause most diseases to such as they love or hate, and this of
\authorfootnote{1273}melancholy amongst the rest. Paracelsus,
\bookcite{\textlatin{Tom. 4. de morbis amentium, Tract. 1.}} in express words
affirms; \li{Multi fascinantur in melancholiam}, many are bewitched into
melancholy, out of his experience. The same saith Danaeus,
\bookcite{\textlatin{lib. 3. de sortiariis}}. \li{Vidi, inquit, qui
Melancholicos morbos gravissimos induxerunt}: I have seen those that have
caused melancholy in the most grievous manner, \authorfootnote{1274}dried up
women's paps, cured gout, palsy; this and apoplexy, falling sickness, which no
physic could help, \li{solu tactu}, by touch alone. Ruland in his
\bookcite{\textlatin{3 Cent. Cura 91.}} gives an instance of one David Helde, a
young man, who by eating cakes which a witch gave him, \li{mox delirare
coepit}, began to dote on a sudden, and was instantly mad: F. H. D. in
\authorfootnote{1275}Hildesheim, consulted about a melancholy man, thought his
disease was partly magical, and partly natural, because he vomited pieces of
iron and lead, and spake such languages as he had never been taught; but such
examples are common in Scribanius, Hercules de Saxonia, and others. The means
by which they work are usually charms, images, as that in Hector Boethius of
King Duffe; characters stamped of sundry metals, and at such and such
constellations, knots, amulets, words, philters, \etc{}, which generally make
the parties affected, melancholy; as \authorfootnote{1276}Monavius discourseth
at large in an epistle of his to Acolsius, giving instance in a Bohemian baron
that was so troubled by a philter taken. Not that there is any power at all in
those spells, charms, characters, and barbarous words; but that the devil doth
use such means to delude them. \li{Ut fideles inde magos} (saith
\authorfootnote{1277}Libanius) \li{in officio retineat, tum in consortium
malefactorum vocet}.

\cleartoleftpage{}
\begin{figure}[p]
  \begingroup
  \centering
  \includegraphics[keepaspectratio,width=\textwidth]{Woman-stars-small.jpg}
  \captionart{WomanStars}
  \label{fig:womanstars}
\end{figure}

% Force float here
\clearpage{}
\thispagestyle{titleontop}

%SECT. II. MEMB. I. SUBSECT. IV.-_Stars a cause. Signs from Physiognomy, Metoposcopy, Chiromancy_.
\section[Heavens, Planets, Stars]{Stars a cause. Signs from Physiognomy, Metoposcopy, Chiromancy.}\label{sec:heavens-planets-stars}

\lettrine{N}{atural} causes are either primary and universal, or secondary and
more particular. Primary causes are the heavens, planets, stars, \etc{}, by
their influence (as our astrologers hold) producing this and such like effects.
I will not here stand to discuss \li{obiter}, whether stars be causes, or
signs; or to apologise for judical astrology. If either Sextus Empericus, Picus
Mirandula, Sextus ab Heminga, Pererius, Erastus, Chambers, \etc{}, have so far
prevailed with any man, that he will attribute no virtue at all to the heavens,
or to sun, or moon, more than he doth to their signs at an innkeeper's post, or
tradesman's shop, or generally condemn all such astrological aphorisms approved
by experience: I refer him to Bellantius, Pirovanus, Marascallerus, Goclenius,
Sir Christopher Heidon, \etc{} If thou shalt ask me what I think, I must
answer, \lit{nam et doctis hisce erroribus versatus sum}{for I am conversant
with these learned errors} they do incline, but not compel; no necessity at
all: \authorfootnote{1278}\li{agunt non cogunt}: and so gently incline, that a
wise man may resist them; \li{sapiens dominabitur astris}: they rule us, but
God rules them. All this (methinks) \authorfootnote{1279}Joh. de Indagine hath
comprised in brief, \li{Quaeris a me quantum in nobis operantur astra}? \etc{}
"Wilt thou know how far the stars work upon us? I say they do but incline, and
that so gently, that if we will be ruled by reason, they have no power over us;
but if we follow our own nature, and be led by sense, they do as much in us as
in brute beasts, and we are no better." So that, I hope, I may justly conclude
with \authorfootnote{1280}Cajetan, \li{Coelum est vehiculum divinae virtutis},
\etc{}, that the heaven is God's instrument, by mediation of which he governs
and disposeth these elementary bodies; or a great book, whose letters are the
stars, (as one calls it,) wherein are written many strange things for such as
can read, \authorfootnote{1281}"or an excellent harp, made by an eminent
workman, on which, he that can but play, will make most admirable music." But
to the purpose.

\begin{figure}[p]
  \begingroup
  \centering
  \includegraphics[keepaspectratio,width=\textwidth]{SkyMap-small.jpg}
  \captionart{SkyMap}
  \label{fig:skymap}
\end{figure}

\authorfootnote{1282}Paracelsus is of opinion, "that a physician without the
knowledge of stars can neither understand the cause or cure of any disease,
either of this or gout, not so much as toothache; except he see the peculiar
geniture and scheme of the party effected." And for this proper malady, he will
have the principal and primary cause of it proceed from the heaven, ascribing
more to stars than humours, \authorfootnote{1283}"and that the constellation
alone many times produceth melancholy, all other causes set apart." He gives
instance in lunatic persons, that are deprived of their wits by the moon's
motion; and in another place refers all to the ascendant, and will have the
true and chief cause of it to be sought from the stars. Neither is it his
opinion only, but of many Galenists and philosophers, though they do not so
peremptorily maintain as much. "This variety of melancholy symptoms proceeds
from the stars," saith \authorfootnote{1284}Melancthon: the most generous
melancholy, as that of Augustus, comes from the conjunction of Saturn and
Jupiter in Libra: the bad, as that of Catiline's, from the meeting of Saturn
and the moon in Scorpio. \idxname{Jovianus}[Jovianus Pontanus], in his tenth book, and thirteenth
chapter \li{de rebus coelestibus}, discourseth to this purpose at large, \li{Ex
atra bile varii generantur morbi}, \etc{}, \authorfootnote{1285}"many diseases
proceed from black choler, as it shall be hot or cold; and though it be cold in
its own nature, yet it is apt to be heated, as water may be made to boil, and
burn as bad as fire; or made cold as ice: and thence proceed such variety of
symptoms, some mad, some solitary, some laugh, some rage," \etc{} The cause of
all which intemperance he will have chiefly and primarily proceed from the
heavens, \authorfootnote{1286}"from the position of Mars, Saturn, and Mercury."
His aphorisms be these, \authorfootnote{1287}"Mercury in any geniture, if he
shall be found in Virgo, or Pisces his opposite sign, and that in the
horoscope, irradiated by those quartile aspects of Saturn or Mars, the child
shall be mad or melancholy." Again, \authorfootnote{1288}"He that shall have
Saturn and Mars, the one culminating, the other in the fourth house, when he
shall be born, shall be melancholy, of which he shall be cured in time, if
Mercury behold them. \authorfootnote{1289}If the moon be in conjunction or
opposition at the birth time with the sun, Saturn or Mars, or in a quartile
aspect with them," (\li{e malo coeli loco}, Leovitius adds,) "many diseases are
signified, especially the head and brain is like to be misaffected with
pernicious humours, to be melancholy, lunatic, or mad," Cardan adds, \li{quarta
luna natos}, eclipses, earthquakes. Garcaeus and Leovitius will have the chief
judgment to be taken from the lord of the geniture, or where there is an aspect
between the moon and Mercury, and neither behold the horoscope, or Saturn and
Mars shall be lord of the present conjunction or opposition in Sagittarius or
Pisces, of the sun or moon, such persons are commonly epileptic, dote,
demoniacal, melancholy: but see more of these aphorisms in the above-named
Pontanus. Garcaeus, \bookcite{\textlatin{cap. 23. de Jud. genitur. Schoner.
lib. 1. cap. 8}}, which he hath gathered out of \authorfootnote{1290}Ptolemy,
Albubater, and some other Arabians, Junctine, Ranzovius, Lindhout, Origen,
\etc{} But these men you will reject peradventure, as astrologers, and
therefore partial judges; then hear the testimony of physicians, Galenists
themselves. \authorfootnote{1291}Carto confesseth the influence of stars to
have a great hand to this peculiar disease, so doth Jason Pratensis, Lonicerius
\bookcite{\textlatin{praefat. de Apoplexia}}, Ficinus, Fernelius, \etc{}
\authorfootnote{1292}P. Cnemander acknowledgeth the stars an universal cause,
the particular from parents, and the use of the six non-natural things.
Baptista Port. \bookcite{\textlatin{mag. l. 1. c. 10, 12, 15}}, will have them
causes to every particular \li{individium}. Instances and examples, to evince
the truth of those aphorisms, are common amongst those astrologian treatises.
Cardan, in his thirty-seventh geniture, gives instance in Matth. Bolognius.
\bookcite{\textlatin{Camerar. hor. natalit. centur. 7. genit. 6. et 7.}} of
Daniel Gare, and others; but see Garcaeus, \bookcite{\textlatin{cap. 33.}} Luc.
Gauricus, \bookcite{\textlatin{Tract. 6. de Azemenis}}, \etc{} The time of this
melancholy is, when the significators of any geniture are directed according to
art, as the hor: moon, hylech, \etc{} to the hostile beams or terms of \saturn{} and \mars{}
especially, or any fixed star of their nature, or if \saturn{} by his revolution or
transitus, shall offend any of those radical promissors in the geniture.

Other signs there are taken from physiognomy, metoposcopy, chiromancy, which
because Joh. de Indagine, and Rotman, the landgrave of Hesse his mathematician,
not long since in his Chiromancy; Baptista Porta, in his celestial Physiognomy,
have proved to hold great affinity with astrology, to satisfy the curious, I am
the more willing to insert.

The general notions \authorfootnote{1293}physiognomers give, be these; "black
colour argues natural melancholy; so doth leanness, hirsuteness, broad veins,
much hair on the brows," saith \authorfootnote{1294}Gratanarolus,
\bookcite{\textlatin{cap. 7}}, and a little head, out of \Aristotle{}, high
sanguine, red colour, shows head melancholy; they that stutter and are bald,
will be soonest melancholy, (as \Avicenna{} supposeth,) by reason of the dryness
of their brains; but he that will know more of the several signs of humour and
wits out of physiognomy, let him consult with old Adamantus and Polemus, that
comment, or rather paraphrase upon \idxname{Pseudo-Aristotle}[Aristotle][\textlatin{Physiognomonica}]'s Physiognomy, Baptista Porta's
four pleasant books, Michael Scot \bookcite{\textlatin{de secretis naturae}},
John de Indagine, Montaltus, Antony Zara. \bookcite{\textlatin{anat.
ingeniorum, sect. 1. memb. 13. et lib. 4.}}

Chiromancy hath these aphorisms to foretell melancholy, Tasneir.
\bookcite{\textlatin{lib. 5. cap. 2}}, who hath comprehended the sum of John de
Indagine: Tricassus, Corvinus, and others in his book, thus hath it;
\authorfootnote{1295}"The Saturnine line going from the rascetta through the
hand, to Saturn's mount, and there intersected by certain little lines, argues
melancholy; so if the vital and natural make an acute angle, Aphorism 100. The
saturnine, hepatic, and natural lines, making a gross triangle in the hand,
argue as much;" which Goclenius, \bookcite{\textlatin{cap. 5. Chiros.}} repeats
verbatim out of him. In general they conclude all, that if Saturn's mount be
full of many small lines and intersections, \authorfootnote{1296}"such men are
most part melancholy, miserable and full of disquietness, care and trouble,
continually vexed with anxious and bitter thoughts, always sorrowful, fearful,
suspicious; they delight in husbandry, buildings, pools, marshes, springs,
woods, walks," \etc{} Thaddaeus Haggesius, in his
\bookcite{\textlatin{Metoposcopia}}, hath certain aphorisms derived from
Saturn's lines in the forehead, by which he collects a melancholy disposition;
and \authorfootnote{1297}Baptista Porta makes observations from those other
parts of the body, as if a spot be over the spleen; \authorfootnote{1298}"or in
the nails; if it appear black, it signifieth much care, grief, contention, and
melancholy;" the reason he refers to the humours, and gives instance in
himself, that for seven years space he had such black spots in his nails, and
all that while was in perpetual lawsuits, controversies for his inheritance,
fear, loss of honour, banishment, grief, care, \etc{} and when his miseries
ended, the black spots vanished. Cardan, in his book \bookcite{\textlatin{de
libris propriis}}, tells such a story of his own person, that a little before
his son's death, he had a black spot, which appeared in one of his nails; and
dilated itself as he came nearer to his end. But I am over tedious in these
toys, which howsoever, in some men's too severe censures, they may be held
absurd and ridiculous, I am the bolder to insert, as not borrowed from
circumforanean rogues and gipsies, but out of the writings of worthy
philosophers and physicians, yet living some of them, and religious professors
in famous universities, who are able to patronise that which they have said,
and vindicate themselves from all cavillers and ignorant persons.

%SECT. II. MEMB. I. SUBSECT. V.-_Old age a cause_.
\section{Old age a cause.}

\lettrine{S}{econdary} peculiar causes efficient, so called in respect of the
other precedent, are either \li{congenitae, internae, innatae}, as they term
them, inward, innate, inbred; or else outward and adventitious, which happen to
us after we are born: congenite or born with us, are either natural, as old
age, or \li{praeter naturam} (as \authorfootnote{1299}Fernelius calls it) that
distemperature, which we have from our parent's seed, it being an hereditary
disease. The first of these, which is natural to all, and which no man living
can avoid, is \authorfootnote{1300}old age, which being cold and dry, and of
the same quality as melancholy is, must needs cause it, by diminution of
spirits and substance, and increasing of adust humours; therefore
\authorfootnote{1301}Melancthon avers out of \Aristotle{}, as an undoubted truth,
\li{Senes plerunque delirasse in senecta}, that old men familiarly dote, \li{ob
atram bilem}, for black choler, which is then superabundant in them: and
Rhasis, that Arabian physician, in his \bookcite{\textlatin{Cont. lib. 1. cap.
9}}, calls it \authorfootnote{1302}"a necessary and inseparable accident," to
all old and decrepit persons. After seventy years (as the Psalmist saith)
\authorfootnote{1303}"all is trouble and sorrow;" and common experience
confirms the truth of it in weak and old persons, especially such as have lived
in action all their lives, had great employment, much business, much command,
and many servants to oversee, and leave off \li{ex abrupto}; as
\authorfootnote{1304}Charles the Fifth did to King Philip, resign up all on a
sudden; they are overcome with melancholy in an instant: or if they do continue
in such courses, they dote at last, (\li{senex bis puer},) and are not able to
manage their estates through common infirmities incident in their age; full of
ache, sorrow and grief, children again, dizzards, they carl many times as they
sit, and talk to themselves, they are angry, waspish, displeased with every
thing, "suspicious of all, wayward, covetous, hard" (saith \Tully{},)
"self-willed, superstitious, self-conceited, braggers and admirers of
themselves," as \authorfootnote{1305}Balthazar Castilio hath truly noted of
them. \authorfootnote{1306}This natural infirmity is most eminent in old women,
and such as are poor, solitary, live in most base esteem and beggary, or such
as are witches; insomuch that Wierus, Baptista Porta, Ulricus Molitor, Edwicus,
do refer all that witches are said to do, to imagination alone, and this humour
of melancholy. And whereas it is controverted, whether they can bewitch cattle
to death, ride in the air upon a cowl-staff out of a chimney-top, transform
themselves into cats, dogs, \etc{}, translate bodies from place to place, meet
in companies, and dance, as they do, or have carnal copulation with the devil,
they ascribe all to this redundant melancholy, which domineers in them, to
\authorfootnote{1307}somniferous potions, and natural causes, the devil's
policy. \li{Non laedunt omnino} (saith Wierus) \li{aut quid mirum faciunt},
(\bookcite{\textlatin{de Lamiis, lib. 3. cap. 36}}), \li{ut putatur, solam
vitiatam habent phantasiam}; they do no such wonders at all, only their
\authorfootnote{1308}brains are crazed. \authorfootnote{1309}"They think they
are witches, and can do hurt, but do not." But this opinion Bodine, Erastus,
Danaeus, Scribanius, Sebastian Michaelis, \idxname{campanella}[Campanella][\textlatin{De sensu rerum et magia}] \bookcite{\textlatin{de
Sensu rerum, lib. 4. cap. 9.}} \authorfootnote{1310}Dandinus the Jesuit,
\bookcite{\textlatin{lib. 2. de Animae explode}}; \authorfootnote{1311}Cicogna
confutes at large. That witches are melancholy, they deny not, but not out of
corrupt phantasy alone, so to delude themselves and others, or to produce such
effects.

%SECT. II. MEMB. I. SUBSECT. VI.-_Parents a cause by Propagation_.
\section{Parents a cause by Propagation.}

\lettrine{T}{hat} other inward inbred cause of Melancholy is our temperature,
in whole or part, which we receive from our parents, which
\authorfootnote{1312}Fernelius calls \li{Praeter naturam}, or unnatural, it
being an hereditary disease; for as he justifies \authorfootnote{1313}\li{Quale
parentum maxime patris semen obtigerit, tales evadunt similares spermaticaeque
paries, quocunque etiam morbo Pater quum generat tenetur, cum semine transfert,
in Prolem}; such as the temperature of the father is, such is the son's, and
look what disease the father had when he begot him, his son will have after
him; \authorfootnote{1314}"and is as well inheritor of his infirmities, as of
his lands. And where the complexion and constitution of the father is corrupt,
there (\authorfootnote{1315}saith Roger Bacon) the complexion and constitution
of the son must needs be corrupt, and so the corruption is derived from the
father to the son." Now this doth not so much appear in the composition of the
body, according to that of Hippocrates, \authorfootnote{1316}"in habit,
proportion, scars, and other lineaments; but in manners and conditions of the
mind," \li{Et patrum in natos abeunt cum semine mores}.

Seleucus had an anchor on his thigh, so had his posterity, as Trogus records,
\bookcite{\textlatin{lib. 15.}} Lepidus, in Pliny \bookcite{\textlatin{l. 7. c.
17}}, was purblind, so was his son. That famous family of Aenobarbi were known
of old, and so surnamed from their red beards; the Austrian lip, and those
Indian flat noses are propagated, the Bavarian chin, and goggle eyes amongst
the Jews, as \authorfootnote{1317}Buxtorfius observes; their voice, pace,
gesture, looks, are likewise derived with all the rest of their conditions and
infirmities; such a mother, such a daughter; their very
\authorfootnote{1318}affections Lemnius contends "to follow their seed, and the
malice and bad conditions of children are many times wholly to be imputed to
their parents;" I need not therefore make any doubt of Melancholy, but that it
is an hereditary disease. \authorfootnote{1319}Paracelsus in express words
affirms it, \bookcite{\textlatin{lib. de morb. amentium to. 4. tr. 1}}; so doth
\authorfootnote{1320}Crato in an Epistle of his to Monavius. So doth Bruno
Seidelius in his book \bookcite{\textlatin{de morbo incurab.}} Montaltus
proves, \bookcite{\textlatin{cap. 11}}, out of Hippocrates and \Plutarch{}, that
such hereditary dispositions are frequent, \li{et hanc (inquit) fieri reor ob
participatam melancholicam intemperantiam} (speaking of a patient) I think he
became so by participation of Melancholy. Daniel Sennertus,
\bookcite{\textlatin{lib. 1. part 2. cap. 9}}, will have his melancholy
constitution derived not only from the father to the son, but to the whole
family sometimes; \li{Quandoque totis familiis hereditativam},
\authorfootnote{1321}Forestus, in his medicinal observations, illustrates this
point, with an example of a merchant, his patient, that had this infirmity by
inheritance; so doth Rodericus a Fonseca, \bookcite{\textlatin{tom. 1. consul.
69}}, by an instance of a young man that was so affected \li{ex matre
melancholica}, had a melancholy mother, \li{et victu melancholico}, and bad
diet together. Ludovicus Mercatus, a Spanish physician, in that excellent Tract
which he hath lately written of hereditary diseases, \bookcite{\textlatin{tom.
2. oper. lib. 5}}, reckons up leprosy, as those \authorfootnote{1322}Galbots in
Gascony, hereditary lepers, pox, stone, gout, epilepsy, \etc{} Amongst the
rest, this and madness after a set time comes to many, which he calls a
miraculous thing in nature, and sticks for ever to them as an incurable habit.
And that which is more to be wondered at, it skips in some families the father,
and goes to the son, \authorfootnote{1323}"or takes every other, and sometimes
every third in a lineal descent, and doth not always produce the same, but some
like, and a symbolizing disease." These secondary causes hence derived, are
commonly so powerful, that (as \authorfootnote{1324}Wolfius holds) \li{saepe
mutant decreta siderum}, they do often alter the primary causes, and decrees of
the heavens. For these reasons, belike, the Church and commonwealth, human and
Divine laws, have conspired to avoid hereditary diseases, forbidding such
marriages as are any whit allied; and as Mercatus adviseth all families to take
such, \li{si fieri possit quae maxime distant natura}, and to make choice of
those that are most differing in complexion from them; if they love their own,
and respect the common good. And sure, I think, it hath been ordered by God's
especial providence, that in all ages there should be (as usually there is)
once in \authorfootnote{1325}600 years, a transmigration of nations, to amend
and purify their blood, as we alter seed upon our land, and that there should
be as it were an inundation of those northern Goths and Vandals, and many such
like people which came out of that continent of Scandia and Sarmatia (as some
suppose) and overran, as a deluge, most part of Europe and Africa, to alter for
our good, our complexions, which were much defaced with hereditary infirmities,
which by our lust and intemperance we had contracted. A sound generation of
strong and able men were sent amongst us, as those northern men usually are,
innocuous, free from riot, and free from diseases; to qualify and make us as
those poor naked Indians are generally at this day; and those about Brazil (as
a late \authorfootnote{1326}writer observes), in the Isle of Maragnan, free
from all hereditary diseases, or other contagion, whereas without help of
physic they live commonly 120 years or more, as in the Orcades and many other
places. Such are the common effects of temperance and intemperance, but I will
descend to particular, and show by what means, and by whom especially, this
infirmity is derived unto us.

\li{Filii ex senibus nati, raro sunt firmi temperamenti}, old men's children
are seldom of a good temperament, as Scoltzius supposeth,
\bookcite{\textlatin{consult. 177}}, and therefore most apt to this disease;
and as \authorfootnote{1327}Levinus Lemnius farther adds, old men beget most
part wayward, peevish, sad, melancholy sons, and seldom merry. He that begets a
child on a full stomach, will either have a sick child, or a crazed son (as
\authorfootnote{1328}Cardan thinks), \bookcite{\textlatin{contradict. med. lib.
1. contradict. 18}}, or if the parents be sick, or have any great pain of the
head, or megrim, headache, (Hieronymus Wolfius \authorfootnote{1329}doth
instance in a child of Sebastian Castalio's); if a drunken man get a child, it
will never likely have a good brain, as Gellius argues,
\bookcite{\textlatin{lib. 12. cap. 1.}} \li{Ebrii gignunt Ebrios}, one drunkard
begets another, saith \authorfootnote{1330}\Plutarch{}, \bookcite{\textlatin{symp.
lib. 1. quest. 5}}, whose sentence \authorfootnote{1331}Lemnius approves,
\bookcite{\textlatin{l. 1. c. 4.}} Alsarius Crutius, \bookcite{\textlatin{Gen.
de qui sit med. cent. 3. fol. 182.}} Macrobius, \bookcite{\textlatin{lib. 1.}}
\Avicenna{}, \bookcite{\textlatin{lib. 3. Fen. 21. Tract 1. cap. 8}}, and
\Aristotle{} himself, \bookcite{\textlatin{sect. 2. prob. 4}}, foolish, drunken,
or hair-brain women, most part bring forth children like unto themselves,
\li{morosos et languidos}, and so likewise he that lies with a menstruous
woman. \li{Intemperantia veneris, quam in nautis praesertim insectatur
\authorfootnote{1332}Lemnius, qui uxores ineunt, nulla menstrui decursus
ratione habita nec observato interlunio, praecipua causa est, noxia,
pernitiosa, concubitum hunc exitialem ideo, et pestiferum vocat.
\authorfootnote{1333}Rodoricus a Castro Lucitanus, detestantur ad unum omnes
medici, tum et quarta luna concepti, infelices plerumque et amentes, deliri,
stolidi, morbosi, impuri, invalidi, tetra lue sordidi minime vitales, omnibus
bonis corporis atque animi destituti: ad laborem nati, si seniores, inquit
Eustathius, ut Hercules, et alii. \authorfootnote{1334}Judaei maxime
insectantur foedum hunc, et immundum apud Christianas Concubitum, ut illicitum
abhorrent, et apud suos prohibent; et quod Christiani toties leprosi, amentes,
tot morbili, impetigines, alphi, psorae, cutis et faciei decolorationes, tam
multi morbi epidemici, acerbi, et venenosi sint, in hunc immundum concubitum
rejiciunt, et crudeles in pignora vocant, qui quarta, luna profluente hac
mensium illuvie concubitum hunc non perhorrescunt. Damnavit olim divina Lex et
morte mulctavit hujusmodi homines, \biblecite{Lev. 18, 20}, et inde nati, siqui
deformes aut mutili, pater dilapidatus, quod non contineret ab
\authorfootnote{1335}immunda muliere. Gregorius Magnus, petenti Augustino
nunquid apud \authorfootnote{1336}Britannos hujusmodi concubitum toleraret,
severe prohibuit viris suis tum misceri foeminas in consuetis suis menstruis},
\etc{} I spare to English this which I have said. Another cause some give,
inordinate diet, as if a man eat garlic, onions, fast overmuch, study too hard,
be over-sorrowful, dull, heavy, dejected in mind, perplexed in his thoughts,
fearful, \etc{}, "their children" (saith \authorfootnote{1337}Cardan
\bookcite{\textlatin{subtil. lib. 18}}) "will be much subject to madness and
melancholy; for if the spirits of the brain be fuzzled, or misaffected by such
means, at such a time, their children will be fuzzled in the brain: they will
be dull, heavy, timorous, discontented all their lives." Some are of opinion,
and maintain that paradox or problem, that wise men beget commonly fools;
Suidas gives instance in Aristarchus the Grammarian, \li{duos reliquit Filios
Aristarchum et Aristachorum, ambos stultos}; and which
\authorfootnote{1338}Erasmus urgeth in his \bookcite{\textlatin{Moria}}, fools
beget wise men. Card. \bookcite{\textlatin{subt. l. 12}}, gives this cause,
\li{Quoniam spiritus sapientum ob studium resolvuntur, et in cerebrum feruntur
a corde}: because their natural spirits are resolved by study, and turned into
animal; drawn from the heart, and those other parts to the brain. Lemnius
subscribes to that of Cardan, and assigns this reason, \li{Quod persolvant
debitum languide, et obscitanter, unde foetus a parentum generositate
desciscit}: they pay their debt (as Paul calls it) to their wives remissly, by
which means their children are weaklings, and many times idiots and fools.

Some other causes are given, which properly pertain, and do proceed from the
mother: if she be over-dull, heavy, angry, peevish, discontented, and
melancholy, not only at the time of conception, but even all the while she
carries the child in her womb (saith Fernelius, \bookcite{\textlatin{path. l.
1, 11}}) her son will be so likewise affected, and worse, as
\authorfootnote{1339}Lemnius adds, \bookcite{\textlatin{l. 4. c. 7}}, if she
grieve overmuch, be disquieted, or by any casualty be affrighted and terrified
by some fearful object, heard or seen, she endangers her child, and spoils the
temperature of it; for the strange imagination of a woman works effectually
upon her infant, that as Baptista Porta proves, \bookcite{\textlatin{Physiog.
caelestis l. 5. c. 2}}, she leaves a mark upon it, which is most especially
seen in such as prodigiously long for such and such meats, the child will love
those meats, saith Fernelius, and be addicted to like humours:
\authorfootnote{1340}"if a great-bellied woman see a hare, her child will often
have a harelip," as we call it. Garcaeus, \bookcite{\textlatin{de Judiciis
geniturarum, cap. 33}}, hath a memorable example of one Thomas Nickell, born in
the city of Brandeburg, 1551, \authorfootnote{1341}"that went reeling and
staggering all the days of his life, as if he would fall to the ground, because
his mother being great with child saw a drunken man reeling in the street."
Such another I find in Martin Wenrichius, \bookcite{\textlatin{com. de ortu
monstrorum, c. 17}}, I saw (saith he) at Wittenberg, in Germany, a citizen that
looked like a carcass; I asked him the cause, he replied,
\authorfootnote{1342}"His mother, when she bore him in her womb, saw a carcass
by chance, and was so sore affrighted with it, that \li{ex eo foetus ei
assimilatus}, from a ghastly impression the child was like it."

So many several ways are we plagued and punished for our father's defaults;
insomuch that as Fernelius truly saith, \authorfootnote{1343}"It is the
greatest part of our felicity to be well born, and it were happy for human
kind, if only such parents as are sound of body and mind should be suffered to
marry." An husbandman will sow none but the best and choicest seed upon his
land, he will not rear a bull or a horse, except he be right shapen in all
parts, or permit him to cover a mare, except he be well assured of his breed;
we make choice of the best rams for our sheep, rear the neatest kine, and keep
the best dogs, \li{Quanto id diligentius in procreandis liberis observandum}?
And how careful then should we be in begetting of our children? In former times
some \authorfootnote{1344}countries have been so chary in this behalf, so
stern, that if a child were crooked or deformed in body or mind, they made him
away; so did the Indians of old by the relation of Curtius, and many other
well-governed commonwealths, according to the discipline of those times.
Heretofore in Scotland, saith \authorfootnote{1345}Hect. Boethius, "if any were
visited with the falling sickness, madness, gout, leprosy, or any such
dangerous disease, which was likely to be propagated from the father to the
son, he was instantly gelded; a woman kept from all company of men; and if by
chance having some such disease, she were found to be with child, she with her
brood were buried alive:" and this was done for the common good, lest the whole
nation should be injured or corrupted. A severe doom you will say, and not to
be used amongst Christians, yet more to be looked into than it is. For now by
our too much facility in this kind, in giving way for all to marry that will,
too much liberty and indulgence in tolerating all sorts, there is a vast
confusion of hereditary diseases, no family secure, no man almost free from
some grievous infirmity or other, when no choice is had, but still the eldest
must marry, as so many stallions of the race; or if rich, be they fools or
dizzards, lame or maimed, unable, intemperate, dissolute, exhaust through riot,
as he said, \authorfootnote{1346}\li{jura haereditario sapere jubentur}; they
must be wise and able by inheritance: it comes to pass that our generation is
corrupt, we have many weak persons, both in body and mind, many feral diseases
raging amongst us, crazed families, \li{parentes, peremptores}; our fathers
bad, and we are like to be worse.

%\chapter{ MEMB. II.}
%SECT. II. MEMB. II.
%SECT. II. MEMB. II. SUBSECT. I.-_Bad Diet a cause. Substance. Quality of Meats_.
\section[Bad Diet]{Bad Diet a cause. Substance. Quality of Meats.}\label{sec:bad-diet}

\lettrine{A}{ccording} to my proposed method, having opened hitherto these
secondary causes, which are inbred with us, I must now proceed to the outward
and adventitious, which happen unto us after we are born. And those are either
evident, remote, or inward, antecedent, and the nearest: continent causes some
call them. These outward, remote, precedent causes are subdivided again into
necessary and not necessary. Necessary (because we cannot avoid them, but they
will alter us, as they are used, or abused) are those six non-natural things,
so much spoken of amongst physicians, which are principal causes of this
disease. For almost in every consultation, whereas they shall come to speak of
the causes, the fault is found, and this most part objected to the patient;
\li{Peccavit circa res sex non naturales}: he hath still offended in one of
those six. Montanus, \bookcite{\textlatin{consil. 22}}, consulted about a
melancholy Jew, gives that sentence, so did Frisemelica in the same place; and
in his 244 counsel, censuring a melancholy soldier, assigns that reason of his
malady, \authorfootnote{1347}"he offended in all those six non-natural things,
which were the outward causes, from which came those inward obstructions;" and
so in the rest.

These six non-natural things are diet, retention and evacuation, which are more
material than the other because they make new matter, or else are conversant in
keeping or expelling of it. The other four are air, exercise, sleeping, waking,
and perturbations of the mind, which only alter the matter. The first of these
is diet, which consists in meat and drink, and causeth melancholy, as it
offends in substance, or accidents, that is, quantity, quality, or the like.
And well it may be called a material cause, since that, as
\authorfootnote{1348}Fernelius holds, "it hath such a power in begetting of
diseases, and yields the matter and sustenance of them; for neither air, nor
perturbations, nor any of those other evident causes take place, or work this
effect, except the constitution of body, and preparation of humours, do concur.
That a man may say, this diet is the mother of diseases, let the father be what
he will, and from this alone, melancholy and frequent other maladies arise."
Many physicians, I confess, have written copious volumes of this one subject,
of the nature and qualities of all manner of meats; as namely, Galen, Isaac the
Jew, Halyabbas, \Avicenna{}, Mesue, also four Arabians, Gordonius, Villanovanus,
Wecker, Johannes Bruerinus, \li{sitologia de Esculentis et Poculentis}, Michael
Savanarola, \bookcite{\textlatin{Tract 2. c. 8}}, Anthony Fumanellus,
\bookcite{\textlatin{lib. de regimine senum}}, Curio in his comment on Schola
Salerna, Godefridus Steckius \bookcite{\textlatin{arte med.}}, Marcilius
Cognatus, Ficinus, Ranzovius, Fonseca, Lessius, Magninus,
\bookcite{\textlatin{regim. sanitatis}}, Frietagius, Hugo Fridevallius, \etc{},
besides many other in \authorfootnote{1349}English, and almost every peculiar
physician, discourseth at large of all peculiar meats in his chapter of
melancholy: yet because these books are not at hand to every man, I will
briefly touch what kind of meats engender this humour, through their several
species, and which are to be avoided. How they alter and change the matter,
spirits first, and after humours, by which we are preserved, and the
constitution of our body, Fernelius and others will show you. I hasten to the
thing itself: and first of such diet as offends in substance.

\subsection{Beef.}
Beef, a strong and hearty meat (cold in the first degree, dry in the second,
saith \bookcite{\textlatin{Gal. l. 3. c. 1. de alim. fac.}}) is condemned by
him and all succeeding Authors, to breed gross melancholy blood: good for such
as are sound, and of a strong constitution, for labouring men if ordered
aright, corned, young, of an ox (for all gelded meats in every species are held
best), or if old, \authorfootnote{1350}such as have been tired out with labour,
are preferred. Aubanus and Sabellicus commend Portugal beef to be the most
savoury, best and easiest of digestion; we commend ours: but all is rejected,
and unfit for such as lead a resty life, any ways inclined to melancholy, or
dry of complexion: \li{Tales} (Galen thinks) \li{de facile melancholicis
aegritudinibus capiuntur}.

\subsection{Pork.}
Pork, of all meats, is most nutritive in his own nature,
\authorfootnote{1351}but altogether unfit for such as live at ease, are any
ways unsound of body or mind: too moist, full of humours, and therefore
\li{noxia delicatis}, saith Savanarola, \li{ex earum usu ut dubitetur an febris
quartana generetur}: naught for queasy stomachs, insomuch that frequent use of
it may breed a quartan ague.

\begin{figure}[H]
  \begingroup
  \centering
  \includegraphics[keepaspectratio,width=\textwidth]{wild-boar-small.jpg}
  \captionart{WildBoar}
  \label{fig:wildboar}
\end{figure}

\subsection{Goat.}
Savanarola discommends goat's flesh, and so doth
\authorfootnote{1352}Bruerinus, \bookcite{\textlatin{l. 13. c. 19}}, calling it
a filthy beast, and rammish: and therefore supposeth it will breed rank and
filthy substance; yet kid, such as are young and tender, Isaac accepts,
Bruerinus and Galen, \bookcite{\textlatin{l. 1. c. 1. de alimentorum
facultatibus}}.

\subsection{Hart.}
Hart and red deer \authorfootnote{1353}hath an evil name: it yields gross
nutriment: a strong and great grained meat, next unto a horse. Which although
some countries eat, as Tartars, and they of China; yet
\authorfootnote{1354}Galen condemns. Young foals are as commonly eaten in Spain
as red deer, and to furnish their navies, about Malaga especially, often used;
but such meats ask long baking, or seething, to qualify them, and yet all will
not serve.

\subsubsection{Venison, Fallow Deer.} All venison is melancholy, and begets bad
blood; a pleasant meat: in great esteem with us (for we have more parks in
England than there are in all Europe besides) in our solemn feasts. 'Tis
somewhat better hunted than otherwise, and well prepared by cookery; but
generally bad, and seldom to be used.

\subsection{Hare.}
Hare, a black meat, melancholy, and hard of digestion, it breeds incubus, often
eaten, and causeth fearful dreams, so doth all venison, and is condemned by a
jury of physicians. Mizaldus and some others say, that hare is a merry meat,
and that it will make one fair, as Martial's epigram testifies to Gellia; but
this is \li{per accidens}, because of the good sport it makes, merry company
and good discourse that is commonly at the eating of it, and not otherwise to
be understood.

\subsection{Conies.}
\authorfootnote{1355}Conies are of the nature of hares. Magninus compares them
to beef, pig, and goat, \bookcite{\textlatin{Reg. sanit. part. 3. c. 17}}; yet
young rabbits by all men are approved to be good.

Generally, all such meats as are hard of digestion breed melancholy. Areteus,
\bookcite{\textlatin{lib. 7. cap. 5}}, reckons up heads and feet,
\authorfootnote{1356}bowels, brains, entrails, marrow, fat, blood, skins, and
those inward parts, as heart, lungs, liver, spleen, \etc{} They are rejected by
Isaac, \bookcite{\textlatin{lib. 2. part. 3}}, Magninus,
\bookcite{\textlatin{part. 3. cap. 17}}, Bruerinus, \bookcite{\textlatin{lib.
12}}, Savanarola, \bookcite{\textlatin{Rub. 32. Tract. 2.}}

\subsection{Milk.}
Milk, and all that comes of milk, as butter and cheese, curds, \etc{}, increase
melancholy (whey only excepted, which is most wholesome):
\authorfootnote{1357}some except asses' milk. The rest, to such as are sound,
is nutritive and good, especially for young children, but because soon turned
to corruption, \authorfootnote{1358}not good for those that have unclean
stomachs, are subject to headache, or have green wounds, stone, \etc{} Of all
cheeses, I take that kind which we call Banbury cheese to be the best, \li{ex
vetustis pessimus}, the older, stronger, and harder, the worst, as Langius
discourseth in his Epistle to Melancthon, cited by Mizaldus, Isaac,
\bookcite{\textlatin{p. 5. Gal. 3. de cibis boni succi}}. \etc{}

\subsection{Fowl.}
Amongst fowl, \authorfootnote{1359}peacocks and pigeons, all fenny fowl are
forbidden, as ducks, geese, swans, herons, cranes, coots, didappers,
water-hens, with all those teals, curs, sheldrakes, and peckled fowls, that
come hither in winter out of Scandia, Muscovy, Greenland, Friesland, which half
the year are covered all over with snow, and frozen up. Though these be fair in
feathers, pleasant in taste, and have a good outside, like hypocrites, white in
plumes, and soft, their flesh is hard, black, unwholesome, dangerous,
melancholy meat; \li{Gravant et putrefaciant stomachum}, saith Isaac,
\bookcite{\textlatin{part. 5. de vol.}}, their young ones are more tolerable,
but young pigeons he quite disapproves.

\begin{figure}[H]
  \begingroup
  \centering
  \includegraphics[keepaspectratio,width=\textwidth]{chickens-small.jpg}
  \captionart{Chickens}
  \label{fig:chickens}
\end{figure}

\subsection{Fishes.}\label{sec:fishes}
Rhasis and \authorfootnote{1360}Magninus discommend all fish, and say, they
breed viscosities, slimy nutriment, little and humorous nourishment. Savanarola
adds, cold, moist: and phlegmatic, Isaac; and therefore unwholesome for all
cold and melancholy complexions: others make a difference, rejecting only
amongst freshwater fish, eel, tench, lamprey, crawfish (which Bright approves,
\bookcite{\textlatin{cap. 6}}), and such as are bred in muddy and standing
waters, and have a taste of mud, as Franciscus Bonsuetus poetically defines,
\bookcite{\textlatin{Lib. de aquatilibus}}.

\begin{figure}[H]
  \begingroup
  \centering
  \includegraphics[keepaspectratio,width=0.7\textwidth]{Three-fishes-arranged-crosswise-small.jpg}
  \captionart{ThreeFishes}
  \label{fig:threefishes}
\end{figure}

\translatedverse{%
\begin{latin}
\begin{verse}
Nam pisces omnes, qui stagna, lacusque frequentant,\\*
Semper plus succi deterioris habent.\\!
\end{verse}
\end{latin}}{%
\begin{verse}
All fish, that standing pools, and lakes frequent,\\*
Do ever yield bad juice and nourishment.\\!
\end{verse}}{}

Lampreys, Paulus Jovius, \bookcite{\textlatin{c. 34. de piscibus fluvial.}},
highly magnifies, and saith, None speak against them, but \li{inepti et
scrupulosi}, some scrupulous persons; but \authorfootnote{1361}eels,
\bookcite{\textlatin{c. 33}}, "he abhorreth in all places, at all times, all
physicians detest them, especially about the solstice." Gomesius,
\bookcite{\textlatin{lib. 1. c. 22, de sale}}, doth immoderately extol
sea-fish, which others as much vilify, and above the rest, dried, soused,
indurate fish, as ling, fumados, red-herrings, sprats, stock-fish, haberdine,
poor-John, all shellfish. \authorfootnote{1362}Tim. Bright excepts lobster and
crab. Messarius commends salmon, which Bruerinus contradicts,
\bookcite{\textlatin{lib. 22. c. 17.}} Magninus rejects conger, sturgeon,
turbot, mackerel, skate.

Carp is a fish of which I know not what to determine. Franciscus Bonsuetus
accounts it a muddy fish. Hippolitus Salvianus, in his Book
\bookcite{\textlatin{de Piscium natura et praeparatione}}, which was printed at
Rome in folio, 1554, with most elegant pictures, esteems carp no better than a
slimy watery meat. Paulus Jovius on the other side, disallowing tench, approves
of it; so doth Dubravius in his Books of Fishponds. Freitagius
\authorfootnote{1363}extols it for an excellent wholesome meat, and puts it
amongst the fishes of the best rank; and so do most of our country gentlemen,
that store their ponds almost with no other fish. But this controversy is
easily decided, in my judgment, by Bruerinus, \bookcite{\textlatin{l. 22. c.
13.}} The difference riseth from the site and nature of pools,
\authorfootnote{1364}sometimes muddy, sometimes sweet; they are in taste as the
place is from whence they be taken. In like manner almost we may conclude of
other fresh fish. But see more in Rondoletius, Bellonius, Oribasius,
\bookcite{\textlatin{lib. 7. cap. 22}}, Isaac, \bookcite{\textlatin{l. 1}},
especially Hippolitus Salvianus, who is \li{instar omnium solus}, \etc{}
Howsoever they may be wholesome and approved, much use of them is not good; P.
Forestus, in his medicinal observations, \authorfootnote{1365}relates, that
Carthusian friars, whose living is most part fish, are more subject to
melancholy than any other order, and that he found by experience, being
sometimes their physician ordinary at Delft, in Holland. He exemplifies it with
an instance of one Buscodnese, a Carthusian of a ruddy colour, and well liking,
that by solitary living, and fish-eating, became so misaffected.

\subsection{Herbs.}

Amongst herbs to be eaten I find gourds, cucumbers, coleworts, melons,
disallowed, but especially cabbage. It causeth troublesome dreams, and sends up
black vapours to the brain. Galen, \bookcite{\textlatin{loc. affect. l. 3. c.
6}}, of all herbs condemns cabbage; and Isaac, \bookcite{\textlatin{lib. 2. c.
1.}} \li{Animae gravitatem facit}, it brings heaviness to the soul. Some are of
opinion that all raw herbs and salads breed melancholy blood, except bugloss
and lettuce. Crato, \bookcite{\textlatin{consil. 21. lib. 2}}, speaks against
all herbs and worts, except borage, bugloss, fennel, parsley, dill, balm,
succory. Magninus, \bookcite{\textlatin{regim. sanitatis, part. 3. cap. 31.}}
\li{Omnes herbae simpliciter malae, via cibi}; all herbs are simply evil to
feed on (as he thinks). So did that scoffing cook in
\authorfootnote{1366}Plautus hold:

\translatedverse{%
\begin{latin}
\begin{verse}
Non ego coenam condio ut alii coqui solent,\\*
Qui mihi condita prata in patinis proferunt,\\*
Boves qui convivas faciunt, herbasque aggerunt.\\!
\end{verse}
\end{latin}}{%
\begin{verse}%
Like other cooks I do not supper dress,\\*
That put whole meadows into a platter,\\*
And make no better of their guests than beeves,\\*
With herbs and grass to feed them fatter.\\!
\end{verse}}{}%

Our Italians and Spaniards do make a whole dinner of herbs and salads (which
our said Plautus calls \li{coenas terrestras}, \Horace{}, \li{coenas sine
sanguine}), by which means, as he follows it,

\translatedverse{%
\begin{latin}
\begin{verse}
Hic homines tam brevem vitam colunt--\\*
Qui herbas hujusmodi in alvum suum congerunt,\\*
Formidolosum dictu, non esu modo,\\*
Quas herbas pecudes non edunt, homines edunt.\\!
\end{verse}
\end{latin}}{%
\begin{verse}%
Their lives, that eat such herbs, must needs be short,\\*
And 'tis a fearful thing for to report,\\*
That men should feed on such a kind of meat,\\*
Which very juments would refuse to eat.\\!
\end{verse}}{%
\attrib{\getauthornote{1367}}}

\authorfootnote{1368}They are windy, and not fit therefore to be eaten of all
men raw, though qualified with oil, but in broths, or otherwise. See more of
these in every \authorfootnote{1369}husbandman, and herbalist.

\subsection{Roots.}
Roots, \li{Etsi quorundam gentium opes sint}, saith Bruerinus, the wealth of
some countries, and sole food, are windy and bad, or troublesome to the head:
as onions, garlic, scallions, turnips, carrots, radishes, parsnips: Crato,
\bookcite{\textlatin{lib. 2. consil. 11}}, disallows all roots, though
\authorfootnote{1370}some approve of parsnips and potatoes.
\authorfootnote{1371}Magninus is of Crato's opinion, \authorfootnote{1372}"They
trouble the mind, sending gross fumes to the brain, make men mad," especially
garlic, onions, if a man liberally feed on them a year together. Guianerius,
\bookcite{\textlatin{tract. 15. cap. 2}}, complains of all manner of roots, and
so doth Bruerinus, even parsnips themselves, which are the best,
\bookcite{\textlatin{Lib. 9. cap. 14.}}

\subsection{Fruits.}
\li{Pastinacarum usus succos gignit improbos}. Crato,
\bookcite{\textlatin{consil. 21. lib. 1}}, utterly forbids all manner of
fruits, as pears, apples, plums, cherries, strawberries, nuts, medlars, serves,
\etc{} \li{Sanguinem inficiunt}, saith Villanovanus, they infect the blood, and
putrefy it, Magninus holds, and must not therefore be taken \li{via cibi, aut
quantitate magna}, not to make a meal of, or in any great quantity.
\authorfootnote{1373}Cardan makes that a cause of their continual sickness at
Fessa in Africa, "because they live so much on fruits, eating them thrice a
day." Laurentius approves of many fruits, in his Tract of Melancholy, which
others disallow, and amongst the rest apples, which some likewise commend,
sweetings, pearmains, pippins, as good against melancholy; but to him that is
any way inclined to, or touched with this malady, \authorfootnote{1374}Nicholas
Piso in his Practics, forbids all fruits, as windy, or to be sparingly eaten at
least, and not raw. Amongst other fruits, \authorfootnote{1375}Bruerinus, out
of Galen, excepts grapes and figs, but I find them likewise rejected.

\subsection{Pulse.}
All \worddef{Dried legume}{pulse} are naught, beans, peas, vetches, \etc{},
they fill the brain (saith Isaac) with gross fumes, breed black thick blood,
and cause troublesome dreams. And therefore, that which Pythagoras said to his
scholars of old, may be for ever applied to melancholy men, \li{A fabis
abstinete}, eat no peas, nor beans; yet to such as will needs eat them, I would
give this counsel, to prepare them according to those rules that Arnoldus
Villanovanus, and Frietagius prescribe, for eating, and dressing. fruits,
herbs, roots, pulse, \etc{}

\subsection{Spices.}
Spices cause hot and head melancholy, and are for that cause forbidden by our
physicians to such men as are inclined to this malady, as pepper, ginger,
cinnamon, cloves, mace, dates, \etc{} honey and sugar.
\authorfootnote{1376}Some except honey; to those that are cold, it may be
tolerable, but \authorfootnote{1377}\lit{Dulcia se in bilem vertunt}{sweets turn
into bile}, they are obstructive. Crato therefore forbids all spice, in a
consultation of his, for a melancholy schoolmaster, \li{Omnia aromatica et
quicquid sanguinem adurit}: so doth Fernelius, \bookcite{\textlatin{consil.
45.}} Guianerius, \bookcite{\textlatin{tract 15. cap. 2.}} Mercurialis,
\bookcite{\textlatin{cons. 189.}} To these I may add all sharp and sour things,
luscious and over-sweet, or fat, as oil, vinegar, verjuice, mustard, salt; as
sweet things are obstructive, so these are corrosive. Gomesius, in his books,
\bookcite{\textlatin{de sale, l. 1. c. 21}}, highly commends salt; so doth
Codronchus in his tract, \bookcite{\textlatin{de sale Absynthii}}, Lemn.
\bookcite{\textlatin{l. 3. c. 9. de occult. nat. mir.}} yet common experience
finds salt, and salt-meats, to be great procurers of this disease. And for that
cause belike those Egyptian priests abstained from salt, even so much, as in
their bread, \li{ut sine perturbatione anima esset}, saith mine author, that
their souls might be free from perturbations.

\subsection{Bread.}

Bread that is made of baser grain, as peas, beans, oats, rye, or
\authorfootnote{1378}over-hard baked, crusty, and black, is often spoken
against, as causing melancholy juice and wind. Joh. Mayor, in the first book of
his History of Scotland, contends much for the wholesomeness of oaten bread: it
was objected to him then living at Paris in France, that his countrymen fed on
oats, and base grain, as a disgrace; but he doth ingenuously confess, Scotland,
Wales, and a third part of England, did most part use that kind of bread, that
it was as wholesome as any grain, and yielded as good nourishment. And yet
Wecker out of Galen calls it horsemeat, and fitter for juments than men to feed
on. But read Galen himself, \bookcite{\textlatin{Lib. 1. De cibis boni et mali
succi}}, more largely discoursing of corn and bread.

\subsection{Wine.}

All black wines, over-hot, compound, strong thick drinks, as Muscadine,
Malmsey, Alicant, Rumney, Brownbastard, Metheglen, and the like, of which they
have thirty several kinds in Muscovy, all such made drinks are hurtful in this
case, to such as are hot, or of a sanguine choleric complexion, young, or
inclined to head-melancholy. For many times the drinking of wine alone causeth
it. Arculanus, \bookcite{\textlatin{c. 16. in 9. Rhasis}}, puts in
\authorfootnote{1379}wine for a great cause, especially if it be immoderately
used. Guianerius, \bookcite{\textlatin{tract. 15. c. 2}}, tells a story of two
Dutchmen, to whom he gave entertainment in his house, "that
\authorfootnote{1380}in one month's space were both melancholy by drinking of
wine, one did nought but sing, the other sigh." Galen, \bookcite{\textlatin{l.
de causis morb. c. 3.}} Matthiolus on Dioscorides, and above all other Andreas
Bachius, \bookcite{\textlatin{l. 3. 18, 19, 20}}, have reckoned upon those
inconveniences that come by wine: yet notwithstanding all this, to such as are
cold, or sluggish melancholy, a cup of wine is good physic, and so doth
Mercurialis grant, \bookcite{\textlatin{consil. 25}}, in that case, if the
temperature be cold, as to most melancholy men it is, wine is much commended,
if it be moderately used.

\subsection{Cider, Perry.}

Cider and perry are both cold and windy drinks, and for that cause to be
neglected, and so are all those hot spiced strong drinks.

\subsubsection{Beer.}

Beer, if it be over-new or over-stale, over-strong, or not sodden, smell of the
cask, sharp, or sour, is most unwholesome, frets, and galls, \etc{} Henricus
Ayrerus, in a \authorfootnote{1381}consultation of his, for one that laboured
of hypochondriacal melancholy, discommends beer. So doth
\authorfootnote{1382}Crato in that excellent counsel of his,
\bookcite{\textlatin{Lib. 2. consil. 21}}, as too windy, because of the hop.
But he means belike that thick black Bohemian beer used in some other parts of
\authorfootnote{1383}Germany.

\translatedverse{%
\begin{latin}
\begin{verse}
------nil spissius illa\\*
Dum bibitur, nil clarius est dum mingitur, unde\\*
Constat, quod multas faeces in corpore linquat.\\!
\end{verse}
\end{latin}}{%
\begin{verse}%
Nothing comes in so thick,\\*
Nothing goes out so thin,\\*
It must needs follow then\\*
The dregs are left within.\\!
\end{verse}}{}

As that \authorfootnote{1384}old poet scoffed, calling it \li{Stygiae monstrum
conforme paludi}, a monstrous drink, like the river Styx. But let them say as
they list, to such as are accustomed unto it, "'tis a most wholesome" (so
\authorfootnote{1385}\idxname{polydorevergil}[Polydore Virgil] calleth it) "and
a pleasant drink," it is more subtle and better, for the hop that rarefies it,
hath an especial virtue against melancholy, as our herbalists confess, Fuchsius
approves, \bookcite{\textlatin{Lib. 2. sec. 2. instit. cap. 11}}, and many
others.

\subsubsection{Waters.}

Standing waters, thick and ill-coloured, such as come forth of pools, and
moats, where hemp hath been steeped, or slimy fishes live, are most
unwholesome, putrefied, and full of mites, creepers, slimy, muddy, unclean,
corrupt, impure, by reason of the sun's heat, and still-standing; they cause
foul distemperatures in the body and mind of man, are unfit to make drink of,
to dress meat with, or to be \authorfootnote{1386}used about men inwardly or
outwardly. They are good for many domestic uses, to wash horses, water cattle,
\etc{}, or in time of necessity, but not otherwise. Some are of opinion, that
such fat standing waters make the best beer, and that seething doth defecate
it, as \authorfootnote{1387}Cardan holds, \bookcite{\textlatin{Lib. 13.
subtil.}} "It mends the substance, and savour of it," but it is a paradox. Such
beer may be stronger, but not so wholesome as the other, as
\authorfootnote{1388}Jobertus truly justifieth out of Galen,
\bookcite{\textlatin{Paradox, dec. 1. Paradox 5}}, that the seething of such
impure waters doth not purge or purify them, Pliny, \bookcite{\textlatin{lib.
31. c. 3}}, is of the same tenet, and P. Crescentius,
\bookcite{\textlatin{agricult. lib. 1. et lib. 4. c. 11. et c. 45.}} Pamphilius
Herilachus, \bookcite{\textlatin{l. 4. de not. aquarum}}, such waters are
naught, not to be used, and by the testimony of \authorfootnote{1389}Galen,
"breed agues, dropsies, pleurisies, splenetic and melancholy passions, hurt the
eyes, cause a bad temperature, and ill disposition of the whole body, with bad
colour." This Jobertus stiffly maintains, \bookcite{\textlatin{Paradox, lib. 1.
part. 5}}, that it causeth blear eyes, bad colour, and many loathsome diseases
to such as use it: this which they say, stands with good reason; for as
geographers relate, the water of Astracan breeds worms in such as drink it.
\authorfootnote{1390}Axius, or as now called Verduri, the fairest river in
Macedonia, makes all cattle black that taste of it. Aleacman now Peleca,
another stream in Thessaly, turns cattle most part white, \li{si polui ducas},
L. Aubanus Rohemus refers that \authorfootnote{1391}struma or poke of the
Bavarians and Styrians to the nature of their waters, as
\authorfootnote{1392}Munster doth that of Valesians in the Alps, and
\authorfootnote{1393}Bodine supposeth the stuttering of some families in
Aquitania, about Labden, to proceed from the same cause, "and that the filth is
derived from the water to their bodies." So that they that use filthy,
standing, ill-coloured, thick, muddy water, must needs have muddy,
ill-coloured, impure, and infirm bodies. And because the body works upon the
mind, they shall have grosser understandings, dull, foggy, melancholy spirits,
and be really subject to all manner of infirmities.

To these noxious simples, we may reduce an infinite number of compound,
artificial, made dishes, of which our cooks afford us a great variety, as
tailors do fashions in our apparel. Such are \authorfootnote{1394}puddings
stuffed with blood, or otherwise composed; baked, meats, soused indurate meats,
fried and broiled buttered meats; condite, powdered, and over-dried,
\authorfootnote{1395}all cakes, simnels, buns, cracknels made with butter,
spice, \etc{}, fritters, pancakes, pies, sausages, and those several sauces,
sharp, or over-sweet, of which \li{scientia popinae}, as \Seneca{} calls it, hath
served those \authorfootnote{1396}Apician tricks, and perfumed dishes, which
Adrian the sixth Pope so much admired in the accounts of his predecessor Leo
Decimus; and which prodigious riot and prodigality have invented in this age.
These do generally engender gross humours, fill the stomach with crudities, and
all those inward parts with obstructions. Montanus,
\bookcite{\textlatin{consil. 22}}, gives instance, in a melancholy Jew, that by
eating such tart sauces, made dishes, and salt meats, with which he was
overmuch delighted, became melancholy, and was evil affected. Such examples are
familiar and common.

%SECT. II. MEMB. II. SUBSECT. II.-_Quantity of Diet a Cause._
\section{Quantity of Diet a Cause.}

\lettrine{T}{here} is not so much harm proceeding from the substance itself of
meat, and quality of it, in ill-dressing and preparing, as there is from the
quantity, disorder of time and place, unseasonable use of it,
\authorfootnote{1397}intemperance, overmuch, or overlittle taking of it. A true
saying it is, \li{Plures crapula quam gladius}. This gluttony kills more than
the sword, this \li{omnivorantia et homicida gula}, this all-devouring and
murdering gut. And that of \authorfootnote{1398}Pliny is truer, "Simple diet is
the best; heaping up of several meats is pernicious, and sauces worse; many
dishes bring many diseases." \authorfootnote{1399}Avicen cries out, "That
nothing is worse than to feed on many dishes, or to protract the time of meats
longer than ordinary; from thence proceed our infirmities, and 'tis the
fountain of all diseases, which arise out of the repugnancy of gross humours."
Thence, saith \authorfootnote{1400}Fernelius, come crudities, wind,
oppilations, cacochymia, plethora, cachexia, bradiopepsia,
\authorfootnote{1401}\li{Hinc subitae, mortes, atque intestata senectus},
sudden death, \etc{}, and what not.

As a lamp is choked with a multitude of oil, or a little fire with overmuch
wood quite extinguished, so is the natural heat with immoderate eating,
strangled in the body. \li{Pernitiosa sentina est abdomen insaturabile}: one
saith, An insatiable paunch is a pernicious sink, and the fountain of all
diseases, both of body and mind. \authorfootnote{1402}Mercurialis will have it
a peculiar cause of this private disease; Solenander,
\bookcite{\textlatin{consil. 5. sect. 3}}, illustrates this of Mercurialis,
with an example of one so melancholy, \li{ab intempestivis commessationibus},
unseasonable feasting. \authorfootnote{1403}Crato confirms as much, in that
often cited counsel, \bookcite{\textlatin{21. lib. 2}}, putting superfluous
eating for a main cause. But what need I seek farther for proofs? Hear
\authorfootnote{1404}Hippocrates himself, \bookcite{\textlatin{lib. 2. aphor.
10}}, "Impure bodies the more they are nourished, the more they are hurt, for
the nourishment is putrefied with vicious humours."

\begin{figure}[H]
  \begingroup
  \centering
  \includegraphics[keepaspectratio,width=\textwidth]{heliogabalus-small.jpg}
  \captionart{Heliogabalus}
  \label{fig:heliogabalus}
\end{figure}

And yet for all this harm, which apparently follows surfeiting and drunkenness,
see how we luxuriate and rage in this kind; read what Johannes Stuckius hath
written lately of this subject, in his great volume \bookcite{\textlatin{De
Antiquorum Conviviis}}, and of our present age; \li{Quam
\authorfootnote{1405}portentosae coenae}, prodigious suppers, \li{Qui dum
invitant ad coenam efferunt ad
sepulchrum}\authorlatintrans{1406.5}\authorfootnote{1406}, what Fagos,
Epicures, Apetios, Heliogables, our times afford? Lucullus' ghost walks
still,\footnoteA{\scriptsize{}Roman politician and famed gastronome. As
recorded in \Plutarch{}'s Parallel Lives, on an occasion without dinner guests his
servant prepared a plain course. Lucullus reprimanded him: \lit{Quid ais,
inquit iratus Lucullus, au nesciebas Lucullum hodie cenaturum esse apud
Lucullum}{What, did not you know, then, that today Lucullus dines with
Lucullus?}. \Plutarch{}, Life of Lucullus, 41.1–6. \theeditor{}} and every man
desires to sup in Apollo; Aesop's costly dish is ordinarily served up.
\li{Magis illa juvant, quae pluris emuntur}\authorlatintrans{1407.5}.\authorfootnote{1407} The dearest
cates are best, and 'tis an ordinary thing to bestow twenty or thirty pounds on
a dish, some thousand crowns upon a dinner: \authorfootnote{1408}Mully-Hamet,
king of Fez and Morocco, spent three pounds on the sauce of a capon: it is
nothing in our times, we scorn all that is cheap. "We loathe the very
\authorfootnote{1409}light" (some of us, as \Seneca{} notes) "because it comes
free, and we are offended with the sun's heat, and those cool blasts, because
we buy them not." This air we breathe is so common, we care not for it; nothing
pleaseth but what is dear. And if we be \authorfootnote{1410}witty in anything,
it is \li{ad gulam}: If we study at all, it is \li{erudito luxu}, to please the
palate, and to satisfy the gut. "A cook of old was a base knave" (as
\authorfootnote{1411}Livy complains), "but now a great man in request; cookery
is become an art, a noble science: cooks are gentlemen:" \li{Venter Deus}: They
wear "their brains in their bellies, and their guts in their heads," as
\authorfootnote{1412}Agrippa taxed some parasites of his time, rushing on their
own destruction, as if a man should run upon the point of a sword, \li{usque
dum rumpantur comedunt}, "They eat till they burst:" \authorfootnote{1413}All
day, all night, let the physician say what he will, imminent danger, and feral
diseases are now ready to seize upon them, that will eat till they vomit,
\li{Edunt ut vomant, vomut ut edant}, saith \Seneca{}; which Dion relates of
Vitellius, \li{Solo transitu ciborum nutriri judicatus}: His meat did pass
through and away, or till they burst again. \authorfootnote{1414}\li{Strage
animantium ventrem onerant}, and rake over all the world, as so many
\authorfootnote{1415}slaves, belly-gods, and land-serpents, \li{Et totus orbis
ventri nimis angustus}, the whole world cannot satisfy their appetite.
\authorfootnote{1416}"Sea, land, rivers, lakes, \etc{}, may not give content to
their raging guts." To make up the mess, what immoderate drinking in every
place? \li{Senem potum pota trahebat anus}, how they flock to the tavern: as if
they were \li{fruges consumere nati}, born to no other end but to eat and
drink, like Offellius Bibulus, that famous Roman parasite, \li{Qui dum vixit,
aut bibit aut minxit}; as so many casks to hold wine, yea worse than a cask,
that mars wine, and itself is not marred by it, yet these are brave men,
Silenus Ebrius was no braver. \li{Et quae fuerunt vitia, mores sunt}: 'tis now
the fashion of our times, an honour: \li{Nunc vero res ista eo rediit} (as
Chrysost. \bookcite{\textlatin{serm. 30. in v. Ephes.}} comments) \li{Ut
effeminatae ridendaeque ignaviae loco habeatur, nolle inebriari}; 'tis now come
to that pass that he is no gentleman, a very milk-sop, a clown, of no bringing
up, that will not drink; fit for no company; he is your only gallant that plays
it off finest, no disparagement now to stagger in the streets, reel, rave,
\etc{}, but much to his fame and renown; as in like case Epidicus told Thesprio
his fellow-servant, in the \authorfootnote{1417}Poet. \li{Aedipol facinus
improbum}, one urged, the other replied, \li{At jam alii fecere idem, erit illi
illa res honori}, 'tis now no fault, there be so many brave examples to bear
one out; 'tis a credit to have a strong brain, and carry his liquor well; the
sole contention who can drink most, and fox his fellow the soonest. 'Tis the
\li{summum bonum} of our tradesmen, their felicity, life, and soul, \li{Tanta
dulcedine affectant}, saith Pliny, \bookcite{\textlatin{lib. 14. cap. 12.}}
\li{Ut magna pars non aliud vitae praemium intelligat}, their chief comfort, to
be merry together in an alehouse or tavern, as our modern Muscovites do in
their mead-inns, and Turks in their coffeehouses, which much resemble our
taverns; they will labour hard all day long to be drunk at night, and spend
\li{totius anni labores}, as St. Ambrose adds, in a tippling feast; convert day
into night, as \Seneca{} taxes some in his times, \li{Pervertunt officia anoctis
et lucis}; when we rise, they commonly go to bed, like our antipodes,

\begin{latin}
\begin{verse}%
Nosque ubi primus equis oriens afflavit anhelis,\\*
Illis sera rubens ascendit lumina vesper.\\!
\end{verse}%
\end{latin}

So did \Petronius in \idxname{Tacitus}[Tacitus][Annals], Heliogabalus in Lampridius.

\translatedverse{%
\begin{latin}
\begin{verse}
------Noctes vigilibat ad ipsum\\*
Mane, diem totum stertebat?------\\!
\end{verse}
\end{latin}}{%
\begin{verse}%
------He drank the night away\\*
Till rising dawn, then snored out all the day.\\!
\end{verse}}{%
\attrib{\getauthornote{1418}}}

Snymdiris the Sybarite never saw the sun rise or set so much as once in twenty
years. Verres, against whom \Tully{} so much inveighs, in winter he never was
\li{extra tectum vix extra lectum}, never almost out of bed,
\authorfootnote{1419}still wenching and drinking; so did he spend his time, and
so do myriads in our days. They have \li{gymnasia bibonum}, schools and
rendezvous; these centaurs and Lapithae toss pots and bowls as so many balls;
invent new tricks, as sausages, anchovies, tobacco, caviar, pickled oysters,
herrings, fumados, \etc{}: innumerable salt meats to increase their appetite,
and study how to hurt themselves by taking antidotes \authorfootnote{1420}"to
carry their drink the better; \authorfootnote{1421}and when nought else serves,
they will go forth, or be conveyed out, to empty their gorge, that they may
return to drink afresh." They make laws, \li{insanas leges, contra bibendi
fallacias}, and \authorfootnote{1422}brag of it when they have done, crowning
that man that is soonest gone, as their drunken predecessors have done, --
\authorfootnote{1423}\li{quid ego video}? Ps. \li{Cum corona Pseudolum ebrium
tuum} --. And when they are dead, will have a can of wine with
\authorfootnote{1424}Maron's old woman to be engraven on their tombs. So they
triumph in villainy, and justify their wickedness; with Rabelais, that French
Lucian, drunkenness is better for the body than physic, because there be more
old drunkards than old physicians. Many such frothy arguments they have,
\authorfootnote{1425}inviting and encouraging others to do as they do, and love
them dearly for it (no glue like to that of good fellowship). So did Alcibiades
in Greece; Nero, Bonosus, Heliogabalus in Rome, or Alegabalus rather, as he was
styled of old (as \authorfootnote{1426}Ignatius proves out of some old coins).
So do many great men still, as \authorfootnote{1427}Heresbachius observes. When
a prince drinks till his eyes stare, like Bitias in the Poet,

\translatedverse{%
\begin{latin}
\begin{verse}%
------(ille impiger hausit\\*
Spumantem vino pateram.)\\!
\end{verse}%
\end{latin}}{%
\begin{verse}%
------a thirsty soul;\\*
He took challenge and embrac'd the bowl;\\*
With pleasure swill'd the gold, nor ceased to draw\\*
Till he the bottom of the brimmer saw.\\!
\end{verse}}{%
\attrib{\getauthornote{1428}}}

and comes off clearly, sound trumpets, fife and drums, the spectators will
applaud him, "the \authorfootnote{1429}bishop himself (if he belie them not)
with his chaplain will stand by and do as much," \li{O dignum principe
haustum}, 'twas done like a prince. "Our Dutchmen invite all comers with a pail
and a dish," \li{Velut infundibula integras obbas exhauriunt, et in monstrosis
poculis, ipsi monstrosi monstrosius epotant}, "making barrels of their
bellies." \li{Incredibile dictu}, as \authorfootnote{1430}one of their own
countrymen complains: \authorfootnote{1431}\li{Quantum liquoris immodestissima
gens capiat}, \etc{} "How they love a man that will be drunk, crown him and
honour him for it," hate him that will not pledge him, stab him, kill him: a
most intolerable offence, and not to be forgiven. \authorfootnote{1432}"He is a
mortal enemy that will not drink with him," as Munster relates of the Saxons.
So in Poland, he is the best servitor, and the honestest fellow, saith
Alexander Gaguinus, \authorfootnote{1433}"that drinketh most healths to the
honour of his master, he shall be rewarded as a good servant, and held the
bravest fellow that carries his liquor best," when a brewer's horse will bear
much more than any sturdy drinker, yet for his noble exploits in this kind, he
shall be accounted a most valiant man, for \authorfootnote{1434}\li{Tam inter
epulas fortis vir esse potest ac in bello}, as much valour is to be found in
feasting as in fighting, and some of our city captains, and carpet knights will
make this good, and prove it. Thus they many times wilfully pervert the good
temperature of their bodies, stifle their wits, strangle nature, and degenerate
into beasts.

Some again are in the other extreme, and draw this mischief on their heads by
too ceremonious and strict diet, being over-precise, cockney-like, and curious
in their observation of meats, times, as that \li{Medicina statica} prescribes,
just so many ounces at dinner, which Lessius enjoins, so much at supper, not a
little more, nor a little less, of such meat, and at such hours, a diet-drink
in the morning, cock-broth, China-broth, at dinner, plum-broth, a chicken, a
rabbit, rib of a rack of mutton, wing of a capon, the merry-thought of a hen,
\etc{}; to sounder bodies this is too nice and most absurd. Others offend in
overmuch fasting: pining adays, saith \authorfootnote{1435}Guianerius, and
waking anights, as many Moors and Turks in these our times do. "Anchorites,
monks, and the rest of that superstitious rank (as the same Guianerius
witnesseth, that he hath often seen to have happened in his time) through
immoderate fasting, have been frequently mad." Of such men belike Hippocrates
speaks, \bookcite{\textlatin{l. Aphor. 5}}, when as he saith,
\authorfootnote{1436}"they more offend in too sparing diet, and are worse
damnified, than they that feed liberally, and are ready to surfeit."

%SECT. II. MEMB. II. SUBSECT. III.-_Custom of Diet, Delight, Appetite, Necessity, how they cause or hinder_.
\section{Custom of Diet, Delight, Appetite, Necessity, how they cause or hinder.}

\lettrine{N}{o} rule is so general, which admits not some exception; to this,
therefore, which hath been hitherto said, (for I shall otherwise put most men
out of commons,) and those inconveniences which proceed from the substance of
meats, an intemperate or unseasonable use of them, custom somewhat detracts and
qualifies, according to that of Hippocrates, \bookcite{\textlatin{2 Aphoris.
50.}} \authorfootnote{1437}"Such things as we have been long accustomed to,
though they be evil in their own nature, yet they are less offensive."
Otherwise it might well be objected that it were a mere
\authorfootnote{1438}tyranny to live after those strict rules of physic; for
custom \authorfootnote{1439}doth alter nature itself, and to such as are used
to them it makes bad meats wholesome, and unseasonable times to cause no
disorder. Cider and perry are windy drinks, so are all fruits windy in
themselves, cold most part, yet in some shires of \authorfootnote{1440}England,
Normandy in France, Guipuscoa in Spain, 'tis their common drink, and they are
no whit offended with it. In Spain, Italy, and Africa, they live most on roots,
raw herbs, camel's \authorfootnote{1441}milk, and it agrees well with them:
which to a stranger will cause much grievance. In Wales, \li{lacticiniis
vescuntur}, as Humphrey Llwyd confesseth, a Cambro-Briton himself, in his
elegant epistle to Abraham Ortelius, they live most on white meats: in Holland
on fish, roots, \authorfootnote{1442}butter; and so at this day in Greece, as
\authorfootnote{1443}Bellonius observes, they had much rather feed on fish than
flesh. With us, \li{Maxima pars victus in carne consistit}, we feed on flesh
most part, saith \authorfootnote{1444}\idxname{polydorevergil}[Polydore
Virgil], as all northern countries do; and it would be very offensive to us to
live after their diet, or they to live after ours. We drink beer, they wine;
they use oil, we butter; we in the north are \authorfootnote{1445}great eaters;
they most sparing in those hotter countries; and yet they and we following our
own customs are well pleased. An Ethiopian of old seeing an European eat bread,
wondered, \li{quomodo stercoribus vescentes viverimus}, how we could eat such
kind of meats: so much differed his countrymen from ours in diet, that as mine
\authorfootnote{1446}author infers, \li{si quis illorum victum apud nos
aemulari vellet}; if any man should so feed with us, it would be all one to
nourish, as Cicuta, Aconitum, or Hellebore itself. At this day in China the
common people live in a manner altogether on roots and herbs, and to the
wealthiest, horse, ass, mule, dogs, cat-flesh, is as delightsome as the rest,
so \authorfootnote{1447}Mat. Riccius the Jesuit relates, who lived many years
amongst them. The Tartars eat raw meat, and most commonly
\authorfootnote{1448}horse-flesh, drink milk and blood, as the nomades of old.
\li{Et lac concretum cum sanguine potat equino}. They scoff at our Europeans
for eating bread, which they call tops of weeds, and horse meat, not fit for
men; and yet \Scaliger{} accounts them a sound and witty nation, living a hundred
years; even in the civilest country of them they do thus, as Benedict the
Jesuit observed in his travels, from the great Mogul's Court by land to Pekin,
which Riccius contends to be the same with Cambulu in Cataia. In Scandia their
bread is usually dried fish, and so likewise in the Shetland Isles; and their
other fare, as in Iceland, saith \authorfootnote{1449}Dithmarus Bleskenius,
butter, cheese, and fish; their drink water, their lodging on the ground. In
America in many places their bread is roots, their meat palmettos, pinas,
potatoes, \etc{}, and such fruits. There be of them too that familiarly drink
\authorfootnote{1450}salt seawater all their lives, eat
\authorfootnote{1451}raw meat, grass, and that with delight. With some, fish,
serpents, spiders: and in divers places they \authorfootnote{1452}eat man's
flesh, raw and roasted, even the Emperor \authorfootnote{1453}Montezuma
himself. In some coasts, again, \authorfootnote{1454}one tree yields them
cocoanuts, meat and drink, fire, fuel, apparel; with his leaves, oil, vinegar,
cover for houses, \etc{}, and yet these men going naked, feeding coarse, live
commonly a hundred years, are seldom or never sick; all which diet our
physicians forbid. In Westphalia they feed most part on fat meats and worts,
knuckle deep, and call it \authorfootnote{1455}\li{cerebrum Iovis}: in the Low
Countries with roots, in Italy frogs and snails are used. The Turks, saith
Busbequius, delight most in fried meats. In Muscovy, garlic and onions are
ordinary meat and sauce, which would be pernicious to such as are unaccustomed
to them, delightsome to others; and all is \authorfootnote{1456}because they
have been brought up unto it. Husbandmen, and such as labour, can eat fat
bacon, salt gross meat, hard cheese, \etc{}, (\li{O dura messorum illa}),
coarse bread at all times, go to bed and labour upon a full stomach, which to
some idle persons would be present death, and is against the rules of physic,
so that custom is all in all. Our travellers find this by common experience
when they come in far countries, and use their diet, they are suddenly
offended, \authorfootnote{1457}as our Hollanders and Englishmen when they touch
upon the coasts of Africa, those Indian capes and islands, are commonly
molested with calentures, fluxes, and much distempered by reason of their
fruits. \authorfootnote{1458}\li{Peregrina, etsi suavia solent vescentibus
perturbationes insignes adferre}, strange meats, though pleasant, cause notable
alterations and distempers. On the other side, use or custom mitigates or makes
all good again. Mithridates by often use, which Pliny wonders at, was able to
drink poison; and a maid, as Curtius records, sent to Alexander from King
Porus, was brought up with poison from her infancy. The Turks, saith Bellonius,
lib. 3. c. 15, eat opium familiarly, a dram at once, which we dare not take in
grains. \authorfootnote{1459}Garcias ab Horto writes of one whom he saw at Goa
in the East Indies, that took ten drams of opium in three days; and yet
\li{consulto loquebatur}, spake understandingly, so much can custom do.
\authorfootnote{1460}Theophrastus speaks of a shepherd that could eat hellebore
in substance. And therefore Cardan concludes out of Galen, \li{Consuetudinem
utcunque ferendam, nisi valde malam}. Custom is howsoever to be kept, except it
be extremely bad: he adviseth all men to keep their old customs, and that by
the authority of \authorfootnote{1461}Hippocrates himself, \li{Dandum aliquid
tempori, aetati regioni, consuetudini}, and therefore to
\authorfootnote{1462}continue as they began, be it diet, bath, exercise,
\etc{}, or whatsoever else.

Another exception is delight, or appetite, to such and such meats: though they
be hard of digestion, melancholy; yet as Fuchsius excepts,
\bookcite{\textlatin{cap. 6. lib. 2. Instit. sect. 2}},
\authorfootnote{1463}"The stomach doth readily digest, and willingly entertain
such meats we love most, and are pleasing to us, abhors on the other side such
as we distaste." Which Hippocrates confirms, \bookcite{\textlatin{Aphoris. 2.
38.}} Some cannot endure cheese, out of a secret antipathy; or to see a roasted
duck, which to others is a \authorfootnote{1464}delightsome meat.

The last exception is necessity, poverty, want, hunger, which drives men many
times to do that which otherwise they are loath, cannot endure, and thankfully
to accept of it: as beverage in ships, and in sieges of great cities, to feed
on dogs, cats, rats, and men themselves. Three outlaws in
\authorfootnote{1465}Hector Boethius, being driven to their shifts, did eat raw
flesh, and flesh of such fowl as they could catch, in one of the Hebrides for
some few months. These things do mitigate or disannul that which hath been said
of melancholy meats, and make it more tolerable; but to such as are wealthy,
live plenteously, at ease, may take their choice, and refrain if they will,
these viands are to be forborne, if they be inclined to, or suspect melancholy,
as they tender their healths: Otherwise if they be intemperate, or disordered
in their diet, at their peril be it. \li{Qui monet amat, Ave et cave}.

\begin{verse}%
He who advises is your friend\\*
Farewell, and to your health attend.\\!
\end{verse}%

%SECT. II. MEMB. II. SUBSECT. IV.-_Retention and Evacuation a cause, and how_.
\section{Retention and Evacuation a cause, and how.}

\lettrine{O}{f} retention and evacuation, there be divers kinds, which are
either concomitant, assisting, or sole causes many times of melancholy.
\authorfootnote{1466}Galen reduceth defect and abundance to this head; others
\authorfootnote{1467}"All that is separated, or remains."

\subsection{Costiveness.}

In the first rank of these, I may well reckon up costiveness, and keeping in of
our ordinary excrements, which as it often causeth other diseases, so this of
melancholy in particular. \authorfootnote{1468}Celsus, lib. 1. cap. 3, saith,
"It produceth inflammation of the head, dullness, cloudiness, headache," \etc{}
Prosper Calenus, \bookcite{\textlatin{lib. de atra bile}}, will have it
distemper not the organ only, \authorfootnote{1469}"but the mind itself by
troubling of it:" and sometimes it is a sole cause of madness, as you may read
in the first book of \authorfootnote{1470}Skenkius's Medicinal Observations. A
young merchant going to Nordeling fair in Germany, for ten days' space never
went to stool; at his return he was \authorfootnote{1471}grievously melancholy,
thinking that he was robbed, and would not be persuaded but that all his money
was gone; his friends thought he had some philtrum given him, but Cnelius, a
physician, being sent for, found his \authorfootnote{1472}costiveness alone to
be the cause, and thereupon gave him a clyster, by which he was speedily
recovered. Trincavellius, \bookcite{\textlatin{consult. 35. lib. 1}}, saith as
much of a melancholy lawyer, to whom he administered physic, and Rodericus a
Fonseca, \bookcite{\textlatin{consult. 85. tom. 2}}, \authorfootnote{1473}of a
patient of his, that for eight days was bound, and therefore melancholy
affected. Other retentions and evacuations there are, not simply necessary, but
at some times; as Fernelius accounts them, \bookcite{\textlatin{Path. lib. 1.
cap. 15}}, as suppression of haemorrhoids, monthly issues in women, bleeding at
nose, immoderate or no use at all of Venus: or any other ordinary issues.

\authorfootnote{1474}Detention of haemorrhoids, or monthly issues, Villanovanus
\bookcite{\textlatin{Breviar. lib. 1. cap. 18.}} Arculanus,
\bookcite{\textlatin{cap. 16. in 9. Rhasis}}, Vittorius Faventinus,
\bookcite{\textlatin{pract. mag. tract. 2. cap. 15.}} Bruel, \etc{} put for
ordinary causes. Fuchsius, \bookcite{\textlatin{l. 2. sect. 5. c. 30}}, goes
farther, and saith, \authorfootnote{1475}"That many men unseasonably cured of
the haemorrhoids have been corrupted with melancholy, seeking to avoid Scylla,
they fall into Charybdis." Galen, \bookcite{\textlatin{l. de hum. commen. 3. ad
text. 26}}, illustrates this by an example of Lucius Martius, whom he cured of
madness, contracted by this means: And \authorfootnote{1476}Skenkius hath two
other instances of two melancholy and mad women, so caused from the suppression
of their months. The same may be said of bleeding at the nose, if it be
suddenly stopped, and have been formerly used, as
\authorfootnote{1477}Villanovanus urgeth: And \authorfootnote{1478}Fuchsius,
\bookcite{\textlatin{lib. 2. sect. 5. cap. 33}}, stiffly maintains, "That
without great danger, such an issue may not be stayed."

Venus omitted produceth like effects. Mathiolus, \bookcite{\textlatin{epist. 5.
l. penult.}}, \authorfootnote{1479}"avoucheth of his knowledge, that some
through bashfulness abstained from venery, and thereupon became very heavy and
dull; and some others that were very timorous, melancholy, and beyond all
measure sad." Oribasius, \bookcite{\textlatin{med. collect. l. 6. c. 37}},
speaks of some, \authorfootnote{1480}"That if they do not use carnal
copulation, are continually troubled with heaviness and headache; and some in
the same case by intermission of it." Not use of it hurts many, Arculanus,
\bookcite{\textlatin{c. 6. in 9. Rhasis, et Magninus, part. 3. cap. 5}}, think,
because it \authorfootnote{1481}"sends up poisoned vapours to the brain and
heart." And so doth Galen himself hold, "That if this natural seed be over-long
kept (in some parties) it turns to poison." Hieronymus Mercurialis, in his
chapter of melancholy, cites it for an especial cause of this malady,
\authorfootnote{1482}priapismus, satyriasis, \etc{} Haliabbas,
\bookcite{\textlatin{5. Theor. c. 36}}, reckons up this and many other
diseases. Villanovanus \bookcite{\textlatin{Breviar. l. 1. c. 18}}, saith, "He
knew \authorfootnote{1483}many monks and widows grievously troubled with
melancholy, and that from this sole cause." \authorfootnote{1484}Ludovicus
Mercatus, \bookcite{\textlatin{l. 2. de mulierum affect. cap. 4}}, and
Rodericus a Castro, \bookcite{\textlatin{de morbis mulier. l. 2. c. 3}}, treat
largely of this subject, and will have it produce a peculiar kind of melancholy
in stale maids, nuns, and widows, \li{Ob suppressionem mensium et venerem
omissam, timidae, moestae anxiae, verecundae, suspicioscae, languentes,
consilii inopes, cum summa vitae et rerum meliorum desperatione}, \etc{}, they
are melancholy in the highest degree, and all for want of husbands. Aelianus
Montaltus, \bookcite{\textlatin{cap. 37. de melanchol.}}, confirms as much out
of Galen; so doth Wierus, Christophorus a Vega \bookcite{\textlatin{de art.
med. lib. 3. c. 14}}, relates many such examples of men and women, that he had
seen so melancholy. Felix Plater in the first book of his Observations,
\authorfootnote{1485}"tells a story of an ancient gentleman in Alsatia, that
married a young wife, and was not able to pay his debts in that kind for a long
time together, by reason of his several infirmities: but she, because of this
inhibition of Venus, fell into a horrible fury, and desired every one that came
to see her, by words, looks, and gestures, to have to do with her," \etc{}
\authorfootnote{1486}Bernardus Paternus, a physician, saith, "He knew a good
honest godly priest, that because he would neither willingly marry, nor make
use of the stews, fell into grievous melancholy fits." Hildesheim,
\bookcite{\textlatin{spicel. 2}}, hath such another example of an Italian
melancholy priest, in a consultation had \emph{Anno} 1580. Jason Pratensis
gives instance in a married man, that from his wife's death abstaining,
\authorfootnote{1487}"after marriage, became exceedingly melancholy," Rodericus
a Fonseca in a young man so misaffected, \bookcite{\textlatin{Tom. 2. consult.
85.}} To these you may add, if you please, that conceited tale of a Jew, so
visited in like sort, and so cured, out of Poggius Florentinus.

Intemperate Venus is all but as bad in the other extreme. Galen,
\bookcite{\textlatin{l. 6. de mortis popular. sect. 5. text. 26}}, reckons up
melancholy amongst those diseases which are \authorfootnote{1488}"exasperated
by venery:" so doth \Avicenna{}, \bookcite{\textlatin{2, 3, c. 11.}} Oribasius,
\bookcite{\textlatin{loc. citat.}} Ficinus, \bookcite{\textlatin{lib. 2. de
sanitate tuenda}}. Marsilius Cognatus, Montaltus, \bookcite{\textlatin{cap.
27.}} Guianerius, \bookcite{\textlatin{Tract. 3. cap. 2.}} Magninus,
\bookcite{\textlatin{cap. 5. part. 3.}} \authorfootnote{1489}gives the reason,
because \authorfootnote{1490}"it infrigidates and dries up the body, consumes
the spirits; and would therefore have all such as are cold and dry to take heed
of and to avoid it as a mortal enemy." Jacchinus \bookcite{\textlatin{in 9
Rhasis, cap. 15}}, ascribes the same cause, and instanceth in a patient of his,
that married a young wife in a hot summer, \authorfootnote{1491}"and so dried
himself with chamber-work, that he became in short space from melancholy, mad:"
he cured him by moistening remedies. The like example I find in Laelius a Fonte
Eugubinus, \bookcite{\textlatin{consult. 129}}, of a gentleman of Venice, that
upon the same occasion was first melancholy, afterwards mad. Read in him the
story at large.

Any other evacuation stopped will cause it, as well as these above named, be it
bile, \authorfootnote{1492}ulcer, issue, \etc{} Hercules de Saxonia,
\bookcite{\textlatin{lib. 1. c. 16}}, and Gordonius, verify this out of their
experience. They saw one wounded in the head who as long as the sore was open,
\li{Lucida habuit mentis intervalla}, was well; but when it was stopped,
\li{Rediit melancholia}, his melancholy fit seized on him again.

Artificial evacuations are much like in effect, as hot houses, baths,
bloodletting, purging, unseasonably and immoderately used.
\authorfootnote{1493}Baths dry too much, if used in excess, be they natural or
artificial, and offend extreme hot, or cold; \authorfootnote{1494}one dries,
the other refrigerates overmuch. Montanus, \bookcite{\textlatin{consil. 137}},
saith, they overheat the liver. Joh. Struthius, \bookcite{\textlatin{Stigmat.
artis. l. 4. c. 9}}, contends, \authorfootnote{1495}"that if one stay longer
than ordinary at the bath, go in too oft, or at unseasonable times, he
putrefies the humours in his body." To this purpose writes Magninus,
\bookcite{\textlatin{l. 3. c. 5.}} Guianerius, \bookcite{\textlatin{Tract. 15.
c. 21}}, utterly disallows all hot baths in melancholy adust.
\authorfootnote{1496}"I saw" (saith he) "a man that laboured of the gout, who
to be freed of this malady came to the bath, and was instantly cured of his
disease, but got another worse, and that was madness." But this judgment varies
as the humour doth, in hot or cold: baths may be good for one melancholy man,
bad for another; that which will cure it in this party, may cause it in a
second.

\subsection{Phlebotomy.}

Phlebotomy, many times neglected, may do much harm to
the body, when there is a manifest redundance of bad humours, and melancholy
blood; and when these humours heat and boil, if this be not used in time, the
parties affected, so inflamed, are in great danger to be mad; but if it be
unadvisedly, importunely, immoderately used, it doth as much harm by
refrigerating the body, dulling the spirits, and consuming them: as Joh.
\authorfootnote{1497}Curio in his 10th chapter well reprehends, such kind of
letting blood doth more hurt than good: \authorfootnote{1498}"The humours rage
much more than they did before, and is so far from avoiding melancholy, that it
increaseth it, and weakeneth the sight." \authorfootnote{1499}Prosper Calenus
observes as much of all phlebotomy, except they keep a very good diet after it;
yea, and as \authorfootnote{1500}Leonartis Jacchinus speaks out of his own
experience, \authorfootnote{1501}"The blood is much blacker to many men after
their letting of blood than it was at first." For this cause belike Salust.
Salvinianus, \bookcite{\textlatin{l. 2. c. 1}}, will admit or hear of no
bloodletting at all in this disease, except it be manifest it proceed from
blood: he was (it appears) by his own words in that place, master of an
hospital of mad men, \authorfootnote{1502}"and found by long experience, that
this kind of evacuation, either in head, arm, or any other part, did more harm
than good." To this opinion of his, \authorfootnote{1503}Felix Plater is quite
opposite, "though some wink at, disallow and quite contradict all phlebotomy in
melancholy, yet by long experience I have found innumerable so saved, after
they had been twenty, nay, sixty times let blood, and to live happily after it.
It was an ordinary thing of old, in Galen's time, to take at once from such men
six pounds of blood, which now we dare scarce take in ounces: \li{sed viderint
medici};" great books are written of this subject.

Purging upward and downward, in abundance of bad humours omitted, may be for
the worst; so likewise as in the precedent, if overmuch, too frequent or
violent, it \authorfootnote{1504}weakeneth their strength, saith Fuchsius,
\bookcite{\textlatin{l. 2. sect., 2 c. 17}}, or if they be strong or able to
endure physic, yet it brings them to an ill habit, they make their bodies no
better than apothecaries' shops, this and such like infirmities must needs
follow.

%SECT. II. MEMB. II. SUBSECT. V.-_Bad Air, a cause of Melancholy_.
\section{Bad Air, a cause of Melancholy.}

\lettrine{A}{ir} is a cause of great moment, in producing this, or any other
disease, being that it is still taken into our bodies by respiration, and our
more inner parts. \authorfootnote{1505}"If it be impure and foggy, it dejects
the spirits, and causeth diseases by infection of the heart," as Paulus hath
it, \bookcite{\textlatin{lib. 1. c. 49.}} \Avicenna{}, \bookcite{\textlatin{lib.
1. Gal. de san. tuenda}}. Mercurialis, Montaltus, \etc{}
\authorfootnote{1506}Fernelius saith, "A thick air thickeneth the blood and
humours." \authorfootnote{1507}Lemnius reckons up two main things most
profitable, and most pernicious to our bodies; air and diet: and this peculiar
disease, nothing sooner causeth \authorfootnote{1508}(Jobertus holds) "than the
air wherein we breathe and live." \authorfootnote{1509}Such as is the air, such
be our spirits; and as our spirits, such are our humours. It offends commonly
if it be too \authorfootnote{1510}hot and dry, thick, fuliginous, cloudy,
blustering, or a tempestuous air. Bodine in his fifth Book,
\bookcite{\textlatin{De repub. cap. 1, 5}}, of his Method of History, proves
that hot countries are most troubled with melancholy, and that there are
therefore in Spain, Africa, and Asia Minor, great numbers of mad men, insomuch
that they are compelled in all cities of note, to build peculiar hospitals for
them. Leo \authorfootnote{1511}Afer, \bookcite{\textlatin{lib. 3. de Fessa
urbe}}, Ortelius and Zuinger, confirm as much: they are ordinarily so choleric
in their speeches, that scarce two words pass without railing or chiding in
common talk, and often quarrelling in their streets.
\authorfootnote{1512}Gordonius will have every man take notice of it: "Note
this" (saith he) "that in hot countries it is far more familiar than in cold."
Although this we have now said be not continually so, for as
\authorfootnote{1513}Acosta truly saith, under the Equator itself, is a most
temperate habitation, wholesome air, a paradise of pleasure: the leaves ever
green, cooling showers. But it holds in such as are intemperately hot, as
\authorfootnote{1514}Johannes a Meggen found in Cyprus, others in Malta,
Aupulia, and the \authorfootnote{1515}Holy Land, where at some seasons of the
year is nothing but dust, their rivers dried up, the air scorching hot, and
earth inflamed; insomuch that many pilgrims going barefoot for devotion sake,
from Joppa to Jerusalem upon the hot sands, often run mad, or else quite
overwhelmed with sand, \li{profundis arenis}, as in many parts of Africa,
Arabia Deserta, Bactriana, now Charassan, when the west wind blows \li{Involuti
arenis transeuntes necantur}\authorlatintrans{1516.5}.\authorfootnote{1516}

\authorfootnote{1517}Hercules de Saxonia, a professor in Venice, gives this
cause why so many Venetian women are melancholy, \li{Quod diu sub sole degant},
they tarry too long in the sun. Montanus, \bookcite{\textlatin{consil. 21}},
amongst other causes assigns this; Why that Jew his patient was mad, \li{Quod
tam multum exposuit se calori et frigori}: he exposed himself so much to heat
and cold, and for that reason in Venice, there is little stirring in those
brick paved streets in summer about noon, they are most part then asleep: as
they are likewise in the great Mogol's countries, and all over the East Indies.
At Aden in Arabia, as \authorfootnote{1518}Lodovicus Vertomannus relates in his
travels, they keep their markets in the night, to avoid extremity of heat; and
in Ormus, like cattle in a pasture, people of all sorts lie up to the chin in
water all day long. At Braga in Portugal; Burgos in Castile; Messina in Sicily,
all over Spain and Italy, their streets are most part narrow, to avoid the
sunbeams. The Turks wear great turbans \li{ad fugandos solis radios}, to
refract the sunbeams; and much inconvenience that hot air of Bantam in Java
yields to our men, that sojourn there for traffic; where it is so hot,
\authorfootnote{1519}"that they that are sick of the pox, lie commonly
bleaching in the sun, to dry up their sores." Such a complaint I read of those
isles of Cape Verde, fourteen degrees from the Equator, they do \li{male
audire}: \authorfootnote{1520}One calls them the unhealthiest clime of the
world, for fluxes, fevers, frenzies, calentures, which commonly seize on
seafaring men that touch at them, and all by reason of a hot distemperature of
the air. The hardiest men are offended with this heat, and stiffest clowns
cannot resist it, as Constantine affirms, \bookcite{\textlatin{Agricult. l. 2.
c. 45.}} They that are naturally born in such air, may not
\authorfootnote{1521}endure it, as Niger records of some part of Mesopotamia,
now called Diarbecha: \li{Quibusdam in locis saevienti aestui adeo subjecta
est, ut pleraque animalia fervore solis et coeli extinguantur}, 'tis so hot
there in some places, that men of the country and cattle are killed with it;
and \authorfootnote{1522}Adricomius of Arabia Felix, by reason of myrrh,
frankincense, and hot spices there growing, the air is so obnoxious to their
brains, that the very inhabitants at some times cannot abide it, much less
weaklings and strangers. \authorfootnote{1523}Amatus Lusitanus,
\bookcite{\textlatin{cent. 1. curat. 45}}, reports of a young maid, that was
one Vincent a currier's daughter, some thirteen years of age, that would wash
her hair in the heat of the day (in July) and so let it dry in the sun,
\authorfootnote{1524}"to make it yellow, but by that means tarrying too long in
the heat, she inflamed her head, and made herself mad."

Cold air in the other extreme is almost as bad as hot, and so doth Montaltus
esteem of it, \bookcite{\textlatin{c. 11}}, if it be dry withal. In those
northern countries, the people are therefore generally dull, heavy, and many
witches, which (as I have before quoted) Saxo Grammaticus, Olaus, Baptista
Porta ascribe to melancholy. But these cold climes are more subject to natural
melancholy (not this artificial) which is cold and dry: for which cause
\authorfootnote{1525}Mercurius Britannicus belike puts melancholy men to
inhabit just under the Pole. The worst of the three is a
\authorfootnote{1526}thick, cloudy, misty, foggy air, or such as come from
fens, moorish grounds, lakes, muck-hills, draughts, sinks, where any carcasses,
or carrion lies, or from whence any stinking fulsome smell comes: Galen,
\Avicenna{}, Mercurialis, new and old physicians, hold that such air is
unwholesome, and engenders melancholy, plagues, and what not?
\authorfootnote{1527}Alexandretta, an haven-town in the Mediterranean Sea,
Saint John de Ulloa, an haven in Nova-Hispania, are much condemned for a bad
air, so are Durazzo in Albania, Lithuania, Ditmarsh, Pomptinae Paludes in
Italy, the territories about Pisa, Ferrara, \etc{} Romney Marsh with us; the
Hundreds in Essex, the fens in Lincolnshire. Cardan, \bookcite{\textlatin{de
rerum varietate, l. 17, c. 96}}, finds fault with the sight of those rich, and
most populous cities in the Low Countries, as Bruges, Ghent, Amsterdam, Leiden,
Utrecht, \etc{} the air is bad; and so at Stockholm in Sweden; Regium in Italy,
Salisbury with us, Hull and Lynn: they may be commodious for navigation, this
new kind of fortification, and many other good necessary uses; but are they so
wholesome? Old Rome hath descended from the hills to the valley, 'tis the site
of most of our new cities, and held best to build in plains, to take the
opportunity of rivers. Leander Albertus pleads hard for the air and site of
Venice, though the black moorish lands appear at every low water: the sea,
fire, and smoke (as he thinks) qualify the air; and \authorfootnote{1528}some
suppose, that a thick foggy air helps the memory, as in them of Pisa in Italy;
and our Camden, out of Plato, commends the site of Cambridge, because it is so
near the fens. But let the site of such places be as it may, how can they be
excused that have a delicious seat, a pleasant air, and all that nature can
afford, and yet through their own nastiness, and sluttishness, immund and
sordid manner of life, suffer their air to putrefy, and themselves to be
chocked up? Many cities in Turkey do \li{male audire} in this kind:
Constantinople itself, where commonly carrion lies in the street. Some find the
same fault in Spain, even in Madrid, the king's seat, a most excellent air, a
pleasant site; but the inhabitants are slovens, and the streets uncleanly kept.

A troublesome tempestuous air is as bad as impure, rough and foul weather,
impetuous winds, cloudy dark days, as it is commonly with us, \li{Coelum visu
foedum}, \authorfootnote{1529}\idxname{polydorevergil}[Polydore] calls it a
filthy sky, \li{et in quo facile generantur nubes}; as \Tully{}'s brother Quintus
wrote to him in Rome, being then quaestor in Britain. "In a thick and cloudy
air" (saith Lemnius) "men are tetric, sad, and peevish: And if the western
winds blow, and that there be a calm, or a fair sunshine day, there is a kind
of alacrity in men's minds; it cheers up men and beasts: but if it be a
turbulent, rough, cloudy, stormy weather, men are sad, lumpish, and much
dejected, angry, waspish, dull, and melancholy." This was
\authorfootnote{1530}\Virgil{}'s experiment of old,

\translatedverse{%
\begin{latin}
\begin{verse}%
Verum ubi tempestas, et coeli mobilis humor\\*
Mutavere vices, et Jupiter humidus Austro,\\*
Vertuntur species animorum, et pectore motus\\*
Concipiunt alios------\\!
\end{verse}%
\end{latin}}{%
\begin{verse}%
But when the face of Heaven changed is\\*
To tempests, rain, from season fair:\\*
Our minds are altered, and in our breasts\\*
Forthwith some new conceits appear.\\!
\end{verse}}{}

And who is not weather-wise against such and such conjunctions of planets,
moved in foul weather, dull and heavy in such tempestuous seasons?
\authorfootnote{1531}\li{Gelidum contristat Aquarius annum}: the time requires,
and the autumn breeds it; winter is like unto it, ugly, foul, squalid, the air
works on all men, more or less, but especially on such as are melancholy, or
inclined to it, as Lemnius holds, \authorfootnote{1532}"They are most moved
with it, and those which are already mad, rave downright, either in, or against
a tempest. Besides, the devil many times takes his opportunity of such storms,
and when the humours by the air be stirred, he goes in with them, exagitates
our spirits, and vexeth our souls; as the sea waves, so are the spirits and
humours in our bodies tossed with tempestuous winds and storms." To such as are
melancholy therefore, Montanus, \bookcite{\textlatin{consil. 24}}, will have
tempestuous and rough air to be avoided, and \bookcite{\textlatin{consil. 27}},
all night air, and would not have them to walk abroad, but in a pleasant day.
Lemnius, \bookcite{\textlatin{l. 3. c. 3}}, discommends the south and eastern
winds, commends the north. Montanus, \bookcite{\textlatin{consil. 31.}}
\authorfootnote{1533}"Will not any windows to be opened in the night."
\bookcite{\textlatin{Consil. 229. et consil. 230}}, he discommends especially
the south wind, and nocturnal air: So doth \authorfootnote{1534}\Plutarch{}. The
night and darkness makes men sad, the like do all subterranean vaults, dark
houses in caves and rocks, desert places cause melancholy in an instant,
especially such as have not been used to it, or otherwise accustomed. Read more
of air in Hippocrates, \bookcite{\textlatin{Aetius, l. 3. a c. 171. ad 175.}}
Oribasius, \bookcite{\textlatin{a c. 1. ad 21.}} Avicen.
\bookcite{\textlatin{l. 1. can. Fen. 2. doc. 2. Fen. 1. c. 123}} to the 12,
\etc{}

%SECT. II. MEMB. II. SUBSECT. VI.-_Immoderate Exercise a cause, and how.
%Solitariness, Idleness_.
\section[Immoderate Exercise, Idleness]{Immoderate Exercise a cause, and how.
Solitariness, Idleness.}

\lettrine{N}{othing} so good but it may be abused: nothing better than exercise
(if opportunely used) for the preservation of the body: nothing so bad if it be
unseasonable. violent, or overmuch. Fernelius out of Galen,
\bookcite{\textlatin{Path. lib. 1. c. 16}}, saith, \authorfootnote{1535}"That
much exercise and weariness consumes the spirits and substance, refrigerates
the body; and such humours which Nature would have otherwise concocted and
expelled, it stirs up and makes them rage: which being so enraged, diversely
affect and trouble the body and mind." So doth it, if it be unseasonably used,
upon a full stomach, or when the body is full of crudities, which Fuchsius so
much inveighs against, \bookcite{\textlatin{lib. 2. instit. sec. 2. c. 4}},
giving that for a cause, why schoolboys in Germany are so often scabbed,
because they use exercise presently after meats. \authorfootnote{1536}Bayerus
puts in a caveat against such exercise, because "it
\authorfootnote{1537}corrupts the meat in the stomach, and carries the same
juice raw, and as yet undigested, into the veins" (saith Lemnius), "which there
putrefies and confounds the animal spirits." Crato,
\bookcite{\textlatin{consil. 21. l. 2}}, \authorfootnote{1538}protests against
all such exercise after meat, as being the greatest enemy to concoction that
may be, and cause of corruption of humours, which produce this, and many other
diseases. Not without good reason then doth Salust. Salvianus,
\bookcite{\textlatin{l. 2. c. 1}}, and Leonartus Jacchinus,
\bookcite{\textlatin{in 9. Rhasis}}, Mercurialis, Arcubanus, and many other,
set down \authorfootnote{1539}immoderate exercise as a most forcible cause of
melancholy.

Opposite to exercise is idleness (the badge of gentry) or want of exercise, the
bane of body and mind, the nurse of naughtiness, stepmother of discipline, the
chief author of all mischief, one of the seven deadly sins, and a sole cause of
this and many other maladies, the devil's cushion, as
\authorfootnote{1540}Gualter calls it, his pillow and chief reposal. "For the
mind can never rest, but still meditates on one thing or other, except it be
occupied about some honest business, of his own accord it rusheth into
melancholy." \authorfootnote{1541}"As too much and violent exercise offends on
the one side, so doth an idle life on the other" (saith Crato), "it fills the
body full of phlegm, gross humours, and all manner of obstructions, rheums,
catarrhs," \etc{} Rhasis, \bookcite{\textlatin{cont. lib. 1. tract. 9}},
accounts of it as the greatest cause of melancholy. \authorfootnote{1542}"I
have often seen" (saith he) "that idleness begets this humour more than
anything else." Montaltus, \bookcite{\textlatin{c. 1}}, seconds him out of his
experience, \authorfootnote{1543}"They that are idle are far more subject to
melancholy than such as are conversant or employed about any office or
business." \authorfootnote{1544}\Plutarch{} reckons up idleness for a sole cause
of the sickness of the soul: "There are they" (saith he) "troubled in mind,
that have no other cause but this." \idxname{Homer}[Homer][Iliad], \bookcite{\textlatin{Iliad. 1}},
brings in Achilles eating of his own heart in his idleness, because he might
not fight. Mercurialis, \bookcite{\textlatin{consil. 86}}, for a melancholy
young man urgeth, \authorfootnote{1545}it as a chief cause; why was he
melancholy? because idle. Nothing begets it sooner, increaseth and continueth
it oftener than idleness. \authorfootnote{1546}A disease familiar to all idle
persons, an inseparable companion to such as live at ease, \li{Pingui otio
desidiose agentes}, a life out of action, and have no calling or ordinary
employment to busy themselves about, that have small occasions; and though they
have, such is their laziness, dullness, they will not compose themselves to do
aught; they cannot abide work, though it be necessary; easy as to dress
themselves, write a letter, or the like; yet as he that is benumbed with cold
sits still shaking, that might relieve himself with a little exercise or
stirring, do they complain, but will not use the facile and ready means to do
themselves good; and so are still tormented with melancholy. Especially if they
have been formerly brought up to business, or to keep much company, and upon a
sudden come to lead a sedentary life; it crucifies their souls, and seizeth on
them in an instant; for whilst they are any ways employed, in action,
discourse, about any business, sport or recreation, or in company to their
liking, they are very well; but if alone or idle, tormented instantly again;
one day's solitariness, one hour's sometimes, doth them more harm, than a
week's physic, labour, and company can do good. Melancholy seizeth on them
forthwith being alone, and is such a torture, that as wise \Seneca{} well saith,
\li{Malo mihi male quam molliter esse}, I had rather be sick than idle. This
idleness is either of body or mind. That of body is nothing but a kind of
benumbing laziness, intermitting exercise, which, if we may believe
\authorfootnote{1547}Fernelius, "causeth crudities, obstructions, excremental
humours, quencheth the natural heat, dulls the spirits, and makes them unapt to
do any thing whatsoever."

\translatedverse{%
\begin{latin}
\begin{verse}%
Neglectis urenda filix innascitur agris.\\!
\end{verse}%
\end{latin}}{%
\begin{verse}%
------for, a neglected field\\*
Shall for the fire its thorns and thistles yield.\\!
\end{verse}}{%
\attrib{\getauthornote{1548}}}

As fern grows in untilled grounds, and all manner of weeds, so do gross humours
in an idle body, \li{Ignavum corrumpunt otia corpus}. A horse in a stable that
never travels, a hawk in a mew that seldom flies, are both subject to diseases;
which left unto themselves, are most free from any such encumbrances. An idle
dog will be mangy, and how shall an idle person think to escape? Idleness of
the mind is much worse than this of the body; wit without employment is a
disease \authorfootnote{1549}\li{Aerugo animi, rubigo ingenii}: the rust of the
soul, \authorfootnote{1550}a plague, a hell itself, \li{Maximum animi
nocumentum}, Galen, calls it. \authorfootnote{1551}"As in a standing pool,
worms and filthy creepers increase, (\li{et vitium capiunt ni moveantur aquae},
the water itself putrefies, and air likewise, if it be not continually stirred
by the wind) so do evil and corrupt thoughts in an idle person," the soul is
contaminated. In a commonwealth, where is no public enemy, there is likely
civil wars, and they rage upon themselves: this body of ours, when it is idle,
and knows not how to bestow itself, macerates and vexeth itself with cares,
griefs, false fears, discontents, and suspicions; it tortures and preys upon
his own bowels, and is never at rest. Thus much I dare boldly say; he or she
that is idle, be they of what condition they will, never so rich, so well
allied, fortunate, happy, let them have all things in abundance and felicity
that heart can wish and desire, all contentment, so long as he or she or they
are idle, they shall never be pleased, never well in body and mind, but weary
still, sickly still, vexed still, loathing still, weeping, sighing, grieving,
suspecting, offended with the world, with every object, wishing themselves gone
or dead, or else earned away with some foolish phantasy or other. And this is
the true cause that so many great men, ladies, and gentlewomen, labour of this
disease in country and city; for idleness is an appendix to nobility; they
count it a disgrace to work, and spend all their days in sports, recreations,
and pastimes, and will therefore take no pains; be of no vocation: they feed
liberally, fare well, want exercise, action, employment, (for to work, I say,
they may not abide,) and Company to their desires, and thence their bodies
become full of gross humours, wind, crudities; their minds disquieted, dull,
heavy, \etc{} care, jealousy, fear of some diseases, sullen fits, weeping fits
seize too \authorfootnote{1552}familiarly on them. For what will not fear and
phantasy work in an idle body? what distempers will they not cause? when the
children of \authorfootnote{1553}Israel murmured against Pharaoh in Egypt, he
commanded his officers to double their task, and let them get straw themselves,
and yet make their full number of bricks; for the sole cause why they mutiny,
and are evil at ease, is, "they are idle." When you shall hear and see so many
discontented persons in all places where you come, so many several grievances,
unnecessary complaints, fears, suspicions, \authorfootnote{1554}the best means
to redress it is to set them awork, so to busy their minds; for the truth is,
they are idle. Well they may build castles in the air for a time, and sooth up
themselves with fantastical and pleasant humours, but in the end they will
prove as bitter as gall, they shall be still I say discontent, suspicious,
\authorfootnote{1555}fearful, jealous, sad, fretting and vexing of themselves;
so long as they be idle, it is impossible to please them, \li{Otio qui nescit
uti, plus habet negotii quam qui negotium in negotio}, as that
\authorfootnote{1556}Agellius could observe: He that knows not how to spend his
time, hath more business, care, grief, anguish of mind, than he that is most
busy in the midst of all his business. \li{Otiosus animus nescit quid volet}:
An idle person (as he follows it) knows not when he is well, what he would
have, or whither he would go, \li{Quum illuc ventum est, illinc lubet}, he is
tired out with everything, displeased with all, weary of his life: \li{Nec bene
domi, nec militiae}, neither at home nor abroad, \li{errat, et praeter vitam
vivitur}, he wanders and lives besides himself. In a word, What the mischievous
effects of laziness and idleness are, I do not find any where more accurately
expressed, than in these verses of Philolaches in the
\authorfootnote{1557}Comical Poet, which for their elegancy I will in part
insert.

\begin{latin}
\begin{verse}%
Novarum aedium esse arbitror similem ego hominem,\\*
Quando hic natus est: Ei rei argumenta dicam.\\*
Aedes quando sunt ad amussim expolitae,\\*
Quisque laudat fabrum, atque exemplum expetit, \etc{}\\*
At ubi illo migrat nequam homo indiligensque, \etc{}\\*
Tempestas venit, confringit tegulas, imbricesque,\\*
Putrifacit aer operam fabri, \etc{}\\*
Dicam ut homines similes esse aedium arbitremini,\\*
Fabri parentes fundamentum substruunt liberorum,\\*
Expoliunt, docent literas, nec parcunt sumptui,\\*
Ego autem sub fabrorum potestate frugi fui,\\*
Postquam autem migravi in ingenium meum,\\*
Perdidi operam fabrorum illico oppido,\\*
Venit ignavia, ea mihi tempestas fuit,\\*
Adventuque suo grandinem et imbrem attulit,\\*
Illa mihi virtutem deturbavit, \etc{}\\!
\end{verse}%
\end{latin}

A young man is like a fair new house, the carpenter leaves it well built, in
good repair, of solid stuff; but a bad tenant lets it rain in, and for want of
reparation, fall to decay, \etc{} Our parents, tutors, friends, spare no cost
to bring us up in our youth, in all manner of virtuous education; but when we
are left to ourselves, idleness as a tempest drives all virtuous motions out of
our minds, et \li{nihili sumus}, on a sudden, by sloth and such bad ways, we
come to nought.

Cousin german to idleness, and a concomitant cause, which goes hand in hand
with it, is \authorfootnote{1558}\li{nimia solitudo}, too much solitariness, by
the testimony of all physicians, cause and symptom both; but as it is here put
for a cause, it is either coact, enforced, or else voluntary. Enforced
solitariness is commonly seen in students, monks, friars, anchorites, that by
their order and course of life must abandon all company, society of other men,
and betake themselves to a private cell: \li{Otio superstitioso seclusi}, as
Bale and Hospinian well term it, such as are the Carthusians of our time, that
eat no flesh (by their order), keep perpetual silence, never go abroad. Such as
live in prison, or some desert place, and cannot have company, as many of our
country gentlemen do in solitary houses, they must either be alone without
companions, or live beyond their means, and entertain all comers as so many
hosts, or else converse with their servants and hinds, such as are unequal,
inferior to them, and of a contrary disposition: or else as some do, to avoid
solitariness, spend their time with lewd fellows in taverns, and in alehouses,
and thence addict themselves to some unlawful disports, or dissolute courses.
Divers again are cast upon this rock of solitariness for want of means, or out
of a strong apprehension of some infirmity, disgrace, or through bashfulness,
rudeness, simplicity, they cannot apply themselves to others' company.
\li{Nullum solum infelici gratius solitudine, ubi nullus sit qui miseriam
exprobret}; this enforced solitariness takes place, and produceth his effect
soonest in such as have spent their time jovially, peradventure in all honest
recreations, in good company, in some great family or populous city, and are
upon a sudden confined to a desert country cottage far off, restrained of their
liberty, and barred from their ordinary associates; solitariness is very
irksome to such, most tedious, and a sudden cause of great inconvenience.

Voluntary solitariness is that which is familiar with melancholy, and gently
brings on like a Siren, a shoeing-horn, or some sphinx to this irrevocable
gulf, \authorfootnote{1559}a primary cause, Piso calls it; most pleasant it is
at first, to such as are melancholy given, to lie in bed whole days, and keep
their chambers, to walk alone in some solitary grove, betwixt wood and water,
by a brook side, to meditate upon some delightsome and pleasant subject, which
shall affect them most; \li{amabilis insania, et mentis gratissimus error}: a
most incomparable delight it is so to melancholise, and build castles in the
air, to go smiling to themselves, acting an infinite variety of parts, which
they suppose and strongly imagine they represent, or that they see acted or
done: \li{Blandae quidem ab initio}, saith Lemnius, to conceive and meditate of
such pleasant things, sometimes, \authorfootnote{1560}"present, past, or to
come," as Rhasis speaks. So delightsome these toys are at first, they could
spend whole days and nights without sleep, even whole years alone in such
contemplations, and fantastical meditations, which are like unto dreams, and
they will hardly be drawn from them, or willingly interrupt, so pleasant their
vain conceits are, that they hinder their ordinary tasks and necessary
business, they cannot address themselves to them, or almost to any study or
employment, these fantastical and bewitching thoughts so covertly, so
feelingly, so urgently, so continually set upon, creep in, insinuate, possess,
overcome, distract, and detain them, they cannot, I say, go about their more
necessary business, stave off or extricate themselves, but are ever musing,
melancholising, and carried along, as he (they say) that is led round about a
heath with a Puck in the night, they run earnestly on in this labyrinth of
anxious and solicitous melancholy meditations, and cannot well or willingly
refrain, or easily leave off, winding and unwinding themselves, as so many
clocks, and still pleasing their humours, until at last the scene is turned
upon a sudden, by some bad object, and they being now habituated to such vain
meditations and solitary places, can endure no company, can ruminate of nothing
but harsh and distasteful subjects. Fear, sorrow, suspicion, \li{subrusticus
pudor}, discontent, cares, and weariness of life surprise them in a moment, and
they can think of nothing else, continually suspecting, no sooner are their
eyes open, but this infernal plague of melancholy seizeth on them, and
terrifies their souls, representing some dismal object to their minds, which
now by no means, no labour, no persuasions they can avoid, \li{haeret lateri
lethalis arundo}, (the arrow of death still remains in the side), they may not
be rid of it, \authorfootnote{1561}they cannot resist. I may not deny but that
there is some profitable meditation, contemplation, and kind of solitariness to
be embraced, which the fathers so highly commended,
\authorfootnote{1562}Hierom, \Chrysostom{}, Cyprian, Austin, in whole tracts,
which Petrarch, Erasmus, Stella, and others, so much magnify in their books; a
paradise, a heaven on earth, if it be used aright, good for the body, and
better for the soul: as many of those old monks used it, to divine
contemplations, as Simulus, a courtier in Adrian's time, Diocletian the
emperor, retired themselves, \etc{}, in that sense, \li{Vatia solus scit
vivere}, Vatia lives alone, which the Romans were wont to say, when they
commended a country life. Or to the bettering of their knowledge, as
Democritus, Cleanthes, and those excellent philosophers have ever done, to
sequester themselves from the tumultuous world, or as in Pliny's villa
Laurentana, \Tully{}'s Tusculan, Jovius' study, that they might better \li{vacare
studiis et Deo}, serve God, and follow their studies. Methinks, therefore, our
too zealous innovators were not so well advised in that general subversion of
abbeys and religious houses, promiscuously to fling down all; they might have
taken away those gross abuses crept in amongst them, rectified such
inconveniences, and not so far to have raved and raged against those fair
buildings, and everlasting monuments of our forefathers' devotion, consecrated
to pious uses; some monasteries and collegiate cells might have been well
spared, and their revenues otherwise employed, here and there one, in good
towns or cities at least, for men and women of all sorts and conditions to live
in, to sequester themselves from the cares and tumults of the world, that were
not desirous, or fit to marry; or otherwise willing to be troubled with common
affairs, and know not well where to bestow themselves, to live apart in, for
more conveniency, good education, better company sake, to follow their studies
(I say), to the perfection of arts and sciences, common good, and as some truly
devoted monks of old had done, freely and truly to serve God. For these men are
neither solitary, nor idle, as the poet made answer to the husbandman in Aesop,
that objected idleness to him; he was never so idle as in his company; or that
Scipio Africanus in \authorfootnote{1563}\Tully{}, \li{Nunquam minus solus, quam
cum solus; nunquam minus otiosus, quam quum esset otiosus}; never less
solitary, than when he was alone, never more busy, than when he seemed to be
most idle. It is reported by Plato in his dialogue \bookcite{\textlatin{de
Amore}}, in that prodigious commendation of Socrates, how a deep meditation
coming into Socrates' mind by chance, he stood still musing, \li{eodem vestigio
cogitabundus}, from morning to noon, and when as then he had not yet finished
his meditation, \li{perstabat cogitans}, he so continued till the evening, the
soldiers (for he then followed the camp) observed him with admiration, and on
set purpose watched all night, but he persevered immovable \li{ad exhortim
solis}, till the sun rose in the morning, and then saluting the sun, went his
ways. In what humour constant Socrates did thus, I know not, or how he might be
affected, but this would be pernicious to another man; what intricate business
might so really possess him, I cannot easily guess; but this is \li{otiosum
otium}, it is far otherwise with these men, according to \Seneca{}, \li{Omnia
nobis mala solitudo persuadet}; this solitude undoeth us, \li{pugnat cum vita
sociali}; 'tis a destructive solitariness. These men are devils alone, as the
saying is, \li{Homo solus aut Deus, aut Daemon}: a man alone, is either a saint
or a devil, \li{mens ejus aut languescit, aut tumescit}; and
\authorfootnote{1564}\li{Vae soli} in this sense, woe be to him that is so
alone. These wretches do frequently degenerate from men, and of sociable
creatures become beasts, monsters, inhumane, ugly to behold, \li{Misanthropi};
they do even loathe themselves, and hate the company of men, as so many Timons,
Nebuchadnezzars, by too much indulging to these pleasing humours, and through
their own default. So that which Mercurialis, \bookcite{\textlatin{consil.
11}}, sometimes expostulated with his melancholy patient, may be justly applied
to every solitary and idle person in particular.
\authorfootnote{1565}\li{Natura de te videtur conqueri posse}, \etc{} "Nature
may justly complain of thee, that whereas she gave thee a good wholesome
temperature, a sound body, and God hath given thee so divine and excellent a
soul, so many good parts, and profitable gifts, thou hast not only contemned
and rejected, but hast corrupted them, polluted them, overthrown their
temperature, and perverted those gifts with riot, idleness, solitariness, and
many other ways, thou art a traitor to God and nature, an enemy to thyself and
to the world." \li{Perditio tua ex te}; thou hast lost thyself wilfully, cast
away thyself, "thou thyself art the efficient cause of thine own misery, by not
resisting such vain cogitations, but giving way unto them."


%SECT. II. MEMB. II. SUBSECT. VII.-_Sleeping and Waking, Causes_.
\section{Sleeping and Waking, Causes.}

\lettrine{W}{hat} I have formerly said of exercise, I may now repeat of sleep.
Nothing better than moderate sleep, nothing worse than it, if it be in
extremes, or unseasonably used. It is a received opinion, that a melancholy man
cannot sleep overmuch; \li{Somnus supra modum prodest}, as an only antidote,
and nothing offends them more, or causeth this malady sooner, than waking, yet
in some cases sleep may do more harm than good, in that phlegmatic, swinish,
cold, and sluggish melancholy which Melancthon speaks of, that thinks of
waters, sighing most part, \etc{} \authorfootnote{1566}It dulls the spirits, if
overmuch, and senses; fills the head full of gross humours; causeth
distillations, rheums, great store of excrements in the brain, and all the
other parts, as \authorfootnote{1567}Fuchsius speaks of them, that sleep like
so many dormice. Or if it be used in the daytime, upon a full stomach, the body
ill-composed to rest, or after hard meats, it increaseth fearful dreams,
incubus, night walking, crying out, and much unquietness; such sleep prepares
the body, as \authorfootnote{1568}one observes, "to many perilous diseases."
But, as I have said, waking overmuch, is both a symptom, and an ordinary cause.
"It causeth dryness of the brain, frenzy, dotage, and makes the body dry, lean,
hard, and ugly to behold," as \authorfootnote{1569}Lemnius hath it. "The
temperature of the brain is corrupted by it, the humours adust, the eyes made
to sink into the head, choler increased, and the whole body inflamed:" and, as
may be added out of Galen, \bookcite{\textlatin{3. de sanitate tuendo}},
\Avicenna{} \bookcite{\textlatin{3. 1.}} \authorfootnote{1570}"It overthrows the
natural heat, it causeth crudities, hurts, concoction," and what not? Not
without good cause therefore Crato, \bookcite{\textlatin{consil. 21. lib. 2}};
Hildesheim, \bookcite{\textlatin{spicel. 2. de delir. et Mania}}, Jacchinus,
Arculanus on Rhasis, Guianerius and Mercurialis, reckon up this overmuch waking
as a principal cause.

%\chapter{ MEMB. III.} SECT. II. MEMB. III. SECT. II. MEMB. III. SUBSECT.
%I.-_Passions and Perturbations of the Mind, how they cause Melancholy_.
\section{Passions and Perturbations of the Mind, how they cause Melancholy.}

\lettrine{A}{s} that gymnosophist in \authorfootnote{1571}\Plutarch{} made answer
to Alexander (demanding which spake best), Every one of his fellows did speak
better than the other: so may I say of these causes; to him that shall require
which is the greatest, every one is more grievous than other, and this of
passion the greatest of all. A most frequent and ordinary cause of melancholy,
\authorfootnote{1572}\li{fulmen perturbationum} (Picolomineus calls it) this
thunder and lightning of perturbation, which causeth such violent and speedy
alterations in this our microcosm, and many times subverts the good estate and
temperature of it. For as the body works upon the mind by his bad humours,
troubling the spirits, sending gross fumes into the brain, and so \li{per
consequens} disturbing the soul, and all the faculties of it,

\translatedverse{%
\begin{latin}
\begin{verse}%
------Corpus onustum,\\*
Hesternis vitiis animum quoque praegravat una,\\!
\end{verse}%
\end{latin}}{%\setauthornote{1573.5}
\begin{verse}%
The body oppressed by yesterday's vices\\*
weighs down the spirit also,\\!
\end{verse}}{%
\attrib{\getauthornote{1573}}}
with fear, sorrow, \etc{}, which are ordinary symptoms of this disease: so on
the other side, the mind most effectually works upon the body, producing by his
passions and perturbations miraculous alterations, as melancholy, despair,
cruel diseases, and sometimes death itself. Insomuch that it is most true which
Plato saith in his Charmides, \li{omnia corporis mala ab anima procedere}; all
the \authorfootnote{1574}mischiefs of the body proceed from the soul: and
Democritus in \authorfootnote{1575}\Plutarch{} urgeth, \li{Damnatam iri animam a
corpore}, if the body should in this behalf bring an action against the soul,
surely the soul would be cast and convicted, that by her supine negligence had
caused such inconveniences, having authority over the body, and using it for an
instrument, as a smith doth his hammer (saith \authorfootnote{1576}Cyprian),
imputing all those vices and maladies to the mind. Even so doth
\authorfootnote{1577}Philostratus, \li{non coinquinatur corpus, nisi
consensuanimae}; the body is not corrupted, but by the soul. Lodovicus Vives
will have such turbulent commotions proceed from ignorance and indiscretion.
\authorfootnote{1578}All philosophers impute the miseries of the body to the
soul, that should have governed it better, by command of reason, and hath not
done it. The Stoics are altogether of opinion (as \authorfootnote{1579}Lipsius
and \authorfootnote{1580}Picolomineus record), that a wise man should be
\textgreek{ἀπαθής}, without all manner of passions and perturbations
whatsoever, as \authorfootnote{1581}\Seneca{} reports of Cato, the
\authorfootnote{1582}Greeks of Socrates, and \authorfootnote{1583}Io. Aubanus
of a nation in Africa, so free from passion, or rather so stupid, that if they
be wounded with a sword, they will only look back.
\authorfootnote{1584}Lactantius, \bookcite{\textlatin{2 instit.}}, will exclude
"fear from a wise man:" others except all, some the greatest passions. But let
them dispute how they will, set down in Thesi, give precepts to the contrary;
we find that of \authorfootnote{1585}Lemnius true by common experience; "No
mortal man is free from these perturbations: or if he be so, sure he is either
a god, or a block." They are born and bred with us, we have them from our
parents by inheritance. \li{A parentibus habemus malum hunc assem}, saith
\authorfootnote{1586}Pelezius, \li{Nascitur una nobiscum, aliturque}, 'tis
propagated from Adam, Cain was melancholy, \authorfootnote{1587}as Austin hath
it, and who is not? Good discipline, education, philosophy, divinity (I cannot
deny), may mitigate and restrain these passions in some few men at some times,
but most part they domineer, and are so violent, \authorfootnote{1588}that as a
torrent (\li{torrens velut aggere rupto}) bears down all before, and overflows
his banks, \lit{sternit agros, sternit sata}{lays waste the fields, prostrates
the crops}, they overwhelm reason, judgment, and pervert the temperature of the
body; \li{Fertur \authorfootnote{1589}equis auriga, nec audit currus habenas}.
Now such a man (saith \authorfootnote{1590}Austin) "that is so led, in a wise
man's eye, is no better than he that stands upon his head." It is doubted by
some, \li{Gravioresne morbi a perturbationibus, an ab humoribus}, whether
humours or perturbations cause the more grievous maladies. But we find that of
our Saviour, \biblecite{Mat. \rn{xxvi.} 41}, most true, "The spirit is willing,
the flesh is weak," we cannot resist; and this of \authorfootnote{1591}Philo
Judeus, "Perturbations often offend the body, and are most frequent causes of
melancholy, turning it out of the hinges of his health." Vives compares them to
\authorfootnote{1592}"Winds upon the sea, some only move as those great gales,
but others turbulent quite overturn the ship." Those which are light, easy, and
more seldom, to our thinking, do us little harm, and are therefore contemned of
us: yet if they be reiterated, \authorfootnote{1593}"as the rain" (saith
Austin) "doth a stone, so do these perturbations penetrate the mind:"
\authorfootnote{1594}and (as one observes) "produce a habit of melancholy at
the last," which having gotten the mastery in our souls, may well be called
diseases.

How these passions produce this effect, \authorfootnote{1595}Agrippa hath
handled at large, \bookcite{\textlatin{Occult. Philos. l. 11. c. 63.}} Cardan,
\bookcite{\textlatin{l. 14. subtil.}} Lemnius, \bookcite{\textlatin{l. 1. c.
12, de occult. nat. mir. et lib. 1. cap. 16.}} Suarez,
\bookcite{\textlatin{Met. disput. 18. sect. 1. art. 25.}} T. Bright,
\bookcite{\textlatin{cap. 12.}} of his Melancholy Treatise. Wright the Jesuit,
in his Book of the Passions of the Mind, \etc{} Thus in brief, to our
imagination cometh by the outward sense or memory, some object to be known
(residing in the foremost part of the brain), which he misconceiving or
amplifying presently communicates to the heart, the seat of all affections. The
pure spirits forthwith flock from the brain to the heart, by certain secret
channels, and signify what good or bad object was presented;
\authorfootnote{1596}which immediately bends itself to prosecute, or avoid it;
and withal, draweth with it other humours to help it: so in pleasure, concur
great store of purer spirits; in sadness, much melancholy blood; in ire,
choler. If the imagination be very apprehensive, intent, and violent, it sends
great store of spirits to, or from the heart, and makes a deeper impression,
and greater tumult, as the humours in the body be likewise prepared, and the
temperature itself ill or well disposed, the passions are longer and stronger;
so that the first step and fountain of all our grievances in this kind, is
\authorfootnote{1597}\li{laesa imaginatio}, which misinforming the heart,
causeth all these distemperatures, alteration and confusion of spirits and
humours. By means of which, so disturbed, concoction is hindered, and the
principal parts are much debilitated; as \authorfootnote{1598}Dr. Navarra well
declared, being consulted by Montanus about a melancholy Jew. The spirits so
confounded, the nourishment must needs be abated, bad humours increased,
crudities and thick spirits engendered with melancholy blood. The other parts
cannot perform their functions, having the spirits drawn from them by vehement
passion, but fail in sense and motion; so we look upon a thing, and see it not;
hear, and observe not; which otherwise would much affect us, had we been free.
I may therefore conclude with \authorfootnote{1599}Arnoldus, \li{Maxima vis est
phantasiae, et huic uni fere, non autem corporis intemperiei, omnis
melancholiae causa est ascribenda}: "Great is the force of imagination, and
much more ought the cause of melancholy to be ascribed to this alone, than to
the distemperature of the body." Of which imagination, because it hath so great
a stroke in producing this malady, and is so powerful of itself, it will not be
improper to my discourse, to make a brief digression, and speak of the force of
it, and how it causeth this alteration. Which manner of digression, howsoever
some dislike, as frivolous and impertinent, yet I am of
\authorfootnote{1600}Beroaldus's opinion, "Such digressions do mightily delight
and refresh a weary reader, they are like sauce to a bad stomach, and I do
therefore most willingly use them."


%SECT. II. MEMB. III. SUBSECT. II.-_Of the Force of Imagination_.
\section{Of the Force of Imagination.}\label{sec:of-the-force-of-imagination}

\lettrine{W}{hat} imagination is, I have sufficiently declared in my digression
of the anatomy of the soul. I will only now point at the wonderful effects and
power of it; which, as it is eminent in all, so most especially it rageth in
melancholy persons, in keeping the species of objects so long, mistaking,
amplifying them by continual and \authorfootnote{1601}strong meditation, until
at length it produceth in some parties real effects, causeth this, and many
other maladies. And although this phantasy of ours be a subordinate faculty to
reason, and should be ruled by it, yet in many men, through inward or outward
distemperatures, defect of organs, which are unapt, or otherwise contaminated,
it is likewise unapt, or hindered, and hurt. This we see verified in sleepers,
which by reason of humours and concourse of vapours troubling the phantasy,
imagine many times absurd and prodigious things, and in such as are troubled
with incubus, or witch-ridden (as we call it), if they lie on their backs, they
suppose an old woman rides, and sits so hard upon them, that they are almost
stifled for want of breath; when there is nothing offends, but a concourse of
bad humours, which trouble the phantasy. This is likewise evident in such as
walk in the night in their sleep, and do strange feats:
\authorfootnote{1602}these vapours move the phantasy, the phantasy the
appetite, which moving the animal spirits causeth the body to walk up and down
as if they were awake. Fracast. \bookcite{\textlatin{l. 3. de intellect}},
refers all ecstasies to this force of imagination, such as lie whole days
together in a trance: as that priest whom \authorfootnote{1603}Celsus speaks
of, that could separate himself from his senses when he list, and lie like a
dead man, void of life and sense. Cardan brags of himself, that he could do as
much, and that when he list. Many times such men when they come to themselves,
tell strange things of heaven and hell, what visions they have seen; as that
St. Owen, in Matthew Paris, that went into St. Patrick's purgatory, and the
monk of Evesham in the same author. Those common apparitions in Bede and
Gregory, Saint Bridget's revelations, Wier. \bookcite{\textlatin{l. 3. de
lamiis, c. 11.}} Caesar Vanninus, in his Dialogues, \etc{} reduceth (as I have
formerly said), with all those tales of witches' progresses, dancing, riding,
transformations, operations, \etc{} to the force of
\authorfootnote{1604}imagination, and the \authorfootnote{1605}devil's
illusions. The like effects almost are to be seen in such as are awake: how
many chimeras, antics, golden mountains and castles in the air do they build
unto themselves? I appeal to painters, mechanicians, mathematicians. Some
ascribe all vices to a false and corrupt imagination, anger, revenge, lust,
ambition, covetousness, which prefers falsehood before that which is right and
good, deluding the soul with false shows and suppositions.
\authorfootnote{1606}Bernardus Penottus will have heresy and superstition to
proceed from this fountain; as he falsely imagineth, so he believeth; and as he
conceiveth of it, so it must be, and it shall be, \li{contra gentes}, he will
have it so. But most especially in passions and affections, it shows strange
and evident effects: what will not a fearful man conceive in the dark? What
strange forms of bugbears, devils, witches, goblins? Lavater imputes the
greatest cause of spectrums, and the like apparitions, to fear, which above all
other passions begets the strongest imagination (saith
\authorfootnote{1607}Wierus), and so likewise love, sorrow, joy, \etc{} Some
die suddenly, as she that saw her son come from the battle at Cannae, \etc{}
Jacob the patriarch, by force of imagination, made speckled lambs, laying
speckled rods before his sheep. Persina, that Ethiopian queen in Heliodorus, by
seeing the picture of Persius and Andromeda, instead of a blackamoor, was
brought to bed of a fair white child. In imitation of whom belike, a
hard-favoured fellow in Greece, because he and his wife were both deformed, to
get a good brood of children, \li{Elegantissimas imagines in thalamo
collocavit}, \etc{} hung the fairest pictures he could buy for money in his
chamber, "That his wife by frequent sight of them, might conceive and bear such
children." And if we may believe Bale, one of Pope Nicholas the Third's
concubines by seeing of \authorfootnote{1608}a bear was brought to bed of a
monster. "If a woman" (saith \authorfootnote{1609}Lemnius), "at the time of her
conception think of another man present or absent, the child will be like him."
Great-bellied women, when they long, yield us prodigious examples in this kind,
as moles, warts, scars, harelips, monsters, especially caused in their children
by force of a depraved phantasy in them: \li{Ipsam speciem quam animo effigiat,
faetui inducit}: She imprints that stamp upon her child which she
\authorfootnote{1610}conceives unto herself. And therefore Lodovicus Vives,
\bookcite{\textlatin{lib. 2. de Christ, faem.}}, gives a special caution to
great-bellied women, \authorfootnote{1611}"that they do not admit such absurd
conceits and cogitations, but by all means avoid those horrible objects, heard
or seen, or filthy spectacles." Some will laugh, weep, sigh, groan, blush,
tremble, sweat, at such things as are suggested unto them by their imagination.
\Avicenna{} speaks of one that could cast himself into a palsy when he list; and
some can imitate the tunes of birds and beasts that they can hardly be
discerned: Dagebertus' and Saint Francis' scars and wounds, like those of
Christ's (if at the least any such were), \authorfootnote{1612}Agrippa
supposeth to have happened by force of imagination: that some are turned to
wolves, from men to women, and women again to men (which is constantly
believed) to the same imagination; or from men to asses, dogs, or any other
shapes. \authorfootnote{1613}Wierus ascribes all those famous transformations
to imagination; that in hydrophobia they seem to see the picture of a dog,
still in their water, \authorfootnote{1614}that melancholy men and sick men
conceive so many fantastical visions, apparitions to themselves, and have such
absurd apparitions, as that they are kings, lords, cocks, bears, apes, owls;
that they are heavy, light, transparent, great and little, senseless and dead
(as shall be showed more at large, in our \authorfootnote{1615}sections of
symptoms), can be imputed to nought else, but to a corrupt, false, and violent
imagination. It works not in sick and melancholy men only, but even most
forcibly sometimes in such as are sound: it makes them suddenly sick, and
\authorfootnote{1616}alters their temperature in an instant. And sometimes a
strong conceit or apprehension, as \authorfootnote{1617}Valesius proves, will
take away diseases: in both kinds it will produce real effects. Men, if they
see but another man tremble, giddy or sick of some fearful disease, their
apprehension and fear is so strong in this kind, that they will have the same
disease. Or if by some soothsayer, wiseman, fortune-teller, or physician, they
be told they shall have such a disease, they will so seriously apprehend it,
that they will instantly labour of it. A thing familiar in China (saith Riccius
the Jesuit), \authorfootnote{1618}"If it be told them they shall be sick on
such a day, when that day comes they will surely be sick, and will be so
terribly afflicted, that sometimes they die upon it." Dr. Cotta in his
discovery of ignorant practitioners of physic, \bookcite{\textlatin{cap. 8}},
hath two strange stories to this purpose, what fancy is able to do. The one of
a parson's wife in Northamptonshire, \emph{An.} 1607, that coming to a
physician, and told by him that she was troubled with the sciatica, as he
conjectured (a disease she was free from), the same night after her return,
upon his words, fell into a grievous fit of a sciatica: and such another
example he hath of another good wife, that was so troubled with the cramp,
after the same manner she came by it, because her physician did but name it.
Sometimes death itself is caused by force of phantasy. I have heard of one that
coming by chance in company of him that was thought to be sick of the plague
(which was not so) fell down suddenly dead. Another was sick of the plague with
conceit. One seeing his fellow let blood falls down in a swoon. Another (saith
\authorfootnote{1619}Cardan out of \Aristotle{}), fell down dead (which is
familiar to women at any ghastly sight), seeing but a man hanged. A Jew in
France (saith \authorfootnote{1620}Lodovicus Vives), came by chance over a
dangerous passage or plank, that lay over a brook in the dark, without harm,
the next day perceiving what danger he was in, fell down dead. Many will not
believe such stories to be true, but laugh commonly, and deride when they hear
of them; but let these men consider with themselves, as
\authorfootnote{1621}Peter Byarus illustrates it, If they were set to walk upon
a plank on high, they would be giddy, upon which they dare securely walk upon
the ground. Many (saith Agrippa), \authorfootnote{1622}"strong-hearted men
otherwise, tremble at such sights, dazzle, and are sick, if they look but down
from a high place, and what moves them but conceit?" As some are so molested by
phantasy; so some again, by fancy alone, and a good conceit, are as easily
recovered. We see commonly the toothache, gout, falling-sickness, biting of a
mad dog, and many such maladies cured by spells, words, characters, and charms,
and many green wounds by that now so much used \li{Unguentum Armarium},
magnetically cured, which Crollius and Goclenius in a book of late hath
defended, Libavius in a just tract as stiffly contradicts, and most men
controvert. All the world knows there is no virtue in such charms or cures, but
a strong conceit and opinion alone, as \authorfootnote{1623}Pomponatius holds,
"which forceth a motion of the humours, spirits, and blood, which takes away
the cause of the malady from the parts affected." The like we may say of our
magical effects, superstitious cures, and such as are done by mountebanks and
wizards. "As by wicked incredulity many men are hurt" (so saith
\authorfootnote{1624}Wierus of charms, spells, \etc{}), "we find in our
experience, by the same means many are relieved." An empiric oftentimes, and a
silly chirurgeon, doth more strange cures than a rational physician. Nymannus
gives a reason, because the patient puts his confidence in him,
\authorfootnote{1625}which \Avicenna{} "prefers before art, precepts, and all
remedies whatsoever." 'Tis opinion alone (saith \authorfootnote{1626}Cardan),
that makes or mars physicians, and he doth the best cures, according to
Hippocrates, in whom most trust. So diversely doth this phantasy of ours
affect, turn, and wind, so imperiously command our bodies, which as another
\authorfootnote{1627}"Proteus, or a chameleon, can take all shapes; and is of
such force (as Ficinus adds), that it can work upon others, as well as
ourselves." How can otherwise blear eyes in one man cause the like affection in
another? Why doth one man's yawning \authorfootnote{1628}make another yawn? One
man's pissing provoke a second many times to do the like? Why doth scraping of
trenchers offend a third, or hacking of files? Why doth a carcass bleed when
the murderer is brought before it, some weeks after the murder hath been done?
Why do witches and old women fascinate and bewitch children: but as Wierus,
Paracelsus, Cardan, Mizaldus, Valleriola, Caesar Vanninus,
\idxname{campanella}[Campanella], and many philosophers think, the forcible
imagination of the one party moves and alters the spirits of the other. Nay
more, they can cause and cure not only diseases, maladies, and several
infirmities, by this means, as \Avicenna{}, \bookcite{\textlatin{de anim. l. 4.
sect. 4}}, supposeth in parties remote, but move bodies from their places,
cause thunder, lightning, tempests, which opinion Alkindus, Paracelsus, and
some others, approve of. So that I may certainly conclude this strong conceit
or imagination is \li{astrum hominis}, and the rudder of this our ship, which
reason should steer, but, overborne by phantasy, cannot manage, and so suffers
itself, and this whole vessel of ours to be overruled, and often overturned.
Read more of this in Wierus, \bookcite{\textlatin{l. 3. de Lamiis, c. 8, 9,
10.}} Franciscus Valesius, \bookcite{\textlatin{med. controv. l. 5. cont. 6.}}
Marcellus Donatus, \bookcite{\textlatin{l. 2. c. 1. de hist. med. mirabil}}.
Levinus Lemnius, \bookcite{\textlatin{de occult. nat. mir. l. 1. c. 12.}}
Cardan, \bookcite{\textlatin{l. 18. de rerum var}}. Corn. Agrippa,
\bookcite{\textlatin{de occult. plilos. cap. 64, 65.}} Camerarius,
\bookcite{\textlatin{1 cent. cap. 54. horarum subcis}}. Nymannus,
\bookcite{\textlatin{morat. de Imag}}. Laurentius, and him that is \li{instar
omnium}, Fienus, a famous physician of Antwerp, that wrote three books \li{de
viribus imaginationis}. I have thus far digressed, because this imagination is
the medium deferens of passions, by whose means they work and produce many
times prodigious effects: and as the phantasy is more or less intended or
remitted, and their humours disposed, so do perturbations move, more or less,
and take deeper impression.

%SECT. II. MEMB. III. SUBSECT. III.-_Division of Perturbations_.
\section{Division of Perturbations.}

\lettrine{P}{erturbations} and passions, which trouble the phantasy, though
they dwell between the confines of sense and reason, yet they rather follow
sense than reason, because they are drowned in corporeal organs of sense. They
are commonly \authorfootnote{1629}reduced into two inclinations, irascible and
concupiscible. The Thomists subdivide them into eleven, six in the coveting,
and five in the invading. \Aristotle{} reduceth all to pleasure and pain, Plato to
love and hatred, \authorfootnote{1630}Vives to good and bad. If good, it is
present, and then we absolutely joy and love; or to come, and then we desire
and hope for it. If evil, we absolute hate it; if present, it is by sorrow; if
to come fear. These four passions \authorfootnote{1631}Bernard compares "to the
wheels of a chariot, by which we are carried in this world." All other passions
are subordinate unto these four, or six, as some will: love, joy, desire,
hatred, sorrow, fear; the rest, as anger, envy, emulation, pride, jealousy,
anxiety, mercy, shame, discontent, despair, ambition, avarice, \etc{}, are
reducible unto the first; and if they be immoderate, they
\authorfootnote{1632}consume the spirits, and melancholy is especially caused
by them. Some few discreet men there are, that can govern themselves, and curb
in these inordinate affections, by religion, philosophy, and such divine
precepts, of meekness, patience, and the like; but most part for want of
government, out of indiscretion, ignorance, they suffer themselves wholly to be
led by sense, and are so far from repressing rebellious inclinations, that they
give all encouragement unto them, leaving the reins, and using all provocations
to further them: bad by nature, worse by art, discipline,
\authorfootnote{1633}custom, education, and a perverse will of their own, they
follow on, wheresoever their unbridled affections will transport them, and do
more out of custom, self-will, than out of reason. \li{Contumax voluntas}, as
Melancthon calls it, \li{malum facit}: this stubborn will of ours perverts
judgment, which sees and knows what should and ought to be done, and yet will
not do it. \li{Mancipia gulae}, slaves to their several lusts and appetite,
they precipitate and plunge \authorfootnote{1634}themselves into a labyrinth of
cares, blinded with lust, blinded with ambition; \authorfootnote{1635}"They
seek that at God's hands which they may give unto themselves, if they could but
refrain from those cares and perturbations, wherewith they continually macerate
their minds." But giving way to these violent passions of fear, grief, shame,
revenge, hatred, malice, \etc{}, they are torn in pieces, as Actaeon was with
his dogs, and \authorfootnote{1636}crucify their own souls.

%SECT. II. MEMB. III. SUBSECT. IV.-_Sorrow a Cause of Melancholy_.
\section{Sorrow a Cause of Melancholy.}

\subsection{Sorrow. \textlatin{Insanus dolor.}}

\lettrine{I}{n} this catalogue of passions, which so much torment the soul of
man, and cause this malady, (for I will briefly speak of them all, and in their
order,) the first place in this irascible appetite, may justly be challenged by
sorrow. An inseparable companion, \authorfootnote{1637}"The mother and daughter
of melancholy, her epitome, symptom, and chief cause:" as Hippocrates hath it,
they beget one another, and tread in a ring, for sorrow is both cause and
symptom of this disease. How it is a symptom shall be shown in its place. That
it is a cause all the world acknowledgeth, \li{Dolor nonnullis insaniae causa
fuit, et aliorum morborum insanabilium}, saith \Plutarch{} to \Apollonius{}; a cause
of madness, a cause of many other diseases, a sole cause of this mischief,
\authorfootnote{1638}Lemnius calls it. So doth Rhasis,
\bookcite{\textlatin{cont. l. 1. tract. 9.}} Guianerius,
\bookcite{\textlatin{Tract. 15. c. 5}}, And if it take root once, it ends in
despair, as \authorfootnote{1639}Felix Plater observes, and as in
\authorfootnote{1640}Cebes' table, may well be coupled with it.
\authorfootnote{1641}\Chrysostom{}, in his seventeenth epistle to Olympia,
describes it to be "a cruel torture of the soul, a most inexplicable grief,
poisoned worm, consuming body and soul, and gnawing the very heart, a perpetual
executioner, continual night, profound darkness, a whirlwind, a tempest, an
ague not appearing, heating worse than any fire, and a battle that hath no end.
It crucifies worse than any tyrant; no torture, no strappado, no bodily
punishment is like unto it." 'Tis the eagle without question which the poets
feigned to gnaw \authorfootnote{1642}Prometheus' heart, and "no heaviness is
like unto the heaviness of the heart," \biblecite{Eccles. \rn{xxv.} 15, 16}.
\authorfootnote{1643}"Every perturbation is a misery, but grief a cruel
torment," a domineering passion: as in old Rome, when the Dictator was created,
all inferior magistracies ceased; when grief appears, all other passions
vanish. "It dries up the bones," saith Solomon, \biblecite{cap. 17. Prov.}, makes
them hollow-eyed, pale, and lean, furrow-faced, to have dead looks, wrinkled
brows, shrivelled cheeks, dry bodies, and quite perverts their temperature that
are misaffected with it. As Eleonara, that exiled mournful duchess (in our
\authorfootnote{1644}English Ovid), laments to her noble husband Humphrey, Duke
of Gloucester,

\begin{verse}%
Sawest thou those eyes in whose sweet cheerful look\\*
Duke Humphrey once such joy and pleasure took,\\*
Sorrow hath so despoil'd me of all grace,\\*
Thou couldst not say this was my Elnor's face.\\*
Like a foul Gorgon, \etc{}\\!
\end{verse}%

\authorfootnote{1645}"It hinders concoction, refrigerates the heart, takes away
stomach, colour, and sleep, thickens the blood,"
(\authorfootnote{1646}Fernelius, \bookcite{\textlatin{l. 1. c. 18. de morb.
causis}},) "contaminates the spirits." (\authorfootnote{1647}Piso.) Overthrows
the natural heat, perverts the good estate of body and mind, and makes them
weary of their lives, cry out, howl and roar for very anguish of their souls.
David confessed as much, \biblecite{Psalm \rn{xxxviii.} 8}, "I have roared for
the very disquietness of my heart." And \biblecite{Psalm \rn{cxix.} 4, part 4 v}.
"My soul melteth away for very heaviness," \biblecite{v. 38}. "I am like a bottle
in the smoke." Antiochus complained that he could not sleep, and that his heart
fainted for grief, \authorfootnote{1648}Christ himself, \li{vir dolorum}, out
of an apprehension of grief, did sweat blood, \biblecite{Mark \rn{xiv.}} "His
soul was heavy to the death, and no sorrow was like unto his." Crato,
\bookcite{\textlatin{consil. 24. l. 2}}, gives instance in one that was so
melancholy by reason of \authorfootnote{1649}grief; and Montanus,
\bookcite{\textlatin{consil. 30}}, in a noble matron,
\authorfootnote{1650}"that had no other cause of this mischief." I. S. D. in
Hildesheim, fully cured a patient of his that was much troubled with
melancholy, and for many years, \authorfootnote{1651}"but afterwards, by a
little occasion of sorrow, he fell into his former fits, and was tormented as
before." Examples are common, how it causeth melancholy,
\authorfootnote{1652}desperation, and sometimes death itself; for
(\biblecite{Eccles. \rn{xxxviii.} 15},) "Of heaviness comes death; worldly sorrow
causeth death." \biblecite{2 Cor. \rn{vii.} 10}, \biblecite{Psalm \rn{xxxi.} 10},
"My life is wasted with heaviness, and my years with mourning." Why was Hecuba
said to be turned to a dog? Niobe into a stone? but that for grief she was
senseless and stupid. Severus the Emperor \authorfootnote{1653}died for grief;
and how \authorfootnote{1654}many myriads besides? \li{Tanta illi est feritas,
tanta est insania luctus}\authorlatintrans{1655}. Melancthon gives a reason of
it, \authorfootnote{1656}"the gathering of much melancholy blood about the
heart, which collection extinguisheth the good spirits, or at least dulleth
them, sorrow strikes the heart, makes it tremble and pine away, with great
pain; and the black blood drawn from the spleen, and diffused under the ribs,
on the left side, makes those perilous hypochondriacal convulsions, which
happen to them that are troubled with sorrow."

\cleartoleftpage{}
\begin{figure}[p]
  \begingroup
  \centering
  \includegraphics[keepaspectratio,width=\textwidth]{memento-mori-small.jpg}
  \captionart{MementoMori}
  \label{fig:mementomori}
\end{figure}

\clearpage{}
\thispagestyle{titleontop}
%SECT. II. MEMB. III. SUBSECT. V.-_Fear, a Cause_.
\section{Fear, a Cause.}

\lettrine{C}{ousin} german to sorrow, is fear, or rather a sister, \li{fidus
Achates}, and continual companion, an assistant and a principal agent in
procuring of this mischief; a cause and symptom as the other. In a word, as
\authorfootnote{1657}\Virgil{} of the Harpies, I may justly say of them both,

\translatedverse{%
\begin{latin}
\begin{verse}%
Tristius haud illis monstrum, nec saevior ulla\\*
Pestis et ira Deum stygiis sese extulit undis.\\!
\end{verse}%
\end{latin}}{%
\begin{verse}%
A sadder monster, or more cruel plague so fell,\\*
Or vengeance of the gods, ne'er came from Styx or Hell.\\!
\end{verse}}{}

This foul fiend of fear was worshipped heretofore as a god by the
Lacedaemonians, and most of those other torturing
\authorfootnote{1658}affections, and so was sorrow amongst the rest, under the
name of Angerona Dea, they stood in such awe of them, as Austin,
\bookcite{\textlatin{de Civitat. Dei, lib. 4. cap. 8}}, noteth out of Varro,
fear was commonly \authorfootnote{1659}adored and painted in their temples with
a lion's head; and as Macrobius records, \bookcite{\textlatin{l. 10.
Saturnalium}}; \authorfootnote{1660}"In the calends of January, Angerona had
her holy day, to whom in the temple of Volupia, or goddess of pleasure, their
augurs and bishops did yearly sacrifice; that, being propitious to them, she
might expel all cares, anguish, and vexation of the mind for that year
following." Many lamentable effects this fear causeth in men, as to be red,
pale, tremble, sweat, \authorfootnote{1661}it makes sudden cold and heat to
come over all the body, palpitation of the heart, syncope, \etc{} It amazeth
many men that are to speak, or show themselves in public assemblies, or before
some great personages, as \Tully{} confessed of himself, that he trembled still at
the beginning of his speech; and Demosthenes, that great orator of Greece,
before Philippus. It confounds voice and memory, as Lucian wittily brings in
Jupiter Tragoedus, so much afraid of his auditory, when he was to make a speech
to the rest of the Gods, that he could not utter a ready word, but was
compelled to use Mercury's help in prompting. Many men are so amazed and
astonished with fear, they know not where they are, what they say,
\authorfootnote{1662}what they do, and that which is worst, it tortures them
many days before with continual affrights and suspicion. It hinders most
honourable attempts, and makes their hearts ache, sad and heavy. They that live
in fear are never free, \authorfootnote{1663}resolute, secure, never merry, but
in continual pain: that, as Vives truly said, \li{Nulla est miseria major quam
metus}, no greater misery, no rack, nor torture like unto it, ever suspicious,
anxious, solicitous, they are childishly drooping without reason, without
judgment, \authorfootnote{1664}"especially if some terrible object be offered,"
as \Plutarch{} hath it. It causeth oftentimes sudden madness, and almost all
manner of diseases, as I have sufficiently illustrated in my
\authorfootnote{1665}digression of the force of imagination, and shall do more
at large in my section of \authorfootnote{1666}terrors. Fear makes our
imagination conceive what it list, invites the devil to come to us, as
\authorfootnote{1667}Agrippa and Cardan avouch, and tyranniseth over our
phantasy more than all other affections, especially in the dark. We see this
verified in most men, as \authorfootnote{1668}Lavater saith, \li{Quae metuunt,
fingunt}; what they fear they conceive, and feign unto themselves; they think
they see goblins, hags, devils, and many times become melancholy thereby.
Cardan, \bookcite{\textlatin{subtil. lib. 18}}, hath an example of such an one,
so caused to be melancholy (by sight of a bugbear) all his life after. Augustus
Caesar durst not sit in the dark, \li{nisi aliquo assidente}, saith
\authorfootnote{1669}Suetonius, \li{Nunquam tenebris exigilavit}. And 'tis
strange what women and children will conceive unto themselves, if they go over
a churchyard in the night, lie, or be alone in a dark room, how they sweat and
tremble on a sudden. Many men are troubled with future events, foreknowledge of
their fortunes, destinies, as Severus the Emperor, Adrian and Domitian,
\li{Quod sciret ultimum vitae diem}, saith Suetonius, \li{valde solicitus},
much tortured in mind because he foreknew his end; with many such, of which I
shall speak more opportunely in another place. \authorfootnote{1670}Anxiety,
mercy, pity, indignation, \etc{}, and such fearful branches derived from these
two stems of fear and sorrow, I voluntarily omit; read more of them in
\authorfootnote{1671}Carolus Pascalius, \authorfootnote{1672}Dandinus, \etc{}

%SECT. II. MEMB. III. SUBSECT. VI.-_Shame and Disgrace, Causes_.
\section{Shame and Disgrace, Causes.}

\lettrine{S}{hame} and disgrace cause most violent passions and bitter pangs.
\li{Ob pudorem et dedecus publicum, ob errorum commissum saepe moventur
generosi animi} (Felix Plater, \bookcite{\textlatin{lib. 3. de alienat
mentis}}.) Generous minds are often moved with shame, to despair for some
public disgrace. And he, saith Philo, \bookcite{\textlatin{lib. 2. de provid.
dei}}, \authorfootnote{1673}"that subjects himself to fear, grief, ambition,
shame, is not happy, but altogether miserable, tortured with continual labour,
care, and misery." It is as forcible a batterer as any of the rest:
\authorfootnote{1674}"Many men neglect the tumults of the world, and care not
for glory, and yet they are afraid of infamy, repulse, disgrace,"
(\bookcite{\textlatin{Tul. offic. l. 1}},) "they can severely contemn pleasure,
bear grief indifferently, but they are quite \authorfootnote{1675}battered and
broken, with reproach and obloquy:" (\li{siquidem vita et fama pari passu
ambulant}) and are so dejected many times for some public injury, disgrace, as
a box on the ear by their inferior, to be overcome of their adversary, foiled
in the field, to be out in a speech, some foul fact committed or disclosed,
\etc{} that they dare not come abroad all their lives after, but melancholise
in corners, and keep in holes. The most generous spirits are most subject to
it; \li{Spiritus altos frangit et generosos}: Hieronymus. \Aristotle{}, because he
could not understand the motion of Euripus, for grief and shame drowned
himself: Caelius Rodigimus \bookcite{\textlatin{antiquar. lec. lib. 29. cap.
8.}} \li{Homerus pudore consumptus}, was swallowed up with this passion of
shame \authorfootnote{1676}"because he could not unfold the fisherman's
riddle." Sophocles killed himself, \authorfootnote{1677}"for that a tragedy of
his was hissed off the stage:" \bookcite{\textlatin{Valer. max. lib. 9. cap.
12.}} Lucretia stabbed herself, and so did \authorfootnote{1678}Cleopatra,
"when she saw that she was reserved for a triumph, to avoid the infamy."
Antonius the Roman, \authorfootnote{1679}"after he was overcome of his enemy,
for three days' space sat solitary in the fore-part of the ship, abstaining
from all company, even of Cleopatra herself, and afterwards for very shame
butchered himself," \Plutarch{}, \bookcite{\textlatin{vita ejus}}. \Apollonius{}
Rhodius \authorfootnote{1680}"wilfully banished himself, forsaking his country,
and all his dear friends, because he was out in reciting his poems," Plinius,
\bookcite{\textlatin{lib. 7. cap. 23.}} Ajax ran mad, because his arms were
adjudged to Ulysses. In China 'tis an ordinary thing for such as are excluded
in those famous trials of theirs, or should take degrees, for shame and grief
to lose their wits, \authorfootnote{1681}Mat Riccius
\bookcite{\textlatin{expedit. ad Sinas, l. 3. c. 9.}} Hostratus the friar took
that book which Reuclin had writ against him, under the name of
\bookcite{\textlatin{Epist. obscurorum virorum}}, so to heart, that for shame
and grief he made away with himself, \authorfootnote{1682}\li{Jovius in
elogiis}. A grave and learned minister, and an ordinary preacher at Alcmar in
Holland, was (one day as he walked in the fields for his recreation) suddenly
taken with a lax or looseness, and thereupon compelled to retire to the next
ditch; but being \authorfootnote{1683}surprised at unawares, by some
gentlewomen of his parish wandering that way, was so abashed, that he did never
after show his head in public, or come into the pulpit, but pined away with
melancholy: (Pet. Forestus \bookcite{\textlatin{med. observat. lib. 10.
observat. 12.}}) So shame amongst other passions can play his prize.

I know there be many base, impudent, brazenfaced rogues, that will
\authorfootnote{1684}\li{Nulla pallescere culpa}, be moved with nothing, take
no infamy or disgrace to heart, laugh at all; let them be proved perjured,
stigmatised, convict rogues, thieves, traitors, lose their ears, be whipped,
branded, carted, pointed at, hissed, reviled, and derided with
\authorfootnote{1685}Ballio the Bawd in Plautus, they rejoice at it,
\li{Cantores probos}; "babe and Bombax," what care they? We have too many such
in our times,

\translatedverse{%
\begin{latin}
\begin{verse}%
------Exclamat Melicerta perisse\\*
------Frontem de rebus.\\!
\end{verse}%
\end{latin}}{%
\begin{verse}%
------Melicerta exclaims,\\*
------all shame has vanished from human transactions.
\end{verse}}{%
\attrib{\getauthornote{1686}}}

Yet a modest man, one that hath grace, a generous spirit, tender of his
reputation, will be deeply wounded, and so grievously affected with it, that he
had rather give myriads of crowns, lose his life, than suffer the least
defamation of honour, or blot in his good name. And if so be that he cannot
avoid it, as a nightingale, \li{Que cantando victa moritur}, (saith
\authorfootnote{1687}Mizaldus,) dies for shame if another bird sing better, he
languisheth and pineth away in the anguish of his spirit.

\cleartoleftpage{}
\begin{figure}[p]
  \begingroup
  \centering
  \includegraphics[keepaspectratio,width=0.9\textwidth]{nvidia-small.jpg}
  \captionart{Invidia}
  \label{fig:invidia}
\end{figure}

% Force float here
\clearpage{}
\thispagestyle{titleontop}

%SECT. II. MEMB. III. SUBSECT. VII.-_Envy, Malice, Hatred, Causes_.
\section{Envy, Malice, Hatred, Causes.}

\lettrine{E}{nvy} and malice are two links of this chain, and both, as
Guianerius, \bookcite{\textlatin{Tract. 15. cap. 2}}, proves out of Galen,
\bookcite{\textlatin{3 Aphorism, com. 22}}, \authorfootnote{1688}"cause this
malady by themselves, especially if their bodies be otherwise disposed to
melancholy." 'Tis Valescus de Taranta, and Felix Platerus' observation,
\authorfootnote{1689}"Envy so gnaws many men's hearts, that they become
altogether melancholy." And therefore belike Solomon, \biblecite{Prov. \rn{xiv.}
13}, calls it, "the rotting of the bones," Cyprian, \li{vulnus occultum};

\begin{latin}
\begin{verse}%
------Siculi non invenere tyranni\\*
Majus tormentum------\\!
\end{verse}%
\end{latin}
\attrib{\getauthornote{1690}}

The Sicilian tyrants never invented the like torment. It crucifies their souls,
withers their bodies, makes them hollow-eyed, \authorfootnote{1691}pale, lean,
and ghastly to behold, Cyprian, \bookcite{\textlatin{ser. 2. de zelo et
livore}}. \authorfootnote{1692}"As a moth gnaws a garment, so," saith
\Chrysostom{}, "doth envy consume a man;" to be a living anatomy: a "skeleton, to
be a lean and \authorfootnote{1693}pale carcass, quickened with a
\authorfootnote{1694}fiend", Hall \bookcite{\textlatin{in Charact.}} for so
often as an envious wretch sees another man prosper, to be enriched, to thrive,
and be fortunate in the world, to get honours, offices, or the like, he repines
and grieves.

\translatedverse{%
\begin{latin}
\begin{verse}
------intabescitque videndo\\*
Successus hominum--suppliciumque suum est.\\!
\end{verse}
\end{latin}}{%\authorlatintrans{1695.5}
\begin{verse}%
He pines away at the sight of another's success\\*
it is his special torture\\!
\end{verse}}{%
\attrib{\getauthornote{1695}}}

He tortures himself if his equal, friend, neighbour, be preferred, commended,
do well; if he understand of it, it galls him afresh; and no greater pain can
come to him than to hear of another man's well-doing; 'tis a dagger at his
heart every such object. He looks at him as they that fell down in Lucian's
rock of honour, with an envious eye, and will damage himself, to do another a
mischief: \li{Atque cadet subito, dum super hoste cadat}. As he did in Aesop,
lose one eye willingly, that his fellow might lose both, or that rich man in
\authorfootnote{1696}Quintilian that poisoned the flowers in his garden,
because his neighbour's bees should get no more honey from them. His whole life
is sorrow, and every word he speaks a satire: nothing fats him but other men's
ruins. For to speak in a word, envy is nought else but \li{Tristitia de bonis
alienis}, sorrow for other men's good, be it present, past, or to come: \li{et
gaudium de adversis}, and \authorfootnote{1697}joy at their harms, opposite to
mercy, \authorfootnote{1698}which grieves at other men's mischances, and
misaffects the body in another kind; so Damascen defines it,
\bookcite{\textlatin{lib. 2. de orthod. fid.}} Thomas, \bookcite{\textlatin{2.
2. quaest. 36. art. 1.}} \Aristotle{}, \bookcite{\textlatin{l. 2. Rhet. c. 4. et
10.}} Plato \bookcite{\textlatin{Philebo}}. \Tully{}, \bookcite{\textlatin{3.
Tusc}}. Greg. Nic. \bookcite{\textlatin{l. de virt. animae, c. 12.}} Basil,
\bookcite{\textlatin{de Invidia}}. Pindarus \bookcite{\textlatin{Od. 1. ser.
5}}, and we find it true. 'Tis a common disease, and almost natural to us, as
\authorfootnote{1699}Tacitus holds, to envy another man's prosperity. And 'tis
in most men an incurable disease. \authorfootnote{1700}"I have read," saith
Marcus Aurelius, "Greek, Hebrew, Chaldee authors; I have consulted with many
wise men for a remedy for envy, I could find none, but to renounce all
happiness, and to be a wretch, and miserable for ever." 'Tis the beginning of
hell in this life, and a passion not to be excused. \authorfootnote{1701}"Every
other sin hath some pleasure annexed to it, or will admit of an excuse; envy
alone wants both. Other sins last but for awhile; the gut may be satisfied,
anger remits, hatred hath an end, envy never ceaseth." Cardan,
\bookcite{\textlatin{lib. 2. de sap.}} Divine and humane examples are very
familiar; you may run and read them, as that of Saul and David, Cain and Abel,
\li{angebat illum non proprium peccatum, sed fratris prosperitas}, saith
Theodoret, it was his brother's good fortune galled him. Rachel envied her
sister, being barren, \biblecite{Gen. xxx}. Joseph's brethren him, \biblecite{Gen.
xxxvii}. David had a touch of this vice, as he confesseth,
\authorfootnote{1702}\biblecite{Psal. 37}. \authorfootnote{1703}Jeremy and
\authorfootnote{1704}Habakkuk, they repined at others' good, but in the end
they corrected themselves, \biblecite{Psal. 75}, "fret not thyself," \etc{}
Domitian spited Agricola for his worth, \authorfootnote{1705}"that a private
man should be so much glorified." \authorfootnote{1706}Cecinna was envied of
his fellow-citizens, because he was more richly adorned. But of all others,
\authorfootnote{1707}women are most weak, \li{ob pulchritudinem invidae sunt
foeminae (Musaeus) aut amat, aut odit, nihil est tertium (Granatensis.)} They
love or hate, no medium amongst them. \li{Implacabiles plerumque laesae
mulieres}, Agrippina like, \authorfootnote{1708}"A woman, if she see her
neighbour more neat or elegant, richer in tires, jewels, or apparel, is
enraged, and like a lioness sets upon her husband, rails at her, scoffs at her,
and cannot abide her;" so the Roman ladies in Tacitus did at Solonina,
Cecinna's wife, \authorfootnote{1709}"because she had a better horse, and
better furniture, as if she had hurt them with it; they were much offended." In
like sort our gentlewomen do at their usual meetings, one repines or scoffs at
another's bravery and happiness. Myrsine, an Attic wench, was murdered of her
fellows, \authorfootnote{1710}"because she did excel the rest in beauty,"
Constantine, \bookcite{\textlatin{Agricult. l. 11. c. 7.}} Every village will
yield such examples.

%SECT. II. MEMB. III. SUBSECT. VIII.-_Emulation, Hatred, Faction, Desire of Revenge, Causes_.
\section{Emulation, Hatred, Faction, Desire of Revenge, Causes.}

\lettrine{O}{ut} of this root of envy \authorfootnote{1711}spring those feral
branches of faction, hatred, livor, emulation, which cause the like grievances,
and are, \li{serrae animae}, the saws of the soul,
\authorfootnote{1712}\li{consternationis pleni affectus}, affections full of
desperate amazement; or as Cyprian describes emulation, it is
\authorfootnote{1713}"a moth of the soul, a consumption, to make another man's
happiness his misery, to torture, crucify, and execute himself, to eat his own
heart. Meat and drink can do such men no good, they do always grieve, sigh, and
groan, day and night without intermission, their breast is torn asunder:" and a
little after, \authorfootnote{1714}"Whomsoever he is whom thou dost emulate and
envy, he may avoid thee, but thou canst neither avoid him nor thyself;
wheresoever thou art he is with thee, thine enemy is ever in thy breast, thy
destruction is within thee, thou art a captive, bound hand and foot, as long as
thou art malicious and envious, and canst not be comforted. It was the devil's
overthrow;" and whensoever thou art thoroughly affected with this passion, it
will be thine. Yet no perturbation so frequent, no passion so common.

\translatedverse{%
\begin{greek}%
\begin{verse}%
καὶ κεραμεὺς κεραμεῖ κοτέει\\*
καὶ τέκτονι τέκτων,\\*
καὶ πτωχὸς πτωχῷ φθονέει\\*
καὶ ἀοιδὸς ἀοιδῷ.\\!
\end{verse}%
\end{greek}}{%
\begin{verse}
A potter emulates a potter:\\*
One smith envies another:\\*
A beggar emulates a beggar;\\*
A singing man his brother.\\!
\end{verse}}{%
\attrib{\getauthornote{1715}\protect\cite{hesiod01}}}

Every society, corporation, and private family is full of it, it takes hold
almost of all sorts of men, from the prince to the ploughman, even amongst
gossips it is to be seen, scarce three in a company but there is siding,
faction, emulation, between two of them, some \li{simultas}, jar, private
grudge, heart-burning in the midst of them. Scarce two gentlemen dwell together
in the country, (if they be not near kin or linked in marriage) but there is
emulation betwixt them and their servants, some quarrel or some grudge betwixt
their wives or children, friends and followers, some contention about wealth,
gentry, precedency, \etc{}, by means of which, like the frog in
\authorfootnote{1716}Aesop, "that would swell till she was as big as an ox,
burst herself at last;" they will stretch beyond their fortunes, callings, and
strive so long that they consume their substance in lawsuits, or otherwise in
hospitality, feasting, fine clothes, to get a few bombast titles, for
\li{ambitiosa paupertate laboramus omnes}, to outbrave one another, they will
tire their bodies, macerate their souls, and through contentions or mutual
invitations beggar themselves. Scarce two great scholars in an age, but with
bitter invectives they fall foul one on the other, and their adherents;
Scotists, Thomists, Reals, Nominals, Plato and \Aristotle{}, Galenists and
Paracelsians, \etc{}, it holds in all professions.

Honest \authorfootnote{1717}emulation in studies, in all callings is not to be
disliked, 'tis \li{ingeniorum cos}, as one calls it, the whetstone of wit, the
nurse of wit and valour, and those noble Romans out of this spirit did brave
exploits. There is a modest ambition, as Themistocles was roused up with the
glory of Miltiades; Achilles' trophies moved Alexander,

\translatedverse{%
\begin{latin}
\begin{verse}%
Ambire semper stulta confidentia est,\\*
Ambire nunquam deses arrogantia est.\\!
\end{verse}%
\end{latin}}{%%\setauthornote{1718.5}
\begin{verse}%
Ambition always is a foolish confidence,\\*
never a slothful arrogance.\\!
\end{verse}}{%
\attrib{\getauthornote{1718}}}

'Tis a sluggish humour not to emulate or to sue at all, to withdraw himself,
neglect, refrain from such places, honours, offices, through sloth,
niggardliness, fear, bashfulness, or otherwise, to which by his birth, place,
fortunes, education, he is called, apt, fit, and well able to undergo; but when
it is immoderate, it is a plague and a miserable pain. What a deal of money did
Henry VIII. and Francis I. king of France, spend at that
\authorfootnote{1719}famous interview? and how many vain courtiers, seeking
each to outbrave other, spent themselves, their livelihood and fortunes, and
died beggars? \authorfootnote{1720}Adrian the Emperor was so galled with it,
that he killed all his equals; so did Nero. This passion made
\authorfootnote{1721}Dionysius the tyrant banish Plato and Philoxenus the poet,
because they did excel and eclipse his glory, as he thought; the Romans exile
Coriolanus, confine Camillus, murder Scipio; the Greeks by ostracism to expel
Aristides, Nicias, Alcibiades, imprison Theseus, make away Phocion, \etc{} When
Richard I. and Philip of France were fellow soldiers together, at the siege of
Acon in the Holy Land, and Richard had approved himself to be the more valiant
man, insomuch that all men's eyes were upon him, it so galled Philip,
\li{Francum urebat Regis victoria}, saith mine \authorfootnote{1722}author,
\li{tam aegre ferebat Richardi gloriam, ut carpere dicta, calumniari facta};
that he cavilled at all his proceedings, and fell at length to open defiance;
he could contain no longer, but hasting home, invaded his territories, and
professed open war. "Hatred stirs up contention," \biblecite{Prov. \rn{x.} 12},
and they break out at last into immortal enmity, into virulency, and more than
Vatinian hate and rage; \authorfootnote{1723}they persecute each other, their
friends, followers, and all their posterity, with bitter taunts, hostile wars,
scurrile invectives, libels, calumnies, fire, sword, and the like, and will not
be reconciled. Witness that Guelph and Ghibelline faction in Italy; that of the
Adurni and Fregosi in Genoa; that of Cneius Papirius, and Quintus Fabius in
Rome; Caesar and Pompey; Orleans and Burgundy in France; York and Lancaster in
England: yea, this passion so rageth \authorfootnote{1724}many times, that it
subverts not men only, and families, but even populous cities.
\authorfootnote{1725}Carthage and Corinth can witness as much, nay, flourishing
kingdoms are brought into a wilderness by it. This hatred, malice, faction, and
desire of revenge, invented first all those racks and wheels, strappadoes,
brazen bulls, feral engines, prisons, inquisitions, severe laws to macerate and
torment one another. How happy might we be, and end our time with blessed days
and sweet content, if we could contain ourselves, and, as we ought to do, put
up injuries, learn humility, meekness, patience, forget and forgive, as in
\authorfootnote{1726}God's word we are enjoined, compose such final
controversies amongst ourselves, moderate our passions in this kind, "and think
better of others," as \authorfootnote{1727}Paul would have us, "than of
ourselves: be of like affection one towards another, and not avenge ourselves,
but have peace with all men." But being that we are so peevish and perverse,
insolent and proud, so factious and seditious, so malicious and envious; we do
\li{invicem angariare}, maul and vex one another, torture, disquiet, and
precipitate ourselves into that gulf of woes and cares, aggravate our misery
and melancholy, heap upon us hell and eternal damnation.

%SECT. II. MEMB. III. SUBSECT. IX.-_Anger, a Cause_.
\section{Anger, a Cause.}

\lettrine{A}{nger}, a perturbation, which carries the spirits outwards,
preparing the body to melancholy, and madness itself: \li{Ira furor brevis
est}, "anger is temporary madness;" and as \authorfootnote{1728}Picolomineus
accounts it, one of the three most violent passions.
\authorfootnote{1729}Areteus sets it down for an especial cause (so doth
\Seneca{}, \bookcite{\textlatin{ep. 18. l. 1}},) of this malady.
\authorfootnote{1730}Magninus gives the reason, \li{Ex frequenti ira supra
modum calefiunt}; it overheats their bodies, and if it be too frequent, it
breaks out into manifest madness, saith St. Ambrose. 'Tis a known saying,
\li{Furor fit Iaesa saepius palienlia}, the most patient spirit that is, if he
be often provoked, will be incensed to madness; it will make a devil of a
saint: and therefore Basil (belike) in his Homily \bookcite{\textlatin{de
Ira}}, calls it \li{tenebras rationis, morbum animae, et daemonem pessimum};
the darkening of our understanding, and a bad angel.
\authorfootnote{1731}Lucian, \bookcite{\textlatin{in Abdicato, tom. 1}}, will
have this passion to work this effect, especially in old men and women. "Anger
and calumny" (saith he) "trouble them at first, and after a while break out
into madness: many things cause fury in women, especially if they love or hate
overmuch, or envy, be much grieved or angry; these things by little and little
lead them on to this malady." From a disposition they proceed to an habit, for
there is no difference between a mad man, and an angry man, in the time of his
fit; anger, as Lactantius describes it, \bookcite{\textlatin{L. de Ira Dei, ad
Donatum, c. 5}}, is \authorfootnote{1732}\li{saeva animi tempestas}, \etc{}, a
cruel tempest of the mind; "making his eye sparkle fire, and stare, teeth gnash
in his head, his tongue stutter, his face pale, or red, and what more filthy
imitation can be of a mad man?"

\begin{latin}
\begin{verse}%
Ora tument ira, fervescunt sanguine venae,\\*
Lumina Gorgonio saevius angue micant.\\!
\end{verse}%
\end{latin}
\attrib{\getauthornote{1733}}

They are void of reason, inexorable, blind, like beasts and monsters for the
time, say and do they know not what, curse, swear, rail, fight, and what not?
How can a mad man do more? as he said in the comedy,
\authorfootnote{1734}\li{Iracundia non sum apud me}, I am not mine own man. If
these fits be immoderate, continue long, or be frequent, without doubt they
provoke madness. Montanus, \bookcite{\textlatin{consil. 21}}, had a melancholy
Jew to his patient, he ascribes this for a principal cause: \li{Irascebatur
levibus de causis}, he was easily moved to anger. Ajax had no other beginning
of his madness; and Charles the Sixth, that lunatic French king, fell into this
misery, out of the extremity of his passion, desire of revenge and malice,
\authorfootnote{1735}incensed against the duke of Britain, he could neither
eat, drink, nor sleep for some days together, and in the end, about the calends
of July, 1392, he became mad upon his horseback, drawing his sword, striking
such as came near him promiscuously, and so continued all the days of his life,
Aemil., \bookcite{\textlatin{lib. 10.}} Gal. \bookcite{\textlatin{hist.}}
Aegesippus \bookcite{\textlatin{de exid. urbis Hieros, l. 1. c. 37}}, hath such
a story of Herod, that out of an angry fit, became mad,
\authorfootnote{1736}leaping out of his bed, he killed Jossippus, and played
many such bedlam pranks, the whole court could not rule him for a long time
after: sometimes he was sorry and repented, much grieved for that he had done,
\li{Postquam deferbuit ira}, by and by outrageous again. In hot choleric
bodies, nothing so soon causeth madness, as this passion of anger, besides many
other diseases, as Pelesius observes, \bookcite{\textlatin{cap. 21. l. 1. de
hum. affect. causis}}; \li{Sanguinem imminuit, fel auget}: and as
\authorfootnote{1737}Valesius controverts, \bookcite{\textlatin{Med. controv.,
lib. 5. contro. 8}}, many times kills them quite out. If this were the worst of
this passion, it were more tolerable, \authorfootnote{1738}"but it ruins and
subverts whole towns, \authorfootnote{1739}cities, families, and kingdoms;"
\li{Nulla pestis humano generi pluris stetit}, saith \Seneca{},
\bookcite{\textlatin{de Ira, lib. 1.}} No plague hath done mankind so much
harm. Look into our histories, and you shall almost meet with no other subject,
but what a company \authorfootnote{1740}of harebrains have done in their rage.
We may do well therefore to put this in our procession amongst the rest; "From
all blindness of heart, from pride, vainglory, and hypocrisy, from envy, hatred
and malice, anger, and all such pestiferous perturbations, good Lord deliver
us."

%SECT. II. MEMB. III. SUBSECT. X.-_Discontents, Cares, Miseries, \&c. Causes_.
\section{Discontents, Cares, Miseries, \&c. Causes.}

\lettrine{D}{iscontents}, cares, crosses, miseries, or whatsoever it is, that
shall cause any molestation of spirits, grief, anguish, and perplexity, may
well be reduced to this head, (preposterously placed here in some men's
judgments they may seem,) yet in that \Aristotle{} in his
\authorfootnote{1741}Rhetoric defines these cares, as he doth envy, emulation,
\etc{} still by grief, I think I may well rank them in this irascible row;
being that they are as the rest, both causes and symptoms of this disease,
producing the like inconveniences, and are most part accompanied with anguish
and pain. The common etymology will evince it, \li{Cura quasi cor uro, Dementes
curae, insomnes curae, damnosae curae, tristes, mordaces, carnifices}, \etc{}
biting, eating, gnawing, cruel, bitter, sick, sad, unquiet, pale, tetric,
miserable, intolerable cares, as the poets \authorfootnote{1742}call them,
worldly cares, and are as many in number as the sea sands.
\authorfootnote{1743}Galen, Fernelius, Felix Plater, Valescus de Taranta,
\etc{}, reckon afflictions, miseries, even all these contentions, and vexations
of the mind, as principal causes, in that they take away sleep, hinder
concoction, dry up the body, and consume the substance of it. They are not so
many in number, but their causes be as divers, and not one of a thousand free
from them, or that can vindicate himself, whom that \li{Ate dea},

\translatedverse{%
\begin{latin}
\begin{verse}%
Per hominum capita molliter ambulans,\\*
Plantas pedum teneras habens:\\!
\end{verse}%
\end{latin}}{%
\begin{verse}%
Over men's heads walking aloft,\\*
With tender feet treading so soft,\\!
\end{verse}}{%
\attrib{\getauthornote{1744}}}

\Homer{}'s Goddess Ate hath not involved into this discontented
\authorfootnote{1745}rank, or plagued with some misery or other. Hyginus,
\bookcite{\textlatin{fab. 220}}, to this purpose hath a pleasant tale. Dame
Cura by chance went over a brook, and taking up some of the dirty slime, made
an image of it; Jupiter eftsoons coming by, put life to it, but Cura and
Jupiter could not agree what name to give him, or who should own him; the
matter was referred to Saturn as judge; he gave this arbitrement: his name
shall be \li{Homo ab humo, Cura eum possideat quamdiu vivat}, Care shall have
him whilst he lives, Jupiter his soul, and Tellus his body when he dies. But to
leave tales. A general cause, a continuate cause, an inseparable accident, to
all men, is discontent, care, misery; were there no other particular affliction
(which who is free from?) to molest a man in this life, the very cogitation of
that common misery were enough to macerate, and make him weary of his life; to
think that he can never be secure, but still in danger, sorrow, grief, and
persecution. For to begin at the hour of his birth, as
\authorfootnote{1746}Pliny doth elegantly describe it, "he is born naked, and
falls \authorfootnote{1747}a whining at the very first: he is swaddled, and
bound up like a prisoner, cannot help himself, and so he continues to his
life's end." \li{Cujusque ferae pabulum}, saith \authorfootnote{1748}\Seneca{},
impatient of heat and cold, impatient of labour, impatient of idleness, exposed
to fortune's contumelies. To a naked mariner \Lucretius{} compares him, cast on
shore by shipwreck, cold and comfortless in an unknown land:
\authorfootnote{1749}no estate, age, sex, can secure himself from this common
misery. "A man that is born of a woman is of short continuance, and full of
trouble," \biblecite{Job \rn{xiv.} 1, 22}. "And while his flesh is upon him he
shall be sorrowful, and while his soul is in him it shall mourn. All his days
are sorrow and his travels griefs: his heart also taketh not rest in the
night." \biblecite{Eccles. \rn{ii.} 23, and \rn{ii.} 11}. "All that is in it is
sorrow and vexation of spirit. \authorfootnote{1750}Ingress, progress, regress,
egress, much alike: blindness seizeth on us in the beginning, labour in the
middle, grief in the end, error in all. What day ariseth to us without some
grief, care, or anguish? Or what so secure and pleasing a morning have we seen,
that hath not been overcast before the evening?" One is miserable, another
ridiculous, a third odious. One complains of this grievance, another of that.
\li{Aliquando nervi, aliquando pedes vexant}, (\Seneca{}) \li{nunc distillatio,
nunc epatis morbus; nunc deest, nunc superest sanguis}: now the head aches,
then the feet, now the lungs, then the liver, \etc{} \li{Huic sensus exuberat,
sed est pudori degener sanguis}, \etc{} He is rich, but base born; he is noble,
but poor; a third hath means, but he wants health peradventure, or wit to
manage his estate; children vex one, wife a second, \etc{} \li{Nemo facile cum
conditione sua concordat}, no man is pleased with his fortune, a pound of
sorrow is familiarly mixed with a dram of content, little or no joy, little
comfort, but \authorfootnote{1751}everywhere danger, contention, anxiety, in
all places: go where thou wilt, and thou shalt find discontents, cares, woes,
complaints, sickness, diseases, encumbrances, exclamations: "If thou look into
the market, there" (saith \authorfootnote{1752}\Chrysostom{}) "is brawling and
contention; if to the court, there knavery and flattery, \etc{}; if to a
private man's house, there's cark and care, heaviness," \etc{} As he said of
old,

\begin{latin}
\begin{verse}%
Nil homine in terra spirat miserum magis alma?\\!
\end{verse}%
\end{latin}
\attrib{\getauthornote{1753}}

No creature so miserable as man, so generally molested,
\authorfootnote{1754}"in miseries of body, in miseries of mind, miseries of
heart, in miseries asleep, in miseries awake, in miseries wheresoever he
turns," as Bernard found, \li{Nunquid tentatio est vita humana super terram}? A
mere temptation is our life, (\idxname{austin}[Austin][\textlatin{confess.}], \bookcite{\textlatin{confess. lib. 10. cap. 28}},) \li{catena
perpetuorum malorum, et quis potest molestias et difficultates pati}? Who can
endure the miseries of it? \authorfootnote{1755}"In prosperity we are insolent
and intolerable, dejected in adversity, in all fortunes foolish and miserable."
\authorfootnote{1756}"In adversity I wish for prosperity, and in prosperity I
am afraid of adversity. What mediocrity may be found? Where is no temptation?
What condition of life is free?" \authorfootnote{1757}"Wisdom hath labour
annexed to it, glory, envy; riches and cares, children and encumbrances,
pleasure and diseases, rest and beggary, go together: as if a man were
therefore born" (as the Platonists hold) "to be punished in this life for some
precedent sins." Or that, as \authorfootnote{1758}Pliny complains, "Nature may
be rather accounted a stepmother, than a mother unto us, all things considered:
no creature's life so brittle, so full of fear, so mad, so furious; only man is
plagued with envy, discontent, griefs, covetousness, ambition, superstition."
Our whole life is an Irish sea, wherein there is nought to be expected but
tempestuous storms and troublesome waves, and those infinite,

\translatedverse{%
\begin{latin}
\begin{verse}%
Tantum malorum pelagus aspicio,\\*
Ut non sit inde enatandi copia,\\!
\end{verse}%
\end{latin}}{%\setauthornote{1759.5}
\begin{verse}%
I perceive such an ocean of troubles before me,\\*
that no means of escape remain\\!
\end{verse}}{%
\attrib{\getauthornote{1759}}}
no halcyonian times, wherein a man can hold himself secure, or agree with his
present estate; but as Boethius infers, \authorfootnote{1760}"there is
something in every one of us which before trial we seek, and having tried
abhor: \authorfootnote{1761}we earnestly wish, and eagerly covet, and are
eftsoons weary of it." Thus between hope and fear, suspicions, angers,
\authorfootnote{1762}\li{Inter spemque metumque, timores inter et iras},
betwixt falling in, falling out, \etc{}, we bangle away our best days, befool
out our times, we lead a contentious, discontent, tumultuous, melancholy,
miserable life; insomuch, that if we could foretell what was to come, and it
put to our choice, we should rather refuse than accept of this painful life. In
a word, the world itself is a maze, a labyrinth of errors, a desert, a
wilderness, a den of thieves, cheaters, \etc{}, full of filthy puddles, horrid
rocks, precipitiums, an ocean of adversity, an heavy yoke, wherein infirmities
and calamities overtake, and follow one another, as the sea waves; and if we
scape Scylla, we fall foul on Charybdis, and so in perpetual fear, labour,
anguish, we run from one plague, one mischief, one burden to another, \li{duram
servientes servitutem}, and you may as soon separate weight from lead, heat
from fire, moistness from water, brightness from the sun, as misery,
discontent, care, calamity, danger, from a man. Our towns and cities are but so
many dwellings of human misery. "In which grief and sorrow"
(\authorfootnote{1763}as he right well observes out of Solon) "innumerable
troubles, labours of mortal men, and all manner of vices, are included, as in
so many pens." Our villages are like molehills, and men as so many emmets,
busy, busy still, going to and fro, in and out, and crossing one another's
projects, as the lines of several sea-cards cut each other in a globe or map.
"Now light and merry," but (\authorfootnote{1764}as one follows it) "by-and-by
sorrowful and heavy; now hoping, then distrusting; now patient, tomorrow crying
out; now pale, then red; running, sitting, sweating, trembling, halting,"
\etc{} Some few amongst the rest, or perhaps one of a thousand, may be Pullus
Jovis, in the world's esteem, \li{Gallinae filius albae}, an happy and
fortunate man, \li{ad invidiam felix}, because rich, fair, well allied, in
honour and office; yet peradventure ask himself, and he will say, that of all
others \authorfootnote{1765}he is most miserable and unhappy. A fair shoe,
\li{Hic soccus novus, elegans}, as he \authorfootnote{1766}said, \li{sed nescis
ubi urat}, but thou knowest not where it pincheth. It is not another man's
opinion can make me happy: but as \authorfootnote{1767}\Seneca{} well hath it, "He
is a miserable wretch that doth not account himself happy, though he be
sovereign lord of a world: he is not happy, if he think himself not to be so;
for what availeth it what thine estate is, or seem to others, if thou thyself
dislike it?" A common humour it is of all men to think well of other men's
fortunes, and dislike their own: \authorfootnote{1768}\li{Cui placet alterius,
sua nimirum est odio sors}; but \authorfootnote{1769}\li{qui fit Mecoenas},
\etc{}, how comes it to pass, what's the cause of it? Many men are of such a
perverse nature, they are well pleased with nothing, (saith
\authorfootnote{1770}Theodoret,) "neither with riches nor poverty, they
complain when they are well and when they are sick, grumble at all fortunes,
prosperity and adversity; they are troubled in a cheap year, in a barren,
plenty or not plenty, nothing pleaseth them, war nor peace, with children, nor
without." This for the most part is the humour of us all, to be discontent,
miserable, and most unhappy, as we think at least; and show me him that is not
so, or that ever was otherwise. Quintus Metellus his felicity is infinitely
admired amongst the Romans, insomuch that as \authorfootnote{1771}Paterculus
mentioneth of him, you can scarce find of any nation, order, age, sex, one for
happiness to be compared unto him: he had, in a word, \li{Bona animi, corporis
et fortunae}, goods of mind, body, and fortune, so had P. Mutianus,
\authorfootnote{1772}Crassus. Lampsaca, that Lacedaemonian lady, was such
another in \authorfootnote{1773}Pliny's conceit, a king's wife, a king's
mother, a king's daughter: and all the world esteemed as much of Polycrates of
Samos. The Greeks brag of their Socrates, Phocion, Aristides; the Psophidians
in particular of their Aglaus, \li{Omni vita felix, ab omni periculo immunis}
(which by the way Pausanias held impossible;) the Romans of their
\authorfootnote{1774}Cato, Curius, Fabricius, for their composed fortunes, and
retired estates, government of passions, and contempt of the world: yet none of
all these were happy, or free from discontent, neither Metellus, Crassus, nor
Polycrates, for he died a violent death, and so did Cato; and how much evil
doth Lactantius and Theodoret speak of Socrates, a weak man, and so of the
rest. There is no content in this life, but as \authorfootnote{1775}he said,
"All is vanity and vexation of spirit;" lame and imperfect. Hadst thou
Sampson's hair, Milo's strength, Scanderbeg's arm, Solomon's wisdom, Absalom's
beauty, Croesus' wealth, \li{Pasetis obulum}, Caesar's valour, Alexander's
spirit, \Tully{}'s or Demosthenes' eloquence, Gyges' ring, Perseus' Pegasus, and
Gorgon's head, Nestor's years to come, all this would not make thee absolute;
give thee content, and true happiness in this life, or so continue it. Even in
the midst of all our mirth, jollity, and laughter, is sorrow and grief, or if
there be true happiness amongst us, 'tis but for a time,

\translatedverse{%
\begin{latin}
\begin{verse}%
Desinat in piscem mulier formosa superne:\\!
\end{verse}%
\end{latin}}{%
\begin{verse}%
A handsome woman with a fish's tail,\\!
\end{verse}}{%
\attrib{\getauthornote{1776}}}

a fair morning turns to a lowering afternoon. Brutus and Cassius, once
renowned, both eminently happy, yet you shall scarce find two (saith
Paterculus) \li{quos fortuna maturius destiturit}, whom fortune sooner forsook.
Hannibal, a conqueror all his life, met with his match, and was subdued at
last, \li{Occurrit forti, qui mage fortis erit}. One is brought in triumph, as
Caesar into Rome, Alcibiades into Athens, \li{coronis aureis donatus}, crowned,
honoured, admired; by-and-by his statues demolished, he hissed out, massacred,
\etc{} \authorfootnote{1777}Magnus Gonsalva, that famous Spaniard, was of the
prince and people at first honoured, approved; forthwith confined and banished.
\li{Admirandas actiones; graves plerunque sequuntur invidiae, et acres
calumniae}: 'tis Polybius his observation, grievous enmities, and bitter
calumnies, commonly follow renowned actions. One is born rich, dies a beggar;
sound today, sick tomorrow; now in most flourishing estate, fortunate and
happy, by-and-by deprived of his goods by foreign enemies, robbed by thieves,
spoiled, captivated, impoverished, as they of \authorfootnote{1778}"Rabbah put
under iron saws, and under iron harrows, and under axes of iron, and cast into
the tile kiln,"

\begin{latin}
\begin{verse}%
Quid me felicem toties jactastis amici,\\*
Qui cecidit, stabili non erat ille gradu.\\!
\end{verse}%
\end{latin}
\attrib{\getauthornote{1779}}

He that erst marched like Xerxes with innumerable armies, as rich as Croesus,
now shifts for himself in a poor cock-boat, is bound in iron chains, with
Bajazet the Turk, and a footstool with Aurelian, for a tyrannising conqueror to
trample on. So many casualties there are, that as \Seneca{} said of a city
consumed with fire, \li{Una dies interest inter maximum civitatem et nullam},
one day betwixt a great city and none: so many grievances from outward
accidents, and from ourselves, our own indiscretion, inordinate appetite, one
day betwixt a man and no man. And which is worse, as if discontents and
miseries would not come fast enough upon us: \lit{homo homini daemon}{man is a
devil to each other}, we maul, persecute, and study how to sting, gall, and vex
one another with mutual hatred, abuses, injuries; preying upon and devouring as
so many, \authorfootnote{1780}ravenous birds; and as jugglers, panders, bawds,
cozening one another; or raging as \authorfootnote{1781}wolves, tigers, and
devils, we take a delight to torment one another; men are evil, wicked,
malicious, treacherous, and \authorfootnote{1782}naught, not loving one
another, or loving themselves, not hospitable, charitable, nor sociable as they
ought to be, but counterfeit, dissemblers, ambidexters, all for their own ends,
hard-hearted, merciless, pitiless, and to benefit themselves, they care not
what mischief they procure to others. \authorfootnote{1783}Praxinoe and Gorgo
in the poet, when they had got in to see those costly sights, they then cried
\li{bene est}, and would thrust out all the rest: when they are rich
themselves, in honour, preferred, full, and have even that they would, they
debar others of those pleasures which youth requires, and they formerly have
enjoyed. He sits at table in a soft chair at ease, but he doth remember in the
mean time that a tired waiter stands behind him, "an hungry fellow ministers to
him full, he is athirst that gives him drink" (saith
\authorfootnote{1784}Epictetus) "and is silent whilst he speaks his pleasure:
pensive, sad, when he laughs." \li{Pleno se proluit auro}: he feasts, revels,
and profusely spends, hath variety of robes, sweet music, ease, and all the
pleasure the world can afford, whilst many an hunger-starved poor creature
pines in the street, wants clothes to cover him, labours hard all day long,
runs, rides for a trifle, fights peradventure from sun to sun, sick and ill,
weary, full of pain and grief, is in great distress and sorrow of heart. He
loathes and scorns his inferior, hates or emulates his equal, envies his
superior, insults over all such as are under him, as if he were of another
species, a demigod, not subject to any fall, or human infirmities. Generally
they love not, are not beloved again: they tire out others' bodies with
continual labour, they themselves living at ease, caring for none else,
\li{sibi nati}; and are so far many times from putting to their helping hand,
that they seek all means to depress, even most worthy and well deserving,
better than themselves, those whom they are by the laws of nature bound to
relieve and help, as much as in them lies, they will let them \worddef{shrill
howling or wailing noise like that of a cat}{caterwaul}, starve, beg, and hang,
before they will any ways (though it be in their power) assist or ease:
\authorfootnote{1785}so unnatural are they for the most part, so unregardful;
so hard-hearted, so churlish, proud, insolent, so dogged, of so bad a
disposition. And being so brutish, so devilishly bent one towards another, how
is it possible but that we should be discontent of all sides, full of cares,
woes, and miseries?

If this be not a sufficient proof of their discontent and misery, examine every
condition and calling apart. Kings, princes, monarchs, and magistrates seem to
be most happy, but look into their estate, you shall \authorfootnote{1786}find
them to be most encumbered with cares, in perpetual fear, agony, suspicion,
jealousy: that, as \authorfootnote{1787}he said of a crown, if they knew but
the discontents that accompany it, they would not stoop to take it up. \li{Quem
mihi regent dabis} (saith \Chrysostom{}) \li{non curis plenum}? What king canst
thou show me, not full of cares? \authorfootnote{1788}"Look not on his crown,
but consider his afflictions; attend not his number of servants, but multitude
of crosses." \li{Nihil aliud potestas culminis, quam tempestas mentis}, as
Gregory seconds him; sovereignty is a tempest of the soul: Sylla like they have
brave titles, but terrible fits: \li{splendorem titulo, cruciatum animo}: which
made \authorfootnote{1789}Demosthenes vow, \li{si vel ad tribunal, vel ad
interitum duceretur}: if to be a judge, or to be condemned, were put to his
choice, he would be condemned. Rich men are in the same predicament; what their
pains are, \li{stulti nesciunt, ipsi sentiunt}: they feel, fools perceive not,
as I shall prove elsewhere, and their wealth is brittle, like children's
rattles: they come and go, there is no certainty in them: those whom they
elevate, they do as suddenly depress, and leave in a vale of misery. The middle
sort of men are as so many asses to bear burdens; or if they be free, and live
at ease, they spend themselves, and consume their bodies and fortunes with
luxury and riot, contention, emulation, \etc{} The poor I reserve for another
\authorfootnote{1790}place and their discontents.

For particular professions, I hold as of the rest, there's no content or
security in any; on what course will you pitch, how resolve? to be a divine,
'tis contemptible in the world's esteem; to be a lawyer, 'tis to be a wrangler;
to be a physician, \authorfootnote{1791}\li{pudet lotii}, 'tis loathed; a
philosopher, a madman; an alchemist, a beggar; a poet, \li{esurit}, an hungry
jack; a musician, a player; a schoolmaster, a drudge; an husbandman, an emmet;
a merchant, his gains are uncertain; a mechanician, base; a chirurgeon,
fulsome; a tradesman, a \authorfootnote{1792}liar; a tailor, a thief; a
serving-man, a slave; a soldier, a butcher; a smith, or a metalman, the pot's
never from his nose; a courtier a parasite, as he could find no tree in the
wood to hang himself; I can show no state of life to give content. The like you
may say of all ages; children live in a perpetual slavery, still under that
tyrannical government of masters; young men, and of riper years, subject to
labour, and a thousand cares of the world, to treachery, falsehood, and
cozenage,

\translatedverse{%
\begin{latin}
\begin{verse}
------Incedit per ignes,\\*
Suppositos cineri doloso,\\!
\end{verse}
\end{latin}}{%
\begin{verse}
------you incautious tread\\*
On fires, with faithless ashes overhead,\\!
\end{verse}}{%
\attrib{\getauthornote{1793}}}

\authorfootnote{1794}old are full of aches in their bones, cramps and
convulsions, \li{silicernia}, dull of hearing, weak sighted, hoary, wrinkled,
harsh, so much altered as that they cannot know their own face in a glass, a
burthen to themselves and others, after 70 years, "all is sorrow" (as David
hath it), they do not live but linger. If they be sound, they fear diseases; if
sick, weary of their lives: \li{Non est vivere, sed valere vita}. One complains
of want, a second of servitude, \authorfootnote{1795}another of a secret or
incurable disease; of some deformity of body, of some loss, danger, death of
friends, shipwreck, persecution, imprisonment, disgrace, repulse,
\authorfootnote{1796}contumely, calumny, abuse, injury, contempt, ingratitude,
unkindness, scoffs, flouts, unfortunate marriage, single life, too many
children, no children, false servants, unhappy children, barrenness,
banishment, oppression, frustrate hopes and ill-success, \etc{}

\translatedverse{%
\begin{latin}
\begin{verse}
Talia de genere hoc adeo sunt multa, loquacem ut\\*
Delassare valent Fabium.------\\!
\end{verse}
\end{latin}}{%
\begin{verse}
But, every various instance to repeat,\\*
Would tire even Fabius of incessant prate.\\!
\end{verse}}{%
\attrib{\getauthornote{1797}}}

Talking Fabius will be tired before he can tell half of them; they are the
subject of whole volumes, and shall (some of them) be more opportunely dilated
elsewhere. In the meantime thus much I may say of them, that generally they
crucify the soul of man, \authorfootnote{1798}attenuate our bodies, dry them,
wither them, shrivel them up like old apples, make them as so many anatomies
(\authorfootnote{1799}\li{ossa atque pellis est totus, ita curis macet}) they
cause \li{tempus foedum et squalidum}, cumbersome days, \li{ingrataque
tempora}, slow, dull, and heavy times: make us howl, roar, and tear our hairs,
as sorrow did in \authorfootnote{1800}Cebes' table, and groan for the very
anguish of our souls. Our hearts fail us as David's did, \biblecite{Psal.
\rn{xl.} 12}, "for innumerable troubles that compassed him;" and we are ready
to confess with Hezekiah, \biblecite{Isaiah \rn{lviii.} 17}, "behold, for
felicity I had bitter grief;" to weep with Heraclitus, to curse the day of our
birth with Jeremy, \biblecite{xx. 14}, and our stars with Job: to hold that axiom
of Silenus, \authorfootnote{1801}"better never to have been born, and the best
next of all, to die quickly:" or if we must live, to abandon the world, as
Timon did; creep into caves and holes, as our anchorites; cast all into the
sea, as Crates Thebanus; or as Theombrotus Ambrociato's 400 auditors,
precipitate ourselves to be rid of these miseries.

%SECT. II. MEMB. III. SUBSECT. XI.-_Concupiscible Appetite, as Desires, Ambition, Causes_.
\section{Concupiscible Appetite, as Desires, Ambition, Causes.}

\lettrine{T}{hese} \worddef{worthy of being desired}{concupiscible} and
\worddef{irritable}{irascible} appetites are as the two twists of a rope,
mutually mixed one with the other, and both twining about the heart: both good,
as \idxname{austin}[Austin][\textlatin{de civ. Dei}], holds, \bookcite{\textlatin{l. 14. c. 9. de civ. Dei}},
\authorfootnote{1802}"if they be moderate; both pernicious if they be
exorbitant." This concupiscible appetite, howsoever it may seem to carry with
it a show of pleasure and delight, and our concupiscences most part affect us
with content and a pleasing object, yet if they be in extremes, they rack and
wring us on the other side. A true saying it is, "Desire hath no rest;" is
infinite in itself, endless; and as \authorfootnote{1803}one calls it, a
perpetual rack, \authorfootnote{1804}or horse-mill, according to \Austin{}, still
going round as in a ring. They are not so continual, as divers, \li{felicius
atomos denumerare possem}, saith \authorfootnote{1805}Bernard, \li{quam motus
cordis; nunc haec, nunc illa cogito}, you may as well reckon up the motes in
the sun as them. \authorfootnote{1806}"It extends itself to everything," as
Guianerius will have it, "that is superfluously sought after:"' or to any
\authorfootnote{1807}fervent desire, as Fernelius interprets it; be it in what
kind soever, it tortures if immoderate, and is (according to
\authorfootnote{1808}Plater and others) an especial cause of melancholy.
\li{Multuosis concupiscentiis dilaniantur cogitationes meae},
\authorfootnote{1809}\Austin{} confessed, that he was torn a pieces with his
manifold desires: and so doth \authorfootnote{1810}Bernard complain, "that he
could not rest for them a minute of an hour: this I would have, and that, and
then I desire to be such and such." 'Tis a hard matter therefore to confine
them, being they are so various and many, impossible to apprehend all. I will
only insist upon some few of the chief, and most noxious in their kind, as that
exorbitant appetite and desire of honour, which we commonly call ambition; love
of money, which is covetousness, and that greedy desire of gain: self-love,
pride, and inordinate desire of vainglory or applause, love of study in excess;
love of women (which will require a just volume of itself), of the other I will
briefly speak, and in their order.

Ambition, a proud covetousness, or a dry thirst of honour, a great torture of
the mind, composed of envy, pride, and covetousness, a gallant madness, one
\authorfootnote{1811}defines it a pleasant poison, Ambrose, "a canker of the
soul, an hidden plague:" \authorfootnote{1812}Bernard, "a secret poison, the
father of livor, and mother of hypocrisy, the moth of holiness, and cause of
madness, crucifying and disquieting all that it takes hold of."
\authorfootnote{1813}\Seneca{} calls it, \li{rem solicitam, timidam, vanam,
ventosam}, a windy thing, a vain, solicitous, and fearful thing. For commonly
they that, like Sisyphus, roll this restless stone of ambition, are in a
perpetual agony, still \authorfootnote{1814}perplexed, \li{semper taciti,
tritesque recedunt} (\Lucretius{}), doubtful, timorous, suspicious, loath to
offend in word or deed, still cogging and colloguing, embracing, capping,
cringing, applauding, flattering, fleering, visiting, waiting at men's doors,
with all affability, counterfeit honesty and humility. \authorfootnote{1815}If
that will not serve, if once this humour (as \authorfootnote{1816}Cyprian
describes it) possess his thirsty soul, \li{ambitionis salsugo ubi bibulam
animam possidet}, by hook and by crook he will obtain it, "and from his hole he
will climb to all honours and offices, if it be possible for him to get up,
flattering one, bribing another, he will leave no means unessay'd to win all."
\authorfootnote{1817}It is a wonder to see how slavishly these kind of men
subject themselves, when they are about a suit, to every inferior person; what
pains they will take, run, ride, cast, plot, countermine, protest and swear,
vow, promise, what labours undergo, early up, down late; how obsequious and
affable they are, how popular and courteous, how they grin and fleer upon every
man they meet; with what feasting and inviting, how they spend themselves and
their fortunes, in seeking that many times, which they had much better be
without; as \authorfootnote{1818}Cyneas the orator told Pyrrhus: with what
waking nights, painful hours, anxious thoughts, and bitterness of mind,
\li{inter spemque metumque}, distracted and tired, they consume the interim of
their time. There can be no greater plague for the present. If they do obtain
their suit, which with such cost and solicitude they have sought, they are not
so freed, their anxiety is anew to begin, for they are never satisfied,
\li{nihil aliud nisi imperium spirant}, their thoughts, actions, endeavours are
all for sovereignty and honour, like \authorfootnote{1819}Lues Sforza that
huffing Duke of Milan, "a man of singular wisdom, but profound ambition, born
to his own, and to the destruction of Italy," though it be to their own ruin,
and friends' undoing, they will contend, they may not cease, but as a dog in a
wheel, a bird in a cage, or a squirrel in a chain, so
\authorfootnote{1820}Budaeus compares them; \authorfootnote{1821}they climb and
climb still, with much labour, but never make an end, never at the top. A
knight would be a baronet, and then a lord, and then a viscount, and then an
earl, \etc{}; a doctor, a dean, and then a bishop; from tribune to praetor;
from bailiff to major; first this office, and then that; as Pyrrhus in
\authorfootnote{1822}\Plutarch{}, they will first have Greece, then Africa, and
then Asia, and swell with Aesop's frog so long, till in the end they burst, or
come down with Sejanus, \li{ad Gemonias scalas}, and break their own necks; or
as Evangelus the piper in Lucian, that blew his pipe so long, till he fell down
dead. If he chance to miss, and have a canvass, he is in a hell on the other
side; so dejected, that he is ready to hang himself, turn heretic, Turk, or
traitor in an instant. Enraged against his enemies, he rails, swears, fights,
slanders, detracts, envies, murders: and for his own part, \li{si appetitum
explere non potest, furore corripitur}; if he cannot satisfy his desire (as
\authorfootnote{1823}Bodine writes) he runs mad. So that both ways, hit or
miss, he is distracted so long as his ambition lasts, he can look for no other
but anxiety and care, discontent and grief in the meantime,
\authorfootnote{1824}madness itself, or violent death in the end. The event of
this is common to be seen in populous cities, or in princes' courts, for a
courtier's life (as Budaeus describes it) "is a \authorfootnote{1825}\worddef{a
hodgepodge; jumble; confused medley}{gallimaufry} of ambition, lust, fraud,
imposture, dissimulation, detraction, envy, pride; \authorfootnote{1826}the
court, a common conventicle of flatterers, time-servers, politicians," \etc{};
or as \authorfootnote{1827}Anthony Perez will, "the suburbs of hell itself." If
you will see such discontented persons, there you shall likely find them.
\authorfootnote{1828}And which he observed of the markets of old Rome,

\begin{latin}
\begin{verse}%
Qui perjurum convenire vult hominem, mitto in Comitium;\\*
Qui mendacem et gloriosum, apud Cluasinae sacrum;\\*
Dites, damnosos maritos, sub basilica quaerito, \etc{}\\!
\end{verse}%
\end{latin}

Perjured knaves, knights of the post, liars, crackers, bad husbands, \etc{} keep their several stations; they do still, and always did in every commonwealth.

%SECT. II. MEMB. III. SUBSECT. XII.-_Φιλαργυρία, Covetousness, a Cause_.
\section{\textgreek{Φιλαργυρία}, Covetousness, a Cause.}

\lettrine{P}{lutarch}, in his \authorfootnote{1829}book whether the diseases of
the body be more grievous than those of the soul, is of opinion, "if you will
examine all the causes of our miseries in this life, you shall find them most
part to have had their beginning from stubborn anger, that furious desire of
contention, or some unjust or immoderate affection, as covetousness, \etc{}"
From whence "are wars and contentions amongst you?" \authorfootnote{1830}St.
James asks: I will add usury, fraud, rapine, simony, oppression, lying,
swearing, bearing false witness, \etc{} are they not from this fountain of
covetousness, that greediness in getting, tenacity in keeping, sordidity in
spending; that they are so wicked, \authorfootnote{1831}"unjust against God,
their neighbour, themselves;" all comes hence. "The desire of money is the root
of all evil, and they that lust after it, pierce themselves through with many
sorrows," \biblecite{1 Tim. \rn{vi.} 10}. Hippocrates therefore in his Epistle to
Crateva, an herbalist, gives him this good counsel, that if it were possible,
\authorfootnote{1832}"amongst other herbs, he should cut up that weed of
covetousness by the roots, that there be no remainder left, and then know this
for a certainty, that together with their bodies, thou mayst quickly cure all
the diseases of their minds." For it is indeed the pattern, image, epitome of
all melancholy, the fountain of many miseries, much discontented care and woe;
this "inordinate, or immoderate desire of gain, to get or keep money," as
\authorfootnote{1833}Bonaventure defines it: or, as \Austin{} describes it, a
madness of the soul, Gregory a torture; \Chrysostom{}, an insatiable drunkenness;
Cyprian, blindness, \li{speciosum supplicium}, a plague subverting kingdoms,
families, an \authorfootnote{1834}incurable disease; Budaeus, an ill habit,
\authorfootnote{1835}"yielding to no remedies:" neither Aesculapius nor Plutus
can cure them: a continual plague, saith Solomon, and vexation of spirit,
another hell. I know there be some of opinion, that covetous men are happy, and
worldly, wise, that there is more pleasure in getting of wealth than in
spending, and no delight in the world like unto it. 'Twas
\authorfootnote{1836}Bias' problem of old, "With what art thou not weary? with
getting money. What is most delectable? to gain." What is it, trow you, that
makes a poor man labour all his lifetime, carry such great burdens, fare so
hardly, macerate himself, and endure so much misery, undergo such base offices
with so great patience, to rise up early, and lie down late, if there were not
an extraordinary delight in getting and keeping of money? What makes a merchant
that hath no need, \li{satis superque domi}, to range all over the world,
through all those intemperate \authorfootnote{1837}Zones of heat and cold;
voluntarily to venture his life, and be content with such miserable famine,
nasty usage, in a stinking ship; if there were not a pleasure and hope to get
money, which doth season the rest, and mitigate his indefatigable pains? What
makes them go into the bowels of the earth, an hundred fathom deep, endangering
their dearest lives, enduring damps and filthy smells, when they have enough
already, if they could be content, and no such cause to labour, but an
extraordinary delight they take in riches. This may seem plausible at first
show, a popular and strong argument; but let him that so thinks, consider
better of it, and he shall soon perceive, that it is far otherwise than he
supposeth; it may be haply pleasing at the first, as most part all melancholy
is. For such men likely have some \li{lucida intervalla}, pleasant symptoms
intermixed; but you must note that of \authorfootnote{1838}\Chrysostom{}, "'Tis
one thing to be rich, another to be covetous:" generally they are all fools,
dizzards, madmen, \authorfootnote{1839}miserable wretches, living besides
themselves, \li{sine arte fruendi}, in perpetual slavery, fear, suspicion,
sorrow, and discontent, \li{plus aloes quam mellis habent}; and are indeed,
"rather possessed by their money, than possessors:" as
\authorfootnote{1840}Cyprian hath it, \li{mancipati pecuniis}; bound prentice
to their goods, as \authorfootnote{1841}Pliny; or as \Chrysostom{}, \li{servi
divitiarum}, slaves and drudges to their substance; and we may conclude of them
all, as \authorfootnote{1842}Valerius doth of Ptolomaeus king of Cyprus, "He
was in title a king of that island, but in his mind, a miserable drudge of
money:"

\begin{latin}
\begin{verse}%
------potiore metallis\\*
libertate carens------\\!
\end{verse}%
\end{latin}
\attrib{\getauthornote{1843}}
wanting his liberty, which is better than gold. Damasippus the Stoic, in
\Horace{}, proves that all mortal men dote by fits, some one way, some another,
but that covetous men \authorfootnote{1844}are madder than the rest; and he
that shall truly look into their estates, and examine their symptoms, shall
find no better of them, but that they are all \authorfootnote{1845}fools, as
Nabal was, \li{Re et nomine} (\biblecite{1. Reg. 15.}) For what greater folly can
there be, or \authorfootnote{1846}madness, than to macerate himself when he
need not? and when, as Cyprian notes, \authorfootnote{1847}"he may be freed
from his burden, and eased of his pains, will go on still, his wealth
increasing, when he hath enough, to get more, to live besides himself," to
starve his genius, keep back from his wife \authorfootnote{1848}and children,
neither letting them nor other friends use or enjoy that which is theirs by
right, and which they much need perhaps; like a hog, or dog in the manger, he
doth only keep it, because it shall do nobody else good, hurting himself and
others: and for a little momentary pelf, damn his own soul? They are commonly
sad and tetric by nature, as Achab's spirit was because he could not get
Naboth's vineyard, (\biblecite{1. Reg. 22.}) and if he lay out his money at any
time, though it be to necessary uses, to his own children's good, he brawls and
scolds, his heart is heavy, much disquieted he is, and loath to part from it:
\li{Miser abstinet et timet uti}, Hor. He is of a wearish, dry, pale
constitution, and cannot sleep for cares and worldly business; his riches,
saith Solomon, will not let him sleep, and unnecessary business which he
heapeth on himself; or if he do sleep, 'tis a very unquiet, interrupt,
unpleasing sleep: with his bags in his arms,

\begin{latin}
\begin{verse}%
------congestis undique sacc\\*
indormit inhians,------\\!
\end{verse}%
\end{latin}

And though he be at a banquet, or at some merry feast, "he sighs for grief of
heart" (as \authorfootnote{1849}Cyprian hath it) "and cannot sleep though it be
upon a down bed; his wearish body takes no rest,"
\authorfootnote{1850}"troubled in his abundance, and sorrowful in plenty,
unhappy for the present, and more unhappy in the life to come." Basil. He is a
perpetual drudge, \authorfootnote{1851}restless in his thoughts, and never
satisfied, a slave, a wretch, a dust-worm, \li{semper quod idolo suo immolet,
sedulus observat} Cypr. \bookcite{\textlatin{prolog. ad sermon}} still seeking
what sacrifice he may offer to his golden god, \li{per fas et nefas}, he cares
not how, his trouble is endless, \authorfootnote{1852}\li{crescunt divitiae,
tamen curtae nescio quid semper abest rei}: his wealth increaseth, and the more
he hath, the more \authorfootnote{1853}he wants: like Pharaoh's lean kine,
which devoured the fat, and were not satisfied. \authorfootnote{1854}\Austin{}
therefore defines covetousness, \li{quarumlibet rerum inhonestam et
insatiabilem cupiditatem} a dishonest and insatiable desire of gain; and in one
of his epistles compares it to hell; \authorfootnote{1855}"which devours all,
and yet never hath enough, a bottomless pit," an endless misery; \li{in quem
scopulum avaritiae cadaverosi senes utplurimum impingunt}, and that which is
their greatest corrosive, they are in continual suspicion, fear, and distrust,
He thinks his own wife and children are so many thieves, and go about to cozen
him, his servants are all false:

\translatedverse{%
\begin{latin}
\begin{verse}
Rem suam periisse, seque eradicarier,\\*
Et divum atque hominum clamat continuo fidem,\\*
De suo tigillo si qua exit foras.\\!
\end{verse}
\end{latin}}{%
\begin{verse}
If his doors creek, then out he cries anon,\\*
His goods are gone, and he is quite undone.\\!
\end{verse}}{}

\lit{Timidus Plutus}{As fearful as Plutus}, an old proverb: so doth
Aristophanes and Lucian bring him in fearful still, pale, anxious, suspicious,
and trusting no man, \authorfootnote{1856}"They are afraid of tempests for
their corn; they are afraid of their friends lest they should ask something of
them, beg or borrow; they are afraid of their enemies lest they hurt them,
thieves lest they rob them; they are afraid of war and afraid of peace, afraid
of rich and afraid of poor; afraid of all." Last of all, they are afraid of
want, that they shall die beggars, which makes them lay up still, and dare not
use that they have: what if a dear year come, or dearth, or some loss? and were
it not that they are both to \authorfootnote{1857}lay out money on a rope, they
would be hanged forthwith, and sometimes die to save charges, and make away
themselves, if their corn and cattle miscarry; though they have abundance left,
as \authorfootnote{1858}Agellius notes. \authorfootnote{1859}Valerius makes
mention of one that in a famine sold a mouse for 200 pence, and famished
himself: such are their cares, \authorfootnote{1860}griefs and perpetual fears.
These symptoms are elegantly expressed by Theophrastus in his character of a
covetous man; \authorfootnote{1861}"lying in bed, he asked his wife whether she
shut the trunks and chests fast, the cap-case be sealed, and whether the hall
door be bolted; and though she say all is well, he riseth out of his bed in his
shirt, barefoot and barelegged, to see whether it be so, with a dark lantern
searching every corner, scarce sleeping a wink all night." Lucian in that
pleasant and witty dialogue called Gallus, brings in Mycillus the cobbler
disputing with his cock, sometimes Pythagoras; where after much speech pro and
con, to prove the happiness of a mean estate, and discontents of a rich man,
Pythagoras' cock in the end, to illustrate by examples that which he had said,
brings him to Gnyphon the usurer's house at midnight, and after that to
Encrates; whom, they found both awake, casting up their accounts, and telling
of their money, \authorfootnote{1862}lean, dry, pale and anxious, still
suspecting lest somebody should make a hole through the wall, and so get in; or
if a rat or mouse did but stir, starting upon a sudden, and running to the door
to see whether all were fast. Plautus, in his Aulularia, makes old Euclio
\authorfootnote{1863}commanding Staphyla his wife to shut the doors fast, and
the fire to be put out, lest anybody should make that an errand to come to his
house: when he washed his hands, \authorfootnote{1864}he was loath to fling
away the foul water, complaining that he was undone, because the smoke got out
of his roof. And as he went from home, seeing a crow scratch upon the
muck-hill, returned in all haste, taking it for \li{malum omen}, an ill sign,
his money was digged up; with many such. He that will but observe their
actions, shall find these and many such passages not feigned for sport, but
really performed, verified indeed by such covetous and miserable wretches, and
that it is,

\begin{latin}
\begin{verse}%
------manifesta phrenesis\\*
Ut locuples moriaris egenti vivere fato.\\!
\end{verse}%
\end{latin}
\attrib{\getauthornote{1865}}

A mere madness, to live like a wretch, and die rich.

%SECT. II. MEMB. III. SUBSECT. XIII.-_Love of Gaming, \etc{} and pleasures immoderate; Causes_.
\section[Love of Gaming]{Love of Gaming, \etc{} and pleasures immoderate; Causes.}

\lettrine{I}{t} is a wonder to see, how many poor, distressed, miserable
wretches, one shall meet almost in every path and street, begging for an alms,
that have been well descended, and sometimes in flourishing estate, now ragged,
tattered, and ready to be starved, lingering out a painful life, in discontent
and grief of body and mind, and all through immoderate lust, gaming, pleasure
and riot. 'Tis the common end of all sensual epicures and brutish prodigals,
that are stupefied and carried away headlong with their several pleasures and
lusts. Cebes in his table, St. Ambrose in his second book of Abel and Cain, and
amongst the rest Lucian in his tract \bookcite{\textlatin{de Mercede
conductis}}, hath excellent well deciphered such men's proceedings in his
picture of Opulentia, whom he feigns to dwell on the top of a high mount, much
sought after by many suitors; at their first coming they are generally
entertained by pleasure and dalliance, and have all the content that possibly
may be given, so long as their money lasts: but when their means fail, they are
contemptibly thrust out at a back door, headlong, and there left to shame,
reproach, despair. And he at first that had so many attendants, parasites, and
followers, young and lusty, richly arrayed, and all the dainty fare that might
be had, with all kind of welcome and good respect, is now upon a sudden
stripped of all, \authorfootnote{1866}pale, naked, old, diseased and forsaken,
cursing his stars, and ready to strangle himself; having no other company but
repentance, sorrow, grief, derision, beggary, and contempt, which are his daily
attendants to his life's end. As the \authorfootnote{1867}prodigal son had
exquisite music, merry company, dainty fare at first; but a sorrowful reckoning
in the end; so have all such vain delights and their followers.
\authorfootnote{1868}\li{Tristes voluptatum exitus, et quisquis voluptatum
suarum reminisci volet, intelliget}, as bitter as gall and wormwood is their
last; grief of mind, madness itself. The ordinary rocks upon which such men do
impinge and precipitate themselves, are cards, dice, hawks, and hounds,
\li{Insanum venandi studium}, one calls it, \li{insanae substructiones}: their
mad structures, disports, plays, \etc{}, when they are unseasonably used,
imprudently handled, and beyond their fortunes. Some men are consumed by mad
fantastical buildings, by making galleries, cloisters, terraces, walks,
orchards, gardens, pools, rillets, bowers, and such like places of pleasure;
\li{Inutiles domos}, \authorfootnote{1869}Xenophon calls them, which howsoever
they be delightsome things in themselves, and acceptable to all beholders, an
ornament, and benefiting some great men: yet unprofitable to others, and the
sole overthrow of their estates. Forestus in his observations hath an example
of such a one that became melancholy upon the like occasion, having consumed
his substance in an unprofitable building, which would afterward yield him no
advantage. Others, I say, are \authorfootnote{1870}overthrown by those mad
sports of hawking and hunting; honest recreations, and fit for some great men,
but not for every base inferior person; whilst they will maintain their
falconers, dogs, and hunting nags, their wealth, saith
\authorfootnote{1871}Salmutze, "runs away with hounds, and their fortunes fly
away with hawks." They persecute beasts so long, till in the end they
themselves degenerate into beasts, as \authorfootnote{1872}Agrippa taxeth them,
\authorfootnote{1873}Actaeon like, for as he was eaten to death by his own
dogs, so do they devour themselves and their patrimonies, in such idle and
unnecessary disports, neglecting in the mean time their more necessary
business, and to follow their vocations. Over-mad too sometimes are our great
men in delighting, and doting too much on it. \authorfootnote{1874}"When they
drive poor husbandmen from their tillage," as
\authorfootnote{1875}Sarisburiensis objects, \bookcite{\textlatin{Polycrat. l.
1. c. 4}}, "fling down country farms, and whole towns, to make parks, and
forests, starving men to feed beasts, and \authorfootnote{1876}punishing in the
mean time such a man that shall molest their game, more severely than him that
is otherwise a common hacker, or a notorious thief." But great men are some
ways to be excused, the meaner sort have no evasion why they should not be
counted mad. Poggius the Florentine tells a merry story to this purpose,
condemning the folly and impertinent business of such kind of persons. A
physician of Milan, saith he, that cured mad men, had a pit of water in his
house, in which he kept his patients, some up to the knees, some to the girdle,
some to the chin, \li{pro modo insaniae}, as they were more or less affected.
One of them by chance, that was well recovered, stood in the door, and seeing a
gallant ride by with a hawk on his fist, well mounted, with his spaniels after
him, would needs know to what use all this preparation served; he made answer
to kill certain fowls; the patient demanded again, what his fowl might be worth
which he killed in a year; he replied 5 or 10 crowns; and when he urged him
farther what his dogs, horse, and hawks stood him in, he told him 400 crowns;
with that the patient bad be gone, as he loved his life and welfare, for if our
master come and find thee here, he will put thee in the pit amongst mad men up
to the chin: taxing the madness and folly of such vain men that spend
themselves in those idle sports, neglecting their business and necessary
affairs. Leo Decimus, that hunting pope, is much discommended by
\authorfootnote{1877}Jovius in his life, for his immoderate desire of hawking
and hunting, in so much that (as he saith) he would sometimes live about Ostia
weeks and months together, leave suitors \authorfootnote{1878}unrespected,
bulls and pardons unsigned, to his own prejudice, and many private men's loss.
\authorfootnote{1879}"And if he had been by chance crossed in his sport, or his
game not so good, he was so impatient, that he would revile and miscall many
times men of great worth with most bitter taunts, look so sour, be so angry and
waspish, so grieved and molested, that it is incredible to relate it." But if
he had good sport, and been well pleased, on the other side, \li{incredibili
munificentia}, with unspeakable bounty and munificence he would reward all his
fellow hunters, and deny nothing to any suitor when he was in that mood. To say
truth, 'tis the common humour of all gamesters, as Galataeus observes, if they
win, no men living are so jovial and merry, but \authorfootnote{1880}if they
lose, though it be but a trifle, two or three games at tables, or a dealing at
cards for two pence a game, they are so choleric and testy that no man may
speak with them, and break many times into violent passions, oaths,
imprecations, and unbeseeming speeches, little differing from mad men for the
time. Generally of all gamesters and gaming, if it be excessive, thus much we
may conclude, that whether they win or lose for the present, their winnings are
not \li{Munera fortunae, sed insidiae} as that wise \Seneca{} determines, not
fortune's gifts, but baits, the common catastrophe is
\authorfootnote{1881}beggary, \authorfootnote{1882}\li{Ut pestis vitam, sic
adimit alea pecuniam}, as the plague takes away life, doth gaming goods, for
\authorfootnote{1883}\li{omnes nudi, inopes et egeni};

\begin{latin}
\begin{verse}%
Alea Scylla vorax, species certissima furti,\\*
Non contenta bonis animum quoque perfida mergit,\\*
Foeda, furax, infamis, iners, furiosa, ruina.\\!
\end{verse}%
\end{latin}
\attrib{\getauthornote{1884}}

For a little pleasure they take, and some small gains and gettings now and
then, their wives and children are ringed in the meantime, and they themselves
with loss of body and soul rue it in the end. I will say nothing of those
prodigious prodigals, \li{perdendae pecuniae, genitos}, as he
\authorfootnote{1885}taxed Anthony, \li{Qui patrimonium sine ulla fori calumnia
amittunt}, saith \authorfootnote{1886}Cyprian, and \authorfootnote{1887}mad
sybaritical spendthrifts, \li{Quique una comedunt patrimonia coena}; that eat
up all at a breakfast, at a supper, or amongst bawds, parasites, and players,
consume themselves in an instant, as if they had flung it into
\authorfootnote{1888}Tiber, with great wages, vain and idle expenses, \etc{},
not themselves only, but even all their friends, as a man desperately swimming
drowns him that comes to help him, by suretyship and borrowing they will
willingly undo all their associates and allies. \authorfootnote{1889}\li{Irati
pecuniis}, as he saith, angry with their money: \authorfootnote{1890}"what with
a wanton eye, a liquorish tongue, and a gamesome hand," when they have
indiscreetly impoverished themselves, mortgaged their wits, together with their
lands, and entombed their ancestors' fair possessions in their bowels, they may
lead the rest of their days in prison, as many times they do; they repent at
leisure; and when all is gone begin to be thrifty: but \li{Sera est in fundo
parsimonia}, 'tis then too late to look about; their \authorfootnote{1891}end
is misery, sorrow, shame, and discontent. And well they deserve to be infamous
and discontent. \authorfootnote{1892}\li{Catamidiari in Amphitheatro}, as by
Adrian the emperor's edict they were of old, \li{decoctores bonorum suorum}, so
he calls them, prodigal fools, to be publicly shamed, and hissed out of all
societies, rather than to be pitied or relieved. \authorfootnote{1893}The
Tuscans and Boetians brought their bankrupts into the marketplace in a bier
with an empty purse carried before them, all the boys following, where they sat
all day \li{circumstante plebe}, to be infamous and ridiculous. At
\authorfootnote{1894}Padua in Italy they have a stone called the stone of
turpitude, near the senate-house, where spendthrifts, and such as disclaim
non-payment of debts, do sit with their hinder parts bare, that by that note of
disgrace others may be terrified from all such vain expense, or borrowing more
than they can tell how to pay. The \authorfootnote{1895}civilians of old set
guardians over such brain-sick prodigals, as they did over madmen, to moderate
their expenses, that they should not so loosely consume their fortunes, to the
utter undoing of their families.

I may not here omit those two main plagues, and common dotages of human kind,
wine and women, which have infatuated and besotted myriads of people; they go
commonly together.

\translatedverse{%
\begin{latin}
\begin{verse}%
Qui vino indulget, quemque aloa decoquit, ille\\*
In venerem putret------\\!
\end{verse}%
\end{latin}}{%\authorlatintrans{1896.5}
\begin{verse}%
One indulges in wine, another the die consumes,\\*
a third is decomposed by venery\\!
\end{verse}}{%
\attrib{\getauthornote{1896}}}

To whom is sorrow, saith Solomon, \biblecite{Pro. \rn{xxiii.} 39}, to whom is
woe, but to such a one as loves drink? it causeth torture, (\li{vino tortus et
ira}) and bitterness of mind, \biblecite{Sirac. 31. 21.} \li{Vinum furoris},
Jeremy calls it, \bookcite{\textlatin{15. cap.}} wine of madness, as well he
may, for \li{insanire facit sanos}, it makes sound men sick and sad, and wise
men \authorfootnote{1897}mad, to say and do they know not what. \li{Accidit
hodie terribilis casus} (saith \authorfootnote{1898}S. \Austin{}) hear a miserable
accident; Cyrillus' son this day in his drink, \li{Matrem praegnantem nequiter
oppressit, sororem violare voluit, patrem occidit fere, et duas alias sorores
ad mortem vulneravit}, would have violated his sister, killed his father,
\etc{} A true saying it was of him, \li{Vino dari laetitiam et dolorem}, drink
causeth mirth, and drink causeth sorrow, drink causeth "poverty and want,"
(\biblecite{Prov. \rn{xxi.}}) shame and disgrace. \li{Multi ignobiles evasere ob
vini potum, et} (\Austin{}) \li{amissis honoribus profugi aberrarunt}: many men
have made shipwreck of their fortunes, and go like rogues and beggars, having
turned all their substance into \li{aurum potabile}, that otherwise might have
lived in good worship and happy estate, and for a few hours' pleasure, for
their Hilary term's but short, or \authorfootnote{1899}free madness, as \Seneca{}
calls it, purchase unto themselves eternal tediousness and trouble.

That other madness is on women, \li{Apostatare facit cor}, saith the wise man,
\authorfootnote{1900}\li{Atque homini cerebrum minuit}. Pleasant at first she
is, like Dioscorides Rhododaphne, that fair plant to the eye, but poison to the
taste, the rest as bitter as wormwood in the end (\biblecite{Prov. \rn{v.} 4.})
and sharp as a two-edged sword, (\biblecite{vii. 27.}) "Her house is the way to
hell, and goes down to the chambers of death." What more sorrowful can be said?
they are miserable in this life, mad, beasts, led like
\authorfootnote{1901}"oxen to the slaughter:" and that which is worse,
whoremasters and drunkards shall be judged, \li{amittunt gratiam}, saith
\Austin{}, \li{perdunt gloriam, incurrunt damnationem aeternam}. They lose grace
and glory;

\translatedverse{%
\begin{latin}
\begin{verse}%
------brevis illa voluptas\\*
Abrogat aeternum caeli decus------\\!
\end{verse}%
\end{latin}}{%\setauthornote{1902}
\begin{verse}%
That momentary pleasure blots out\\*
the eternal glory of a heavenly life.
\end{verse}}{%
\attrib{\getauthornote{1902}}}

they gain hell and eternal damnation.

%SECT. II. MEMB. III. SUBSECT. XIV.-_Philautia, or Self-love, Vainglory,
%Praise, Honour, Immoderate Applause, Pride, overmuch Joy, \etc{}, Causes_.
\section[\textgreek{Φιλαυτία}, or Self-love]{\textgreek{Φιλαυτία}, or
Self-love, Vainglory, Praise, Honour, Immoderate Applause, Pride, overmuch Joy,
\etc{}, Causes.}

\lettrine{S}{elf-love}, pride, and vainglory, \authorfootnote{1903}\li{caecus
amor sui}, which \Chrysostom{} calls one of the devil's three great nets;
\authorfootnote{1904}"Bernard, an arrow which pierceth the soul through, and
slays it; a sly, insensible enemy, not perceived," are main causes. Where
neither anger, lust, covetousness, fear, sorrow, \etc{}, nor any other
perturbation can lay hold; this will slyly and insensibly pervert us, \li{Quem
non gula vicit, Philautia, superavit}, (saith Cyprian) whom surfeiting could
not overtake, self-love hath overcome. \authorfootnote{1905}"He hath scorned
all money, bribes, gifts, upright otherwise and sincere, hath inserted himself
to no fond imagination, and sustained all those tyrannical concupiscences of
the body, hath lost all his honour, captivated by vainglory." \Chrysostom{},
\bookcite{\textlatin{sup. Io.}} \li{Tu sola animum mentemque peruris, gloria}.
A great assault and cause of our present malady, although we do most part
neglect, take no notice of it, yet this is a violent batterer of our souls,
causeth melancholy and dotage. This pleasing humour; this soft and whispering
popular air, \li{Amabilis insania}; this delectable frenzy, most irrefragable
passion, \li{Mentis gratissimus error}, this acceptable disease, which so
sweetly sets upon us, ravisheth our senses, lulls our souls asleep, puffs up
our hearts as so many bladders, and that without all feeling,
\authorfootnote{1906}insomuch as "those that are misaffected with it, never so
much as once perceive it, or think of any cure." We commonly love him best in
this \authorfootnote{1907}malady, that doth us most harm, and are very willing
to be hurt; \li{adulationibus nostris libentur facemus} (saith
\authorfootnote{1908}Jerome) we love him, we love him for it:
\authorfootnote{1909}\li{O Bonciari suave, suave fuit a te tali haec tribui};
'Twas sweet to hear it. And as \authorfootnote{1910}Pliny doth ingenuously
confess to his dear friend Augurinus, "all thy writings are most acceptable,
but those especially that speak of us." Again, a little after to Maximus,
\authorfootnote{1911}"I cannot express how pleasing it is to me to hear myself
commended." Though we smile to ourselves, at least ironically, when parasites
bedaub us with false encomiums, as many princes cannot choose but do, \li{Quum
tale quid nihil intra se repererint}, when they know they come as far short, as
a mouse to an elephant, of any such virtues; yet it doth us good. Though we
seem many times to be angry, \authorfootnote{1912}"and blush at our own
praises, yet our souls inwardly rejoice, it puffs us up;" 'tis \li{fallax
suavitas, blandus daemon}, "makes us swell beyond our bounds, and forget
ourselves." Her two daughters are lightness of mind, immoderate joy and pride,
not excluding those other concomitant vices, which \authorfootnote{1913}Iodocus
Lorichius reckons up; bragging, hypocrisy, peevishness, and curiosity.

Now the common cause of this mischief, ariseth from ourselves or others,
\authorfootnote{1914}we are active and passive. It proceeds inwardly from
ourselves, as we are active causes, from an overweening conceit we have of our
good parts, own worth, (which indeed is no worth) our bounty, favour, grace,
valour, strength, wealth, patience, meekness, hospitality, beauty, temperance,
gentry, knowledge, wit, science, art, learning, our
\authorfootnote{1915}excellent gifts and fortunes, for which, Narcissus-like,
we admire, flatter, and applaud ourselves, and think all the world esteems so
of us; and as deformed women easily believe those that tell them they be fair,
we are too credulous of our own good parts and praises, too well persuaded of
ourselves. We brag and venditate our \authorfootnote{1916}own works, and scorn
all others in respect of us; \li{Inflati scientia}, (saith Paul) our wisdom,
\authorfootnote{1917}our learning, all our geese are swans, and we as basely
esteem and vilify other men's, as we do over-highly prize and value our own. We
will not suffer them to be \li{in secundis}, no, not \li{in tertiis}; what,
\li{Mecum confertur Ulysses}? they are \li{Mures, Muscae, culices prae se},
nits and flies compared to his inexorable and supercilious, eminent and
arrogant worship: though indeed they be far before him. Only wise, only rich,
only fortunate, valorous, and fair, puffed up with this tympany of
self-conceit; \authorfootnote{1918}as that proud Pharisee, they are not (as
they suppose) "like other men," of a purer and more precious metal:
\authorfootnote{1919}\li{Soli rei gerendi sunt efficaces}, which that wise
Periander held of such: \authorfootnote{1920}\li{meditantur omne qui prius
negotium}, \etc{} \li{Novi quendam} (saith \authorfootnote{1921}Erasmus) I knew
one so arrogant that he thought himself inferior to no man living, like
\authorfootnote{1922}Callisthenes the philosopher, that neither held
Alexander's acts, or any other subject worthy of his pen, such was his
insolency; or Seleucus king of Syria, who thought none fit to contend with him
but the Romans. \authorfootnote{1923}\li{Eos solos dignos ratus quibuscum de
imperio certaret}. That which \Tully{} writ to Atticus long since, is still in
force. \authorfootnote{1924}"There was never yet true poet nor orator, that
thought any other better than himself." And such for the most part are your
princes, potentates, great philosophers, historiographers, authors of sects or
heresies, and all our great scholars, as \authorfootnote{1925}Hierom defines;
"a natural philosopher is a glorious creature, and a very slave of rumour,
fame, and popular opinion," and though they write \li{de contemptu gloriae},
yet as he observes, they will put their names to their books. \li{Vobis et
famae, me semper dedi}, saith Trebellius Pollio, I have wholly consecrated
myself to you and fame. "'Tis all my desire, night and day, 'tis all my study
to raise my name." Proud \authorfootnote{1926}Pliny seconds him; \li{Quamquam
O}! \etc{} and that vainglorious \authorfootnote{1927}orator is not ashamed to
confess in an Epistle of his to Marcus Lecceius, \li{Ardeo incredibili
cupididate}, \etc{} "I burn with an incredible desire to have my
\authorfootnote{1928}name registered in thy book." Out of this fountain proceed
all those cracks and brags,-- \authorfootnote{1929}\li{speramus carmina fingi
Posse linenda cedro, et leni servanda cupresso} -- \authorfootnote{1930}\li{Non
usitata nec tenui ferar penna.--nec in terra morabor longius. Nil parvum aut
humili modo, nil mortale loquor. Dicar qua violens obstrepit Ausidus.--Exegi
monumentum aere perennius. Iamque opus exegi, quod nec Jovis ira, nec ignis,
\etc{} cum venit ille dies, \etc{} parte tamen meliore mei super alta perennis
astra ferar, nomenque erit indelebile nostrum}. (This of \Ovid{} I have
paraphrased in English.)

\begin{verse}
And when I am dead and gone,\\*
My corpse laid under a stone\\*
My fame shall yet survive,\\*
And I shall be alive,\\*
In these my works for ever,\\*
My glory shall persever, \etc{}\\!
\end{verse}

And that of Ennius,

\translatedverse{%
\begin{latin}
\begin{verse}%
Nemo me lachrymis decoret, neque funera fletu\\*
Faxit, cur? volito docta per ora virum.\\!
\end{verse}
\end{latin}}{%
\settowidth{\versewidth}{Let none shed tears over me, or adorn my bier with sorrow}
\begin{verse}
Let none shed tears over me,\\
or adorn my bier with sorrow\\
because I am eternally in the mouths of men.
\end{verse}}{}

"Let none shed tears over me, or adorn my bier with sorrow--because I am
eternally in the mouths of men." With many such proud strains, and foolish
flashes too common with writers. Not so much as Democharis on the
\authorfootnote{1931}Topics, but he will be immortal. \li{Typotius de fama},
shall be famous, and well he deserves, because he writ of fame; and every
trivial poet must be renowned,-- \li{Plausuque petit clarescere vulgi}. "He
seeks the applause of the public." This puffing humour it is, that hath
produced so many great tomes, built such famous monuments, strong castles, and
Mausolean tombs, to have their acts eternised,-- \li{Digito monstrari, et
dicier hic est}; "to be pointed at with the finger, and to have it said 'there
he goes,'" to see their names inscribed, as Phryne on the walls of Thebes,
\li{Phryne fecit}; this causeth so many bloody battles,-- \li{Et noctes cogit
vigilare serenas}; "and induces us to watch during calm nights." Long journeys,
\li{Magnum iter intendo, sed dat mihi gloria vires}, "I contemplate a monstrous
journey, but the love of glory strengthens me for it," gaining honour, a little
applause, pride, self-love, vainglory. This is it which makes them take such
pains, and break out into those ridiculous strains, this high conceit of
themselves, to \authorfootnote{1932}scorn all others; \li{ridiculo fastu et
intolerando contemptu}; as \authorfootnote{1933}Palaemon the grammarian
contemned Varro, \li{secum et natas et morituras literas jactans}, and brings
them to that height of insolency, that they cannot endure to be contradicted,
\authorfootnote{1934}"or hear of anything but their own commendation," which
Hierom notes of such kind of men. And as \authorfootnote{1935}\Austin{} well
seconds him, "'tis their sole study day and night to be commended and
applauded." When as indeed, in all wise men's judgments, \li{quibus cor sapit},
they are \authorfootnote{1936}mad, empty vessels, funges, beside themselves,
derided, \li{et ut Camelus in proverbio quaerens cornua, etiam quas habebat
aures amisit}\authorlatintrans{1937}, their works are toys, as an almanac out of
date, \authorfootnote{1938}\li{authoris pereunt garrulitate sui}, they seek
fame and immortality, but reap dishonour and infamy, they are a common obloquy,
\li{insensati}, and come far short of that which they suppose or expect.
\authorfootnote{1939}\li{O puer ut sis vitalis metuo},

\begin{verse}%
------How much I dread\\*
Thy days are short, some lord shall strike thee dead.\\!
\end{verse}%

Of so many myriads of poets, rhetoricians, philosophers, sophisters, as
\authorfootnote{1940}Eusebius well observes, which have written in former ages,
scarce one of a thousand's works remains, \li{nomina et libri simul cum
corporibus interierunt}, their books and bodies are perished together. It is
not as they vainly think, they shall surely be admired and immortal, as one
told Philip of Macedon insultingly, after a victory, that his shadow was no
longer than before, we may say to them,

\translatedverse{%
\begin{latin}
\begin{verse}
Nos demiramur, sed non cum deside vulgo,\\*
Sed velut Harpyas, Gorgonas, et Furias.\\!
\end{verse}
\end{latin}}{%
\begin{verse}
We marvel too, not as the vulgar we,\\*
But as we Gorgons, Harpies, or Furies see.\\!
\end{verse}}{}

Or if we do applaud, honour and admire, \li{quota pars}, how small a part, in
respect of the whole world, never so much as hears our names, how few take
notice of us, how slender a tract, as scant as Alcibiades' land in a map! And
yet every man must and will be immortal, as he hopes, and extend his fame to
our antipodes, when as half, no not a quarter of his own province or city,
neither knows nor hears of him--but say they did, what's a city to a kingdom, a
kingdom to Europe, Europe to the world, the world itself that must have an end,
if compared to the least visible star in the firmament, eighteen times bigger
than it? and then if those stars be infinite, and every star there be a sun, as
some will, and as this sun of ours hath his planets about him, all inhabited,
what proportion bear we to them, and where's our glory? \li{Orbem terrarum
victor Romanus habebat}, as he cracked in \Petronius, all the world was under
Augustus: and so in Constantine's time, Eusebius brags he governed all the
world, \li{universum mundum praeclare admodum administravit,--et omnes orbis
gentes Imperatori subjecti}: so of Alexander it is given out, the four
monarchies, \etc{} when as neither Greeks nor Romans ever had the fifteenth
part of the now known world, nor half of that which was then described. What
braggadocios are they and we then? \li{quam brevis hic de nobis sermo}, as
\authorfootnote{1941}he said, \authorfootnote{1942}\li{pudebit aucti nominis},
how short a time, how little a while doth this fame of ours continue? Every
private province, every small territory and city, when we have all done, will
yield as generous spirits, as brave examples in all respects, as famous as
ourselves, Cadwallader in Wales, Rollo in Normandy, Robin Hood and Little John,
are as much renowned in Sherwood, as Caesar in Rome, Alexander in Greece, or
his Hephestion, \authorfootnote{1943}\li{Omnis aetas omnisque populus in
exemplum et admirationem veniet}, every town, city, book, is full of brave
soldiers, senators, scholars; and though \authorfootnote{1944}Bracyclas was a
worthy captain, a good man, and as they thought, not to be matched in
Lacedaemon, yet as his mother truly said, \li{plures habet Sparta Bracyda
meliores}, Sparta had many better men than ever he was; and howsoever thou
admirest thyself, thy friend, many an obscure fellow the world never took
notice of, had he been in place or action, would have done much better than he
or he, or thou thyself.

Another kind of mad men there is opposite to these, that are insensibly mad,
and know not of it, such as contemn all praise and glory, think themselves most
free, when as indeed they are most mad: \li{calcant sed alio fastu}: a company
of cynics, such as are monks, hermits, anchorites, that contemn the world,
contemn themselves, contemn all titles, honours, offices: and yet in that
contempt are more proud than any man living whatsoever. They are proud in
humility, proud in that they are not proud, \li{saepe homo de vanae gloriae
contemptu, vanius gloriatur}, as \idxname{austin}[Austin][\textlatin{confess.}] hath it, \bookcite{\textlatin{confess.
lib. 10, cap. 38}}, like Diogenes, \li{intus gloriantur}, they brag inwardly,
and feed themselves fat with a self-conceit of sanctity, which is no better
than hypocrisy. They go in sheep's russet, many great men that might maintain
themselves in cloth of gold, and seem to be dejected, humble by their outward
carriage, when as inwardly they are swollen full of pride, arrogancy, and
self-conceit. And therefore \Seneca{} adviseth his friend Lucilius,
\authorfootnote{1945}"in his attire and gesture, outward actions, especially to
avoid all such things as are more notable in themselves: as a rugged attire,
hirsute head, horrid beard, contempt of money, coarse lodging, and whatsoever
leads to fame that opposite way."

All this madness yet proceeds from ourselves, the main engine which batters us
is from others, we are merely passive in this business: from a company of
parasites and flatterers, that with immoderate praise, and bombast epithets,
glossing titles, false eulogiums, so \worddef{smeared, soiled}{bedaub} and
applaud, gild over many a silly and undeserving man, that they clap him quite
out of his wits. \li{Res imprimis violenta est}, as Hierom notes, this common
applause is a most violent thing, \li{laudum placenta}, a drum, fife, and
trumpet cannot so animate; that fattens men, erects and dejects them in an
instant. \authorfootnote{1946}\li{Palma negata macrum, donata reducit opimum}.
It makes them fat and lean, as frost doth conies. \authorfootnote{1947}"And who
is that mortal man that can so contain himself, that if he be immoderately
commended and applauded, will not be moved?" Let him be what he will, those
parasites will overturn him: if he be a king, he is one of the nine worthies,
more than a man, a god forthwith,-- \authorfootnote{1948}\li{edictum Domini
Deique nostri}: and they will sacrifice unto him,

\translatedverse{%
\begin{latin}
\begin{verse}
\authorfootnote{1949}------divinos si tu patiaris honores,\\*
Ultro ipsi dabimus meritasque sacrabimus aras.\\!
\end{verse}
\end{latin}}{% \authorlatintrans{1949.5}
\begin{verse}%
If you will accept divine honours,\\*
we will willingly erect and consecrate altars to you.
\end{verse}}{%
\attrib{\getauthornote{1949}}}

If he be a soldier, then Themistocles, Epaminondas, Hector, Achilles, \li{duo
fulmina belli, triumviri terrarum}, \etc{}, and the valour of both Scipios is
too little for him, he is \li{invictissimus, serenissimus, multis trophaeus
ornatissimus, naturae, dominus}, although he be \li{lepus galeatus}, indeed a
very coward, a milk-sop, \authorfootnote{1950}and as he said of Xerxes,
\li{postremus in pugna, primus in fuga}, and such a one as never durst look his
enemy in the face. If he be a big man, then is he a Samson, another Hercules;
if he pronounce a speech, another \Tully{} or Demosthenes; as of Herod in the
Acts, "the voice of God and not of man:" if he can make a verse, \Homer{}, \Virgil{},
\etc{}, And then my silly weak patient takes all these eulogiums to himself; if
he be a scholar so commended for his much reading, excellent style, method,
\etc{}, he will eviscerate himself like a spider, study to death, \li{Laudatas
ostendit avis Junonia pennas}, peacock-like he will display all his feathers.
If he be a soldier, and so applauded, his valour extolled, though it be
\li{impar congressus}, as that of Troilus and Achilles, \li{Infelix puer}, he
will combat with a giant, run first upon a breach, as another
\authorfootnote{1951}Philippus, he will ride into the thickest of his enemies.
Commend his housekeeping, and he will beggar himself; commend his temperance,
he will starve himself.

\translatedverse{%
\begin{latin}
\begin{verse}
------laudataque virtus\\*
Crescit, et immensum gloria calcar habet.\\!
\end{verse}
\end{latin}}{%\setauthornote{1952}
\begin{verse}%
Applauded virtue grows apace,\\*
and glory includes within it an immense impulse.\\!
\end{verse}}{}%
he is mad, mad, mad, no woe with him:-- \li{impatiens consortis erit}, he will
over the \authorfootnote{1953}Alps to be talked of, or to maintain his credit.
Commend an ambitious man, some proud prince or potentate, \li{si plus aequo
laudetur} (saith \authorfootnote{1954}Erasmus) \li{cristas erigit, exuit
hominem, Deum se putat}, he sets up his crest, and will be no longer a man but
a God.

\translatedverse{%
\begin{latin}
\begin{verse}
------nihil est quod credere de se\\*
Non audet quum laudatur diis aequa potestas.\\!
\end{verse}
\end{latin}}{%\authorlatintrans{1956}.
\begin{verse}%
There is nothing which overlauded power\\*
will not presume to imagine of itself
\end{verse}}{%
\attrib{\getauthornote{1955}}}

How did this work with Alexander, that would needs be Jupiter's son, and go
like Hercules in a lion's skin? Domitian a god,
\authorfootnote{1957}(\li{Dominus Deus noster sic fieri jubet},) like the
\authorfootnote{1958}Persian kings, whose image was adored by all that came
into the city of Babylon. Commodus the emperor was so gulled by his flattering
parasites, that he must be called Hercules. \authorfootnote{1959}Antonius the
Roman would be crowned with ivy, carried in a chariot, and adored for Bacchus.
Cotys, king of Thrace, was married to \authorfootnote{1960}Minerva, and sent
three several messengers one after another, to see if she were come to his
bedchamber. Such a one was \authorfootnote{1961}Jupiter Menecrates, Maximinus,
Jovianus, Dioclesianus Herculeus, Sapor the Persian king, brother of the sun
and moon, and our modern Turks, that will be gods on earth, kings of kings,
God's shadow, commanders of all that may be commanded, our kings of China and
Tartary in this present age. Such a one was Xerxes, that would whip the sea,
fetter Neptune, \li{stulta jactantia}, and send a challenge to Mount Athos; and
such are many sottish princes, brought into a fool's paradise by their
parasites, 'tis a common humour, incident to all men, when they are in great
places, or come to the solstice of honour, have done, or deserved well, to
applaud and flatter themselves. \li{Stultitiam suam produnt}, \etc{}, (saith
\authorfootnote{1962}Platerus) your very tradesmen if they be excellent, will
crack and brag, and show their folly in excess. They have good parts, and they
know it, you need not tell them of it; out of a conceit of their worth, they go
smiling to themselves, a perpetual meditation of their trophies and plaudits,
they run at last quite mad, and lose their wits. \authorfootnote{1963}Petrarch,
\bookcite{\textlatin{lib. 1 de contemptu mundi}}, confessed as much of himself,
and Cardan, in his fifth book of wisdom, gives an instance in a smith of Milan,
a fellow-citizen of his, \authorfootnote{1964}one Galeus de Rubeis, that being
commended for refining of an instrument of Archimedes, for joy ran mad.
\Plutarch{} in the life of Artaxerxes, hath such a like story of one Chamus, a
soldier, that wounded king Cyrus in battle, and "grew thereupon so
\authorfootnote{1965}arrogant, that in a short space after he lost his wits."
So many men, if any new honour, office, preferment, booty, treasure,
possession, or patrimony, \li{ex insperato} fall unto them for immoderate joy,
and continual meditation of it, cannot sleep \authorfootnote{1966}or tell what
they say or do, they are so ravished on a sudden; and with vain conceits
transported, there is no rule with them. Epaminondas, therefore, the next day
after his Leuctrian victory, \authorfootnote{1967}"came abroad all squalid and
submiss," and gave no other reason to his friends of so doing, than that he
perceived himself the day before, by reason of his good fortune, to be too
insolent, overmuch joyed. That wise and virtuous lady,
\authorfootnote{1968}Queen Katherine, Dowager of England, in private talk, upon
like occasion, said, that \authorfootnote{1969}"she would not willingly endure
the extremity of either fortune; but if it were so, that of necessity she must
undergo the one, she would be in adversity, because comfort was never wanting
in it, but still counsel and government were defective in the other:" they
could not moderate themselves.

\cleartoleftpage{}
\begin{figure}[p]
  \begingroup
  \centering
  \includegraphics[keepaspectratio,width=0.85\textwidth]{scholar-small.jpg}
  \captionart{Scholar}
  \label{fig:scholar}
\end{figure}

% Force float here
\clearpage{}
\thispagestyle{titleontop}

%SECT. II. MEMB. III. SUBSECT. XV.-_Love of Learning, or overmuch study. With a
%Digression of the misery of Scholars, and why the Muses are Melancholy_.
\section[Love of Learning, or overmuch study.]{Love of Learning, or overmuch
study. With a Digression of the misery of Scholars, and why the Muses are
Melancholy.}

\lettrine{L}{eonartus} Fuchsius \bookcite{\textlatin{Instit. lib. iii. sect. 1.
cap. 1.}} Felix Plater, \bookcite{\textlatin{lib. iii. de mentis alienat}}.
Herc. de Saxonia, \bookcite{\textlatin{Tract. post. de melanch. cap. 3}}, speak
of a \authorfootnote{1970}peculiar fury, which comes by overmuch study.
Fernelius, \bookcite{\textlatin{lib. 1, cap. 18}}, \authorfootnote{1971}puts
study, contemplation, and continual meditation, as an especial cause of
madness: and in his \bookcite{\textlatin{86 consul.}} cites the same words. Jo.
Arculanus, \bookcite{\textlatin{in lib. 9, Rhasis ad Alnansorem, cap. 16}},
amongst other causes reckons up \li{studium vehemens}: so doth Levinus Lemnius,
\bookcite{\textlatin{lib. de occul. nat. mirac. lib. 1, cap. 16.}}
\authorfootnote{1972}"Many men" (saith he) "come to this malady by continual
\authorfootnote{1973}study, and night-waking, and of all other men, scholars
are most subject to it:" and such Rhasis adds, \authorfootnote{1974}"that have
commonly the finest wits." \bookcite{\textlatin{Cont. lib. 1, tract. 9}},
Marsilius Ficinus, \bookcite{\textlatin{de sanit. tuenda, lib. 1. cap. 7}},
puts melancholy amongst one of those five principal plagues of students, 'tis a
common Maul unto them all, and almost in some measure an inseparable companion.
Varro belike for that cause calls \li{Tristes Philosophos et severos}, severe,
sad, dry, tetric, are common epithets to scholars: and
\authorfootnote{1975}Patritius therefore, in the institution of princes, would
not have them to be great students. For (as Machiavel holds) study weakens
their bodies, dulls the spirits, abates their strength and courage; and good
scholars are never good soldiers, which a certain Goth well perceived, for when
his countrymen came into Greece, and would have burned all their books, he
cried out against it, by no means they should do it,
\authorfootnote{1976}"leave them that plague, which in time will consume all
their vigour, and martial spirits." The \authorfootnote{1977}Turks abdicated
Cornutus the next heir from the empire, because he was so much given to his
book: and 'tis the common tenet of the world, that learning dulls and
diminisheth the spirits, and so \li{per consequens} produceth melancholy.

Two main reasons may be given of it, why students should be more subject to
this malady than others. The one is, they live a sedentary, solitary life,
\li{sibi et musis}, free from bodily exercise, and those ordinary disports
which other men use: and many times if discontent and idleness concur with it,
which is too frequent, they are precipitated into this gulf on a sudden: but
the common cause is overmuch study; too much learning (as
\authorfootnote{1978}Festus told Paul) hath made thee mad; 'tis that other
extreme which effects it. So did Trincavelius, \bookcite{\textlatin{lib. 1,
consil. 12 and 13}}, find by his experience, in two of his patients, a young
baron, and another that contracted this malady by too vehement study. So
Forestus, \bookcite{\textlatin{observat. l. 10, observ. 13}}, in a young divine
in Louvain, that was mad, and said \authorfootnote{1979}"he had a Bible in his
head:" Marsilius Ficinus \bookcite{\textlatin{de sanit. tuend. lib. 1, cap. 1,
3, 4}}, and \bookcite{\textlatin{lib. 2, cap. 16}}, gives many reasons,
\authorfootnote{1980}"why students dote more often than others." The first is
their negligence; \authorfootnote{1981}"other men look to their tools, a
painter will wash his pencils, a smith will look to his hammer, anvil, forge; a
husbandman will mend his plough-irons, and grind his hatchet if it be dull; a
falconer or huntsman will have an especial care of his hawks, hounds, horses,
dogs, \etc{}; a musician will string and unstring his lute, \etc{}; only
scholars neglect that instrument, their brain and spirits (I mean) which they
daily use, and by which they range overall the world, which by much study is
consumed." \li{Vide} (saith Lucian) \li{ne funiculum nimis intendendo aliquando
abrumpas}: "See thou twist not the rope so hard, till at length it
\authorfootnote{1982}break." Facinus in his fourth chap. gives some other
reasons; Saturn and Mercury, the patrons of learning, they are both dry
planets: and Origanus assigns the same cause, why Mercurialists are so poor,
and most part beggars; for that their president Mercury had no better fortune
himself. The destinies of old put poverty upon him as a punishment; since when,
poetry and beggary are Gemelli, twin-born brats, inseparable companions;

\begin{verse}%
\end{verse}%
\begin{verse}%
And to this day is every scholar poor;\\*
Gross gold from them runs headlong to the boor:\\!
\end{verse}%
\attrib{\getauthornote{1983}}

Mercury can help them to knowledge, but not to money. The second is
contemplation, \authorfootnote{1984}"which dries the brain and extinguisheth
natural heat; for whilst the spirits are intent to meditation above in the
head, the stomach and liver are left destitute, and thence come black blood and
crudities by defect of concoction, and for want of exercise the superfluous
vapours cannot exhale," \etc{} The same reasons are repeated by Gomesius,
\bookcite{\textlatin{lib. 4, cap. 1, de sale}} \authorfootnote{1985}Nymannus
\bookcite{\textlatin{orat. de Imag.}} Jo. Voschius, \bookcite{\textlatin{lib.
2, cap. 5, de peste}}: and something more they add, that hard students are
commonly troubled with gouts, catarrhs, rheums, cachexia, bradiopepsia, bad
eyes, stone and colic, \authorfootnote{1986}crudities, oppilations, vertigo,
winds, consumptions, and all such diseases as come by overmuch sitting; they
are most part lean, dry, ill-coloured, spend their fortunes, lose their wits,
and many times their lives, and all through immoderate pains, and extraordinary
studies. If you will not believe the truth of this, look upon great Tostatus
and Thomas Aquinas's works, and tell me whether those men took pains? peruse
\Austin{}, Hierom, \etc{}, and many thousands besides.

\translatedverse{%
\begin{latin}
\begin{verse}
Qui cupit optatam cursu contingere metam,\\*
Multa tulit, fecitque puer, sudavit et alsit.\\!
\end{verse}
\end{latin}}{%
\begin{verse}%
He that desires this wished goal to gain,\\*
Must sweat and freeze before he can attain,\\!
\end{verse}}{}
and labour hard for it. So did \Seneca{}, by his own confession,
\bookcite{\textlatin{ep. 8.}} \authorfootnote{1987}"Not a day that I spend
idle, part of the night I keep mine eyes open, tired with waking, and now
slumbering to their continual task." Hear \Tully{} \li{pro Archia Poeta}: "whilst
others loitered, and took their pleasures, he was continually at his book," so
they do that will be scholars, and that to the hazard (I say) of their healths,
fortunes, wits, and lives. How much did \Aristotle{} and Ptolemy spend? \li{unius
regni precium} they say, more than a king's ransom; how many crowns per annum,
to perfect arts, the one about his History of Creatures, the other on his
Almagest? How much time did Thebet Benchorat employ, to find out the motion of
the eighth sphere? forty years and more, some write: how many poor scholars
have lost their wits, or become dizzards, neglecting all worldly affairs and
their own health, wealth, \li{esse} and \li{bene esse}, to gain knowledge for
which, after all their pains, in this world's esteem they are accounted
ridiculous and silly fools, idiots, asses, and (as oft they are) rejected,
contemned, derided, doting, and mad. Look for examples in Hildesheim
\bookcite{\textlatin{spicel. 2, de mania et delirio}}: read Trincavellius,
\bookcite{\textlatin{l. 3. consil. 36, et c. 17.}} Montanus,
\bookcite{\textlatin{consil. 233.}} \authorfootnote{1988}Garceus
\bookcite{\textlatin{de Judic. genit. cap. 33.}} Mercurialis,
\bookcite{\textlatin{consil. 86, cap. 25.}} Prosper
\authorfootnote{1989}Calenius in his Book \bookcite{\textlatin{de atra bile}};
Go to Bedlam and ask. Or if they keep their wits, yet they are esteemed scrubs
and fools by reason of their carriage: "after seven years' study"

\translatedverse{%
\begin{latin}
\begin{verse}%
------statua, taciturnius exit,\\*
Plerumque et risum populi quatit.------\\!
\end{verse}%
\end{latin}}{%
\begin{verse}%
He becomes more silent than a statue,\\*
and generally excites people's laughter.\\!
\end{verse}}{}

Because they cannot ride a horse, which every clown can do; salute and court a
gentlewoman, carve at table, cringe and make conges, which every common swasher
can do, \authorfootnote{1990}\li{hos populus ridet}, \etc{}, they are laughed
to scorn, and accounted silly fools by our gallants. Yea, many times, such is
their misery, they deserve it: \authorfootnote{1991}a mere scholar, a mere ass.

\translatedverse{%
\begin{latin}
\begin{verse}
Obstipo capite, et figentes lumine terram,\\*
Murmura cum secum, et rabiosa silentia rodunt,\\*
Atque experrecto trutinantur verba labello,\\*
Aegroti veteris meditantes somnia, gigni\\*
De nihilo nihilum; in nihilum nil posse reverti.\\!
\end{verse}
\end{latin}}{%
\begin{verse}%
\vin\vin who do lean awry\\*
------who do lean awry\\*
Their heads, piercing the earth with a fixt eye;\\*
When, by themselves, they gnaw their murmuring,\\*
And furious silence, as 'twere balancing\\*
Each word upon their out-stretched lip, and when\\*
They meditate the dreams of old sick men,\\*
As, 'Out of nothing, nothing can be brought;\\*
And that which is, can ne'er be turn'd to nought.'\\!
\end{verse}}{%
\attrib{\getauthornote{1992}\authormarginnote{1992.5} \getauthornote{1993}}}

Thus they go commonly meditating unto themselves, thus they sit, such is their
action and gesture. Fulgosus, \bookcite{\textlatin{l. 8, c. 7}}, makes mention
how Th. Aquinas supping with king Lewis of France, upon a sudden knocked his
fist upon the table, and cried, \li{conclusum est contra Manichaeos}, his wits
were a wool-gathering, as they say, and his head busied about other matters,
when he perceived his error, he was much \authorfootnote{1994}abashed. Such a
story there is of Archimedes in Vitruvius, that having found out the means to
know how much gold was mingled with the silver in king Hieron's crown, ran
naked forth of the bath and cried \textgreek{ἕυρηκα}, I have found:
\authorfootnote{1995}"and was commonly so intent to his studies, that he never
perceived what was done about him: when the city was taken, and the soldiers
now ready to rifle his house, he took no notice of it." St. Bernard rode all
day long by the Lemnian lake, and asked at last where he was, Marullus,
\bookcite{\textlatin{lib. 2, cap. 4.}} It was Democritus's carriage alone that
made the Abderites suppose him to have been mad, and send for Hippocrates to
cure him: if he had been in any solemn company, he would upon all occasions
fall a laughing. Theophrastus saith as much of Heraclitus, for that he
continually wept, and Laertius of Menedemus Lampsacus, because he ran like a
madman, \authorfootnote{1996}saying, "he came from hell as a spy, to tell the
devils what mortal men did." Your greatest students are commonly no better,
silly, soft fellows in their outward behaviour, absurd, ridiculous to others,
and no whit experienced in worldly business; they can measure the heavens,
range over the world, teach others wisdom, and yet in bargains and contracts
they are circumvented by every base tradesman. Are not these men fools? and how
should they be otherwise, "but as so many sots in schools, when" (as
\authorfootnote{1997}he well observed) "they neither hear nor see such things
as are commonly practised abroad?" how should they get experience, by what
means? \authorfootnote{1998}"I knew in my time many scholars," saith Aeneas
Sylvius (in an epistle of his to Gasper Scitick, chancellor to the emperor),
"excellent well learned, but so rude, so silly, that they had no common
civility, nor knew how to manage their domestic or public affairs."
"Paglarensis was amazed, and said his farmer had surely cozened him, when he
heard him tell that his sow had eleven pigs, and his ass had but one foal." To
say the best of this profession, I can give no other testimony of them in
general, than that of Pliny of Isaeus; \authorfootnote{1999}"He is yet a
scholar, than which kind of men there is nothing so simple, so sincere, none
better, they are most part harmless, honest, upright, innocent, plain-dealing
men."

Now because they are commonly subject to such hazards and inconveniences as
dotage, madness, simplicity, \etc{} Jo. Voschius would have good scholars to be
highly rewarded, and had in some extraordinary respect above other men, "to
have greater \authorfootnote{2000}privileges than the rest, that adventure
themselves and abbreviate their lives for the public good." But our patrons of
learning are so far nowadays from respecting the muses, and giving that honour
to scholars, or reward which they deserve, and are allowed by those indulgent
privileges of many noble princes, that after all their pains taken in the
universities, cost and charge, expenses, irksome hours, laborious tasks,
wearisome days, dangers, hazards, (barred interim from all pleasures which
other men have, mewed up like hawks all their lives) if they chance to wade
through them, they shall in the end be rejected, contemned, and which is their
greatest misery, driven to their shifts, exposed to want, poverty, and beggary.
Their familiar attendants are,

\translatedverse{%
\begin{latin}
\begin{verse}
Pallentes morbi, luctus, curaeque laborque\\*
Et metus, et malesuada fames, et turpis egestas,\\*
Terribiles visu formae------\\!
\end{verse}
\end{latin}}{%
\begin{verse}%
Grief, labour, care, pale sickness, miseries,\\*
Fear, filthy poverty, hunger that cries,\\*
Terrible monsters to be seen with eyes.\\!
\end{verse}}{%
\attrib{\getauthornote{2001}}}

If there were nothing else to trouble them, the conceit of this alone were
enough to make them all melancholy. Most other trades and professions, after
some seven years' apprenticeship, are enabled by their craft to live of
themselves. A merchant adventures his goods at sea, and though his hazard be
great, yet if one ship return of four, he likely makes a saving voyage. An
husbandman's gains are almost certain; \li{quibus ipse Jupiter nocere non
potest} (whom Jove himself can't harm) ('tis \authorfootnote{2002}Cato's
hyperbole, a great husband himself); only scholars methinks are most uncertain,
unrespected, subject to all casualties, and hazards. For first, not one of a
many proves to be a scholar, all are not capable and docile,
\authorfootnote{2003}\li{ex omniligno non fit Mercurius}: we can make majors
and officers every year, but not scholars: kings can invest knights and barons,
as Sigismund the emperor confessed; universities can give degrees; and \li{Tu
quod es, e populo quilibet esse potest}; but he nor they, nor all the world,
can give learning, make philosophers, artists, orators, poets; we can soon say,
as \Seneca{} well notes, \li{O virum bonum, o divitem}, point at a rich man, a
good, a happy man, a prosperous man, \li{sumptuose vestitum, Calamistratum,
bene olentem, magno temporis impendio constat haec laudatio, o virum
literarum}, but 'tis not so easily performed to find out a learned man.
Learning is not so quickly got, though they may be willing to take pains, to
that end sufficiently informed, and liberally maintained by their patrons and
parents, yet few can compass it. Or if they be docile, yet all men's wills are
not answerable to their wits, they can apprehend, but will not take pains; they
are either seduced by bad companions, \li{vel in puellam impingunt, vel in
poculum} (they fall in with women or wine) and so spend their time to their
friends' grief and their own undoings. Or put case they be studious,
industrious, of ripe wits, and perhaps good capacities, then how many diseases
of body and mind must they encounter? No labour in the world like unto study.
It may be, their temperature will not endure it, but striving to be excellent
to know all, they lose health, wealth, wit, life and all. Let him yet happily
escape all these hazards, \li{aereis intestinis} with a body of brass, and is
now consummate and ripe, he hath profited in his studies, and proceeded with
all applause: after many expenses, he is fit for preferment, where shall he
have it? he is as far to seek it as he was (after twenty years' standing) at
the first day of his coming to the University. For what course shall he take,
being now capable and ready? The most parable and easy, and about which many
are employed, is to teach a school, turn lecturer or curate, and for that he
shall have falconer's wages, ten pound per annum, and his diet, or some small
stipend, so long as he can please his patron or the parish; if they approve him
not (for usually they do but a year or two) as inconstant, as
\authorfootnote{2004}they that cried "Hosanna" one day, and "Crucify him" the
other; serving-man-like, he must go look a new master; if they do, what is his
reward?

\translatedverse{%
\begin{latin}
\begin{verse}
Hoc quoque te manet ut pueros elementa docentem\\*
Occupet extremis in vicis alba senectus.\\!
\end{verse}
\end{latin}}{%
\begin{verse}%
At last thy snow-white age in suburb schools,\\*
Shall toil in teaching boys their grammar rules.\\!
\end{verse}}{%
\attrib{\getauthornote{2005}}}

Like an ass, he wears out his time for provender, and can show a stump rod,
\li{togam tritam et laceram} saith \authorfootnote{2006}Haedus, an old torn
gown, an ensign of his infelicity, he hath his labour for his pain, a modicum
to keep him till he be decrepit, and that is all. \li{Grammaticus non est
felix}, \etc{} If he be a trencher chaplain in a gentleman's house, as it
befell \authorfootnote{2007}Euphormio, after some seven years' service, he may
perchance have a living to the halves, or some small rectory with the mother of
the maids at length, a poor kinswoman, or a cracked chambermaid, to have and to
hold during the time of his life. But if he offend his good patron, or
displease his lady mistress in the mean time,

\begin{latin}
\begin{verse}%
Ducetur Planta velut ictus ab Hercule Cacus,\\*
Poneturque foras, si quid tentaverit unquam\\*
Hiscere------\\!
\end{verse}%
\end{latin}
\attrib{\getauthornote{2008}}
as Hercules did by Cacus, he shall be dragged forth of doors by the heels, away
with him. If he bend his forces to some other studies, with an intent to be
\li{a secretis} to some nobleman, or in such a place with an ambassador, he
shall find that these persons rise like apprentices one under another, and in
so many tradesmen's shops, when the master is dead, the foreman of the shop
commonly steps in his place. Now for poets, rhetoricians, historians,
philosophers, \authorfootnote{2009}mathematicians, sophisters, \etc{}; they are
like grasshoppers, sing they must in summer, and pine in the winter, for there
is no preferment for them. Even so they were at first, if you will believe that
pleasant tale of Socrates, which he told fair Phaedrus under a plane-tree, at
the banks of the river Iseus; about noon when it was hot, and the grasshoppers
made a noise, he took that sweet occasion to tell him a tale, how grasshoppers
were once scholars, musicians, poets, \etc{}, before the Muses were born, and
lived without meat and drink, and for that cause were turned by Jupiter into
grasshoppers. And may be turned again, \li{In Tythoni Cicadas, aut Lyciorum
ranas}, for any reward I see they are like to have: or else in the mean time, I
would they could live, as they did, without any viaticum, like so many
\authorfootnote{2010}manucodiatae, those Indian birds of paradise, as we
commonly call them, those I mean that live with the air and dew of heaven, and
need no other food; for being as they are, their \authorfootnote{2011}"rhetoric
only serves them to curse their bad fortunes," and many of them for want of
means are driven to hard shifts; from grasshoppers they turn humble-bees and
wasps, plain parasites, and make the muses, mules, to satisfy their
hunger-starved paunches, and get a meal's meat. To say truth, 'tis the common
fortune of most scholars, to be servile and poor, to complain pitifully, and
lay open their wants to their respectless patrons, as
\authorfootnote{2012}Cardan doth, as \authorfootnote{2013}Xilander and many
others: and which is too common in those dedicatory epistles, for hope of gain,
to lie, flatter, and with hyperbolical eulogiums and commendations, to magnify
and extol an illiterate unworthy idiot, for his excellent virtues, whom they
should rather, as \authorfootnote{2014}Machiavel observes, vilify, and rail at
downright for his most notorious villainies and vices. So they prostitute
themselves as fiddlers, or mercenary tradesmen, to serve great men's turns for
a small reward. They are like \authorfootnote{2015}Indians, they have store of
gold, but know not the worth of it: for I am of Synesius's opinion,
\authorfootnote{2016}"King Hieron got more by Simonides' acquaintance, than
Simonides did by his;" they have their best education, good institution, sole
qualification from us, and when they have done well, their honour and
immortality from us: we are the living tombs, registers, and as so many
trumpeters of their fames: what was Achilles without \Homer{}? Alexander without
Arian and Curtius? who had known the Caesars, but for Suetonius and Dion?

\translatedverse{%
\begin{latin}
\begin{verse}
Vixerunt fortes ante Agamemnona\\*
Multi: sed omnes illachrymabiles\\*
Urgentur, ignotique longa\\*
Nocte, carent quia vate sacro.\\!
\end{verse}
\end{latin}}{%
\begin{verse}%
Before great Agamemnon reign'd,\\*
Reign'd kings as great as he, and brave,\\*
Whose huge ambition's now contain'd\\*
In the small compass of a grave:\\*
In endless night, they sleep, unwept, unknown,\\*
No bard they had to make all time their own.\\!
\end{verse}}{%
\attrib{\getauthornote{2017}}}
they are more beholden to scholars, than scholars to them; but they undervalue
themselves, and so by those great men are kept down. Let them have that
encyclopaedian, all the learning in the world; they must keep it to themselves,
\authorfootnote{2018}"live in base esteem, and starve, except they will
submit," as Budaeus well hath it, "so many good parts, so many ensigns of arts,
virtues, be slavishly obnoxious to some illiterate potentate, and live under
his insolent worship, or honour, like parasites," \li{Qui tanquam mures alienum
panem comedunt}. For to say truth, \li{artes hae, non sunt Lucrativae}, as
Guido Bonat that great astrologer could foresee, they be not gainful arts
these, \li{sed esurientes et famelicae}, but poor and hungry.

\translatedverse{%
\begin{latin}
\begin{verse}
Dat Galenus opes, dat Justinianus honores,\\*
Sed genus et species cogitur ire pedes:\\!
\end{verse}
\end{latin}}{%
\begin{verse}%
The rich physician, honour'd lawyers ride,\\*
Whilst the poor scholar foots it by their side.\\!
\end{verse}}{%
\attrib{\getauthornote{2019}}}

Poverty is the muses' patrimony, and as that poetical divinity teacheth us,
when Jupiter's daughters were each of them married to the gods, the muses alone
were left solitary, Helicon forsaken of all suitors, and I believe it was,
because they had no portion.

\translatedverse{%
\begin{latin}
\begin{verse}
Calliope longum caelebs cur vixit in aevum?\\*
Nempe nihil dotis, quod numeraret, erat.\\!
\end{verse}
\end{latin}}{%
\begin{verse}%
Why did Calliope live so long a maid?\\*
Because she had no dowry to be paid.\\!
\end{verse}}{}%

Ever since all their followers are poor, forsaken and left unto themselves.
Insomuch, that as \authorfootnote{2020}\Petronius argues, you shall likely know
them by their clothes. "There came," saith he, "by chance into my company, a
fellow not very spruce to look on, that I could perceive by that note alone he
was a scholar, whom commonly rich men hate: I asked him what he was, he
answered, a poet: I demanded again why he was so ragged, he told me this kind
of learning never made any man rich."

\translatedverse{%
\begin{latin}
\begin{verse}
Qui Pelago credit, magno se faenore tollit,\\*
Qui pugnas et rostra petit, praecingitur auro:\\*
Vilis adulator picto jacet ebrius ostro,\\*
Sola pruinosis horret facundia pannis.\\!
\end{verse}
\end{latin}}{%
\begin{verse}%
A merchant's gain is great, that goes to sea;\\*
A soldier embossed all in gold;\\*
A flatterer lies fox'd in brave array;\\*
A scholar only ragged to behold.\\!
\end{verse}}{%
\attrib{\getauthornote{2021}}}

All which our ordinary students, right well perceiving in the universities, how
unprofitable these poetical, mathematical, and philosophical studies are, how
little respected, how few patrons; apply themselves in all haste to those three
commodious professions of law, physic, and divinity, sharing themselves between
them, \authorfootnote{2022}rejecting these arts in the mean time, history,
philosophy, philology, or lightly passing them over, as pleasant toys fitting
only table-talk, and to furnish them with discourse. They are not so behoveful:
he that can tell his money hath arithmetic enough: he is a true geometrician,
can measure out a good fortune to himself; a perfect astrologer, that can cast
the rise and fall of others, and mark their errant motions to his own use. The
best optics are, to reflect the beams of some great man's favour and grace to
shine upon him. He is a good engineer that alone can make an instrument to get
preferment. This was the common tenet and practice of Poland, as Cromerus
observed not long since, in the first book of his history; their universities
were generally base, not a philosopher, a mathematician, an antiquary, \etc{},
to be found of any note amongst them, because they had no set reward or
stipend, but every man betook himself to divinity, \li{hoc solum in votis
habens, opimum sacerdotium}, a good parsonage was their aim. This was the
practice of some of our near neighbours, as \authorfootnote{2023}Lipsius
inveighs, "they thrust their children to the study of law and divinity, before
they be informed aright, or capable of such studies." \li{Scilicet omnibus
artibus antistat spes lucri, et formosior est cumulus auri, quam quicquid
Graeci Latinique delirantes scripserunt. Ex hoc numero deinde veniunt ad
gubernacula reipub. intersunt et praesunt consiliis regum, o pater, o patria}?
so he complained, and so may others. For even so we find, to serve a great man,
to get an office in some bishop's court (to practise in some good town) or
compass a benefice, is the mark we shoot at, as being so advantageous, the
highway to preferment.

Although many times, for aught I can see, these men fail as often as the rest
in their projects, and are as usually frustrate of their hopes. For let him be
a doctor of the law, an excellent civilian of good worth, where shall he
practise and expatiate? Their fields are so scant, the civil law with us so
contracted with prohibitions, so few causes, by reason of those all-devouring
municipal laws, \li{quibus nihil illiteratius}, saith
\authorfootnote{2024}Erasmus, an illiterate and a barbarous study, (for though
they be never so well learned in it, I can hardly vouchsafe them the name of
scholars, except they be otherwise qualified) and so few courts are left to
that profession, such slender offices, and those commonly to be compassed at
such dear rates, that I know not how an ingenious man should thrive amongst
them. Now for physicians, there are in every village so many mountebanks,
empirics, quacksalvers, Paracelsians, as they call themselves, \li{Caucifici et
sanicidae} so \authorfootnote{2025}Clenard terms them, wizards, alchemists,
poor vicars, cast apothecaries, physicians' men, barbers, and good wives,
professing great skill, that I make great doubt how they shall be maintained,
or who shall be their patients. Besides, there are so many of both sorts, and
some of them such harpies, so covetous, so clamorous, so impudent; and as
\authorfootnote{2026}he said, litigious idiots,

\translatedverse{%
\begin{latin}
\begin{verse}%
Quibus loquacis affatim arrogantiae est\\*
Pentiae parum aut nihil,\\*
Nec ulla mica literarii salis,\\*
Crumenimulga natio:\\*
Loquuteleia turba, litium strophae,\\*
Maligna litigantium cohors, togati vultures,\\*
Lavernae alumni, Agyrtae, \etc{}\\!
\end{verse}%
\end{latin}}{%
\begin{verse}%
Which have no skill but prating arrogance,\\*
No learning, such a purse-milking nation:\\*
Gown'd vultures, thieves, and a litigious rout\\*
Of cozeners, that haunt this occupation,\\!
\end{verse}}{}%
that they cannot well tell how to live one by another, but as he jested in the
Comedy of Clocks, they were so many, \authorfootnote{2027}\li{major pars populi
arida reptant fame}, they are almost starved a great part of them, and ready to
devour their fellows, \authorfootnote{2028}\li{Et noxia callidilate se
corripere}, such a multitude of pettifoggers and empirics, such impostors, that
an honest man knows not in what sort to compose and behave himself in their
society, to carry himself with credit in so vile a rout, \li{scientiae nomen,
tot sumptibus partum et vigiliis, profiteri dispudeat, postquam}, \etc{}

Last of all to come to our divines, the most noble profession and worthy of
double honour, but of all others the most distressed and miserable. If you will
not believe me, hear a brief of it, as it was not many years since publicly
preached at Paul's cross, \authorfootnote{2029}by a grave minister then, and
now a reverend bishop of this land: "We that are bred up in learning, and
destinated by our parents to this end, we suffer our childhood in the
grammar-school, which \Austin{} calls \li{magnam tyrannidem, et grave malum}, and
compares it to the torments of martyrdom; when we come to the university, if we
live of the college allowance, as Phalaris objected to the Leontines,
\textgreek{παν τῶν ἐνδεῖς πλὴν λιμοὺ καὶ φόβου}, needy of all things but hunger
and fear, or if we be maintained but partly by our parents' cost, do expend in
unnecessary maintenance, books and degrees, before we come to any perfection,
five hundred pounds, or a thousand marks. If by this price of the expense of
time, our bodies and spirits, our substance and patrimonies, we cannot purchase
those small rewards, which are ours by law, and the right of inheritance, a
poor parsonage, or a vicarage of 50 \emph{l.} per annum, but we must pay to the
patron for the lease of a life (a spent and out-worn life) either in annual
pension, or above the rate of a copyhold, and that with the hazard and loss of
our souls, by simony and perjury, and the forfeiture of all our spiritual
preferments, in \li{esse} and \li{posse}, both present and to come. What father
after a while will be so improvident to bring up his son to his great charge,
to this necessary beggary? What Christian will be so irreligious, to bring up
his son in that course of life, which by all probability and necessity,
\li{cogit ad turpia}, enforcing to sin, will entangle him in simony and
perjury, when as the poet said, \li{Invitatus ad haec aliquis de ponte
negabit}: a beggar's brat taken from the bridge where he sits a begging, if he
knew the inconvenience, had cause to refuse it." This being thus, have not we
fished fair all this while, that are initiate divines, to find no better fruits
of our labours, \authorfootnote{2030}\li{hoc est cur palles, cur quis non
prandeat hoc est}? do we macerate ourselves for this? Is it for this we rise so
early all the year long? \authorfootnote{2031}"Leaping" (as he saith) "out of
our beds, when we hear the bell ring, as if we had heard a thunderclap." If
this be all the respect, reward and honour we shall have,
\authorfootnote{2032}\li{frange leves calamos, et scinde Thalia libellos}: let
us give over our books, and betake ourselves to some other course of life; to
what end should we study? \authorfootnote{2033}\li{Quid me litterulas stulti
docuere parentes}, what did our parents mean to make us scholars, to be as far
to seek of preferment after twenty years' study, as we were at first: why do we
take such pains? \li{Quid tantum insanis juvat impallescere chartis}? If there
be no more hope of reward, no better encouragement, I say again, \li{Frange
leves calamos, et scinde Thalia libellos}; let's turn soldiers, sell our books,
and buy swords, guns, and pikes, or stop bottles with them, turn our
philosopher's gowns, as Cleanthes once did, into millers' coats, leave all and
rather betake ourselves to any other course of life, than to continue longer in
this misery. \authorfootnote{2034}\li{Praestat dentiscalpia radere, quam
literariis monumentis magnatum favorem emendicare}.

Yea, but methinks I hear some man except at these words, that though this be
true which I have said of the estate of scholars, and especially of divines,
that it is miserable and distressed at this time, that the church suffers
shipwreck of her goods, and that they have just cause to complain; there is a
fault, but whence proceeds it? If the cause were justly examined, it would be
retorted upon ourselves, if we were cited at that tribunal of truth, we should
be found guilty, and not able to excuse it That there is a fault among us, I
confess, and were there not a buyer, there would not be a seller; but to him
that will consider better of it, it will more than manifestly appear, that the
fountain of these miseries proceeds from these griping patrons. In accusing
them, I do not altogether excuse us; both are faulty, they and we: yet in my
judgment, theirs is the greater fault, more apparent causes and much to be
condemned. For my part, if it be not with me as I would, or as it should, I do
ascribe the cause, as \authorfootnote{2035}Cardan did in the like case; \li{meo
infortunio potius quam illorum sceleri}, to \authorfootnote{2036}mine own
infelicity rather than their naughtiness: although I have been baffled in my
time by some of them, and have as just cause to complain as another: or rather
indeed to mine own negligence; for I was ever like that Alexander in
\authorfootnote{2037}\Plutarch{}, Crassus his tutor in philosophy, who, though he
lived many years familiarly with rich Crassus, was even as poor when from,
(which many wondered at) as when he came first to him; he never asked, the
other never gave him anything; when he travelled with Crassus he borrowed a hat
of him, at his return restored it again. I have had some such noble friends'
acquaintance and scholars, but most part (common courtesies and ordinary
respects excepted) they and I parted as we met, they gave me as much as I
requested, and that was--And as Alexander ab Alexandro
\bookcite{\textlatin{Genial. dier. l. 6. c. 16.}} made answer to Hieronymus
Massainus, that wondered, \li{quum plures ignavos et ignobiles ad dignitates et
sacerdotia promotos quotidie videret}, when other men rose, still he was in the
same state, \li{eodem tenore et fortuna cui mercedem laborum studiorumque
deberi putaret}, whom he thought to deserve as well as the rest. He made
answer, that he was content with his present estate, was not ambitious, and
although \li{objurgabundus suam segnitiem accusaret, cum obscurae sortis
homines ad sacerdotia et pontificatus evectos}, \etc{}, he chid him for his
backwardness, yet he was still the same: and for my part (though I be not
worthy perhaps to carry Alexander's books) yet by some overweening and
well-wishing friends, the like speeches have been used to me; but I replied
still with Alexander, that I had enough, and more peradventure than I deserved;
and with Libanius Sophista, that rather chose (when honours and offices by the
emperor were offered unto him) to be \li{talis Sophista, quam tails
Magistratus}. I had as lief be still Democritus junior, and \li{privus
privatus, si mihi jam daretur optio, quam talis fortasse Doctor, talis
Dominus.--Sed quorsum haec}? For the rest 'tis on both sides \li{facinus
detestandum}, to buy and sell livings, to detain from the church, that which
God's and men's laws have bestowed on it; but in them most, and that from the
covetousness and ignorance of such as are interested in this business; I name
covetousness in the first place, as the root of all these mischiefs, which,
Achan-like, compels them to commit sacrilege, and to make simoniacal compacts,
(and what not) to their own ends, \authorfootnote{2038}that kindles God's
wrath, brings a plague, vengeance, and a heavy visitation upon themselves and
others. Some out of that insatiable desire of filthy lucre, to be enriched,
care not how they come by it \li{per fas et nefas}, hook or crook, so they have
it. And others when they have with riot and prodigality embezzled their
estates, to recover themselves, make a prey of the church, robbing it, as
\authorfootnote{2039}Julian the apostate did, spoil parsons of their revenues
(in keeping half back, \authorfootnote{2040}as a great man amongst us
observes:) "and that maintenance on which they should live:" by means whereof,
barbarism is increased, and a great decay of Christian professors: for who will
apply himself to these divine studies, his son, or friend, when after great
pains taken, they shall have nothing whereupon to live? But with what event do
they these things?

\begin{latin}
\begin{verse}%
Opesque totis viribus venamini\\*
At inde messis accidit miserrima.\\!
\end{verse}%
\end{latin}
\attrib{\getauthornote{2041}}

They toil and moil, but what reap they? They are commonly unfortunate families
that use it, accursed in their progeny, and, as common experience evinceth,
accursed themselves in all their proceedings. "With what face" (as
\authorfootnote{2042}he quotes out of Aust.) "can they expect a blessing or
inheritance from Christ in heaven, that defraud Christ of his inheritance here
on earth?" I would all our simoniacal patrons, and such as detain tithes, would
read those judicious tracts of Sir Henry Spelman, and Sir James Sempill,
knights; those late elaborate and learned treatises of Dr. Tilslye, and Mr.
Montague, which they have written of that subject. But though they should read,
it would be to small purpose, \li{clames licet et mare coelo Confundas};
thunder, lighten, preach hell and damnation, tell them 'tis a sin, they will
not believe it; denounce and terrify, they have \authorfootnote{2043}cauterised
consciences, they do not attend, as the enchanted adder, they stop their ears.
Call them base, irreligious, profane, barbarous, pagans, atheists, epicures,
(as some of them surely are) with the bawd in Plautus, \li{Euge, optime}, they
cry and applaud themselves with that miser, \authorfootnote{2044}\li{simul ac
nummos contemplor in arca}: say what you will, \li{quocunque modo rem}: as a
dog barks at the moon, to no purpose are your sayings: Take your heaven, let
them have money. A base, profane, epicurean, hypocritical rout: for my part,
let them pretend what zeal they will, counterfeit religion, blear the world's
eyes, bombast themselves, and stuff out their greatness with church spoils,
shine like so many peacocks; so cold is my charity, so defective in this
behalf, that I shall never think better of them, than that they are rotten at
core, their bones are full of epicurean hypocrisy, and atheistical marrow, they
are worse than heathens. For as Dionysius Halicarnassaeus observes,
\bookcite{\textlatin{Antiq. Rom. lib. 7.}} \authorfootnote{2045}\li{Primum
locum}, \etc{} "Greeks and Barbarians observe all religious rites, and dare not
break them for fear of offending their gods;" but our simoniacal contractors,
our senseless Achans, our stupefied patrons, fear neither God nor devil, they
have evasions for it, it is no sin, or not due \li{jure divino}, or if a sin,
no great sin, \etc{} And though they be daily punished for it, and they do
manifestly perceive, that as he said, frost and fraud come to foul ends; yet as
\authorfootnote{2046}\Chrysostom{} follows it \li{Nulla ex poena sit correctio, et
quasi adversis malitia hominum provocetur, crescit quotidie quod puniatur}:
they are rather worse than better,-- \li{iram atque animos a crimine sumunt},
and the more they are corrected, the more they offend: but let them take their
course, \authorfootnote{2047}\li{Rode caper vites}, go on still as they begin,
'tis no sin, let them rejoice secure, God's vengeance will overtake them in the
end, and these ill-gotten goods, as an eagle's feathers,
\authorfootnote{2048}will consume the rest of their substance; it is
\authorfootnote{2049}\li{aurum Tholosanum}, and will produce no better effects.
\authorfootnote{2050}"Let them lay it up safe, and make their conveyances never
so close, lock and shut door," saith \Chrysostom{}, "yet fraud and covetousness,
two most violent thieves are still included, and a little gain evil gotten will
subvert the rest of their goods." The eagle in Aesop, seeing a piece of flesh
now ready to be sacrificed, swept it away with her claws, and carried it to her
nest; but there was a burning coal stuck to it by chance, which unawares
consumed her young ones, nest, and all together. Let our simoniacal
church-chopping patrons, and sacrilegious harpies, look for no better success.

A second cause is ignorance, and from thence contempt, \li{successit odium in
literas ab ignorantia vulgi}; which \authorfootnote{2051}Junius well perceived:
this hatred and contempt of learning proceeds out of
\authorfootnote{2052}ignorance; as they are themselves barbarous, idiots, dull,
illiterate, and proud, so they esteem of others. \li{Sint Mecaenates, non
deerunt Flacce Marones}: Let there be bountiful patrons, and there will be
painful scholars in all sciences. But when they contemn learning, and think
themselves sufficiently qualified, if they can write and read, scramble at a
piece of evidence, or have so much Latin as that emperor had, \li{qui nescit
dissimulare, nescit vivere}\authorlatintrans{2053}, they are unfit to do their
country service, to perform or undertake any action or employment, which may
tend to the good of a commonwealth, except it be to fight, or to do country
justice, with common sense, which every yeoman can likewise do. And so they
bring up their children, rude as they are themselves, unqualified, untaught,
uncivil most part. \authorfootnote{2054}\li{Quis e nostra juventute legitime
instituitur literis? Quis oratores aut Philosophos tangit? quis historiam
legit, illam rerum agendarum quasi animam? praecipitant parentes vota sua},
\etc{} 'twas Lipsius' complaint to his illiterate countrymen, it may be ours.
Now shall these men judge of a scholar's worth, that have no worth, that know
not what belongs to a student's labours, that cannot distinguish between a true
scholar and a drone? or him that by reason of a voluble tongue, a strong voice,
a pleasing tone, and some trivially polyanthean helps, steals and gleans a few
notes from other men's harvests, and so makes a fairer show, than he that is
truly learned indeed: that thinks it no more to preach, than to speak,
\authorfootnote{2055}"or to run away with an empty cart;" as a grave man said:
and thereupon vilify us, and our pains; scorn us, and all learning.
\authorfootnote{2056}Because they are rich, and have other means to live, they
think it concerns them not to know, or to trouble themselves with it; a fitter
task for younger brothers, or poor men's sons, to be pen and inkhorn men,
pedantical slaves, and no whit beseeming the calling of a gentleman, as
Frenchmen and Germans commonly do, neglect therefore all human learning, what
have they to do with it? Let mariners learn astronomy; merchants, factors study
arithmetic; surveyors get them geometry; spectacle-makers optics; land-leapers
geography; town-clerks rhetoric, what should he do with a spade, that hath no
ground to dig; or they with learning, that have no use of it? thus they reason,
and are not ashamed to let mariners, apprentices, and the basest servants, be
better qualified than themselves. In former times, kings, princes, and
emperors, were the only scholars, excellent in all faculties. Julius Caesar
mended the year, and writ his own Commentaries,

\begin{latin}
\begin{verse}%
------media inter prealia semper,\\*
Stellarum coelique plagis, superisque vacavit.\\!
\end{verse}%
\end{latin}
\attrib{\getauthornote{2057}}

\authorfootnote{2058}Antonius, Adrian, Nero, Seve. Jul. \etc{}
\authorfootnote{2059}Michael the emperor, and Isacius, were so much given to
their studies, that no base fellow would take so much pains: Orion, Perseus,
Alphonsus, Ptolomeus, famous astronomers; Sabor, Mithridates, Lysimachus,
admired physicians: Plato's kings all: Evax, that Arabian prince, a most expert
jeweller, and an exquisite philosopher; the kings of Egypt were priests of old,
chosen and from thence,-- \li{Idem rex hominum, Phoebique sacerdos}: but those
heroical times are past; the Muses are now banished in this bastard age, \li{ad
sordida tuguriola}, to meaner persons, and confined alone almost to
universities. In those days, scholars were highly beloved,
\authorfootnote{2060}honoured, esteemed; as old Ennius by Scipio Africanus,
\Virgil{} by Augustus; \Horace{} by Meceanas: princes' companions; dear to them, as
Anacreon to Polycrates; Philoxenus to Dionysius, and highly rewarded. Alexander
sent Xenocrates the philosopher fifty talents, because he was poor, \li{visu
rerum, aut eruditione praestantes viri, mensis olim regum adhibiti}, as
Philostratus relates of Adrian and Lampridius of Alexander Severus: famous
clerks came to these princes' courts, \li{velut in Lycaeum}, as to a
university, and were admitted to their tables, \li{quasi divum epulis
accumbentes}; Archilaus, that Macedonian king, would not willingly sup without
Euripides, (amongst the rest he drank to him at supper one night, and gave him
a cup of gold for his pains) \li{delectatus poetae suavi sermone}; and it was
fit it should be so; because as \authorfootnote{2061}Plato in his Protagoras
well saith, a good philosopher as much excels other men, as a great king doth
the commons of his country; and again, \authorfootnote{2062}\li{quoniam illis
nihil deest, et minime egere solent, et disciplinas quas profitentur, soli a
contemptu vindicare possunt}, they needed not to beg so basely, as they compel
\authorfootnote{2063}scholars in our times to complain of poverty, or crouch to
a rich chuff for a meal's meat, but could vindicate themselves, and those arts
which they professed. Now they would and cannot: for it is held by some of
them, as an axiom, that to keep them poor, will make them study; they must be
dieted, as horses to a race, not pampered, \authorfootnote{2064}\li{Alendos
volunt, non saginandos, ne melioris mentis flammula extinguatur}; a fat bird
will not sing, a fat dog cannot hunt, and so by this depression of theirs
\authorfootnote{2065}some want means, others will, all want
\authorfootnote{2066}encouragement, as being forsaken almost; and generally
contemned. 'Tis an old saying, \li{Sint Mecaenates, non deerunt Flacce
Marones}, and 'tis a true saying still. Yet oftentimes I may not deny it the
main fault is in ourselves. Our academics too frequently offend in neglecting
patrons, as \authorfootnote{2067}Erasmus well taxeth, or making ill choice of
them; \li{negligimus oblatos aut amplectimur parum aptos}, or if we get a good
one, \li{non studemus mutuis officiis favorem ejus alere}, we do not ply and
follow him as we should. \li{Idem mihi accidit Adolescenti} (saith Erasmus)
acknowledging his fault, \li{et gravissime peccavi}, and so may
\authorfootnote{2068}I say myself, I have offended in this, and so peradventure
have many others. We did not \li{spondere magnatum favoribus, qui caeperunt nos
amplecti}, apply ourselves with that readiness we should: idleness, love of
liberty, \li{immodicus amor libertatis effecit ut diu cum perfidis amicis}, as
he confesseth, \li{et pertinaci pauperate colluctarer}, bashfulness,
melancholy, timorousness, cause many of us to be too backward and remiss. So
some offend in one extreme, but too many on the other, we are most part too
forward, too solicitous, too ambitious, too impudent; we commonly complain
\li{deesse Maecenates}, of want of encouragement, want of means, when as the
true defect is in our own want of worth, our insufficiency: did Maecenas take
notice of \Horace{} or \Virgil{} till they had shown themselves first? or had Bavius
and Mevius any patrons? \li{Egregium specimen dent}, saith Erasmus, let them
approve themselves worthy first, sufficiently qualified for learning and
manners, before they presume or impudently intrude and put themselves on great
men as too many do, with such base flattery, parasitical colloguing, such
hyperbolical elogies they do usually insinuate that it is a shame to hear and
see. \li{Immodicae laudes conciliant invidiam, potius quam laudem}, and vain
commendations derogate from truth, and we think in conclusion, \li{non melius
de laudato, pejus de laudante}, ill of both, the commender and commended. So we
offend, but the main fault is in their harshness, defect of patrons. How
beloved of old, and how much respected was Plato to Dionysius? How dear to
Alexander was \Aristotle{}, Demeratus to Philip, Solon to Croesus, Auexarcus and
Trebatius to Augustus, Cassius to Vespasian, \Plutarch{} to Trajan, \Seneca{} to
Nero, Simonides to Hieron? how honoured?

\begin{latin}
\begin{verse}%
Sed haec prius fuere, nunc recondita\\*
Senent quiete,\\!
\end{verse}%
\end{latin}
\attrib{\getauthornote{2069}}

those days are gone; \li{Et spes, et ratio studiorum in Caesare
tantum}\authorlatintrans{2070}: as he said of old, we may truly say now, he is
our amulet, our \authorfootnote{2071}sun, our sole comfort and refuge, our
Ptolemy, our common Maecenas, \li{Jacobus munificus, Jacobus pacificus, mysta
Musarum, Rex Platonicus: Grande decus, columenque nostrum}: a famous scholar
himself, and the sole patron, pillar, and sustainer of learning: but his worth
in this kind is so well known, that as Paterculus of Cato, \li{Jam ipsum
laudare nefas sit}: and which \authorfootnote{2072}Pliny to Trajan. \li{Seria
te carmina, honorque aeternus annalium, non haec brevis et pudenda praedicatio
colet}. But he is now gone, the sun of ours set, and yet no night follows,
\li{Sol occubuit, nox nulla sequuta est}. We have such another in his room,
\authorfootnote{2073}\li{aureus alter. Avulsus, simili frondescit virga
metallo}, and long may he reign and flourish amongst us.

Let me not be malicious, and lie against my genius, I may not deny, but that we
have a sprinkling of our gentry, here and there one, excellently well learned,
like those Fuggeri in Germany; Dubartus, Du Plessis, Sadael, in France; Picus
Mirandula, Schottus, Barotius, in Italy; \li{Apparent rari nantes in gurgite
vasto}. But they are but few in respect of the multitude, the major part (and
some again excepted, that are indifferent) are wholly bent for hawks and
hounds, and carried away many times with intemperate lust, gaming and drinking.
If they read a book at any time (\li{si quod est interim otii a venatu,
poculis, alea, scortis}) 'tis an English Chronicle, St. Huon of Bordeaux,
Amadis de Gaul, \etc{}, a play-book, or some pamphlet of news, and that at such
seasons only, when they cannot stir abroad, to drive away time,
\authorfootnote{2074}their sole discourse is dogs, hawks, horses, and what
news? If some one have been a traveller in Italy, or as far as the emperor's
court, wintered in Orleans, and can court his mistress in broken French, wear
his clothes neatly in the newest fashion, sing some choice outlandish tunes,
discourse of lords, ladies, towns, palaces, and cities, he is complete and to
be admired: \authorfootnote{2075}otherwise he and they are much at one; no
difference between the master and the man, but worshipful titles; wink and
choose betwixt him that sits down (clothes excepted) and him that holds the
trencher behind him: yet these men must be our patrons, our governors too
sometimes, statesmen, magistrates, noble, great, and wise by inheritance.

Mistake me not (I say again) \li{Vos o Patritius sanguis}, you that are worthy
senators, gentlemen, I honour your names and persons, and with all
submissiveness, prostrate myself to your censure and service. There are amongst
you, I do ingenuously confess, many well-deserving patrons, and true patriots,
of my knowledge, besides many hundreds which I never saw, no doubt, or heard
of, pillars of our commonwealth, \authorfootnote{2076}whose worth, bounty,
learning, forwardness, true zeal in religion, and good esteem of all scholars,
ought to be consecrated to all posterity; but of your rank, there are a
debauched, corrupt, covetous, illiterate crew again, no better than stocks,
\li{merum pecus (testor Deum, non mihi videri dignos ingenui hominis
appellatione)} barbarous Thracians, \li{et quis ille thrax qui hoc neget}? a
sordid, profane, pernicious company, irreligious, impudent and stupid, I know
not what epithets to give them, enemies to learning, confounders of the church,
and the ruin of a commonwealth; patrons they are by right of inheritance, and
put in trust freely to dispose of such livings to the church's good; but (hard
taskmasters they prove) they take away their straw, and compel them to make
their number of brick: they commonly respect their own ends, commodity is the
steer of all their actions, and him they present in conclusion, as a man of
greatest gifts, that will give most; no penny, \authorfootnote{2077}no
paternoster, as the saying is. \li{Nisi preces auro fulcias, amplius irritas:
ut Cerberus offa}, their attendants and officers must be bribed, feed, and
made, as Cerberus is with a sop by him that goes to hell. It was an old saying,
\lit{Omnia Romae venalia}{all things are venal at Rome}, 'tis a rag of Popery,
which will never be rooted out, there is no hope, no good to be done without
money. A clerk may offer himself, approve his \authorfootnote{2078}worth,
learning, honesty, religion, zeal, they will commend him for it; but
\authorfootnote{2079}\li{probitas laudatur et alget}. If he be a man of
extraordinary parts, they will flock afar off to hear him, as they did in
\Apuleius, to see Psyche: \li{multi mortales confluebant ad videndum saeculi
decus, speculum gloriosum, laudatur ab omnibus, spectatur ob omnibus, nec
quisquam non rex, non regius, cupidus ejus nuptiarium petitor accedit; mirantur
quidem divinam formam omnes, sed ut simulacrum fabre politum mirantur}; many
mortal men came to see fair Psyche the glory of her age, they did admire her,
commend, desire her for her divine beauty, and gaze upon her; but as on a
picture; none would marry her, \li{quod indotato}, fair Psyche had no money.
\authorfootnote{2080}So they do by learning;

\translatedverse{%
\begin{latin}
\begin{verse}%
------didicit jam dives avarus\\*
Tantum admirari, tantum laudare disertos,\\*
Ut pueri Junonis avem------\\!
\end{verse}%
\end{latin}}{%
\begin{verse}%
Your rich men have now learn'd of latter days\\*
T'admire, commend, and come together\\*
To hear and see a worthy scholar speak,\\*
As children do a peacock's feather.\\!
\end{verse}}{%
\attrib{\getauthornote{2081}}}

He shall have all the good words that may be given, \authorfootnote{2082}a
proper man, and 'tis pity he hath no preferment, all good wishes, but
inexorable, indurate as he is, he will not prefer him, though it be in his
power, because he is \li{indotatus}, he hath no money. Or if he do give him
entertainment, let him be never so well qualified, plead affinity,
consanguinity, sufficiency, he shall serve seven years, as Jacob did for
Rachel, before he shall have it. \authorfootnote{2083}If he will enter at
first, he must get in at that Simoniacal gate, come off soundly, and put in
good security to perform all covenants, else he will not deal with, or admit
him. But if some poor scholar, some parson chaff, will offer himself; some
trencher chaplain, that will take it to the halves, thirds, or accepts of what
he will give, he is welcome; be conformable, preach as he will have him, he
likes him before a million of others; for the host is always best cheap: and
then as Hierom said to Cromatius, \li{patella dignum operculum}, such a patron,
such a clerk; the cure is well supplied, and all parties pleased. So that is
still verified in our age, which \authorfootnote{2084}\Chrysostom{} complained of
in his time, \li{Qui opulentiores sunt, in ordinem parasitorum cogunt eos, et
ipsos tanquam canes ad mensas suas enutriunt, eorumque impudentes. Venires
iniquarum coenarum reliquiis differtiunt, iisdem pro arbitro abulentes}: Rich
men keep these lecturers, and fawning parasites, like so many dogs at their
tables, and filling their hungry guts with the offals of their meat, they abuse
them at their pleasure, and make them say what they propose.
\authorfootnote{2085}"As children do by a bird or a butterfly in a string, pull
in and let him out as they list, do they by their trencher chaplains,
prescribe, command their wits, let in and out as to them it seems best." If the
patron be precise, so must his chaplain be; if he be papistical, his clerk must
be so too, or else be turned out. These are those clerks which serve the turn,
whom they commonly entertain, and present to church livings, whilst in the
meantime we that are University men, like so many hidebound calves in a
pasture, tarry out our time, wither away as a flower ungathered in a garden,
and are never used; or as so many candles, illuminate ourselves alone,
obscuring one another's light, and are not discerned here at all, the least of
which, translated to a dark room, or to some country benefice, where it might
shine apart, would give a fair light, and be seen over all. Whilst we lie
waiting here as those sick men did at the Pool of
\authorfootnote{2086}Bethesda, till the Angel stirred the water, expecting a
good hour, they step between, and beguile us of our preferment. I have not yet
said, if after long expectation, much expense, travel, earnest suit of
ourselves and friends, we obtain a small benefice at last; our misery begins
afresh, we are suddenly encountered with the flesh, world, and devil, with a
new onset; we change a quiet life for an ocean of troubles, we come to a
ruinous house, which before it be habitable, must be necessarily to our great
damage repaired; we are compelled to sue for dilapidations, or else sued
ourselves, and scarce yet settled, we are called upon for our predecessor's
arrearages; first-fruits, tenths, subsidies, are instantly to be paid,
benevolence, procurations, \etc{}, and which is most to be feared, we light
upon a cracked title, as it befell Clenard of Brabant, for his rectory, and
charge of his \li{Beginae}; he was no sooner inducted, but instantly sued,
\li{cepimusque} \authorfootnote{2087}(saith he) \li{strenue litigare, et
implacabili bello confligere}: at length after ten years' suit, as long as
Troy's siege, when he had tired himself, and spent his money, he was fain to
leave all for quietness' sake, and give it up to his adversary. Or else we are
insulted over, and trampled on by domineering officers, fleeced by those greedy
harpies to get more fees; we stand in fear of some precedent lapse; we fall
amongst refractory, seditious sectaries, peevish puritans, perverse papists, a
lascivious rout of atheistical Epicures, that will not be reformed, or some
litigious people (those wild beasts of Ephesus must be fought with) that will
not pay their dues without much repining, or compelled by long suit; \li{Laici
clericis oppido infesti}, an old axiom, all they think well gotten that is had
from the church, and by such uncivil, harsh dealings, they make their poor
minister weary of his place, if not his life; and put case they be quiet honest
men, make the best of it, as often it falls out, from a polite and terse
academic, he must turn rustic, rude, melancholise alone, learn to forget, or
else, as many do, become maltsters, graziers, chapmen, \etc{} (now banished
from the academy, all commerce of the muses, and confined to a country village,
as \Ovid{} was from Rome to Pontus), and daily converse with a company of idiots
and clowns.

\begin{latin}
Nos interim quod, attinet (nec enim immunes ab hac noxa sumus) idem realus
manet, idem nobis, et si non multo gravius, crimen objici potest: nostra enim
culpa sit, nostra incuria, nostra avaritia, quod tam frequentes, foedaeque
fiant in Ecclesia nundinationes, (templum est vaenale, deusque) tot sordes
invehantur, tanta grassetur impietas, tanta nequitia, tam insanus miseriarum
Euripus, et turbarum aestuarium, nostro inquam, omnium (Academicorum
imprimis) vitio sit. Quod tot Resp. malis afficiatur, a nobis seminarium;
ultro malum hoc accersimus, et quavis contumelia, quavis interim miseria
digni, qui pro virili non occurrimus. Quid enim fieri posse speramus, quum
tot indies sine delectu pauperes alumni, terrae filii, et cujuscunque ordinis
homunciones ad gradus certatim admittantur? qui si definitionem,
distinctionemque unam aut alteram memoriter edidicerint, et pro more tot
annos in dialectica posuerint, non refert quo profectu, quales demum sint,
idiotae, nugatores, otiatores, aleatores, compotores, indigni, libidinis
voluptatumque administri, "Sponsi Penelopes, nebulones, Alcinoique," modo tot
annos in academia insumpserint, et se pro togatis venditarint; lucri causa,
et amicorum intercessu praesentantur; addo etiam et magnificis nonnunquam
elogiis morum et scientiae; et jam valedicturi testimonialibus hisce
litteris, amplissime conscriptis in eorum gratiam honorantur, abiis, qui
fidei suae et existimationis jacturam proculdubio faciunt. "Doctores enim et
professores" (quod ait \authorfootnote{2088}ille) "id unum curant, ut ex
professionibus frequentibus, et tumultuariis potius quam legitimis, commoda
sua promoverant, et ex dispendio publico suum faciant incrementum." Id solum
in votis habent annui plerumque magistratus, ut ab incipientium numero
\authorfootnote{2089}pecunias emungant, nec multum interest qui sint,
literatores an literati, modo pingues, nitidi, ad aspectum speciosi, et quod
verbo dicam, pecuniosi sint. \authorfootnote{2090}Philosophastri licentiantur
in artibus, artem qui non habent, \authorfootnote{2091}"Eosque sapientes esse
jubent, qui nulla praediti sunt sapientia, et nihil ad gradum praeterquam
velle adferunt." Theologastri (solvant modo) satis superque docti, per omnes
honorum gradus evehuntur et ascendunt. Atque hinc fit quod tam viles scurrae,
tot passim idiotae, literarum crepusculo positi, larvae pastorum,
circumforanei, vagi, barbi, fungi, crassi, asini, merum pecus in sacrosanctos
theologiae aditus, illotis pedibus irrumpant, praeter inverecundam frontem
adferentes nihil, vulgares quasdam quisquilias, et scholarium quaedam
nugamenta, indigna quae vel recipiantur in triviis. Hoc illud indignum genus
hominum et famelicum, indigum, vagum, ventris mancipium, ad stivam potius
relegandum, ad haras aptius quam ad aras, quod divinas hasce literas turpiter
prostituit; hi sunt qui pulpita complent, in aedes nobilium irrepunt, et quum
reliquis vitae destituantur subsidiis, ob corporis et animi egestatem,
aliarum in repub. partium minime capaces sint; ad sacram hanc anchoram
confugiunt, sacerdotium quovis modo captantes, non ex sinceritate, quod
\authorfootnote{2092}Paulus ait, "sed cauponantes verbum Dei." Ne quis
interim viris bonis detractum quid putet, quos habet ecclesia Anglicana
quamplurimos, eggregie doctos, illustres, intactae famae, homines, et plures
forsan quam quaevis Europae provincia; ne quis a florentisimis Academiis,
quae viros undiquaque doctissimos, omni virtutum genere suspiciendos, abunde
producunt. Et multo plures utraque habitura, multo splendidior futura, si non
hae sordes splendidum lumen ejus obfuscarent, obstaret corruptio, et
cauponantes quaedam harpyae, proletariique bonum hoc nobis non inviderent.
Nemo enim tam caeca mente, qui non hoc ipsum videat: nemo tam stolido
ingenio, qui non intelligat; tam pertinaci judicio, qui non agnoscat, ab his
idiotis circumforaneis, sacram pollui Theologiam, ac caelestes Musas quasi
prophanum quiddam prostitui. "Viles animae et effrontes" (sic enim Lutherus
\authorfootnote{2093}alicubi vocat) "lucelli causa, ut muscae ad mulctra, ad
nobilium et heroum mensas advolant, in spem sacerdotii," cujuslibet honoris,
officii, in quamvis aulam, urbem se ingerunt, ad quodvis se ministerium
componunt.-- "Ut nervis alienis mobile lignum--Ducitur"--Hor.
\bookcite{\textlatin{Lib. \rn{II.} Sat. 7}}. \authorfootnote{2094}"offam
sequentes, psittacorum more, in praedae spem quidvis effutiunt:"
obsecundantes Parasiti \authorfootnote{2095}(Erasmus ait) "quidvis docent,
dicunt, scribunt, suadent, et contra conscientiam probant, non ut salutarem
reddant gregem, sed ut magnificam sibi parent fortunam."
\authorfootnote{2096}"Opiniones quasvis et decreta contra verbum Dei
astruunt, ne non offendant patronum, sed ut retineant favorem procerum, et
populi plausum, sibique ipsis opes accumulent." Eo etenim plerunque animo ad
Theologiam accedunt, non ut rem divinam, sed ut suam facient; non ad
Ecclesiae bonum promovendum, sed expilandum; quaerentes, quod Paulus ait,
"non quae Jesu Christi, sed quae sua," non domini thesaurum, sed ut sibi,
suisque thesaurizent. Nec tantum iis, qui vilirrie fortunae, et abjectae,
sortis sunt, hoc in usu est: sed et medios, summos elatos, ne dicam
Episcopos, hoc malum invasit. \authorfootnote{2097}"Dicite pontifices, in
sacris quid facit aurum?" \authorfootnote{2098}"summos saepe viros
transversos agit avaritia," et qui reliquis morum probitate praelucerent; hi
facem praeferunt ad Simoniam, et in corruptionis hunc scopulum impingentes,
non tondent pecus, sed deglubunt, et quocunque se conferunt, expilant,
exhauriunt, abradunt, magnum famae suae, si non animae naufragium facientes;
ut non ab infimis ad summos, sed a summis ad infimos malum promanasse
videatur, et illud verum sit quod ille olim lusit, "emerat ille prius,
vendere jure potest. Simoniacus enim" (quod cum Leone dicam) "gratiam non
accepit, si non accipit, non habet, et si non habet, nec gratus potest esse;"
tantum enim absunt istorum nonnulli, qui ad clavum sedent a promovendo
reliquos, ut penitus impediant, probe sibi conscii, quibus artibus illic
pervenerint. \authorfootnote{2099}"Nam qui ob literas emersisse illos credat,
desipit; qui vero ingenii, eruditionis, experientiae, probitatis, pietatis,
et Musarum id esse pretium putat" (quod olim revera fuit, hodie promittitur)
"planissime insanit." Utcunque vel undecunque malum hoc originem ducat, non
ultra quaeram, ex his primordiis caepit vitiorum colluvies, omnis calamitas,
omne miseriarum agmen in Ecclesiam invehitur. Hinc tam frequens simonia, hinc
ortae querelae, fraudes, imposturae, ab hoc fonte se derivarunt omnes
nequitiae. Ne quid obiter dicam de ambitione, adulatione plusquam aulica, ne
tristi domicaenio laborent, de luxu, de foedo nonnunquam vitae exemplo, quo
nonnullos offendunt, de compotatione Sybaritica, \etc{} hinc ille squalor
academicus, "tristes hac tempestate Camenae," quum quivis homunculus artium
ignarus, hic artibus assurgat, hunc in modum promoveatur et ditescat,
ambitiosis appellationibus insignis, et multis dignitatibus augustus vulgi
oculos perstringat, bene se habeat, et grandia gradiens majestatem quandam ac
amplitudinem prae se ferens, miramque sollicitudinem, barba reverendus, toga
nitidus, purpura coruscus, supellectilis splendore, et famulorum numero
maxime conspicuus. "Quales statuae" (quod ait \authorfootnote{2100}ille)
"quae sacris in aedibus columnis imponuntur, velut oneri cedentes videntur,
ac si insudarent, quum revera sensu sint carentes, et nihil saxeam adjuvent
firmitatem:" atlantes videri volunt, quum sint statuae lapideae, umbratiles
revera homunciones, fungi, forsan et bardi, nihil a saxo differentes. Quum
interim docti viri, et vilae sanctioris ornamentis praediti, qui aestum diei
sustinent, his iniqua sorte serviant, minimo forsan salario contenti, puris
nominibus nuncupati, humiles, obscuri, multoque digniores licet, egentes,
inhonorati vitam privam privatam agant, tenuique sepulti sacerdotio, vel in
collegiis suis in aeternum incarcerati, inglorie delitescant. Sed nolo
diutius hanc movere sentinam, hinc illae lachrymae, lugubris musarum habitus,
\authorfootnote{2101}hinc ipsa religio (quod cum Secellio dicam) "in
ludibrium et contemptum adducitur," abjectum sacerdotium (atque haec ubi
fiunt, ausim dicere, et pulidum \authorfootnote{2102}putidi dicterium de clero
usurpare) "putidum vulgus," inops, rude, sordidum, melancholicum, miserum,
despicabile, contemnendum.\authorfootnote{2103}
\end{latin}

\subsection{A note}

As for ourselves (for neither are we free from this fault) the same guilt, the
same crime, may be objected against us: for it is through our fault,
negligence, and avarice, that so many and such shameful corruptions occur in
the church (both the temple and the Deity are offered for sale), that such
sordidness is introduced, such impiety committed, such wickedness, such a mad
gulf of wretchedness and irregularity-these I say arise from all our faults,
but more particularly from ours of the University. We are the nursery in which
those ills are bred with which the state is afflicted; we voluntarily introduce
them, and are deserving of every opprobrium and suffering, since we do not
afterwards encounter them according to our strength. For what better can we
expect when so many poor, beggarly fellows, men of every order, are readily and
without election, admitted to degrees? Who, if they can only commit to memory a
few definitions and divisions, and pass the customary period in the study of
logics, no matter with what effect, whatever sort they prove to be, idiots,
triflers, idlers, gamblers, sots, sensualists, --mere ciphers in the book of
life Like those who boldly woo'd Ulysses' wife; Born to consume the fruits of
earth: in truth, As vain and idle as Pheacia's youth; only let them have passed
the stipulated period in the University, and professed themselves collegians:
either for the sake of profit, or through the influence of their friends, they
obtain a presentation; nay, sometimes even accompanied by brilliant eulogies
upon their morals and acquirements; and when they are about to take leave, they
are honoured with the most flattering literary testimonials in their favour, by
those who undoubtedly sustain a loss of reputation in granting them. For
doctors and professors (as an author says) are anxious about one thing only,
\lit{viz.}{namely}, that out of their various callings they may promote their own advantage,
and convert the public loss into their private gains. For our annual officers
wish this only, that those who commence, whether they are taught or untaught is
of no moment, shall be sleek, fat, pigeons, worth the plucking. The
Philosophastic are admitted to a degree in Arts, because they have no
acquaintance with them. And they are desired to be wise men, because they are
endowed with no wisdom, and bring no qualification for a degree, except the
wish to have it. The Theologastic (only let them pay) thrice learned, are
promoted to every academic honour. Hence it is that so many vile buffoons, so
many idiots everywhere, placed in the twilight of letters, the mere ghosts of
scholars, wanderers in the market place, vagrants, barbels, mushrooms, dolts,
asses, a growling herd, with unwashed feet, break into the sacred precincts of
theology, bringing nothing along with them but an impudent front, some vulgar
trifles and foolish scholastic technicalities, unworthy of respect even at the
crossing of the highways. This is the unworthy, vagrant, voluptuous race,
fitter for the hog sty (haram) than the altar (aram), that basely prostitute
divine literature; these are they who fill the pulpits, creep into the palaces
of our nobility after all other prospects of existence fail them, owing to
their imbecility of body and mind, and their being incapable of sustaining any
other parts in the commonwealth; to this sacred refuge they fly, undertaking
the office of the ministry, not from sincerity, but as St. Paul says,
huckstering the word of God.

Let not any one suppose that it is here intended to detract from those many
exemplary men of which the Church of England may boast, learned, eminent, and
of spotless fame, for they are more numerous in that than in any other church
of Europe: nor from those most learned universities which constantly send forth
men endued with every form of virtue. And these seminaries would produce a
still greater number of inestimable scholars hereafter if sordidness did not
obscure the splendid light, corruption interrupt, and certain truckling harpies
and beggars envy them their usefulness.

Nor can any one be so blind as not to perceive this-any so stolid as not to
understand it-any so perverse as not to acknowledge how sacred Theology has
been contaminated by those notorious idiots, and the celestial Muse treated
with profanity.

Vile and shameless souls (says Luther) for the sake of gain, like flies to a
milk-pail, crowd round the tables of the nobility in expectation of a church
living, any office, or honour, and flock into any public hall or city ready to
accept of any employment that may offer. A thing of wood and wires by others
played.

Following the paste as the parrot, they stutter out anything in hopes of
reward: obsequious parasites, says Erasmus, teach, say, write, admire, approve,
contrary to their conviction, anything you please, not to benefit the people
but to improve their own fortunes. They subscribe to any opinions and decisions
contrary to the word of God, that they may not offend their patron, but retain
the favour of the great, the applause of the multitude, and thereby acquire
riches for themselves; for they approach Theology, not that they may perform a
sacred duty, but make a fortune: nor to promote the interests of the church,
but to pillage it: seeking, as Paul says, not the things which are of Jesus
Christ, but what may be their own: not the treasure of their Lord, but the
enrichment of themselves and their followers. Nor does this evil belong to
those of humbler birth and fortunes only, it possesses the middle and higher
ranks, \emph{bishops excepted}. O Pontiffs, tell the efficacy of gold in sacred
matters! Avarice often leads the highest men astray, and men, admirable in all
other respects: these find a salvo for simony; and, striking against this rock
of corruption, they do not shear but flay the flock; and, wherever they teem,
plunder, exhaust, raze, making shipwreck of their reputation, if not of their
souls also. Hence it appears that this malady did not flow from the humblest to
the highest classes, but \emph{vice versa}, so that the maxim is true although
spoken in jest-he bought first, therefore has the best right to sell. For a
Simoniac (that I may use the phraseology of Leo) has not received a favour;
since he has not received one he does not possess one; and since he does not
possess one he cannot confer one. So far indeed are some of those who are
placed at the helm from promoting others, that they completely obstruct them,
from a consciousness of the means by which themselves obtained the honour. For
he who imagines that they emerged from their obscurity through their learning,
is deceived; indeed, whoever supposes promotion to be the reward of genius,
erudition, experience, probity, piety, and poetry (which formerly was the case,
but nowadays is only promised) is evidently deranged.

How or when this malady commenced, I shall not further inquire; but from these
beginnings, this accumulation of vices, all her calamities and miseries have
been brought upon the Church; hence such frequent acts of simony, complaints,
fraud, impostures- from this one fountain spring all its conspicuous
iniquities. I shall not press the question of ambition and courtly flattery,
lest they may be chagrined about luxury, base examples of life, which offend
the honest, wanton drinking parties, \&c. Yet; hence is that academic squalor,
the muses now look sad, since every low fellow ignorant of the arts, by those
very arts rises, is promoted, and grows rich, distinguished by ambitious
titles, and puffed up by his numerous honours; he just shows himself to the
vulgar, and by his stately carriage displays a species of majesty, a remarkable
solicitude, letting down a flowing beard, decked in a brilliant toga
resplendent with purple, and respected also on account of the splendour of his
household and number of his servants. There are certain statues placed in
sacred edifices that seem to sink under their load, and almost to perspire,
when in reality they are void of sensation, and do not contribute to the stony
stability, so these men would wish to look like Atlases, when they are no
better than statues of stone, insignificant scrubs, funguses, dolts, little
different from stone. Meanwhile really learned men, endowed with all that can
adorn a holy life, men who have endured the heat of mid-day, by some unjust lot
obey these, dizzards, content probably with a miserable salary, known by honest
appellations, humble, obscure, although eminently worthy, needy, leading a
private life without honour, buried alive in some poor benefice, or
incarcerated for ever in their college chambers, lying hid ingloriously.

But I am unwilling to stir this sink any longer or any deeper; hence those
tears, this melancholy habit of the muses; hence (that I may speak with
Secellius) is it that religion is brought into disrepute and contempt, and the
priesthood abject; (and since this is so, I must speak out and use a filthy
witticism of the filthy) a foetid crowd, poor, sordid, melancholy, miserable,
despicable, contemptible.

%\chapter{ MEMB. IV.} SECT. II. MEMB. IV. SECT. II. MEMB. IV. SUBSECT.
%I-_Non-necessary, remote, outward, adventitious, or accidental causes: as
%first from the Nurse_.
\section[Remote or accidental causes]{Non-necessary, remote, outward,
adventitious, or accidental causes: as first from the Nurse.}

\lettrine{O}{f} those remote, outward, ambient, necessary causes, I have
sufficiently discoursed in the precedent member, the non-necessary follow; of
which, saith \authorfootnote{2104}Fuchsius, no art can be made, by reason of
their uncertainty, casualty, and multitude; so called "not necessary" because
according to \authorfootnote{2105}Fernelius, "they may be avoided, and used
without necessity." Many of these accidental causes, which I shall entreat of
here, might have well been reduced to the former, because they cannot be
avoided, but fatally happen to us, though accidentally, and unawares, at some
time or other; the rest are contingent and inevitable, and more properly
inserted in this rank of causes. To reckon up all is a thing impossible; of
some therefore most remarkable of these contingent causes which produce
melancholy, I will briefly speak and in their order.

From a child's nativity, the first ill accident that can likely befall him in
this kind is a bad nurse, by whose means alone he may be tainted with this
\authorfootnote{2106}malady from his cradle, Aulus Gellius
\bookcite{\textlatin{l. 12. c. 1.}} brings in Phavorinus, that eloquent
philosopher, proving this at large, \authorfootnote{2107}"that there is the
same virtue and property in the milk as in the seed, and not in men alone, but
in all other creatures; he gives instance in a kid and lamb, if either of them
suck of the other's milk, the lamb of the goat's, or the kid of the ewe's, the
wool of the one will be hard, and the hair of the other soft." Giraldus
Cambrensis \bookcite{\textlatin{Itinerar. Cambriae, l. 1. c. 2.}} confirms this
by a notable example which happened in his time. A sow-pig by chance sucked a
brach, and when she was grown \authorfootnote{2108}"would miraculously hunt all
manner of deer, and that as well, or rather better, than any ordinary hound."
His conclusion is, \authorfootnote{2109}"that men and beasts participate of her
nature and conditions by whose milk they are fed." Phavorinus urges it farther,
and demonstrates it more evidently, that if a nurse be
\authorfootnote{2110}"misshapen, unchaste, dishonest, impudent,
\authorfootnote{2111}cruel, or the like, the child that sucks upon her breast
will be so too;" all other affections of the mind and diseases are almost
engrafted, as it were, and imprinted into the temperature of the infant, by the
nurse's milk; as pox, leprosy, melancholy, \etc{} Cato for some such reason
would make his servants' children suck upon his wife's breast, because by that
means they would love him and his the better, and in all likelihood agree with
them. A more evident example that the minds are altered by milk cannot be
given, than that of \authorfootnote{2112}Dion, which he relates of Caligula's
cruelty; it could neither be imputed to father nor mother, but to his cruel
nurse alone, that anointed her paps with blood still when he sucked, which made
him such a murderer, and to express her cruelty to a hair: and that of
Tiberius, who was a common drunkard, because his nurse was such a one. \li{Et
si delira fuerit} (\authorfootnote{2113}one observes) \li{infantulum delirum
faciet}, if she be a fool or dolt, the child she nurseth will take after her,
or otherwise be misaffected; which Franciscus Barbarus \bookcite{\textlatin{l.
2. c. ult. de re uxoria}} proves at full, and Ant. Guivarra,
\bookcite{\textlatin{lib. 2. de Marco Aurelio}}: the child will surely
participate. For bodily sickness there is no doubt to be made. Titus,
Vespasian's son, was therefore sickly, because the nurse was so, Lampridius.
And if we may believe physicians, many times children catch the pox from a bad
nurse, Botaldus \bookcite{\textlatin{cap. 61. de lue vener.}} Besides evil
attendance, negligence, and many gross inconveniences, which are incident to
nurses, much danger may so come to the child. \authorfootnote{2114}For these
causes \Aristotle{} \bookcite{\textlatin{Polit. lib. 7. c. 17.}} Phavorinus and
Marcus Aurelius would not have a child put to nurse at all, but every mother to
bring up her own, of what condition soever she be; for a sound and able mother
to put out her child to nurse, is \li{naturae intemperies}, so
\authorfootnote{2115}Guatso calls it, 'tis fit therefore she should be nurse
herself; the mother will be more careful, loving, and attendant, than any
servile woman, or such hired creatures; this all the world acknowledgeth,
\li{convenientissimum est} (as Rod. a Castro \bookcite{\textlatin{de nat.
mulierum. lib. 4. c. 12.}} in many words confesseth) \li{matrem ipsam lactare
infantem}, "It is most fit that the mother should suckle her own infant"--who
denies that it should be so?--and which some women most curiously observe;
amongst the rest, \authorfootnote{2116}that queen of France, a Spaniard by
birth, that was so precise and zealous in this behalf, that when in her absence
a strange nurse had suckled her child, she was never quiet till she had made
the infant vomit it up again. But she was too jealous. If it be so, as many
times it is, they must be put forth, the mother be not fit or well able to be a
nurse, I would then advise such mothers, as \authorfootnote{2117}\Plutarch{} doth
in his book \bookcite{\textlatin{de liberis educandis}} and
\authorfootnote{2118}S. Hierom, \bookcite{\textlatin{li. 2. epist. 27. Laetae
de institut. fil. Magninus part 2. Reg. sanit. cap. 7.}} and the said
Rodericus, that they make choice of a sound woman, of a good complexion,
honest, free from bodily diseases, if it be possible, all passions and
perturbations of the mind, as sorrow, fear, grief, \authorfootnote{2119}folly,
melancholy. For such passions corrupt the milk, and alter the temperature of
the child, which now being \authorfootnote{2120}\li{Udum et molle lutum}, "a
moist and soft clay," is easily seasoned and perverted. And if such a nurse may
be found out, that will be diligent and careful withal, let Phavorinus and M.Aurelius plead how they can against it, I had rather accept of her in some
cases than the mother herself, and which Bonacialus the physician, Nic. Biesius
the politician, \bookcite{\textlatin{lib. 4. de repub. cap. 8.}} approves,
\authorfootnote{2121}"Some nurses are much to be preferred to some mothers."
For why may not the mother be naught, a peevish drunken flirt, a waspish
choleric slut, a crazed piece, a fool (as many mothers are), unsound as soon as
the nurse? There is more choice of nurses than mothers; and therefore except
the mother be most virtuous, staid, a woman of excellent good parts, and of a
sound complexion, I would have all children in such cases committed to discreet
strangers. And 'tis the only way; as by marriage they are engrafted to other
families to alter the breed, or if anything be amiss in the mother, as
Ludovicus Mercatus contends, \bookcite{\textlatin{Tom. 2. lib. de morb.
haered.}} to prevent diseases and future maladies, to correct and qualify the
child's ill-disposed temperature, which he had from his parents. This is an
excellent remedy, if good choice be made of such a nurse.

%SECT. II. MEMB. IV. SUBSECT. II.-_Education a Cause of Melancholy_.
\section{Education a Cause of Melancholy.}

\lettrine{E}{ducation}, of these accidental causes of melancholy, may justly
challenge the next place, for if a man escape a bad nurse, he may be undone by
evil bringing up. \authorfootnote{2122}Jason Pratensis puts this of education
for a principal cause; bad parents, stepmothers, tutors, masters, teachers, too
rigorous, too severe, too remiss or indulgent on the other side, are often
fountains and furtherers of this disease. Parents and such as have the tuition
and oversight of children, offend many times in that they are too stern, always
threatening, chiding, brawling, whipping, or striking; by means of which their
poor children are so disheartened and cowed, that they never after have any
courage, a merry hour in their lives, or take pleasure in anything. There is a
great moderation to be had in such things, as matters of so great moment to the
making or marring of a child. Some fright their children with beggars,
bugbears, and hobgoblins, if they cry, or be otherwise unruly: but they are
much to blame in it, many times, saith Lavater, \bookcite{\textlatin{de
spectris, part. 1, cap. 5.}} \li{ex metu in morbos graves incidunt et noctu
dormientes clamant}, for fear they fall into many diseases, and cry out in
their sleep, and are much the worse for it all their lives: these things ought
not at all, or to be sparingly done, and upon just occasion. Tyrannical,
impatient, hair-brain schoolmasters, \li{aridi magistri}, so
\authorfootnote{2123}Fabius terms them, \li{Ajaces flagelliferi}, are in this
kind as bad as hangmen and executioners, they make many children endure a
martyrdom all the while they are at school, with bad diet, if they board in
their houses, too much severity and ill-usage, they quite pervert their
temperature of body and mind: still chiding, railing, frowning, lashing,
tasking, keeping, that they are \li{fracti animis}, moped many times, weary of
their lives, \authorfootnote{2124}\li{nimia severitate deficiunt et desperant},
and think no slavery in the world (as once I did myself) like to that of a
grammar scholar. \li{Praeceptorum ineptiis discruciantur ingenia
puerorum}\authorlatintrans{2125}, saith Erasmus, they tremble at his voice,
looks, coming in. St. \idxname{austin}[Austin][\textlatin{confess.}], in the first book of his
\bookcite{\textlatin{confess. et 4 ca.}} calls this schooling \li{meliculosam
necessitatem}, and elsewhere a martyrdom, and confesseth of himself, how
cruelly he was tortured in mind for learning Greek, \li{nulla verba noveram, et
saevis terroribus et poenis, ut nossem, instabatur mihi vehementer}, I know
nothing, and with cruel terrors and punishment I was daily compelled.
\authorfootnote{2126}Beza complains in like case of a rigorous schoolmaster in
Paris, that made him by his continual thunder and threats once in a mind to
drown himself, had he not met by the way with an uncle of his that vindicated
him from that misery for the time, by taking him to his house. Trincavellius,
\bookcite{\textlatin{lib. 1. consil. 16.}} had a patient nineteen years of age,
extremely melancholy, \li{ob nimium studium, Tarvitii et praeceptoris minas},
by reason of overmuch study, and his \authorfootnote{2127}tutor's threats. Many
masters are hard-hearted, and bitter to their servants, and by that means do so
deject, with terrible speeches and hard usage so crucify them, that they become
desperate, and can never be recalled.

Others again, in that opposite extreme, do as great harm by their too much
remissness, they give them no bringing up, no calling to busy themselves about,
or to live in, teach them no trade, or set them in any good course; by means of
which their servants, children, scholars, are carried away with that stream of
drunkenness, idleness, gaming, and many such irregular courses, that in the end
they rue it, curse their parents, and mischief themselves. Too much indulgence
causeth the like, \authorfootnote{2128}\li{inepta patris lenitas et facilitas
prava}, when as Mitio-like, with too much liberty and too great allowance, they
feed their children's humours, let them revel, wench, riot, swagger, and do
what they will themselves, and then punish them with a noise of musicians;

\translatedverse{%
\begin{latin}
\begin{verse}%
Obsonet, potet, oleat unguenta de meo;\\*
Amat? dabitur a me argentum ubi erit commodum.\\*
Fores effregit? restituentur: descidit\\*
Vestem? resarcietur.--Faciat quod lubet,\\*
Sumat, consumat, perdat, decretum est pati.\\!
\end{verse}%
\end{latin}}{%%\authorlatintrans{2129.5}
\begin{verse}
Let him feast, drink, perfume himself at my expense:\\*
If he be in love, I shall supply him with money.\\*
Has he broken in the gates? they shall be repaired.\\*
Has he torn his garments? they shall be replaced.\\*
Let him do what he pleases, take, spend, waste, I am resolved to submit.
\end{verse}}{}

But as Demeo told him, \lit{tu illum corrumpi sinis}{your lenity will be his
undoing}, \li{praevidere videor jam diem, illum, quum hic egens profugiet
aliquo militatum}, I foresee his ruin. So parents often err, many fond mothers
especially, dote so much upon their children, like \authorfootnote{2130}Aesop's
ape, till in the end they crush them to death, \li{Corporum nutrices animarum
novercae}, pampering up their bodies to the undoing of their souls: they will
not let them be \authorfootnote{2131}corrected or controlled, but still soothed
up in everything they do, that in conclusion "they bring sorrow, shame,
heaviness to their parents" (\biblecite{Ecclus. cap. \rn{xxx.} 8, 9}), "become
wanton, stubborn, wilful, and disobedient;" rude, untaught, headstrong,
incorrigible, and graceless; "they love them so foolishly," saith
\authorfootnote{2132}Cardan, "that they rather seem to hate them, bringing them
not up to virtue but injury, not to learning but to riot, not to sober life and
conversation, but to all pleasure and licentious behaviour." Who is he of so
little experience that knows not this of Fabius to be true?
\authorfootnote{2133}"Education is another nature, altering the mind and will,
and I would to God" (saith he) "we ourselves did not spoil our children's
manners, by our overmuch cockering and nice education, and weaken the strength
of their bodies and minds, that causeth custom, custom nature," \etc{} For
these causes \Plutarch{} in his book \bookcite{\textlatin{de lib. educ.}} and
Hierom. \bookcite{\textlatin{epist. lib. 1. epist. 17. to Laeta de institut.
filiae}}, gives a most especial charge to all parents, and many good cautions
about bringing up of children, that they be not committed to indiscreet,
passionate, bedlam tutors, light, giddy-headed, or covetous persons, and spare
for no cost, that they may be well nurtured and taught, it being a matter of so
great consequence. For such parents as do otherwise, \Plutarch{} esteems of them
\authorfootnote{2134}"that are more careful of their shoes than of their feet,"
that rate their wealth above their children. And he, saith
\authorfootnote{2135}Cardan, "that leaves his son to a covetous schoolmaster to
be informed, or to a close Abbey to fast and learn wisdom together, doth no
other, than that he be a learned fool, or a sickly wise man."

%SECT. II. MEMB. IV. SUBSECT. III.-_Terrors and Affrights, Causes of
%Melancholy_.
\section{Terrors and Affrights, Causes of
Melancholy.}\label{sec:terrors-and-affrights}

\lettrine{T}{ully}, in the fourth of his Tusculans, distinguishes these terrors
which arise from the apprehension of some terrible object heard or seen, from
other fears, and so doth Patritius \bookcite{\textlatin{lib. 5. Tit. 4. de
regis institut.}} Of all fears they are most pernicious and violent, and so
suddenly alter the whole temperature of the body, move the soul and spirits,
strike such a deep impression, that the parties can never be recovered, causing
more grievous and fiercer melancholy, as Felix Plater, \bookcite{\textlatin{c.
3. de mentis alienat}}. \authorfootnote{2136}speaks out of his experience, than
any inward cause whatsoever: "and imprints itself so forcibly in the spirits,
brain, humours, that if all the mass of blood were let out of the body, it
could hardly be extracted. This horrible kind of melancholy" (for so he terms
it) "had been often brought before him, and troubles and affrights commonly men
and women, young and old of all sorts." \authorfootnote{2137}Hercules de
Saxonia calls this kind of melancholy (\li{ab agitatione spirituum}) by a
peculiar name, it comes from the agitation, motion, contraction, dilatation of
spirits, not from any distemperature of humours, and produceth strong effects.
This terror is most usually caused, as \authorfootnote{2138}\Plutarch{} will have,
"from some imminent danger, when a terrible object is at hand," heard, seen, or
conceived, \authorfootnote{2139}"truly appearing, or in a
\authorfootnote{2140}dream:" and many times the more sudden the accident, it is
the more violent.

\translatedverse{%
\begin{latin}
\begin{verse}%
Stat terror animis, et cor attonitum salit,\\*
Pavidumque trepidis palpitat venis jecur.\\!
\end{verse}%
\end{latin}}{%
\begin{verse}%
Their soul's affright, their heart amazed quakes,\\*
The trembling liver pants i' th' veins, and aches.\\!
\end{verse}}{%
\attrib{\getauthornote{2141}}}

Arthemedorus the grammarian lost his wits by the unexpected sight of a
crocodile, Laurentius \bookcite{\textlatin{7. de melan}}.
\authorfootnote{2142}The massacre at Lyons, 1572, in the reign of Charles \rn{IX.},
was so terrible and fearful, that many ran mad, some died, great-bellied women
were brought to bed before their time, generally all affrighted aghast. Many
lose their wits \authorfootnote{2143}"by the sudden sight of some spectrum or
devil, a thing very common in all ages," saith Lavater
\bookcite{\textlatin{part 1. cap. 9.}} as Orestes did at the sight of the
Furies, which appeared to him in black (as \authorfootnote{2144}Pausanias
records). The Greeks call them \textgreek{μορμολύχεια}, which so terrify their
souls, or if they be but affrighted by some counterfeit devils in jest,

\begin{latin}
\begin{verse}%
------ut pueri trepidant, atque omnia caecis\\*
In tenebris metuunt------\\!
\end{verse}%
\end{latin}
\attrib{\getauthornote{2145}}
as children in the dark conceive hobgoblins, and are so afraid, they are the
worse for it all their lives. Some by sudden fires, earthquakes, inundations,
or any such dismal objects: Themiscon the physician fell into a hydrophobia, by
seeing one sick of that disease: (Dioscorides \bookcite{\textlatin{l. 6. c.
33.}}) or by the sight of a monster, a carcase, they are disquieted many months
following, and cannot endure the room where a corpse hath been, for a world
would not be alone with a dead man, or lie in that bed many years after in
which a man hath died. At \authorfootnote{2146}Basil many little children in
the springtime went to gather flowers in a meadow at the town's end, where a
malefactor hung in gibbets; all gazing at it, one by chance flung a stone, and
made it stir, by which accident, the children affrighted ran away; one slower
than the rest, looking back, and seeing the stirred carcase wag towards her,
cried out it came after, and was so terribly affrighted, that for many days she
could not rest, eat, or sleep, she could not be pacified, but melancholy, died.
\authorfootnote{2147}In the same town another child, beyond the Rhine, saw a
grave opened, and upon the sight of a carcase, was so troubled in mind that she
could not be comforted, but a little after departed, and was buried by it.
Platerus \bookcite{\textlatin{observat. l. 1}}, a gentlewoman of the same city
saw a fat hog cut up, when the entrails were opened, and a noisome savour
offended her nose, she much misliked, and would not longer abide: a physician
in presence, told her, as that hog, so was she, full of filthy excrements, and
aggravated the matter by some other loathsome instances, insomuch, this nice
gentlewoman apprehended it so deeply, that she fell forthwith a-vomiting, was
so mightily distempered in mind and body, that with all his art and
persuasions, for some months after, he could not restore her to herself again,
she could not forget it, or remove the object out of her sight, \li{Idem}. Many
cannot endure to see a wound opened, but they are offended: a man executed, or
labour of any fearful disease, as possession, apoplexies, one bewitched;
\authorfootnote{2148}or if they read by chance of some terrible thing, the
symptoms alone of such a disease, or that which they dislike, they are
instantly troubled in mind, aghast, ready to apply it to themselves, they are
as much disquieted as if they had seen it, or were so affected themselves.
\li{Hecatas sibi videntur somniare}, they dream and continually think of it. As
lamentable effects are caused by such terrible objects heard, read, or seen,
\li{auditus maximos motus in corpore facit}, as \authorfootnote{2149}\Plutarch{}
holds, no sense makes greater alteration of body and mind: sudden speech
sometimes, unexpected news, be they good or bad, \li{praevisa minus oratio},
will move as much, \li{animum obruere, et de sede sua dejicere}, as a
\authorfootnote{2150}philosopher observes, will take away our sleep and
appetite, disturb and quite overturn us. Let them bear witness that have heard
those tragical alarms, outcries, hideous noises, which are many times suddenly
heard in the dead of the night by irruption of enemies and accidental fires,
\etc{}, those \authorfootnote{2151}panic fears, which often drive men out of
their wits, bereave them of sense, understanding and all, some for a time, some
for their whole lives, they never recover it. The
\authorfootnote{2152}Midianites were so affrighted by Gideon's soldiers, they
breaking but every one a pitcher; and \authorfootnote{2153}Hannibal's army by
such a panic fear was discomfited at the walls of Rome. Augusta Livia hearing a
few tragical verses recited out of \Virgil{}, \li{Tu Marcellus eris}, \etc{}, fell
down dead in a swoon. Edinus king of Denmark, by a sudden sound which he heard,
\authorfootnote{2154}"was turned into fury with all his men," Cranzius,
\bookcite{\textlatin{l. 5, Dan. hist.}} and Alexander ab Alexandro
\bookcite{\textlatin{l. 3. c. 5.}} Amatus Lusitanus had a patient, that by
reason of bad tidings became epilepticus, \bookcite{\textlatin{cen. 2. cura
90}}, Cardan \bookcite{\textlatin{subtil. l. 18}}, saw one that lost his wits
by mistaking of an echo. If one sense alone can cause such violent commotions
of the mind, what may we think when hearing, sight, and those other senses are
all troubled at once? as by some earthquakes, thunder, lightning, tempests,
\etc{} At Bologna in Italy, \emph{anno} 1504, there was such a fearful
earthquake about eleven o'clock in the night (as \authorfootnote{2155}Beroaldus
in his book \bookcite{\textlatin{de terrae motu}}, hath commended to posterity)
that all the city trembled, the people thought the world was at an end,
\li{actum de mortalibus}, such a fearful noise, it made such a detestable
smell, the inhabitants were infinitely affrighted, and some ran mad. \li{Audi
rem atrocem, et annalibus memorandam} (mine author adds), hear a strange story,
and worthy to be chronicled: I had a servant at the same time called Fulco
Argelanus, a bold and proper man, so grievously terrified with it, that he
\authorfootnote{2156}was first melancholy, after doted, at last mad, and made
away himself. At \authorfootnote{2157}Fuscinum in Japona "there was such an
earthquake, and darkness on a sudden, that many men were offended with
headache, many overwhelmed with sorrow and melancholy. At Meacum whole streets
and goodly palaces were overturned at the same time, and there was such a
hideous noise withal, like thunder, and filthy smell, that their hair stared
for fear, and their hearts quaked, men and beasts were incredibly terrified. In
Sacai, another city, the same earthquake was so terrible unto them, that many
were bereft of their senses; and others by that horrible spectacle so much
amazed, that they knew not what they did." Blasius a Christian, the reporter of
the news, was so affrighted for his part, that though it were two months after,
he was scarce his own man, neither could he drive the remembrance of it out of
his mind. Many times, some years following, they will tremble afresh at the
\authorfootnote{2158}remembrance or conceit of such a terrible object, even all
their lives long, if mention be made of it. Cornelius Agrippa relates out of
Gulielmus Parisiensis, a story of one, that after a distasteful purge which a
physician had prescribed unto him, was so much moved,
\authorfootnote{2159}"that at the very sight of physic he would be
distempered," though he never so much as smelled to it, the box of physic long
after would give him a purge; nay, the very remembrance of it did effect it;
\authorfootnote{2160}"like travellers and seamen," saith \Plutarch{}, "that when
they have been sanded, or dashed on a rock, for ever after fear not that
mischance only, but all such dangers whatsoever."

%SECT. II. MEMB. IV. SUBSECT. IV.-_Scoffs, Calumnies, bitter Jests, how they cause Melancholy_.
\section[Scoffs, bitter Jests]{Scoffs, Calumnies, bitter Jests, how they cause Melancholy.}\label{sec:scoffs-bitter-jests}

\lettrine{I}{t} is an old saying, \authorfootnote{2161}"A blow with a word
strikes deeper than a blow with a sword:" and many men are as much galled with
a calumny, a scurrilous and bitter jest, a libel, a pasquil, satire, apologue,
epigram, stage-play or the like, as with any misfortune whatsoever. Princes and
potentates, that are otherwise happy, and have all at command, secure and free,
\li{quibus potentia sceleris impunitatem fecit}, are grievously vexed with
these pasquilling libels, and satires: they fear a railing
\authorfootnote{2162}Aretine, more than an enemy in the field, which made most
princes of his time (as some relate) "allow him a liberal pension, that he
should not tax them in his satires." \authorfootnote{2163}The Gods had their
Momus, \Homer{} his Zoilus, Achilles his Thersites, Philip his Demades: the
Caesars themselves in Rome were commonly taunted. There was never wanting a
\Petronius, a Lucian in those times, nor will be a Rabelais, an Euphormio, a
Boccalinus in ours. Adrian the sixth pope \authorfootnote{2164}was so highly
offended, and grievously vexed with pasquillers at Rome, he gave command that
his statue should be demolished and burned, the ashes flung into the river
Tiber, and had done it forthwith, had not Ludovicus Suessanus, a facete
companion, dissuaded him to the contrary, by telling him, that pasquil's ashes
would turn to frogs in the bottom of the river, and croak worse and louder than
before,-- \li{genus irritabile vatum}, and therefore
\authorfootnote{2165}Socrates in Plato adviseth all his friends, "that respect
their credits, to stand in awe of poets, for they are terrible fellows, can
praise and dispraise as they see cause." \li{Hinc quam sit calamus saevior ense
patet}. The prophet David complains, \biblecite{Psalm \rn{cxxiii.} 4.} "that his
soul was full of the mocking of the wealthy, and of the despitefulness of the
proud," and \biblecite{Psalm \rn{lv.} 4.} "for the voice of the wicked, \etc{},
and their hate: his heart trembled within him, and the terrors of death came
upon him; fear and horrible fear," \etc{}, and \biblecite{Psal. \rn{lxix.} 20.}
"Rebuke hath broken my heart, and I am full of heaviness." Who hath not like
cause to complain, and is not so troubled, that shall fall into the mouths of
such men? for many are of so \authorfootnote{2166}petulant a spleen; and have
that figure Sarcasmus so often in their mouths, so bitter, so foolish, as
\authorfootnote{2167}Balthazar Castilio notes of them, that "they cannot speak,
but they must bite;" they had rather lose a friend than a jest; and what
company soever they come in, they will be scoffing, insulting over their
inferiors, especially over such as any way depend upon them, humouring,
misusing, or putting gulleries on some or other till they have made by their
humouring or gulling \authorfootnote{2168}\li{ex stulto insanum}, a mope or a
noddy, and all to make themselves merry:

\translatedverse{%
\begin{latin}
\begin{verse}%
------dummodo risum\\*
Excutiat sibi; non hic cuiquam parcit amico;\\!
\end{verse}%
\end{latin}}{%\setauthornote{2169.5}
\begin{verse}%
Provided he can only excite laughter,\\*
he spares not his best friend.\\!
\end{verse}}{%
\attrib{\getauthornote{2169}}}

Friends, neuters, enemies, all are as one, to make a fool a madman, is their
sport, and they have no greater felicity than to scoff and deride others; they
must sacrifice to the god of laughter, with them in
\authorfootnote{2170}\Apuleius, once a day, or else they shall be melancholy
themselves; they care not how they grind and misuse others, so they may
exhilarate their own persons. Their wits indeed serve them to that sole
purpose, to make sport, to break a scurrile jest, which is \li{levissimus
ingenii fructus}, the froth of wit, as \authorfootnote{2171}\Tully{} holds, and
for this they are often applauded, in all other discourse, dry, barren,
stramineous, dull and heavy, here lies their genius, in this they alone excel,
please themselves and others. Leo Decimus, that scoffing pope, as Jovius hath
registered in the Fourth book of his life, took an extraordinary delight in
humouring of silly fellows, and to put gulleries upon them,
\authorfootnote{2172}by commending some, persuading others to this or that: he
made \li{ex stolidis stultissimos, et maxime ridiculos, ex stultis insanos};
soft fellows, stark noddies; and such as were foolish, quite mad before he left
them. One memorable example he recites there, of Tarascomus of Parma, a
musician that was so humoured by Leo Decimus, and Bibiena his second in this
business, that he thought himself to be a man of most excellent skill, (who was
indeed a ninny) they \authorfootnote{2173}"made him set foolish songs, and
invent new ridiculous precepts, which they did highly commend," as to tie his
arm that played on the lute, to make him strike a sweeter stroke,
\authorfootnote{2174}"and to pull down the arras hangings, because the voice
would be clearer, by reason of the reverberation of the wall." In the like
manner they persuaded one Baraballius of Caieta, that he was as good a poet as
Petrarch; would have him to be made a laureate poet, and invite all his friends
to his instalment; and had so possessed the poor man with a conceit of his
excellent poetry, that when some of his more discreet friends told him of his
folly, he was very angry with them, and said \authorfootnote{2175}"they envied
his honour, and prosperity:" it was strange (saith Jovius) to see an old man of
60 years, a venerable and grave old man, so gulled. But what cannot such
scoffers do, especially if they find a soft creature, on whom they may work?
nay, to say truth, who is so wise, or so discreet, that may not be humoured in
this kind, especially if some excellent wits shall set upon him; he that mads
others, if he were so humoured, would be as mad himself, as much grieved and
tormented; he might cry with him in the comedy, \li{Proh Jupiter tu homo me,
adigas ad insaniam}. For all is in these things as they are taken; if he be a
silly soul, and do not perceive it, 'tis well, he may haply make others sport,
and be no whit troubled himself; but if he be apprehensive of his folly, and
take it to heart, then it torments him worse than any lash: a bitter jest, a
slander, a calumny, pierceth deeper than any loss, danger, bodily pain, or
injury whatsoever; \li{leviter enim volat}, (it flies swiftly) as Bernard of an
arrow, \li{sed graviter vulnerat}, (but wounds deeply), especially if it shall
proceed from a virulent tongue, "it cuts" (saith David) "like a two-edged
sword. They shoot bitter words as arrows," \biblecite{Psal. \rn{lxiv.} 5.} "And
they smote with their tongues," \biblecite{Jer. \rn{xviii.} 18}, and that so
hard, that they leave an incurable wound behind them. Many men are undone by
this means, moped, and so dejected, that they are never to be recovered; and of
all other men living, those which are actually melancholy, or inclined to it,
are most sensible, (as being suspicious, choleric, apt to mistake) and
impatient of an injury in that kind: they aggravate, and so meditate
continually of it, that it is a perpetual corrosive, not to be removed, till
time wear it out. Although they peradventure that so scoff, do it alone in
mirth and merriment, and hold it \li{optimum aliena frui insania}, an excellent
thing to enjoy another man's madness; yet they must know, that it is a mortal
sin (as \authorfootnote{2176}Thomas holds) and as the prophet
\authorfootnote{2177}David denounceth, "they that use it, shall never dwell in
God's tabernacle."

Such scurrilous jests, flouts, and sarcasms, therefore, ought not at all to be
used; especially to our betters, to those that are in misery, or any way
distressed: for to such, \li{aerumnarum incrementa sunt}, they multiply grief,
and as \authorfootnote{2178}he perceived, \li{In multis pudor, in multis
iracundia}, \etc{}, many are ashamed, many vexed, angered, and there is no
greater cause or furtherer of melancholy. Martin Cromerus, in the Sixth book of
his history, hath a pretty story to this purpose, of Vladislaus, the second
king of Poland, and Peter Dunnius, earl of Shrine; they had been hunting late,
and were enforced to lodge in a poor cottage. When they went to bed, Vladislaus
told the earl in jest, that his wife lay softer with the abbot of Shrine; he
not able to contain, replied, \li{Et tua cum Dabesso}, and yours with Dabessus,
a gallant young gentleman in the court, whom Christina the queen loved.
\li{Tetigit id dictum Principis animum}, these words of his so galled the
prince, that he was long after \li{tristis et cogitabundus}, very sad and
melancholy for many months; but they were the earl's utter undoing: for when
Christina heard of it, she persecuted him to death. Sophia the empress,
Justinian's wife, broke a bitter jest upon Narsetes the eunuch, a famous
captain then disquieted for an overthrow which he lately had: that he was
fitter for a distaff and to keep women company, than to wield a sword, or to be
general of an army: but it cost her dear, for he so far distasted it, that he
went forthwith to the adverse part, much troubled in his thoughts, caused the
Lombards to rebel, and thence procured many miseries to the commonwealth.
Tiberius the emperor withheld a legacy from the people of Rome, which his
predecessor Augustus had lately given, and perceiving a fellow round a dead
corse in the ear, would needs know wherefore he did so; the fellow replied,
that he wished the departed soul to signify to Augustus, the commons of Rome
were yet unpaid: for this bitter jest the emperor caused him forthwith to be
slain, and carry the news himself. For this reason, all those that otherwise
approve of jests in some cases, and facete companions, (as who doth not?) let
them laugh and be merry, \li{rumpantur et illa Codro}, 'tis laudable and fit,
those yet will by no means admit them in their companies, that are any way
inclined to this malady: \li{non jocandum cum iis qui miseri sunt, et
aerumnosi}, no jesting with a discontented person. 'Tis Castilio's caveat,
\authorfootnote{2179}Jo. Pontanus, and \authorfootnote{2180}Galateus, and every
good man's.

\begin{verse}%
Play with me, but hurt me not:\\*
Jest with me, but shame me not.\\!
\end{verse}%
\attrib{\getauthornote{2180}}

\li{Comitas} is a virtue between rusticity and scurrility, two extremes, as
affability is between flattery and contention, it must not exceed; but be still
accompanied with that \authorfootnote{2181}\textgreek{ἀβλάβεια} or innocency,
\li{quae nemini nocet, omnem injuriae, oblationem abhorrens}, hurts no man,
abhors all offer of injury. Though a man be liable to such a jest or obloquy,
have been overseen, or committed a foul fact, yet it is no good manners or
humanity, to upbraid, to hit him in the teeth with his offence, or to scoff at
such a one; 'tis an old axiom, \li{turpis in reum omnis
exprobratio}\authorlatintrans{2182}. I speak not of such as generally tax vice,
Barclay, Gentilis, Erasmus, Agrippa, Fishcartus, \etc{}, the Varronists and
Lucians of our time, satirists, epigrammists, comedians, apologists, \etc{},
but such as personate, rail, scoff, calumniate, perstringe by name, or in
presence offend;

\begin{latin}
\begin{verse}%
Ludit qui stolida procacitate\\*
Non est Sestius ille sed caballus:\\!
\end{verse}%
\end{latin}
\attrib{\getauthornote{2183}}

'Tis horse-play this, and those jests (as he \authorfootnote{2184}saith) "are
no better than injuries," biting jests, \li{mordentes et aculeati}, they are
poisoned jests, leave a sting behind them, and ought not to be used.

\begin{verse}%
Set not thy foot to make the blind to fall;\\*
Nor wilfully offend thy weaker brother:\\*
Nor wound the dead with thy tongue's bitter gall,\\*
Neither rejoice thou in the fall of other.\\!
\end{verse}%
\attrib{\getauthornote{2185}}

If these rules could be kept, we should have much more ease and quietness than
we have, less melancholy, whereas on the contrary, we study to misuse each
other, how to sting and gall, like two fighting boors, bending all our force
and wit, friends, fortune, to crucify \authorfootnote{2186}one another's souls;
by means of which, there is little content and charity, much virulency, hatred,
malice, and disquietness among us.

%SECT. II. MEMB. IV. SUBSECT. V.-_Loss of Liberty, Servitude, Imprisonment, how
%they cause Melancholy_.
\section[Loss of Liberty, Servitude]{Loss of Liberty, Servitude, Imprisonment,
how they cause Melancholy.}

\lettrine{T}{o} this catalogue of causes, I may well annex loss of liberty,
servitude, or imprisonment, which to some persons is as great a torture as any
of the rest. Though they have all things convenient, sumptuous houses to their
use, fair walks and gardens, delicious bowers, galleries, good fare and diet,
and all things correspondent, yet they are not content, because they are
confined, may not come and go at their pleasure, have and do what they will,
but live \authorfootnote{2187}\li{aliena quadra}, at another man's table and
command. As it is \authorfootnote{2188}in meats so it is in all other things,
places, societies, sports; let them be never so pleasant, commodious,
wholesome, so good; yet \li{omnium rerum est satietas}, there is a loathing
satiety of all things. The children of Israel were tired with manna, it is
irksome to them so to live, as to a bird in his cage, or a dog in his kennel,
they are weary of it. They are happy, it is true, and have all things, to
another man's judgment, that heart can wish, or that they themselves can
desire, \li{bona si sua norint}: yet they loathe it, and are tired with the
present: \li{Est natura hominum novitatis avida}; men's nature is still
desirous of news, variety, delights; and our wandering affections are so
irregular in this kind, that they must change, though it must be to the worst.
Bachelors must be married, and married men would be bachelors; they do not love
their own wives, though otherwise fair, wise, virtuous, and well qualified,
because they are theirs; our present estate is still the worst, we cannot
endure one course of life long, \li{et quod modo voverat, odit}, one calling
long, \li{esse in honore juvat, mox displicet}; one place long,
\authorfootnote{2189}\li{Romae Tibur amo, ventosus Tybure Romam}, that which we
earnestly sought, we now contemn. \li{Hoc quosdam agit ad mortem}, (saith
\authorfootnote{2190}\Seneca{}) \li{quod proposita saepe mutando in eadem
revolvuntur, et non relinquunt novitati locum: Fastidio caepit esse vita, et
ipsus mundus, et subit illud rapidissimarum deliciarum, Quousque eadem}? this
alone kills many a man, that they are tied to the same still, as a horse in a
mill, a dog in a wheel, they run round, without alteration or news, their life
groweth odious, the world loathsome, and that which crosseth their furious
delights, what? still the same? Marcus Aurelius and Solomon, that had
experience of all worldly delights and pleasure, confessed as much of
themselves; what they most desired, was tedious at last, and that their lust
could never be satisfied, all was vanity and affliction of mind.

Now if it be death itself, another hell, to be glutted with one kind of sport,
dieted with one dish, tied to one place; though they have all things otherwise
as they can desire, and are in heaven to another man's opinion, what misery and
discontent shall they have, that live in slavery, or in prison itself? \li{Quod
tristius morte, in servitute vivendum}, as Hermolaus told Alexander in
\authorfootnote{2191}Curtius, worse than death is bondage:
\authorfootnote{2192}\li{hoc animo scito omnes fortes, ut mortem servituti
anteponant}, All brave men at arms (\Tully{} holds) are so affected.
\authorfootnote{2193}\li{Equidem ego is sum, qui servitutem extremum omnium
malorum esse arbitror}: I am he (saith Boterus) that account servitude the
extremity of misery. And what calamity do they endure, that live with those
hard taskmasters, in gold mines (like those 30\thinspace{}000
\authorfootnote{2194}Indian slaves at Potosi, in Peru), tin-mines, lead-mines,
stone-quarries, coal-pits, like so many mouldwarps under ground, condemned to
the galleys, to perpetual drudgery, hunger, thirst, and stripes, without all
hope of delivery? How are those women in Turkey affected, that most part of the
year come not abroad; those Italian and Spanish dames, that are mewed up like
hawks, and locked up by their jealous husbands? how tedious is it to them that
live in stoves and caves half a year together? as in Iceland, Muscovy, or under
the \authorfootnote{2195}pole itself, where they have six months' perpetual
night. Nay, what misery and discontent do they endure, that are in prison? They
want all those six non-natural things at once, good air, good diet, exercise,
company, sleep, rest, ease, \etc{}, that are bound in chains all day long,
suffer hunger, and (as \authorfootnote{2196}Lucian describes it) "must abide
that filthy stink, and rattling of chains, howlings, pitiful outcries, that
prisoners usually make; these things are not only troublesome, but
intolerable." They lie nastily among toads and frogs in a dark dungeon, in
their own dung, in pain of body, in pain of soul, as Joseph did, \biblecite{Psal.
\rn{cv.} 18}, "they hurt his feet in the stocks, the iron entered his soul."
They live solitary, alone, sequestered from all company but heart-eating
melancholy; and for want of meat, must eat that bread of affliction, prey upon
themselves. Well might \authorfootnote{2197}Arculanus put long imprisonment for
a cause, especially to such as have lived jovially, in all sensuality and lust,
upon a sudden are estranged and debarred from all manner of pleasures: as were
Huniades, Edward, and Richard \rn{II.}, Valerian the Emperor, Bajazet the Turk. If
it be irksome to miss our ordinary companions and repast for once a day, or an
hour, what shall it be to lose them for ever? If it be so great a delight to
live at liberty, and to enjoy that variety of objects the world affords; what
misery and discontent must it needs bring to him, that shall now be cast
headlong into that Spanish inquisition, to fall from heaven to hell, to be
cubbed up upon a sudden, how shall he be perplexed, what shall become of him?
\authorfootnote{2198}Robert Duke of Normandy being imprisoned by his youngest
brother Henry \rn{I.}, \li{ab illo die inconsolabili dolore in carcere contabuit},
saith Matthew Paris, from that day forward pined away with grief.
\authorfootnote{2199}Jugurtha that generous captain, "brought to Rome in
triumph, and after imprisoned, through anguish of his soul, and melancholy,
died." \authorfootnote{2200}Roger, Bishop of Salisbury, the second man from
King Stephen (he that built that famous castle of \authorfootnote{2201}Devizes
in Wiltshire,) was so tortured in prison with hunger, and all those calamities
accompanying such men, \authorfootnote{2202}\li{ut vivere noluerit, mori
nescierit}, he would not live, and could not die, between fear of death, and
torments of life. Francis King of France was taken prisoner by Charles \rn{V.},
\li{ad mortem fere melancholicus}, saith Guicciardini, melancholy almost to
death, and that in an instant. But this is as clear as the sun, and needs no
further illustration.


\begin{figure}[H]
  \begingroup
  \centering
  \includegraphics[keepaspectratio,width=0.8\textwidth]{Bright-sun-Albert-Flamen-small.jpg}
  \captionart{BrightSun}
  \label{fig:brightsun}
\end{figure}

\cleartoleftpage{}
\begin{figure}[p]
  \begingroup
  \centering
  \includegraphics[keepaspectratio,width=\textwidth]{carrying-povery-small.jpg}
  \captionart{PovertyRiches}
  \label{fig:povertyriches}
\end{figure}

% Force float here
\clearpage{}
\thispagestyle{titleontop}
%SECT. II. MEMB. IV SUBSECT. VI.-_Poverty and Want, Causes of Melancholy_.
\section{Poverty and Want, Causes of Melancholy.}\label{sec:poverty-and-want}

\lettrine{P}{overty} and want are so violent oppugners, so unwelcome guests, so
much abhorred of all men, that I may not omit to speak of them apart. Poverty,
although (if considered aright, to a wise, understanding, truly regenerate, and
contented man) it be \li{donum Dei}, a blessed estate, the way to heaven, as
\authorfootnote{2203}\Chrysostom{} calls it, God's gift, the mother of modesty,
and much to be preferred before riches (as shall be shown in his
\authorfootnote{2204}place), yet as it is esteemed in the world's censure, it
is a most odious calling, vile and base, a severe torture, \li{summum scelus},
a most intolerable burden; we \authorfootnote{2205}shun it all, \lit{cane pejus
et angue}{worse than a dog or a snake}, we abhor the name of it,
\authorfootnote{2206}\li{Paupertas fugitur, totoque arcessitur orbe}, as being
the fountain of all other miseries, cares, woes, labours, and grievances
whatsoever. To avoid which, we will take any pains,-- \li{extremos currit
mercator ad Indos}, we will leave no haven, no coast, no creek of the world
unsearched, though it be to the hazard of our lives, we will dive to the bottom
of the sea, to the bowels of the earth, \authorfootnote{2207}five, six, seven,
eight, nine hundred fathom deep, through all five zones, and both extremes of
heat and cold: we will turn parasites and slaves, prostitute ourselves, swear
and lie, damn our bodies and souls, forsake God, abjure religion, steal, rob,
murder, rather than endure this insufferable yoke of poverty, which doth so
tyrannise, crucify, and generally depress us.

For look into the world, and you shall see men most part esteemed according to
their means, and happy as they are rich: \authorfootnote{2208}\li{Ubique tanti
quisque quantum habuit fuit}. If he be likely to thrive, and in the way of
preferment, who but he? In the vulgar opinion, if a man be wealthy, no matter
how he gets it, of what parentage, how qualified, how virtuously endowed, or
villainously inclined; let him be a bawd, a gripe, an usurer, a villain, a
pagan, a barbarian, a wretch, \authorfootnote{2209}Lucian's tyrant, "on whom
you may look with less security than on the sun;" so that he be rich (and
liberal withal) he shall be honoured, admired, adored, reverenced, and highly
\authorfootnote{2210}magnified. "The rich is had in reputation because of his
goods," \biblecite{Eccl. \rn{x.} 31}. He shall be befriended: "for riches gather
many friends," \biblecite{Prov. \rn{xix.} 4},-- \lit{multos numerabit amicos}{all
happiness ebbs and flows with his money},\authorfootnote{2211}. He shall be
accounted a gracious lord, a Mecaenas, a benefactor, a wise, discreet, a
proper, a valiant, a fortunate man, of a generous spirit, \li{Pullus Jovis, et
gallinae, filius albae}: a hopeful, a good man, a virtuous, honest man.
\li{Quando ego ie Junonium puerum, et matris partum vere aureum}, as
\authorfootnote{2212}\Tully{} said of Octavianus, while he was adopted Caesar, and
an heir \authorfootnote{2213}apparent of so great a monarchy, he was a golden
child. All \authorfootnote{2214}honour, offices, applause, grand titles, and
turgent epithets are put upon him, \li{omnes omnia bona dicere}; all men's eyes
are upon him, God bless his good worship, his honour;
\authorfootnote{2215}every man speaks well of him, every man presents him,
seeks and sues to him for his love, favour, and protection, to serve him,
belong unto him, every man riseth to him, as to Themistocles in the Olympics,
if he speak, as of Herod, \lit{Vox Dei, non hominis}{the voice of God, not of
man}. All the graces, Veneres, pleasures, elegances attend him,
\authorfootnote{2216}golden fortune accompanies and lodgeth with him; and as to
those Roman emperors, is placed in his chamber.

\begin{latin}
\begin{verse}%
------Secura naviget aura,\\*
Fortunamque suo temperet arbitrio:\\!
\end{verse}%
\end{latin}
\attrib{\getauthornote{2217}}
he may sail as he will himself, and temper his estate at his pleasure, jovial
days, splendour and magnificence, sweet music, dainty fare, the good things,
and fat of the land, fine clothes, rich attires, soft beds, down pillows are at
his command, all the world labours for him, thousands of artificers are his
slaves to drudge for him, run, ride, and post for him:
\authorfootnote{2218}Divines (for \li{Pythia Philippisat}) lawyers, physicians,
philosophers, scholars are his, wholly devote to his service. Every man seeks
his \authorfootnote{2219}acquaintance, his kindred, to match with him, though
he be an oaf, a ninny, a monster, a goose-cap, \li{uxorem ducat
Danaen}\authorlatintrans{2220}, when, and whom he will, \li{hunc optant generum
Rex et Regina} --he is an excellent \authorfootnote{2221}match for my son, my
daughter, my niece, \etc{} \li{Quicquid calcaverit hic, Rosa fiet}, let him go
whither he will, trumpets sound, bells ring, \etc{}, all happiness attends him,
every man is willing to entertain him, he sups in \authorfootnote{2222}Apollo
wheresoever he comes; what preparation is made for his
\authorfootnote{2223}entertainment? fish and fowl, spices and perfumes, all
that sea and land affords. What cookery, masking, mirth to exhilarate his
person?

\begin{latin}
\begin{verse}%
Da Trebio, pone ad Trebium, vis frater ab illia\\*
Ilibus?------\\!
\end{verse}%
\end{latin}
\attrib{\getauthornote{2224}}

What dish will your good worship eat of?

\translatedverse{%
\begin{latin}
\begin{verse}%
------dulcia poma,\\*
Et quoscunque feret cultus tibi fundus honores,\\*
Ante Larem, gustet venerabilior Lare dives.\\!
\end{verse}%
\end{latin}}{%
\begin{verse}%
Sweet apples, and whate'er thy fields afford,\\*
Before thy Gods be serv'd, let serve thy Lord.\\!
\end{verse}}{%
\attrib{\getauthornote{2225}}}

What sport will your honour have? hawking, hunting, fishing, fowling, bulls,
bears, cards, dice, cocks, players, tumblers, fiddlers, jesters, \etc{}, they
are at your good worship's command. Fair houses, gardens, orchards, terraces,
galleries, cabinets, pleasant walks, delightsome places, they are at hand:
\authorfootnote{2226}\li{in aureis lac, vinum in argenteis, adolescentulae ad
nutum speciosae}, wine, wenches, \etc{} a Turkish paradise, a heaven upon
earth. Though he be a silly soft fellow, and scarce have common sense, yet if
he be borne to fortunes (as I have said) \authorfootnote{2227}\li{jure
haereditario sapere jubetur}, he must have honour and office in his course:
\authorfootnote{2228}\li{Nemo nisi dives honore dignus} (Ambros.
\bookcite{\textlatin{offic. 21.}}) none so worthy as himself: he shall have it,
\li{atque esto quicquid Servius aut Labeo}. Get money enough and command
\authorfootnote{2229}kingdoms, provinces, armies, hearts, hands, and
affections; thou shalt have popes, patriarchs to be thy chaplains and
parasites: thou shalt have (Tamerlane-like) kings to draw thy coach, queens to
be thy laundresses, emperors thy footstools, build more towns and cities than
great Alexander, Babel towers, pyramids and Mausolean tombs, \etc{} command
heaven and earth, and tell the world it is thy vassal, \li{auro emitur diadema,
argento caelum panditur, denarius philosophum conducit, nummus jus cogit,
obolus literatum pascit, metallum sanitatem conciliat, aes amicos
conglutinat}\authorlatintrans{2230}. And therefore not without good cause, John
de Medicis, that rich Florentine, when he lay upon his death-bed, calling his
sons, Cosmo and Laurence, before him, amongst other sober sayings, repeated
this, \li{animo quieto digredior, quod vos sanos et divites post me relinquam},
"It doth me good to think yet, though I be dying, that I shall leave you, my
children, sound and rich:" for wealth sways all. It is not with us, as amongst
those Lacedaemonian senators of Lycurgus in \Plutarch{}, "He preferred that
deserved best, was most virtuous and worthy of the place,
\authorfootnote{2231}not swiftness, or strength, or wealth, or friends carried
it in those days:" but \li{inter optimos optimus, inter temperantes
temperantissimus}, the most temperate and best. We have no aristocracies but in
contemplation, all oligarchies, wherein a few rich men domineer, do what they
list, and are privileged by their greatness. \authorfootnote{2232}They may
freely trespass, and do as they please, no man dare accuse them, no not so much
as mutter against them, there is no notice taken of it, they may securely do
it, live after their own laws, and for their money get pardons, indulgences,
redeem their souls from purgatory and hell itself,-- \li{clausum possidet arca
Jovem}. Let them be epicures, or atheists, libertines, Machiavellians, (as they
often are) \authorfootnote{2233}\li{Et quamvis perjuris erit, sine gente,
cruentus}, they may go to heaven through the eye of a needle, if they will
themselves, they may be canonised for saints, they shall be
\authorfootnote{2234}honourably interred in Mausolean tombs, commended by
poets, registered in histories, have temples and statues erected to their
names,-- \li{e manibus illis--nascentur violae}.--If he be bountiful in his
life, and liberal at his death, he shall have one to swear, as he did by
Claudius the Emperor in Tacitus, he saw his soul go to heaven, and be miserably
lamented at his funeral. \li{Ambubalarum collegia, \etc{} Trimalcionis topanta}
in \Petronius \li{recta in caelum abiit}, went right to heaven: a, base quean,
\authorfootnote{2235}"thou wouldst have scorned once in thy misery to have a
penny from her;" and why? \li{modio nummos metiit}, she measured her money by
the bushel. These prerogatives do not usually belong to rich men, but to such
as are most part seeming rich, let him have but a good
\authorfootnote{2236}outside, he carries it, and shall be adored for a god, as
\authorfootnote{2237}Cyrus was amongst the Persians, \li{ob splendidum
apparatum}{for his gay attires}; now most men are esteemed according to their
clothes. In our gullish times, whom you peradventure in modesty would give
place to, as being deceived by his habit, and presuming him some great
worshipful man, believe it, if you shall examine his estate, he will likely be
proved a serving man of no great note, my lady's tailor, his lordship's barber,
or some such gull, a Fastidius Brisk, Sir Petronel Flash, a mere outside. Only
this respect is given him, that wheresoever he comes, he may call for what he
will, and take place by reason of his outward habit.

But on the contrary, if he be poor, \biblecite{Prov. \rn{xv.} 15}, "all his days
are miserable," he is under hatches, dejected, rejected and forsaken, poor in
purse, poor in spirit; \authorfootnote{2238}\li{prout res nobis fluit, ita et
animus se habet}; \authorfootnote{2239}money gives life and soul. Though he be
honest, wise, learned, well-deserving, noble by birth, and of excellent good
parts; yet in that he is poor, unlikely to rise, come to honour, office, or
good means, he is contemned, neglected, \li{frustra sapit, inter literas
esurit, amicus molestus}. \authorfootnote{2240}"If he speak, what babbler is
this?" \biblecite{Ecclus}, his nobility without wealth, is \li{projecta vilior
alga}\authorlatintrans{2241}, and he not esteemed: \li{nos viles pulli nati
infelicibus ovis}, if once poor, we are metamorphosed in an instant, base
slaves, villains, and vile drudges; \authorfootnote{2242}for to be poor, is to
be a knave, a fool, a wretch, a wicked, an odious fellow, a common eyesore, say
poor and say all; they are born to labour, to misery, to carry burdens like
juments, \li{pistum stercus comedere} with Ulysses' companions, and as
Chremilus objected in Aristophanes, \authorfootnote{2243}\li{salem lingere},
lick salt, to empty jakes, fay channels, \authorfootnote{2244}carry out dirt
and dunghills, sweep chimneys, rub horse-heels, \etc{} I say nothing of Turks,
galley-slaves, which are bought \authorfootnote{2245}and sold like juments, or
those African Negroes, or poor \authorfootnote{2246}Indian drudges, \li{qui
indies hinc inde deferendis oneribus occumbunt, nam quod apud nos boves et
asini vehunt, trahunt}\authorlatintrans{2247}, \etc{} \li{Id omne misellis
Indis}, they are ugly to behold, and though erst spruce, now rusty and squalid,
because poor, \authorfootnote{2248}\li{immundas fortunas aquum est squalorem
sequi}, it is ordinarily so. \authorfootnote{2249}"Others eat to live, but they
live to drudge," \authorfootnote{2250}\li{servilis et misera gens nihil
recusare audet}, a servile generation, that dare refuse no task.--
\authorfootnote{2251}\li{Heus tu Dromo, cape hoc flabellum, ventulum hinc
facito dum lavamus}, sirrah blow wind upon us while we wash, and bid your
fellow get him up betimes in the morning, be it fair or foul, he shall run
fifty miles afoot tomorrow, to carry me a letter to my mistress, \li{Socia ad
pistrinam}, Socia shall tarry at home and grind malt all day long, Tristan
thresh. Thus are they commanded, being indeed some of them as so many
footstools for rich men to tread on, blocks for them to get on horseback, or as
\authorfootnote{2252}"walls for them to piss on." They are commonly such
people, rude, silly, superstitious idiots, nasty, unclean, lousy, poor,
dejected, slavishly humble: and as \authorfootnote{2253}Leo Afer observes of
the commonalty of Africa, \li{natura viliores sunt, nec apud suos duces majore
in precio quam si canes essent}: \authorfootnote{2254}base by nature, and no
more esteemed than dogs, \li{miseram, laboriosam, calamitosam vitam agunt, et
inopem, infelicem, rudiores asinis, ut e brutis plane natos dicas}: no
learning, no knowledge, no civility, scarce common, sense, nought but barbarism
amongst them, \li{belluino more vivunt, neque calceos gestant, neque vestes},
like rogues and vagabonds, they go barefooted and barelegged, the soles of
their feet being as hard as horse-hoofs, as \authorfootnote{2255}Radzivilus
observed at Damietta in Egypt, leading a laborious, miserable, wretched,
unhappy life, \authorfootnote{2256}"like beasts and juments, if not worse:"
(for a \authorfootnote{2257}Spaniard in Incatan, sold three Indian boys for a
cheese, and a hundred Negro slaves for a horse) their discourse is scurrility,
their \li{summum bonum}, a pot of ale. There is not any slavery which these
villains will not undergo, \li{inter illos plerique latrinas evacuant, alii
culinariam curant, alii stabularios agunt, urinatores et id genus similia
exercent}, \etc{} like those people that dwell in the
\authorfootnote{2258}Alps, chimney-sweepers, jakes-farmers, dirt-daubers,
vagrant rogues, they labour hard some, and yet cannot get clothes to put on, or
bread to eat. For what can filthy poverty give else, but
\authorfootnote{2259}beggary, fulsome nastiness, squalor, contempt, drudgery,
labour, ugliness, hunger and thirst; \li{pediculorum, et pulicum numerum}? as
\authorfootnote{2260}he well followed it in Aristophanes, fleas and lice,
\li{pro pallio vestem laceram, et pro pulvinari lapidem bene magnum ad caput},
rags for his raiment, and a stone for his pillow, \li{pro cathedra, ruptae
caput urnae}, he sits in a broken pitcher, or on a block for a chair, \li{et
malvae, ramos pro panibus comedit}, he drinks water, and lives on wort leaves,
pulse, like a hog, or scraps like a dog, \li{ut nunc nobis vita afficitur, quis
non putabit insaniam esse, infelicitatemque}? as Chremilus concludes his
speech, as we poor men live nowadays, who will not take our life to be
\authorfootnote{2261}infelicity, misery, and madness?

If they be of little better condition than those base villains, hunger-starved
beggars, wandering rogues, those ordinary slaves, and day-labouring drudges;
yet they are commonly so preyed upon by \authorfootnote{2262}polling officers
for breaking the laws, by their tyrannising landlords, so flayed and fleeced by
perpetual \authorfootnote{2263}exactions, that though they do drudge, fare
hard, and starve their genius, they cannot live in \authorfootnote{2264}some
countries; but what they have is instantly taken from them, the very care they
take to live, to be drudges, to maintain their poor families, their trouble and
anxiety "takes away their sleep," \biblecite{Sirac. \rn{xxxi.} 1}, it makes them
weary of their lives: when they have taken all pains, done their utmost and
honest endeavours, if they be cast behind by sickness, or overtaken with years,
no man pities them, hard-hearted and merciless, uncharitable as they are, they
leave them so distressed, to beg, steal, murmur, and
\authorfootnote{2265}rebel, or else starve. The feeling and fear of this misery
compelled those old Romans, whom Menenius Agrippa pacified, to resist their
governors: outlaws, and rebels in most places, to take up seditious arms, and
in all ages hath caused uproars, murmurings, seditions, rebellions, thefts,
murders, mutinies, jars and contentions in every commonwealth: grudging,
repining, complaining, discontent in each private family, because they want
means to live according to their callings, bring up their children, it breaks
their hearts, they cannot do as they would. No greater misery than for a lord
to have a knight's living, a gentleman a yeoman's, not to be able to live as
his birth and place require. Poverty and want are generally corrosives to all
kinds of men, especially to such as have been in good and flourishing estate,
are suddenly distressed, \authorfootnote{2266}nobly born, liberally brought up,
and, by some disaster and casualty miserably dejected. For the rest, as they
have base fortunes, so have they base minds correspondent, like beetles, \li{e
stercore orti, e stercore victus, in stercore delicium}, as they were obscurely
born and bred, so they delight in obscenity; they are not thoroughly touched
with it. \li{Angustas animas angusto in pectore versant}\authorfootnote{2267}.
Yet, that which is no small cause of their torments, if once they come to be in
distress, they are forsaken of their fellows, most part neglected, and left
unto themselves; as poor \authorfootnote{2268}Terence in Rome was by Scipio,
Laelius, and Furius, his great and noble friends.

\translatedverse{%
\begin{latin}
\begin{verse}%
Nil Publius Scipio profuit, nil ei Laelius, nil Furius,\\*
Tres per idem tempus qui agitabant nobiles facillime,\\*
Horum ille opera ne domum quident habuit conductitiam.\\!
\end{verse}%
\end{latin}}{%\authorlatintrans{2269}
\begin{verse}%
Publius Scipio, Laelius and Furius,\\*
three of the most distinguished noblemen at that day in Rome,\\*
were of so little service to him,\\*
that he could scarcely procure a lodging through their patronage.\\!
\end{verse}}{}

'Tis generally so, \li{Tempora si fuerint nubila, solus eris}, he is left cold
and comfortless, \li{nullas ad amissas ibit amicus opes}, all flee from him as
from a rotten wall, now ready to fall on their heads. \biblecite{Prov. \rn{xix.}
1.} "Poverty separates them from their \authorfootnote{2270}neighbours."

\translatedverse{%
\begin{latin}
\begin{verse}%
Dum fortuna favet vultum servatis amici,\\*
Cum cecidit, turpi vertitis ora fuga.\\!
\end{verse}%
\end{latin}}{%
\begin{verse}%
Whilst fortune favour'd, friends, you smil'd on me,\\*
But when she fled, a friend I could not see.\\!
\end{verse}}{%
\attrib{\getauthornote{2271}}}

Which is worse yet, if he be poor \authorfootnote{2272}every man contemns him,
insults over him, oppresseth him, scoffs at, aggravates his misery.

\translatedverse{%
\begin{latin}
\begin{verse}%
Quum caepit quassata domus subsidere, partes\\*
In proclinatas omne recumbit onus.\\!
\end{verse}%
\end{latin}}{%
\begin{verse}%
When once the tottering house begins to shrink,\\*
Thither comes all the weight by an instinct.\\!
\end{verse}}{%
\attrib{\getauthornote{2273}}}

Nay they are odious to their own brethren, and dearest friends, \biblecite{Pro.
\rn{xix.} 7}. "His brethren hate him if he be poor,"
\authorfootnote{2274}\li{omnes vicini oderunt}, "his neighbours hate him," \biblecite{Pro.
\rn{xiv.} 20}, \authorfootnote{2275}\li{omnes me noti ac ignoti deserunt}, as he
complained in the comedy, friends and strangers, all forsake me. Which is most
grievous, poverty makes men ridiculous, \li{Nil habet infelix paupertas durius
in se, quam quod ridiculos homines facit}, they must endure
\authorfootnote{2276}jests, taunts, flouts, blows of their betters, and take
all in good part to get a meal's meat: \authorfootnote{2277}\li{magnum
pauperies opprobrium, jubet quidvis et facere et pati}. He must turn parasite,
jester, fool, \li{cum desipientibus desipere}; saith
\authorfootnote{2278}Euripides, slave, villain, drudge to get a poor living,
apply himself to each man's humours, to win and please, \etc{}, and be buffeted
when he hath all done, as Ulysses was by Melanthius \authorfootnote{2279}in
\Homer{}, be reviled, baffled, insulted over, for
\authorfootnote{2280}\li{potentiorum stultitia perferenda est}, and may not so
much as mutter against it. He must turn rogue and villain; for as the saying
is, \li{Necessitas cogit ad turpia}, poverty alone makes men thieves, rebels,
murderers, traitors, assassins, "because of poverty we have sinned,"
\biblecite{Ecclus. \rn{xxvii.} 1}, swear and forswear, bear false witness, lie,
dissemble, anything, as I say, to advantage themselves, and to relieve their
necessities: \authorfootnote{2281}\li{Culpae scelerisque magistra est}, when a
man is driven to his shifts, what will he not do?

\translatedverse{%
\begin{latin}
\begin{verse}%
------si miserum fortuna Sinonem\\*
Finxit, vanum etiam mendacemque improba finget.\\!
\end{verse}%
\end{latin}}{%\setauthornote{2282}
\begin{verse}%
Since cruel fortune has made Sinon poor,\\*
she has made him vain and mendacious.\\!
\end{verse}}{}%
he will betray his father, prince, and country, turn Turk, forsake religion,
abjure God and all, \li{nulla tam horrenda proditio, quam illi lucri causa}
(saith \authorfootnote{2283}Leo Afer) \li{perpetrare nolint}.
\authorfootnote{2284}Plato, therefore, calls poverty, "thievish, sacrilegious,
filthy, wicked, and mischievous:" and well he might. For it makes many an
upright man otherwise, had he not been in want, to take bribes, to be corrupt,
to do against his conscience, to sell his tongue, heart, hand, \etc{}, to be
churlish, hard, unmerciful, uncivil, to use indirect means to help his present
estate. It makes princes to exact upon their subjects, great men tyrannise,
landlords oppress, justice mercenary, lawyers vultures, physicians harpies,
friends importunate, tradesmen liars, honest men thieves, devout assassins,
great men to prostitute their wives, daughters, and themselves, middle sort to
repine, commons to mutiny, all to grudge, murmur, and complain. A great
temptation to all mischief, it compels some miserable wretches to counterfeit
several diseases, to dismember, make themselves blind, lame, to have a more
plausible cause to beg, and lose their limbs to recover their present wants.
Jodocus Damhoderius, a lawyer of Bruges, \bookcite{\textlatin{praxi rerum
criminal. c. 112.}} hath some notable examples of such counterfeit cranks, and
every village almost will yield abundant testimonies amongst us; we have
dummerers, Abraham men, \etc{} And that which is the extent of misery, it
enforceth them through anguish and wearisomeness of their lives, to make away
themselves; they had rather be hanged, drowned, \etc{}, than to live without
means.

\translatedverse{%
\begin{latin}
\begin{verse}%
In mare caetiferum, ne te premat aspera egestas,\\*
Desili, et a celsis corrue Cerne jugis.\\!
\end{verse}%
\end{latin}}{%
\begin{verse}%
Much better 'tis to break thy neck,\\*
Or drown thyself i' the sea,\\*
Than suffer irksome poverty;\\*
Go make thyself away.\\!
\end{verse}}{%
\attrib{\getauthornote{2285}}}

A Sybarite of old, as I find it registered in \authorfootnote{2286}Athenaeus,
supping in Phiditiis in Sparta, and observing their hard fare, said it was no
marvel if the Lacedaemonians were valiant men; "for his part, he would rather
run upon a sword point (and so would any man in his wits,) than live with such
base diet, or lead so wretched a life." \authorfootnote{2287}In Japonia, 'tis a
common thing to stifle their children if they be poor, or to make an abortion,
which \Aristotle{} commends. In that civil commonwealth of China,
\authorfootnote{2288}the mother strangles her child, if she be not able to
bring it up, and had rather lose, than sell it, or have it endure such misery
as poor men do. Arnobius, \bookcite{\textlatin{lib. 7, adversus gentes}},
\authorfootnote{2289}Lactantius, \bookcite{\textlatin{lib. 5. cap. 9.}} objects
as much to those ancient Greeks and Romans, "they did expose their children to
wild beasts, strangle, or knock out their brains against a stone, in such
cases." If we may give credit to \authorfootnote{2290}Munster, amongst us
Christians in Lithuania, they voluntarily mancipate and sell themselves, their
wives and children to rich men, to avoid hunger and beggary;
\authorfootnote{2291}many make away themselves in this extremity. Apicius the
Roman, when he cast up his accounts, and found but 100\thinspace{}000 crowns
left, murdered himself for fear he should be famished to death. P. Forestus, in
his medicinal observations, hath a memorable example of two brothers of Louvain
that, being destitute of means, became both melancholy, and in a discontented
humour massacred themselves. Another of a merchant, learned, wise otherwise and
discreet, but out of a deep apprehension he had of a loss at seas, would not be
persuaded but as \authorfootnote{2292}Ventidius in the poet, he should die a
beggar. In a word, thus much I may conclude of poor men, that though they have
good \authorfootnote{2293}parts they cannot show or make use of them:
\authorfootnote{2294}\li{ab inopia ad virtutem obsepta est via}, 'tis hard for
a poor man to \authorfootnote{2295}rise, \li{haud facile emergunt, quorum
virtutibus obstat res angusta domi}\authorlatintrans{2296}. "The wisdom of the
poor is despised, and his words are not heard." \biblecite{Eccles. \rn{vi.} 19}.
His works are rejected, contemned, for the baseness and obscurity of the
author, though laudable and good in themselves, they will not likely take.

\translatedverse{%
\begin{latin}
\begin{verse}%
Nulla placere diu, neque vivere carmina possunt,\\*
Quae scribuntur atquae potoribus.------\\!
\end{verse}%
\end{latin}}{%
\begin{verse}%
No verses can please men or live long\\*
that are written by water-drinkers.\\!
\end{verse}}{}

Poor men cannot please, their actions, counsels, consultations, projects, are
vilified in the world's esteem, \li{amittunt consilium in re}, which Gnatho
long since observed. \authorfootnote{2297}\li{Sapiens crepidas sibi nunquam nec
soleas fecit}, a wise man never cobbled shoes; as he said of old, but how doth
he prove it? I am sure we find it otherwise in our days,
\authorfootnote{2298}\li{pruinosis horret facundia pannis}. \Homer{} himself must
beg if he want means, and as by report sometimes he did
\authorfootnote{2299}"go from door to door, and sing ballads, with a company of
boys about him." This common misery of theirs must needs distract, make them
discontent and melancholy, as ordinarily they are, wayward, peevish, like a
weary traveller, for \authorfootnote{2300}\li{Fames et mora bilem in nares
conciunt}, still murmuring and repining: \li{Ob inopiam morosi sunt, quibus est
male}, as \Plutarch{} quotes out of Euripides, and that comical poet well seconds,

\begin{latin}
\begin{verse}%
Omnes quibus res sunt minus secundae, nescio quomodo\\*
Suspitiosi, ad contumeliam omnia accipiunt magis,\\*
Propter suam impotentiam se credunt negligi.\\!
\end{verse}%
\end{latin}
\attrib{\getauthornote{2301}}

"If they be in adversity, they are more suspicious and apt to mistake: they
think themselves scorned by reason of their misery:" and therefore many
generous spirits in such cases withdraw themselves from all company, as that
comedian \authorfootnote{2302}Terence is said to have done; when he perceived
himself to be forsaken and poor, he voluntarily banished himself to Stymphalus,
a base town in Arcadia, and there miserably died.

\translatedverse{%
\begin{latin}
\begin{verse}%
------ad summam inopiam redactus,\\*
Itaque e conspectu omnium abiit Graeciae in terram ultimam.\\!
\end{verse}%
\end{latin}}{%%\authorlatintrans{2303}.
\begin{verse}%
Reduced to the greatest necessity,\\
he withdrew from the gaze of the public to the most remote village in Greece.\\!
\end{verse}}{}

Neither is it without cause, for we see men commonly respected according to
their means, (\authorfootnote{2304}\li{an dives sit omnes quaerunt, nemo an
bonus}) and vilified if they be in bad clothes.
\authorfootnote{2305}Philophaemen the orator was set to cut wood, because he
was so homely attired, \authorfootnote{2306}Terentius was placed at the lower
end of Cecilius' table, because of his homely outside.
\authorfootnote{2307}Dante, that famous Italian poet, by reason his clothes
were but mean, could not be admitted to sit down at a feast. Gnatho scorned his
old familiar friend because of his apparel, \authorfootnote{2308}\li{Hominem
video pannis, annisque obsitum, hic ego illum contempsi prae me}. King Persius
overcome sent a letter to \authorfootnote{2309}Paulus Aemilius, the Roman
general; Persius P. Consuli. S. but he scorned him any answer, \li{tacite
exprobrans fortunam suam} (saith mine author) upbraiding him with a present
fortune. \authorfootnote{2310}Carolus Pugnax, that great duke of Burgundy, made
H. Holland, late duke of Exeter, exiled, run after his horse like a lackey, and
would take no notice of him: \authorfootnote{2311}'tis the common fashion of
the world. So that such men as are poor may justly be discontent, melancholy,
and complain of their present misery, and all may pray with
\authorfootnote{2312}Solomon, "Give me, O Lord, neither riches nor poverty;
feed me with food convenient for me."

%SECT. II. MEMB. IV. SUBSECT. VII.-_A heap of other Accidents causing
%Melancholy, Death of Friends, Losses, \etc{}._
\section[Accidents, Death of Friends, Losses]{A heap of other Accidents causing
Melancholy, Death of Friends, Losses,
\etc{}}\label{sec:accidents-death-of-friends}

\lettrine{I}{n} this labyrinth of accidental causes, the farther I wander, the
more intricate I find the passage, \li{multae ambages}, and new causes as so
many by-paths offer themselves to be discussed: to search out all, were an
Herculean work, and fitter for Theseus: I will follow mine intended thread; and
point only at some few of the chiefest.

\subsection{Death of Friends.}

Amongst which, loss and death of friends may challenge a first place, \li{multi
tristantur}, as \authorfootnote{2313}Vives well observes, \li{post delicias,
convivia, dies festos}, many are melancholy after a feast, holiday, merry
meeting, or some pleasing sport, if they be solitary by chance, left alone to
themselves, without employment, sport, or want their ordinary companions, some
at the departure of friends only whom they shall shortly see again, weep and
howl, and look after them as a cow lows after her calf, or a child takes on
that goes to school after holidays. \li{Ut me levarat tuus adventus, sic
discessus afflixit}, (which \authorfootnote{2314}\Tully{} writ to Atticus) thy
coming was not so welcome to me, as thy departure was harsh. Montanus,
\bookcite{\textlatin{consil. 132.}} makes mention of a country woman that
parting with her friends and native place, became grievously melancholy for
many years; and Trallianus of another, so caused for the absence of her
husband: which is an ordinary passion amongst our good wives, if their husband
tarry out a day longer than his appointed time, or break his hour, they take on
presently with sighs and tears, he is either robbed, or dead, some mischance or
other is surely befallen him, they cannot eat, drink, sleep, or be quiet in
mind, till they see him again. If parting of friends, absence alone can work
such violent effects, what shall death do, when they must eternally be
separated, never in this world to meet again? This is so grievous a torment for
the time, that it takes away their appetite, desire of life, extinguisheth all
delights, it causeth deep sighs and groans, tears, exclamations,

\translatedverse{%
\begin{latin}
\begin{verse}%
(O dulce germen matris, o sanguis meus,\\*
Eheu tepentes, \etc{}--o flos tener.)\\!
\end{verse}%
\end{latin}}{%\authorlatintrans{2315}
\begin{verse}%
Oh sweet offspring; oh my very blood;\\*
oh tender flower, \etc{}
\end{verse}}{}
howling, roaring, many bitter pangs, \authorfootnote{2316}\li{lamentis
gemituque et faemineo ululatu Tecta fremunt}) and by frequent meditation
extends so far sometimes, \authorfootnote{2317}"they think they see their dead
friends continually in their eyes," \li{observantes imagines}, as Conciliator
confesseth he saw his mother's ghost presenting herself still before him.
\li{Quod nimis miseri volunt, hoc facile credunt}, still, still, still, that
good father, that good son, that good wife, that dear friend runs in their
minds: \li{Totus animus hac una cogitatione defixus est}, all the year long, as
\authorfootnote{2318}Pliny complains to Romanus, "methinks I see Virginius, I
hear Virginius, I talk with Virginius," \etc{}

\translatedverse{%
\begin{latin}
\begin{verse}%
Te sine, vae misero mihi, lilia nigra videntur,\\*
Pallentesque rosae, nec dulce rubens hyacinthus,\\*
Nullos nec myrtus, noc laurus spirat odores.\\!
\end{verse}%
\end{latin}}{% \authorlatintrans{2319.5}%
\begin{verse}
Without thee, ah! wretched me,\\*
the lillies lose their whiteness, the roses become pallid, the hyacinth forgets to blush\\*
neither the myrtle nor the laurel retains its odours.
\end{verse}}{%
\attrib{\getauthornote{2319}}}

\cleartoleftpage{}
\begin{figure}[p]
  \begingroup
  \centering
  \includegraphics[keepaspectratio,width=0.75\textwidth]{fortuna-small.jpg}
  \captionart{Fortuna}
  \label{fig:fortuna}
\end{figure}

% Force float here
\clearpage{}
They that are most staid and patient, are so furiously carried headlong by the
passion of sorrow in this case, that brave discreet men otherwise, oftentimes
forget themselves, and weep like children many months together,
\authorfootnote{2320}as if that they to water would, and will not be comforted.
They are gone, they are gone; what shall I do?

\translatedverse{%
\begin{latin}
\begin{verse}%
Abstulit atra dies et funere mersit acerbo,\\*
Quis dabit in lachrymas fontem mihi? quis satis altos\\*
Accendet gemitus, et acerbo verba dolori?\\*
Exhaurit pietas oculos, et hiantia frangit\\*
Pectora, nec plenos avido sinit edere questus,\\*
Magna adeo jactura premit, \etc{}\\!
\end{verse}%
\end{latin}}{%
\begin{verse}%
Fountains of tears who gives, who lends me groans,\\*
Deep sighs sufficient to express my moans?\\*
Mine eyes are dry, my breast in pieces torn,\\*
My loss so great, I cannot enough mourn.\\!
\end{verse}}{}%

So Stroza Filius, that elegant Italian poet, in his Epicedium, bewails his
father's death, he could moderate his passions in other matters, (as he
confesseth) but not in this, lie yields wholly to sorrow,

\begin{latin}
\begin{verse}%
Nunc fateor do terga malis, mens illa fatiscit,\\*
Indomitus quondam vigor et constantia mentis.\\!
\end{verse}%
\end{latin}

How doth \authorfootnote{2321}Quintilian complain for the loss of his son, to
despair almost: Cardan lament his only child in his book
\bookcite{\textlatin{de libris propriis}}, and elsewhere in many of his tracts,
\authorfootnote{2322}St. Ambrose his brother's death? \li{an ego possum non
cogitare de te, aut sine lachrymis cogitare? O amari dies, o flebiles noctes},
\etc{} "Can I ever cease to think of thee, and to think with sorrow? O bitter
days, O nights of sorrow," \etc{} Gregory Nazianzen, that noble Pulcheria!
\li{O decorem, \etc{} flos recens, pullulans}, \etc{} Alexander, a man of most
invincible courage, after Hephestion's death, as Curtius relates, \li{triduum
jacuit ad moriendum obstinatus}, lay three days together upon the ground,
obstinate, to die with him, and would neither eat, drink, nor sleep. The woman
that communed with Esdras (\bookcite{\textlatin{lib. 2. cap. 10.}}) when her
son fell down dead. "fled into the field, and would not return into the city,
but there resolved to remain, neither to eat nor drink, but mourn and fast
until she died." "Rachel wept for her children, and would not be comforted
because they were not." \biblecite{Matt. \rn{ii.} 18}. So did Adrian the emperor
bewail his Antinous; Hercules, Hylas; Orpheus, Eurydice; David, Absalom; (O my
dear son Absalom) \Austin{} his mother Monica, Niobe her children, insomuch that
the \authorfootnote{2323}poets feigned her to be turned into a stone, as being
stupefied through the extremity of grief. \authorfootnote{2324}\li{Aegeas,
signo lugubri filii consternatus, in mare se proecipitatem dedit}, impatient of
sorrow for his son's death, drowned, himself. Our late physicians are full of
such examples. Montanus \bookcite{\textlatin{consil. 242.}}
\authorfootnote{2325}had a patient troubled with this infirmity, by reason of
her husband's death, many years together. Trincavellius,
\bookcite{\textlatin{l. 1. c. 14.}} hath such another, almost in despair, after
his \authorfootnote{2326}mother's departure, \li{ut se ferme proecipitatem
daret}; and ready through distraction to make away himself: and in his
Fifteenth counsel, tells a story of one fifty years of age, "that grew
desperate upon his mother's death;" and cured by Fallopius, fell many years
after into a relapse, by the sudden death of a daughter which he had, and could
never after be recovered. The fury of this passion is so violent sometimes,
that it daunts whole kingdoms and cities. Vespasian's death was pitifully
lamented all over the Roman empire, \li{totus orbis lugebat}, saith Aurelius
Victor. Alexander commanded the battlements of houses to be pulled down, mules
and horses to have their manes shorn off, and many common soldiers to be slain,
to accompany his dear Hephestion's death; which is now practised amongst the
Tartars, when \authorfootnote{2327}a great Cham dieth, ten or twelve thousand
must be slain, men and horses, all they meet; and among those the
\authorfootnote{2328}Pagan Indians, their wives and servants voluntarily die
with them. Leo Decimus was so much bewailed in Rome after his departure, that
as Jovius gives out, \authorfootnote{2329}\li{communis salus, publica
hilaritas}, the common safety of all good fellowship, peace, mirth, and plenty
died with him, \li{tanquam eodem sepulchro cum Leone condita lugebantur}: for
it was a golden age whilst he lived, \authorfootnote{2330}but after his decease
an iron season succeeded, \li{barbara vis et foeda vastitas, et dira malorum
omnium incommoda}, wars, plagues, vastity, discontent. When Augustus Caesar
died, saith Paterculus, \li{orbis ruinam timueramus}, we were all afraid, as if
heaven had fallen upon our heads. \authorfootnote{2331}Budaeus records, how
that, at Lewis the Twelfth his death, \li{tam subita mutatio, ut qui prius
digito coelum attingere videbantur, nunc humi derepente serpere, sideratos esse
diceres}, they that were erst in heaven, upon a sudden, as if they had been
planet-strucken, lay grovelling on the ground;

\translatedverse{%
\begin{latin}
\begin{verse}%
Concussis cecidere animis, seu frondibus ingens\\*
Sylva dolet lapsis------\\!
\end{verse}%
\end{latin}}{%\authorlatintrans{2332.5}
\begin{verse}%
They became fallen in feelings,\\*
as the great forest laments its fallen leaves
\end{verse}}{%
\attrib{\getauthornote{2332}}}
they looked like cropped trees. \authorfootnote{2333}At Nancy in Lorraine, when
Claudia Valesia, Henry the Second French king's sister, and the duke's wife
deceased, the temples for forty days were all shut up, no prayers nor masses,
but in that room where she was. The senators all seen in black, "and for a
twelvemonth's space throughout the city, they were forbid to sing or dance."

\translatedverse{%
\begin{latin}
\begin{verse}%
Non ulli pastos illis egre diebus\\*
Frigida (Daphne) boves ad flumina, nulla nec amnem\\*
Libavit quadrupes, nec graminis attigit herbam.\\!
\end{verse}%
\end{latin}}{%
\begin{verse}%
The swains forgot their sheep, nor near the brink\\*
Of running waters brought their herds to drink;\\*
The thirsty cattle, of themselves, abstained\\*
From water, and their grassy fare disdain'd.\\!
\end{verse}}{%
\attrib{\getauthornote{2334}}}

How were we affected here in England for our Titus, \li{deliciae, humani
generis}, Prince Henry's immature death, as if all our dearest friends' lives
had exhaled with his? \authorfootnote{2335}Scanderbeg's death was not so much
lamented in Epirus. In a word, as \authorfootnote{2336}he saith of Edward the
First at the news of Edward of Caernarvon his son's birth, \li{immortaliter
gavisus}, he was immortally glad, may we say on the contrary of friends'
deaths, \li{immortaliter gementes}, we are diverse of us as so many turtles,
eternally dejected with it.

There is another sorrow, which arises from the loss of temporal goods and
fortunes, which equally afflicts, and may go hand in hand with the preceding;
loss of time, loss of honour, office, of good name, of labour, frustrate hopes,
will much torment; but in my judgment, there is no torture like unto it, or
that sooner procureth this malady and mischief:

\translatedverse{%
\begin{latin}
\begin{verse}%
Ploratur lachrymis amissa pecunia veris:\\!
\end{verse}%
\end{latin}}{%
\begin{verse}%
Lost money is bewailed with grief sincere.\\!
\end{verse}}{%
\attrib{\getauthornote{2337}}}

it wrings true tears from our eyes, many sighs, much sorrow from our hearts,
and often causes habitual melancholy itself, Guianerius
\bookcite{\textlatin{tract. 15. 5.}} repeats this for an especial cause:
\authorfootnote{2338}"Loss of friends, and loss of goods, make many men
melancholy, as I have often seen by continual meditation of such things." The
same causes Arnoldus Villanovanus inculcates, \bookcite{\textlatin{Breviar. l.
1. c. 18.}} \li{ex rerum amissione, damno, amicorum morte}, \etc{} Want alone
will make a man mad, to be Sans argent will cause a deep and grievous
melancholy. Many persons are affected like \authorfootnote{2339}Irishmen in
this behalf, who if they have a good scimitar, had rather have a blow on their
arm, than their weapon hurt: they will sooner lose their life, than their
goods: and the grief that cometh hence, continueth long (saith
\authorfootnote{2340}Plater) "and out of many dispositions, procureth an
habit." \authorfootnote{2341}Montanus and Frisemelica cured a young man of 22
years of age, that so became melancholy, \li{ab amissam pecuniam}, for a sum of
money which he had unhappily lost. Sckenkius hath such another story of one
melancholy, because he overshot himself, and spent his stock in unnecessary
building. \authorfootnote{2342}Roger that rich bishop of Salisbury, \li{exutus
opibus et castris a Rege Stephano}, spoiled of his goods by king Stephen,
\li{vi doloris absorptus, atque in amentiam versus, indecentia fecit}, through
grief ran mad, spoke and did he knew not what. Nothing so familiar, as for men
in such cases, through anguish of mind to make away themselves. A poor fellow
went to hang himself, (which Ausonius hath elegantly expressed in a neat
\authorfootnote{2343}Epigram) but finding by chance a pot of money, flung away
the rope, and went merrily home, but he that hid the gold, when he missed it,
hanged himself with that rope which the other man had left, in a discontented
humour.

\begin{latin}
\begin{verse}%
At qui condiderat, postquam non reperit aurum,\\*
Aptavit collo, quem reperit laqueum.\\!
\end{verse}%
\end{latin}

Such feral accidents can want and penury produce. Be it by suretyship,
shipwreck, fire, spoil and pillage of soldiers, or what loss soever, it boots
not, it will work the like effect, the same desolation in provinces and cities,
as well as private persons. The Romans were miserably dejected after the battle
of Cannae, the men amazed for fear, the stupid women tore their hair and cried.
The Hungarians, when their king Ladislaus and bravest soldiers were slain by
the Turks, \li{Luctus publicus}, \etc{} The Venetians when their forces were
overcome by the French king Lewis, the French and Spanish kings, pope, emperor,
all conspired against them, at Cambray, the French herald denounced open war in
the senate: \li{Lauredane Venetorum dux}, \etc{}, and they had lost Padua,
Brixia, Verona, Forum Julii, their territories in the continent, and had now
nothing left, but the city of Venice itself, \li{et urbi quoque ipsi} (saith
\authorfootnote{2344}Bembus) \li{timendum putarent}, and the loss of that was
likewise to be feared, \li{tantus repente dolor omnes tenuit, ut nunquam,
alias}, \etc{}, they were pitifully plunged, never before in such lamentable
distress. \emph{Anno} 1527, when Rome was sacked by Burbonius, the common
soldiers made such spoil, that fair \authorfootnote{2345}churches were turned
to stables, old monuments and books made horse-litter, or burned like straw;
relics, costly pictures defaced; altars demolished, rich hangings, carpets,
\etc{}, trampled in the dirt. \authorfootnote{2346}Their wives and loveliest
daughters constuprated by every base cullion, as Sejanus' daughter was by the
hangman in public, before their fathers and husbands' faces. Noblemen's
children, and of the wealthiest citizens, reserved for princes' beds, were
prostitute to every common soldier, and kept for concubines; senators and
cardinals themselves dragged along the streets, and put to exquisite torments,
to confess where their money was hid; the rest, murdered on heaps, lay stinking
in the streets; infants' brains dashed out before their mothers' eyes. A
lamentable sight it was to see so goodly a city so suddenly defaced, rich
citizens sent a begging to Venice, Naples, Ancona, \etc{}, that erst lived in
all manner of delights. \authorfootnote{2347}"Those proud palaces that even now
vaunted their tops up to heaven, were dejected as low as hell in an instant."
Whom will not such misery make discontent? Terence the poet drowned himself
(some say) for the loss of his comedies, which suffered shipwreck. When a poor
man hath made many hungry meals, got together a small sum, which he loseth in
an instant; a scholar spent many an hour's study to no purpose, his labours
lost, \etc{}, how should it otherwise be? I may conclude with Gregory,
\li{temporalium amor, quantum afficit, cum haeret possessio, tantum quum
subtrahitur, urit dolor}; riches do not so much exhilarate us with their
possession, as they torment us with their loss.

Next to sorrow still I may annex such accidents as procure fear; for besides
those terrors which I have \authorfootnote{2348}before touched, and many other
fears (which are infinite) there is a superstitious fear, one of the three
great causes of fear in \Aristotle{}, commonly caused by prodigies and dismal
accidents, which much trouble many of us, (\li{Nescio quid animus mihi
praesagit mali.}) As if a hare cross the way at our going forth, or a mouse
gnaw our clothes: if they bleed three drops at nose, the salt falls towards
them, a black spot appear in their nails, \etc{}, with many such, which Delrio
\bookcite{\textlatin{Tom. 2. l. 3. sect. 4.}} Austin Niphus in his book
\bookcite{\textlatin{de Auguriis.}} \idxname{polydorevergil}[Polydore Virgil]
\bookcite{\textlatin{l. 3. de Prodigas}}. Sarisburiensis
\bookcite{\textlatin{Polycrat. l. 1. c. 13.}} discuss at large. They are so
much affected, that with the very strength of imagination, fear, and the
devil's craft, \authorfootnote{2349}"they pull those misfortunes they suspect,
upon their own heads, and that which they fear, shall come upon them," as
Solomon fortelleth, \biblecite{Prov. \rn{x.} 24.} and Isaiah denounceth,
\biblecite{lxvi. 4.} which if \authorfootnote{2350}"they could neglect and
contemn, would not come to pass," \li{Eorum vires nostra resident opinione, ut
morbi gravitas?grotantium cogitatione}, they are intended and remitted, as our
opinion is fixed, more or less. \li{N. N. dat poenas}, saith
\authorfootnote{2351}Crato of such a one, \li{utinam non attraheret}: he is
punished, and is the cause of it \authorfootnote{2352}himself:

\li{Dum fata fugimus fata stulti incurrimus}, the thing that I feared, saith
Job, is fallen upon me.\authorfootnote{2353}

As much we may say of them that are troubled with their fortunes; or ill
destinies foreseen: \li{multos angit praecientia malorum}: The foreknowledge of
what shall come to pass, crucifies many men: foretold by astrologers, or
wizards, \li{iratum ob coelum}, be it ill accident, or death itself: which
often falls out by God's permission; \li{quia daemonem timent} (saith
\Chrysostom{}) \li{Deus ideo permittit accidere}. Severus, Adrian, Domitian, can
testify as much, of whose fear and suspicion, Sueton, Herodian, and the rest of
those writers, tell strange stories in this behalf.
\authorfootnote{2354}Montanus \bookcite{\textlatin{consil. 31.}} hath one
example of a young man, exceeding melancholy upon this occasion. Such fears
have still tormented mortal men in all ages, by reason of those lying oracles,
and juggling priests. \authorfootnote{2355}There was a fountain in Greece, near
Ceres' temple in Achaia, where the event of such diseases was to be known; "A
glass let down by a thread," \etc{} Amongst those Cyanean rocks at the springs
of Lycia, was the oracle of Thrixeus Apollo, "where all fortunes were foretold,
sickness, health, or what they would besides:" so common people have been
always deluded with future events. At this day, \li{Metus futurorum maxime
torquet Sinas}, this foolish fear, mightily crucifies them in China: as
\authorfootnote{2356}Matthew Riccius the Jesuit informeth us, in his
commentaries of those countries, of all nations they are most superstitious,
and much tormented in this kind, attributing so much to their divinators,
\li{ut ipse metus fidem faciat}, that fear itself and conceit, cause it to
\authorfootnote{2357}fall out: If he foretell sickness such a day, that very
time they will be sick, \li{vi metus afflicti in aegritudinem cadunt}; and many
times die as it is foretold. A true saying, \li{Timor mortis, morte pejor}, the
fear of death is worse than death itself, and the memory of that sad hour, to
some fortunate and rich men, "is as bitter as gall," \biblecite{Eccl. \rn{xli.}
1.} \li{Inquietam nobis vitam facit mortis metus}, a worse plague cannot happen
to a man, than to be so troubled in his mind; 'tis \li{triste divortium}, a
heavy separation, to leave their goods, with so much labour got, pleasures of
the world, which they have so deliciously enjoyed, friends and companions whom
they so dearly loved, all at once. Axicchus the philosopher was bold and
courageous all his life, and gave good precepts \li{de contemnenda morte}, and
against the vanity of the world, to others; but being now ready to die himself,
he was mightily dejected, \li{hac luce privabor? his orbabor
bonis?}\authorfootnote{2358} he lamented like a child, \etc{} And though
Socrates himself was there to comfort him, \li{ubi pristina virtutum jactatio O
Axioche}? "where is all your boasted virtue now, my friend?" yet he was very
timorous and impatient of death, much troubled in his mind, \li{Imbellis pavor
et impatientia}, \etc{} "O Clotho," Megapetus the tyrant in Lucian exclaims,
now ready to depart, "let me live a while longer. \authorfootnote{2359}I will
give thee a thousand talents of gold, and two boles besides, which I took from
Cleocritus, worth a hundred talents apiece." "Woe's me,"
\authorfootnote{2360}saith another, "what goodly manors shall I leave! what
fertile fields! what a fine house! what pretty children! how many servants! who
shall gather my grapes, my corn? Must I now die so well settled? Leave all, so
richly and well provided? Woe's me, what shall I do?"
\authorfootnote{2361}\li{Animula vagula, blandula, qua nunc abibis in loca}?

To these tortures of fear and sorrow, may well be annexed curiosity, that
irksome, that tyrannising care, \li{nimia solicitudo},
\authorfootnote{2362}"superfluous industry about unprofitable things, and their
qualities," as Thomas defines it: an itching humour or a kind of longing to see
that which is not to be seen, to do that which ought not to be done, to know
that \authorfootnote{2363}secret which should not be known, to eat of the
forbidden fruit. We commonly molest and tire ourselves about things unfit and
unnecessary, as Martha troubled herself to little purpose. Be it in religion,
humanity, magic, philosophy, policy, any action or study, 'tis a needless
trouble, a mere torment. For what else is school divinity, how many doth it
puzzle? what fruitless questions about the Trinity, resurrection, election,
predestination, reprobation, hell-fire, \etc{}, how many shall be saved,
damned? What else is all superstition, but an endless observation of idle
ceremonies, traditions? What is most of our philosophy but a labyrinth of
opinions, idle questions, propositions, metaphysical terms? Socrates,
therefore, held all philosophers, cavillers, and mad men, \li{circa subtilia
Cavillatores pro insanis habuit, palam eos arguens}, saith
\authorfootnote{2364}Eusebius, because they commonly sought after such things
\li{quae nec percipi a nobis neque comprehendi posset}, or put case they did
understand, yet they were altogether unprofitable. For what matter is it for us
to know how high the Pleiades are, how far distant Perseus and Cassiopeia from
us, how deep the sea, \etc{}, we are neither wiser, as he follows it, nor
modester, nor better, nor richer, nor stronger for the knowledge of it.
\li{Quod supra nos nihil ad, nos}, I may say the same of those genethliacal
studies, what is astrology but vain elections, predictions? all magic, but a
troublesome error, a pernicious foppery? physic, but intricate rules and
prescriptions? philology, but vain criticisms? logic, needless sophisms?
metaphysics themselves, but intricate subtleties, and fruitless abstractions?
alchemy, but a bundle of errors? to what end are such great tomes? why do we
spend so many years in their studies? Much better to know nothing at all, as
those barbarous Indians are wholly ignorant, than as some of us, to be so sore
vexed about unprofitable toys: \li{stultus labor est ineptiarum}, to build a
house without pins, make a rope of sand, to what end? \li{cui bono?} He studies
on, but as the boy told St. \Austin{}, when I have laved the sea dry, thou shalt
understand the mystery of the Trinity. He makes observations, keeps times and
seasons; and as \authorfootnote{2365}Conradus the emperor would not touch his
new bride, till an astrologer had told him a masculine hour, but with what
success? He travels into Europe, Africa, Asia, searcheth every creek, sea,
city, mountain, gulf, to what end? See one promontory (said Socrates of old),
one mountain, one sea, one river, and see all. An alchemist spends his fortunes
to find out the philosopher's stone forsooth, cure all diseases, make men
long-lived, victorious, fortunate, invisible, and beggars himself, misled by
those seducing impostors (which he shall never attain) to make gold; an
antiquary consumes his treasure and time to scrape up a company of old coins,
statues, rules, edicts, manuscripts, \etc{}, he must know what was done of old
in Athens, Rome, what lodging, diet, houses they had, and have all the present
news at first, though never so remote, before all others, what projects,
counsels, consultations, \etc{}, \li{quid Juno in aurem insusurret Jovi},
what's now decreed in France, what in Italy: who was he, whence comes he, which
way, whither goes he, \etc{} \Aristotle{} must find out the motion of Euripus;
Pliny must needs see Vesuvius, but how sped they? One loseth goods, another his
life; Pyrrhus will conquer Africa first, and then Asia: he will be a sole
monarch, a second immortal, a third rich; a fourth commands.
\authorfootnote{2366}\li{Turbine magno spes solicitae in urbibus errant}; we
run, ride, take indefatigable pains, all up early, down late, striving to get
that which we had better be without, (Ardelion's busybodies as we are) it were
much fitter for us to be quiet, sit still, and take our ease. His sole study is
for words, that they be-- \li{Lepidae lexeis compostae, ut tesserulae omnes},
not a syllable misplaced, to set out a stramineous subject: as thine is about
apparel, to follow the fashion, to be terse and polite, 'tis thy sole business:
both with like profit. His only delight is building, he spends himself to get
curious pictures, intricate models and plots, another is wholly ceremonious
about titles, degrees, inscriptions: a third is over-solicitous about his diet,
he must have such and such exquisite sauces, meat so dressed, so far-fetched,
\li{peregrini aeris volucres}, so cooked, \etc{}, something to provoke thirst,
something anon to quench his thirst. Thus he redeems his appetite with
extraordinary charge to his purse, is seldom pleased with any meal, whilst a
trivial stomach useth all with delight and is never offended. Another must have
roses in winter, \li{alieni temporis flores}, snow-water in summer, fruits
before they can be or are usually ripe, artificial gardens and fishponds on the
tops of houses, all things opposite to the vulgar sort, intricate and rare, or
else they are nothing worth. So busy, nice, curious wits, make that
insupportable in all vocations, trades, actions, employments, which to duller
apprehensions is not offensive, earnestly seeking that which others so
scornfully neglect. Thus through our foolish curiosity do we macerate
ourselves, tire our souls, and run headlong, through our indiscretion, perverse
will, and want of government, into many needless cares, and troubles, vain
expenses, tedious journeys, painful hours; and when all is done, \li{quorsum
haec? cui bono}? to what end?

\translatedverse{%
\begin{latin}
\begin{verse}%
Nescire velle quae Magister maximus\\*
Docere non vult, erudita inscitia est.\\!
\end{verse}%
\end{latin}}{%
\begin{verse}%
To profess a disinclination\\*
for that knowledge which is beyond our reach,\\*
is pedantic ignorance.\\!
\end{verse}}{%
\attrib{\getauthornote{2367}}}

\subsection{Unfortunate marriage.}

Amongst these passions and irksome accidents, unfortunate marriage may be
ranked: a condition of life appointed by God himself in Paradise, an honourable
and happy estate, and as great a felicity as can befall a man in this world,
\authorfootnote{2368}if the parties can agree as they ought, and live as
\authorfootnote{2369}\Seneca{} lived with his Paulina; but if they be unequally
matched, or at discord, a greater misery cannot be expected, to have a scold, a
slut, a harlot, a fool, a fury or a fiend, there can be no such plague.
\biblecite{Eccles. \rn{xxvi.} 14}, "He that hath her is as if he held a scorpion,
\etc{}" \biblecite{xxvi. 25}, "a wicked wife makes a sorry countenance, a heavy
heart, and he had rather dwell with a lion than keep house with such a wife."
Her \authorfootnote{2370}properties \idxname{Jovianus}[Jovianus Pontanus] hath described at large,
\bookcite{\textlatin{Ant. dial. Tom. 2}}, under the name of Euphorbia. Or if
they be not equal in years, the like mischief happens. Cecilius in
\bookcite{\textlatin{Agellius lib. 2. cap. 23}}, complains much of an old wife,
\li{dum ejus morti inhio, egomet mortuus vivo inter vivos}, whilst I gape after
her death, I live a dead man amongst the living, or if they dislike upon any
occasion,

\begin{verse}%
Judge who that are unfortunately wed\\*
What 'tis to come into a loathed bed.\\!
\end{verse}%
\attrib{\getauthornote{2371}}

The same inconvenience befalls women.

\translatedverse{%
\begin{latin}
\begin{verse}%
At vos o duri miseram lugete parentes,\\*
Si ferro aut laqueo laeva hac me exsolvere sorte\\*
Sustineo:------\\!
\end{verse}%
\end{latin}}{%
\begin{verse}%
Hard hearted parents both lament my fate,\\*
If self I kill or hang, to ease my state.\\!
\end{verse}}{%
\attrib{\getauthornote{2372}}}

\authorfootnote{2373}A young gentlewoman in Basil was married, saith Felix
Plater, \bookcite{\textlatin{observat. l. 1}}, to an ancient man against her
will, whom she could not affect; she was continually melancholy, and pined away
for grief; and though her husband did all he could possibly to give her
content, in a discontented humour at length she hanged herself. Many other
stories he relates in this kind. Thus men are plagued with women; they again
with men, when they are of divers humours and conditions; he a spendthrift, she
sparing; one honest, the other dishonest, \etc{} Parents many times disquiet
their children, and they their parents. \authorfootnote{2374}"A foolish son is
an heaviness to his mother." \li{Injusta noverca}: a stepmother often vexeth a
whole family, is matter of repentance, exercise of patience, fuel of
dissension, which made Cato's son expostulate with his father, why he should
offer to marry his client Solinius' daughter, a young wench, \li{Cujus causa
novercam induceret}; what offence had he done, that he should marry again?

Unkind, unnatural friends, evil neighbours, bad servants, debts and debates,
\etc{}, 'twas Chilon's sentence, \li{comes aeris alieni et litis est miseria},
misery and usury do commonly together; suretyship is the bane of many families,
\li{Sponde, praesto noxa est}: "he shall be sore vexed that is surety for a
stranger," \biblecite{Prov. \rn{xi.} 15}, "and he that hateth suretyship is
sure." Contention, brawling, lawsuits, falling out of neighbours and friends.--
\li{discordia demens} (Virg. \bookcite{\textlatin{Aen. 6}},) are equal to the
first, grieve many a man, and vex his soul. \li{Nihil sane miserabilius eorum
mentibus}, (as \authorfootnote{2375}Boter holds) "nothing so miserable as such
men, full of cares, griefs, anxieties, as if they were stabbed with a sharp
sword, fear, suspicion, desperation, sorrow, are their ordinary companions."
Our Welshmen are noted by some of their \authorfootnote{2376}own writers, to
consume one another in this kind; but whosoever they are that use it, these are
their common symptoms, especially if they be convict or overcome,
\authorfootnote{2377}cast in a suit. Arius put out of a bishopric by
Eustathius, turned heretic, and lived after discontented all his life.
\authorfootnote{2378}Every repulse is of like nature; \li{heu quanta de spe
decidi}! Disgrace, infamy, detraction, will almost effect as much, and that a
long time after. Hipponax, a satirical poet, so vilified and lashed two
painters in his iambics, \li{ut ambo laqueo se suffocarent},
\authorfootnote{2379}Pliny saith, both hanged themselves. All oppositions,
dangers, perplexities, discontents, \authorfootnote{2380}to live in any
suspense, are of the same rank: \li{potes hoc sub casu ducere somnos}? Who can
be secure in such cases? Ill-bestowed benefits, ingratitude, unthankful
friends, much disquiet and molest some. Unkind speeches trouble as many;
uncivil carriage or dogged answers, weak women above the rest, if they proceed
from their surly husbands, are as bitter as gall, and not to be digested. A
glassman's wife in Basil became melancholy because her husband said he would
marry again if she died. "No cut to unkindness," as the saying is, a frown and
hard speech, ill respect, a browbeating, or bad look, especially to courtiers,
or such as attend upon great persons, is present death: \li{Ingenium vultu
statque caditque suo}, they ebb and flow with their masters' favours. Some
persons are at their wits' ends, if by chance they overshoot themselves, in
their ordinary speeches, or actions, which may after turn to their disadvantage
or disgrace, or have any secret disclosed. Ronseus \bookcite{\textlatin{epist.
miscel. 2}}, reports of a gentlewoman 25 years old, that falling foul with one
of her gossips, was upbraided with a secret infirmity (no matter what) in
public, and so much grieved with it, that she did thereupon \li{solitudines
quaerere omnes ab se ablegare, ac tandem in gravissimam incidens melancholiam,
contabescere}, forsake all company, quite moped, and in a melancholy humour
pine away. Others are as much tortured to see themselves rejected, contemned,
scorned, disabled, defamed, detracted, undervalued, or
\authorfootnote{2381}"left behind their fellows." Lucian brings in Aetamacles,
a philosopher in his \bookcite{\textlatin{Lapith. convivio}}, much discontented
that he was not invited amongst the rest, expostulating the matter, in a long
epistle, with Aristenetus their host. Praetextatus, a robed gentleman in
\Plutarch{}, would not sit down at a feast, because he might not sit highest, but
went his ways all in a chafe. We see the common quarrelings, that are ordinary
with us, for taking of the wall, precedency, and the like, which though toys in
themselves, and things of no moment, yet they cause many distempers, much
heart-burning amongst us. Nothing pierceth deeper than a contempt or disgrace,
\authorfootnote{2382}especially if they be generous spirits, scarce anything
affects them more than to be despised or vilified. Crato,
\bookcite{\textlatin{consil. 16, l. 2}}, exemplifies it, and common experience
confirms it. Of the same nature is oppression, \biblecite{Ecclus. 77}, "surely
oppression makes a man mad," loss of liberty, which made Brutus venture his
life, Cato kill himself, and \authorfootnote{2383}\Tully{} complain, \li{Omnem
hilaritatem in perpetuum amisi}, mine heart's broken, I shall never look up, or
be merry again, \authorfootnote{2384}\li{haec jactura intolerabilis}, to some
parties 'tis a most intolerable loss. Banishment a great misery, as Tyrteus
describes it in an epigram of his,

\translatedverse{%
\begin{latin}
\begin{verse}%
Nam miserum est patria amissa, laribusque vagari\\*
Mendicum, et timida voce rogare cibos:\\*
Omnibus invisus, quocunque accesserit exul\\*
Semper erit, semper spretus egensque jacet, \etc{}\\!
\end{verse}%
\end{latin}}{%
\begin{verse}%
A miserable thing 'tis so to wander,\\*
And like a beggar for to whine at door,\\*
Contemn'd of all the world, an exile is,\\*
Hated, rejected, needy still and poor.\\!
\end{verse}}{}

Polynices in his conference with Jocasta in \authorfootnote{2385}Euripides,
reckons up five miseries of a banished man, the least of which alone were
enough to deject some pusillanimous creatures. Oftentimes a too great feeling
of our own infirmities or imperfections of body or mind, will shrivel us up; as
if we be long sick:

\begin{latin}
\begin{verse}%
O beata sanitas, te praesente, amaenum\\*
Ver florit gratiis, absque te nemo beatus:\\!
\end{verse}%
\end{latin}

O blessed health! "thou art above all gold and treasure," \biblecite{Ecclus.
\rn{xxx.} 15}, the poor man's riches, the rich man's bliss, without thee there
can be no happiness: or visited with some loathsome disease, offensive to
others, or troublesome to ourselves; as a stinking breath, deformity of our
limbs, crookedness, loss of an eye, leg, hand, paleness, leanness, redness,
baldness, loss or want of hair, \etc{}, \li{hic ubi fluere caepit, diros ictus
cordi infert}, saith \authorfootnote{2386}Synesius, he himself troubled not a
little \li{ob comae defectum}, the loss of hair alone, strikes a cruel stroke
to the heart. Acco, an old woman, seeing by chance her face in a true glass
(for she used false flattering glasses belike at other times, as most
gentlewomen do,) \li{animi dolore in insaniam delapsa est}, (Caelius Rhodiginus
\bookcite{\textlatin{l. 17, c. 2}},) ran mad. \authorfootnote{2387}Brotheus,
the son of Vulcan, because he was ridiculous for his imperfections, flung
himself into the fire. Lais of Corinth, now grown old, gave up her glass to
Venus, for she could hot abide to look upon it. \authorfootnote{2388}\li{Qualis
sum nolo, qualis eram nequeo}. Generally to fair nice pieces, old age and foul
linen are two most odious things, a torment of torments, they may not abide the
thought of it,

\translatedverse{%
\begin{latin}
\begin{verse}%
------o deorum\\*
Quisquis haec audis, utinam inter errem\\*
Nuda leones,\\*
Antequam turpis macies decentes\\*
Occupet malas, teneraeque succus\\*
Defluat praedae, speciosa quaerro\\*
Pascere tigres.\\!
\end{verse}%
\end{latin}}{
\begin{verse}%
Hear me, some gracious heavenly power,\\*
Let lions dire this naked corse devour.\\*
My cheeks ere hollow wrinkles seize.\\*
Ere yet their rosy bloom decays:\\*
While youth yet rolls its vital flood,\\*
Let tigers friendly riot in my blood.\\!
\end{verse}}{%
\attrib{\getauthornote{2389}}}

To be foul, ugly, and deformed, much better be buried alive. Some are fair but
barren, and that galls them. "Hannah wept sore, did not eat, and was troubled
in spirit, and all for her barrenness," \biblecite{1 Sam. 1.} and \biblecite{Gen.
30.} Rachel said "in the anguish of her soul, give me a child, or I shall die:"
another hath too many: one was never married, and that's his hell, another is,
and that's his plague. Some are troubled in that they are obscure; others by
being traduced, slandered, abused, disgraced, vilified, or any way injured:
\li{minime miror eos} (as he said) \li{qui insanire occipiunt ex injuria}, I
marvel not at all if offences make men mad. Seventeen particular causes of
anger and offence \Aristotle{} reckons them up, which for brevity's sake I must
omit. No tidings troubles one; ill reports, rumours, bad tidings or news, hard
hap, ill success, cast in a suit, vain hopes, or hope deferred, another:
expectation, \li{adeo omnibus in rebus molesta semper est expectatio}, as
\authorfootnote{2390}Polybius observes; one is too eminent, another too base
born, and that alone tortures him as much as the rest: one is out of action,
company, employment; another overcome and tormented with worldly cares, and
onerous business. But what \authorfootnote{2391}tongue can suffice to speak of
all?

Many men catch this malady by eating certain meats, herbs, roots, at unawares;
as henbane, nightshade, cicuta, mandrakes, \etc{} \authorfootnote{2392}A
company of young men at Agrigentum in Sicily, came into a tavern; where after
they had freely taken their liquor, whether it were the wine itself, or
something mixed with it 'tis not yet known, \authorfootnote{2393}but upon a
sudden they began to be so troubled in their brains, and their phantasy so
crazed, that they thought they were in a ship at sea, and now ready to be cast
away by reason of a tempest. Wherefore to avoid shipwreck and present drowning,
they flung all the goods in the house out at the windows into the street, or
into the sea, as they supposed; thus they continued mad a pretty season, and
being brought before the magistrate to give an account of this their fact, they
told him (not yet recovered of their madness) that what was done they did for
fear of death, and to avoid imminent danger: the spectators were all amazed at
this their stupidity, and gazed on them still, whilst one of the ancientest of
the company, in a grave tone, excused himself to the magistrate upon his knees,
\li{O viri Tritones, ego in imo jacui}, I beseech your deities, \etc{} for I
was in the bottom of the ship all the while: another besought them as so many
sea gods to be good unto them, and if ever he and his fellows came to land
again, \authorfootnote{2394}he would build an altar to their service. The
magistrate could not sufficiently laugh at this their madness, bid them sleep
it out, and so went his ways. Many such accidents frequently happen, upon these
unknown occasions. Some are so caused by philters, wandering in the sun, biting
of a mad dog, a blow on the head, stinging with that kind of spider called
tarantula, an ordinary thing if we may believe Skeuck. \bookcite{\textlatin{l.
6. de Venenis}}, in Calabria and Apulia in Italy, Cardan,
\bookcite{\textlatin{subtil. l. 9.}} \idxname{scaliger}[Scaliger][\textlatin{exercitat.}] \bookcite{\textlatin{exercitat.
185.}} Their symptoms are merrily described by \idxname{Jovianus}[Jovianus Pontanus],
\bookcite{\textlatin{Ant. dial.}} how they dance altogether, and are cured by
music. \authorfootnote{2395}Cardan speaks of certain stones, if they be carried
about one, which will cause melancholy and madness; he calls them unhappy, as
an \authorfootnote{2396}\li{adamant, selenites}, \etc{} "which dry up the body,
increase cares, diminish sleep:" Ctesias in Persicis, makes mention of a well
in those parts, of which if any man drink, \authorfootnote{2397}"he is mad for
24 hours." Some lose their wits by terrible objects (as elsewhere I have more
\authorfootnote{2398}copiously dilated) and life itself many times, as
Hippolitus affrighted by Neptune's seahorses, Athemas by Juno's furies: but
these relations are common in all writers.

\translatedverse{%
\begin{latin}
\begin{verse}%
Hic alias poteram, et plures subnectere causas,\\*
Sed jumenta vocant, et Sol inclinat, Eundum est.\\!
\end{verse}%
\end{latin}}{%
\begin{verse}%
Many such causes, much more could I say,\\*
But that for provender my cattle stay:\\*
The sun declines, and I must needs away.\\!
\end{verse}}{%
\attrib{\getauthornote{2399}}}

These causes if they be considered, and come alone, I do easily yield, can do
little of themselves, seldom, or apart (an old oak is not felled at a blow)
though many times they are all sufficient every one: yet if they concur, as
often they do, \li{vis unita fortior; et quae non obsunt singula, multa
nocent}, they may batter a strong constitution; as \authorfootnote{2400}\Austin{}
said, "many grains and small sands sink a ship, many small drops make a flood,"
\etc{}, often reiterated; many dispositions produce an habit.

%\chapter{ MEMB. V.} SECT. II. MEMB. V. SUBSECT. I.-_Continent, inward,
%antecedent, next causes and how the body works on the mind_.
\section{Continent, inward, antecedent, next causes and how the body works on
the mind.}

\lettrine{A}{s} a purlieu hunter, I have hitherto beaten about the circuit of
the forest of this microcosm, and followed only those outward adventitious
causes. I will now break into the inner rooms, and rip up the antecedent
immediate causes which are there to be found. For as the distraction of the
mind, amongst other outward causes and perturbations, alters the temperature of
the body, so the distraction and distemper of the body will cause a
distemperature of the soul, and 'tis hard to decide which of these two do more
harm to the other. Plato, Cyprian, and some others, as I have formerly said,
lay the greatest fault upon the soul, excusing the body; others again accusing
the body, excuse the soul, as a principal agent. Their reasons are, because
\authorfootnote{2401}"the manners do follow the temperature of the body," as
Galen proves in his book of that subject, Prosper Calenius
\bookcite{\textlatin{de Atra bile}}, Jason Pratensis \bookcite{\textlatin{c. de
Mania}}, Lemnius \bookcite{\textlatin{l. 4. c. 16.}} and many others. And that
which Gualter hath commented, \bookcite{\textlatin{hom. 10. in epist.
Johannis}}, is most true, concupiscence and originals in, inclinations, and bad
humours, are \authorfootnote{2402}radical in every one of us, causing these
perturbations, affections, and several distempers, offering many times violence
unto the soul. "Every man is tempted by his own concupiscence (James i. 14),
the spirit is willing but the flesh is weak, and rebelleth against the spirit,"
as our \authorfootnote{2403}apostle teacheth us: that methinks the soul hath
the better plea against the body, which so forcibly inclines us, that we cannot
resist, \li{Nec nos obniti contra, nec tendere tantum sufficimus}. How the body
being material, worketh upon the immaterial soul, by mediation of humours and
spirits, which participate of both, and ill-disposed organs, Cornelius Agrippa
hath discoursed \bookcite{\textlatin{lib. 1. de occult. Philos. cap. 63, 64,
65.}} Levinus Lemnius \bookcite{\textlatin{lib. 1. de occult. nat. mir. cap.
12. et 16. et 21. institut. ad opt. vit}}. Perkins \bookcite{\textlatin{lib. 1.
Cases of Cons. cap. 12.}} T. Bright \bookcite{\textlatin{c. 10, 11, 12.}} "in
his treatise of melancholy," for as, \authorfootnote{2404}anger, fear, sorrow,
obtrectation, emulation, \etc{} \li{si mentis intimos recessus occuparint},
saith \authorfootnote{2405}Lemnius, \li{corpori quoque infesta sunt, et illi
teterrimos morbos inferunt}, cause grievous diseases in the body, so bodily
diseases affect the soul by consent. Now the chiefest causes proceed from the
\authorfootnote{2406}heart, humours, spirits: as they are purer, or impurer, so
is the mind, and equally suffers, as a lute out of tune, if one string or one
organ be distempered, all the rest miscarry, \authorfootnote{2407}\li{corpus
onustum hesternis vitiis, animum quoque praegravat una}. The body is
\li{domicilium animae}, her house, abode, and stay; and as a torch gives a
better light, a sweeter smell, according to the matter it is made of; so doth
our soul perform all her actions, better or worse, as her organs are disposed;
or as wine savours of the cask wherein it is kept; the soul receives a tincture
from the body, through which it works. We see this in old men, children,
Europeans; Asians, hot and cold climes; sanguine are merry, melancholy sad,
phlegmatic dull, by reason of abundance of those humours, and they cannot
resist such passions which are inflicted by them. For in this infirmity of
human nature, as Melancthon declares, the understanding is so tied to, and
captivated by his inferior senses, that without their help he cannot exercise
his functions, and the will being weakened, hath but a small power to restrain
those outward parts, but suffers herself to be overruled by them; that I must
needs conclude with Lemnius, \li{spiritus et humores maximum nocumentum
obtinent}, spirits and humours do most harm in \authorfootnote{2408}troubling
the soul. How should a man choose but be choleric and angry, that hath his body
so clogged with abundance of gross humours? or melancholy, that is so inwardly
disposed? That thence comes then this malady, madness, apoplexies, lethargies,
\etc{} it may not be denied.

Now this body of ours is most part distempered by some precedent diseases,
which molest his inward organs and instruments, and so \li{per consequens}
cause melancholy, according to the consent of the most approved physicians.
\authorfootnote{2409}"This humour" (as \Avicenna{} \bookcite{\textlatin{l. 3. Fen.
1. Tract. 4. c. 18.}} Arnoldus \bookcite{\textlatin{breviar. l. 1. c. 18.}}
Jacchinus \bookcite{\textlatin{comment. in 9 Rhasis, c. 15.}} Montaltus,
\bookcite{\textlatin{c. 10.}} Nicholas Piso \bookcite{\textlatin{c. de Melan.}}
\etc{} suppose) "is begotten by the distemperature of some inward part, innate,
or left after some inflammation, or else included in the blood after an
\authorfootnote{2410}ague, or some other malignant disease." This opinion of
theirs concurs with that of Galen, \bookcite{\textlatin{l. 3. c. 6. de locis
affect}}. Guianerius gives an instance in one so caused by a quartan ague, and
Montanus \bookcite{\textlatin{consil. 32.}} in a young man of twenty-eight
years of age, so distempered after a quartan, which had molested him five years
together; Hildesheim \bookcite{\textlatin{spicel. 2. de Mania}}, relates of a
Dutch baron, grievously tormented with melancholy after a long
\authorfootnote{2411}ague: Galen, \bookcite{\textlatin{l. de atra bile, c. 4.}}
puts the plague a cause. Botaldus in his book \bookcite{\textlatin{de lue
vener. c. 2.}} the French pox for a cause, others, frenzy, epilepsy, apoplexy,
because those diseases do often degenerate into this. Of suppression of
haemorrhoids, haemorrhagia, or bleeding at the nose, menstruous retentions,
(although they deserve a larger explication, as being the sole cause of a
proper kind of melancholy, in more ancient maids, nuns and widows, handled
apart by Rodericus a Castro, and Mercatus, as I have elsewhere signified,) or
any other evacuation stopped, I have already spoken. Only this I will add, that
this melancholy which shall be caused by such infirmities, deserves to be
pitied of all men, and to be respected with a more tender compassion, according
to Laurentius, as coming from a more inevitable cause.
%SECT. II. MEMB. V. SUBSECT. II.-_Distemperature of particular Parts, causes_.
\section{Distemperature of particular Parts, causes.}

\lettrine{T}{here} is almost no part of the body, which being distempered, doth
not cause this malady, as the brain and his parts, heart, liver, spleen,
stomach, matrix or womb, pylorus, mirach, mesentery, hypochondries, mesaraic
veins; and in a word, saith \authorfootnote{2412}Arculanus, "there is no part
which causeth not melancholy, either because it is adust, or doth not expel the
superfluity of the nutriment." Savanarola \bookcite{\textlatin{Pract. major.
rubric. 11. Tract. 6. cap. 1.}} is of the same opinion, that melancholy is
engendered in each particular part, and \authorfootnote{2413}Crato
\bookcite{\textlatin{in consil. 17. lib. 2.}} Gordonius, who is
\bookcite{\textlatin{instar omnium, lib. med. partic. 2. cap. 19.}} confirms as
much, putting the \authorfootnote{2414}"matter of melancholy, sometimes in the
stomach, liver, heart, brain, spleen, mirach, hypochondries, when as the
melancholy humour resides there, or the liver is not well cleansed from
melancholy blood."

The brain is a familiar and frequent cause, too hot, or too cold,
\authorfootnote{2415}"through adust blood so caused," as Mercurialis will have
it, "within or without the head," the brain itself being distempered. Those are
most apt to this disease, \authorfootnote{2416}"that have a hot heart and moist
brain," which Montaltus \bookcite{\textlatin{cap. 11. de Melanch.}} approves
out of Halyabbas, Rhasis, and \Avicenna{}. Mercurialis
\bookcite{\textlatin{consil. 11.}} assigns the coldness of the brain a cause,
and Salustius Salvianus \bookcite{\textlatin{med. lect. l. 2. c. 1.}}
\authorfootnote{2417}will have it "arise from a cold and dry distemperature of
the brain." Piso, Benedictus Victorius Faventinus, will have it proceed from a
\authorfootnote{2418}"hot distemperature of the brain;" and
\authorfootnote{2419}Montaltus \bookcite{\textlatin{cap. 10.}} from the brain's
heat, scorching the blood. The brain is still distempered by himself, or by
consent: by himself or his proper affection, as Faventinus calls it,
\authorfootnote{2420}"or by vapours which arise from the other parts, and fume
up into the head, altering the animal facilities."

Hildesheim \bookcite{\textlatin{spicel. 2. de Mania}}, thinks it may be caused
from a \authorfootnote{2421}"distemperature of the heart; sometimes hot;
sometimes cold." A hot liver, and a cold stomach, are put for usual causes of
melancholy: Mercurialis \bookcite{\textlatin{consil. 11. et consil. 6. consil.
86.}} assigns a hot liver and cold stomach for ordinary causes.
\authorfootnote{2422}Monavius, in an epistle of his to Crato in Scoltzius, is
of opinion, that hypochondriacal melancholy may proceed from a cold liver; the
question is there discussed. Most agree that a hot liver is in fault;
\authorfootnote{2423}"the liver is the shop of humours, and especially causeth
melancholy by his hot and dry distemperature." \authorfootnote{2424}"The
stomach and mesaraic veins do often concur, by reason of their obstructions,
and thence their heat cannot be avoided, and many times the matter is so adust
and inflamed in those parts, that it degenerates into hypochondriacal
melancholy." Guianerius \bookcite{\textlatin{c. 2. Tract. 15.}} holds the
mesaraic veins to be a sufficient \authorfootnote{2425}cause alone. The spleen
concurs to this malady, by all their consents, and suppression of haemorrhoids,
\li{dum non expurget alter a causa lien}, saith Montaltus, if it be
\authorfootnote{2426}"too cold and dry, and do not purge the other parts as it
ought," \bookcite{\textlatin{consil. 23.}} Montanus puts the
\authorfootnote{2427}"spleen stopped" for a great cause.
\authorfootnote{2428}Christophorus a Vega reports of his knowledge, that he
hath known melancholy caused from putrefied blood in those seed-veins and womb;
\authorfootnote{2429}"Arculanus, from that menstruous blood turned into
melancholy, and seed too long detained (as I have already declared) by
putrefaction or adustion."

The mesenterium, or midriff, diaphragma, is a cause which the
\authorfootnote{2430}Greeks called \textgreek{φρένας}: because by his
inflammation, the mind is much troubled with convulsions and dotage. All these,
most part, offend by inflammation, corrupting humours and spirits, in this
non-natural melancholy: for from these are engendered fuliginous and black
spirits. And for that reason \authorfootnote{2431}Montaltus
\bookcite{\textlatin{cap. 10. de causis melan.}} will have "the efficient cause
of melancholy to be hot and dry, not a cold and dry distemperature, as some
hold, from the heat of the brain, roasting the blood, immoderate heat of the
liver and bowels, and inflammation of the pylorus. And so much the rather,
because that," as Galen holds, "all spices inflame the blood, solitariness,
waking, agues, study, meditation, all which heat: and therefore he concludes
that this distemperature causing adventitious melancholy is not cold and dry,
but hot and dry." But of this I have sufficiently treated in the matter of
melancholy, and hold that this may be true in non-natural melancholy, which
produceth madness, but not in that natural, which is more cold, and being
immoderate, produceth a gentle dotage. \authorfootnote{2432}Which opinion
Geraldus de Solo maintains in his comment upon Rhasis.

%SECT. II. MEMB. V. SUBSECT. III.-_Causes of Head-Melancholy_.
\section{Causes of Head-Melancholy.}

\lettrine{A}{fter} a tedious discourse of the general causes of melancholy, I
am now returned at last to treat in brief of the three particular species, and
such causes as properly appertain unto them. Although these causes
promiscuously concur to each and every particular kind, and commonly produce
their effects in that part which is most ill-disposed, and least able to
resist, and so cause all three species, yet many of them are proper to some one
kind, and seldom found in the rest. As for example, head-melancholy is commonly
caused by a cold or hot distemperature of the brain, according to Laurentius
\bookcite{\textlatin{cap. 5 de melan}}. but as \authorfootnote{2433}Hercules de
Saxonia contends, from that agitation or distemperature of the animal spirits
alone. Salust. Salvianus, before mentioned, \bookcite{\textlatin{lib. 2. cap.
3. de re med.}} will have it proceed from cold: but that I take of natural
melancholy, such as are fools and dote: for as Galen writes
\bookcite{\textlatin{lib. 4. de puls. 8.}} and \Avicenna{},
\authorfootnote{2434}"a cold and moist brain is an inseparable companion of
folly." But this adventitious melancholy which is here meant, is caused of a
hot and dry distemperature, as \authorfootnote{2435}Damascen the Arabian
\bookcite{\textlatin{lib. 3. cap. 22.}} thinks, and most writers: Altomarus and
Piso call it \authorfootnote{2436}"an innate burning intemperateness, turning
blood and choler into melancholy." Both these opinions may stand good, as Bruel
maintains, and Capivaccius, \li{si cerebrum sit calidius},
\authorfootnote{2437}"if the brain be hot, the animal spirits will be hot, and
thence comes madness; if cold, folly." David Crusius
\bookcite{\textlatin{Theat. morb. Hermet. lib. 2. cap. 6. de atra bile}},
grants melancholy to be a disease of an inflamed brain, but cold
notwithstanding of itself: \li{calida per accidens, frigida per se}, hot by
accident only; I am of Capivaccius' mind for my part. Now this humour,
according to Salvianus, is sometimes in the substance of the brain, sometimes
contained in the membranes and tunicles that cover the brain, sometimes in the
passages of the ventricles of the brain, or veins of those ventricles. It
follows many times \authorfootnote{2438}"frenzy, long diseases, agues, long
abode in hot places, or under the sun, a blow on the head," as Rhasis informeth
us: Piso adds solitariness, waking, inflammations of the head, proceeding most
part \authorfootnote{2439}from much use of spices, hot wines, hot meats: all
which Montanus reckons up \bookcite{\textlatin{consil. 22.}} for a melancholy
Jew; and Heurnius repeats \bookcite{\textlatin{cap. 12. de Mania}}: hot baths,
garlic, onions, saith Guianerius, bad air, corrupt, much
\authorfootnote{2440}waking, \etc{}, retention of seed or abundance, stopping
of haemorrhagia, the midriff misaffected; and according to Trallianus
\bookcite{\textlatin{l. 1. 16.}} immoderate cares, troubles, griefs,
discontent, study, meditation, and, in a word, the abuse of all those six
non-natural things. Hercules de Saxonia, \bookcite{\textlatin{cap. 16. lib.
1.}} will have it caused from a \authorfootnote{2441}cautery, or boil dried up,
or an issue. Amatus Lusitanus \bookcite{\textlatin{cent. 2. cura. 67.}} gives
instance in a fellow that had a hole in his arm, \authorfootnote{2442}"after
that was healed, ran mad, and when the wound was open, he was cured again."
Trincavellius \bookcite{\textlatin{consil. 13. lib. 1.}} hath an example of a
melancholy man so caused by overmuch continuance in the sun, frequent use of
venery, and immoderate exercise: and in his \bookcite{\textlatin{cons. 49. lib.
3.}} from a \authorfootnote{2443}headpiece overheated, which caused
head-melancholy. Prosper Calenus brings in Cardinal Caesius for a pattern of
such as are so melancholy by long study; but examples are infinite.

%SECT. II. MEMB. V. SUBSECT. IV.-_Causes of Hypochondriacal, or Windy
%Melancholy_.
\section{Causes of Hypochondriacal, or Windy Melancholy.}

\lettrine{I}{n} repeating of these causes, I must \li{crambem bis coctam
apponere}, say that again which I have formerly said, in applying them to their
proper species. Hypochondriacal or flatuous melancholy, is that which the
Arabians call mirachial, and is in my judgment the most grievous and frequent,
though Bruel and Laurentius make it least dangerous, and not so hard to be
known or cured. His causes are inward or outward. Inward from divers parts or
organs, as midriff, spleen, stomach, liver, pylorus, womb, diaphragma, mesaraic
veins, stopping of issues, \etc{} Montaltus \bookcite{\textlatin{cap. 15.}} out
of Galen recites, \authorfootnote{2444}"heat and obstruction of those mesaraic
veins, as an immediate cause, by which means the passage of the chilus to the
liver is detained, stopped or corrupted, and turned into rumbling and wind."
Montanus, \bookcite{\textlatin{consil. 233}}, hath an evident demonstration,
Trincavelius another, \bookcite{\textlatin{lib. 1, cap. 1}}, and Plater a
third, \bookcite{\textlatin{observat. lib. 1}}, for a doctor of the law visited
with this infirmity, from the said obstruction and heat of these mesaraic
veins, and bowels; \li{quoniam inter ventriculum et jecur venae effervescunt},
the veins are inflamed about the liver and stomach. Sometimes those other parts
are together misaffected; and concur to the production of this malady: a hot
liver and cold stomach, or cold belly: look for instances in Hollerius, Victor
Trincavelius, \bookcite{\textlatin{consil. 35, l. 3}}, Hildesheim
\bookcite{\textlatin{Spicel. 2, fol. 132}}, Solenander
\bookcite{\textlatin{consil. 9, pro cive Lugdunensi}}, Montanus
\bookcite{\textlatin{consil. 229}}, for the Earl of Montfort in Germany, 1549,
and Frisimelica in the 233 consultation of the said Montanus. I.Caesar
Claudinus gives instance of a cold stomach and over-hot liver, almost in every
consultation, \bookcite{\textlatin{con. 89}}, for a certain count; and
\bookcite{\textlatin{con. 106}}, for a Polonian baron, by reason of heat the
blood is inflamed, and gross vapours sent to the heart and brain. Mercurialis
subscribes to them, \bookcite{\textlatin{cons. 89}}, \authorfootnote{2445}"the
stomach being misaffected," which he calls the king of the belly, because if he
be distempered, all the rest suffer with him, as being deprived of their
nutriment, or fed with bad nourishment, by means of which come crudities,
obstructions, wind, rumbling, griping, \etc{} Hercules de Saxonia, besides
heat, will have the weakness of the liver and his obstruction a cause,
\li{facultatem debilem jecinoris}, which he calls the mineral of melancholy.
Laurentius assigns this reason, because the liver over-hot draws the meat
undigested out of the stomach, and burneth the humours. Montanus,
\bookcite{\textlatin{cons. 244}}, proves that sometimes a cold liver may be a
cause. Laurentius \bookcite{\textlatin{c. 12}}, Trincavelius
\bookcite{\textlatin{lib. 12, consil.}}, and Gualter Bruel, seems to lay the
greatest fault upon the spleen, that doth not his duty in purging the liver as
he ought, being too great, or too little, in drawing too much blood sometimes
to it, and not expelling it, as P. Cnemiandrus in a
\authorfootnote{2446}consultation of his noted \li{tumorem lienis}, he names
it, and the fountain of melancholy. Diocles supposed the ground of this kind of
melancholy to proceed from the inflammation of the pylorus, which is the nether
mouth of the ventricle. Others assign the mesenterium or midriff distempered by
heat, the womb misaffected, stopping of haemorrhoids, with many such. All which
Laurentius, \bookcite{\textlatin{cap. 12}}, reduceth to three, mesentery,
liver, and spleen, from whence he denominates hepatic, splenetic, and mesaraic
melancholy. Outward causes, are bad diet, care, griefs, discontents, and in a
word all those six non-natural things, as Montanus found by his experience,
\bookcite{\textlatin{consil. 244.}} Solenander \bookcite{\textlatin{consil.
9}}, for a citizen of Lyons, in France, gives his reader to understand, that he
knew this mischief procured by a medicine of cantharides, which an unskilful
physician ministered his patient to drink \li{ad venerem excitandam}. But most
commonly fear, grief, and some sudden commotion, or perturbation of the mind,
begin it, in such bodies especially as are ill-disposed. Melancthon,
\bookcite{\textlatin{tract. 14, cap. 2, de anima}}, will have it as common to
men, as the mother to women, upon some grievous trouble, dislike, passion, or
discontent. For as Camerarius records in his life, Melancthon himself was much
troubled with it, and therefore could speak out of experience. Montanus,
\bookcite{\textlatin{consil. 22, pro delirante Judaeo}}, confirms it,
\authorfootnote{2447}grievous symptoms of the mind brought him to it.
Randolotius relates of himself, that being one day very intent to write out a
physician's notes, molested by an occasion, he fell into a hypochondriacal fit,
to avoid which he drank the decoction of wormwood, and was freed.
\authorfootnote{2448}Melancthon "(being the disease is so troublesome and
frequent) holds it a most necessary and profitable study, for every man to know
the accidents of it, and a dangerous thing to be ignorant," and would therefore
have all men in some sort to understand the causes, symptoms, and cures of it.

%SECT. II. MEMB. V. SUBSECT. V.-_Causes of Melancholy from the whole Body_.
\section{Causes of Melancholy from the whole Body.}

\lettrine{A}{s} before, the cause of this kind of melancholy is inward or
outward. Inward, \authorfootnote{2449}"when the liver is apt to engender such a
humour, or the spleen weak by nature, and not able to discharge his office." A
melancholy temperature, retention of haemorrhoids, monthly issues, bleeding at
nose, long diseases, agues, and all those six non-natural things increase it.
But especially \authorfootnote{2450}bad diet, as Piso thinks, pulse, salt meat,
shellfish, cheese, black wine, \etc{} Mercurialis out of Averroes and \Avicenna{}
condemns all herbs: Galen, \bookcite{\textlatin{lib. 3, de loc. affect. cap.
7}}, especially cabbage. So likewise fear, sorrow, discontents, \etc{}, but of
these before. And thus in brief you have had the general and particular causes
of melancholy.

Now go and brag of thy present happiness, whosoever thou art, brag of thy
temperature, of thy good parts, insult, triumph, and boast; thou seest in what
a brittle state thou art, how soon thou mayst be dejected, how many several
ways, by bad diet, bad air, a small loss, a little sorrow or discontent, an
ague, \etc{}; how many sudden accidents may procure thy ruin, what a small
tenure of happiness thou hast in this life, how weak and silly a creature thou
art. "Humble thyself, therefore, under the mighty hand of God," \biblecite{1
Peter, \rn{v.} 6}, know thyself, acknowledge thy present misery, and make right
use of it. \li{Qui stat videat ne cadat}. Thou dost now flourish, and hast
\li{bona animi, corporis, et fortunae}, goods of body, mind, and fortune,
\li{nescis quid serus secum vesper ferat}, thou knowest not what storms and
tempests the late evening may bring with it. Be not secure then, "be sober and
watch," \authorfootnote{2451}\li{fortunam reverenter habe}, if fortunate and
rich; if sick and poor, moderate thyself. I have said.
