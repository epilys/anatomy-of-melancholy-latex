\setauthornote{6302}{Called religious because it is still conversant about religion and such divine objects.}
\setauthornote{6303}{Grotius.}
%\setauthornote{6303}{Grotius. Proceed, ye muses, nor desert me in the middle of my journey, where no footsteps lead me, no wheeltracks indicate the transit of former chariots.}
\setauthornote{6304}{Lib. 1. cap. 16. nonnulli opinionibus addicti sunt, et futura se praedicere arbitrantur.}
\setauthornote{6305}{Aliis videtur quod sunt prophetae et inspirati a Spiritu sancto, et incipiunt prophetare, et multa futura praedicunt.}
\setauthornote{6306}{Cap. 6. de Melanch.}
\setauthornote{6307}{Cap. 5, Tractat. multi ob timorem Dei sunt melancholici, et timorem gehennae. They are still troubled for their sins.}
\setauthornote{6308}{Plater c. 13.}
\setauthornote{6309}{Melancholia Erotica vel quae cum amore est, duplex est: prima quae ab aliis forsan non meretur nomen melancholiae, est affectio eorum quae pro objecto proponunt Deum et ideo nihil aliud curant aut cogitant quam Deum, jejunia, vigilias: altera ob mulieres.}
\setauthornote{6310}{Alia reperitur furoris species a prima vel a secunda, deorum rogantium, vel afflatu numinum furor hic venit.}
\setauthornote{6311}{Qui in Delphis futura praedicunt vates, et in Dodona sacerdotes furentes quidem multa jocunda Graecis deferunt, sani vero exigua am nulla.}
\setauthornote{6312}{Deus bonus, Justus, pulcher, juxta Platonem.}
\setauthornote{6313}{Miror et stupeo cum coelum aspicio et pulchritudinem siderum, angelorum, \&c. et quis digne laudet quod an nobis viget, corpus tam pulchrum, frontem pulchram, nares, genas, oculos, in ellectum, omnia pulchra; si sic in creaturis laboramus; quid in ipso deo?}
\setauthornote{6314}{Drexelius Nicet. lib. 2. cap. 11.}
\setauthornote{6315}{Fulgor divinae majestatis. Aug.}
\setauthornote{6316}{In Psal. \rn{lxiv.} misit ad nos Epistolas et totam scripturam, quibus nobis faceret amandi desiderium.}
\setauthornote{6317}{Epist. 48. \rn{l.} 4. quid est tota scriptura nisi Epistola omnipotentis Dei ad creaturum suam?}
\setauthornote{6318}{Cap. \rn{vi.} 8.}
\setauthornote{6319}{Cap. \rn{xxvii.} 11.}
\setauthornote{6320}{In Psal. \rn{lxxxv.} omnes pulchritudines terrenas auri, argenti, nemorum et camporum pulchritudinem Solis et Lunae, stellarum, omnia pulchra superans.}
\setauthornote{6321}{Immortalis haec visio immortalis amor, indefessus amor et visio.}
\setauthornote{6322}{Osorius; ubicunque visio et pulchritudo divini aspectus, ibi voluptas ex eodem fonte omnisque beatitudo, nec ab ejus aspectu voluptas, nec ab illa voluptate aspectus separari potest.}
\setauthornote{6323}{Leon Haebreus. Dubitatur an humana felicitas Deo cognoscendo an amando terminetur.}
\setauthornote{6324}{Lib. de anima. Ad hoc objectum amandum et fruendum nati sumus; et hunc expetisset, unicum hunc amasset humana, voluntas, ut summum bonum, et caeteras res omnes eo ordine.}
\setauthornote{6325}{9. de Repub.}
\setauthornote{6326}{Hom. 9. in epist. Johannis cap. 2. Multos conjugium decepit, res alioqui salutaris et necessaria, eo quod caeco ejus amore decepti, divini amoris et gloriae studium in universum abjecerunt; plurimos cibus et potus perdit.}
\setauthornote{6327}{In mundo splendor opum gloriae majestas, amicitiarum praesidia, verborum blanditiae, voluptatum omnis generis illecebrae, victoriae, triumphi, et infinita alia ab amore dei nos abstrahunt, \&c.}
\setauthornote{6328}{In Psal. \rn{xxxii.} Dei amicus esse non potest qui mundi studiis delectatur; ut hanc, formam videas munda cor, serena cor, \&c.}
\setauthornote{6329}{Contemplationis pluma nos sublevat, atque inde erigimur intentione cordis, dulcedine contemplationis distinct. 6. de 7. Itineribus.}
\setauthornote{6330}{Lib. de victimis: amans Deum, sublimia petit, sumptis alis et in coelum recte volat, relicta terra, cupidus aberrandi cum sole, luna, stellarumque sacra militia, ipso Deo duce.}
\setauthornote{6331}{In com. Plat. cap. 7. ut Solem videas oculis, fieri debes Solaris: ut divinam aspicias pulchritudinem, demitte materiam, demitte sensum, et Deum qualis sit videbis.}
\setauthornote{6332}{Avare, quid inhias his, \&c. pulchrior est qui te ambit ipsum visurus, ipsum habiturus.}
\setauthornote{6333}{Prov. \rn{viii.}}
\setauthornote{6334}{Cap. 18. Rom. Amorem hunc divinum totis viribus amplexamini; Deum vobis omni officiorum genere propitium facite.}
\setauthornote{6335}{Cap. 7. de pulchritudine regna et imperia totius terras et maris et coeli oportet abjicere si ad ipsum conversus veils inseri.}
\setauthornote{6336}{Habitus a Deo infusus, per quem inclinatur homo ad diligendum Deum super omnia.}
\setauthornote{6337}{Dial. 1. Omnia. convertit amor in ipsius pulchri naturam.}
\setauthornote{6338}{Stromatum lib. 2.}
\setauthornote{6339}{Greenham.}
\setauthornote{6340}{De primo praecepto.}
\setauthornote{6341}{De relig. l. 2. Thes. 1.}
\setauthornote{6342}{2 De nat. deorum.}
\setauthornote{6343}{Hist. Belgic. lib. 8.}
\setauthornote{6344}{Superstitio error insanus est epist. 223.}
\setauthornote{6345}{Nam qui superstitione imbutus est, quietus esse nunquam potest.}
\setauthornote{6346}{Greg.}
\setauthornote{6347}{Polit. lib. 1. cap. 13.}
\setauthornote{6348}{Hor.}
\setauthornote{6349}{Epist, Phalar.}
\setauthornote{6350}{In Psal. \rn{iii.}}
\setauthornote{6351}{Lib. 9. cap. 6.}
\setauthornote{6352}{Lib. 3.}
\setauthornote{6353}{Lib. 6. descrip. Graec. nulla est via qua non innumeris idolis est referta. Tantum tunc temporis in miserrimos mortales potentiae et crudelis Tyrannidis Satan exercuit.}
\setauthornote{6354}{The devil divides the empire with Jupiter.}
\setauthornote{6355}{Alex. ab. Alex. lib. 6. cap. 26.}
\setauthornote{6356}{Purchas Pilgrim. lib. 1. c. 3.}
\setauthornote{6357}{Lib. 3.}
\setauthornote{6358}{2 Part. sect. 3. lib. 1. cap. et deinceps.}
\setauthornote{6359}{Titelmannus. Maginus. Bredenbachius. Fr. Aluaresius Itin. de Abyssinis Herbis solum vescuntur votarii, aquis mento tenus dormiunt, \&c.}
\setauthornote{6360}{Bredenbactoius Jod. a Meggen.}
\setauthornote{6361}{See Passevinus Herbastein, Magin. D. Fletcher, Jovius, Hacluit. Purchas, \&c. of their errors.}
\setauthornote{6362}{Deplorat. Gentis Lapp.}
\setauthornote{6363}{Gens superstitioni obnoxia, religionibus adversa.}
\setauthornote{6364}{Boissardus de Magia. Intra septimum aut nonum a baptismo diem moriuntur. Hinc fit, \&c.}
\setauthornote{6365}{Cap. de Incolis terrae sanctae.}
\setauthornote{6366}{Plato in Crit. Daemones custodes sunt hominum et eorum domini, ut nos animalium; nec hominibus, sed et regionibus imperant, vaticiniis, auguriis, nos regunt. Idem fere Max. Tyrius ser. 1. et 26. 27. medios vult daemones inter Deos et homines deorum ministros, praesides hominum, a coelo ad homines descendentes.}
\setauthornote{6367}{Depraeparat. Evangel.}
\setauthornote{6368}{Vel in abusum Dei vel in aemulationem. Dandinus com. in lib. 2. Arist. de An. Text. 29.}
\setauthornote{6369}{Daemones consulunt, et familiares habent daemones plerique sacerdotes. Riccius lib. 1. cap. 10. expedit Sinar.}
\setauthornote{6370}{Vitam turbant, somnos inquietant, irrepentes etiam in corpora merites terrent, valetudinem frangunt, morbos lacessant, ut ad cultum sui cogant, nec aliud his studium, quam ut a vera religione, ad superstitionem vertant: cum sint ipsi poenales, quaerunt sibi adpoenas comites, ut habeant erroris participes.}
\setauthornote{6371}{Lib. 4. praeparat. Evangel, c. Tantamque victoriam amentia hominum consequuti sunt, ut si colligere in unum velis, universum orbem istis scelestibus spiritibus subjectum fuisse invenies: Usque ad Salvaloris adventum hominum caede perniciosissimos daemones placabant, \&c.}
\setauthornote{6372}{Plato.}
\setauthornote{6373}{Strozzius Cicogna omnif. mag. lib. 3. cap. 7. Ezek. viii. 4.; Reg. 11. 4.; Reg. 3. et 17. 14; Jer. xlix.; Num. xi. 3.; Reg. 13.}
\setauthornote{6374}{Lib. 4. cap. 8. praepar.}
\setauthornote{6375}{Bapt. Mant. 4. Fast, de Sancto Georgio.}
\setauthornote{6375.5}{O great master of war, whom our youths worship as if he were Mars self.}
\setauthornote{6376}{Part. 1. cap. 1. et lib. 2. cap. 9.}
\setauthornote{6377}{Polyd. Virg. lib. 1. de prodig.}
\setauthornote{6378}{Hor. l. 3. od. 6.}
\setauthornote{6379}{Lib. 3. hist.}
\setauthornote{6380}{Orata lege me dicastis mulieres Dion. Halicarn.}
\setauthornote{6381}{Tully de nat. deorum lib. 2. Aequa Venus Teucris Pallas iniqua fuit.}
\setauthornote{6382}{Jo. Molanus lib. 3. cap. 59.}
\setauthornote{6383}{Pet. Oliver. de Johanne primo Portugalliae Rege strenue pugnans, et diversae partis ictus clypeo excipiens.}
\setauthornote{6384}{L. 14. Loculos sponte aperuisse et pro iis pugnasse.}
\setauthornote{6385}{Religion, as they hold, is policy, invented alone to keep men in awe.}
\setauthornote{6386}{Annal.}
\setauthornote{6387}{Omnes religione moventur. 5. in Verrem.}
\setauthornote{6388}{Zeleuchus, praefat. legis qui urbem aut regionem inhabitant, persuasos esse oportet esse Deos.}
\setauthornote{6389}{10. de legibus. Religio neglecta maximam pestem in civitatem infert, omnium scelerum fenestram aperit.}
\setauthornote{6390}{Cardarius Com. in Ptolomeum quadripart.}
\setauthornote{6391}{Lipsius l. 1. c. 3.}
\setauthornote{6392}{Homo sine religione, sicut equus sine fraeno.}
\setauthornote{6393}{Vaninus dial. 52. de oraculis.}
\setauthornote{6394}{If a religion be false, only let it be supposed to be true, and it will tame mental ferocity, restrain lusts, and make loyal subjects.}
\setauthornote{6395}{Lib. 10. Ideo Lycurgus, \&c. non quod ipse superstitiosus, sed quod videret mortales paradoxa facilius amplecti, nec res graves audere sine periculo deorum.}
\setauthornote{6396}{Cleonardus epist. 1. Novas leges suas ad Angelum Gabrielem referebat, pro monitore mentiebatur omnia se gerere.}
\setauthornote{6397}{Lib. 16. belli Gallici. Ut metu mortis neglecto, ad virtutem incitarent.}
\setauthornote{6398}{De his lege Luciatium de luctu tom. 1. Homer. Odyss. 11. Virg. Aen. 6.}
\setauthornote{6399}{Baratheo sulfure et flamma stagnante sternum demergebantur.}
\setauthornote{6400}{Et 3. de repub. omnis institutio adolescentum eo referenda ut de deo bene sentiant ob commune bonum.}
\setauthornote{6401}{Boterus.}
\setauthornote{6402}{Citra aquam, viridarium plantavit maximum et pulcherrimum, floribus odoriferis et suavibus plenum, \&c.}
\setauthornote{6403}{Potum quendam dedit quo inescatus, et gravi sopore oppressus, in viridarium interim ducebatur, \&c.}
\setauthornote{6404}{Atque iterum memoratum potum bibendum exhibuit, et sic extra Paradisum reduxit, ut cum evigilaret, sopore soluto, \&c.}
\setauthornote{6405}{Lib. 1. de orb. Concord. cap. 7.}
\setauthornote{6406}{Lib. 4.}
\setauthornote{6407}{Lib. 4.}
\setauthornote{6408}{Exerc. 228.}
\setauthornote{6409}{S. Ed. Sands.}
\setauthornote{6410}{In consult. de princ. inter provinc. Europ.}
\setauthornote{6411}{Lucian. By themselves sustain the brunt of every battle.}
\setauthornote{6412}{S. Ed. Sands in his Relation.}
\setauthornote{6413}{Seneca.}
\setauthornote{6414}{Vice cotis, acutum Reddere quae ferrum valet, exors ipsa secandi.}
\setauthornote{6415}{De civ. Dei lib. 4. cap. 31.}
\setauthornote{6416}{Seeking their own, saith Paul, not Christ's.}
\setauthornote{6417}{He hath the Duchy of Spoleto in Italy, the Marquisate of Ancona, beside Rome, and the territories adjacent, Bologna, Ferrara, \&c. Avignon in France, \&c.}
\setauthornote{6418}{Estote fratres mei, et principes hujus mundi.}
\setauthornote{6419}{The Laity suspect their greatness, witness those statutes of mortmain.}
\setauthornote{6420}{Lib. 8. de Academ.}
\setauthornote{6421}{Praefat. lib. de paradox. Jesuit-Rom. provincia habet Col. 36. Neapol. 23. Veneta 13. Lucit. 15. India, orient. 17. Brazil. 20, \&c.}
\setauthornote{6422}{In his Chronic. vit. Hen. 8.}
\setauthornote{6423}{15. cap. of his funeral monuments.}
\setauthornote{6424}{Pausanias in Laconicis lib. 3. Idem de Achaicas lib. 7. cujus summae opes, et valde inclyta fama.}
\setauthornote{6425}{Exercit. Eth. Colleg. 3. disp. 3.}
\setauthornote{6426}{Act. xix. 28.}
\setauthornote{6427}{Pontifex Romanus prorsus inermis regibus terrae jura dat, ad regna evehit ad pacem cogit, et peccantes castigat, \&c. quod imperatores Romani 40. legionibus armati non effecerunt.}
\setauthornote{6428}{Mirum quanta passus sit H. 2. quomodo se submisit, ea se facturum pollicitus, quorum hodie ne privatus quidem partem faceret.}
\setauthornote{6429}{Sigonius 9. hist. Ital.}
\setauthornote{6430}{Curio lib. 4. Fox Martyrol.}
\setauthornote{6431}{Hierocles contends Apollonius to have been as great a prophet as Christ, whom Eusebius confutes.}
\setauthornote{6432}{Munster Cosmog. l. 3. c. 37. Artifices ex officinis, arator e stiva, foeminae e colo, \&c. quasi numine quodam rapti, nesciis parentibus et dominis recta adeunt, \&c. Combustus demum ab Herbipolensi Episcopo; haeresis evanuit.}
\setauthornote{6433}{Nulla non provincia haeresibus, Atheismis, \&c, plena. Nullus orbis angulus ab hisce belluis immunis.}
\setauthornote{6434}{Lib. 1. de nat. Deorum. He gave to man an upward gaze, commanding him to fix his eyes on heaven.}
\setauthornote{6435}{Zanchius.}
\setauthornote{6436}{Virg. 6. Aen.}
\setauthornote{6437}{Superstitio ex ignorantia divinitatis emersit, ex vitiosa aemulatione et daemonis illecebris, inconstans, timens, fluctuans, et cui se addicat nesciens, quem imploret, cui se committat, a daemone facile decepta. Lemnius, lib. 3. c. 8.}
\setauthornote{6438}{Seneca.}
\setauthornote{6439}{Vide Baronium 3 Annalium ad annum 324. vit. Constantin.}
\setauthornote{6440}{De rerum varietate, l. 3. c. 38. Parum vero distat sapientia virorum a puerili, multo minus senum et mulierum, cum metu et superstitione et aliena stultitia et improbitate simplices agitantur.}
\setauthornote{6441}{In all superstition wise men follow fools. Bacon's Essays.}
\setauthornote{6442}{Peregrin, Hieros. ca. 5. totum scriptum confusum sine ordine vel colore, absque sensu et ratione ad rusticissimos, idem dedit, rudissimos, et prorsus agrestes, qui nullius erant discretionis, ut dijudicare possent.}
\setauthornote{6443}{Lib. 1. cap. 9. Valent. haeres. 9.}
\setauthornote{6444}{Meteranus li. 8. hist. Belg.}
\setauthornote{6445}{Si doctores suum fecissent officium, et plebem fidei commissam recte instituissent de doctrirnae christianae, capitib. nec sacris scripturis interdixissent, de multis proculdubio recte sensissent.}
\setauthornote{6446}{Curtius li. 4.}
\setauthornote{6447}{See more in Kemnisius' Examen Concil. Trident. de Purgatorio.}
\setauthornote{6448}{Part 1. c. 16, part 3. cap. 18. et 14.}
\setauthornote{6449}{Austin.}
\setauthornote{6450}{Curtius, lib. 8.}
\setauthornote{6451}{Lampridius vitae ejus. Virgines vestales, et sacrum ignem Romae extinxit, et omnes ubique per orbem terrae religiones, unum hoc studens ut solus deus coleretur.}
\setauthornote{6452}{Flagellatorum secta. Munster. lib. 3. Cosmog. cap. 19.}
\setauthornote{6453}{Votum coelibatus, monachatus.}
\setauthornote{6454}{Mater sanitatis, clavis coelorum, ala animae quae leves pennas producat, ut in sublime ferat; currus spiritus sancti, vexilium fidei, porta paradisi, vita angelorum, \&c.}
\setauthornote{6455}{Castigo corpus meum.}
\setauthornote{6456}{Mor. necom.}
\setauthornote{6457}{Lib. 8. cap. 10. de rerum varietate: admiratione digna sunt quae per jejunium hoc modo contingunt: somnia, superstitio, contemptus tormentorum, mortis desiderium obstinata opinio, insania: jejunium naturaliter preparat ad haec omnia.}
\setauthornote{6458}{Epist. \rn{i.} 3. Ita attenuatus fuit jejunio et vigiliis, in tantum exeso corpora ut ossibus vix haerebat, undo nocte infantum vagitus, balatus pecorum, mugitus boum, voces et ludibria daemonum, \&c.}
\setauthornote{6459}{Lib. de abstinentia, Sobrietas et continentia mentem deo conjungunt.}
\setauthornote{6460}{Extasis nihil est aliud quam gustus futurae beatitudinis. Erasmus epist. ad Dorpium in qua toti absorbemur in Deum.}
\setauthornote{6461}{Si religiosum nimis jejunia videris observantem, audaciter melancholicum pronunciabis. Tract. 5. cap. 5.}
\setauthornote{6462}{Solitudo ipsa, mens aegra laboribus anxiis et jejuniis, tum temperatura cibis mutata agrestibus, et humor melancholicus Heremitis illusionum causa sunt.}
\setauthornote{6463}{Solitudo est causa apparitionum; nulli visionibus et hinc delirio magis obnoxii sunt quam qui collegis et eremo vivunt monachi: tales plerumque melancholici ob victum, solitudinem.}
\setauthornote{6464}{Monachi sese putant prophetare ex Deo, et qui solitariam agunt vitam, quum sit instinctu daemonum; et sic falluntur fatidicae; a malo genio habent, quas putant a Deo, et sic enthusiastae.}
\setauthornote{6465}{Sibylla, Pythii, et prophetae qui divinare solent, omnes fanatici sunt melancholici.}
\setauthornote{6466}{Exercit. c. 1.}
\setauthornote{6467}{De divinatione et magicis praestigiis.}
\setauthornote{6468}{Idem.}
\setauthornote{6469}{Post. 15 dierum preces et jejunia, mirabiles videbat visiones.}
\setauthornote{6470}{Fol. 84. vita Stephani, et fol. 177. post trium mensium inediam et languorem per 9 dies nihil comedens aut bibens.}
\setauthornote{6471}{After contemplation in an ecstasy; so Hierom was whipped for reading Tully; see millions of examples in our annals.}
\setauthornote{6472}{Bede, Gregory, Jacobus de Voragine, Lippomanus, Hieronymus, John Major de vitiis patrum, \&c.}
\setauthornote{6473}{Fol. 199. post abstinentiae curas miras illusiones daemonum audivit.}
\setauthornote{6474}{Fol. 155. post seriam meditationem in vigila dici dominicae visionem habuit de purgatorio.}
\setauthornote{6475}{Ubi multos dies manent jejuni consilio sacerdotum auxilia invocantes.}
\setauthornote{6476}{In Necromant. Et cibus quidem glandes erant, potus aqua, lectus sub divo, \&c.}
\setauthornote{6477}{John Everardus Britanno. Romanus lib. edit. 1611 describes all the manner of it.}
\setauthornote{6478}{Varius mappa componere risum vix poterat.}
\setauthornote{6479}{Pleno ridet Catphurnius ore. Hor.}
\setauthornote{6480}{Alanus de Insulis.}
\setauthornote{6481}{Cicero 1. de finibus.}
\setauthornote{6482}{In Micah comment.}
\setauthornote{6483}{Gall. hist. lib. 1.}
\setauthornote{6484}{Lactantius.}
\setauthornote{6485}{Juv. Sat. 15.}
\setauthornote{6486}{Comment in Micah. Ferre non possunt ut illorum Messias communis servator sit, nostrum gaudium, \&c. Messias vel decem decies crucifixuri essent, ipsumque Deum si id fieri posset, una cum angelis et creaturis omnibus, nec absterretur ab hoc facto et si mille interna subeunda forent.}
\setauthornote{6487}{Lucret.}
\setauthornote{6488}{Lucan.}
\setauthornote{6489}{Ad Galat. comment. Nomen odiosius meum quam ullus homicida aut fur.}
\setauthornote{6490}{In comment. Micah. Adeo incomprehensibilis et aspera eorum superbia, \&c.}
\setauthornote{6491}{Synagog. Judaeorum, ca. 1. Inter eorum intelligentissimos Rabbinos nil praeter ignorantiam et insipientiam grandem invenies, horrendam indurationem, et obsti nationem, \&c.}
\setauthornote{6492}{Great is Diana of the Ephesians, Act. xv.}
\setauthornote{6493}{Malunt cum illis insanire, quam cum aliis bene sentire.}
\setauthornote{6494}{Acosta, l. 5.}
\setauthornote{6495}{O Aegypte, religionis tuae solae supersunt fabulae eaeque incredibiles posteris tuis.}
\setauthornote{6496}{Meditat. 19. de coena domin.}
\setauthornote{6497}{Lib. 1. de trin. cap. 2. si decepti sumus, \&c.}
\setauthornote{6498}{Vide Samsatis Isphocanis objectiones in monachum Milesium.}
\setauthornote{6499}{Lege Hossman. Mus exenteratus.}
\setauthornote{6500}{As true as Homer's Iliad, Ovid's Metamorphoses, Aesop's Fables.}
\setauthornote{6501}{Dial. 52. de oraculis.}
\setauthornote{6502}{O sanctas gentes quibus haec nascuntur in horto Numina! Juven. Sat. 15.}
\setauthornote{6503}{Prudentius. Having proceeded to deify leeks and onions, you, oh Egypt, worship such gods.}
\setauthornote{6504}{Praefat. ver. hist.}
\setauthornote{6505}{Tiguri. fol. 1494.}
\setauthornote{6506}{Rosin, antiq. Rom. l. 2. c. 1. et deinceps.}
\setauthornote{6507}{Lib. de divinatione et magicis praestigiis in Mopso.}
\setauthornote{6508}{Cosmo Paccio Interpret. nihil ab aeris caligine aut figurarum varietate impeditus meram pulchritudinem meruit, exultans et misericordia motus, cognatos amicos qui adhuc morantur in terra tuetur, errantibus succurrit, \&c. Deus hoc jussit ut essent genii dii tutelares hominibus, bonos juvantes, males punientes, \&c.}
\setauthornote{6509}{Sacrorum gent. descript. non bene meritos solum, sed et tyrannos pro diis colunt, qui genus humanum horrendum in modum portentosa immanitate divexarunt, \&c. foedas meretrices, \&c.}
\setauthornote{6510}{Cap. 22. de ver. rel. Deos finxerunt eorum poetae, ut infiantium puppas.}
\setauthornote{6511}{Proem, lib. Contra, philos.}
\setauthornote{6512}{Livius, lib. 1. Deus vobis in posterum propitius, Quirites.}
\setauthornote{6513}{Anth. Verdure Imag. deorum.}
\setauthornote{6514}{Mulieris candido splendentes ainicimine varioque laetentes gestimine, verno florentes conamine, solum sternentes. \&c. \Apuleius, lib. 11. de Asino aureo.}
\setauthornote{6515}{Magna religione quaeritur quae possit adultoria plura numerare Minut.}
\setauthornote{6516}{Lib. de sacrificiis, Fumo inhiantes. et muscarum in morem sanguinem exugentes circum aras effusum.}
\setauthornote{6517}{Imagines Deorum lib. sic. inscript.}
\setauthornote{6518}{De ver. relig. cap. 22. Indigni qui terram calcent, \&c.}
\setauthornote{6519}{Octaviano.}
\setauthornote{6520}{Jupiter Tragoedus, de sacrificiis, et passim alias.}
\setauthornote{6521}{666 several kinds of sacrifices in Egypt Major reckons up, tom. 2. coll. of which read more in cap. 1. of Laurentius Pignorius his Egypt characters, a cause of which Sanubiua gives subcis. lib. 3. cap. 1.}
\setauthornote{6522}{Herod. Clio. Immolavit lecta pecora ter mille Delphis, una cum lectis phialis tribus.}
\setauthornote{6523}{Superstitiosus Julianus innumeras sine parsirnonia pecudes mactavit. Amianus 25. Boves albi. M. Caesari salutem, si tu viceris perimus; lib. 3. Romara observantissimi sunt ceremoniarum, bello praesertim.}
\setauthornote{6524}{De sacrificiis: nuculam pro bona valetudine, boves quatuor pro divitiis, centum tauros pro sospite a Trojae reditu, \&c.}
\setauthornote{6525}{De sacris Gentil. et sacrific. Tyg. 1596.}
\setauthornote{6526}{Enimvero si quis recenseret quae stulti mortales in festis, sacrificiis, diis adorandis, \&c. quae vota faciant, quid de iis statuant, \&c. haud scio an risurus, \&c.}
\setauthornote{6527}{Max. Tyrius ser. 1. Croesus regum omnium stultissimus de lebete consulit, alius de numero arenarum, dimensione maris, \&c.}
\setauthornote{6528}{Lib. 4.}
\setauthornote{6529}{Perigr. Hierosol.}
\setauthornote{6530}{Solinus.}
\setauthornote{6531}{Herodotus.}
\setauthornote{6532}{Boterus polit. lib. 2. cap. 16.}
\setauthornote{6533}{Plutarch vit. Crassi.}
\setauthornote{6534}{They were of the Greek church.}
\setauthornote{6535}{Lib. 5. de gestis Scanderbegis.}
\setauthornote{6536}{In templis immania Idolorum monstra conspiciuntur, marmorea, lignea, lutea, \&c. Riccius.}
\setauthornote{6537}{Deum enim placare non est opus, quia non nocet; sed daemonem sacrifices placant, \&c.}
\setauthornote{6538}{Fer. Cortesius.}
\setauthornote{6539}{M. Polas. Lod. Vertomannus navig. lib. 6. cap. 9. P. Martyr. Ocean, dec.}
\setauthornote{6540}{Propertius lib. 3. eleg. 12. There is a contest amongst the living wives as to which shall follow the husband, and not be allowed to die for him is accounted a disgrace.}
\setauthornote{6541}{Matthias a Michou.}
\setauthornote{6542}{Epist. Jesuit. anno. 1549. a Xaverto et socus. Idemque Riccius expedid. ad Sinas l. 1. per totum Jejunatores apud eos toto die carnibus abstinent et piscibus ob religionem, nocte et die Idola colentes; nusquam egredientes.}
\setauthornote{6543}{Ad immortalitatem morte aspirant summi magistrates, \&c. Et multi mortales hac insania, et praepostero immortalitatis studio laborant, et misere pereunt: rex ipse clam venenum hausisset, nisi a servo fuisset detentus.}
\setauthornote{6544}{Cantione in lib. 10. Bonini de repub. fol. 111.}
\setauthornote{6545}{Quin ipsius diaboli ut nequitiam referant.}
\setauthornote{6546}{Lib. de superstit.}
\setauthornote{6547}{Hominibus vitas finis mors, non autem superstitionis, profert haec suos terminos ultra vitae finem.}
\setauthornote{6548}{Buxtorfius Synagog. Jud. c. 4. Inter precandum nemo pediculos attingat, vel pulicem, aut per guttur inferius ventum emittas, \&c. Id. c. 5. et. seq. cap. 36.}
\setauthornote{6549}{Illic omnia animalia, pisces, aves, quos Deus unquam creavit mactabuntur, et vinum generosum, \&c.}
\setauthornote{6550}{Cujus lapsu cedri altissimi 300 dejecti sunt, quumque e lapsu ovum fuerat confractum, pagi 160 inde submersi, et alluvione inundati.}
\setauthornote{6551}{Every king of the world shall send him one of his daughters to be his wife, because it is written, Ps. xlv. 10. Kings' daughters shall attend on him, \&c.}
\setauthornote{6552}{Quum quadringentis adhuc milliaribus ab imperatore Leo hic abesset, tam fortiter rugiebat, ut mulieres Romanae abortierint omnes, mutique, \&c.}
\setauthornote{6553}{Strozzius Cicogna omnif. mag. lib. 1. c. 1. putida multa recenset ex Alcorano, de coelo, stellis, Angelis, Lonicerus c. 21, 22. l. 1.}
\setauthornote{6554}{Quinquies in die orare Turcae tenentur ad meridiem. Bredenbachius cap. 5.}
\setauthornote{6555}{In quolibet anno mensem integrum jejunant interdiu, nec comedentes nec bibentes, \&c.}
\setauthornote{6556}{Nullis unquam multi per totam aetatem carnibus vescuntur. Leo Afer.}
\setauthornote{6557}{Lonicerus to. 1. cap. 17. 18.}
\setauthornote{6558}{Gotardus Arthus ca. 33. hist. orient. Indiae; opinio est expiatorium esse Gangem; et nec mundum ab omni peccato nec salvum fieri posse, qui non hoc flumine se abluat: quam ob causam ex tota India, \&c.}
\setauthornote{6559}{Quia nil volunt deinceps videre.}
\setauthornote{6560}{Nullum se conflictandi finem facit.}
\setauthornote{6561}{Ut in aliquem angulum se reciperet, ne reus fieret ejus delicti quod ipse erat admissurus.}
\setauthornote{6562}{Gregor. Hom.}
\setauthornote{6563}{Bound to the dictates of no master}
\setauthornote{6564}{Epist. 190.}
\setauthornote{6565}{Orat. 8. ut vertigine correptis videntur omnia moveri, omnia iis falsa sunt, quum error in ipsorum cerebro sit.}
\setauthornote{6566}{Res novas affectant et inutiles, falsa veris praeferunt. 2. quod temeritas effutierit, id superbia post modum tuebitur et contumaciae, \&c.}
\setauthornote{6567}{See more in Vincent. Lyrin.}
\setauthornote{6568}{Aust. de haeres. usus mulierum indifferens.}
\setauthornote{6569}{Quod ante peccavit Adam, nudus erat.}
\setauthornote{6570}{Alii nudis pedibus semper ambulant.}
\setauthornote{6571}{Insana feritate sibi non parcunt nam per mortes varias praecipitiorum aquarum et ignium. seipsos necant, et in istum furorem alios cogunt, mortem minantes ni faciant.}
\setauthornote{6572}{Elench. haeret. ab orbe condito.}
\setauthornote{6573}{Nubrigensis. lib. cap. 19.}
\setauthornote{6574}{Jovian. Pont. Ant. Dial.}
\setauthornote{6575}{Cum per Paganos nomen ejus persequi non poterat, sub specie religionis fraudulenter subvertere disponebat.}
\setauthornote{6576}{That writ de professo against Christians, et palestinum deum (ut Socrates lib. 3. cap. 19.) scripturam nugis plenam, \&c. vide Cyrillum in Julianum, Originem in Celsum, \&c.}
\setauthornote{6577}{One image had one gown worth 400 crowns and more.}
\setauthornote{6578}{As at our lady's church at Bergamo in Italy.}
\setauthornote{6579}{Lucilius lib. 1. cap. 22. de falsa relig.}
\setauthornote{6580}{An. 441.}
\setauthornote{6581}{Hospinian Osiander. An haec propositio Deus sit cucurbita vel scarabeus, sit aeque possibilis ac Deus et homo? An possit respectum producere sine fundamento et termino. An levius sit hominem jugulare quam die dominico calceum consuere?}
\setauthornote{6582}{De doct. Christian.}
\setauthornote{6583}{Daniel.}
\setauthornote{6584}{Whilst these fools avoid one vice they run into another of an opposite character}
\setauthornote{6585}{Agrip. ep. 29.}
\setauthornote{6586}{Alex. Gaguin. 22. Discipulis ascitis mirum in modum populum decepit.}
\setauthornote{6587}{Guicciard. descrip. Belg. com. plures habuit asseclas ab iisdem honoratus.}
\setauthornote{6588}{Hen. Nicholas at Leiden 1580. such a one.}
\setauthornote{6589}{See Camden's Annals fo. 242. et 285.}
\setauthornote{6590}{Arius his bowels burst; Montanus hanged himself, \&c. Eudo de stellis, his disciples, ardere potius quam ad vitam corrigi maluerunt; tanta vis infixi semel erroris, they died blaspheming. Nubrigensis c. 9. lib. 1. Jer. vii. 23. Amos. v. 5.}
\setauthornote{6591}{5. Cap.}
\setauthornote{6592}{Poplinerius Lerius praef. hist. Rich. Dinoth.}
\setauthornote{6593}{Advers. gerites lib. 1. postquam in mundo Christiana gens coepit, terrarum orbem periise, et multis malis affectum esse genus humanum videmus.}
\setauthornote{6594}{Quod nec hyeme, nec aestate tanta imbrium copia, nec frugibus torrendis solita flagrantia, nec vernali temperie sata tam laeta sint, nec arboreis foetibus autumni foecundi, minus de montibus marmor ernatur, minus aurum, \&c.}
\setauthornote{6595}{Solitus erat oblectare se fidibus, et voce musica canentium; sed hoc omne sublatum Sybillae cujusdam interventu, \&c. Inde quicquid erat instrumentorum Symphoniacorum, aura gemmisque egregio opere distinctorum comminuit, et in ignem injecit, \&c.}
\setauthornote{6596}{Ob id genus observatiunculas videmus homines misere affligi, et denique mori, et sibi ipsis Christianos videri quum revera sint Judaei.}
\setauthornote{6597}{Ita in corpora nostra fortunasque decretis suis saeviit ut parum obfuerat nisi Deus Lutherum virum perpetua memoria dignissimum excitasset, quin nobis faeno mox communi cum jumentis cibo utendum fuisset.}
\setauthornote{6598}{The Gentiles in India will eat no sensible creatures, or aught that hath blood in it.}
\setauthornote{6599}{Vandormilius de Aucupio. cap. 27.}
\setauthornote{6600}{Some explode all human authors, arts, and sciences, poets, histories, \&c., so precise, their zeal overruns their wits; and so stupid, they oppose all human learning, because they are ignorant themselves and illiterate, nothing must be read but Scriptures; but these men deserve to be pitied, rather than confuted. Others are so strict they will admit of no honest game and pleasure, no dancing, singing, other plays, recreations and games, hawking, hunting, cock-fighting, bear-baiting, \&c., because to see one beast kill another is the fruit of our rebellion against God, \&c.}
\setauthornote{6601}{Nuda ac tremebunda cruentis Irrepet genibus si candida jusserit Ino. Juvenalis. Sect. 6.}
\setauthornote{6602}{Munster Cosmog. lib. 3. cap. 444. Incidit in cloacam, unde se non possit eximere, implorat opem sociorum, sed illi negant, \&c.}
\setauthornote{6603}{De benefic. 7. 2.}
\setauthornote{6604}{Numen venerare praesertim quod civitas colit.}
\setauthornote{6605}{Octavio dial.}
\setauthornote{6606}{Annal. tom. 3. ad annum 324. 1.}
\setauthornote{6607}{Ovid.}
\setauthornote{6607}{Saturn is dead, his laws died with him; now that Jupiter rules the world, let us obey his laws.}
\setauthornote{6608}{In epist. Sym.}
\setauthornote{6609}{Quia deus immensum quiddam est, et infinitum cujus natura perfecte cognosci non potest, aequum ergo est, ut diversa ratione colatur prout quisque aliquid de Deo percipit aut intelligit.}
\setauthornote{6610}{Campanella Calcaginus, and others.}
\setauthornote{6611}{Aeternae beatitudinis consortes fore, qui sancte innocenterque hanc vitam traduxerint, quamcunque illi religionem sequuti sunt.}
\setauthornote{6612}{Comment. in C. Tim. 6. ver. 20. et 21. severitate cum agendum, et non aliter.}
\setauthornote{6613}{Quod silentium haereticis indixerit.}
\setauthornote{6614}{Igne et fuste potius agendum cum haereticis quam cum disputationibus; os alia loquens, \&c.}
\setauthornote{6615}{Praefat. Hist.}
\setauthornote{6616}{Quidam conquestus est mihi de hoc morbo, et deprecatus est ut ego illum curarem; ego quaesivi ab eo quid sentiret; respondit, semper imaginor et cogito de Deo et angelis, \&c. et ita demersus sum hac imaginatione, ut nec edam nec dormiam, nec negotiis, \&c. Ego curavi medicine et persuasione; et sic plures alios.}
\setauthornote{6617}{De anima, c. de humoribus.}
%\setauthornote{6618}{Juvenal. That there are many ghosts and subterranean realms, and a boat-pole, and black frogs in the Stygian gulf, and that so many thousands pass over in one boat, not even boys believe, unless those not as yet washed for money.}
\setauthornote{6618}{Juvenal.}
\setauthornote{6619}{Lib. 5. Gal. hist, quamplurimi reperti sunt qui tot pericula subeuntes irridebant; et quae de fide, religione, \&c. dicebant, ludibrio habebant, nihil eorum admittentes de futura vita.}
\setauthornote{6620}{50,000 atheists at this day in Paris, Mercennus thinks.}
\setauthornote{6621}{Eat, drink, be merry; there is no more pleasure after death}
%\setauthornote{6622}{Hor. l. 2. od. 13. One day succeeds another, and new moons hasten to their wane.}
\setauthornote{6622}{Hor. l. 2. od. 13.}
\setauthornote{6623}{Luke \rn{xvii.}}
\setauthornote{6624}{Wisd. \rn{ii.} 2.}
\setauthornote{6625}{Vers. 6, 7, 8.}
\setauthornote{6626}{Catullus.}
\setauthornote{6627}{Prov. \rn{vii.} 8.}
\setauthornote{6628}{Time glides away, and we grow old by years insensibly accumulating.}
\setauthornote{6629}{Lib. 1.}
\setauthornote{6630}{M. Montan. lib. 1. cap. 4.}
\setauthornote{6631}{Orat. Cont. Hispan. ne proximo decennio deum adorarent, \&c.}
\setauthornote{6632}{Talem se exhibuit, ut nec in Christum, nec Mahometan crederet, unde effectum ut promissa nisi quatenus in suum commodum cederent minime servaret, nec ullo scelere peccatum statueret, ut suis desideriis satisfaceret.}
\setauthornote{6633}{Lib. de mor. Germ.}
\setauthornote{6634}{Or Breslau.}
\setauthornote{6635}{Usque adeo insanus, ut nec inferos, nec superos esse dicat, animasque cum corporibus interire credat, \&c.}
\setauthornote{6636}{Europae deser. cap. 24.}
\setauthornote{6637}{Fratres a Bry Amer. par. 6. librum a Vincentio monacho datum abjecit, nihil se videre ibi hujusmodi dicens rogansque unde haec sciret, quum de coelo et Tartaro contineri ibi diceret.}
\setauthornote{6638}{Non minus hi furunt quam Hercules, qui conjugem et liberos interfecit; habet haec aetas plura hujusmodi portentosa monstra.}
\setauthornote{6639}{De orbis con. lib. 1. cap. 7.}
\setauthornote{6640}{Nonne Romani sine Deo vestro regnant et fruuntur orbe toto, et vos et Deos vestros captivos tenent, \&c. Minutius Octaviano.}
\setauthornote{6641}{Comment. in Genesin copiosus in hoc subjecto.}
\setauthornote{6642}{Ecce pars vestrum et major et melior alget, fame laborat, et deus patitur, dissimulat, non vult, non potest opitulari suis, et vel invalidus vel iniquus est. Cecilius in Minut. Dum rapiunt mala fata bonos, ignoscite fasso, Sollicitor nullos esse putare deos. Ovid. Vidi ego diis fretos, multos decipi. Plautus Casina act. 2. scen. 5.}
\setauthornote{6643}{Martial. l. 4. epig. 21.}
\setauthornote{6644}{Ser. 30. in 5. cap. ad Ephes. hic fractii est pedibus, alter furit, alius ad extremam senectam progressus omnem vitam paupertate peragit, ille morbis gravissimis: sunt haec Providentiae opera? hic surdus, ille mutus, \&c.}
\setauthornote{6645}{Oh! Jupiter, do you hear those things? Collecting many such facts, they weave a tissue of reproaches against God's providence.}
\setauthornote{6646}{Omnia contingenter fieri volunt. Melancthon in praeceptum primum.}
\setauthornote{6647}{Dial. 1. lib. 4. de admir. nat. Arcanis.}
\setauthornote{6648}{Anima mea sit cum animis philosophorum.}
\setauthornote{6649}{Deum unum multis designant nominibus, \&c.}
\setauthornote{6650}{Non intelligis te quum haec dicis, negare te ipsum nomen Dei: quid enim est aliud Natura quam Deus? \&c. tot habet appellationes quot munera.}
\setauthornote{6651}{Austin.}
\setauthornote{6652}{Principio phaemer.}
\setauthornote{6653}{In cities, kings, religions, and in individual men, these things are true and obvious, as Aristotle appears to imply, and daily experience teaches to the reader of history: for what was more sacred and illustrious, by Gentile law, than Jupiter? what now more vile and execrable? In this way celestial objects suggest religions for worldly motives, and when the influx ceases, so does the law, \&c.}
\setauthornote{6654}{And again a great Achilles shall be sent against Troy: religions and their ceremonies shall be born again; however affairs relapse into the same track, there is nothing now that was not formerly and Will not be again, \&c.}
\setauthornote{6655}{Vaninus dial. 52. de oraculis.}
\setauthornote{6656}{Varie homines affecti, alii dei judicium ad tam pii exilium, alii ad naturam referebant, nec ab indignatione dei, sed humanis causis, \&c. 12. Natural, quaest. 33. 39.}
%\setauthornote{6657}{Juv. Sat. 13. There are those who ascribe everything to chance, and believe that the world is made without a director, nature influencing the vicissitudes, \&c.}
\setauthornote{6657}{Juv. Sat. 13.}
\setauthornote{6658}{\textlatin{Epist. ad C. Caesar. Romani olim putabant fortunam regna et imperia dare: Credebant antea mortales fortunam solam opes et honores largiri, idque duabus de causis; primum quod indignus quisque dives honoratus, potens; alterum, vix quisquam perpetuo bonis iis frui visus. Postea prudentiores didicere fortunam suam quemque fingere.}}
\setauthornote{6659}{10 de legib. Alii negant esse deos, alii deos non curare res humanas, alii utraque concedunt.}
\setauthornote{6660}{Lib. 8. ad mathern.}
\setauthornote{6661}{Origen. contra Celsum. l. 3. hos immerito nobiscum conferri fuse declarat.}
\setauthornote{6662}{Crucifixum deum ignominiose Lucianus vita peregrin. Christum vocat.}
\setauthornote{6663}{De ira, 16. 34. Iratus coelo quod obstreperet, ad pugnam vocans Jovem, quanta dementia? putavit sibi nocere non posse, et se nocere tamen Jovi posse.}
\setauthornote{6664}{Lib. 1. 1.}
\setauthornote{6665}{Idem status post mortem, ac fuit antequam nasceremur, et Seneca. Idem erit post me quod ante me fuit.}
\setauthornote{6666}{Lucernae eadem conditio quum extinguitur, ac fuit antequam accenderetur; ita et hominis.}
\setauthornote{6667}{Dissert, cum nunc sider.}
\setauthornote{6668}{Campanella, cap. 18. Atheism, triumphat.}
\setauthornote{6669}{Comment. in Gen. cap. 7.}
\setauthornote{6670}{So that a man may meet an atheist as soon in his study as in the street.}
\setauthornote{6671}{Simonis religio incerto auctore Cracoviae edit. 1588, conclusio libri est, Ede itaque, bibe, lude, \&c. jam Deus figmentum est.}
\setauthornote{6672}{Lib. de immortal. animae.}
\setauthornote{6673}{Pag. 645. an. 1238. ad finem Henrici tertii. Idem Pisterius, pag. 743. in compilat. sua.}
\setauthornote{6674}{Virg. They place fear, fate, and the sound of craving Acheron under their feet.}
\setauthornote{6675}{Rom. \rn{xii.} 2.}
\setauthornote{6676}{Omnis Aristippum decuit color, et status, et res.}
\setauthornote{6677}{Psal. \rn{xiii.} 1.}
\setauthornote{6678}{Guicciardini.}
\setauthornote{6679}{Erasmus.}
\setauthornote{6680}{Hierom.}
\setauthornote{6681}{Senec. consol. ad Polyb. ca. 21.}
\setauthornote{6682}{Disput. 4. Philosophiae adver. Atheos. Venetiis 1627, quarto.}
\setauthornote{6683}{Edit. Romae, fol. 1631.}
\setauthornote{6684}{Abernethy, c. 24. of his Physic of the Soul.}
\setauthornote{6685}{Omissa spe victoriae in destinatam mortem conspirant, tantusque ardor singulos cepit, ut victores se putarent si non inulti morerentur. Justin. l. 20.}
\setauthornote{6686}{Method. hist. cap. 5.}
\setauthornote{6687}{Hosti abire volenti iter minime interscindas, \&c.}
\setauthornote{6688}{Poster volum.}
\setauthornote{6689}{Super praeceptum primum de Relig. et partibus ejus. Non loquor de omni desperatione, sed tantum de ea qua desperare solent homines de Deo; opponitur spei, et est peccatum gravissimum, \&c.}
\setauthornote{6690}{Lib. 5. lit. 21. de regis institut. Omnium pertubationum deterrima.}
\setauthornote{6691}{Reprobi usque ad finem pertinaciter persistunt. Zanchius.}
\setauthornote{6692}{Vitium ab infidelitate proficiscens.}
\setauthornote{6693}{Abernethy.}
\setauthornote{6694}{1 Sam. \rn{ii.} 16.}
\setauthornote{6695}{Psal. \rn{xxxviii.} vers. 9. 14.}
\setauthornote{6696}{Immiscent se mali genii, Lem. lib. 1. cap. 16.}
\setauthornote{6697}{Cases of conscience, l. 1. 16.}
\setauthornote{6698}{Tract. Melan. capp. 33 et 34.}
\setauthornote{6699}{Cap. 3. de mentis alien. Deo minus se curae esse, nec ad salutem praedestinatos esse. Ad desperationem saepe ducit haec melancholia, et est frequentissima ob supplicii metum aeternumque judicium; meror et metus in desperationem plerumque desinunt.}
\setauthornote{6700}{Comment. in 1. cap. gen. artic. 3. quia impii florent boni opprimuntur, \&c. alius ex consideratione hujus seria desperabundus.}
\setauthornote{6701}{Lib. 20. c. 17.}
\setauthornote{6702}{Damnatam se putavit, et quatuor menses Gehennae poenam sentire.}
\setauthornote{6703}{1566. ob triticum diutius servatum conscientiae stimulis agitatur, \&c.}
\setauthornote{6704}{Tom. 2. c. 27. num. 282. conversatio cum scrupulosis, vigiliae, jejunia.}
\setauthornote{6705}{Solitarios et superstitiosos plerumque exagitat conscientia, non mercatores, lenones, caupones, foeneratore?, \&c. largiorem hi nacti sunt conscientiam. Juvenes plerumque conscientiam negligunt, senes autem, \&c.}
\setauthornote{6706}{Annon sentis sulphur inquit?}
\setauthornote{6707}{Desperabundus misere periit.}
\setauthornote{6708}{In 17. Johannis. Non pauci se cruciant, et excarnificant in tantum, ut non parum absint ab insania; neque tamen aliud hac mentis anxietate efficiunt, quam ut diabolo potestatem faciant ipsos per desperationem ad infernos producendi.}
\setauthornote{6709}{Drexelius Nicet. lib. 2. cap. 11. Eternity, that word, that tremendous word, more threatening than thunders and the artillery of heaven-Eternity, that word, without end or origin. No torments affright us which are limited to years: Eternity, eternity, occupies and inflames the heart-this it is that daily augments our sufferings, and multiplies our heart-burnings a hundredfold.}
\setauthornote{6710}{Ecclesiast. 1. 1. Haud scio an majus discrimen ab his qui blandiuntur, an ab his qui territant; ingens utrinque periculum: alii ad securitatem ducunt, alii afflictionum magnitudine mentem absorbent, et in desperationem trahunt.}
\setauthornote{6711}{Bern. sup. 16. cant. 1. alterum sine altero proferre non expedit; recordatio solius judicii in desperationem praecipitat, et misericordis; fallax ostentatio pessimam generat securitatem.}
\setauthornote{6712}{In Luc. hom. 103. exigunt ab aliis charitatem, beneficentiam, cum ipsi nil spectent praeter libidinem, invidiam, avaritiam.}
\setauthornote{6713}{Leo Decimus.}
\setauthornote{6714}{Deo futuro judicio, de damnatione horrendum crepunt, et amaras illas potationes in ore semper habent, ut multos inde in desperationem cogant.}
\setauthornote{6715}{Euripides. O wretched Orestes, what malady consumes you?}
\setauthornote{6716}{Conscience, for I am conscious of evil.}
\setauthornote{6717}{Pierius.}
\setauthornote{6718}{Gen. iv.}
\setauthornote{6719}{9 causes Musculus makes.}
\setauthornote{6720}{Plutarch.}
\setauthornote{6721}{Alios misere castigat plena scrupulis conscientia, nodum in scirpo quaerunt, et ubi nulla causa subest, misericordiae divinae diffidentes, se Oreo destinant.}
\setauthornote{6722}{Coelius, lib. 6.}
\setauthornote{6723}{Juvenal. Night and day they carry their witnesses in the breast.}
\setauthornote{6724}{Lucian. de dea Syria. Si adstiteris, te aspicit; si transeas, visu te sequitur.}
\setauthornote{6725}{Prima haec est ultio, quod se judice nemo nocens absolvitur, improba quamvis gratia fallacis praetoris vicerit urnam. Juvenal.}
\setauthornote{6726}{Quis unquam vidit avarum ringi, dum lucrum adest, adulterum dum potitur voto, lugere in perpetrando scelere? voluptate sumus ebrii, proinde non sentimus, \&c.}
\setauthornote{6727}{Buchanan, lib. 6. Hist. Scot.}
\setauthornote{6728}{Animus conscientia sceleris inquietus, nullum admisit gaudium, sed semper vexatus noctu et interdiu per somnum visis horrore plenis putremefactus, \&c.}
\setauthornote{6729}{De bello Neapol.}
\setauthornote{6730}{Thirens de locis infestis, part. 1. cap. 2. Nero's mother was still in his eyes.}
\setauthornote{6731}{Psal. \rn{xliv.} 1.}
\setauthornote{6732}{And Nemesis pursues and notices the steps of men, lest you commit any evil.}
\setauthornote{6733}{Regina causarum et arbitra rerum, nunc erectas cervices opprimit, \&c.}
\setauthornote{6734}{Alex. Gaguinus catal. reg. Pol.}
\setauthornote{6735}{Cosmog. Munster, et Magde.}
\setauthornote{6736}{Plinius, cap. 10. l. 35. Consumptis affectibus, Agamemnonis caput velavit, ut omnes quem possent, maximum moerorem in virginis patre cogitarent.}
\setauthornote{6737}{Cap. 15. in 9. Rhasis.}
\setauthornote{6738}{Juv. Sat. 13.}
\setauthornote{6739}{Mentem eripit timor hic; vultum, totumque corporis habitum immutat, etiam in deliciis, in tripudiis, in symposiis, in amplexu conjugis carnificinam exercet, lib. 4. cap. 21.}
\setauthornote{6740}{Non sinit conscientia tales homines recta verba proferre, aut rectis quenquam oculis aspicere, ab omni hominum coetu eosdem exterminat, et dormientes perterrefacit. Philost. lib. 1. de vita Apollonii.}
\setauthornote{6741}{Eusebius, Nicephorus eccles. hist. lib. 4. c. 17.}
\setauthornote{6742}{Seneca, lib. 18. epist. 106. Conscientia aliud agere non patitur, perturbatam vitam agunt, nunquam vacant, \&c.}
\setauthornote{6743}{Artic. 3. ca. 1. fol. 230. quod horrendum dictu, desperabundus quidam me presente cum ad patientiam hortaretur, \&c.}
\setauthornote{6744}{Lib. 1. obser. cap. 3.}
\setauthornote{6745}{Ad maledicendum Deo.}
\setauthornote{6746}{Goulart.}
\setauthornote{6747}{Dum haec scribo, implorat opem meam monacha, in reliquis sana, et judicio recta, per. 5. annos melancholica; damnatum se dicit, conscientiae stimultis oppressa, \&c.}
\setauthornote{6748}{Alios conquerentes audivi se esse ex damnatorum numero. Deo non esse curae aliaque infinita quae proferre non audebant, vel abhorrebant.}
\setauthornote{6749}{Musculus, Patritius, ad vim sibi inferendam cogit homines.}
\setauthornote{6750}{De mentis alienat. observ. lib. 1.}
\setauthornote{6751}{Uxor Mercatoris diu vexationibus tentata, \&c.}
\setauthornote{6752}{Abernethy.}
\setauthornote{6753}{Busbequius.}
\setauthornote{6754}{John Major vitis patrum: quidam negavit Christum, per Chirographum post restitutus.}
\setauthornote{6755}{Trincavelius lib. 3.}
\setauthornote{6756}{My brother, George Burton, M. James Whitehall, rector of Checkley, in Staffordshire, my quondam chamber-fellow, and late fellow student in Christ Church, Oxon.}
\setauthornote{6757}{Scio quam vana sit et inefficax humanorum verborum penes afflictos consolatio, nisi verbum Dei audiatur, a quo vita, refrigeratio, solatium, poenitentia.}
\setauthornote{6758}{Antid. adversus desperationem.}
\setauthornote{6759}{Tom. 2. c. 27. num. 282.}
\setauthornote{6760}{Aversio cogitationis a re scrupulosa, contraventio scrupulorum.}
\setauthornote{6761}{Magnam injuriam Deo facit qui diffidit de ejus misericordia.}
\setauthornote{6762}{Bonitas invicti non vincitur; infiniti misericordia non finitur.}
\setauthornote{6763}{Hom. 3. De poenitentia: Tua quidem malitia mensuram habet. Dei autem misericordia mensuram non habet. Tua malitia circumscripta est, \&c. Pelagus etsi magnum mensuram habet; dei autem, \&c.}
\setauthornote{6764}{Non ut desidiores vos faciam, sed ut alacriores reddam.}
\setauthornote{6765}{Pro peccatis veniam poscere, et mala de novo iterare.}
\setauthornote{6766}{Si bis, si ter, si centies, si centies millies, toties poenitentiam age.}
\setauthornote{6767}{Conscientia mea meruit damnationem, poenitentia non sufficit ad satisfactionem: sed tua misericordia superat omnem offensionem.}
\setauthornote{6768}{Multo efficacior Christi mors in bonum, quam peccata nostra in malum. Christus potentior ad salvandum, quam daemon ad perdendum.}
\setauthornote{6769}{Peritus medicus potest omnes infirmitates sanare; si misericors, vult.}
\setauthornote{6770}{Omnipotenti medico nullus languor insanabilis occurrit: tu tantum doceri te sine, manum ejus ne repelle: novit quid agat; non tantum delecteris cum fovet, sed toleres quum secat.}
\setauthornote{6771}{Chrys. hom. 3. de poenit.}
\setauthornote{6772}{Spes salutis per quam peccatores salvantur, Deus ad misericordiam provocatur. Isidor. omnia ligata tu solvis, contrita sanas, confusa lucidas, desperata animas.}
\setauthornote{6773}{Chrys. hom 5. non fornicatorem abnuit, non ebrium avertit, non superbum repellit, non aversatur Idololatram, non adulterum, sed omnes suscipit, omnibus communicat.}
\setauthornote{6774}{Chrys. hom. 5.}
\setauthornote{6775}{Qui turpibus cantilenis aliquando inquinavit os, divinis hymnis animum purgabit.}
\setauthornote{6776}{Hom. 5. Introivit hic quis accipiter, columba exit; introivit lupus, ovis egreditur, \&c.}
\setauthornote{6777}{Omnes languores sanat, caecis visum, claudis gressum, gratiam confert, \&c.}
\setauthornote{6778}{Seneca. He who repents of his sins is well nigh innocent.}
\setauthornote{6779}{Delectatur Deus conversione peccatoris; omne tempus vitae conversioni deputatur; pro praesentibus habentur tam praeterita quam futura.}
\setauthornote{6780}{Austin. Semper poenitentiae portus apertus est ne desperemus.}
\setauthornote{6781}{Quicquid feceris, quantumcunque peccaveris, adhuc in vita es, unde te omnino si sanare te nollet Deus, auferret; parcendo clamat ut redeas, \&c.}
\setauthornote{6782}{Matt. vi. 23.}
\setauthornote{6783}{Rev. xxi. 6.}
\setauthornote{6784}{Abernethy, Perkins.}
\setauthornote{6785}{Non est poenitentia, sed Dei misericordia annexa.}
\setauthornote{6786}{Caecilius Minutio, Omnia ista figmenta mala sanae religionis, et inepta solatia a poetis inventa, vel ab aliis ob commodum, superstitiosa misteria, \&c.}
\setauthornote{6787}{These temptations and objections are well answered in John Downam's Christian Warfare.}
\setauthornote{6788}{Seneca.}
\setauthornote{6789}{Licinus lies in a marble tomb, but Cato in a mean one; Pomponius has none, who can think therefore that there are Gods?}
\setauthornote{6790}{Vid. Campanella cap. 6. Atheis. triumphal, et c. 2. ad argumentum 12. ubi plura. Si Deus bonus unde colum, \&c.}
\setauthornote{6791}{Lucan. It can't be true that Just Jove reigns.}
\setauthornote{6792}{Perkins.}
\setauthornote{6793}{Hemingius. Nemo peccat in spiritum sanctum nisi qui finaliter et voluntarie renunciat Christum, eumque et ejus verbum extreme contemnit, sine qua nulla salus; a quo peccato liberet nos Dominus Jesus Christus. Amen.}
\setauthornote{6794}{Abernethy.}
\setauthornote{6795}{See whole books of these arguments.}
\setauthornote{6796}{Lib. 3. fol. 122. Praejudicata opinio, invida, maligna, et apta ad impellendos animos in desperationem.}
\setauthornote{6797}{See the Antidote in Chamier's tom. 3. lib. 7. Downam's Christian Warfare, \&c.}
\setauthornote{6798}{Potentior est Deo diabolus et mundi princeps, et in multitudine hominum sita est majestas.}
\setauthornote{6799}{Homicida qui non subvenit quum potest; hoc de Deo sine scelere cogitari non potest, utpote quum quod vult licet. Boni natura communicari. Bonus Deus, quomodo misericordiae, pater, \&c.}
\setauthornote{6800}{Vide Cyrillum lib. 4. adversus Julianum, qui poterimus illi gratias agere qui nobis non misit Mosen et prophetas, et contempsit boni amimarum nostrarum.}
\setauthornote{6801}{Venia danda est iis qui non audiunt ob ignoratiam. Non est tam iniquus Judex Deus: ut quenquam indicia causa damnare velit. Ii solum damnantur, qui oblatam Christi gratium rejiciunt.}
\setauthornote{6802}{Busbequius Lonicerus, Tur. hist. To. 1 l. 2.}
\setauthornote{6803}{Olem. Alex.}
\setauthornote{6804}{Paulus Jovius Elog. vir. Illust.}
\setauthornote{6805}{Non homines sed et ipsi daemones aliquando servandi.}
\setauthornote{6806}{Vid Pelsii Harmoniam art. 22. p. 2.}
\setauthornote{6807}{Epist. Erasmi de utilitate colloquior. ad lectorem.-Let whoever wishes dispute, I think the laws of our forefathers should be received with reverence, and religiously observed, as coming from God; neither is it safe or pious to conceive, or contrive, an injurious suspicion of the public authority; and should any tyranny, likely to drive men into the commission of wickedness, exist, it is better to endure it than to resist it by sedition.}
\setauthornote{6808}{Vastata conscientia sequitur sensus irae divinae. (Hemingius) fremitus cordis, ingens animae cruciatus, \&c.}
\setauthornote{6809}{Austin.}
\setauthornote{6810}{not from pleasures to pleasures}
\setauthornote{6811}{Super Psal. \rn{lii.} Convertar ad liberandum eum, quia conversus est ad peccatum suum puniendum.}
\setauthornote{6812}{Antiqui soliti sunt hanc herbam ponere in coemiteriis ideo quod, \&c.}
\setauthornote{6813}{Non desunt nostra aetate sacrificuli, qui tale quid attentant, sed a cacodaemone irrisi pudore suffecti sunt et re infecta abicrunt.}
\setauthornote{6814}{Done into English by W. B., 1613.}
\setauthornote{6815}{Tom. 2. cap. 27, num. 282. Let him avert his thoughts from the painful object.}
\setauthornote{6816}{Navarrus.}
\setauthornote{6817}{Is. l. 4.}
