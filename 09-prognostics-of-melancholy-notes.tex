\setauthornote{2716}{Si melancholicis haemorroides supervenerint varices, vel ut quibusdam placet, aqua inter cutem, solvilur malum.}
\setauthornote{2717}{Cap. 10. de quartana.}
\setauthornote{2718}{Cum sanguis exit per superficiem et residet melancholia per scabiem, morpheam nigram, vel expurgatur per inferiores partes, vel urinam, \&c., non erit, \&c. spen magnificatur et varices apparent.}
\setauthornote{2719}{Quia jam conversa in naturam.}
\setauthornote{2720}{In quocunque sit a quacunque causa Hypocon. praesertim, semper est longa, morosa, nec facile curari potest.}
\setauthornote{2721}{Regina morborum et inexorabilis.}
\setauthornote{2722}{Omne delirium quod oritur a paucitate cerebri incurabile, Hildesheim, spicel. 2. de mania.}
\setauthornote{2723}{Si sola imaginatio laedatur, et non ratio.}
\setauthornote{2724}{Mala a sanguine fervente, deterior a bile assata, pessima ab atra bile putrefacta.}
\setauthornote{2725}{Difficilior cura ejus quae fit vitio corporis totius et cerebri.}
\setauthornote{2726}{Difficilis curatu in viris, multo difficilio in faeminis.}
\setauthornote{2727}{Ad interitum plerumque homines comitatur, licet medici levent plerumque, tamen non tollunt unquam, sed recidet acerbior quam antea minima occasione, aut errore.}
\setauthornote{2728}{Periculum est ne degenereret in Epilepsiam, Apoplexiam, Convulsionem, caecitatem.}
\setauthornote{2729}{Montal. c. 25. Laurentius. Nic. Piso.}
\setauthornote{2730}{Her. de Soxonia, Aristotle, Capivaccius.}
\setauthornote{2731}{Favent. Humor frigidus sola delirii causa, furoris vero humor calidus.}
\setauthornote{2732}{Heurnius calls madness sobolem malancholiae.}
\setauthornote{2733}{Alesander l. 1. c. 18.}
\setauthornote{2734}{Lib. 1. part. 2. c. 11.}
\setauthornote{2735}{Montalt. c. 15. Raro mors aut nunquam, nisi sibi ipsis inferant.}
\setauthornote{2736}{Lib. de Insan. Fabio Calico Interprete.}
\setauthornote{2737}{Nonulli violentas manus sibi inferunt.}
\setauthornote{2738}{Lucret. l. 3.}
\setauthornote{2739}{Lib. 2. de intell. saepe mortem sibi consciscunt ob timorem et tristitiam taedeio vitae affecti ob furorem et desperationem. Est enim infera, \&c. Ergo sic perpetuo afflictati vitam oderunt, se praecipitant, his malis carituri aut interficiunt se, aut tale quid committunt.}
\setauthornote{2740}{Psal. cvii. 10.}
\setauthornote{2741}{Job xxxiii.}
\setauthornote{2742}{Job. vi. 8.}
\setauthornote{2743}{Vi doloris et tristitiae ad insaniam pene redactus.}
\setauthornote{2744}{Seneca.}
\setauthornote{2745}{In salutis suae desperatione proponunt sibi mortis desiderium, Oct. Horat l. 2. c. 5.}
\setauthornote{2746}{Lib. de insania. Sic sic juvat ire per umbras.}
\setauthornote{2747}{Cap. 3. de mentis alienat. maesti degunt, dum tandem mortem quam timent, suspendio aut submersione, aut aliqua alia vi, ut multa tristia exempla vidimus.}
\setauthornote{2748}{Arculanus in 9. Rhasis, c. 16. cavendum ne ex alto se praecipitent aut alias laedant.}
\setauthornote{2749}{O omnium opinionibus incogitabile malum. Lucian. Mortesque mille, mille dum vivit neces gerit, peritque Hensius Austriaco.}
\setauthornote{2750}{Regina morborum cui famulantur omnes et obediunt. Cardan.}
\setauthornote{2751}{Eheu quis intus Scorpio, \&c. Seneca Act. 4. Herc. O Et.}
\setauthornote{2752}{Silius Italicus.}
\setauthornote{2753}{Lib. 29.}
\setauthornote{2754}{Hic omnis imbonitas et insuavitas consistit, ut Tertulliani verbis utar, orat. ad. martyr.}
\setauthornote{2755}{\Plautus{}.}
\setauthornote{2756}{Vit. Herculis.}
\setauthornote{2757}{Persius.}
\setauthornote{2758}{Quid est miserius in vita, quam velle mori? Seneca.}
\setauthornote{2759}{Tom. 2. Libello, an graviores passiones, \&c.}
\setauthornote{2760}{Ter.}
\setauthornote{2761}{Patet exitus; si pugnare non vultis, licet fugere; quis vos tenet invitos? De provid. cap. 8.}
\setauthornote{2762}{Agamus Deo gratias, quod nemo invitus in vita teneri potest.}
\setauthornote{2763}{Epist. 26. Seneca et de sacra. 2. cap. 15. et Epist. 70. et 12.}
\setauthornote{2764}{Lib. 2. cap. 83. Terra mater nostri miserta.}
\setauthornote{2765}{Epist. 24. 71. 22.}
\setauthornote{2766}{Mac. 14. 42.}
\setauthornote{2767}{Vindicatio Apoc. lib.}
\setauthornote{2768}{Finding that he would be destined to endure excruciating pain of the feet, and additional tortures, he abstained from food altogether.}
\setauthornote{2769}{As amongst Turks and others.}
\setauthornote{2770}{Bohemus de moribus gent.}
\setauthornote{2771}{Aelian. lib. 4. cap. 1. omnes 70. annum egressos interficiunt.}
\setauthornote{2772}{Lib. 2. Praesertim quum tormentum ei vita sit, bona spe fretus, acerba vita velut a carcere se eximat, vel ab aliis eximi sua voluntate patiatur.}
\setauthornote{2773}{Nam quis amphoram exsiccans foecem exorberet (Seneca epist. 58.) quis in poenas et risum viveret? stulti est manere in vita cum sit miser.}
\setauthornote{2774}{Expedit. ad Sinas l. 1. c. 9. Vel bonorum desperatione, vel malorum perpessione fracti et fagitati, vel manus violentas sibi inferunt vel ut inimicis suis aegre faciant, \&c.}
\setauthornote{2775}{No one ever died in this way, who would not have died some time or other; but what does it signify how life itself may be ended, since he who comes to the end is not obliged to die a second time?}
\setauthornote{2776}{So did Anthony, Galba, Vitellius, Otho, Aristotle himself, \&c. Ajax in despair; Cleopatra to save her honour.}
\setauthornote{2777}{Incertius deligitur diu vivere quam in timore tot morborum semel moriendo, nullum deinceps formidare.}
%\setauthornote{2778}{And now when Ambrociotes was bidding farewell to the light of day, and about to cast himself into the Stygian pool, although he had not been guilty of any crime that merited death: but, perhaps, he had read that divine work of Plato upon Death.}
\setauthornote{2779}{Curtius l. 16.}
\setauthornote{2780}{Laqueus praecisus, cont. 1. l. 5. quidam naufragio facto, amissis tribus liberis, et uxore, suspendit se; praecidit illi quidam ex praetereuntibus laqueum: A liberato reus fit maleficii. Seneca.}
\setauthornote{2781}{See Lipsius Manuduc. ad Stoicam philosophiam lib. 3. dissert. 22. D. Kings 14. Lect. on Jonas. D. Abbot's 6 Lect. on the same prophet.}
\setauthornote{2782}{\Plautus{}.}
\setauthornote{2783}{Martial.}
\setauthornote{2784}{As to be buried out of Christian burial with a stake. Idem. Plato 9. de legibus, vult separatim sepeliri, qui sibi ipsis mortem consciscunt, \&c. lose their goods, \&c.}
\setauthornote{2785}{Navis destitutae nauclero, in terribilem aliquem scopulum impingit.}
\setauthornote{2786}{Observat.}
\setauthornote{2787}{Seneca tract. 1. 1. 8. c. 4. Lex Homicida in se insepultus abjiciatur contradicitur; Eo quod afferre sibi manus coactus sit assiduis malis: summam infelicitatem suam in hoc removit, quod existimabat licere misero mori.}
\setauthornote{2788}{Buchanan, Eleg. lib.}
