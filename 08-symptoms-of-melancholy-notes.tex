\setmarginnote{2452}{Seneca cont. lib. 10. cont. 5.}
\setmarginnote{2453}{Quaedam universalia, particulariae, quaedam manifesta, quaedam in corpore, quaedam in cogitatione et animo, quaedam a stellis, quaedam ab humoribus, quae ut vinum corpus varie disponit, \&c. Diversa phantasmata pro varietate causae externae, internae.}
\setmarginnote{2454}{Lib. 1. de risu. fol. 17. Ad ejus esum alii sudant, alii vomunt, stent, bibunt, saltant, alii rident, tremunt, dormiunt, \&c.}
\setmarginnote{2455}{T. Bright. cap. 20.}
\setmarginnote{2456}{Nigrescit hic humer aliquando supercalefactus, aliquando superfrigefactus. Melanel. a Gal.}
\setmarginnote{2457}{Interprete F. Calvo.}
\setmarginnote{2458}{Oculi his excavantur, venti gignuntur circum praecordia et acidi ructus, sicci fere ventres, vertigo, tinnitus aurium, somni pusilli, somnia terribilia et interrupta.}
\setmarginnote{2459}{Virg. Aen.}
\setmarginnote{2460}{Assiduae eaeque acidae ructationes quae cibum virulentum culentumque nidorem, et si nil tale ingestum sit, referant ob cruditatem. Ventres hisce aridi, somnus plerumque parcus et interruptus, somnia absurdissima, turbulenta, corporis tremor, capitis gravedo, strepitus circa aures et visiones ante oculos, ad venerem prodigi.}
\setmarginnote{2461}{Altomarus, Bruel, Piso, Montaltus.}
\setmarginnote{2462}{Frequentes habent oculorum nictationes, aliqui tamen fixis oculis plerumque sunt.}
\setmarginnote{2463}{Cent. lib. 1. Tract. 9. Signa hujus morbi sunt plurimus saltus, sonitus aurium, capitis gravedo, lingua titubat, oculi excavantur, \&c.}
\setmarginnote{2464}{In Pantheon cap. de Melancholia.}
\setmarginnote{2465}{Alvus arida nihil dejiciens cibi capaces, nihilominus tamen extenuati sunt.}
\setmarginnote{2466}{Nic Piso Inflatio carotidum, \&c.}
\setmarginnote{2467}{Andreas Dudith Rahamo. cp. lib. 3. Crat epist. multa in pulsibus superstitio, ausim etiam dicere, tot differentias quae describuntur a Galeno, neque intelligi a quoquam nec observari posse.}
\setmarginnote{2468}{T. Bright. cap. 20.}
\setmarginnote{2469}{Post. 40. aetat. annum, saith Jacchinus in 15. 9. Rhasis Idem. Mercurialis consil. 86. Trincavelius, Tom. 2. cons. 17.}
\setmarginnote{2470}{Gordonius, modo rident, modo flent, silent, \&c.}
\setmarginnote{2471}{Fernelius consil. 43. et 45. Montanus consil. 230. Galen de locis affectis, lib. 3 cap. 6.}
\setmarginnote{2472}{Aphorism et lib. de Melan.}
\setmarginnote{2473}{Lib. 2. cap. 6. de locis affect. timor et moestitia, si diutius perseverent, \&c.}
\setmarginnote{2474}{Tract. posthumo de Melan. edit. Venetiis 1620. per Bolzettam Bibliop. Mihi diligentius hanc rem consideranti, patet quosdam esse, qui non laborant maerore et timore.}
\setmarginnote{2475}{Prob. lib. 3.}
\setmarginnote{2476}{Physiog lib. 1. c. 8. Quibus multa frigida bilis atra, stolidi et timidi, at qui calidi, ingeniosi, amasii, divinosi, spiritu instigati, \&c.}
\setmarginnote{2477}{Omnes exercent metus et tristitia, et sine causa.}
\setmarginnote{2478}{Omnes timent licet non omnibus idem timendi modus Aetius Tetrab. lib. 2. sect. c. 9.}
\setmarginnote{2479}{Ingenti pavore trepidant.}
\setmarginnote{2480}{Multi mortem timent, et tamen sibi ipsis mortem consciscunt, alii coeli ruinam timent.}
\setmarginnote{2481}{Affligit eos plena scrupulis conscientia, divinae misericordiae diffidentes, Orco se destinant foeda lamentatione deplorantes.}
\setmarginnote{2482}{Non ausus egredi domo ne deficeret.}
\setmarginnote{2483}{Multi daemones timent, latrones, insidias, Avicenna.}
\setmarginnote{2484}{Alii comburi, alii de Rege, Rhasis.}
\setmarginnote{2485}{Ne terra absorbeantur. Forestus.}
\setmarginnote{2486}{Ne terra dehiscat. Gordon.}
\setmarginnote{2487}{Alii timore mortis timentur et mala gratia principum putant se aliquid commisisse et ad supplicium requiri.}
\setmarginnote{2488}{Alius domesticos timet, alius omnes. Aetius.}
\setmarginnote{2489}{Alii timent insidias. Aurel. lib. 1. de morb. Chron. cap. 6.}
\setmarginnote{2490}{Ille charissimos, hic omnes homines citra discrimen timet.}
\setmarginnote{2491}{Virgil.}
\setmarginnote{2492}{Hic in lucem prodire timet, tenebrasque quaerit, contra, ille caliginosa fugit.}
\setmarginnote{2493}{Quidam larvas, et malos spiritus ab inimicis veneficius et incantationibus sibi putant objectari, Hippocrates, potionem se veneficam sumpsisse putat, et de hac ructare sibi crebro videtur. Idem Montaltus cap. 21. Aetius lib. 2. et alii. Trallianus l. 1. cap. 16.}
\setmarginnote{2494}{Observat. l. 1. Quando iis nil nocet, nisi quod mulieribus melancholicis.}
\setmarginnote{2495}{tamen metusque causae nescius, causa est metus. Heinsius Austriaco.}
\setmarginnote{2496}{Cap. 15. in 9. Rhasis, in multis vidi, praeter rationem semper aliquid timent, in caeteris tamen optime se gerunt, neque aliquid praeter dignitatem committunt.}
\setmarginnote{2497}{Altomarus cap. 7. Areteus, triste, sunt.}
\setmarginnote{2498}{Mant. Egl. 1.}
\setmarginnote{2499}{Ovid. Met. 4.}
\setmarginnote{2500}{Inquies animus.}
\setmarginnote{2501}{Hor. l. 3. Od. 1. Dark care rides behind him.}
\setmarginnote{2502}{Virg.}
\setmarginnote{2503}{Mened. Heautont. Act. 1. sc. 1.}
\setmarginnote{2504}{Altomarus.}
\setmarginnote{2505}{Seneca.}
\setmarginnote{2506}{Cap. 31. Quo stomachi dolore correptum se, etiam de consciscenda morte cogitasse dixit.}
\setmarginnote{2507}{Luget et semper tristatur, solitudinem amat, mortem sibi precatur, vitam propriam odio habet.}
\setmarginnote{2508}{Facile in iram incidunt. Aret.}
\setmarginnote{2509}{Ira sine causa, velocitas irae. Savanarola. pract. major. velocitas irae signum. Avicenna l. 3. Fen. 1. Tract. 4. cap. 18. Angor sine causa.}
\setmarginnote{2510}{Suspicio, diffidentia, symptomata, Crato Ep. Julio Alexandrino cons. 185 Scoltzii.}
\setmarginnote{2511}{Hor. At Rome, wishing for the fields, in the country, extolling the city to the skies.}
\setmarginnote{2512}{Pers. Sat. 3. And like the children of nobility, require to eat pap, and, angry at the nurse, refuse her to sing lullaby.}
\setmarginnote{2513}{In his Dutch work picture.}
\setmarginnote{2514}{Howard cap. 7. differ.}
\setmarginnote{2515}{Tract. de mel. cap. 2. Noctu ambulant per sylvas, et loca periculosa, neminem timent.}
\setmarginnote{2516}{Facile amant. Altom.}
\setmarginnote{2517}{Bodine.}
\setmarginnote{2518}{Io. Major vitis patrum fol. 202. Paulus Abbas Eremita tanta solitudine, perseverat, ut nec vestem, nec vultum mulieris ferre possit, \&c.}
\setmarginnote{2519}{Consult, lib. 1. 17. Cons.}
\setmarginnote{2520}{Generally as they are pleased or displeased, so are their continual cogitations pleasing or displeasing.}
\setmarginnote{2521}{Omnes excercent, vanae intensaeque animi cogitationes, (N. Piso Bruel) et assiduae.}
\setmarginnote{2522}{Curiosi de rebus minimis. Areteus.}
\setmarginnote{2523}{Lib. 2. de Intell.}
\setmarginnote{2524}{Hoc melancholicis omnibus proprium, ut quas semel imaginationes valde reciperint, non facile rejiciant, sed hae etiam vel invitis semper occurrant.}
\setmarginnote{2525}{Tullius de sen.}
\setmarginnote{2526}{Consil. med. pro Hypochondriaco.}
\setmarginnote{2527}{Consil. 43.}
\setmarginnote{2528}{Cap. 5.}
\setmarginnote{2529}{Lib. 2. de Intell.}
\setmarginnote{2530}{Consult. 15. et 16. lib. 1.}
\setmarginnote{2531}{Virg. Aen. 6.}
\setmarginnote{2532}{Iliad. 3.}
\setmarginnote{2533}{Si malum exasperantur, homines odio habent et solitaria petunt.}
\setmarginnote{2534}{Democritus solet noctes et dies apud se degere, plerumque autem in speluncis, sub amaenis arborum umbris vel in tenebris, et mollibus herbis, vel ad aquarum crebra et quieta fluenta, \&c.}
\setmarginnote{2535}{Gaudet tenebris, aliturque dolor. Ps. lxii. Vigilavi et factus sum velut nycticorax in domicilio, passer solitarius in templo.}
\setmarginnote{2536}{Et quae vix audet fabula, monstra parit.}
\setmarginnote{2537}{In cap. 18. l. 10. de civ. dei, Lunam ab Asino epotam videus.}
\setmarginnote{2538}{Vel. l. 4. c. 5.}
\setmarginnote{2539}{Sect. 2. Memb. 1. Subs. 4.}
\setmarginnote{2540}{De reb. coelest. lib. 10. c. 13.}
\setmarginnote{2541}{l. de Indagine Goclenius.}
\setmarginnote{2542}{Hor. de art. poet.}
\setmarginnote{2543}{Tract. 7. de Melan.}
\setmarginnote{2544}{Humidum, calidum, frigidum, siccum.}
\setmarginnote{2545}{Com. in 1 c. Johannis de Sacrobosco.}
\setmarginnote{2546}{Si residet melancholia naturalis, tales plumbei coloris aut nigri, stupidi, solitarii.}
\setmarginnote{2547}{Non una melancholiae causa est, nec unus humor vitii parens, sed plures, et alius aliter mutatus, unde non omnes eadem sentiunt symptomata.}
\setmarginnote{2548}{Humor frigidus delirii causa, humor calidus furoris.}
\setmarginnote{2549}{Multum refert qua quisque melancholia teneatur, hunc fervens et accensa agitat, illum tristis et frigens occupat: hi timidi, illi inverecundi, intrepidi, \&c.}
\setmarginnote{2550}{Cap. 7. et 8. Tract. de Mel.}
\setmarginnote{2551}{Signa melancholiae ex intemperie et agitatione spirituum sine materia.}
\setmarginnote{2552}{T. Bright cap. 16. Treat. Mel.}
\setmarginnote{2553}{Cap. 16. in 9. Rhasis.}
\setmarginnote{2554}{Bright, c. 16.}
\setmarginnote{2555}{Pract. major. Somnians, piger, frigidus.}
\setmarginnote{2556}{De anima cap. de humor. si a Phlegmate semper in aquis fere sunt, et circa fluvios plorant multum.}
\setmarginnote{2557}{Pigra nascitur ex colore pallido et albo, Her. de Saxon.}
\setmarginnote{2558}{Savanarola.}
\setmarginnote{2559}{Muros cadere in se, aut submergi timent, cum torpore et segnitie, et fluvios amant tales, Alexand. c. 16. lib. 7.}
\setmarginnote{2560}{Semper fere dormit somnolenta c. 16. l. 7.}
\setmarginnote{2561}{Laurentius.}
\setmarginnote{2562}{Ca. 6. de mel. Si a sanguine, venit rubedo oculorum et faciei, plurimus risus.}
\setmarginnote{2563}{Venae oculorum sunt rubrae, vide an praecesserit vini et aromatum usus, et frequens balneum, Trallian. lib. 1. 16. an praecesserit mora sub sole.}
\setmarginnote{2564}{Ridet patiens si a sanguine, putat se videre choreas, musicam audire, ludos, \&c.}
\setmarginnote{2565}{Cap. 2. Tract. de Melan.}
\setmarginnote{2566}{Hor. ep. lib. 2. quidam haud ignobilis Argis, \&c.}
\setmarginnote{2567}{Lib. de reb. mir.}
\setmarginnote{2568}{Cum inter concionandum mulier dormiens e subsellio caderet, et omnes reliqui qui id viderent, riderent, tribus post diebus, \&c.}
\setmarginnote{2569}{Juvenis et non vulgaris eruditionis.}
\setmarginnote{2570}{Si a cholera, furibundi, interficiunt, se et alios, putant se videre pugnas.}
\setmarginnote{2571}{Urina subtilis et ignea, parum dormiunt.}
\setmarginnote{2572}{Tract. 15. c. 4.}
\setmarginnote{2573}{Ad haec perpetranda furore rapti ducuntur, cruciatus quosvis tolerant, et mortem, et furore exacerbato audent et ad supplicia plus irritantur, mirum est quantam habeant in tormentis patientiam.}
\setmarginnote{2574}{Tales plus caeteris timent, et continue tristantur, valde suspiciosi, solitudinem diligunt, corruptissimas habent imaginationes, \&c.}
\setmarginnote{2575}{Si a melancholia adusta, tristes, de sepulchris somniant, timent ne fascinentur, putant se mortuos, aspici nolunt.}
\setmarginnote{2576}{Videntur sibi videre monachos nigros et daemonos, et suspensos et mortuos.}
\setmarginnote{2577}{Quavis nocte se cum daemone coire putavit.}
\setmarginnote{2578}{Semper fere vidisse militem nigrum praesentem.}
\setmarginnote{2579}{Anthony de Verdeur.}
\setmarginnote{2580}{Quidam mugitus boum aemulantur, et pecora se putant, ut Praeti filiae.}
\setmarginnote{2581}{Baro quidam mugitus boum et rugitus asinorum, et aliorum animalium voces effingit.}
\setmarginnote{2582}{Omnia magna putabat, uxorem magnam, grandes equos, abhorruit omnia parva, magna pocula, et calceamenta pedibus majora.}
\setmarginnote{2583}{Lib. 1. cap. 16. putavit se uno digito posse totum mundum conterere.}
\setmarginnote{2584}{Sustinet humeris coelum cum Atlante. Alii coeli ruinam timent.}
\setmarginnote{2585}{Cap. 1. Tract. 15. alius se gallum putat, alius lusciniam.}
\setmarginnote{2586}{Trallianus.}
\setmarginnote{2587}{Cap. 7. de mel.}
\setmarginnote{2588}{Anthony de Verdeur.}
\setmarginnote{2589}{Cap. 7. de mel.}
\setmarginnote{2590}{Laurentius cap. 6.}
\setmarginnote{2591}{Lib. 3. cap. 14. qui se regem putavit regno expulsum.}
\setmarginnote{2592}{Dipnosophist. lib. Thrasilaus putavit omnes naves in Pireum portum appellantes suas esse.}
\setmarginnote{2593}{De hist. Med. mirab. lib. 2. cap. 1.}
\setmarginnote{2594}{Genibus flexis loqui cum illo voluit, et adstare jam tum putavit, \&c.}
\setmarginnote{2595}{Gordonius, quod sit propheta, et inflatus a spiritu sancto.}
\setmarginnote{2596}{Qui forensibus causis insudat, nil nisi arresta cogitat, et supplices libellos, alius non nisi versus facit. P. Forestus.}
\setmarginnote{2597}{Gordonius.}
\setmarginnote{2598}{Verbo non exprimunt, nec opere, sed alta mente recondunt, et sunt viri prudentissimi, quos ego saepe novi, cum multi sint sine timore, ut qui se reges et mortuis putant, plura signa quidam habent, pauciora, majora, minora.}
\setmarginnote{2599}{Trallianus, lib. 1. 16. alii intervalla quaedam habent, ut etiam consueta administrent, alii in continuo delirio sunt, \&c.}
\setmarginnote{2600}{Prac. mag. Vera tantum et autumno.}
\setmarginnote{2601}{Lib. de humeribus.}
\setmarginnote{2602}{Guianerius.}
\setmarginnote{2603}{De mentis alienat. cap. 3.}
\setmarginnote{2604}{Levinus Lemnius, Jason Pratensis, blanda ab initio.}
\setmarginnote{2605}{A most agreeable mental delusion.}
\setmarginnote{2606}{Hor.}
\setmarginnote{2607}{Facilis descensus averni.}
\setmarginnote{2608}{Virg.}
\setmarginnote{2609}{Corpus cadaverosum. Psa. lxvii. cariosa est facies mea prae aegritudine animae.}
\setmarginnote{2610}{Lib. 9. ad Ahnansorem.}
\setmarginnote{2611}{Practica majore.}
\setmarginnote{2612}{Quum ore loquitur quae corde concepit, quum subito de una re ad aliud transit, neque rationem de aliquo reddit, tunc est in medio, at quum incipit operari quae loquitur, in summo gradu est.}
\setmarginnote{2613}{Cap. 19. Partic. 2. Loquitur secum et ad alios, ac si vere praesentes. Aug. cap. 11. li. de cura pro mortuis gerenda. Rhasis.}
\setmarginnote{2614}{Quum res ad hoc devenit, ut ea quae cogitare caeperit, ore promat, atque acta permisceat, tum perfecta melancholia est.}
\setmarginnote{2615}{Melancholicus se videre et audire putat daemones. Lavater de spectris, part. 3. cap. 2.}
\setmarginnote{2616}{Wierus, lib. 3. cap. 31.}
\setmarginnote{2617}{Michael a musian.}
\setmarginnote{2618}{Malleo malef.}
\setmarginnote{2619}{Lib. de atra bile.}
\setmarginnote{2620}{Part. 1. Subs. 2, Memb. 2.}
\setmarginnote{2621}{De delirio, melancholia et mania.}
\setmarginnote{2622}{Nicholas Piso. Si signa circa ventriculum non apparent nec sanguis male affectus, et adsunt timor et maestitia, cerebrum ipsum existimandum est, \&c.}
\setmarginnote{2623}{Tract. de mel. cap. 13, \&c. Ex intemperie spirituum, et cerebri motu, tenebrositate.}
\setmarginnote{2624}{Facie sunt rubente et livescente, quibus etiam aliquando adsunt pustulae.}
\setmarginnote{2625}{Jo. Pantheon. cap. de Mel. Si cerebrum primario afficiatur adsunt capitis gravitas, fixi oculi, \&c.}
\setmarginnote{2626}{Laurent. cap. 5. si a cerebro ex siccitate, tum capitis erit levitas, sitis, vigilia, paucitas superfluitatum in oculis et naribus.}
\setmarginnote{2627}{Si nulla digna laesio, ventriculo, quoniam in hac melancholia capitis, exigua nonnunquam ventriculi pathemata coeunt, duo enim haec membra sibi invicem affectionem transmittunt.}
\setmarginnote{2628}{Postrema magis flatuosa.}
\setmarginnote{2629}{Si minus molestiae circa ventriculum aut ventrem, in iis cerebrum primario afficitur, et curare oportet hunc affectum, per cibos flatus exortes, et bonae concoctionis, \&c. raro cerebrum afficitur sine ventriculo.}
\setmarginnote{2630}{Sanguinem adurit caput calidius, et inde fumi melancholici adusti, animum exagitant.}
\setmarginnote{2631}{Lib. de loc. affect. cap. 6.}
\setmarginnote{2632}{Cap. 6.}
\setmarginnote{2633}{Hildesheim spicel. 1. de mel. In Hypochondriaca melancholia adeo ambigua sunt symptomata, ut etiam exercitatissimi medici de loco affecto statuere non possint.}
\setmarginnote{2634}{Medici de loco affecto nequeunt statuere.}
\setmarginnote{2635}{Tract. posthumo de mel. Patavii edit. 1620. per Bozettum Bibliop. cap. 2.}
\setmarginnote{2636}{Acidi ructus, cruditates, aestus in praecordiis, flatus, interdum ventriculi dolores vehementes, sumptoque cibo concoctu difficili, sputum humidum idque multum sequetur, \&c. Hip. lib. de mel. Galenus, Melanelius e Ruffo et Aetio, Altomarus, Piso, Montaltus, Bruel, Wecker, \&c.}
\setmarginnote{2637}{Circa praecordia de assidua in flatione queruntur, et cum sudore totius corporis importuno, frigidos articulos saepe patiuntur, indigestione laborant, ructus suos insuaves perhorrescunt, viscerum dolores habent.}
\setmarginnote{2638}{Montaltus, c. 13. Wecker, Fuchsius c. 13. Altomarus c. 7. Laurentius c. 73. Bruel, Gordon.}
\setmarginnote{2639}{Pract. major: dolor in eo et ventositas, nausea.}
\setmarginnote{2640}{Ut atra densaque nubes soli effusa, radios et lumen ejus intercipit et offuscat; sic, etc.}
\setmarginnote{2641}{Ut fumus e camino.}
\setmarginnote{2642}{Hypochondriaci maxime affectant coire, et multiplicatur coitus in ipsis, eo quod ventositates multiplicantur in hypochondriis, et coitus saepe allevat has ventositates.}
\setmarginnote{2643}{Cont. lib. 1. tract. 9.}
\setmarginnote{2644}{Wecker, Melancholicus succus toto corpore redundans.}
\setmarginnote{2645}{Splen natura imbecilior. Montaltus cap. 22.}
\setmarginnote{2646}{Lib. 1. cap. 16. Interrogare convenit, an aliqua evacuationis retentio obvenerit, viri in haemmorrhoid, mulierum menstruis, et vide faciem similiter an sit rubicunda.}
\setmarginnote{2647}{Naturales nigri acquisiti a toto corpore, saepe rubicundi.}
\setmarginnote{2648}{Montaltus cap. 22. Piso. Ex colore sanguinis si minuas venam, si fluat niger, \&c.}
\setmarginnote{2649}{Apul. lib. 1. semper obviae species mortuorum quicquid umbrarum est uspiam, quicquid lemurum et larvarum oculis suis aggerunt, sibi fingunt omnia noctium occursacula, omnia busforum formidamina, omnia sepulchrorum terriculamenta.}
\setmarginnote{2650}{Differt enim ab ea quae viris et reliquis feminis communiter contingit, propriam habens causam.}
\setmarginnote{2651}{Ex menstrui sanguinis tetra ad cor et cerebrum exhalatione, vitiatum semen mentem perturbat, \&c. non per essentiam, sed per consensum. Animus moerens et anxius inde malum trahit, et spiritus cerebrum obfuscantur, quae cuncta augentur, \&c.}
\setmarginnote{2652}{Cum tacito delirio ac dolore alicujus partis internae, dorsi, hypochondrii, cordis regionem et universam mammam interdum occupantis, \&c. Cutis aliquando squalida, aspera, rugosa, praecipue cubitis, genibus, et digitorum articulis, praecordia ingenti saepe torrore aestuant et pulsant, cumque vapor excitatus sursum evolat, cor palpitat aut premitur, animus deficit, \&c.}
\setmarginnote{2653}{Animi dejectio, perversa rerum existimatio, praeposterum judicium. Fastidiosae, languentes, taediosae, consilii inopes, lachrymosae, timentes, moestae, cum summa rerum meliorum desperatione, nulla re delectantur, solitudinem amant, \&c.}
\setmarginnote{2654}{Nolunt aperire molestiam quam patiuntur, sed conqueruntur tamen de capite, corde, mammis, \&c. In puteos fere maniaci prosilire, ac strangulari cupiunt, nulla orationis suavitate ad spem salutis recuperandam erigi, \&c. Familiares non curant, non loquuntur, non respondent, \&c. et haec graviora, si, \&c.}
\setmarginnote{2655}{Clisteres et Helleborismum Mathioli summe laudat.}
\setmarginnote{2656}{Examen conc. Trident. de coelibatu sacerd.}
\setmarginnote{2657}{Cap. de Satyr. et Priapis.}
\setmarginnote{2658}{Part. 3. sect. 2. Memb. 5. Sub. 5.}
\setmarginnote{2659}{Lest you may imagine that I patronise that widow or this virgin, I shall not add another word.}
\setmarginnote{2660}{Vapores crassi et nigri, a ventriculo in cerebrum exhalant. Fel. Platerus.}
\setmarginnote{2661}{Calidi hilares, frigidi indispositi ad laetitiam, et ideo solitarii, taciturni, non ob tenebras internas, ut medici volunt, sed ob frigus: multi melancholici nocte ambulant intrepidi.}
\setmarginnote{2662}{Vapores melancholici, spiritibus misti, tenebrarum causse sunt, cap. 1.}
\setmarginnote{2663}{Intemperies facit succum nigrum, nigrities, obscurat spiritum, obscuratio spiritus facit metum et tristiam.}
\setmarginnote{2664}{Ut nubecula Solern offuscat. Constantinus lib. de melanch.}
\setmarginnote{2665}{Altomarus c. 7. Causam timoris circumfert aler humor passionis materia, et atri spiritus perpetuam animae domicilio offundunt noctem.}
\setmarginnote{2666}{Pone exemplum, quod quis potest ambulare super trahem quae est in via: sed si sit super aquam profundam, loco pontis, non ambulabit super eam, eo quod imaginetur in animo et timet vehementer, forma cadendi impressa, cui obediunt membra omnia, et facultates reliquae.}
\setmarginnote{2667}{Lib. 2. de intellectione. Susoiciosi ob timorem et obliquum discursum, et semper inde putant sibi fieri insidias. Lauren. 5.}
\setmarginnote{2668}{Tract. de mel. cap. 7. Ex dilatione, contractione, confusione, tenebrositate spirituum, calida, frigida intemperie, \&c.}
\setmarginnote{2669}{Illud inquisitione dignum, cur tam falsa recipiant, habere se cornua, esse mortuos, nasutos, esse aves, \&c.}
\setmarginnote{2670}{1. Dispositio corporis. 2. Occasio Imaginationis.}
\setmarginnote{2671}{In pro. li. de coelo. Vehemens et assidua cogitatio rei erga quam afficitur, spiritus in cerebrum evocat.}
\setmarginnote{2672}{Melancholici ingeniosi omnes, summi viri in artibus et disciplinis, sive circum imperatoriam aut reip. disciplinam omnes fere melancholici, Aristoteles.}
\setmarginnote{2673}{Adeo miscentur, ut sit duplum sanguinis ad reliqua duo.}
\setmarginnote{2674}{Lib. 2. de intellectione. Pingui sunt Minerva phlegmatici: sanguinei amabiles, grati, hilares, at non ingeniosi; cholerici celerna motu, et ob id contemplationis impatientes: Melancholici solum excellentes, \&c.}
\setmarginnote{2675}{Trepidantium vox tremula, quia cor quatitur.}
\setmarginnote{2676}{Ob ariditatem quae reddit nervos linguae torpidos.}
\setmarginnote{2677}{Incontinentia linguae ex copia flatuum, et velocitate imaginationis.}
\setmarginnote{2678}{Calvities ob ficcitatis excessum.}
\setmarginnote{2679}{Aetius.}
\setmarginnote{2680}{Lauren. c. 13.}
\setmarginnote{2681}{Tetrab. 2. ser. 2. cap. 10.}
\setmarginnote{2682}{Ant. Lodovicus prob. lib. 1. sect. 5. de atrabilariis.}
\setmarginnote{2683}{Subrusticus pudor vitiosus pudor.}
\setmarginnote{2684}{Ob ignominiam aut turpedinem facti, \&c.}
\setmarginnote{2685}{De symp. et Antip. cap. 12. laborat facies ob praesentiam ejus qui defectum nostrum videt, et natura quasi opem latura calorem illuc mittit, calor sanguinem trahit, undo rubor, audaces non rubent, \&c.}
\setmarginnote{2686}{Ob gaudium et voluptatem foras exit sanguis, aut ob melioris reverentiam, aut ob subitum occursum, aut si quid incautius exciderit.}
\setmarginnote{2687}{Com. in Arist. de anima. Coeci ut plurimum impudentes, nox facit impudentes.}
\setmarginnote{2688}{Alexander Aphrodisiensis makes all bashfulness a virtue, eamque se refert in seipso experiri solitum, etsi esset admodum sanex.}
\setmarginnote{2689}{Saepe post cibum apti ad ruborem, ex potu vini ex timore saepe, et ab hepate calido, cerebro calido, \&c.}
\setmarginnote{2690}{Com. in Arist. de anima, tam a vi et inexperientia quam a vitio.}
\setmarginnote{2691}{De oratore, quid ipse risus, quo pacto concitatur, ubi sit, \&c.}
\setmarginnote{2692}{Diaphragma titillant, quia transversum et nervosum, quia titillatione moto sensu atque arteriis distentis, spiritus inde latera, venas, os, oculos occupant.}
\setmarginnote{2693}{Ex calefactione humidi cerebri: nam ex sicco lachrymae non fluunt.}
\setmarginnote{2694}{Res mirandas imaginantur: et putant se videre quae nec vident, nec audiunt.}
\setmarginnote{2695}{Laet. lib. 13. cap. 2. descript. Indiae Occident.}
\setmarginnote{2696}{Lib. 1. ca. 17. cap. de mel.}
\setmarginnote{2697}{Insani, et qui morti vicini sunt, res quas extra se videre putant, intra oculos habent.}
\setmarginnote{2698}{Cap. 10. de Spirit apparitione.}
\setmarginnote{2699}{De occult. Nat. mirac.}
\setmarginnote{2700}{O mother! I beseech you not to persecute me with those horrible-looking furies. See! see! they attack, they assault me!}
\setmarginnote{2701}{Peace! peace! unhappy being, for you do not see what you think you see.}
\setmarginnote{2702}{Seneca. Quod metuunt nimis, nunquam amoveri posse, nec tolli putant.}
\setmarginnote{2703}{Sanguis upupoe cum melle compositus et centaurea, \&c. Albertus.}
\setmarginnote{2704}{Lib. 1. occult. philos. Imperiti homines daemonum et umbrarum imagines videre se putant, quum nihil sint aliud, quam simulachra animae expertia.}
\setmarginnote{2705}{Pythonissae vocum varietatem in ventre et gutture fingentes formant voces humanas a longe vel prope, prout volunt, ac si spiritus cum homine loqueretur, et sonos brutorum fingunt, \&c.}
\setmarginnote{2706}{Gloucester cathedral.}
\setmarginnote{2707}{Tam clare et articulate audies repetitum, ut perfectior sit Echo quam ipse dixeris.}
\setmarginnote{2708}{Blowing of bellows, and knocking of hammers, if they apply their ear to the cliff.}
\setmarginnote{2709}{Memb. 1. Sub. 3. of this partition, cap. 16, in 9. Rhasis.}
\setmarginnote{2710}{Signa daemonis nulla sunt nisi quod loquantur ea quae ante nesciebant, ut Teutonicum aut aliud Idioma, \&c.}
\setmarginnote{2711}{Cap 12. tract. de mel.}
\setmarginnote{2712}{Tract. 15. c. 4.}
\setmarginnote{2713}{Cap 9.}
\setmarginnote{2714}{Mira vis concitat humores, ardorque vehemens mentem exagitat, quum, \&c.}
\setmarginnote{2715}{Praefat. Iamblici mysteriis.}
