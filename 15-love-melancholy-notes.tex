\setauthornote{4630}{Memb. 1. Subs. 2.}
\setauthornote{4631}{Amor et amicitia.}
\setauthornote{4632}{Phaedrus orat. in laudem amoris Platonis convivio.}
\setauthornote{4633}{Vide Boccas. de Genial deorum.}
\setauthornote{4634}{See the moral in Plut. of that fiction.}
\setauthornote{4635}{Affluentiae Deus.}
\setauthornote{4636}{Cap. 7. Comment. in Plat. convivium.}
\setauthornote{4637}{See more in Valesius, lib. 3. cont. med. et cont. 13.}
\setauthornote{4638}{Vives 3. de anima; oramus te ut tuis artibus et caminis nos refingas, et ex duobus unum facias; quod et fecit, et exinde amatores unum sunt et unum esse petunt.}
\setauthornote{4639}{See more in Natalis Comes Imag. Deorum Philostratus de Imaginibus. Litius Giraldus Syntag. de diis. Phornutus, \&c.}
\setauthornote{4640}{Juvenis pingitur quod amore plerumque juvenes capiuntur; sic et mollis, formosus, nudus, quod simplex et apertus hic affectus; ridet quod oblectamentum prae se ferat, cum pharetra, \&c.}
\setauthornote{4641}{A petty Pope claves habet superorum et inferorum, as Orpheus, \&c.}
\setauthornote{4642}{Lib. 13. cap. 5. Dypnoso.}
\setauthornote{4643}{Regnat et in superos jus habet ille deos. \Ovid{}.}
\setauthornote{4644}{\Plautus{}.}
\setauthornote{4645}{Selden pro leg. 3. cap. de diis Syris.}
\setauthornote{4646}{Dial. 3.}
\setauthornote{4647}{A concilia Deorum rejectus et ad majorem ejus ignominiam, \&c.}
\setauthornote{4648}{Fulmine concitatior.}
\setauthornote{4649}{Sophocles.}
\setauthornote{4650}{He divides the empire of the sea with Thetis,-of the Shades, with Aeacus,-of the Heaven, with Jove.}
\setauthornote{4651}{Tom. 4.}
\setauthornote{4652}{Dial. deorum, tom. 3.}
\setauthornote{4653}{Quippe matrem ipsius quibus modis me afficit, nunc in Idam adigens Anchisae causa, \&c.}
\setauthornote{4654}{Jampridem et plagas ipsi in nates incussi sandalio.}
\setauthornote{4655}{Altopilus, fol. 79.}
\setauthornote{4656}{Nullis amor est medicabilis herbis.}
\setauthornote{4657}{Plutarch in Amatorio. Dictator quo creato cessant reliqui magistratus.}
\setauthornote{4658}{Claadian. descript. vener. aulae. Trees are influenced by love, and every flourishing tree in turn feels the passion: palms nod mutual vows, poplar sighs to poplar, plane to plane, and alder breathes to alder.}
\setauthornote{4659}{Neque prius in iis desiderium cessat dum dejectus consoletur; videre enim est ipsam arborem incurvatam, ultro ramis ab utrisque vicissim ad osculum exporrectis. Manifesta dant mutui desiderii signa.}
\setauthornote{4660}{Multas palmas contingens quae simul crescant, rursusque ad amantem regrediens, eamque manu attingens, quasi osculum mutuo ministrare videtur, et expediti concubitus gratiam facit.}
\setauthornote{4661}{Quam vero ipsa desideret affectu ramorum significat, et adullam respicit; amantur, \&c.}
\setauthornote{4662}{Virg. 3. Georg.}
\setauthornote{4663}{Propertius.}
\setauthornote{4664}{Dial. deorum. Confide mater, leonibus ipsis familiaris jam factus sum, et saepe conscendi eorum terga et apprehendi jubas; equorum more insidens eos agito, et illi mihi caudis adblandiuntur.}
\setauthornote{4665}{Leones prae amore furunt, Plin. l. 8. c. 16. Arist. l. 6. hist. animal.}
\setauthornote{4666}{Cap. 17. of his book of hunting.}
\setauthornote{4667}{\Lucretius{}.}
\setauthornote{4668}{De sale lib. 1. c. 21. Pisces ob amorem marcescunt, pallescunt, \&c.}
\setauthornote{4669}{Hauriendae aquae causa venientes ex insidiis a Tritone comprehensae, \&c.}
\setauthornote{4670}{Plin. l. 10. c. 5 quumque aborta tempestate periisset Hernias in sicco piscis expiravit.}
\setauthornote{4671}{Postquam puer morbo abiit, et ipse delphinus periit.}
\setauthornote{4672}{Pleni sunt libri quibus ferae in homines inflammatae fuerunt, in quibus ego quidem semper assensum sustinui, veritus ne fabulosa crederem; Donec vidi lyncem quem habui ab Assyria, sic affectum erga unum de meis hominibus, \&c.}
\setauthornote{4673}{Desiderium suum testatus post inediam aliquot dierum interiit.}
\setauthornote{4674}{Orpheus hymno Ven. Venus keeps the keys of the air, earth, sea, and she alone retains the command of all.}
\setauthornote{4675}{Qui haec in artrae bilis aut Imaginationis vim referre conati sunt, nihil faciunt.}
\setauthornote{4676}{Cantantem audies et vinum bibes, quale antea nunquam bibisti; te rivalis turbabit nullus; pulchra autem pulchro autem pulchro contente vivam, et moriar.}
\setauthornote{4677}{Multi factum hoc cognovere, quod in media Graecia gestum sit.}
\setauthornote{4678}{Rem curans domesticam, ut ante, peperit aliquot liberos, semper tamen tristis et pallida.}
\setauthornote{4679}{Haec audivi a multis fide dignis qui asseverabant ducem Bavariae eadem retulisse Duci Saxoniae pro veris.}
\setauthornote{4680}{Fabula Damarati et Aristonis in Herodoto lib. 6. Erato.}
\setauthornote{4681}{Interpret. Mersio.}
\setauthornote{4682}{Deus Angelos misit ad tutelam cultumque generis humani; sed illos cum hominibus commorantes, dominator ille terrae salacissimus paulatim ad vitia pellexit, et mulierum congressibus inquinavit.}
\setauthornote{4683}{Quidam ex illo capti sunt amore virginum, et libidine victi defecerunt, ex quibus gigantes qui vocantur, nati sunt.}
\setauthornote{4684}{Pererius in Gen. lib. 8. c. 6. ver. 1. Zanc. \&c.}
\setauthornote{4685}{Purchas Hack posth. par. 1. lib. 4. Cap. 1. S. 7.}
\setauthornote{4686}{In Clio.}
\setauthornote{4687}{Deus ipse hoc cubili requiescens.}
\setauthornote{4688}{Physiologiae Stoicorum l. 1. cap. 20. Si spiritus unde semen iis, \&c. at exempla turbant nos; mulierum quotidianae confessiones de mistione omnes asserunt, et sunt in hac urbe Loviano exempla.}
\setauthornote{4689}{Unum dixero, non opinari me ullo retro aevo tantam copiam Satyrorum, et salacium istorum Geniorum se ostendisse, quantum nunc quotidianae narrationes, et judiciales sententiae proferunt.}
\setauthornote{4690}{Virg.}
\setauthornote{4691}{For it is a shame to speak of those things which are done of them in secret, Eph. v. 12.}
\setauthornote{4692}{Plutarch, amator lib.}
\setauthornote{4693}{Lib. 13.}
\setauthornote{4694}{Rom. i. 27.}
\setauthornote{4695}{Lilius Giraldus, vita ejus.}
\setauthornote{4696}{Pueros amare solis Philosophis relinquendum vult Lucianus dial. Amorum.}
\setauthornote{4697}{Busbequius.}
\setauthornote{4698}{Achilles Tatius lib. 2.}
\setauthornote{4699}{Lucianus Charidemo.}
\setauthornote{4700}{Non est haec mentula demens. Mart.}
\setauthornote{4701}{Jovius Musc.}
\setauthornote{4702}{Praefat. lectori lib. de vitis pontif.}
\setauthornote{4703}{Mercurialis cap. de Priapismo. Coelius l. 11. antic. lect. cap. 14. Galenis 6. de locis aff.}
\setauthornote{4704}{De morb. mulier. lib. I. c. 15.}
\setauthornote{4705}{Herodotus l. 2. Euterpae: uxores insignium virorum non statim vita functas tradunt condendas, ac ne eas quidem foeminas quae formosae sunt, sed quatriduo ante defunctas, ne cum iis salinarii concumbant, \&c.}
\setauthornote{4706}{Metam. 13.}
\setauthornote{4707}{Seneca de ira, l. 11. c. 18.}
\setauthornote{4708}{Nullus est meatus ad quem non pateat aditus impudicitiae. Clem Alex. paedag, lib. 3. c 3.}
\setauthornote{4709}{Seneca 1. nat. quaest.}
\setauthornote{4710}{Tom. P. Gryllo.}
\setauthornote{4711}{De morbis mulierum l. 1. c. 15.}
\setauthornote{4712}{Amphitheat. amore. cap. 4. interpret. Curtio.}
\setauthornote{4713}{Aeneas Sylvius Juvenal. And he who has not felt the influence of love is either a stone or a beast.}
\setauthornote{4714}{Tertul. prover. lib.}
\setauthornote{4715}{One whom no maiden's beauty has ever affected.}
\setauthornote{4716}{Chaucer.}
\setauthornote{4717}{Tom. 1. dial. deorum Lucianus. Amore non ardent Musae.}
\setauthornote{4718}{As matter seeks form, so woman turns towards man.}
\setauthornote{4719}{In amator. dialog.}
\setauthornote{4720}{Hor.}
\setauthornote{4721}{\Lucretius{}.}
\setauthornote{4722}{Fonseca.}
\setauthornote{4723}{Hor.}
\setauthornote{4724}{Propert.}
\setauthornote{4725}{Simonides, graec.}
\setauthornote{4725.5}{She grows old in love and in years together.}
\setauthornote{4726}{Ausonius.}
\setauthornote{4727}{Geryon amicitae symbolum.}
\setauthornote{4728}{Propert. l. 2.}
\setauthornote{4729}{Plutarch. c. 30. Rom. Hist.}
\setauthornote{4730}{Junonem habeam iratam, si unquam meminerim me virginem fuisse. Infans enim paribus inquinata sum, et subinde majoribus me applicui, donec ad aetatem perveni; ut Milo vitulum, \&c.}
\setauthornote{4731}{Parnodidasc. dial. lat. interp. Casp. Barthio ex Ital.}
\setauthornote{4732}{Angelico scriptur concentu.}
\setauthornote{4733}{Epictetus c. 42. mulieres statim ab anno 14. movere incipiunt, \&c. attrectari se sinunt et exponunt. Levinu Lemnius.}
\setauthornote{4734}{Lib. 3. fol. 126.}
\setauthornote{4735}{Catullus.}
\setauthornote{4736}{Whithersoever enraged you fly there is no escape. Although you reach the Tanais, love will still pursue you.}
\setauthornote{4737}{De mulierum inexhausta libidine luxuque insatiabili omnes aeque regiones conqueri posse existimo. Steph.}
\setauthornote{4738}{What have lust and unrestrained desire left chaste or enviolate upon earth?}
\setauthornote{4739}{\Plautus{}.}
\setauthornote{4740}{Oculi caligant, aures graviter audiunt, capilli fluunt, cutis arescit, flatus olet, tussis, \&c. Cyprian.}
\setauthornote{4741}{Lib. 8. Epist. Ruffinus.}
\setauthornote{4742}{Hiatque turpis inter aridas nates podex.}
\setauthornote{4743}{Cadaverosa adeo ut ab inferis reversa videri possit, vult adhuc catullire.}
\setauthornote{4744}{Nam et matrimoniis est despectum senium. Aeneas Silvius.}
\setauthornote{4745}{Quid toto terrarum orbe communius? quae civitas, quod oppidum, quae familia vacat amatorum exemplis? Aeneas Silvius. Quis trigesimum annum natus nullum amoris causa peregit insigne facinus? ego de me facio conjecturam, quem amor in mille pericula misit.}
\setauthornote{4746}{Forestus. Plato.}
\setauthornote{4747}{Pract. major. Tract. 6. cap. 1. Rub. 11. de aegrit. cap. quod his multum contingat.}
\setauthornote{4748}{Haec aegritudo est solicitudo melancholica in qua homo applicat sibi continuam cogitationem super pulchritudine ipsius quam amat, gestuum morum.}
\setauthornote{4749}{Animi forte accidens quo quis rem habere nimia aviditate concupiscit, ut ludos venatores, aurum et opes avari.}
\setauthornote{4750}{Assidua cogitatio super rem desideratum, cum confidentia obtinendi, ut spe apprehensum delectabile, \&c.}
\setauthornote{4751}{Morbus corporis potius quam animi.}
\setauthornote{4752}{Amor est passio melancholica.}
\setauthornote{4753}{Ob calefactionem spirituum pars anterior capitis laborat ob consumptionem humiditatis.}
\setauthornote{4754}{Affectus animi concupiscibilis e desiderio rei amatae per oculus in mente concepto, spiritus in corde et jecore incendens.}
\setauthornote{4755}{Odyss. et Metamor. 4. \Ovid{}.}
\setauthornote{4756}{Quod talem carnificinam in adolescentum visceribus amor faciat inexplebilis.}
\setauthornote{4757}{Testiculi quoad causam conjunctam, epar antecedentem, possunt esse subjectum.}
\setauthornote{4758}{Proprie passio cerebri est ob corruptam imaginationem.}
\setauthornote{4759}{Cap. de affectibus.}
\setauthornote{4760}{Est corruptio imaginativae et aestimativae facultatis, ob formam fortiter affixam, corruptumque judicium, ut semper de eo cogitet, ideoque recte melancholicus appellatur. Concupiscentia vehemens ex corrupto judicio aestimativae virtutis.}
\setauthornote{4761}{Comment. in convivium Platonis. Irretiuntur cito quibus nascentibus Venus fuerit in Leone, vel Luna venerem vehementer aspexerit, et qui eadem complexione sunt praediti.}
\setauthornote{4762}{Plerumque amatores sunt, et si foeminae meretrices, 1. de audiend.}
\setauthornote{4763}{Comment, in Genes, cap. 3.}
\setauthornote{4764}{Et si in hoc parum a praeclara infamia stultitiaque abero, vincit tamen amor veritatis.}
\setauthornote{4765}{Edit. Basil. 1553. Cum Commentar. in Ptolomaei quadripartitum.}
\setauthornote{4766}{Fol. 445. Basil. Edit.}
\setauthornote{4767}{Dial, amorum.}
\setauthornote{4768}{Citius maris fluctus et nives coelo delabentes numeraris quam amores meos; alii amores aliis succedunt, ac priusquam desinant priores, incipiunt sequentes. Adeo humidis oculis meus inhabitat Asylus omnem formam ad se rapiens, ut nulla satietate expleatur. Quaenam haec ira Veneris, \&c.}
\setauthornote{4769}{Num. \rn{xxxii}.}
\setauthornote{4770}{Qui calidum testiculorum crisin habent, \&c.}
\setauthornote{4771}{Printed at Paris 1624, seven years after my first edition.}
\setauthornote{4772}{\Ovid{} de art.}
\setauthornote{4773}{Gerbelius, descript. Graeciae. Rerum omnium affluentia et loci mira opportunitas, nullo non die hospites in portas advertebant. Templo Veneris mille meretrices se prostituebant.}
\setauthornote{4774}{Tota Cypri insula delitiis incumbit, et ob id tantum luxuriae dedita ut sit olim Veneri sacrata. Ortelius, Lampsacus, olim Priapo sacer ob vinum generosum, et loci delicias. Idem.}
\setauthornote{4775}{Agri Neapolitani delectatio, elegantia, amoenitas, vix intra modum humanum consistere videtur; unde, \&c. Leand, Alber. in Campania.}
\setauthornote{4776}{Lib. de laud. urb. Neap. Disputat. de morbis animi. Reinoldo Interpret.}
\setauthornote{4777}{Lampridius, Quod decem noctibus centum virgines fecisset mulieres.}
\setauthornote{4778}{Vita ejus.}
\setauthornote{4779}{If they contain themselves, many times it is not \textlatin{virtutis amore; non deest voluntas sed facultas.}}
\setauthornote{4780}{In Muscov.}
\setauthornote{4781}{Catullus ad Lesbiam.}
\setauthornote{4782}{Hor.}
\setauthornote{4783}{Polit. 8. num. 28. ut naptha, ad ignem, sic amor ad illos qui torpescunt ocio.}
\setauthornote{4784}{Pausanias Attic, lib. 1. Cephalus egregiae formae juvenis ab aurora raptus quod ejus amore capta esset.}
\setauthornote{4785}{In amatorio.}
\setauthornote{4786}{E. Stobaeo ser. 62.}
\setauthornote{4787}{Amor otiosae cura est sollicitudinus.}
\setauthornote{4788}{Principes plerumque ob licentiam et adfluentiam divitiarum istam passionem solent incurrere.}
\setauthornote{4789}{Ardenter appetit qui otiosam vitam agit, et communiter incurrit haec passio solitarios delitiose viventes, incontinentes, religiosos, \&c.}
\setauthornote{4790}{Plutarch. vit. ejus.}
\setauthornote{4791}{Vina parant animos veneri.}
\setauthornote{4792}{Sed nihil erucae faciunt bulbique salaces; Improba nec prosit jam satureia tibi. \Ovid{}.}
\setauthornote{4793}{Petronius.}
\setauthornote{4794}{Uti ille apud Skenkium, qui post potionem, uxorem et quatuor ancillas proximo cubiculo cubantes, compressit.}
\setauthornote{4795}{Pers. Sat. 3.}
\setauthornote{4796}{Siracides. Nox, et amor vinumque nihil moderabile suadent.}
\setauthornote{4797}{Lip. ad Olympiam.}
\setauthornote{4798}{Hymno.}
\setauthornote{4799}{Hor. l. 3. Od. 25.}
\setauthornote{4800}{De sale lib. cap. 21.}
\setauthornote{4801}{Kornmannus lib. de virginitate.}
\setauthornote{4802}{Garcias ab horto aromatum, lib. 1. cap. 28.}
\setauthornote{4803}{Surax radix ad coitum summe facit si quis comedat, aut infusionem bibat, membrum subito erigitur. Leo Afer. lib. 9. cap. ult.}
\setauthornote{4804}{Quae non solum edentibus sed et genitale tangentibus tantum valet, ut coire summe desiderent; quoties fere velint, possint; alios duodecies profecisse, alios ad 60 vices pervenisse refert.}
\setauthornote{4805}{Lucian. Tom. 4. Dial. amorum.}
\setauthornote{4806}{Sight, conference, association, kisses, touch.}
\setauthornote{4807}{Ea enim hominum intemperantium libido est ut etiam fama ad amandum impellantur, et audientes aeque afficiuntur ac videntes.}
\setauthornote{4808}{Formosam Sostrato filiam audiens, uxorem cupit, et sola illius, auditione ardet.}
\setauthornote{4809}{Quoties de Panthea Xenophontis locum perlego, ita animo affectus ac si coram intuerer.}
\setauthornote{4810}{Pulchritudinem sibi ipsis configunt, Imagines.}
\setauthornote{4811}{De aulico lib. 2. fol. 116.'tis a pleasant story, and related at large by him.}
\setauthornote{4812}{Gratia venit ab auditu aeque ac visu et species amoris in phantasiam recipiunt sola relatione. Picolomineus grad. 8. c. 38.}
\setauthornote{4813}{Lips. cent. 2. epist. 22. Beautie's Encomions.}
\setauthornote{4814}{Propert.}
\setauthornote{4815}{Amoris primum gradum visus habet, ut aspiciat rem amatam.}
\setauthornote{4816}{Achilles Tatius lib. 1. Forma telo quovis acutior ad inferendum vulnus, perque oculos amatorio vulneri aditum patefaciens in animum penetrat.}
\setauthornote{4817}{In tota rerum natura nihil forma divinius, nihil augustius, nihil pretiosius, cujus vires hinc facile intelliguntur, \&c.}
\setauthornote{4818}{Christ. Fonseca.}
\setauthornote{4819}{S. L.}
\setauthornote{4820}{Bruys prob. 11. de forma e Lucianos.}
\setauthornote{4821}{Lib. de calumnia. Formosi Calumninia vacant; dolemus alios meliore loco positos, fortunam nobis novercam illis, \&c.}
\setauthornote{4822}{Invidemus sapientibus, justis, nisi beneficiis assidue amorem extorquent; solos formosos amamus et primo velut aspectu benevolentia conjungimur, et eos tanquam Deos colimus, libentius iis servimus quam aliis imperamus, majoremque, \&c.}
\setauthornote{4823}{Formae majestatem Barbari verentur, nec alii majores quam quos eximia forma natura donata est, Herod, lib. 5. Curtius G. Arist. Polit.}
\setauthornote{4824}{Serm. 63. Plutarch, vit. ejus. Brisonius Strabo.}
\setauthornote{4825}{Virtue appears more gracefully in a lovely personage.}
\setauthornote{4826}{Lib. 5. magnorumque; operum non alios capaces putant quam quos eximia specie natura donavit.}
\setauthornote{4827}{Lib. de vitis Pontificum. Rom.}
\setauthornote{4828}{Lib. 2. cap. 6.}
\setauthornote{4829}{Dial. amorum. c. 2. de magia. Lib. 2. connub. cap. 27. Virgo formosa et si oppido pauper, abunde est dotata.}
\setauthornote{4830}{Isocrates plures ob formam immortalitatem adepti sunt quam ob reliquas omnes virtutes.}
\setauthornote{4831}{Lucian Tom. 4. Charidaemon. Qui pulchri, merito apud Deos et apud homines honore affecti. Muta commentatio, quavis epistola ad commendandum efficacior.}
\setauthornote{4832}{Lib. 9. Var. hist, tanta formae elegantia ut ab ea nuda, \&c.}
\setauthornote{4833}{Esdras, iv. 29.}
\setauthornote{4834}{Origen hom. 23. in Numb. In ipsos tyrannos tyrannidem exercet.}
\setauthornote{4835}{Illud certe magnum ob quod gloriari possunt formosi, quod robustis necessarium sit laborare, fortem periculis se objicere, sapientem, \&c.}
\setauthornote{4836}{Majorem vim habet ad commendandam forma, quam accurate scripta epistola. Arist.}
\setauthornote{4837}{Heliodor. lib. I.}
\setauthornote{4838}{Knowles. hist. Turcica.}
\setauthornote{4839}{Daniel in complaint of Rosamond.}
\setauthornote{4840}{Stroza filius Epig.}
\setauthornote{4840.5}{The king of the gods on account of this beauty became a bull, a shower, a swan.}
\setauthornote{4841}{Sect. 2. Mem. 1. Sub. 1.}
\setauthornote{4842}{Stromatum l. post captam Trojam cum impetu ferretur, ad occidendam Helenam, stupore adeo pulchritudinis correptus ut ferrum excideret, \&c.}
\setauthornote{4843}{Tantae formae fuit ut cum vincta loris, feris exposita foret, equorum calcibus obterenda, ipsis jumentis admiratione fuit; laedere noluerunt.}
\setauthornote{4844}{Lib. 8. mules.}
\setauthornote{4845}{If you will restore me to my parents, and my beautiful lover, what thanks, what honour shall I owe you, what provender shall I not supply you?}
\setauthornote{4846}{Aethiop. l. 3.}
\setauthornote{4847}{Atheneus, lib. 8.}
\setauthornote{4848}{\Apuleius Aur. asino.}
\setauthornote{4849}{Shakespeare.}
\setauthornote{4850}{Marlowe.}
\setauthornote{4851}{Ov. Met. 1.}
\setauthornote{4852}{\Ovid{}. Met. lib. 5.}
\setauthornote{4853}{And with her hand wiping off the drops from her green tresses, thus began to relate the loves of Alpheus. I was formerly an Achaian nymph.}
\setauthornote{4854}{Leland.}
\setauthornote{4854.5}{Their lips resound with thousand kisses, their arms are pallid with the close embrace, and their necks are mutually entwined by their fond caresses.}
\setauthornote{4855}{Angerianus.}
\setauthornote{4856}{Si longe aspiciens haec urit lumine divos atque homines prope, cur urere lina nequit? Angerianus.}
\setauthornote{4857}{We wonder how great the vapour, and whence it comes.}
\setauthornote{4858}{Idem Anger.}
\setauthornote{4859}{Obstupuit mirabundas membrorum elegantiam, \&c. Ep. 7.}
\setauthornote{4860}{Stobaeus e graeco.}
\setauthornote{4860.5}{My limbs became relaxed, I was overcome from head to foot, all self-possession fled, so great a stupor overburdened my mind.}
\setauthornote{4861}{Parum abfuit quo minus saxum ex nomine factus sum, ipsis statuis immobiliorem me fecit.}
\setauthornote{4862}{Veteres Gorgonis fabulam confinxerunt, eximium formae decus stupidos reddens.}
\setauthornote{4863}{Hor. Ode 5.}
\setauthornote{4864}{Marlos Hero.}
\setauthornote{4865}{Aspectum virginis sponte fugit insanus fere, et impossibile existimans ut simul eam aspicere quis possit, et intra temperantiae metas se continere.}
\setauthornote{4866}{\Apuleius, l. 4. Multi mortales longis itineribus, \&c.}
\setauthornote{4867}{Nic. Gerbel. l. 5. Achaia.}
\setauthornote{4868}{I. Secundus basiorum lib.}
\setauthornote{4869}{Musaeus Illa autem bene morata, per aedem quocunque vagabatur, sequentem mentem habebat, e oculos, et corda virorum.}
\setauthornote{4870}{Homer.}
\setauthornote{4871}{Marlowe.}
\setauthornote{4872}{Perno didascalo dial. Ital. Latin. donat. a Gasp. Barthio Germano.}
\setauthornote{4873}{Propertius.}
\setauthornote{4874}{Vestium splendore et elegantia ambitione incessus, donis, cantilenis, \&c. gratiam adipisci.}
\setauthornote{4875}{Prae caeteris corporis proceritate et egregia indole mirandus apparebat, caeteri autem capti ejus amore videbantur, \&c.}
\setauthornote{4876}{Aristenaetus, ep. 10.}
\setauthornote{4877}{Tom. 4. dial. meretr. respicientes et ad formam ejus obstupescentes.}
\setauthornote{4878}{In Charidemo sapientiae merito pulchritudo praefertur et opibus.}
\setauthornote{4879}{Indignum nihil est Troas fortes et Achivos tempore tam longo perpessos esse labore.}
\setauthornote{4880}{Digna quidem facies pro qua vel obiret Achilles, vel Priamus, belli causa probanda fuit. Proper. lib. 2.}
\setauthornote{4881}{Coecus qui Helenae formam carpserat.}
\setauthornote{4882}{Those mutinous Turks that murmured at Mahomet, when they saw Irene, excused his absence. Knowls.}
\setauthornote{4883}{In laudem Helenae erat.}
\setauthornote{4884}{Apul. miles. lib. 4.}
\setauthornote{4885}{Secun. bas. 13.}
\setauthornote{4886}{Curtius, l. 1.}
\setauthornote{4887}{Confessi.}
\setauthornote{4888}{Seneca. Amor in oculis oritur.}
\setauthornote{4889}{\Ovid{} Fast.}
\setauthornote{4890}{Plutarch.}
\setauthornote{4891}{Lib. de pulchrit. Jesu et Mariae.}
\setauthornote{4892}{Lucian Charidemon supra omnes mortales felicissimum si hac frui possit.}
\setauthornote{4893}{Lucian amor. Insanum quiddam ac furibundum exclamans. O fortunatissime deorum Mars qui propter hanc vinctus fuisti.}
\setauthornote{4894}{Ov. Met. l. 3.}
\setauthornote{4895}{Omnes dii complexi sunt, et in uxorem sibi petierunt, Nat. Comes de Venere.}
\setauthornote{4896}{Ut cum lux noctis affulget, omnium oculos incurrit: sic Antiloquus \&c.}
\setauthornote{4897}{Dolovit omnes ex animo mulieres.}
\setauthornote{4898}{Nam vincit et vel ignem, ferrumque si qua pulchra est. Anacreon, 2.}
\setauthornote{4899}{Spenser in his Faerie Queene.}
\setauthornote{4900}{Achilles Tatius, lib. 1.}
\setauthornote{4901}{Statim ac eam contemplatus sum, occidi; oculos a virgine avertere conatus sum, sed illi repugnabant.}
\setauthornote{4902}{Pudet dicere, non celabo tamen. Memphim veniens me vicit, et continentiam expugnavit, quam ad senectutem usque servarum, oculis corporis, \&c.}
\setauthornote{4903}{Nunc primum circa hanc anxius animi haereo. Aristaenetus, ep. 17.}
\setauthornote{4904}{Virg. Aen. 4.}
\setauthornote{4904.5}{She alone hath captivated my feelings, and fixed my wavering mind.}
\setauthornote{4905}{Amaranto dial.}
\setauthornote{4906}{Comasque ad speculum disposuit.}
\setauthornote{4907}{Imag. Polystrato. Si illam saltem intuearis, statuis immobiliorem te faciet: si conspexeris eam, non relinquetur facultas oculos ab ea amovendi; abducet te alligatum quocunque voluerit, ut ferrum ad se trahere ferunt adamantem.}
\setauthornote{4908}{Plaut. Merc.}
\setauthornote{4909}{In the Knight's Tale.}
\setauthornote{4910}{Ex debita totius proportione aptaque partium compositione. Picolomineus.}
\setauthornote{4911}{Hor. Od. 19. lib. 1.}
\setauthornote{4912}{Ter. Eunuch. Act. 2. Scen. 3.}
\setauthornote{4913}{Petronius Catall.}
\setauthornote{4914}{Sophocles. Antigone.}
\setauthornote{4915}{Jo. Secundus bas. 19.}
\setauthornote{4916}{Loecheus.}
\setauthornote{4917}{Arandus. Vallis amoenissima e duobus montibus composita niveis.}
\setauthornote{4918}{\Ovid{}.}
\setauthornote{4919}{Fol. 77. Dapsiles hilares amatores, \&c.}
\setauthornote{4920}{When Cupid slept. Caesariem auream habentem, ubi Psyche vidit, mollemque ex ambrosia cervicem inspexit, crines crispos, purpureas genas candidasque, \&c. \Apuleius.}
\setauthornote{4921}{In laudem calvi; splendida coma quisque adulter est; allicit aurea coma.}
\setauthornote{4922}{Venus ipsa non placeret comis nudata, capite spoliata, si qualis ipsa Venus cum fuit virgo omni gratiarum choro stipata, et toto cupidinun populo concinnata, baltheo suo cincta, cinnama fragrans, et balsama, si calva processerit, placere non potest Vulcano suo.}
\setauthornote{4923}{Arandus. Capilli retia Cupidinis, sylva caedua, in qua nidificat Cupido, sub cujus umbra amores mille modis se exercent.}
\setauthornote{4924}{Theod. Prodromus Amor. lib. 1.}
\setauthornote{4925}{Epist. 72. Ubi pulchram tibiam, bene compactum tenuemque pedem vidi.}
\setauthornote{4926}{Plaut. Cas.}
\setauthornote{4927}{Claudus optime rem agit.}
\setauthornote{4928}{Fol. 5. Si servum viderint, aut flatorem altius cinctum, aut pulvere perfusum, aut histrionem in scenam traductum, \&c.}
\setauthornote{4929}{Me pulchra fateor carere forma, verum luculenta-nostra est. Petronius Catal. de Priapo.}
\setauthornote{4930}{Galen.}
\setauthornote{4931}{Calcagninus Apologis. Quae pars maxime desiderabilis? Alius frontem, alius genas, \&c.}
\setauthornote{4932}{Inter foemineum.}
\setauthornote{4933}{Hensius.}
\setauthornote{4934}{Sunt enim oculi, praecipuae pulchritudinis sedes. lib. 6.}
\setauthornote{4935}{Amoris hami, duces, judices et indices qui momento insanos sanant, sanos insanire cogunt, oculatissimi corporis excubitores, quid non agunt? Quid non cogunt?}
\setauthornote{4936}{Ocelli carna. 17. cujus et Lipsius epist. quaest. lib. 3. cap. 11. meminit ob elegantiam.}
\setauthornote{4937}{Cynthia prima suis miserum me cepit ocellis, contactum nullis ante cupidinibus. Propert. l. 1.}
\setauthornote{4938}{In catalect.}
\setauthornote{4939}{De Sulpicio, lib. 4.}
\setauthornote{4940}{Pulchritudo ipsa per occultos radios in pectus amantis dimanans amatae rei formam insculpsit, Tatius, l. 5.}
\setauthornote{4941}{Jacob Cornelius Amnon Tragoed. Act. 1. sc. 1.}
\setauthornote{4942}{Rosae formosaram oculis nascuntur, et hilaritas vultus elegantiae corona. Philostratus deliciis.}
\setauthornote{4943}{Epist. et in deliciis, abi et oppugnationem relinque, quam flamma non extinguit; nam ab amore ipsa flamma sentit incendium: quae corporum penetratio, quae tyrannis haec? \&c.}
\setauthornote{4944}{Loecheus Panthea.}
\setauthornote{4945}{Propertius.}
\setauthornote{4945.5}{The wretched Cynthia first captivates with her sparkling eyes.}
\setauthornote{4946}{\Ovid{}, amorum, lib. 2. eleg. 4.}
\setauthornote{4947}{Scut. Hercul.}
\setauthornote{4948}{Calcagninus dial.}
\setauthornote{4949}{Iliad 1.}
\setauthornote{4950}{Hist. lib. 1.}
\setauthornote{4951}{Sands' relation, fol. 67.}
\setauthornote{4952}{Mantuan.}
\setauthornote{4953}{Amor per oculos, nares, poros influens, \&c. Mortales tum summopere fascinantur quando frequentissimo intuitu aciem dirigentes, \&c. Ideo si quis nitore polleat oculorum, \&c.}
\setauthornote{4954}{Spiritus puriores fascinantur, oculus a se radios emittit, \&c.}
\setauthornote{4955}{Lib. de pulch. Jes. et Mar.}
\setauthornote{4956}{Lib. 2. c. 23. colore triticum referente, crine, flava, acribus oculis.}
\setauthornote{4957}{Lippi solo intuitu alios lippos faciunt, et patet una cum radio vaporem corrupti sanguinis emanare, cujus contagione oculis spectantis inficitur.}
\setauthornote{4958}{Vita Apollon.}
\setauthornote{4959}{Comment. in Aristot. Probl.}
\setauthornote{4960}{Sic radius a corde percutientis missus, regimen proprium repetit, cor vulnerat, per oculos et sanguinem inficit et spiritus, subtili quadam vi. Castil. lib. 3. de aulico.}
\setauthornote{4961}{Lib. 10. Causa omnis et origo omnis prae sentis doloris tute es; isti enim tui oculi, per meos oculos ad intima delapsi praecordia, acerrimum meis medullis commovent incendium; ergo miserere tui causa pereuntis.}
\setauthornote{4962}{Lycias in Phaedri vultum inhiat, Phaedrus in oculos Lyciae scintillas suorum defigit oculorum; cumque scintillis, \&c. Sequitur Phaedrus Lyciam, quia cor suum petit spiritum; Phaedrum Lycias, quia spiritus propriam sedem postulat. Verum Lycias, \&c.}
\setauthornote{4963}{Daemonia inquit quae in hoc Eremo nuper occurebant.}
\setauthornote{4964}{Castilio de aulico, l. 3. fol. 228. Oculi ut milites in insidiis semper recubant, et subito ad visum sagittas emittunt, \&c.}
\setauthornote{4965}{Nec mirum si reliquos morbos qui ex contagione nascuntur consideremus, pestem, pruritum, scabiem, \&c.}
\setauthornote{4966}{\Lucretius{}.}
\setauthornote{4966.5}{And the body naturally seeks whence it is that the mind is so wounded by love.}
\setauthornote{4967}{Bacon's Essays.}
\setauthornote{4967.5}{In beauty, that of favour is preferred before that of colours, and decent motion is more than that of favour.}
\setauthornote{4968}{Martialis.}
\setauthornote{4969}{Multi tacit e opinantur commercium illud adeo frequens cum barbaris nudis, ac presertim cum foeminis ad libidinem provocare, at minus multo noxia illorum nuditas quam nostrarum foeminarum cultus. Ausim asseverare splendidum illum cultum, fucos, \&c.}
\setauthornote{4970}{Harmo. evangel. lib. 6. cap. 6.}
\setauthornote{4971}{Serm. de concep. Virg. Physiognomia virginis omnes movet ad casitatem.}
\setauthornote{4972}{3. sent. d. 3. q. 3. mirum virgo formosissima, sed a nemine concupita.}
\setauthornote{4973}{Met. 10.}
\setauthornote{4974}{Rosamond's complaint, by Sam. Daniel.}
\setauthornote{4975}{Aeneas Silv.}
\setauthornote{4976}{Heliodor. l. 2. Rodolphe Thracia tam inevitabili fascino instructa, tam exacte oculis intuens attraxit, ni si in illam quis incidisset, fieri non posset quin capertur.}
\setauthornote{4977}{Lib. 3. de providentia: Animi fenestrae oculi, et omnis improba cupiditas per ocellos tanquam canales introit.}
\setauthornote{4978}{Buchanan.}
\setauthornote{4979}{\Ovid{} de arte amandi.}
\setauthornote{4980}{Pers. 3. Sat.}
\setauthornote{4981}{Vel centum Chariles ridere putaret. Museus of Hero.}
\setauthornote{4982}{Hor. Od. 22 lib. 1.}
\setauthornote{4983}{Eustathius, l. 5.}
\setauthornote{4984}{Mantuan.}
\setauthornote{4985}{Tom. 4. merit, dial. Exornando seipsam eleganter, facilem et hilarem se gerendo erga cunctos, ridendo suave ac blandum quid, \&c.}
\setauthornote{4986}{Angeriaims.}
\setauthornote{4987}{Vel si forte vestimentum de industria elevetur, ut pedum ac tibiarum pars aliqua conspiciatur, dum templum aut locum aliquem adierit.}
\setauthornote{4988}{Sermone, quod non foeminae. viris cohabitent. Non loquuta es lingua, sed loquuta es gressu: non loquuta es voce, sed oculis loquuta es clarius quam voce.}
\setauthornote{4989}{Jovianus Pontanus Baiar. lib. 1. ad Hermionem. For why do you exhibit your 'milky way,' your uncovered bosoms? What else is it but to say plainly. Ask me, ask me, I will surrender; and what is that but love's call?}
\setauthornote{4990}{De luxu vestium discurs. 6. Nihil aliud deest nisi ut praeco vos praecedat, \&c.}
\setauthornote{4991}{If you can tell how, you may sing this to the tune a sow-gelder blows.}
\setauthornote{4992}{Auson. epig. 28. Neither draped Diana nor naked Venus pleases me. One has too much voluptuousness about her, the other none.}
\setauthornote{4993}{Plin. lib. 33. cap. 10. Gampaspen nudam picturus Apelles, amore ejus illaqueatus est.}
\setauthornote{4994}{In Tyrrhenis conviviis nudae mulieres ministrabant.}
\setauthornote{4995}{Amatoria miscentes vidit, et in ipsis complexibus audit, \&c. emersit inde cupido in pectus virginis.}
\setauthornote{4996}{Epist. 7. lib. 2.}
\setauthornote{4997}{Spartian.}
\setauthornote{4998}{Sidney's Arcadia.}
\setauthornote{4999}{De immod. mulier. cultu.}
\setauthornote{5000}{Discurs. 6. de luxu vestium.}
\setauthornote{5001}{Petronius fol. 95. quo spectant flexae comae? quo facies medicamine attrita et oculorum mollis petulantia? quo incessus tam compositus, \&c.}
\setauthornote{5002}{Ter.}
\setauthornote{5002.5}{They take a year to deck and comb themselves.}
\setauthornote{5003}{P. Aretine. Hortulanus non ita exercetur visendis hortis, eques equis, armis, nauta navibus, \&c.}
\setauthornote{5004}{Epist. 4. Sonus armillarum bene sonantium, odor unguentorm, \&c.}
\setauthornote{5005}{Tom. 4. dial. Amor. vascula plena multae infelicitatis omnem mariotorum opulentiam in haec inpendunt, dracones pro monilibus habent, qui utinam vere dracones essent. Lucian.}
\setauthornote{5006}{Seneca.}
\setauthornote{5007}{Castilio de aulic. lib. I. Mulieribus omnibus hoc imprimis in votis est, ut formosae sint, aut si reipsa non sint, videantur tamen esse; et si qua parte natura defuit, artis supetias adjungunt: unde illae faciei unctiones, dolor et cruciatus in arctandis corporibus, \&c.}
\setauthornote{5008}{\Ovid{}. epist. Med. Jasoni.}
\setauthornote{5009}{A distorted dwarf, an Europa.}
\setauthornote{5010}{Modo caudatas tunicas, \&c. Bossus.}
\setauthornote{5011}{Scribanius philos. Christ. cap. 6.}
\setauthornote{5012}{Ter. Eunuc. Act. 2. scen. 3.}
\setauthornote{5013}{Stroza fil.}
\setauthornote{5014}{\Ovid{}.}
\setauthornote{5015}{S. Daniel.}
\setauthornote{5016}{Lib. de victimis. Fracto incessu obtuitu lascivo, calamistrata, cincinnata, fucata, recens lota, purpurissata, pretioso que amicta palliolo, spirans unguenta, ut juvenum animos circumveniat.}
\setauthornote{5017}{Orat. in ebrios. Impudenter so masculorum aspectibus exponunt, insolenter comas jactantes, trahunt tunicas pedibus collidentes, oculoque petulanti, risu effuso, ad tripudium insanientes, omnem adolescentum intemperantiam in se provocantes, inque in templis memoriae martyrum consecratis; pomoerium civitatis officinam fecerunt impudentiae.}
\setauthornote{5018}{Hymno Veneri dicato.}
\setauthornote{5019}{Argonaut. l. 4.}
\setauthornote{5020}{Vit. Anton.}
\setauthornote{5021}{Regia domo ornatuque certantes, sese ac formam suam Antonio offerentes, \&c. Cum ornatu et incredibili pompa per Cydnum fluvium navigarent aurata puppi, ipsa ad similitudinem Veneris ornata, puellae Gratiis similes, pueri Cupidinibus, Antonius ad visum stupefactus.}
\setauthornote{5022}{Amictum Chlamyde et coronis, quum primum aspexit Cnemonem, ex potestate mentis excidit.}
\setauthornote{5023}{Lib. de lib. prop.}
\setauthornote{5024}{Ruth, iii. 3.}
\setauthornote{5025}{Cap. ix. 5.}
\setauthornote{5026}{Juv. Sat. 6.}
\setauthornote{5027}{Hor. lib. 2. Od. 11.}
\setauthornote{5028}{Cap. 27.}
\setauthornote{5029}{Epist. 90.}
\setauthornote{5030}{Quicquid est boni moris levitate extinguitur, et politura corporis muliebres munditias antecessimus colores meretricios viri sumimus, tenero et molli gradu suspendimus gradum, non ambulamus, nat. quaest. lib. 7. cap. 31.}
\setauthornote{5031}{Liv. lib. 4. dec. 4.}
\setauthornote{5032}{Quid exultas in pulchritudine panni? Quid gloriaris in gemmis ut facilius invites ad libidiniosum incendium? Mat. Bossus de immoder. mulie. cultu.}
\setauthornote{5033}{Epist. 113. fulgent monilibus, moribus sordent, purpurata vestis, conscientia pannosa, cap. 3. 17.}
\setauthornote{5034}{De virginali habitu: dum ornari cultius, dum evagari virgines volunt, desinunt esse virgines. Clemens Alexandrinus, lib. de pulchr. animae, ibid.}
\setauthornote{5035}{Lib. 2. de cultu mulierum, oculos depictos verecundia, inferentes in aures sermonem dei, annectentes crinibus jugum Christi, caput maritis subjicientes, sic facile et satis eritis ornatae: vestite vos serico probitatis, byssino sanctitatis, purpura pudicitiae; taliter pigmentatae deum habebitis amatorem.}
\setauthornote{5036}{Suas habeant Romanae? lascivias; purpurissa, ac cerussa ora perungant, fomenta libidinum, et corruptae mentis indicia; vestrum ornamentum deus sit, pudicitia, virtutis studium. Rossus \Plautus{}.}
\setauthornote{5037}{Sollicitiores de capitis sui decore quam de salute, inter pectinem et speculum diem perdunt, concinniores esse malunt quam honestiores, et rempub. minus turbari curant quam comam. Seneca.}
\setauthornote{5038}{Lucian.}
\setauthornote{5039}{Non sic Furius de Gallis, not Papyrius de Samnitibus, Scipio de Numantia triumphavit, ac illa se vincendo in hac parte.}
\setauthornote{5040}{Anacreon. 4. solum intuemur aurum.}
\setauthornote{5041}{Asser tecum si vis vivere mecum.}
\setauthornote{5042}{Theognis.}
\setauthornote{5043}{Chaloner, l. 9. de Repub. Ang.}
\setauthornote{5044}{Uxorem ducat Danaen, \&c.}
\setauthornote{5045}{\Ovid{}.}
\setauthornote{5046}{Epist. 14 formam spectant alii per gratias, ego pecuniam, \&c. ne mihi negotium facesse.}
\setauthornote{5047}{Qui caret argento, frustra utitur argumento.}
\setauthornote{5048}{Juvenalis.}
\setauthornote{5049}{Tom. 4. merit. dial. multos amatores rejecit, quia pater ejus nuper mortuus, ac dominus ipse factus bonorum omnium.}
\setauthornote{5050}{Lib. 3. cap. 14. quis nobilium eo tempore, sibi aut filio aut nepoti uxorem accipere cupiens, oblatam sibi aliquam propinquarum ejus non acciperet obviis manibus? Quarum turbam acciverat e Normannia in Angliam ejus rei gratia.}
\setauthornote{5051}{Alexander Gaguinus Sarmat. Europ. descript.}
\setauthornote{5052}{Tom. 3. Annal.}
\setauthornote{5053}{Libido statim deferbuit, fastidium caepit, et quod in ea tantopere adamavit aspernatur, et ab aegritudine liberatus in angorem incidit.}
\setauthornote{5054}{De puellae voluntate periculum facere solis oculis non est satis, sed efficacius aliquid agere oportet, ibique etiam machinam alteram ahibere: itaque manus tange, digitos constringe, atque inter stringendum suspira; si haec agentem aequo se animo feret, neque facta hujusmodi aspernabitur, tum vero dominam appella, ejusque collum suaviare.}
\setauthornote{5055}{Hungry dogs will eat dirty puddings.}
\setauthornote{5056}{Shakspeare.}
\setauthornote{5057}{Tatius, lib. 1.}
\setauthornote{5058}{In mammarum attractu, non aspernanda inest jucunditas, et attrectatus, \&c.}
\setauthornote{5059}{Mantuam.}
\setauthornote{5060}{\Ovid{}. 1. Met.}
\setauthornote{5061}{Manus ad cubitum nuda, coram astans, fortius intuita, tenuem de pectore spiritum ducens, digitum meum pressit, et bibens pedem pressit; mutuae compressiones corporum, labiorum commixtiones, pedum connexiones, \&c. Et bibit eodem loco, \&c.}
\setauthornote{5062}{Epist. 4. Respexi, respexit et, illa subridens, \&c.}
\setauthornote{5063}{Vir. Aen. 4.}
\setauthornote{5063.5}{That was the first hour of destruction, and the first beginning of my miseries.}
\setauthornote{5064}{Propertius.}
\setauthornote{5065}{\Ovid{}. amor. lib. 2. eleg. 2. Place modesty itself in such a situation, desire will intrude.}
\setauthornote{5066}{Romae vivens flore fortunae, et opulentiae meae, aetas, forma, gratia conversationis, maxime me fecerunt expetibilem, \&c.}
\setauthornote{5067}{De Aulic. lib. 1. fol. 63.}
\setauthornote{5068}{Ut adulterini mercatorum panni.}
\setauthornote{5069}{Busbeq. epist.}
\setauthornote{5070}{Paranympha in cubiculum adducta capillos ad cutem referebat; sponsus inde ad eam ingressus cingulum solvebat, nec prius sponsam aspexit interdiu quam ex illa factus esset pater.}
\setauthornote{5071}{Serm. cont. concub.}
\setauthornote{5072}{Lib. 2. epist. ad filium, et virginem et matrem viduam epist. 10. dabit tibi barbatulus quispiam manum, sustentabit lassam, et pressis digitis aut tentabitur aut tentabit, \&c.}
\setauthornote{5073}{Loquetur alius nutibus, et quicquid metuit dicere, significabit affectibus. Inter bas tantas voluptatum illecebras etiam ferreas mentes libido domat. Difficile inter epulas servatur pudicitia.}
\setauthornote{5074}{Clamore vestium ad se juvenes vocat; capilli fasciolis comprimuntur crispati, cingulo pectus arctatur, capilii vel in frontem, vel in aures defluunt: palliolum interdum cadit, ut nudet humeros, et quasi videri noluerit, festinans celat, quod volens detexerit.}
\setauthornote{5075}{Serm. cont. concub. In sancto et reverendo sacramentorum tempore multas occasiones, ut illis placeant qui eas vident, praebent.}
\setauthornote{5076}{Pont. Baia. l. 1.}
\setauthornote{5077}{Descr. Brit.}
\setauthornote{5078}{Res est blanda canor, discant cantare puellae pro facie, \&c. \Ovid{}. 3. de art. amandi.}
\setauthornote{5079}{Epist. l. 1. Cum loquitur Lais, quanta, O dii boni, vocis ejus dulcedo!}
\setauthornote{5080}{The sweet sound of his voice reanimates my soul through my covetous ears.}
\setauthornote{5081}{Aristenaetus, lib. 2. epist. 5. Quam suave canit! verbum audax dixi, omnium quos vidi formosissimus, utinam amare me dignetur!}
\setauthornote{5082}{Imagines, si cantantem audieris, ita demulcebere, ut parentum et patriae statim obliviscaris.}
\setauthornote{5083}{Edyll. 18. neque sane ulla sic Cytharam pulsare novit.}
\setauthornote{5084}{Amatorio Dialogo.}
\setauthornote{5085}{Puellam Cythara canentem vidimus.}
\setauthornote{5086}{Apollonius, Argonaut. l. 3. The mind is delighted as much by eloquence as beauty.}
\setauthornote{5087}{Catullus.}
\setauthornote{5088}{Parnodidascalo dial. Ital. Latin. interp. Jasper. Barthio. Germ. Fingebam honestatem plusquam virginis vestalis, intuebar oculis uxoris, addebam gestus, \&c.}
\setauthornote{5089}{Tom. 4. dial. merit.}
\setauthornote{5090}{Amatorius sermo vehemens vehementis cupiditatis incitatio est, Tatius l. 1.}
\setauthornote{5091}{De luxuria et deliciis compositi.}
\setauthornote{5092}{Aeneas Sylvius. Nulla machina validior quam lecto lascivae historiae: saepe etiam hujusmodi fabulis ad furorem incenduntur.}
\setauthornote{5093}{Martial. l. 4.}
\setauthornote{5094}{Lib. 1. c. 7.}
\setauthornote{5095}{Eustathius, l. 1. Pictures parant animum ad Venerem, \&c. Horatius ed res venereas intemperantior traditur; nam cubiculo suo sic specula dicitur habuisse disposita, ut quocunque respexisset imaginem coitus referrent. Suetonius vit. ejus.}
\setauthornote{5096}{Osculum ut phylangium inficit.}
\setauthornote{5097}{Hor. Venus hath imbued with the quintessence of her nectar.}
\setauthornote{5098}{Heinsius.}
\setauthornote{5098.5}{You may conquer with the sword, but you are conquered by a kiss.}
\setauthornote{5099}{Applico me illi proximius et spisse deosculata sagum peto.}
\setauthornote{5100}{Petronius catalect.}
\setauthornote{5101}{Catullus ad Lesbiam: da mihi basia mille, deinde centum, \&c.}
\setauthornote{5102}{Petronius.}
\setauthornote{5102.5}{Only attempt to touch her person, and immediately your members will be filled with a glow of delicious warmth.}
\setauthornote{5103}{\Apuleius, l. 30. et Catalect.}
\setauthornote{5104}{Petronius.}
\setauthornote{5105}{\Apuleius.}
\setauthornote{5106}{Petronius Proselios ad Circen.}
\setauthornote{5107}{Petronius.}
\setauthornote{5108}{Animus conjungitur, et spiritus etiam noster per osculum effluit; alternatim se in utriusque corpus infundentes commiscent; animae potius quam corporis connectio.}
\setauthornote{5109}{Catullus.}
\setauthornote{5110}{Lucian. Tom. 4.}
\setauthornote{5111}{Non dat basia, dat Nera nectar, dat rores animae suaveolentes, dat nardum, thymumque, cinnamumque et mel, \&c. Secundus bas. 4.}
\setauthornote{5112}{Eustathius lib. 4.}
\setauthornote{5113}{Catullus.}
\setauthornote{5114}{Buchanan.}
\setauthornote{5115}{\Ovid{}. art. am. Eleg. 18.}
\setauthornote{5116}{\Ovid{}.}
\setauthornote{5116.5}{She folded her arms around my neck.}
\setauthornote{5117}{Cum capita liment solitis morsiunculis, et cum mammillarum pressiunculis. Lip. od. ant. lec. lib. 3.}
\setauthornote{5118}{Tom. 4. dial. meretr.}
\setauthornote{5119}{\Apuleius Miles. 6. Et unum blandientis linguae admulsum longe mellitum: et post lib. 11. Arctius eam complexus caepi suaviari jamque pariter patentis oris inhalitu cinnameo et occursantis linguae illisu nectareo, \&c.}
\setauthornote{5120}{Lib. 1 advers. Jovin. cap. 30.}
\setauthornote{5121}{Oscula qui sumpsit, si non et cetera sumpsit, \&c.}
\setauthornote{5122}{Corpus Placuit mariti sui tolli ex arca, atque illi quae vacabat cruci adfigi.}
\setauthornote{5123}{Novi ingenium mulierum, nolunt, ubi velis, ubi nolis capiunt ultro. Ter. Eunuc. act. 4. sc. 7.}
\setauthornote{5124}{Marlowe.}
\setauthornote{5125}{Pornodidascolo dial. Ital. Latin. donat. a Gasp. Barthio Germano. Quanquam natura, et arte eram formosissima, isto tamen astu tanto speciosior videbar, quod enim oculis cupitum aegre praebetur, multo magis affectus humanos incendit.}
\setauthornote{5126}{Quo majoribus me donis probatiabat, eo pejoribus illum modis tractabam, ne basium impetravis, \&c.}
\setauthornote{5127}{Comes de monte Turco Hispanus has de venatione sua partes misit, jussitque peramanter orare, ut hoc qualecunque donum suo nomine accipias.}
\setauthornote{5128}{His artibus hominem ita excantabam, ut pro me ille ad omnia parutas, \&c.}
\setauthornote{5129}{Tom. 4. dial, merit.}
\setauthornote{5130}{Relicto illo, aegre ipsi interim faciens, et omnino difficilis.}
\setauthornote{5131}{Si quis enim nec Zelotypus irascitur, nec pugnat aliquando amator, nec perjurat, non est habendus amator, \&c. Totus hic ignis Zelotypia constat, \&c. maxime amores inde nascuntur. Sed si persuasum illi fuerit te solum habere, elanguescit illico amor suus.}
\setauthornote{5132}{Venientem videbis ipsum denuo inflammatum et prorsus insanientem.}
\setauthornote{5133}{Et sic cum fere de illo desperassem, post menses quatuor ad me rediit.}
\setauthornote{5134}{Petronius Catal.}
\setauthornote{5135}{Imagines deorum. fol. 327. varios amores facit, quos aliqui interpretantur multiplices affectus et illecebras, alios puellos, puellas, alatos, alios poma aurea, alios sagittas, alios laqueos, \&c.}
\setauthornote{5136}{Epist. lib. 3. vita Pauli Eremitae.}
\setauthornote{5137}{Meretrix speciosa cepit delicatius stringere colla complexibus, et corpora in libidinem concitato, \&c.}
\setauthornote{5138}{Camden in Gloucestershire, huic praefuit nobilis et formosa abbatissa, Godwinus comes indole subtilis, non ipsam, sed sua cupiens, reliquit nepotem suum forma elegantissimum, tanquam infirmum donec reverteretur, instruit, \&c.}
\setauthornote{5139}{Ille impiger regem adit, abatissam et suas praegnantes edocet, exploratoribus missis probat, et iis ejectis, a domino suo manerium accepit.}
\setauthornote{5140}{Post sermones de casu suo suavitate sermones conciliat animum hominis, manumque inter colloquia et risus ad barbam protendit et palpare coepit cervicem suam et osculari; quid multa? Captivum ducit militem Christi. Complexura evanescit, demones in aere monachum riserunt.}
\setauthornote{5141}{Choraea circulus, cujus centrum diab.}
\setauthornote{5142}{Multae inde impudicae domum rediere, plures ambiguae, melior nulla.}
\setauthornote{5143}{Turpium deliciarum comes est externa saltatio; neque certe facile dictu quae mala hinc visus hauriat, et quae pariat, colloquia, monstrosus, inconditos gestus, \&c.}
\setauthornote{5144}{Juv. Sat. 11. Perhaps you may expect that a Gaditanian with a tuneful company may begin to wanton, and girls approved with applause lower themselves to the ground in a lascivious manner, a provocative of languishing desire.}
\setauthornote{5145}{Justin. l. 10. Adduntur instrumenta luxuriae, tympana et tripudia; nec tam spectator rex, sed nequitiae magister, \&c.}
\setauthornote{5146}{Hor. l. 5. od. 6.}
\setauthornote{5147}{Havarde vita ejus.}
\setauthornote{5148}{Of whom he begat William the Conqueror; by the same token she tore her smock down, saying, \&c.}
\setauthornote{5149}{Epist. \&c. Quis non miratus est saltantem? Quis non vidit et amavit? veterem et novam vidi Romam, sed tibi similem non vidi Panareta; felix qui Panareta fruitur, \&c.}
\setauthornote{5150}{Prinicipio Ariadne velut sponsa prodit, ac sola recedit; prodiens illico Dionysius ad numeros cantante tibia saltabat; admirati sunt omnes saltantem juvenem, ipsaque Ariadne, ut vix potuerit conquiescere; post ea vero cum Dionysius eam aspexit, \&c. ut autem surrexit Dionysius, erexit simul Ariadnem, licebatque spectare gestus osculantium, et inter se complectentium; qui autem spectabant, \&c. Ad extremum videntes eos mutuis amplexibus implicatos et jamjam ad thalamum ituros; qui non duxerant uxores jurabant uxores se ductoreos; qui autem duxerant conscensis equis et incitatis, ut iisdem fruerentur, domum festinarunt.}
\setauthornote{5151}{Lib. 4. de contemnend. amoribus.}
\setauthornote{5152}{Ad Anysium epist. 57.}
\setauthornote{5153}{Intempestivum enim est, et a nuptiis abhorrens, inter saltantes podagricum videre senem, et episcopum.}
\setauthornote{5154}{Rem omnium in mortalium vita optimam innocenter accusare.}
\setauthornote{5155}{Quae honestam voluptatem respicit, aut corporis exercitium, contemni non debet.}
\setauthornote{5156}{Elegantissima res est, quae et mentem acuit, corpus exerceat, et spectantes oblectet, multos gestus decoros docens, oculos, aures, animum ex aeque demulcens.}
\setauthornote{5157}{\Ovid{}.}
\setauthornote{5158}{System, moralis philosophiae.}
\setauthornote{5159}{\Apuleius. 10. Pueili, puellaeque virenti florentes aetatula, forma conspicui, veste nitidi, incessu gratiosi, Graecanicam saltantes Pyrrhicam, dispositis ordinationibus, decoros ambitus inerrabant, nunc in orbem flexi, nunc in obliquam seriem connexi, nunc in quadrum cuneati, nunc inde separati, \&c.}
\setauthornote{5160}{Lib. 1. cap. 11.}
\setauthornote{5161}{Vit. Epaminondae.}
\setauthornote{5162}{Lib. 5.}
\setauthornote{5163}{Read P. Martyr Ocean Decad. Benzo, Lerius Hacluit, \&c.}
\setauthornote{5164}{Angerianus Erotopaedium.}
\setauthornote{5165}{10 Leg. \textgreek{τῆς γὰρ τοιαύτης σπεδῆς ἔνεκα}, \&c. \textlatin{hujus causa oportuit disciplinam constitui, ut tam pueri quam puellae choreas celebrent, spectenturque ac spectent, \&c.}}
\setauthornote{5166}{Aspectus enim nudorum corporum tam mares quam feminas irritare solet ad enormes lasciviae appetitus.}
\setauthornote{5167}{Camden Annal. anno 1578, fol. 276. Amatoriis facetiis et illecebris exquisitissimus.}
\setauthornote{5168}{Met. 1. \Ovid{}.}
\setauthornote{5169}{Erasmus egl. mille mei siculis errant in montibus agni.}
\setauthornote{5170}{Virg.}
\setauthornote{5171}{58 Lecheus.}
\setauthornote{5172}{Tom. 4. merit. dial. amare se jurat et lachrimatur dicitque uxorem me ducere velle, quum pater oculos claussisset.}
\setauthornote{5173}{Quum dotem alibi multo majorem aspiciet, \&c.}
\setauthornote{5174}{Or upper garment. Quem Juno miserata veste contexit.}
\setauthornote{5175}{Hor.}
\setauthornote{5176}{Dejeravit illa secundum supra trigesimum ad proximum Decembrem completuram se esse.}
\setauthornote{5177}{\Ovid{}.}
\setauthornote{5178}{Nam donis vincitur omnis amor. Catullus 1. el. 5.}
\setauthornote{5179}{Fox, act. 3. sc. 3.}
\setauthornote{5180}{Catullus.}
\setauthornote{5181}{Perjuria ridet amantum Jupiter, et ventos irrita ferre jubet Tibul. lib. 3. et 6.}
\setauthornote{5182}{In Philebo. pejerantibus, nis dii soli ignoscunt.}
\setauthornote{5183}{Catul.}
\setauthornote{5184}{Lib. 1. de contemnendis amoribus.}
\setauthornote{5185}{Dial. Ital. argentum ut paleas projiciebat. Biliosum habui amatorem qui supplex flexis genibus, \&c. Nullus recens allatus terrae fructus, nullum cupediarum genus tam carum erat, nullum vinum Creticum pretiosum, quin ad me ferret illico; credo alterum oculum pignori daturus, \&c.}
\setauthornote{5186}{Post musicam opiperas epulas, et tantis juramentis, donis, \&c.}
\setauthornote{5187}{Nunquam aliquis umbrarum conjurator tanta attentione, tamque potentibus verbis usus est, quam ille exquisitis mihi dictis, \&c.}
\setauthornote{5188}{Chaucer.}
\setauthornote{5189}{Ah crudele genas nec tutum foemina nomen! Tibul. l. 3. eleg. 4.}
\setauthornote{5190}{Jovianus Pon.}
\setauthornote{5191}{Aristaenetus, lib. 2. epist. 13.}
\setauthornote{5192}{Suaviter flebam, ut persuasum habeat lachrymas prae gaudio illius reditus mihi emanare.}
\setauthornote{5193}{Lib. 3. his accedunt, vultus subtristis, color pallidus, gemebunda vox, ignita suspiria, lachrymae prope innumerabiles. Istae se statim umbrae offerunt tanto squalore et in omni fere diverticulo tanta macie, ut illas jamjam moribundas putes.}
\setauthornote{5194}{Petronius.}
\setauthornote{5194.5}{Trust not your heart to women, for the wave is less treacherous than their fidelity.}
\setauthornote{5195}{Coelestina, act 7. Barthio interpret omnibus arridet, et a singulis amari se solam dicit.}
\setauthornote{5196}{\Ovid{}.}
\setauthornote{5196.5}{They have made the same promises to a thousand girls that they make to you.}
\setauthornote{5197}{Seneca Hippol.}
\setauthornote{5198}{Tom. 4. dial. merit. tu vero aliquando maerore afficieris ubi andieris me a meipsa laqueo tui causa suffocatam aut in puteum praecipitatam.}
\setauthornote{5199}{Epist. 20. l. 2.}
\setauthornote{5200}{Matronae flent duobus oculis, moniales quatuor, virgines uno, meretrices nullo.}
\setauthornote{5201}{\Ovid{}.}
\setauthornote{5202}{Imagines deorum, fol. 332. e Moschi amore fugitive, quem Politianus Latinum fecit.}
\setauthornote{5203}{Lib. 3. mille vix anni sufficerent ad omnes illas machinationes, dolosque commemorandos, quos viri et mulieres ut se invicem circumveniant, excogitare solent.}
\setauthornote{5204}{Petronius.}
\setauthornote{5205}{\Plautus{} Tritemius. Three hundred verses would not comprise their indecencies.}
\setauthornote{5206}{De Magnet. Philos. lib. 4. cap. 10.}
\setauthornote{5207}{Catul. eleg. 5. lib. 1. Venit in exitium callida lena meum.}
\setauthornote{5208}{\Ovid{}. 10. met.}
\setauthornote{5209}{Parabosc. Barthii.}
\setauthornote{5210}{De vit. Erem c. 3. ad sororem vix aliquam reclusarum hujus temporis solam invenies, ante cujus fenestram non anus garrula, vel nugigerula mulier sedet, quae eam fabulis occupet, rumoribus pascat, hujus vel illius monachi, \&c.}
\setauthornote{5211}{Agreste olus anus vendebat, et rogo inquam, mater, nunquid scis ubi ego habitem? delectata illa urbanitate tam stulta, et quid nesciam inquit? consurrexitque et cepit me praecedere; divinam ego putabam, \&c. nudas video meretrices et in lupanar me adductum, sero execrutus aniculae insidias.}
\setauthornote{5212}{\Plautus{} Menech. These harlots send little maidens down to the quays to ascertain the name and nation of every ship that arrives, after which they themselves hasten to address the new-comers.}
\setauthornote{5213}{Promissis everberant, molliunt dulciloquiis, et opportunum tempus aucupantes laqueos ingerunt quos vix Lucretia vitare; escam parant quam vel satur Hippolitus sumeret, \&c. Hae sane sunt virgae soporiferae quibus contactae animae ad Orcum descendunt; hoc gluten quo compactae mentium alae evolare nequeunt, daemonis ancillae, quae sollicitant, \&c.}
\setauthornote{5214}{See the practices of the Jesuits, Anglice, edit. 1630.}
\setauthornote{5215}{Aen. Sylv.}
\setauthornote{5216}{Chaucer, in the wife of Bath's tale.}
\setauthornote{5217}{H. Stephanus Apol. Herod, lib. 1. cap. 21.}
\setauthornote{5218}{Bale. Puellae in lectis dormire non poterant.}
\setauthornote{5219}{Idem Josephus, lib. 18. cap. 4.}
\setauthornote{5220}{Lib credit. Augustae Vindelicorum, An. 1608.}
\setauthornote{5221}{Quarum animas lucrari debent Deo, sacrificant diabolo.}
\setauthornote{5222}{M. Drayton, Her. epist.}
\setauthornote{5223}{Pornodidascolo dial. Ital. Latin, fact. a Gasp. Barthio. Plus possum quam omnes philosophi, astrologi, necromantici, \&c. sola saliva inungens, 1. amplexu et basiis tam furiose furere, tam bestialiter obstupesieri coegi, ut instar idoli me adorarint.}
\setauthornote{5224}{Sagae omnes sibi arrogant notitiam, et facultatem in amorem alliciendi quos velint; odia inter conjuges serendi, tempestates excitandi, morbos infligendi, \&c.}
\setauthornote{5225}{Juvenalis Sat.}
\setauthornote{5226}{Idem refert Hen. Kormannus de mir. mort. lib. 1 cap. 14. Perdite amavit mulierculam quandam, illius amplexibus acquiescens, summa cum indignatione suorum et dolore.}
\setauthornote{5227}{Et inde totus in Episcopum furere, illum colere.}
\setauthornote{5228}{Aquisgranum, vulgo Aixe.}
\setauthornote{5229}{Immenso sumptu templum et aedes, \&c.}
\setauthornote{5230}{Apolog. quod Pudentillam viduam ditem et provectioris aetatis foeminam cantaminibus in amorem sui pellexisset.}
\setauthornote{5231}{Philopseude, tom. 3.}
\setauthornote{5232}{Impudicae mulieres opera veneficarum, diaboli coquarum, amatores suos ad se nuctu ducunt et reducunt, ministerio hirci in aere volantis: multos novi qui hoc fassi sunt, \&c.}
\setauthornote{5233}{Mandrake apples, Lemnius lib. herb. bib. c. 3.}
\setauthornote{5234}{Of which read Plin. lib. 8. cap. 22. et lib. 13. c. 25. et Quintilianum, lib. 7.}
\setauthornote{5235}{Lib. 11. c. 8. Venere implicat eos, qui ex eo bibunt. Idem Ov. Met. 4. Strabo. Geog. l. 14.}
\setauthornote{5236}{Lod. Guicciardine's descript. Ger. in Aquisgrano.}
\setauthornote{5237}{Baltheus Veneris, in quo suavitas, et dulcia colloquia, benevolentiae, et blanditiae, suasiones, fraudes et veneficia includebantur. Whence that heat to waters bubbling from the cold moist earth? Cupid, once upon a time, playfully dipped herein his arrows of steel, and delighted with the hissing sound, he said, boil on for ever, and retain the memory of my quiver. From that time it is a thermal spring, in which few venture to bathe, but whosoever does, his heart is instantly touched with love.}
\setauthornote{5238}{\Ovid{}. Facit hunc amor ipse colorem. Met. 4.}
\setauthornote{5239}{Signa ejus profunditas oculorum, privatio lachrymarum, suspiria, saepe rident sibi, ac si quod delectabile; viderent, aut audirent.}
\setauthornote{5240}{Seneca Hip.}
\setauthornote{5241}{Seneca Hip.}
\setauthornote{5242}{De moris cerebri de erot. amore. Ob spirituum distractionem hepar officio suo non fungitur, nec vertit alimentum in sanguinem, ut debeat. Ergo membra debilia, et penuria alibilis succi marcescunt, squalentque ut herbae in horto meo hoc mense Maio Zeriscae, ob imbrium defectum.}
\setauthornote{5243}{Faerie Queene, l. 3. cant. 11.}
\setauthornote{5244}{Amator Emblem. 3.}
\setauthornote{5245}{Lib. 4. Animo errat, et quidvis obvium loquitur, vigilias absque causa sustinet, et succum corporis subito amisit.}
\setauthornote{5246}{\Apuleius.}
\setauthornote{5247}{Chaucer, in the Knight's Tale.}
\setauthornote{5248}{Virg. Aen. 4.}
\setauthornote{5249}{Dum vaga passim sidera fulgent, numerat longas tetricus horas, et sollicito nixus cubito suspirando viscera rumpit.}
\setauthornote{5250}{Saliebat crebro tepidum cor ad aspectum Ismenes.}
\setauthornote{5251}{Gordonius c. 20. amittunt saepe cibum, potum, et merceratur inde totum corpus.}
\setauthornote{5252}{Ter. Eunuch. Dii boni, quid hoc est, adeone homines mutari ex amore, ut non cognoscas eundem esse!}
\setauthornote{5253}{\Ovid{}. Met. 4. The more it is concealed the more it struggles to break through its concealment.}
\setauthornote{5254}{Ad ejus nomen, rubebut, et ad aspectum pulsus variebatur. Plutar.}
\setauthornote{5255}{Epist. 13.}
\setauthornote{5256}{Barck. lib. 1. Oculi medico tremore errabant.}
\setauthornote{5257}{Pulsus eorum velox et inordinatus, si mulier quam amat forte transeat.}
\setauthornote{5258}{Signa sunt cessatio ab omni opere insueto, privatio somni, suspiria crebra, rubor cum sit sermo de re amata, et commotio pulsus.}
\setauthornote{5259}{Si noscere vis an homines suspecti tales sint, tangito eorum arterias.}
\setauthornote{5260}{Amor facit inaequales, inordinatos.}
\setauthornote{5261}{In nobilis cujusdam uxore quum subolfacerem adulteri amore fuisse correptam et quam maritus, \&c.}
\setauthornote{5262}{Cepit illico pulsus variari et ferri celerius et sic inveni.}
\setauthornote{5263}{Eunuch, act. 2. scen. 2.}
\setauthornote{5264}{Epist. 7. lib. 2. Tener sudor et creber anhelitus, palpitatio cordis, \&c.}
\setauthornote{5265}{Lib. 1.}
\setauthornote{5266}{Lexoviensis episcopus.}
\setauthornote{5267}{Theodorus prodromus Amaranto dial. Gaulimo interpret.}
\setauthornote{5268}{Petron. Catal.}
\setauthornote{5269}{Sed unum ego usque et unum Petam a tuis labellis, postque unum et unum et unum, dari rogabo. Loecheus Anacreon.}
\setauthornote{5270}{Jo. Secundus, bas. 7.}
\setauthornote{5271}{Translated or imitated by M. B. Johnson, our arch poet, in his 119 ep.}
\setauthornote{5272}{Lucret. l. 4.}
\setauthornote{5273}{Lucian. dial. Tom. 4. Merit, sed et aperientes, \&c.}
\setauthornote{5274}{Epist. 16.}
\setauthornote{5275}{Deducto ore longo me basio demulcet.}
\setauthornote{5276}{In deliciis mammas tuas tango, \&c.}
\setauthornote{5277}{Terent.}
\setauthornote{5278}{Tom. 4. merit, dial.}
\setauthornote{5279}{Attente adeo in me aspexit, et interdum ingemiscebat, et lachrymabatur. Et si quando bibens, \&c.}
\setauthornote{5280}{Quique omnia cernere debes Leucothoen spectas, et virgine figis in una quos mundo debes oculos, \Ovid{}. Met. 4.}
\setauthornote{5281}{Lucian. tom. 3. quoties ad cariam venis currum sistis, et desuper aspectas.}
\setauthornote{5282}{Ex quo te primum vidi Pythia alio oculos vertere non fuit.}
\setauthornote{5283}{Lib. 4.}
\setauthornote{5284}{Dial, amorum.}
\setauthornote{5285}{Ad occasum solis aegre domum rediens, atque totum die ex adverso deae sedens recto, in ipsam perpetuo oculorum ictus direxit, \&c.}
\setauthornote{5286}{Lib. 3.}
\setauthornote{5287}{Regum palatium non tam diligenti custodia septum fuit, ac aedes meas stipabant, \&c.}
\setauthornote{5288}{Uno, et eodem die sexties vel septies ambulant per eandem plateam ut vel unico amicae suae fruantur aspectu, lib. 3. Theat. Mundi.}
\setauthornote{5289}{Hor.}
\setauthornote{5290}{\Ovid{}.}
\setauthornote{5291}{\Ovid{}.}
\setauthornote{5292}{Hyginus, fab. 59. Eo die dicitur nonies ad littus currisse.}
\setauthornote{5293}{Chaucer.}
\setauthornote{5294}{Gen. xxix. 20.}
\setauthornote{5295}{\Plautus{} Cistel.}
\setauthornote{5296}{Stobaeus e Graeco. Sweeter than honey it pleases me, more bitter than gall, it teases me.}
\setauthornote{5297}{\Plautus{}: Credo ego ad hominis carnificinam amorem inventum esse.}
\setauthornote{5298}{De civitat. lib. 22. cap. 20. Ex eo oriuntur mordaces curae, perturbationes, maerores, formidines, insana gaudia, discordiae, lites, bella, insidiae, iracundiae, inimicitiae, fallaciae, adulatio, fraus, furtum, nequitia, impudentia.}
\setauthornote{5299}{Marullus, l. 1.}
\setauthornote{5300}{Ter. Eunuch.}
\setauthornote{5301}{\Plautus{} Mercat.}
\setauthornote{5302}{\Ovid{}.}
\setauthornote{5303}{Adelphi, Act. 4. scen. 5. M. Bono animo es, duces uxorem hanc Aeschines. Ae. Hem. pater, num tu ludis me nunc? M. Egone te, quamobrem? Ae. Quod tam misere cupio, \&c.}
\setauthornote{5304}{Tom. 4. dial. amorum.}
\setauthornote{5305}{Aristotle, 2. Rhet. puts love therefore in the irascible part. \Ovid{}.}
\setauthornote{5306}{Ter. Eunuch. Act. 1. sc. 2.}
\setauthornote{5307}{\Plautus{}.}
\setauthornote{5308}{Tom. 3.}
\setauthornote{5309}{Scis quod posthac dicturus fuerim.}
\setauthornote{5310}{Tom. 4. dial. merit. Tryphena, amor me perdit, neque malum hoc amplius sustinere possum.}
\setauthornote{5311}{Aristaenetus, lib. 2. epist. 8.}
\setauthornote{5312}{Coelestinae, act 1. Sancti majora laetitia non fruuntur. Si mihi Deus omnium votorum mortalium summam concedat, non magis, \&c.}
\setauthornote{5313}{Catullus de Lesbia.}
\setauthornote{5314}{Hor. ode 9. lib. 3.}
\setauthornote{5315}{Act. 3. scen. 5. Eunuch. Ter.}
\setauthornote{5316}{Act. 5. scen. 9.}
\setauthornote{5317}{Mantuan.}
\setauthornote{5318}{Ter. Adelph. 3. 4.}
\setauthornote{5319}{Lib. 1. de contemn. amoribus. Si quem alium respexerit amica suavius, et familiarius, si quem aloquuta fuerit, si nutu, nuncio, \&c. statim cruciatar.}
\setauthornote{5320}{Calisto in Celestina.}
\setauthornote{5321}{Pornodidasc. dial. Ital. Patre et matre se singultu orbos censebant, quod meo contubernio carendum esset.}
\setauthornote{5322}{Ter. tui carendum quod erat.}
\setauthornote{5323}{Si responsum esset dominam occupatam esse aliisque vacaret, ille statim vix hoc audito velut in amor obriguit, alii se damnare, \&c. at cui favebam, in campis Elysiis esse videbatur, \&c.}
\setauthornote{5324}{Mantuan.}
\setauthornote{5325}{Laecheus.}
\setauthornote{5326}{Sole se occultante, aut tempestate veniente, statim clauditur ac languescit.}
\setauthornote{5327}{Emblem, amat. 13.}
\setauthornote{5328}{Calisto de Melebaea.}
\setauthornote{5329}{Anima non est ubi animat, sed ubi amat.}
\setauthornote{5330}{Celestine, act. 1. credo in Melebaeam, \&c.}
\setauthornote{5331}{Ter. Eunuch, act. 1. sc. 2.}
\setauthornote{5332}{Virg. 4. Aen.}
\setauthornote{5333}{Interdiu oculi, et aures occupatae distrahunt animum, at noctu solus jactor, ad auroram somnus paulum misertus, nec tamen ex animo puella abiit, sed omnia mihi de Leucippe somnia erant.}
\setauthornote{5334}{Tota hac nocte somnum hisce oculis non vidi. Ter.}
\setauthornote{5335}{Buchanan. syl.}
\setauthornote{5336}{Aen. Sylv. Te dies, noctesque amo, te cogito, te desidero, te voco, te expecto, te spero, tecum oblecto me, totus in te sum.}
\setauthornote{5337}{Hor. lib. 2. ode 9.}
\setauthornote{5338}{Petronius.}
\setauthornote{5339}{Tibullus, l. 3. Eleg. 3.}
\setauthornote{5340}{\Ovid{}. Fast. 2. ver. 775.}
\setauthornote{5340.5}{Although the presence of her fair form is wanting, the love which it kindled remains.}
\setauthornote{5341}{Virg. Aen. 4.}
\setauthornote{5342}{De Pythonissa.}
\setauthornote{5343}{Juno, nec ira deum tantum, nec tela, nec hostis, quantum tute potis animis illapsus. Silius Ital. 15. bel. Punic. de amore.}
\setauthornote{5344}{Philostratus vita ejus. Maximum tormentum quod excogitare, vel docere te possum, est ipse amor.}
\setauthornote{5345}{Ausonius c. 35.}
\setauthornote{5346}{Et caeco carpitur igne; et mihi sese offert ultra meus ignis Amyntas.}
\setauthornote{5347}{Ter. Eunuc.}
\setauthornote{5348}{Sen. Hippol.}
\setauthornote{5349}{Theocritus, edyl. 2. Levibus cor est violabile telis.}
\setauthornote{5350}{Ignis tangentes solum urit, at forma procul astantes inflammat.}
\setauthornote{5351}{Nonius.}
\setauthornote{5352}{Major illa flamma quae consumit unam animam, quam quae centum millia corporum.}
\setauthornote{5353}{Mant. egl. 2.}
\setauthornote{5354}{Marullus Epig. lib. 1.}
\setauthornote{5355}{Imagines deorum.}
\setauthornote{5356}{\Ovid{}.}
\setauthornote{5357}{Aeneid. 4.}
\setauthornote{5358}{Seneca.}
\setauthornote{5359}{Cor totum combustum, jecur suffumigatum, pulmo arefactus, ut credam miseram illam animam bis elixam aut combustam, ob maximum ardorem quem patiuntur ob ignem amoris.}
\setauthornote{5360}{Embl. Amat. 4. et 5.}
\setauthornote{5361}{Grotius.}
\setauthornote{5362}{Lib. 4. nam istius amoris neque principia, neque media aliud habent quid, quam molestias, dolores, cruciatus, defatigationes, adeo ut miserum esse maerore, gemitu, solitudine torqueri, mortem optare. semperque debacchari, sint certa amantium signa et certae actiones.}
\setauthornote{5363}{Virg. Aen. 4.}
\setauthornote{5363.5}{The works are interrupted, promises of great walls, and scaffoldings rising towards the skies, are all suspended.}
\setauthornote{5364}{Seneca Hip. act.}
\setauthornote{5364.5}{The shuttle stops, and the web hangs unfinished from her hands.}
\setauthornote{5365}{Eclog. 1.}
\setauthornote{5365.5}{No rest, no business pleased my lovesick breast, my faculties became dormant, my mind torpid, and I lost my taste for poetry and song.}
\setauthornote{5366}{Edyl. 14.}
\setauthornote{5367}{Mant. Eclog.}
\setauthornote{5368}{Ter. Eunuch.}
\setauthornote{5369}{Ov. Met. de Polyphemo: uritur oblitus pecorum, antrorumque suorum; jamque tibi formae, \&c.}
\setauthornote{5370}{Qui quaeso? Amo.}
\setauthornote{5371}{Ter. Eunuch.}
\setauthornote{5372}{Qui olim cogitabat quae vellet, et pulcherrimis philosophiae praeceptis operam insumpsit, qui universi circuitiones coelique naturam, \&c. Hanc unam intendit operam, de sola cogitat, noctes et dies se componit ad hanc, et ad acerbam servitutem redactus animus, \&c.}
\setauthornote{5373}{Pars epitaphii ejus.}
\setauthornote{5374}{Epist. prima.}
\setauthornote{5375}{Boethius l. 3 Met. ult.}
\setauthornote{5376}{Epist. lib. 6. Valeat pudor, valeat honestas, valeat honor.}
\setauthornote{5377}{Theodor. prodromus, lib. 3. Amor Mystili genibus ovolutis, ubertemque lachrimas, \&c. Nihil ex tota praeda praeter Rhodanthem virginem accipiam.}
\setauthornote{5378}{Lib. 2. Certe vix credam, et bona fide fateare Aratine, te no amasse adeo vehementer; si enim vere amasses, nihil prius aut potius optasses, quam amatae mulieri placere. Ea enim amoris lex est idem velle et nolle.}
\setauthornote{5379}{Stroza, sil. Epig.}
\setauthornote{5380}{Quippe haec omnia ex atra bile et amore proveniunt. Jason Pratensis.}
\setauthornote{5381}{Immense amor ipse stultitia est. Carda, lib. 1. de sapientia.}
\setauthornote{5382}{Mantuan.}
\setauthornote{5382.5}{Whoever is in love is in slavery, he follows his sweetheart as a captive his captor, and wears a yoke on his sumbissibe neck.}
\setauthornote{5383}{Virg. Aen. 4.}
\setauthornote{5383.5}{She began to speak but stopped in the middle of her discourse.}
\setauthornote{5384}{Seneca, Hippol. What reason requires, raging love forbids.}
\setauthornote{5385}{Met. 10.}
\setauthornote{5386}{Buchanan.}
\setauthornote{5386.5}{Oh fraud, and love, and distraction of mind, whither have you led me?}
\setauthornote{5387}{An immodest woman is like a bear.}
\setauthornote{5388}{Feram induit cum rosas comedat, idem ad se redeat.}
\setauthornote{5389}{Alciatus de upupa Embl. Animal immundum upupa stercora amans; ave hac nihil foedius, nihil libidinosius. Sabin in \Ovid{}. Met.}
\setauthornote{5390}{is like a false glass, which represents everything fairer than it is.}
\setauthornote{5391}{Hor. ser. lib. sat. l. 3. These very things please him, as the wen of Agna did Balbinus.}
\setauthornote{5392}{The daughter and heir of Carolus Pugnax.}
\setauthornote{5393}{Seneca in Octavia.}
\setauthornote{5393.5}{Her beauty excels the Tyndarian Helen's, which caused such dreadful wars.}
\setauthornote{5394}{Loecheus.}
\setauthornote{5395}{Mantuan, Egl 1.}
\setauthornote{5396}{Angerianus.}
\setauthornote{5397}{Faerie Queene, Cant. lyr. 4.}
\setauthornote{5398}{Epist. 12. Quis unquam formas vidit orientis, quis occidentis, veniant undique omnes, et dicant veraces an tam insignem viderint formam.}
\setauthornote{5399}{Nulla vox formam ejus possit comprehendere.}
\setauthornote{5400}{Caleagnini dit. Galat.}
\setauthornote{5401}{Catullus.}
\setauthornote{5402}{Petronii Catalect.}
\setauthornote{5403}{Chaucer, in the Knight's Tale.}
\setauthornote{5404}{\Ovid{}, Met. 13.}
\setauthornote{5405}{It is envy evidently that prompts you, because Polyphemus does not love you as he does me.}
\setauthornote{5406}{Plutarch. sibi dixit tam pulchram non videri, \&c.}
\setauthornote{5407}{Quanto quam Lucifer aurea Phoebe, tanto virginibus conspectior omnibus Herce. \Ovid{}.}
\setauthornote{5408}{M. D. Son. 30.}
\setauthornote{5409}{Martial., l. 5. Epig. 38.}
\setauthornote{5410}{Ariosto.}
\setauthornote{5411}{Tully lib. 1. de nat. deor. pulchrior deo, et tamen erat oculis perversissimis.}
\setauthornote{5412}{Marullus ad Neaeram epig. 1. lib.}
\setauthornote{5413}{Barthius.}
\setauthornote{5414}{Ariosto, lib. 29. hist. 8.}
\setauthornote{5415}{Tibulius.}
\setauthornote{5416}{Marul. lib. 2.}
\setauthornote{5417}{Tibullus l. 4. de Sulpicia.}
\setauthornote{5418}{Aristenaetus, Epist. 1.}
\setauthornote{5419}{Epist. 24. veni cito charissime Lycia, cito veni; prae te Satyri omnes videntur non homines, nullo loco solus es, \&c.}
\setauthornote{5420}{Lib. 3. de aulico, alterius affectui se totum componit, totus placere studet, et ipsius animam amatae pedisequam facit.}
\setauthornote{5421}{Cyropaed. l. 5. amor servitus, et qui amant optat se liberari non secus ac alio quovis morbo, neque liberari tamen possunt, sed validiori necessitate ligati sunt quam si in ferrea vincula confectiforent.}
\setauthornote{5422}{In paradoxis, An ille mihi liber videtur cui mulier imperat? Cui leges imponit, praescribit, jubet, vetat quod videtur. Qui nihil imperanti negat, nihil audet, \&c. poscit? dandum; vocat? veniendum; minatur? extimiscendum.}
\setauthornote{5423}{Illane parva est servitus amatorum singulis fere horis pectine capillum, calimistroque barbam componere, faciem aquis redolentibus diluere, \&c.}
\setauthornote{5424}{Si quando in pavimentum incautius quid mihi excidisset, elevare inde quam promptissime, nec nisi osculo compacto mihi commendare, \&c.}
\setauthornote{5425}{Nor will the rude rocks affright, me, nor the crooked-tusked bear, so that I shall not visit my mistress in pleasant mood.}
\setauthornote{5426}{Plutarchus amat. dial.}
\setauthornote{5427}{Lib. 1. de contem. amor. quid referam eorum pericula et clades, qui in amicarum aedes per fenestras ingressi stillicidiaque egressi indeque deturbati, sed aut praecipites, membra frangunt, collidunt, aut animam amittunt.}
\setauthornote{5428}{Ter. Eunuch. Act. 5. Scen. 8.}
\setauthornote{5429}{Paratus sum ad obeundum mortem, si tu jubeas; hanc sitim aestuantis seda, quam tuum sidus perdidit, aquae et fontes non negant, \&c.}
\setauthornote{5430}{Si occidere placet, ferrum meum vides, si verberibus contenta es, curro nudus ad poenam.}
\setauthornote{5431}{Act. 15. 18. Impera mihi; occidam decem viros, \&c.}
\setauthornote{5432}{Gasper Ens. puellam misere deperiens, per jocum ab ea in Padum desilire jussus statim e ponte se praecipitavit. Alius Ficino insano amore ardens ab amica jussus se suspendere, illico fecit.}
\setauthornote{5433}{Intelligo pecuniam rem esse jucundissimam, meam tamen libentius darem Cliniae quam ab aliis acciperem; libentius huic servirem, quam aliis imperarem, \&c. Noctem et somnum accuso, quod illum non videam, luci autem et soli gratiam habeo quod mihi Cliniam ostendant. Ego etiam cum Clinia in ignem currerem; et scio vos quoque mecum ingressuros si videretis.}
\setauthornote{5434}{Impera quidvis; navigare jube, navem conscendo; plagas accipere, plector; animum profundere, in ignem currere, non recuso, lubens facio.}
\setauthornote{5435}{Seneca in Hipp. act. 2.}
\setauthornote{5436}{Hujus ero vivus, mortuus hujus ero. Propert. lib. 2. vivam si vivat; si cadat illa, cadam, Id.}
\setauthornote{5437}{Dial. Amorum. Mihi o dii coelestes ultra sit vita haec perpetua ex adverso amicae sedere, et suave loquentem audire, \&c. si moriatur, vivere non sustinebo, et idem erit se pulchrum utrisque.}
\setauthornote{5438}{Buchanan.}
\setauthornote{5438.5}{When she dies my love shall also be at rest in the tomb.}
\setauthornote{5439}{Epist. 21. Sit hoc votum a diis amare Delphidem, ab ea amari, adloqui pulchram et loquentem audire.}
\setauthornote{5440}{Hor.}
\setauthornote{5441}{Mart.}
\setauthornote{5442}{Lege Calimitates Pet. Abelhardi Epist. prima.}
\setauthornote{5443}{Ariosto.}
\setauthornote{5444}{Chaucer, in the Knight's Tale.}
\setauthornote{5445}{Theodorus prodromus, Amorum lib. 6. Interpret. Gaulmino.}
\setauthornote{5446}{\Ovid{}. 10. Met. Higinius, c. 185.}
\setauthornote{5447}{Ariost. lib. 1. Cant. 1. staff. 5.}
\setauthornote{5448}{Plut. dial. amor.}
\setauthornote{5449}{Faerie Queene, cant. 1. lib. 4. et cant. 3. lib. 4.}
\setauthornote{5450}{Dum cassis pertusa, ensis instar Serrae excisus, scutum, \&c. Barthius Caelestina.}
\setauthornote{5451}{Lesbia sex cyathis, septem Justina bibatur.}
\setauthornote{5452}{As Xanthus for the love of Eurippe, omnem Europam peragravit. Parthenius Erot cap. 8.}
\setauthornote{5453}{Beroaldus e Bocatio.}
\setauthornote{5454}{Epist. 17. l. 2.}
\setauthornote{5455}{\Lucretius{}.}
\setauthornote{5455.5}{For if the object of your love be absent, her image is present, and her sweet name is still familiar in my ears.}
\setauthornote{5456}{Aeneas Sylvius, Lucretie quum accepit Euriali literas hilaris statim milliesqua papirum basiavit.}
\setauthornote{5457}{Mediis inseruit papillis litteram ejus, mille prius pangens suavia. Arist. 2. epist. 13.}
\setauthornote{5458}{\Plautus{} Asinar.}
\setauthornote{5459}{Hor.}
\setauthornote{5459.5}{Some token snatched from her arm or her gently resisting finger.}
\setauthornote{5460}{Illa domi sedens imaginem ejus fixis oculis assidue conspicata.}
\setauthornote{5461}{And distracted will imprint kisses on the doors.}
\setauthornote{5462}{Buchanan Sylva.}
\setauthornote{5463}{Fracastorius Naugerio.}
\setauthornote{5463.5}{Ye alpine winds, ye mountain breezes, bear these gifts to her.}
\setauthornote{5464}{Happy servants that serve her, happy men that are in her company.}
\setauthornote{5465}{Non ipsos solum sed ipsorum memoriam amant. Lucian.}
\setauthornote{5466}{Epist. O ter felix solum! beatus ego, si me calcaveris; vultus tuus amnes sistere potest, \&c.}
\setauthornote{5467}{Idem epist. in prato cum sit flores superat; illi pulchri sed unius tantum diei; fluvius gratis sed evanescit; at tuus fluvius mari major. Si coelum aspicio, solem exis timo cecidisse, et in terra ambulare, \&c.}
\setauthornote{5468}{Si civitate egrederis, sequentur te dii custodes, spectaculo commoti; si naviges sequentur; quis fluvius salum tuum non rigaret?}
\setauthornote{5469}{El. 15. 2.}
\setauthornote{5470}{Oh, if I might only dally with thee, and alleviate the wasting sorrows of my mind.}
\setauthornote{5471}{Carm. 30.}
\setauthornote{5472}{Englished by M. B. Holliday, in his Technog. act 1. scen. 7.}
\setauthornote{5473}{\Ovid{}. Met. lib. 4.}
\setauthornote{5474}{Xenophon Cyropaed. lib. 5.}
\setauthornote{5475}{\Plautus{} de milite.}
\setauthornote{5476}{Lucian.}
\setauthornote{5477}{E Graeco Ruf.}
\setauthornote{5478}{Petronius.}
\setauthornote{5479}{He is happy who sees thee, more happy who hears, a god who enjoys thee.}
\setauthornote{5480}{Lod. Vertomannus navig. lib. 2. c. 5. O deus, hunc creasti sole candidiorem, e diverso me et conjugem meum et natos meos omnes nigricantes. Utinam hic, \&c. Ibit Gazella, Tegeia, Galzerana, et promissis oneravit, et donis. \&c.}
\setauthornote{5481}{M. D.}
\setauthornote{5482}{Hor. Ode 9. lib. 3.}
\setauthornote{5483}{Ov. Met. 10.}
\setauthornote{5484}{Buchanan. Hendecasyl.}
\setauthornote{5485}{Petrarch.}
\setauthornote{5486}{Cardan, lib. 2. de sap ex vilibus generosos efficere solet, ex timidis audaces, ex avaris splendidos, ex agrestibus civiles, ex crudelibus mansuetos, ex impiis religiosos, ex sordidis nitidos atque cultos, ex duris misericordes, ex mutis eloquentes.}
\setauthornote{5487}{Anima hominis amore capti tota referta suffitibus et odoribus: Paeanes resonat, \&c.}
\setauthornote{5488}{\Ovid{}.}
\setauthornote{5489}{In convivio, amor Veneris Martem detinet, et fortem facit; adolescentem maxime erubescere cernimus quum amatrixeum eum turpe quid committentem ostendit.}
\setauthornote{5490}{Plutarch. Amator. dial.}
\setauthornote{5491}{Si quo pacto fieri civitas aut exercitus posset partim ex his qui amant, partim ex his, \&c.}
\setauthornote{5492}{Angerianus.}
\setauthornote{5493}{Faerie Qu. lib. 4. cant. 2.}
\setauthornote{5494}{Zened. proverb. cont. 6.}
\setauthornote{5495}{Plat. conviv.}
\setauthornote{5496}{Lib. 3. de Aulico. Non dubito quin is qui talem exercitum haberet, totius orbis statim victor esset, nisi forte cum aliquo exercitu confligendum esset in quo omnes amatores essent.}
\setauthornote{5497}{Higinus de cane et lepore coelesti, et decimator.}
\setauthornote{5498}{Vix dici potest quantam inde audaciam assumerent Hispani, inde pauci infinitas Maurorum copias superarunt.}
\setauthornote{5499}{Lib. 5. de legibus.}
\setauthornote{5500}{Spenser's Faerie Queene, 3. book. cant. 8.}
\setauthornote{5501}{Hyginus, l. 2.}
\setauthornote{5501.5}{For love both inspires us with stratagems, and suggests to us frauds.}
\setauthornote{5502}{Aratus in phaenom.}
\setauthornote{5503}{Virg.}
\setauthornote{5503.5}{Who can deceive a lover.}
\setauthornote{5504}{Hanc ubi conspicatus est Cymon, baculo innixus, immobilis stetit, et mirabundus, \&c.}
\setauthornote{5505}{\Plautus{} Casina, act. 2. sc. 4.}
\setauthornote{5506}{\Plautus{}.}
\setauthornote{5507}{\Ovid{}. Met. 2.}
\setauthornote{5508}{\Ovid{}. Met. 4.}
\setauthornote{5509}{Virg. 1. Aen.}
\setauthornote{5509.5}{He resembled a god as to his head and shoulders, for his mother had made his hair seem beautiful, bestowed upon him the lovely bloom of youth, and given the happiest lustre to his eyes.}
\setauthornote{5510}{\Ovid{}. Met. 13.}
\setauthornote{5511}{Virg. E. l. 2.}
\setauthornote{5511.5}{I am not so deformed, I lately saw myself in the tranquil glassy sea, as I stood upon the shore.}
\setauthornote{5512}{Epist. An uxor literato sit ducenda. Noctes insomnes traducendae, literis renunciandum, saepe gemendum, nonnunquam et illacrymandum sorti et conditioni tuae. Videndum quae vestes, quis cultus, te deceat, quis in usu sit, utrum latus barbae, \&c. Cum cura loquendum, incedendum, bibendum et cum cura insaniendum.}
\setauthornote{5513}{Mart. Epig. 5.}
\setauthornote{5514}{Chil. 4. cent. 5. pro. 16.}
\setauthornote{5515}{Martianus. Capella lib. 1. de nupt. philol. Jam. Illum sentio amore teneri, ejusque studio plures habere comparatas in famultio disciplinas, \&c.}
\setauthornote{5516}{Lib. 3. de aulico. Quis choreis insudaret, nisi foeminarum causa? Quis musicae tantam navaret operam nisi quod illius dulcedine permulcere speret? Quis tot carmina componeret, nisi ut inde affectus suos in mulieres explicaret?}
\setauthornote{5517}{Craterem nectaris evertit saltans apud Deos, qui in terram cadens, rosam prius albam rubore infecit.}
\setauthornote{5518}{Puellas choreantes circa juvenilem Cupidinis statuam fecit. Philostrat. Imag. lib. 3. de statuis. Exercitium amori aptissimum.}
\setauthornote{5519}{Lib. 6. Met.}
\setauthornote{5520}{Tom. 4.}
\setauthornote{5521}{Kornman de cur. mort. part. 5 cap. 28. Sat. puellae dormienti insultantium, \&c.}
\setauthornote{5522}{View of Fr.}
\setauthornote{5523}{Vita ejus Puellae, amore septuagenarius senex usque ad insaniam correptus, multis liberis susceptis: multi non sine pudore conspexerunt senem et philosophum podagricium, non sine risu saltantem ad tibiae modos.}
\setauthornote{5524}{Anacreon. Carm. 7.}
\setauthornote{5525}{Joach. Bellius Epig.}
\setauthornote{5525.5}{Thus youth dies, thus in death he loves.}
\setauthornote{5526}{De taciturno loquacem facit, et de verecundo officiosum reddit, de negligente industrium, de socorde impigrum.}
\setauthornote{5527}{Josephus antiq. Jud. lib. 18. cap. 4.}
\setauthornote{5528}{Gellius, l. 1. cap. 8. Pretium noctis centum sestertia.}
\setauthornote{5529}{Ipsi enim volunt suarum amasiarum pulchritudinis praeecones ac testes esse, eas laudibus, et cantilenis et versibus exonare, ut auro statuas, ut memorentur, et ab omnibus admirentur.}
\setauthornote{5530}{Tom. 2. Ant. Dialogo.}
\setauthornote{5531}{Flores hist. fol. 298.}
\setauthornote{5532}{Per totum annum cantarunt, pluvia super illos non cecidit; non frigus, non calor, non sitis, nec lassitudo illos affecit, \&c.}
\setauthornote{5533}{His eorum nomina inscribuntur de quibus quaerunt.}
\setauthornote{5534}{Huic munditias, ornatum, leporem, delicias, ludos, elegantiam, omnem denique vitae suavitatem debemus.}
\setauthornote{5535}{Hyginus cap. 272.}
\setauthornote{5536}{E Graeco.}
\setauthornote{5537}{Angerianus.}
\setauthornote{5538}{Lib. 4. tit. 11. de prin. instit.}
\setauthornote{5539}{Plin. lib. 35. cap. 12.}
\setauthornote{5540}{Gerbelius, l. 6. descript. Gr.}
\setauthornote{5541}{Fransus, l. 3. de symbolis qui primus symbolum excogitavit voluit nimirum hac ratione implicatum animum evolvere, eumque vel dominae vel aliis intuentibus ostendere.}
\setauthornote{5542}{Lib. 4. num. 102. Sylvae nuptialis poetae non inveniunt fabulas, aut versus laudatos faciunt, nisi qui ab amore fuerint excitati.}
\setauthornote{5543}{Martial, ep. 73. lib. 9.}
\setauthornote{5544}{Virg. Eclog. 4.}
\setauthornote{5544.5}{None shall excel me in poetry, neither the Thracian Orpheus, nor Apollo.}
\setauthornote{5545}{Teneris arboribus amicarum nomina inscribentes ut simul crescant. Haed.}
\setauthornote{5546}{S. R. 1600.}
\setauthornote{5547}{Lib. 13. cap. Dipnosophist.}
\setauthornote{5548}{See Putean. epist. 33 de sua Margareta Beroaldus, \&c.}
\setauthornote{5549}{Hen. Steph. apol. pro Herod.}
\setauthornote{5550}{Tully orat. 5. ver.}
\setauthornote{5551}{Esth. v.}
\setauthornote{5552}{Mat. l. 47.}
\setauthornote{5553}{Gravissimis regni negotiis nihil sine amasiae suae consensu fecit, omnesque actiones suas scortillo communicavit, \&c. Nich. Bellus. discours. 26. de amat.}
\setauthornote{5554}{Amoris famulus omnem scientiam diffitetur, amandi tamen se scientissimum doctorem agnoscit.}
\setauthornote{5555}{Serm. 8.}
\setauthornote{5556}{Quis horum scribere molestias potest, nisi qui et is aliquantum insanit?}
\setauthornote{5557}{Lib. 1. de non temnendis amoribus; opinor hac de re neminem aut desceptare recte posse aut judicare qui non in ea versatur, aut magnum fecerit periculum.}
\setauthornote{5558}{I am not in love, nor do I know what love may be.}
\setauthornote{5559}{Semper moritur, nunquam mortuus est qui amat. Aen. Sylv.}
\setauthornote{5560}{Eurial. ep. ad Lucretiam, apud Aeneam Sylvium; Rogas ut amare deficiam? roga montes ut in planum deveniant, ut fontes flumina repetant; tam possum te non amare ac suum Phoebus relinquere cursum.}
\setauthornote{5561}{Buchanan Syl.}
\setauthornote{5562}{Propert. lib. 2. eleg. 1.}
\setauthornote{5563}{Est orcus illa vis, est immedicabilis, est rabies insana.}
\setauthornote{5564}{Lib. 2.}
\setauthornote{5565}{Virg. Ecl. 3.}
\setauthornote{5566}{R. T.}
\setauthornote{5567}{Qui quidem amor utrosque et totam Egyptum extremis calamitatibus involvit.}
\setauthornote{5568}{\Plautus{}.}
\setauthornote{5569}{Ut corpus pondere, sic animus amore praecipitatur. \Austin{}. l. 2. de civ. dei, c. 28.}
\setauthornote{5570}{Dial. hinc oritur paenitentia desperatio, et non vident ingenium se cum re simul amisisse.}
\setauthornote{5571}{Idem Savanarola, et plures alii, \&c. Rabidem facturus Orexin. Juven.}
\setauthornote{5572}{Cap. de Heroico Amore. Haec passio durans sanguinem torridum et atrabiliarum reddit; ale vero ad cerebrum delatus, insaniam parat, vigilia et crebro desiderio exsiccans.}
\setauthornote{5573}{Virg. Egl. 2.}
\setauthornote{5573.5}{Oh Corydon, Corydon! what madness possesses you?}
\setauthornote{5574}{Insani fiunt aut sibi ipsis desperantes mortem afferunt. Languentes cito mortem aut maniam patiuntur.}
\setauthornote{5575}{Calcagninus.}
\setauthornote{5576}{Lucian Imag.}
\setauthornote{5576.5}{So for Lucian's mistress, all that saw her, and could not enjoy her, ran mad, or hanged themselves.}
\setauthornote{5577}{Musaeus.}
\setauthornote{5578}{\Ovid{}. Met. 10. Aeneas Sylvius. Ad ejus decessum nunquam visa Lucretia ridere, nullis facetiis, jocis, nullo gaudio potuit ad laetitiam renovari, mox in aegritudinem incidit, et sic brevi contabuit.}
\setauthornote{5579}{Anacreon.}
\setauthornote{5580}{But let me die, she says, thus; thus it is better to descend to the shades.}
\setauthornote{5581}{Pausanias Achaicis, l. 7.}
\setauthornote{5582}{Megarensis amore flagrans Lucian. Tom. 4.}
\setauthornote{5583}{\Ovid{}. 3. met.}
\setauthornote{5584}{Furibundus putavit se videre imaginem puellae, et coram loqui blandiens illi, \&c.}
\setauthornote{5585}{Juven. Hebreus.}
\setauthornote{5586}{Juvenis Medicinae operam dans doctoris filiam deperibat, \&c.}
\setauthornote{5587}{Gotardus Arthus Gallobelgicus, nund. vernal. 1615. collum novacula aperuit: et inde expiravit.}
\setauthornote{5588}{Cum renuente parente utroque et ipsa virgine frui non posset, ipsum et ipsam interfecit, hoc a magistratu petens, ut in eodem sepulchro sepeliri possent.}
\setauthornote{5589}{Boccaccio.}
\setauthornote{5590}{Sedes eorum qui pro amoris impatientia pereunt, Virg. 6. Aenid.}
\setauthornote{5591}{Whom cruel love with its wasting power destroyed.}
\setauthornote{5592}{And a myrtle grove overshadow thee; nor do cares relinquish thee even in death itself.}
\setauthornote{5593}{Sal. Val.}
\setauthornote{5594}{Sabel. lib. 3. En. 6.}
\setauthornote{5595}{Curtius, lib. 5.}
\setauthornote{5596}{Chalcocondilas de reb. Tuscicis, lib. 9. Nerei uxor Athenarum domina, \&c.}
\setauthornote{5597}{Nicephorus Greg. hist. lib. 8. Uxorem occidit liberos et Michaelem filium videre abhorruit. Thessalonicae amore captus pronotarii, filiae, \&c.}
\setauthornote{5598}{Parthenius Erot. lib. cap. 5.}
\setauthornote{5599}{Idem ca. 21. Gubernatoris alia Achillis amore capta civitatem proditit.}
\setauthornote{5600}{Idem. cap. 9.}
\setauthornote{5601}{Virg. Aen 6.}
\setauthornote{5602}{Otium naufragium castitatis. \Austin{}.}
\setauthornote{5603}{Buchanan. Hendeca syl.}
\setauthornote{5604}{\Ovid{} lib. 1. remed.}
\setauthornote{5604.5}{Love yields to business; be employed, and you'll be safe.}
\setauthornote{5605}{Cap. 16. circares arduas exerceri.}
\setauthornote{5606}{Part 2. c. 23. reg. San. His, praeter horam somni, nulla per otium transeat.}
\setauthornote{5607}{Hor. lib. I. epist. 2.}
\setauthornote{5608}{Seneca.}
\setauthornote{5609}{Poverty has not the means of feeding her passion.}
\setauthornote{5610}{Tract. 16. cap. 18. saepe nuda carne cilicium portent tempore frigido sine caligis, et nudis pedibus incedant, in pane et aqua jejunent, saepius se verberbus caedant, \&c.}
\setauthornote{5611}{Daemonibus referta sunt corpora nostra, illorum praecipue qui delicatis vescuntur eduliis, advolitant, et corporibus inherent; hanc ob rem jejunium impendio probatur ad pudicitiam.}
\setauthornote{5612}{Victus sit attenuatus, balnei frequens usus et sudationes, cold baths, not hot, saith Magninus, part 3. ca. 23. to dive over head and ears in a cold river, \&c.}
\setauthornote{5613}{Ser. de gula; fames amica virginitati est, inimica lasciviae: saturitas vero castitatem perdit, et nutrit illecebras.}
\setauthornote{5614}{Vita Hilarionis, lib. 3. epist. cum tentasset eum daemon titillatione inter caetera, Ego inquit, aselle, ad corpus suum, faciam, \&c.}
\setauthornote{5615}{Strabo. l. 15. Geog. sub pellibus, cubant, \&c.}
\setauthornote{5616}{Cup. 2. part. 2. Si sit juvenis, et non vult obedire, flagelletur frequenter et fortiter, dum incipiat foetere.}
\setauthornote{5617}{Laertius, lib. 6. cap. 5. amori medetur fames; sin aliter, tempus; sin non hoc, laqueus.}
\setauthornote{5618}{Vina parant animos Veneri, \&c.}
\setauthornote{5619}{3. de Legibus.}
\setauthornote{5620}{Non minus si vinum bibissent ac si adulterium admisissent, Gellius, lib. 10. c. 23.}
\setauthornote{5621}{Rer. Sam. part. 3. cap. 23. Mirabilem vim habet.}
\setauthornote{5622}{Cum muliere aliqua gratiosa saepe coire erit utilissimum. Idem Laurentius, cap. 11.}
\setauthornote{5623}{Hor.}
\setauthornote{5624}{Cap. 29. de morb. cereb.}
\setauthornote{5625}{Beroaldus orat. de amore.}
\setauthornote{5626}{Amatori, cujus est pro impotentia mens amota, opus est ut paulatim animus velut a peregrinatione domum revocetur per musicam, convivia, \&c. Per aucupium, fabulas, et festivas narrationes, laborem usque ad sudorem, \&c.}
\setauthornote{5627}{Caelestinae, Act. 2 Barthio interpret.}
\setauthornote{5628}{Cap. de Illishi. Multus hoc affectu sanat cantilena, laetitia, musica; et quidam sunt quoshaec angent.}
\setauthornote{5629}{This author came to my hands since the third edition of this book.}
\setauthornote{5630}{Cent. 3 curat. 56. Syrupo helleborato et aliis quae ad atram bilem pertinent.}
\setauthornote{5631}{Purgetur si ejus dispositio venerit ad adust, humoris, et phlebotomizetur.}
\setauthornote{5632}{Amantium morbus ut pruritus solvitur, venae sectione et cucurbitulus.}
\setauthornote{5633}{Cura a venae sectione per aures, unde semper steriles.}
\setauthornote{5634}{Seneca.}
\setauthornote{5635}{Cum in mulierem incident, quae cum forma morum suavitatem conjunctam habet, et jam oculos persenserit formae ad se imaginem cum aviditate quadam rapere cum eadem, \&c.}
\setauthornote{5636}{23 \Ovid{}, de rem. lib. 1.}
\setauthornote{5637}{Aeneas Silvius.}
\setauthornote{5638}{\Plautus{} gurcu. Remove and throw her quite out of doors, she who has drank my lovesick blood.}
\setauthornote{5639}{Tom. 2. lib. 4. cap. 10. Syntag. med. arc. Mira. vitentur oscula, tactus sermo, et scripta impudica, literae, \&c.}
\setauthornote{5640}{Lib. de singul. Cler.}
\setauthornote{5641}{Tam admirabilem splendorem declinet, gratiam, scintillas, amabiles risus, gestus suavissimos, \&c.}
\setauthornote{5642}{Lipsius, hort. leg. lib. 3. antiq. lec.}
\setauthornote{5643}{Lib. 3. de vit. coelitus compar. cap. 6.}
\setauthornote{5644}{\Lucretius{}.}
\setauthornote{5644.5}{It is best to shun the semblance and the food of love, to abstain from it, and totally avert the mind from the object.}
\setauthornote{5645}{Lib. 3. eleg. 10.}
\setauthornote{5646}{Job xxxi. Pepigi foedus cum oculis meis ne cogitarem de virgine.}
\setauthornote{5647}{Dial. 3. de contemptu mundi; nihil facilius recrudescit quam amor; ut pompa visa renovat ambitionem, auri species avaritiam, spectata corporis forma incendit luxuriam.}
\setauthornote{5648}{Seneca cont. lib. 2. cont. 9.}
\setauthornote{5649}{\Ovid{}.}
\setauthornote{5650}{Met. 7. ut solet a ventis alimenta resumere, quaeque Pavia sub inducta latuit scintilla favilla. Crescere et in veteres agitata resurgere flammas.}
\setauthornote{5651}{Eustathii l. 3. aspectus amorem incendit, ut marcescentem in palea ignem ventus; ardebam interea majore concepto incendio.}
\setauthornote{5652}{Heliodorus, l. 4. inflammat mentem novus aspectus, perinde ac ignis materiae admotus, Chariclia, \&c.}
\setauthornote{5653}{Epist. 15. l. 2.}
\setauthornote{5654}{Epist. 4. l. 2.}
\setauthornote{5655}{Curtius, lib. 3. cum uxorem Darii laudatam audivisset, tantum cupiditati suae fraenum injecit, ut illam vix vellet intueri.}
\setauthornote{5656}{Cyropaedia. cum Pantheae forman evexisset Araspus, tanto magis, inquit Cyrus abstinere oportet, quanto pulchrior est.}
\setauthornote{5657}{Livius, cum eam regulo cuidam desponsaram audivisset muneribus cumulatam remisit.}
\setauthornote{5658}{Ep. 39. lib. 7.}
\setauthornote{5659}{Et ea loqui posset quae soli amatores loqui solent.}
\setauthornote{5660}{Platonis Convivio.}
\setauthornote{5661}{Heliodorus, lib. 4. expertem esse amoris beatitudo est; at quum captus sis, ad moderationem revocare animum prudentia singularis.}
\setauthornote{5662}{\Lucretius{}, l. 4.}
\setauthornote{5663}{Haedus, lib. 1. de amor. contem.}
\setauthornote{5664}{Loci mutatione tanquam non convalescens curandus est. cap. 11.}
\setauthornote{5665}{Fly the cherished shore. It is advisable to withdraw from the places near it.}
\setauthornote{5666}{\Ovid{}. I Amorum, l. 2.}
\setauthornote{5666.5}{Depart, and take a long journey-safety is in flight only.}
\setauthornote{5667}{Quisquis amat, loca nota nocent; dies aegritudinem adimit, absentia delet. Ire licet procul hinc patriaeque relinquere fines. \Ovid{}.}
\setauthornote{5668}{Lib. 3. eleg. 20.}
\setauthornote{5669}{Lib. 1. Socrat. memor. Tibi O Critobule consulo ut integrum annum absis, \&c.}
\setauthornote{5670}{Proximum est ut esurias 2. ut moram temporis opponas. 3. et locum mutes. 4. ut de laqueo cogites.}
\setauthornote{5671}{Philostratus de vita Sophistratum.}
\setauthornote{5672}{Virg, 6. Aen.}
\setauthornote{5673}{Buchanan.}
\setauthornote{5674}{Annuncientur valde tristia, ut major tristitia possit minorem obfuscare.}
\setauthornote{5675}{Aut quod sit factus senescallus, aut habeat honorem magnum.}
\setauthornote{5676}{Adolescens Graecus erat in Egypti coenobio qui nulla operis magnitudine, nulla persuasione flammam poterat sedare: monasterii pater hac arte servavit. Imperat cuidam e sociis, \&c. Flebat ille, omnes adversabantur; solus pater calide opponere, ne abundantia tristitiae absorberetur, quid multa? hoc invento curatus est, et a cogitationibus pristinis avocatus.}
\setauthornote{5677}{Tom. 4.}
\setauthornote{5678}{Ter.}
\setauthornote{5679}{Hypatia Alexandrina quendam se adamantem prolatis muliebribus pannis, et in cum conjectis ab amoris insania laboravit. Suidas et Eunapius.}
\setauthornote{5680}{Savanarola, reg. 5.}
\setauthornote{5681}{Virg. Ecl. 3}
\setauthornote{5681.5}{You will easily find another if this Alexis disdains you.}
\setauthornote{5682}{Distributio amoris fiat in plures, ad plures amicas animum applicet.}
\setauthornote{5683}{\Ovid{}.}
\setauthornote{5683.5}{I recommend you to have two mistresses.}
\setauthornote{5684}{Higinus, sab. 43.}
\setauthornote{5685}{Petronius.}
\setauthornote{5686}{Lib. de salt.}
\setauthornote{5687}{E theatro egressus hilaris, ac si pharmacum oblivionis bibisset.}
\setauthornote{5688}{Mus in cista natus, \&c.}
\setauthornote{5689}{In quem e specu subterraneo modicum lucis illabitur.}
\setauthornote{5690}{Deplorabant eorum miseriam qui subterraneis illis locis vitam degunt.}
\setauthornote{5691}{Tatius lib. 6.}
\setauthornote{5692}{Aristaenetus, epist. 4.}
\setauthornote{5693}{Calcaguin. Dial. Galat. Mox aliam praetulit, aliam praelaturus quam primum occasio arriserit.}
\setauthornote{5694}{Epist. lib. 2. 16. Philosophi saeculi veterem amorem novo, quasi clavum clavo repellere, quod et Assuero regi septem principes Persarum fecere, ut Vastae reginae desiderium amore compensarent.}
\setauthornote{5695}{\Ovid{}. One love extracts the influence of another.}
\setauthornote{5696}{Lugubri veste indutus; consolationes non admisit, donec Caesar ex ducali sanguine, formosam virginem matrimonio conjunxit. Aeneas Sylvias hist. de Euryalo et Lucretia.}
\setauthornote{5697}{Ter.}
\setauthornote{5698}{Virg. Ecl. 2.}
\setauthornote{5698.5}{For what limit has love?}
\setauthornote{5699}{Lib. de beat. vit. cap. 14.}
\setauthornote{5700}{Longo usu dicimus, longa desuetudine dediscendum est. Petrarch, epist. lib. 5. 8.}
\setauthornote{5701}{Tom. 4. dial. meret. Fortusse etiam ipsa ad amorem istum connihil contulero.}
\setauthornote{5702}{Quid enim meretrix nisi juventutis expilatrix, virorum rapina seu mors; patrimonii devoratrix, honoris pernicies, pabulum diaboli, janua mortis, inferni supplementum?}
\setauthornote{5703}{Sanguinem hominum sorbent.}
\setauthornote{5704}{Contemplatione Idiotae, c. 34. discrimen vitae, mors blanda, mel sclleum, dulce venenum, pernicies delicata, mallum spontaneum, \&c.}
\setauthornote{5705}{Pornodidasc. dial. Ital. gula, ira, invidia, superbia, sacritegia, latrocinia, caedes, eo die nata sunt, quo primum meretrix professionem fecit. Superbia major quam opulenti rustici, invidia quam luis venerae inimicitia nocentior melancholia, avaritia in immensum profunda.}
\setauthornote{5706}{Qualis extra sum vides, qualis intra novit Deus.}
\setauthornote{5707}{Virg. He calls Mnestheus, Surgestus, and the brave Cloanthus, and orders them silently to prepare the fleet.}
\setauthornote{5708}{He is moved by no tears, he cannot he induced to hear her words.}
\setauthornote{5709}{Tom. 2. in votis. Caivus cum sis, nasum habeas simum, \&c.}
\setauthornote{5710}{Petronius.}
\setauthornote{5711}{\Ovid{}.}
\setauthornote{5712}{In Catarticis, lib. 2.}
\setauthornote{5713}{Si ferveat deformis, ecce formosa est; si frigeat formosa, jam sis informis. Th. Morus Epigram.}
\setauthornote{5714}{Amorum dial. tom. 4. si quis ad auroram contempletur multas mulieres a nocte lecto surgentes, turpiores putabit esse bestiis.}
\setauthornote{5715}{Hugo de claustro Animae, lib. 1. c. 1. If you quietly reflect upon what passes through her mouth, nostrils, and other conduits of her body, you never saw viler stuff.}
\setauthornote{5716}{Hist. nat. 11. cap. 35. A fly that hath golden wings but a poisoned body.}
\setauthornote{5717}{Buchanan, Hendecasyl.}
\setauthornote{5718}{Apol. pro Rem. Seb.}
\setauthornote{5719}{6 \Ovid{}. 2. rem.}
\setauthornote{5720}{Post unam noctem incertum unde offensam cepit propter foetentem ejus spiritum alii dicunt, vel latentem foeditatem repudiavit, rem faciens plane illicitam, et regiae personae multum indecoram.}
\setauthornote{5721}{Hall and Grafton belike.}
\setauthornote{5722}{Juvenal.}
\setauthornote{5722.5}{When the wrinkled skin becomes flabby, and the teeth black.}
\setauthornote{5723}{Mart.}
\setauthornote{5724}{Tully in Cat.}
\setauthornote{5724.5}{Because wrinkles and hoary locks disfigure you.}
\setauthornote{5725}{Hor. ode. 13. lib. 4.}
\setauthornote{5726}{Locheus.}
\setauthornote{5726.5}{Beautiful cheeks, rosy lips, and languishing eyes}
\setauthornote{5727}{Qualis fuit Venus cum fuit virgo, balsamum spirans, \&c.}
\setauthornote{5728}{Seneca.}
\setauthornote{5729}{Seneca Hyp.}
%\setauthornote{5729.5}{Beauty is a gift of dubious worth to mortals, and of brief duration.}
\setauthornote{5730}{Camerarius, emb. 68. cent. 1. flos omnium pulcherrimus statim languescit, formae typus.}
\setauthornote{5731}{Bernar. Bauhusius Ep. l. 4.}
\setauthornote{5732}{Pausanias Lacon. lib. 3. uxorem duxit Spartae mulierum omnium post Helenam formosissimam, at ob mores omnium turpissimam.}
\setauthornote{5733}{Epist. 76. gladium bonum dices, non cui deauratus est baltheus, nec cui vagina gemmis distinguitur, sed cui ad secandum subtilis acies et mucro munimentum omne rupturus.}
\setauthornote{5734}{Pulchritudo corporis, temporis et fugacior ludibrium. orat. 2.}
\setauthornote{5735}{Florum mutabilitate fugacior, nec sua natura formosas facit, sed spectantium infirmitas.}
\setauthornote{5736}{Epist. 11. Quem ego depereo juvenis mihi pulcherimus videtur; sed forsan amore percita de amore non recte judico.}
\setauthornote{5737}{Luc. Brugensis.}
\setauthornote{5737.5}{Bright eyes and snow-white neck.}
\setauthornote{5738}{Idem.}
\setauthornote{5738.5}{Let my Melita's eyes be like Juno's, her hand Minerva's, her breasts Venus', her leg Ampbitiles'.}
\setauthornote{5739}{Bebelius adagiis Ger.}
\setauthornote{5740}{Petron. Cat.}
%\setauthornote{5740.5}{Let her eyes be as bright as the stars, her neck smell like the rose, her hair shine more than gold, her honied lips be ruby coloured; let her beauty be resplendent, and superior to Venus, let her be in all respects a deity, \&c.}
\setauthornote{5741}{M. Drayton.}
\setauthornote{5742}{Senec. act. 2. Herc. Oeteus.}
\setauthornote{5743}{Vides venustam mulierem, fulgidum habentem oculum, vultu hilari coruscantem, eximium quendam aspectum et decorem praese ferentem, urentem mentem tuam, et concupiscentiam agentem; cogita terram esse id quod amas, et quod admiraris stercus, et quod te urit, \&c., cogita illam jam senescere jam rugosam cavis genis, aegrotam; tantis sordibus intus plena est, pituita, stercore; reputa quid intra nares, oculos, cerebrum gestat, quas sordes, \&c.}
\setauthornote{5744}{Subtil. 13.}
\setauthornote{5745}{Cardan, subtil. lib. 13.}
\setauthornote{5746}{Show me your company and I'll tell you who you are.}
\setauthornote{5747}{Hark, you merry maids, do not dance so, for see the he-goat is at hand, ready to pounce upon you.}
\setauthornote{5748}{Lib. de centum amoribus, earum mendas volvant animo, saepe ante oculos constituant, saepe damnent.}
\setauthornote{5749}{In deliciis.}
\setauthornote{5750}{Quum amator annulum se amicae optaret, ut ejus amplexu frui posset, \&c. O te miserum ait annulus, si meas vices obires, videres, audires, \&c. nihil non odio dignimi observares.}
\setauthornote{5751}{Laedieus.}
\setauthornote{5751.5}{Snares of the human species, torments of life, spoils of the night, bitterest cares of day, the torture of husbands, the ruin of youths.}
\setauthornote{5752}{See our English Tatius, lib. 1.}
\setauthornote{5753}{Chaucer, in Romaunt of the Rose.}
\setauthornote{5754}{Qui se facilem in amore probarit, hanc succendito. At qui succendat, ad hunc diem repertus nemo. Calcagninus.}
\setauthornote{5755}{Ariosto.}
\setauthornote{5756}{Hor.}
\setauthornote{5757}{Christoph. Fonseca.}
\setauthornote{5758}{Encom. Demonthen.}
\setauthornote{5759}{Febris hectica uxor, et non nisi morte avellenda.}
\setauthornote{5760}{Synesius, libros ego liberos genui Lipsius antiq. Lect. lib.}
\setauthornote{5761}{Avaunt, ye nymphs, maidens, ye are a deceitful race, no married life for me, \etc{}}
\setauthornote{5762}{\Plautus{} Asin. act. 1.}
\setauthornote{5763}{Senec. in Hercul.}
\setauthornote{5764}{Seneca.}
\setauthornote{5765}{Amator. Emblem.}
\setauthornote{5766}{De rebus Hibernicis l. 3.}
\setauthornote{5767}{Gemmea pocula, argentea vasa, caelata candelabra, aurea. \&c. Conchileata aulaea, buccinarum clangorem, tibiarum cantnum, et symphoniae suavitatem, majestatemque principis coronati cum vidissent sella deaurata \&c.}
\setauthornote{5768}{Eubulus in Crisil. Athenaeus dypnosophist, l. 13. c. 3.}
\setauthornote{5769}{Translated by my brother, Ralph Burton.}
\setauthornote{5770}{Juvenal.}
\setauthornote{5770.5}{Who thrusts his foolish neck a second time into the halter.}
\setauthornote{5771}{Haec in speciem dicta cave ut credas.}
\setauthornote{5772}{Bachelors always are the bravest men. Bacon. Seek eternity in memory, not in posterity, like Epaminondas, that instead of children, left two great victories behind him, which he called his two daughters.}
\setauthornote{5773}{Ecclus. \rn{xxviii}. 1.}
\setauthornote{5774}{Euripides Andromach.}
\setauthornote{5775}{Aelius Verus imperator. Spar. vit. ejus.}
\setauthornote{5776}{Hor.}
\setauthornote{5777}{Quod licet, ingratum est.}
\setauthornote{5778}{For better for worse, for richer for poorer, in sickness and in health, \&c.}
\setauthornote{5779}{Ter. act. 1 Sc. 2. Eunuch.}
\setauthornote{5780}{Lucian. tom. 4. neque cum una aliqua rem habere contentus forem.}
\setauthornote{5781}{Juvenal.}
\setauthornote{5782}{Lib. 28.}
\setauthornote{5783}{Camerar. 82. cent. 3.}
\setauthornote{5784}{Simonides.}
\setauthornote{5785}{Children make misfortunes more bitter. Bacon.}
\setauthornote{5786}{She will sink your whole establishment by her fecundity.}
\setauthornote{5787}{Heinsius. Epist. Primiero. Nihil miserius quam procreare liberos ad quos nihil ex haereditate tua pervenire videas praeter famem et sitim.}
\setauthornote{5788}{\Chrysostom{} Fonseca.}
\setauthornote{5789}{Liberi sibi carcinomata.}
\setauthornote{5790}{Melius fuerat eos sine liberis discessisse.}
\setauthornote{5791}{Lemnius, cap. 6. lib. 1. Si morosa, si non in omnibus obsequaris, omnia impacata in aedibus, omnia sursum misceri videas, multae tempestates, \&c. Lib. 2. numer. 101. sil. nup.}
\setauthornote{5792}{Juvenal. I would rather have a Venusinian wench than thee, Cornelia, mother of the Gracchi, \&c.}
\setauthornote{5793}{Tom. 4. Amores, omnem mariti opulentiam profundet, totam Arabiam capillis redolens.}
\setauthornote{5794}{Idem, et quis sanae mentis sustinere queat, \&c.}
\setauthornote{5795}{Subegit ancillas quod uxor ejus deformior esset.}
\setauthornote{5796}{Perhaps she will not suit you.}
\setauthornote{5797}{Sil. nup. l. 2. num. 25. Dives inducit tempestatem, pauper curam; ducens viduam se inducit in laqueum.}
\setauthornote{5798}{Sic quisque dicit, alteram ducit tamen Who can endure a virago for a wife?}
\setauthornote{5799}{Si dotata erit, imperiosa, continuoque viro inequitare conabitur. Petrarch.}
\setauthornote{5800}{If a woman nourish her husband, she is angry and impudent, and full of reproach. Eccles. \rn{xxv}. 22. \textlatin{Scilicet uxori nubere nolo meae}.}
\setauthornote{5801}{\Plautus{} Mil. Glor. act. 3. sc. 1.}
\setauthornote{5801.5}{To be a father is very pleasant, but to be a freeman still more so}
\setauthornote{5802}{Stobaeus, fer. 66. Alex. ab Alexand. lib. 4. cap. 8.}
\setauthornote{5803}{They shall attend the lamb in heaven, because they were not defiled with women, Apoc 14.}
\setauthornote{5804}{Nuptiae repleat terram, virginitas Paradisum. Hier.}
\setauthornote{5805}{Daphne in laurum semper virentem, immortalem docet gloriam paratam virginibus pudicitiam servantibus.}
\setauthornote{5806}{Catul. car. nuptiali.}
%\setauthornote{5806.5}{As the flower that grows in the secret inclosure of the garden, unknown to the flocks, impressed by the ploughshare, which also the breezes refresh, the heat strengthens, the rain makes grow: so is a virgin whilst untouched, whilst dear to her relatives, but when once she forfeits her chastity, \&c.}
\setauthornote{5807}{Diet. salut. c. 22. pulcherrimum sertum infiniti precii, gemma, et pictura speciosa.}
\setauthornote{5808}{Mart.}
\setauthornote{5809}{Lib. 24. qua obsequiorum diversitate colantur homines sine liberis.}
\setauthornote{5810}{Hunc alii ad coenam invitant, princeps huic famulatur, oratores gratis patrocinantur. Lib. de amore Prolis.}
\setauthornote{5811}{Annal. 11.}
%\setauthornote{5811.5}{If you wish to be master of your house, let no little ones play in your halls, nor any little daughter yet more dear, a barren wife makes a pleasant and affectionate companion.}
\setauthornote{5812}{60 de benefic. 38.}
\setauthornote{5813}{E Graeco.}
\setauthornote{5814}{Ter. Adelph.}
\setauthornote{5814.5}{I have married a wife; what misery it has entailed upon me! sons were born and other cares followed.}
\setauthornote{5815}{Itineraria in psalmo instructione ad lectorem.}
\setauthornote{5816}{Bruson, lib. 7. 22. cap. Si uxor deesset, nihil mihi ad summam felicitatem defuisset.}
\setauthornote{5817}{Extinguitur virilitas ex incantamentorum maleficiis; neque enim fabula est, nonnulli reperti sunt, qui ex veneficiis amore privati sunt, ut ex multis historiis patet.}
\setauthornote{5818}{Curat omnes morbos, phthises, hydropes et oculorum morbos, et febre quartana laborantes et amore captos, miris artibus eos demulcet.}
\setauthornote{5819}{The moral is, vehement fear expels love.}
\setauthornote{5820}{Catullus.}
\setauthornote{5821}{Quum Junonem deperiret Jupiter impotenter, ibi solitus lavare, \&c.}
\setauthornote{5822}{Menander.}
\setauthornote{5822.5}{Stricken by the gad-fly of love, rushed headlong from the summit.}
\setauthornote{5823}{\Ovid{}. ep. 21.}
\setauthornote{5824}{Apud antiquos amor Lethes olim fuit, is ardentes faeces in profluentum inclinabat; hujus statua Veneris Eleusinae templo visebatur, quo amantes confluebant, qui amicae memoriam deponere volebant.}
\setauthornote{5825}{Lib. 10. Vota ei nuncupant amatores, multis de causis, sed imprimis viduae mulieres, ut sibi alteras a dea nuptias exposcant.}
\setauthornote{5826}{Rodiginus, ant. lect. lib. 16. cap. 25. calls it Selenus, Omni amore liberat.}
\setauthornote{5827}{Seneca. The rise and remedy of love the same.}
\setauthornote{5828}{Cupido crucifixus: Lepidum poema.}
\setauthornote{5829}{Cap. 19. de morb. cerebri.}
\setauthornote{5830}{Patiens potiatur re amata, si fieri possit, optima cura, cap. 16. in 9 Rhasis.}
\setauthornote{5831}{Si nihil aliud, nuptiae et copulatio cum ea.}
\setauthornote{5832}{Petronius Catal.}
\setauthornote{5833}{Cap. de Ilishi. Non invenitur cura, nisi regimen connexionis inter eos, secundum modum promissionis, et legis, et sic vidimus ad carnem restitutum, qui jam venerat ad arofactionem; evanuit cura postquam sensit, \&c.}
\setauthornote{5834}{Fama est melancholicum quendam ex amore insanabiliter se habentem, ubi puellae se conjunxisset, restitutum, \&c.}
\setauthornote{5835}{Jovian. Pontanus, Basi. lib. 1.}
\setauthornote{5836}{Speede's hist. e M.S. Ber. Andreae.}
\setauthornote{5837}{Lucretia in Ocelestina, act. 19. Barthio interpret.}
\setauthornote{5838}{Virg. 4 Aen.}
\setauthornote{5838.5}{How shall I begin?}
\setauthornote{5839}{E Graecho Moschi.}
\setauthornote{5840}{\Ovid{}. Met. 1.}
\setauthornote{5840.5}{The efficacious one is golden.}
\setauthornote{5841}{Pausanias Achaicis, lib. 7. Perdite amabat Callyrhoen virginem, et quanto erat Choresi amor vehememior erat, tanto erat puellae animus ab ejus amore alienior.}
\setauthornote{5842}{Virg. 6 Aen.}
\setauthornote{5843}{Erasmus Egl. Galatea.}
\setauthornote{5844}{Having no compassion for my tears, she avoids my prayers, and is inflexible to my plaints}
\setauthornote{5845}{Angerianus Erotopaegnion.}
\setauthornote{5846}{Virg.}
\setauthornote{5847}{Laecheus.}
\setauthornote{5848}{\Ovid{}. Met. 1.}
\setauthornote{5849}{Erot. lib. 2.}
\setauthornote{5850}{T. H.}
\setauthornote{5850.5}{To captivate the men, but despise them when captive.}
\setauthornote{5851}{Virg. 4 Aen.}
\setauthornote{5852}{Metamor. 3.}
\setauthornote{5853}{Fracastorius Dial. de anim.}
\setauthornote{5854}{Dial. Am.}
\setauthornote{5855}{Ausonius.}
\setauthornote{5856}{\Ovid{}. Met.}
\setauthornote{5857}{Hom. 5. in 1. epist. Thess. cap. 4, vers. 1.}
\setauthornote{5858}{Ter.}
\setauthornote{5859}{Ter. Heaut. Scen. ult.}
%\setauthornote{5859.5}{He will marry the daughter of rich parents, a red-haired, blear-eyed, big-mouthed, crooked-nosed wench.}
\setauthornote{5860}{Plebeius et nobilis ambiebant puellam, puellae certamen in partes venit, \&c.}
\setauthornote{5861}{\Apuleius apol.}
\setauthornote{5862}{Gen. \rn{xxvi.}}
\setauthornote{5863}{Non peccat venialiter qui mulierem ducit ob pulchritudinem.}
\setauthornote{5864}{Lib. 6. de leg. Ex usu reipub. est ut in nuptiis juvenes neque pauperum affinitatem fugiant, neque divitum sectentur.}
\setauthornote{5865}{Philost. ep. Quoniam pauper sum, idcirco contemptior et abjectior tibi videar? Amor ipse nundus est, gratiae et astra; Hercules pelle leonina indutus.}
\setauthornote{5866}{Juvenal.}
\setauthornote{5867}{Lib. 2. ep. 7.}
\setauthornote{5868}{Ejulans inquit, non mentem una addixit mihi fortuna servitute.}
\setauthornote{5869}{De repub. c. de period, rerumpub.}
\setauthornote{5870}{Com. in car. Chron.}
\setauthornote{5871}{Plin. in pan.}
\setauthornote{5872}{Declam. 306.}
\setauthornote{5873}{Puellis imprimis nulla danda occasio lapsus. Lemn. lib. l. 54. de vit instit.}
\setauthornote{5874}{See more part 1. s. mem. 2. subs. 4.}
\setauthornote{5875}{Filia excedens annum 25. potest inscio patre nubere, licet indignus sit maritus, et eum cogere ad congrue dotandum.}
\setauthornote{5876}{Ne appetentiae procacioris reputetur auctor.}
\setauthornote{5877}{Expetitia enim magis debet vider a viro quam ipsa virum expetisse.}
\setauthornote{5878}{Mulier apud nos 24. annorum vetula est et projectitia.}
\setauthornote{5879}{Comoed. Lycistrat. And. Divo Interpr.}
\setauthornote{5880}{Ausonius edy. 14.}
\setauthornote{5881}{Idem.}
\setauthornote{5882}{Catullus.}
\setauthornote{5883}{Translated by M. B. Johnson.}
\setauthornote{5884}{Horn. 5. in 1. Thes. cap. 4. 1.}
\setauthornote{5885}{\Plautus{}.}
\setauthornote{5886}{\Ovid{}.}
\setauthornote{5887}{Epist. 12. l. 2. Eligit conjugem pauperem, indotatatam et subito deamavit, et commiseratione ejus inopiae.}
\setauthornote{5888}{Virg. Aen.}
\setauthornote{5889}{Fabius pictor: amor ipse conjunxit populos, \&c.}
\setauthornote{5890}{Lipsius polit. Sebast. Mayer. Select. Sect. 1. cap. 13.}
\setauthornote{5891}{Mayerus select. sect. 1. c. 14. et Aelian. l. 13. c. 33. cum famulae lavantis vestes incuriosus custodirent, \&c. mandavit per universam Aegyptum ut foemina quaereretur, cujus is calceus esset eamque sic inventam. in matrimonium accepit.}
\setauthornote{5892}{Pausnnias lib. 3. de Laconicis. Dimisit que nunciarunt, \&c. optionem puellis dedit, ut earum quaelibet eum sibi virum deligeret, cujus maxime esset forma complacita.}
\setauthornote{5893}{Illius conjugium abominabitur.}
\setauthornote{5894}{Socera quinque circiter annos natu minor.}
\setauthornote{5895}{Vit. Caleat. secundi.}
\setauthornote{5896}{\\Apuleius in Catel. nobis cupido velle dat, posse abnegat.}
\setauthornote{5897}{Anacreon. 56.}
\setauthornote{5898}{Continentiae donum ex fide postulet quia certum sit eum vocari ad coelibatum cui domis, \&c.}
\setauthornote{5899}{Act. \rn{xvi.} 7.}
\setauthornote{5900}{Rom. \rn{i.} 13.}
\setauthornote{5901}{Praefix. gen. Leovitii.}
%\setauthornote{5902}{The stars in the skies preside over our persons, for they are made of humble matter. They cannot bind a rational mind, for that is under the control of God only.}
\setauthornote{5903}{Idem Wolfius dial.}
\setauthornote{5904}{That is, make the best of it, and take his lot as it falls.}
\setauthornote{5905}{\Ovid{}. 1.}
\setauthornote{5905.5}{Met Their beauty is inconsistent with their vows.}
\setauthornote{5906}{Mercurialis de Priapismo.}
\setauthornote{5907}{Memorabile quod Ulricus epistola refert Gregorium quum ex piscina quadam allata plus quam sex mille infantum capita vidisset, ingemuisse et decretum de coelibatu tantam caedis causam confesses condigno illud poenitentiae fructu purgasse. Kemnisius ex concil. Trident, part. 3. de coelibatu sacerdotum.}
\setauthornote{5908}{Si nubat, quam si domi concubinam alat.}
\setauthornote{5909}{Alphonsus Cicaonius lib. de gest. pontificum.}
\setauthornote{5910}{Cum medici suaderent ut aut nuberet aut coitu uteretur, sic mortem vitari posse mortem potius intrepidus expectavit, \&c.}
\setauthornote{5911}{Epist. 30.}
\setauthornote{5912}{Vide vitam ejus edit. 1623. by D. T. James.}
\setauthornote{5913}{Lidgate, in Chaucer's Flower of Curtesie.}
\setauthornote{5914}{'Tis not multitude but idleness which causeth beggary.}
\setauthornote{5915}{Or to set them awork, and bring them up in some honest trades.}
\setauthornote{5916}{Dion. Cassius, lib. 56.}
\setauthornote{5917}{Sardus Buxtorphius.}
\setauthornote{5918}{Claude Albaville in his hist. of the Frenchmen to the Isle of Maragnan. An. 1614.}
\setauthornote{5919}{Rara quidem dea tu es O chastitas in his terris, nec facile perfecta, rarius perpetua, cogi nonnunquam potest, ob naturae defectum, vel si disciplina pervaserit, censura compresserit.}
\setauthornote{5920}{Peregrin. Hierosol.}
\setauthornote{5921}{Plutarch, vita ejus, adolescentiae medio constitutus.}
\setauthornote{5922}{Ancilias duas egregia forma et aetatis flore.}
\setauthornote{5923}{Alex. ab. Alex. l. 4. c. 8.}
\setauthornote{5924}{Tres filii patrem ab excubiis, quinque ab omnibus officiis liberabanto.}
\setauthornote{5925}{Praecepto primo, cogatur nubere aut mulctetur et pecunia templo Junonis dedicetur et publica fiat.}
\setauthornote{5926}{Consol. 3. pros. 7.}
\setauthornote{5927}{Nic. Hill. Epic. philos.}
\setauthornote{5928}{Qui se capistro matrimonii alligari non patiuntur, Lemn, lib. 4. 13. de occult. nat. Abhorrent multi a matrimonio, ne morosam, querulam, acerbam, amaram uxorem perferre cogantur.}
\setauthornote{5929}{Senec. Hippol.}
\setauthornote{5930}{Caelebs enim vixerat nec ad uxorem ducendam unquam induci potuit.}
\setauthornote{5931}{Senec. Hip. There is nothing better, nothing preferable to a single life.}
\setauthornote{5932}{Hor.}
\setauthornote{5933}{Aeneas Sylvius de dictis Sigismundi. Hensius. Primiero.}
\setauthornote{5934}{Habeo uxorem ex animi sententia Camillam Paleotti Jurisconsulti filiam.}
\setauthornote{5935}{Legentibus et meditantibus candelas et candelabrum tenuerunt.}
\setauthornote{5936}{Hor. Neither despise agreeable love, nor mirthful pleasure.}
\setauthornote{5937}{\Ovid{}.}
\setauthornote{5938}{Aphranius.}
\setauthornote{5938.5}{He who chooses a wife, takes a brother and a sister.}
\setauthornote{5939}{Locheus.}
\setauthornote{5939.5}{The delight of mankind, the solace of life, the blandishments of night, delicious cares of day, the wishes of older men, the hopes of young.}
\setauthornote{5940}{Bacon's Essays.}
\setauthornote{5941}{Euripides.}
\setauthornote{5942}{How harmoniously do a loving wife and constant husband lead their lives.}
\setauthornote{5943}{Cum juxta mare agrum coleret: Omnis enim miseriae immemorem, conjugalis amor eum fecerat. Non sine ingenti admiratione, tanta hominis charitate motus rex liberos esse jussit, \&c.}
\setauthornote{5944}{Qui vult vitare molestias vitet mundum.}
\setauthornote{5945}{\textgreek{Τίδε βίος τίθε τερπνὸν ἄτερ χρυσῆς ἀφροδίτης}. \textlatin{Quid vita est quaeso quidve est sine Cypride dulce? Mimner.}}
\setauthornote{5946}{Erasmus.}
\setauthornote{5947}{E Stobeo.}
\setauthornote{5948}{Menander.}
\setauthornote{5949}{Seneca Hyp. lib. 3. num. 1.}
\setauthornote{5950}{Hist. lib. 4.}
\setauthornote{5951}{Palingenius. He lives contemptibly by whom no other lives.}
\setauthornote{5952}{Bruson. lib. 7. cap. 23.}
\setauthornote{5953}{Noli societatem habere, \&c.}
\setauthornote{5954}{Lib. 1. cap. 6. Si, inquit, Quirites, sine uxore esse possemus, omnes careremus; Sed quoniam sic est, saluti potius publicae quam voluptati consulendum.}
\setauthornote{5955}{Beatum foret si liberos auro et argento mercari, \&c.}
\setauthornote{5956}{Seneca. Hyp.}
\setauthornote{5957}{Gen. \rn{ii.} Adjutorium simile, \&c.}
\setauthornote{5958}{\Ovid{}. Find her to whom you may say, 'thou art my only pleasure.'}
\setauthornote{5959}{Euripides. Unhappy the man who has met a bad wife, happy who found a good one.}
\setauthornote{5960}{E Graeco Valerius, lib. 7. cap. 7. To marry, and not to marry, are equally base.}
\setauthornote{5961}{Pervigilium Veneris e vetere poeta.}
\setauthornote{5962}{Donaus non potest consistere sine uxore. Nevisanus lib. 2. num. 18.}
\setauthornote{5963}{Nemo in severissima Stoicorum familia qui non barbam quoque et supercilium amplexibus uxores submiserit, aut in ista parte a reliquis dissenserit. Hensius Primiero.}
\setauthornote{5964}{Quid libentius homo masculus videre debet quam bellam uxorem?}
\setauthornote{5965}{Chaucer.}
\setauthornote{5966}{Conclusio Theod. Podro. mi. 9. l. Amor.}
\setauthornote{5967}{\Ovid{}.}
\setauthornote{5968}{Epist. 4. l. 2. Jucundiores multo et suaviores longe post molestas turbas amantium nuptiae.}
\setauthornote{5969}{Olim meminisse juvabit.}
\setauthornote{5970}{Quid expectatis, intus fiunt nuptiae, the music, guests, and all the good cheer is within.}
\setauthornote{5971}{The conclusion of Chaucer's poem of Troilus and Creseid.}
\setauthornote{5972}{Catullus.}
\setauthornote{5973}{Catullus. J. Secundus Sylvar. lib. Jam Virgo thalamum subibit unde ne virgo redeat, marite cura.}
\setauthornote{5974}{Ecclus. \rn{xxxix.} 14.}
\setauthornote{5975}{Galeni Epithal.}
\setauthornote{5976}{O noctem quater et quater beatam.}
\setauthornote{5977}{Theocritus idyl. 18.}
\setauthornote{5978}{Erasm. Epithal. P. Aegidij. Nec saltent modo sed duo charissima pectora indissolubili mutuae benevolentiae nodo corpulent, ut nihil unquam eos incedere possit irae vel taedii. Illa perpetuo nihil audiat nisi, mea lux: ille vicissim nihil nisi anime mi: atque huic jucunditati ne senectus detrahat, imo potius aliquid adaugeat.}
\setauthornote{5979}{Happy both, if my verses have any charms, nor shall time ever detract from the memorable example of your lives.}
\setauthornote{5980}{Kornmannus de linea amoris.}
\setauthornote{5981}{Finis 3 book of Troilus and Creseid.}
