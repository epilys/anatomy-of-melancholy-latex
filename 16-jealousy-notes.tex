\setmarginnote{5982}{In his Oration of Jealousy, put out by Fr. Sansavin.}
\setmarginnote{5983}{Benedetto Varchi.}
\setmarginnote{5984}{Exercitat. 317. Cum metuimus ne amatae rei exturbimur possessione.}
\setmarginnote{5985}{Zelus de forma est invidentiae species ne quis forma quam amamus fruatur.}
\setmarginnote{5986}{3 de Anima.}
\setmarginnote{5987}{Has not every one of the slaves that went to meet him returned this night from the supper?}
\setmarginnote{5988}{R. de Anima. Tangimur zelotypia de pupillis, liberis charisque curae nostrae concreditis, non de forma, sed ne male sit iis, aut ne nobis sibique parent ignominiam.}
\setmarginnote{5989}{Plutarch.}
\setmarginnote{5990}{Senec. in Herc. fur.}
\setmarginnote{5991}{Exod. xx.}
\setmarginnote{5992}{Lucan.}
\setmarginnote{5993}{Danaeus Aphoris. polit. semper metuunt ne eorum auctoritas minuatur.}
\setmarginnote{5994}{Belli Neapol. lib. 5.}
\setmarginnote{5995}{Dici non potest quam tenues et infirmas causas habent moeroris et suspicionis, et hic est morbus occultus, qui in familiis principum regnat.}
\setmarginnote{5996}{Omnes aemulos interfect. Lamprid.}
\setmarginnote{5997}{Constant. agricult. lib. 10. c. 5. Cyparissae Eteoclis filiae, saltantes ad emulationem dearum in puteum demolitae sunt, sed terra miserata, cupressos inde produxit.}
\setmarginnote{5998}{Ovid. Met.}
\setmarginnote{5999}{Seneca.}
\setmarginnote{6000}{Quis autem carifex addictum supplicio crudelius afficiat, quam metus? Metus inquam mortis, infamiae cruciatus, sunt ille utrices furiae quae tyrannos exagitant, \&c. Multo acerbius sauciant et pungunt, quam crudeles domini servos vinctos fustibus ac tormentis exulcerare possunt.}
\setmarginnote{6001}{Lonicerus, To. 1. Turc. hist. c. 24.}
\setmarginnote{6002}{Jovius vita ejus.}
\setmarginnote{6003}{Knowles. Busbequius. Sand. fol. 52.}
\setmarginnote{6004}{Nicephorus, lib. 11. c. 45. Socrates, lib. 7. cap. 35. Neque Valens alicui pepercit qui Theo cognomine vocaretur.}
\setmarginnote{6005}{Alexand. Gaguin. Muscov. hist. descrip. c. 5.}
\setmarginnote{6006}{D. Fletcher, timet omnes ne insidiae essent, Herodot. l. 7. Maximinus invisum se sentiens, quod ex infimo loco in tantam fortunam venisset moribus ac genere barbarus, metuens ne natalium obscuritas objiceretur, omnes Alexandri praedecessoris ministros ex aula ejecit, pluribus interfectis quod moesti essent ad mortem Alexandri, insidias inde metuens.}
\setmarginnote{6007}{Lib. 8. tanquam ferae solitudine vivebant, terrentes alios, timentes.}
\setmarginnote{6008}{Serres, fol. 56.}
\setmarginnote{6009}{Neap. belli, lib. 5 nulli prorsus homini fidebat, omnes insidiari sibi putabat.}
\setmarginnote{6010}{Camden's Remains.}
\setmarginnote{6011}{Mat. Paris.}
\setmarginnote{6012}{R. T. notis in blason jealousie.}
\setmarginnote{6013}{Daniel in his Panegyric to the king.}
\setmarginnote{6014}{3. de anima, cap. de zel. Animalia quaedam zelotypia tanguntur, ut olores, columbae, galli, tauri, \&c. ob metum communionis.}
\setmarginnote{6015}{Seneca.}
\setmarginnote{6016}{Lib. 11. Cynoget.}
\setmarginnote{6017}{Chaucer, in his Assembly of Fowls.}
\setmarginnote{6018}{Alderovand.}
\setmarginnote{6019}{Lib. 12.}
\setmarginnote{6020}{Sibi timens circa res venereas, solitudines amat quo solus sola foemina fruatur.}
\setmarginnote{6021}{Crocodili zelotypi et uxorum amantissimi, \&c.}
\setmarginnote{6022}{Qui dividit agrum communem; inde deducitur ad amantes.}
\setmarginnote{6023}{Erasmus chil. 1. cent. 9. adag. 99.}
\setmarginnote{6024}{Ter. Eun. Act. 1. sc. 1. Munus nostrum ornato verbis, et istum aemulum, quoad poteris, ab ea pellito.}
\setmarginnote{6025}{Pinus puella quondam fuit, \&c.}
\setmarginnote{6026}{Mars zelotypus Adonidem interfecit.}
\setmarginnote{6027}{R. T.}
\setmarginnote{6028}{1 Sam. i. 6.}
\setmarginnote{6029}{Blazon of Jealousy.}
\setmarginnote{6030}{Mulierum conditio misera; nullam honestam credunt nisi domo conclusa vivat.}
\setmarginnote{6031}{Fines Morison.}
\setmarginnote{6032}{Nomen zelotypiae apud istos locum non habet, lib. 3. c. 8.}
\setmarginnote{6033}{Fines Moris. part. 3. cap. 2.}
\setmarginnote{6034}{Busbequius. Sands.}
\setmarginnote{6035}{Prae amore et zelotypia saepius insaniunt.}
\setmarginnote{6036}{Australes ne sacra quidem publica fieri patiuntur, nisi uterque sexus pariete medio dividatur: et quum in Angliam inquit, legationis causa profectus essem, audivi Mendozam legatum Hispaniarum dicentem turpe esse viros et foeminas in, \&c.}
\setmarginnote{6037}{Idea: mulieres praeterquam quod sunt infidae, suspicaces, inconstantes, insidiosae, simulatrices, superstitiosae, et si potentes, intolerabiles, amore zelotypae supra modum. Ovid. 2. de art.}
\setmarginnote{6038}{Bartello.}
\setmarginnote{6039}{R. T.}
\setmarginnote{6040}{Lib. 2. num. 8. mulier otiosa facile praesumitur luxuriosa, et saepe zelotypa.}
\setmarginnote{6041}{And now she requires other youths and other loves, calls me the imbecile and decrepit old man.}
\setmarginnote{6042}{Lib 2. num. 4.}
\setmarginnote{6043}{Quum omnibus infideles foeminae, senibus infidelissimae.}
\setmarginnote{6044}{Mimnermus.}
\setmarginnote{6045}{Vix aliqua non impudica, et quam non suspectam merito quis habeat.}
\setmarginnote{6046}{Lib. 5. de aur. asino. At ego misera patre meo seniorem maritum nacta sum, cum cucurbita calviorom et quovis puero pumiliorem, cunctam domum seris et catenis obditam custodientem.}
\setmarginnote{6047}{Chaloner.}
\setmarginnote{6048}{Lib. 4. n. 80.}
\setmarginnote{6049}{Ovid 2. de art. amandi.}
\setmarginnote{6050}{Every Man out of his Humour.}
\setmarginnote{6051}{Calcagninus Apol. Tiberini ab uxorum partu earum vices subeunt, ut aves per vices incubant, \&c.}
\setmarginnote{6052}{Exiturus fascia uxoris pectus alligabat, nec momento praeesentia ejus carere poterat, potumque non hauriebat nisi praegustatum labris ejus.}
\setmarginnote{6053}{Chaloner.}
\setmarginnote{6054}{Panegyr. Trajano.}
\setmarginnote{6055}{Ter. Adelph. act. 1. sce. 1.}
\setmarginnote{6056}{Fab. Calvo. Ravennate interprete.}
\setmarginnote{6057}{Dum rediero domum meam habitabis, et licet cum parentibus habitet, hac mea peregrinatione; eam tamen et ejus mores observabis uti absentia viri sui probe degat, nec alios viros cogitet aut quaerat.}
\setmarginnote{6058}{Foemina semper custode eget qui se pudicam contineat; suapte enim natura nequitias insitas habet, quas nisi indies comprimat, ut arbores stolones emittunt, \&c.}
\setmarginnote{6059}{Heinsius.}
\setmarginnote{6060}{Uxor cujusdam nobilis quum debitum maritale sacro passionis hebdomada non obtineret, alterum adiit.}
\setmarginnote{6061}{Ne tribus prioribus noctibus rem haberet cum ea. ut esset in pecoribus fortunatus, ab uxore morae impatiente, \&c.}
\setmarginnote{6062}{Totam noctem bene et pudice nemini molestus dormiendo transegit; mane autem quum nullius conscius facinoris sibi esset, et inertiae puderet, audisse se dicebat eum dolore calculi solere eam conflictari. Duo praecepta juris una nocte expressit, neminem laeserat et honeste vixerat, sed an suum cuique reddidisset, quaeri poterat. Mutius opinor et Trebatius hoc negassent. lib. 1.}
\setmarginnote{6063}{Alterius loci emendationem serio optabat, quem corruptum esse ille non invenit.}
\setmarginnote{6064}{Such another tale is in Neander de Jocoseriis, his first tale.}
\setmarginnote{6065}{Lib. 2. Ep. 3. Si pergit alienis negotiis operam dare sui negligens, erit alius mihi orator qui rem meam agat.}
\setmarginnote{6066}{Ovid. rara est concordia formae atque pudicitiae.}
\setmarginnote{6067}{Epist.}
\setmarginnote{6068}{Quod strideret ejus calceamentum.}
\setmarginnote{6069}{Hor. epist. 15. Often has the serpent lain hid beneath the coloured grass, under a beauliful aspect, and often has the evil inclination affected a sale without the husband's privity.}
\setmarginnote{6070}{De re uxoria, lib. 1. cap. 5.}
\setmarginnote{6071}{Cum steriles sunt, ex mutatione viri se putant concipere.}
\setmarginnote{6072}{Tibullus, eleg. 6.}
\setmarginnote{6073}{Wither's Sat.}
\setmarginnote{6074}{3 de Anima. Crescit ac decrescit zelotypia cum personis, locis, temporibus, negotiis.}
\setmarginnote{6075}{Marullus.}
\setmarginnote{6076}{Tibullus Epig.}
\setmarginnote{6077}{Prov. ix. 17.}
\setmarginnote{6078}{Propert. eleg. 2.}
\setmarginnote{6079}{Ovid. lib. 9. Met. Pausanias Strabo, quum crevit imbribus hyemalibus. Deianiram suscipit, Herculem nando sequi jubet.}
\setmarginnote{6080}{Lucian, tom. 4.}
\setmarginnote{6081}{Plutarch.}
\setmarginnote{6082}{Cap. v. 8.}
\setmarginnote{6083}{Seneca.}
\setmarginnote{6084}{Lib. 2. cap. 23.}
\setmarginnote{6085}{Petronius Catal.}
\setmarginnote{6086}{Sueton.}
\setmarginnote{6087}{Pontus Heuter, vita ejus.}
\setmarginnote{6088}{Lib. 8. Flor. hist. Dux omnium optimus et sapientissimus, sed in re venerea prodigiosus.}
\setmarginnote{6089}{Vita Castruccii. Idem uxores maritis abalienavit.}
\setmarginnote{6090}{Sesellius, lib. 2. de Repub. Gallorum. Ita nunc apud infimos obtinuit hoc vitium, ut nullius fere pretii sit, et ignavus miles qui non in scortatione maxime excellat, et adulterio.}
\setmarginnote{6091}{Virg. Aen. 4. What now must have been Dido's sensations when she witnessed these doings?}
\setmarginnote{6092}{Epig. 9. lib. 4.}
\setmarginnote{6093}{Virg. 4. Aen.}
\setmarginnote{6094}{Secundus syl.}
\setmarginnote{6095}{And belches out the smell of onions and garlic.}
\setmarginnote{6096}{Aeneas Sylvius.}
\setmarginnote{6097}{Neither a god honoured him with his table, nor a goddess with her bed.}
\setmarginnote{6098}{Virg. 4. Aen. Such beauty shines in his graceful features.}
\setmarginnote{6099}{S. Graeco Simonides.}
\setmarginnote{6100}{Cont. 2. ca. 38. Oper. subcis. mulieris liberius et familiarius communicantis cum omnibus licentia et immodestia, sinistri sermonis et suspicionis materiam viro praebet.}
\setmarginnote{6101}{Voces liberae, oculorum colloquia, contractiones parum verecundae, motus immodici, \&c. Heinsius.}
\setmarginnote{6102}{Challoner.}
\setmarginnote{6103}{What is here said, is not prejudicial to honest women.}
\setmarginnote{6104}{Lib. 28, sc 13.}
\setmarginnote{6105}{Dial. amor. Pendet fallax et blanda circa oscula mariti, quem in cruce, si fieri posset, deosculari velit: illius vitam chariorem esse sua jurejurando affirmat: quem certe non redimeret anima catelli si posset.}
\setmarginnote{6106}{Adeunt templum ut rem divinam audiant, ut ipsae simulant, sed vel ut monachum fratrem, vel adulterum lingua, oculis, ad libidinem provoceat.}
\setmarginnote{6107}{Lib. 4. num. 81. Ipse sibi persuadent, quod adulterium cum principe vel cum praesule, non est pudor nec peccatum.}
\setmarginnote{6108}{Deum rogat, non pro salute mariti, filii, cognati vota suscipit, sed pro reditu moechi si abest, pro valetudine lenonis si aegrotet.}
\setmarginnote{6109}{Tibullus.}
\setmarginnote{6110}{Gortardus Arthus descrip. Indiae Orient. Linchoften.}
\setmarginnote{6111}{Garcias ab Horto, hist. lib. 2. cap. 24. Daturam herbam vocat et describit, tam proclives sunt ad venerem mulieres ut viros inebrient per 24 horas, liquore quodam, ut nihil videant, recordentur at dormiant, et post lotionem pedum, ad se restituunt, \&c.}
\setmarginnote{6112}{Ariosto, lib. 28. st. 75.}
\setmarginnote{6113}{Lipsius polit.}
\setmarginnote{6114}{Seneca, lib. 2. controv. 8.}
\setmarginnote{6115}{Bodicher. Sat.}
\setmarginnote{6116}{Sitting close to her, and shaking her hand lovingly.}
\setmarginnote{6117}{Tibullus.}
\setmarginnote{6118}{After wine the mistress is often unable to distinguish her own lover.}
\setmarginnote{6119}{Epist. 85. ad Oceanum. Ad unius horae ebrietatem nudat femora, quae per sexcentos annos sobrietate contexerat.}
\setmarginnote{6120}{Juv. Sat. 13.}
\setmarginnote{6121}{Nihil audent primo, post ab aliis confirmatae, audaces et confidentes sunt. Ubi semel verecundiae limites transierint.}
\setmarginnote{6122}{Euripides, l. 63. Love of gain induces one to break her marriage vow, a wish to have associates to keep her in countenance actuates others.}
\setmarginnote{6123}{De miser. Curialium. Aut alium cum ea invenies, aut isse alium reperies.}
\setmarginnote{6124}{Cap. 18 de Virg.}
\setmarginnote{6125}{Hom. 38. in c. 17. Gen. Etsi magnis affluunt divitiis, \&c.}
\setmarginnote{6126}{3 de Anima. Omnes voces, auras, omnes susurros captat zelotypus, et amplificat apud se cum iniquissima de singulis calumnia. Maxime suspiciosi, et ad pejora credendum proclives.}
\setmarginnote{6127}{These thunders pour down their peculiar showers.}
\setmarginnote{6128}{Propertius.}
\setmarginnote{6129}{Aeneas Silv.}
\setmarginnote{6130}{Ant. Dial.}
\setmarginnote{6131}{Rabie concepta, caesariem abrasit, puellaeque mirabiliter insultans faciem vibicibus foedavit.}
\setmarginnote{6132}{Daniel.}
\setmarginnote{6133}{Annal. lib. 12. Principis mulieris zelotypae est in alias mulieres quas suspectas habet, odium inseparabile.}
\setmarginnote{6134}{Seneca in Medea.}
\setmarginnote{6135}{Alcoran cap. Bovis, interprete Ricardo praed. c. 8. Confutationis.}
\setmarginnote{6136}{Plautus.}
\setmarginnote{6137}{Expedit. in Sinas. l. 3. c. 9.}
\setmarginnote{6138}{Decem eunuchorum millia numerantur in regia familia, qui servant uxores ejus.}
\setmarginnote{6139}{Lib. 57. ep. 81.}
\setmarginnote{6140}{Semotis a viris servant in interioribus, ab eorum conspectu immanes.}
\setmarginnote{6141}{Lib. 1. fol. 7.}
\setmarginnote{6142}{Diruptiones hymenis flunt a propriis digitis vel ab aliis instrumentis.}
\setmarginnote{6143}{Idem Rhasis Arab. cont.}
\setmarginnote{6144}{Ita clausae pharmacis ut non possunt coitum exercere.}
\setmarginnote{6145}{Qui et pharmacum praescribit docetque.}
\setmarginnote{6146}{Epist. 6. Mercero Inter.}
\setmarginnote{6147}{Barthius. Ludus illi temeratum pudicitiae florem mentitis machinis pro integro vendere. Ego docebo te, qui mulier ante nuptias sponso te probes virginem.}
\setmarginnote{6148}{Qui mulierem violasset, virilia execabant, et mille virgas dabant.}
\setmarginnote{6149}{Dion. Halic.}
\setmarginnote{6150}{Viridi gaudens Feronia luco. Virg.}
\setmarginnote{6151}{Ismene was so tried by Dian's well, in which maids did swim, unchaste were drowned. Eustathius, lib. 8.}
\setmarginnote{6152}{Contra mendac. an confess. 21 cap.}
\setmarginnote{6153}{Phaerus Aegipti rex captus oculis per decennium, oraculum consuluit de uxoris pudicitia.}
\setmarginnote{6154}{Caesar. lib. 6. bello Gall. vitae necisque in uxores habuerunt potestatem.}
\setmarginnote{6155}{Animi dolores et zelotypia si diutius perserverent, dementes reddunt. Acak. comment. in par. art. Galeni.}
\setmarginnote{6156}{Ariosto, lib. 31. staff. 6.}
\setmarginnote{6157}{3 de anima, c. 3. de zelotyp. transit in rabiem et odium, et sibi et aliis violentas saepe manus injiciunt.}
\setmarginnote{6158}{Higinus, cap. 189. Ovid, \&c.}
\setmarginnote{6159}{Phaerus Aegypti rex de caecitate oraculum consulens, visum ei rediturum accepit, si oculos abluisset lotio mulieris quae aliorum virorum esset expers; uxoris urinam expertus nihil profecit, et aliarum frustra, eas omnes (ea excepta per quam curatus fuit) unum in locum coactas concremavit. Herod. Euterp.}
\setmarginnote{6160}{Offic. lib. 2.}
\setmarginnote{6161}{Aurelius Victor.}
\setmarginnote{6162}{Herod, lib. 9. in Calliope. Masistae uxorem excarnificat, mammillas praescindit, aesque canibus abjicit, filiae nares praescidit, labra, linguam, \&c.}
\setmarginnote{6163}{Lib. 1. Dum formae curandae intenta capillum in sole pectit, a marito per lusum leviter percussa furtirm superveniente virga, risu suborto, mi Landrice dixit, frontem vir fortis petet, \&c. Marito conspecto attonita, cum Landrico mox in ejus mortem conspirat, et statim inter venandum efficit.}
\setmarginnote{6164}{Qui Goae uxorem habens, Gotherinum principem quendam virum quod uxori suae oculos adjecisset, ingenti vulnere deformavit in facie, et tibiam abscidit, unde mutuae caedes.}
\setmarginnote{6165}{Eo quod infans natus involutus esset panniculo, credebat eum filium fratris Francisci, \&c.}
\setmarginnote{6166}{Zelotypia reginas regis mortem acceleravit paulo post, ut Martianus medicus mihi retulit. Illa autem atra bile inde exagitata in latebras se subducens prae aegritudine animi reliquum tempus consumpsit.}
\setmarginnote{6167}{A zelotypia redactus ad insaniam et desperationem.}
\setmarginnote{6168}{Uxorem interemit, inde desperabundus ex alto se praecipitavit.}
\setmarginnote{6169}{Tollere nodosam nescit medicina podagram.}
\setmarginnote{6170}{Ariosto, lib. 31. staff.}
\setmarginnote{6171}{Veteres mature suadent ungues amoris esse radendos, priusquam producant se nimis.}
\setmarginnote{6172}{In Jovianum.}
\setmarginnote{6173}{Gomesius, lib. 3. de reb. gestis Ximenii.}
\setmarginnote{6174}{Urit enim praecordia aegritudo animi compressa, et in angustiis adducta mentem. subvertit, nec alio medicamine facilius erigitur, quam cordati hominis sermone.}
\setmarginnote{6175}{3 De anima.}
\setmarginnote{6176}{Lib. 3.}
\setmarginnote{6177}{Argetocoxi Caledoni Reguli uxor, Juliae Augustae cum ipsam morderet quod inhoneste versaretur, respondet, nos cum optimis viris consuetudinem habemus; vos Romanas autem occulte passim homines constuprant.}
\setmarginnote{6178}{Leges de moechis fecit, ex civibus plures in jus vocati.}
\setmarginnote{6179}{L. 3. Epig. 26.}
\setmarginnote{6180}{Asser Arthuri; parcerem libenter heroinarum laesae majestati, si non historiae veritas aurem vellicaret, Leland.}
\setmarginnote{6181}{Leland's assert. A thuri.}
\setmarginnote{6182}{Epigram.}
\setmarginnote{6183}{Cogita an sic aliis tu unquam feceris; an hoc tibi nunc fieri dignum sit? severus aliis, indulgens tibi, cur. ab uxore exigis quod nori ipse praestas? Plutar.}
\setmarginnote{6184}{Vaga libidine cum ipse quovis rapiaris, cur si vel modicum aberret ipsa, insanias?}
\setmarginnote{6185}{Ariosto, li. 28. staffe 80.}
\setmarginnote{6186}{Sylva nupt. l. 4. num. 72.}
\setmarginnote{6187}{Lemnius, lib. 4. cap. 13. de occult. nat. mir.}
\setmarginnote{6188}{Optimum bene nasci.}
\setmarginnote{6189}{Mart.}
\setmarginnote{6190}{Ovid. amor. lib. 3. eleg.}
\setmarginnote{6191}{Lib. 4. St. 72.}
\setmarginnote{6192}{Policrat. lib. 8. c. 11. De amor.}
\setmarginnote{6193}{Euriel. et Lucret. qui uxores occludunt, meo judicio minus utiliter faciunt; sunt enim eo ingenio mulieres ut id potissimum cupiant, quod maxime denegatur: si liberas habent habenas, minus delinquunt; frustra seram adhibes, si non sit sponte casta.}
\setmarginnote{6194}{Quando cognoscunt maritos hoc advertere.}
\setmarginnote{6195}{Ausonius.}
\setmarginnote{6196}{Opes suas, mundum suum, thesaurum suum, \&c.}
\setmarginnote{6197}{Virg. Aen.}
\setmarginnote{6198}{Daniel.}
\setmarginnote{6199}{1 de serm. d. in monte ros. 16.}
\setmarginnote{6200}{O quam formosus lacertus hic quidam inquit ad aequales conversus; at illa, publicus, inquit, non est.}
\setmarginnote{6201}{Bilia Dinutum virum senem habuit et spiritum foetidum habentem, quem quum quidam exprobrasset, \&c.}
\setmarginnote{6202}{Numquid tibi, Armena, Tigranes videbatur esse pulcher? et illum, inquit, aedepol, \&c. Xenoph. Cyropaed. l. 3.}
\setmarginnote{6203}{Ovid.}
\setmarginnote{6204}{Read Petrarch's Tale of Patient Grizel in Chaucer.}
\setmarginnote{6205}{Sil. nup. lib. 4. num. 80.}
\setmarginnote{6206}{Erasmus.}
\setmarginnote{6207}{Quum accepisset uxorem peperisse secundo a nuptiis mense, cunas quinas vel senas coemit, ut si forte uxor singulis bimensibus pareret.}
\setmarginnote{6208}{Julius Capitol, vita ejus, quum palam Citharaedus uxorem diligeret, minime curiosus fuit.}
\setmarginnote{6209}{Disposuit armatos qui ipsum interficerent: hi protenus mandatum exequentes, \&c. Ille et rex declarator, et Stratonicem quae fratri nupserat, uxorem ducit: sed postquam audivit fratrem vivere, \&c. Attalum comiter accepit, pristinamque uxorem complexus, magno honore apud se habuit.}
\setmarginnote{6210}{See John Harrington's notes in 28. book of Ariosto.}
\setmarginnote{6211}{Amator. dial.}
\setmarginnote{6212}{Plautus scen. ult. Amphit.}
\setmarginnote{6213}{Idem.}
\setmarginnote{6214}{T. Daniel conjurat. French.}
\setmarginnote{6215}{Lib. 4. num. 80.}
\setmarginnote{6216}{R. T.}
\setmarginnote{6217}{Lib. de heres. Quum de zele culparetur, purgandi se causa permisisse fertur ut ea qui vellet uteretur; quod ejus factum in sectam turpissimam versum est, qua placet usus indifferens foeminamm.}
\setmarginnote{6218}{Sleiden, Com.}
\setmarginnote{6219}{Alcoran.}
\setmarginnote{6220}{Alcoran edit, et Bibliandro.}
\setmarginnote{6221}{De mor. gent. lib. 1. cap. 6. Nupturae regi de virginandae exhibentur.}
\setmarginnote{6222}{Lumina extinguebantur, nec persons) et aetatis habila reverentia, in quam quisque per tenebras incidit, mulierem cognoscit.}
\setmarginnote{6223}{Leander Albertus. Flagitioso ritu cuncti in aedem convenientes post impuram concionem, extinctis luminibus in Venerem ruunt.}
\setmarginnote{6224}{Lod. Vertomannus navig. lib. 6. cap. 8. et Marcus Polus lib. 1. cap. 46. Uxores viatoribus prostituunt.}
\setmarginnote{6225}{Dithmarus, Bleskenius, ut Agetas Aristoni, pulcherrimam uxorem habens prostituit.}
\setmarginnote{6226}{Herodot. in Erato. Mulieres Babyloni caecum hospite permiscentur ob argentum quod post Veneri sacrum. Bohernus, lib. 2.}
\setmarginnote{6227}{Navigat. lib. 5. cap. 4. prius thorum non init, quam a digniore sacerdote nova nupta deflorata sit.}
\setmarginnote{6228}{Bohemus lib. 2. cap. 3. Ideo nubere nollent ob mulierum intemperantiam, nullam servare viro fidem putabant.}
\setmarginnote{6229}{Stephanus praefat. Herod. Alius e lupanari meretricem, Pitho dictam, in uxorem duxit; Ptolomaeus Thaidem nobile scortum duxit et ex ea duos filios suscepit, \&c.}
\setmarginnote{6230}{Poggius Floreno.}
\setmarginnote{6231}{Felix Plater.}
\setmarginnote{6232}{Plutarch, Lucian, Salmutz Tit. 2. de porcellanis cum in Panciro 1. de nov. repert. et Plutarchus.}
\setmarginnote{6233}{Stephanus e 1. confor. Bonavent. c. 6. vit. Francisci.}
\setmarginnote{6234}{Plutarch. vit. ejus.}
\setmarginnote{6235}{Vecker. lib. 7. secret.}
\setmarginnote{6236}{Citatur a Gellio.}
\setmarginnote{6237}{Lib. 1. Til. 4. de instit. reipub. de officio mariti.}
\setmarginnote{6238}{Ne cum ea blande nimis agas, ne objurges praesentibus extraneis.}
\setmarginnote{6239}{Epist. 70.}
\setmarginnote{6240}{Ovid. How badly steers of different ages are yoked to the plough.}
\setmarginnote{6241}{Alciat. emb. 116.}
\setmarginnote{6242}{Deipnosoph. l. 3. cap. 12.}
\setmarginnote{6243}{Euripides.}
\setmarginnote{6244}{Pontanus hiarum lib. 1. Maidens shun their embraces; Love, Venus, Hymen, all abhor them.}
\setmarginnote{6245}{Offic. lib. Luxuria cum omni aetati turpis, tum senectuti foedissima.}
\setmarginnote{6246}{Ecclus. xxv. 2. An old man that dotes, \&c.}
\setmarginnote{6247}{Hor. lib. 3. ode 26. He was lately a match for a maid, and contended not ingloriously.}
\setmarginnote{6248}{Alecto herself holds the torch at such nuptials, and malicious Hymen sadly howls.}
\setmarginnote{6249}{Cap. 5. instit. ad optimum vitam; maxima mortalium pars praecipitanter et inconsiderate nubit, idque ea aetate quae minus apta est, quum senex adolescentulae, sanus morbidae, dives pauperi, \&c.}
\setmarginnote{6250}{Obsoleto, intempestivo, turpi remedio fatentur se uti; recordatione pristinarum voluptatum se recreant, et adversante natura, pollinctam carnetn et enectam excitant.}
\setmarginnote{6251}{Lib. 2. nu. 25.}
\setmarginnote{6252}{Qui vero non procreandae prolis, sed explendae; libidinis causa sibi invicem copulantur, non tam conjuges quam fornicarii habentur.}
\setmarginnote{6253}{Lex Papia. Sueton. Claud. c. 23.}
\setmarginnote{6254}{Pontanus biarum lib. 1. More salacious than the sparrow in spring, or the snow-white ring-doves.}
\setmarginnote{6255}{Plautus mercator.}
\setmarginnote{6256}{Symposio.}
\setmarginnote{6257}{Vide Thuani historiam.}
\setmarginnote{6258}{Calabect. vet. poetarum.}
\setmarginnote{6259}{Martial, lib. 3. 62. Epig.}
\setmarginnote{6260}{Lib. 1. Miles.}
\setmarginnote{6261}{Ovid. If you would marry suitably, marry your equal in every respect.}
\setmarginnote{6262}{Parental virtue is a rich inheritance, as well as that chastity which habitually avoids a second husband.}
\setmarginnote{6263}{Rabelais hist. Pantagruel: l. 3. cap. 33.}
\setmarginnote{6264}{Hom. 80. Qui pulchram habet uxorem, nihil pejus habere potest.}
\setmarginnote{6265}{Arniseus.}
\setmarginnote{6266}{Itinerar. Ital. Coloniae edit. 1620. Nomine trium. Ger. fol. 304. displicuit quod dominae filiabus immutent nomen inditum, in Baptisime, et pro Catharina, Margareta, \&c. ne quid desit ad luxuriam, appellant ipsas nominibus Cynthiae, Camaenae, \&c.}
\setmarginnote{6267}{Leonicus de var. lib. 3. c. 43. Asylus virginum deformium Cassandrae templum. Plutarch.}
\setmarginnote{6268}{Polycrat. l. 8. cap. 11.}
\setmarginnote{6269}{If your wife seem deformed, your maid beautiful, still abstain from the latter.}
\setmarginnote{6270}{Marullus. Not the most fair but the most virtuous pleases me.}
\setmarginnote{6271}{Chaloner lib. 9. de repub. Ang.}
\setmarginnote{6272}{Lib. 2. num. 159.}
\setmarginnote{6273}{Si genetrix caste, caste quoque filia vivit; si meretrix mater, filia talis erit.}
\setmarginnote{6274}{Juven. Sat. 6.}
\setmarginnote{6275}{Camerarius cent. 2. cap. 54. oper. subcis.}
\setmarginnote{6276}{Ser. 72. Quod amicus quidam uxorem habens mihi dixit, dicam vobis. In cubili cavendae adulationes vesperi, mane clamores.}
\setmarginnote{6277}{Lib. 4. tit. 4. de institut. Reipub. cap. de officio mariti et uxoris.}
\setmarginnote{6278}{Lib. 4. syl. nup. num. 81. Non curant de uxoribus, nec volunt iis subvenire de victu, vestitu, \&c.}
\setmarginnote{6279}{In Clio. Speciem uxoris supra modum extollens, fecit ut illam nudam coram aspiceret.}
\setmarginnote{6280}{Juven. Sat. 6. He cannot kiss his wife for paint.}
\setmarginnote{6281}{Orat, contra ebr.}
\setmarginnote{6282}{That a matron should not be seen in public without her husband as her spokesman.}
\setmarginnote{6283}{Helpless deer, what are we but a prey?}
\setmarginnote{6284}{Ad baptismum, matrimonium et tumultum.}
\setmarginnote{6285}{Non vociferatur illa si maritus obganniat.}
\setmarginnote{6286}{Fraudem aperiens ostendit ei non aquam sed silentium iracundiae moderari.}
\setmarginnote{6287}{Horol. princi. lib. 2. cap. 8. Diligenter cavendum foeminis illustribus ne frequenter exeant.}
\setmarginnote{6288}{Chaloner. One who delights in the labour of the distaff, and beguiles the hours of labour with a song: her duties assume an air of virtuous beauty when she is busied at the wheel and the spindle with her maids.}
\setmarginnote{6289}{Menander. Whoever guards his wife with bolts and bars will repent his narrow policy.}
\setmarginnote{6290}{Lib. 5. num. 11.}
\setmarginnote{6291}{Ctesias in Persicis finxit vulvae morbum esse nec curari posse nisi cum viro concumberet, hac arte voti compos, \&c.}
\setmarginnote{6292}{Exsolvit vinculis solutumque demisit, at ille inhumanus stupravit conjugem.}
\setmarginnote{6293}{Plutarch. vita ejus.}
\setmarginnote{6294}{Rosinus lib. 2. 19. Valerius lib. 2. cap. 1.}
\setmarginnote{6295}{Alexander ab Alexandro l. 4. cap. 8. gen. dier.}
\setmarginnote{6296}{Fr. Rueus de gemmis l. 2. cap. 8. et 15.}
\setmarginnote{6297}{Strozzius Cicogna lib. 2. cap. 15. spiritet in can. habent ibidem uxores quot volunt cum oculis clarissimis, quos nunquam in aliquem praeter maritum fixuri sunt, \&c. Bredenbacchius, Idem et Bohemus, \&c.}
\setmarginnote{6298}{Uxor caeca ducat maritum surdum, \&c.}
\setmarginnote{6299}{See Valent. Nabod. differ. com. in Alcabitium, ubi plura.}
\setmarginnote{6300}{Cap. 46. Apol. quod mulieres sine concupiscentia aspicere non posset, \&c.}
\setmarginnote{6301}{Ye gods avert such a pestilence from the world.}
