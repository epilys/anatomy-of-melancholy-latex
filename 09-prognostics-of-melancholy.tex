\chapter{Prognostics of Melancholy}\label{ch:prognostics}
%SECT. IV. MEMB. I.
%\section{Prognostics of Melancholy.}

\lettrine{P}{rognostics}, or signs of things to come, are either good or bad. If
this malady be not hereditary, and taken at the beginning, there is
good hope of cure, recens curationem non habet difficilem, saith
\Avicenna{}, l. 3, Fen. 1, Tract. 4, c. 18. That which is with laughter,
of all others is most secure, gentle, and remiss, Hercules de Saxonia.
\authorfootnote{2716}If that evacuation of haemorrhoids, or varices, which they call
the water between the skin, shall happen to a melancholy man, his
misery is ended, Hippocrates Aphor. 6, 11. Galen l. 6, de morbis
vulgar. com. 8, confirms the same; and to this aphorism of Hippocrates,
all the Arabians, new and old Latins subscribe; Montaltus c. 25,
Hercules de Saxonia, Mercurialis, Vittorius Faventinus, \&c. Skenkius,
l. 1, observat. med. c. de Mania, illustrates this aphorism, with an
example of one Daniel Federer a coppersmith that was long melancholy,
and in the end mad about the 27th year of his age, these varices or
water began to arise in his thighs, and he was freed from his madness.

Marius the Roman was so cured, some, say, though with great pain.

Skenkius hath some other instances of women that have been helped by
flowing of their mouths, which before were stopped. That the opening of
the haemorrhoids will do as much for men, all physicians jointly
signify, so they be voluntary, some say, and not by compulsion. All
melancholy are better after a quartan; \authorfootnote{2717}Jobertus saith, scarce any
man hath that ague twice; but whether it free him from this malady,
'tis a question; for many physicians ascribe all long agues for
especial causes, and a quartan ague amongst the rest. \authorfootnote{2718}Rhasis
cont. lib. 1, tract. 9. When melancholy gets out at the superficies of
the skin, or settles breaking out in scabs, leprosy, morphew, or is
purged by stools, or by the urine, or that the spleen is enlarged, and
those varices appear, the disease is dissolved. Guianerius, cap. 5,
tract. 15, adds dropsy, jaundice, dysentery, leprosy, as good signs, to
these scabs, morphews, and breaking out, and proves it out of the 6th
of Hippocrates' Aphorisms.

Evil prognostics on the other part. \li{Inveterata melancholia incurabilis,
if it be inveterate}, it is \authorfootnote{2719}incurable, a common axiom, \li{aut
difficulter curabilis} as they say that make the best, hardly cured.

This Galen witnesseth, l. 3, de loc. affect. cap. 6, \authorfootnote{2720}be it in
whom it will, or from what cause soever, it is ever long, wayward,
tedious, and hard to be cured, if once it be habituated. As Lucian said
of the gout, she was \authorfootnote{2721}the queen of diseases, and inexorable, may
we say of melancholy. Yet Paracelsus will have all diseases whatsoever
curable, and laughs at them which think otherwise, as T. Erastus par.
3, objects to him; although in another place, hereditary diseases he
accounts incurable, and by no art to be removed. \authorfootnote{2722}Hildesheim
spicel. 2, de mel. holds it less dangerous if only \authorfootnote{2723}imagination be
hurt, and not reason, \authorfootnote{2724}the gentlest is from blood. Worse from
choler adust, but the worst of all from melancholy putrefied.
\authorfootnote{2725}Bruel esteems hypochondriacal least dangerous, and the other two
species (opposite to Galen) hardest to be cured. \authorfootnote{2726}The cure is hard
in man, but much more difficult in women. And both men and women must
take notice of that saying of Montanus consil. 230, pro Abate Italo,
\authorfootnote{2727}This malady doth commonly accompany them to their grave;
physicians may ease, and it may lie hid for a time, but they cannot
quite cure it, but it will return again more violent and sharp than at
first, and that upon every small occasion or error: as in Mercury's
weather-beaten statue, that was once all over gilt, the open parts were
clean, yet there was in fimbriis aurum, in the chinks a remnant of
gold: there will be some relics of melancholy left in the purest bodies
(if once tainted) not so easily to be rooted out. \authorfootnote{2728} Oftentimes it
degenerates into epilepsy, apoplexy, convulsions, and blindness: by the
authority of Hippocrates and Galen, \authorfootnote{2729}all aver, if once it possess
the ventricles of the brain, Frambesarius, and Salust. Salvianus adds,
if it get into the optic nerves, blindness. Mercurialis, consil. 20,
had a woman to his patient, that from melancholy became epileptic and
blind. \authorfootnote{2730}If it come from a cold cause, or so continue cold, or
increase, epilepsy; convulsions follow, and blindness, or else in the
end they are moped, sottish, and in all their actions, speeches, and
gestures, ridiculous. \authorfootnote{2731}If it come from a hot cause, they are more
furious, and boisterous, and in conclusion mad. Calescentem
melancholiam saepius sequitur mania. \authorfootnote{2732}If it heat and increase,
that is the common event, \authorfootnote{2733}per circuitus, aut semper insanit, he
is mad by fits, or altogether. For as \authorfootnote{2734}Sennertus contends out of
Crato, there is seminarius ignis in this humour, the very seeds of
fire. If it come from melancholy natural adust, and in excess, they are
often demoniacal, Montanus.

\authorfootnote{2735}Seldom this malady procures death, except (which is the greatest,
most grievous calamity, and the misery of all miseries,) they make away
themselves, which is a frequent thing, and familiar amongst them. 'Tis
\authorfootnote{2736}Hippocrates' observation, Galen's sentence, Etsi mortem timent,
tamen plerumque sibi ipsis mortem consciscunt, l. 3. de locis affec.
cap. 7. The doom of all physicians. 'Tis \authorfootnote{2737}Rabbi Moses' Aphorism,
the prognosticon of \Avicenna{}, Rhasis, Aetius, Gordonius, Valescus,
Altomarus, Salust. Salvianus, Capivaccius, Mercatus, Hercules de
Saxonia, Piso, Bruel, Fuchsius, all, \etc{}

\begin{latin}%
\begin{verse}%
Et saepe usque adeo mortis formidine vitae\\*
Percipit infelix odium lucisque videndae,\\*
Ut sibi consciscat maerenti pectore lethum.\\!
\end{verse}%
\end{latin}%
\translationrule%
\begin{verse}%
And so far forth death's terror doth affright,\\*
He makes away himself, and hates the light\\*
To make an end of fear and grief of heart,\\*
He voluntary dies to ease his smart.\\!
\end{verse}%
\attrib{\getauthornote{2738}}%

In such sort doth the torture and extremity of his misery torment him,
that he can take no pleasure in his life, but is in a manner enforced
to offer violence unto himself, to be freed from his present
insufferable pains. So some (saith \authorfootnote{2739}Fracastorius) in fury, but
most in despair, sorrow, fear, and out of the anguish and vexation of
their souls, offer violence to themselves: for their life is unhappy
and miserable. They can take no rest in the night, nor sleep, or if
they do slumber, fearful dreams astonish them. In the daytime they are
affrighted still by some terrible object, and torn in pieces with
suspicion, fear, sorrow, discontents, cares, shame, anguish, \&c. as so
many wild horses, that they cannot be quiet an hour, a minute of time,
but even against their wills they are intent, and still thinking of it,
they cannot forget it, it grinds their souls day and night, they are
perpetually tormented, a burden to themselves, as Job was, they can
neither eat, drink or sleep. Psal. \rn{cvii.} 18. Their soul abhorreth all
meat, and they are brought to death's door, \authorfootnote{2740}being bound in misery
and iron: they \authorfootnote{2741}curse their stars with Job, \authorfootnote{2742}and day of their
birth, and wish for death: for as Pineda and most interpreters hold,
Job was even melancholy to despair, and almost \authorfootnote{2743}madness itself;
they murmur many times against the world, friends, allies, all mankind,
even against God himself in the bitterness of their passion,
\authorfootnote{2744}vivere nolunt, mori nesciunt, live they will not, die they
cannot. And in the midst of these squalid, ugly, and such irksome days,
they seek at last, finding no comfort, \authorfootnote{2745}no remedy in this wretched
life, to be eased of all by death. Omnia appetunt bonum, all creatures
seek the best, and for their good as they hope, sub specie, in show at
least, vel quia mori pulchrum putant (saith \authorfootnote{2746}Hippocrates) vel quia
putant inde se majoribus malis liberari, to be freed as they wish.

Though many times, as Aesop's fishes, they leap from the frying-pan
into the fire itself, yet they hope to be eased by this means: and
therefore (saith Felix \authorfootnote{2747}Platerus) after many tedious days at last,
either by drowning, hanging, or some such fearful end, they precipitate
or make away themselves: many lamentable examples are daily seen
amongst us: \li{alius ante, fores se laqueo suspendit} (as \Seneca{} notes),
\li{alius se praecipitavit a tecto, ne dominum stomachantem audiret, alius
ne reduceretur a fuga ferrum redegit in viscera}, one hangs himself
before his own door,-another throws himself from the house-top, to
avoid his master's anger,-a third, to escape expulsion, plunges a
dagger into his heart,-so many causes there are-\li{His amor exitio est,
furor his}-love, grief, anger, madness, and shame, \&c. 'Tis a common
calamity, \authorfootnote{2748}a fatal end to this disease, they are condemned to a
violent death, by a jury of physicians, furiously disposed, carried
headlong by their tyrannising wills, enforced by miseries, and there
remains no more to such persons, if that heavenly Physician, by his
assisting grace and mercy alone do not prevent, (for no human
persuasion or art can help) but to be their own butchers, and execute
themselves. Socrates his cicuta, Lucretia's dagger, Timon's halter, are
yet to be had; Cato's knife, and Nero's sword are left behind them, as
so many fatal engines, bequeathed to posterity, and will be used to the
world's end, by such distressed souls: so intolerable, insufferable,
grievous, and violent is their pain, \authorfootnote{2749}so unspeakable and
continuate. One day of grief is an hundred years, as Cardan observes:
'Tis carnificina hominum, angor animi, as well saith Areteus, a plague
of the soul, the cramp and convulsion of the soul, an epitome of hell;
and if there be a hell upon earth, it is to be found in a melancholy
man's heart.

\begin{verse}%
For that deep torture may be call'd an hell,\\*
When more is felt, than one hath power to tell.\\!
\end{verse}%

Yea, that which scoffing Lucian said of the gout in jest, I may truly
affirm of melancholy in earnest.

\begin{latin}%
\begin{verse}%
O triste nomen! o diis odibile\\*
Melancholia lacrymosa, Cocyti filia,\\*
Tu Tartari specubus opacis edita\\*
Erinnys, utero quam Megara suo tulit,\\*
Et ab uberibus aluit, cuique parvidae\\*
Amarulentum in os lac Alecto dedit,\\*
Omnes abominabilem te daemones\\*
Produxere in lucem, exitio mortalium. Et paulo post\\*
Non Jupiter ferit tale telum fulminis,\\*
Non ulla sic procella saevit aequoris,\\*
Non impetuosi tanta vis est turbinis.\\*
An asperos sustineo morsus Cerberi?\\*
Num virus Echidnae membra mea depascitur?\\*
Aut tunica sanie tincta Nessi sanguinis?\\*
Illacrymabile et immedicabile malum hoc.\\!
\end{verse}%
\end{latin}%
\translationrule%
\begin{verse}%
O sad and odious name! a name so fell,\\*
Is this of melancholy, brat of hell.\\*
There born in hellish darkness doth it dwell,\\*
The Furies brought it up, Megara's teat,\\*
Alecto gave it bitter milk to eat.\\*
And all conspir'd a bane to mortal men,\\*
To bring this devil out of that black den.\\*
Jupiter's thunderbolt, not storm at sea,\\*
Nor whirlwind doth our hearts so much dismay.\\*
What? am I bit by that fierce Cerberus?\\*
Or stung by \authorfootnote{2751}serpent so pestiferous?\\*
Or put on shirt that's dipt in Nessus' blood?\\*
My pain's past cure; physic can do no good.\\!
\end{verse}%
\attrib{\getauthornote{2750}}%

No torture of body like unto it, \li{Siculi non invenere tyranni majus
tormentum, no strappadoes}, hot irons, Phalaris' bulls,

\begin{latin}%
\begin{verse}%
Nec ira deum tantum, nec tela, nec hostis,\\*
Quantum sola noces animis illapsa.\\!
\end{verse}%
\end{latin}%
\translationrule%
\begin{verse}%
Jove's wrath, nor devils can\\*
Do so much harm to th' soul of man.\\!
\end{verse}%
\attrib{\getauthornote{2752}}%

All fears, griefs, suspicions, discontents, imbonites, insuavities are
swallowed up, and drowned in this Euripus, this Irish sea, this ocean
of misery, as so many small brooks; 'tis coagulum omnium aerumnarum:
which \authorfootnote{2753}Ammianus applied to his distressed Palladins. I say of our
melancholy man, he is the cream of human adversity, the \authorfootnote{2754}
quintessence, and upshot; all other diseases whatsoever, are but
flea-bitings to melancholy in extent: 'Tis the pith of them all,
%
\begin{latin}%
\begin{verse}%
Hospitium est calamitatis; quid verbis opus est?\\*
Quamcunque malam rem quaeris, illic reperies:\\!
\end{verse}%
\end{latin}%
\translationrule%
\begin{verse}%
What need more words? 'tis calamities inn,\\*
Where seek for any mischief, 'tis within;\\!
\end{verse}%
\attrib{\getauthornote{2755}}%
%
and a melancholy man is that true Prometheus, which is bound to
Caucasus; the true Titius, whose bowels are still by a vulture devoured
(as poets feign) for so doth \authorfootnote{2756}Lilius Geraldus interpret it, of
anxieties, and those griping cares, and so ought it to be understood.

In all other maladies, we seek for help, if a leg or an arm ache,
through any distemperature or wound, or that we have an ordinary
disease, above all things whatsoever, we desire help and health, a
present recovery, if by any means possible it may be procured; we will
freely part with all our other fortunes, substance, endure any misery,
drink bitter potions, swallow those distasteful pills, suffer our
joints to be seared, to be cut off, anything for future health: so
sweet, so dear, so precious above all other things in this world is
life: 'tis that we chiefly desire, long life and happy days,
\authorfootnote{2757}multos da Jupiter annos, increase of years all men wish; but to a
melancholy man, nothing so tedious, nothing so odious; that which they
so carefully seek to preserve \authorfootnote{2758}he abhors, he alone; so intolerable
are his pains; some make a question, graviores morbi corporis an animi,
whether the diseases of the body or mind be more grievous, but there is
no comparison, no doubt to be made of it, multo enim saevior longeque
est atrocior animi, quam corporis cruciatus (Lem. l. 1. c. 12.) the
diseases of the mind are far more grievous.-Totum hic pro vulnere
corpus, body and soul is misaffected here, but the soul especially. So
Cardan testifies de rerum var. lib. 8. 40. \authorfootnote{2759}Maximus Tyrius a
Platonist, and Plutarch, have made just volumes to prove it. \authorfootnote{2760}Dies
adimit aegritudinem hominibus, in other diseases there is some hope
likely, but these unhappy men are born to misery, past all hope of
recovery, incurably sick, the longer they live the worse they are, and
death alone must ease them.

Another doubt is made by some philosophers, whether it be lawful for a
man in such extremity of pain and grief, to make away himself: and how
these men that so do are to be censured. The Platonists approve of it,
that it is lawful in such cases, and upon a necessity; Plotinus l. de
beatitud. c. 7. and Socrates himself defends it, in Plato's Phaedon, if
any man labour of an incurable disease, he may despatch himself, if it
be to his good. Epicurus and his followers, the cynics and stoics in
general affirm it, Epictetus and \authorfootnote{2761}\Seneca{} amongst the rest,
\li{quamcunque veram esse viam ad libertatem}, any way is allowable that
leads to liberty, \authorfootnote{2762}let us give God thanks, that no man is
compelled to live against his will; \authorfootnote{2763} \li{quid ad hominem claustra,
career, custodia? liberum ostium habet}, death is always ready and at
hand. \li{Vides illum praecipitem locum, illud flumen}, dost thou see that
steep place, that river, that pit, that tree, there's liberty at hand,
effugia servitutis et doloris sunt, as that Laconian lad cast himself
headlong (\li{non serviam aiebat puer}) to be freed of his misery: every
vein in thy body, if these be \li{nimis operosi exitus}, will set thee free,
\li{quid tua refert finem facias an accipias?} there's no necessity for a
man to live in misery. \li{Malum est necessitati vivere; sed in necessitate
vivere, necessitas nulla est. Ignavus qui sine causa moritur, et
stultus qui cum dolore vivit.} Idem epi. 58. Wherefore hath our mother
the earth brought out poisons, saith \authorfootnote{2764}\Pliny{}, in so great a
quantity, but that men in distress might make away themselves? which
kings of old had ever in a readiness, ad incerta fortunae venenum sub
custode promptum, Livy writes, and executioners always at hand.

Speusippes being sick was met by Diogenes, and carried on his slaves'
shoulders, he made his moan to the philosopher; but I pity thee not,
quoth Diogenes, \li{qui cum talis vivere sustines}, thou mayst be freed when
thou wilt, meaning by death. \authorfootnote{2765}\Seneca{} therefore commends Cato,
Dido, and Lucretia, for their generous courage in so doing, and others
that voluntarily die, to avoid a greater mischief, to free themselves
from misery, to save their honour, or vindicate their good name, as
Cleopatra did, as Sophonisba, Syphax's wife did, Hannibal did, as
Junius Brutus, as Vibius Virus, and those Campanian senators in Livy
(Dec. 3. lib. 6.) to escape the Roman tyranny, that poisoned
themselves. Themistocles drank bull's blood, rather than he would fight
against his country, and Demosthenes chose rather to drink poison,
Publius Crassi filius, Censorius and Plancus, those heroical Romans to
make away themselves, than to fall into their enemies' hands. How many
myriads besides in all ages might I remember, qui sibi lethum Insontes
pepperere manu, \&c. \authorfootnote{2766}Rhasis in the Maccabees is magnified for it,
Samson's death approved. So did Saul and Jonas sin, and many worthy men
and women, quorum memoria celebratur in Ecclesia, saith
\authorfootnote{2767}Leminchus, for killing themselves to save their chastity and
honour, when Rome was taken, as \Austin instances, l. 1. de Civit. Dei,
cap. 16. Jerome vindicateth the same in Ionam and Ambrose, l. 3. de
virginitate commendeth Pelagia for so doing. Eusebius, lib. 8. cap. 15.
admires a Roman matron for the same fact to save herself from the lust
of Maxentius the Tyrant. Adelhelmus, abbot of Malmesbury, calls them
Beatas virgines quae sic, \&c. Titus Pomponius Atticus, that wise,
discreet, renowned Roman senator, Tully's dear friend, when he had been
long sick, as he supposed, of an incurable disease, vitamque produceret
ad augendos dolores, sine spe salutis, was resolved voluntarily by
famine to despatch himself to be rid of his pain; and when as Agrippa,
and the rest of his weeping friends earnestly besought him, osculantes
obsecrarent ne id quod natura cogeret, ipse acceleraret, not to offer
violence to himself, with a settled resolution he desired again they
would approve of his good intent, and not seek to dehort him from it:
and so constantly died, precesque eorum taciturna sua obstinatione
depressit. Even so did Corellius Rufus, another grave senator, by the
relation of Plinius Secundus, epist. lib. 1. epist. 12. famish himself
to death; pedibus correptus cum incredibiles cruciatus et indignissima
tormenta pateretur, a cibis omnino abstinuit; \authorfootnote{2768}neither he nor
Hispilla his wife could divert him, but destinatus mori obstinate
magis, \&c. die he would, and die he did. So did Lycurgus, Aristotle,
Zeno, Chrysippus, Empedocles, with myriads, \&c. In wars for a man to
run rashly upon imminent danger, and present death, is accounted valour
and magnanimity, \authorfootnote{2769}to be the cause of his own, and many a
thousand's ruin besides, to commit wilful murder in a manner, of
himself and others, is a glorious thing, and he shall be crowned for
it. The \authorfootnote{2770} Massegatae in former times, \authorfootnote{2771}Barbiccians, and I
know not what nations besides, did stifle their old men, after seventy
years, to free them from those grievances incident to that age. So did
the inhabitants of the island of Choa, because their air was pure and
good, and the people generally long lived, antevertebant fatum suum,
priusquam manci forent, aut imbecillitas accederet, papavere vel
cicuta, with poppy or hemlock they prevented death. Sir Thomas More in
his Utopia commends voluntary death, if he be sibi aut aliis molestus,
troublesome to himself or others, (\authorfootnote{2772} especially if to live be a
torment to him,) let him free himself with his own hands from this
tedious life, as from a prison, or suffer himself to be freed by
others. \authorfootnote{2773}And 'tis the same tenet which Laertius relates of Zeno,
of old, Juste sapiens sibi mortem consciscit, si in acerbis doloribus
versetur, membrorum mutilatione aut morbis aegre curandis, and which
Plato 9. de legibus approves, if old age, poverty, ignominy, \&c.
oppress, and which Fabius expresseth in effect. (Praefat. 7. Institut.)
Nemo nisi sua culpa diu dolet. It is an ordinary thing in China, (saith
Mat. Riccius the Jesuit,) \authorfootnote{2774}if they be in despair of better
fortunes, or tired and tortured with misery, to bereave themselves of
life, and many times, to spite their enemies the more, to hang at their
door. Tacitus the historian, Plutarch the philosopher, much approve a
voluntary departure, and Aust. de civ. Dei, l. 1. c. 29. defends a
violent death, so that it be undertaken in a good cause, nemo sic
mortuus, qui non fuerat aliquando moriturus; quid autem interest, quo
mortis genere vita ista finiatur, quando ille cui finitur, iterum mori
non cogitur? \&c. \authorfootnote{2775}no man so voluntarily dies, but volens nolens,
he must die at last, and our life is subject to innumerable casualties,
who knows when they may happen, utrum satius est unam perpeti moriendo,
an omnes timere vivendo, \authorfootnote{2776} rather suffer one, than fear all. Death
is better than a bitter life, Eccl. \rn{xxx.} 17. \authorfootnote{2777}and a harder choice
to live in fear, than by once dying, to be freed from all. Theombrotus
Ambraciotes persuaded I know not how many hundreds of his auditors, by
a luculent oration he made of the miseries of this, and happiness of
that other life, to precipitate themselves. And having read Plato's
divine tract de anima, for example's sake led the way first. That neat
epigram of Callimachus will tell you as much,
%
\begin{latin}%
\begin{verse}%
Jamque vale Soli cum diceret Ambrociotes,\\*
In Stygios fertur desiluisse lacus,\\!

Morte nihil dignum passus: sed forte Platonis\\*
Divini eximum de nece legit opus.\\!
\end{verse}%
\end{latin}%
\translationrule%\setauthornote{2778}
\begin{verse}%
And now when Ambrociotes was bidding farewell to the light of day,\\*
and about to cast himself into the Stygian pool,\\!

although he had not been guilty of any crime that merited death: but, perhaps,\\*
he had read that divine work of Plato upon Death.\\!
\end{verse}%
\attrib{\getauthornote{2778}}%

Calenus and his Indians hated of old to die a natural death:\authormarginnote{2779} the
Circumcellians and Donatists, loathing life, compelled others to make
them away, with many such: \authorfootnote{2780}but these are false and pagan
positions, profane stoical paradoxes, wicked examples, it boots not
what heathen philosophers determine in this kind, they are impious,
abominable, and upon a wrong ground. No evil is to be done that good
may come of it; reclamat Christus, reclamat Scriptura, God, and all
good men are \authorfootnote{2781}against it: He that stabs another, can kill his
body; but he that stabs himself, kills his own soul. \authorfootnote{2782}Male
meretur, qui dat mendico, quod edat; nam et illud quod dat, perit; et
illi producit vitam ad miseriam: he that gives a beggar an alms (as
that comical poet said) doth ill, because he doth but prolong his
miseries. But Lactantius l. 6. c. 7. de vero cultu, calls it a
detestable opinion, and fully confutes it, lib. 3. de sap. cap. 18. and
S. \Austin, epist. 52. ad Macedonium, cap. 61. ad Dulcitium Tribunum: so
doth Hierom to Marcella of Blesilla's death, Non recipio tales animas,
\&c., he calls such men martyres stultae Philosophiae: so doth Cyprian
de duplici martyrio; Si qui sic moriantur, aut infirmitas, aut ambitio,
aut dementia cogit eos; 'tis mere madness so to do, \authorfootnote{2783}furore est ne
moriare mori. To this effect writes Arist. 3. Ethic. Lipsius Manuduc.
ad Stoicam Philosophiaem lib. 3. dissertat. 23. but it needs no
confutation. This only let me add, that in some cases, those \authorfootnote{2784}hard
censures of such as offer violence to their own persons, or in some
desperate fit to others, which sometimes they do, by stabbing,
slashing, \&c. are to be mitigated, as in such as are mad, beside
themselves for the time, or found to have been long melancholy, and
that in extremity, they know not what they do, deprived of reason,
judgment, all, \authorfootnote{2785}as a ship that is void of a pilot, must needs
impinge upon the next rock or sands, and suffer shipwreck. \authorfootnote{2786}P.
Forestus hath a story of two melancholy brethren, that made away
themselves, and for so foul a fact, were accordingly censured to be
infamously buried, as in such cases they use: to terrify others, as it
did the Milesian virgins of old; but upon farther examination of their
misery and madness, the censure was \authorfootnote{2787}revoked, and they were
solemnly interred, as Saul was by David, 2 Sam. ii. 4. and \Seneca{} well
adviseth, \lit{Be justly offended with him as he was a murderer, but pity him now as a dead man}{Irascere interfectori, sed miserere interfecti}.

Thus of their goods and bodies we can dispose; but what shall become of
their souls, God alone can tell; his mercy may come inter pontem et
fontem, inter gladium et jugulum, betwixt the bridge and the brook, the
knife and the throat. Quod cuiquam contigit, quivis potest: Who knows
how he may be tempted? It is his case, it may be thine: \authorfootnote{2788}Quae sua
sors hodie est, eras fore vestra potest. We ought not to be so rash and
rigorous in our censures, as some are; charity will judge and hope the
best: God be merciful unto us all.
