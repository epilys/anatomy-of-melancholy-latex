\chapter{Particular Cures}
%SECT. V. MEMB. I.

%SUBSECT. I.-_Particular Cure of the three several Kinds; of Head Melancholy_.
\section[Cure of three kinds]{Particular Cure of the three several Kinds; of Head Melancholy.}

\lettrine{T}{he} general cures thus briefly examined and discussed, it remains now
to apply these medicines to the three particular species or kinds,
that, according to the several parts affected, each man may tell in
some sort how to help or ease himself. I will treat of head melancholy
first, in which, as in all other good cures, we must begin with diet,
as a matter of most moment, able oftentimes of itself to work this
effect. I have read, saith Laurentius, cap. 8. de Melanch. that in old
diseases which have gotten the upper hand or a habit, the manner of
living is to more purpose, than whatsoever can be drawn out of the most
precious boxes of the apothecaries. This diet, as I have said, is not
only in choice of meat and drink, but of all those other non-natural
things. Let air be clear and moist most part: diet moistening, of good
juice, easy of digestion, and not windy: drink clear, and well brewed,
not too strong, nor too small. Make a melancholy man fat, as
\authorfootnote{4244}Rhasis saith, and thou hast finished the cure. Exercise not too
remiss, nor too violent. Sleep a little more than ordinary.

\authorfootnote{4245}Excrements daily to be voided by art or nature; and which
Fernelius enjoins his patient, consil. 44, above the rest, to avoid all
passions and perturbations of the mind. Let him not be alone or idle
(in any kind of melancholy), but still accompanied with such friends
and familiars he most affects, neatly dressed, washed, and combed,
according to his ability at least, in clean sweet linen, spruce,
handsome, decent, and good apparel; for nothing sooner dejects a man
than want, squalor, and nastiness, foul, or old clothes out of fashion.

Concerning the medicinal part, he that will satisfy himself at large
(in this precedent of diet) and see all at once the whole cure and
manner of it in every distinct species, let him consult with Gordonius,
Valescus, with Prosper Calenius, lib. de atra bile ad Card. Caesium,
Laurentius, cap. 8. et 9. de mela. \AE{}lian Montaltus, de mel. cap. 26.
27. 28. 29. 30. Donat. ab Altomari, cap. 7. artis med. Hercules de
Saxonia, in Panth. cap. 7. et Tract. ejus peculiar. de melan. per
Bolzetam, edit. Venetiis 1620. cap. 17. 18. 19. Savanarola, Rub. 82.
Tract. 8. cap. 1. Sckenkius, in prax. curat. Ital. med. Heurnius, cap.
12. de morb. Victorius Faventius, pract. Magn. et Empir. Hildesheim,
Spicel. 2. de man. et mel. Fel. Plater, Stockerus, Bruel. P. Baverus,
Forestus, Fuchsius, Capivaccius, Rondoletius, Jason Pratensis, Sullust.
Salvian. de remed. lib. 2. cap. 1. Jacchinus, in 9. Rhasis, Lod.
Mercatus, de Inter. morb. cur. lib. 1. cap. 17. Alexan. Messaria,
pract. med. lib. 1. cap. 21. de mel. Piso. Hollerius, \etc{} that have
culled out of those old Greeks, Arabians, and Latins, whatsoever is
observable or fit to be used. Or let him read those counsels and
consultations of Hugo Senensis, consil. 13. et 14. Reinerus Solenander,
consil. 6. sec. 1. et consil. 3. sec. 3. Crato, consil. 16. lib. 1.
Montanus 20. 22. and his following counsels, Laelius a Fonte Egubinus,
consult. 44. 69. 77. 125. 129. 142. Fernelius, consil. 44. 45. 46. Jul.
Caesar Claudinus, Mercurialis, Frambesarius, Sennertus, \etc{}

Wherein he shall find particular receipts, the whole method, preparatives,
purgers, correctors, averters, cordials in great variety and abundance:
out of which, because every man cannot attend to read or peruse them, I
will collect for the benefit of the reader, some few more notable
medicines.

%SUBSECT. II.-_Bloodletting_.
\subsection{Bloodletting.}

Phlebotomy is promiscuously used before and after physic, commonly
before, and upon occasion is often reiterated, if there be any need at
least of it. For Galen, and many others, make a doubt of bleeding at
all in this kind of head-melancholy. If the malady, saith Piso, cap.
23. and Altomarus, cap. 7. Fuchsius, cap. 33. \authorfootnote{4246}shall proceed
primarily from the misaffected brain, the patient in such case shall
not need at all to bleed, except the blood otherwise abound, the veins
be full, inflamed blood, and the party ready to run mad. In immaterial
melancholy, which especially comes from a cold distemperature of
spirits, Hercules de Saxonia, cap. 17. will not admit of phlebotomy;
Laurentius, cap. 9, approves it out of the authority of the Arabians;
but as Mesue, Rhasis, Alexander appoint, \authorfootnote{4247}especially in the head,
to open the veins of the forehead, nose and ears is good. They commonly
set cupping-glasses on the party's shoulders, having first scarified
the place, they apply horseleeches on the head, and in all melancholy
diseases, whether essential or accidental, they cause the haemorrhoids
to be opened, having the eleventh aphorism of the sixth book of
Hippocrates for their ground and warrant, which saith, That in
melancholy and mad men, the varicose tumour or haemorrhoids appearing
doth heal the same. Valescus prescribes bloodletting in all three
kinds, whom Sallust. Salvian follows. \authorfootnote{4248}If the blood abound, which
is discerned by the fullness of the veins, his precedent diet, the
party's laughter, age, \etc{}, begin with the median or middle vein of the
arm; if the blood be ruddy and clear, stop it, but if black in the
spring time, or a good season, or thick, let it run, according to the
party's strength: and some eight or twelve days after, open the head
vein, and the veins in the forehead, or provoke it out of the nostrils,
or cupping-glasses, \etc{} Trallianus allows of this, \authorfootnote{4249}If there have
been any suppression or stopping of blood at nose, or haemorrhoids, or
women's months, then to open a vein in the head or about the ankles.

Yet he doth hardly approve of this course, if melancholy be situated in
the head alone, or in any other dotage, \authorfootnote{4250}except it primarily
proceed from blood, or that the malady be increased by it; for
bloodletting refrigerates and dries up, except the body be very full of
blood, and a kind of ruddiness in the face. Therefore I conclude with
Areteus, \authorfootnote{4251}before you let blood, deliberate of it, and well
consider all circumstances belonging to it.

%SUBSECT. III.-_Preparatives and Purgers_.
\subsection{Preparatives and Purgers.}

After bloodletting we must proceed to other medicines; first prepare,
and then purge, Augeae stabulum purgare, make the body clean before we
hope to do any good. Walter Bruel would have a practitioner begin first
with a clyster of his, which he prescribes before bloodletting: the
common sort, as Mercurialis, Montaltus cap. 30. \etc{} proceed from
lenitives to preparatives, and so to purgers. Lenitives are well known,
electuarium lenitivum, diaphenicum diacatholicon, \etc{} Preparatives are
usually syrups of borage, bugloss, apples, fumitory, thyme and
epithyme, with double as much of the same decoction or distilled water,
or of the waters of bugloss, balm, hops, endive, scolopendry, fumitory,
\etc{} or these sodden in whey, which must be reiterated and used for many
days together. Purges come last, which must not be used at all, if the
malady may be otherwise helped, because they weaken nature and dry so
much, and in giving of them, \authorfootnote{4252} we must begin with the gentlest
first. Some forbid all hot medicines, as Alexander, and Salvianus, \etc{}
Ne insaniores inde fiant, hot medicines increase the disease \authorfootnote{4253}by
drying too much. Purge downward rather than upward, use potions rather
than pills, and when you begin physic, persevere and continue in a
course; for as one observes, \authorfootnote{4254}movere et non educere in omnibus
malum est; to stir up the humour (as one purge commonly doth) and not
to prosecute, doth more harm than good. They must continue in a course
of physic, yet not so that they tire and oppress nature, danda quies
naturae, they must now and then remit, and let nature have some rest.

The most gentle purges to begin with, are \authorfootnote{4255}senna, cassia,
epithyme, myrabolanea, catholicon: if these prevail not, we may proceed
to stronger, as the confection of hamech, pil. Indae, fumitoriae, de
assaieret, of lapis armenus and lazuli, diasena. Or if pills be too
dry; \authorfootnote{4256}some prescribe both hellebores in the last place, amongst
the rest Aretus, \authorfootnote{4257}because this disease will resist a gentle
medicine. Laurentius and Hercules de Saxonia would have antimony tried
last, if the \authorfootnote{4258}party be strong, and it warily given.

\authorfootnote{4259}Trincavelius prefers hierologodium, to whom Francis Alexander in
his Apol. rad. 5. subscribes, a very good medicine they account it. But
Crato in a counsel of his, for the duke of Bavaria's chancellor, wholly
rejects it.

I find a vast chaos of medicines, a confusion of receipts and
magistrals, amongst writers, appropriated to this disease; some of the
chiefest I will rehearse. \authorfootnote{4260}To be seasick first is very good at
seasonable times. Helleborismus Matthioli, with which he vaunts and
boasts he did so many several cures, \authorfootnote{4261}I never gave it (saith he),
but after once or twice, by the help of God, they were happily cured.

The manner of making it he sets down at large in his third book of
Epist. to George Hankshius a physician. Walter Bruel, and Heurnius,
make mention of it with great approbation; so doth Sckenkius in his
memorable cures, and experimental medicines, cen. 6. obser. 37. That
famous Helleborisme of Montanus, which he so often repeats in his
consultations and counsels, as 28. pro. melan. sacerdote, et consil.
148. pro hypochondriaco, and cracks, \authorfootnote{4262} to be a most sovereign
remedy for all melancholy persons, which he hath often given without
offence, and found by long experience and observations to be such.

Quercetan prefers a syrup of hellebore in his Spagirica Pharmac. and
Hellebore's extract cap. 5. of his invention likewise (a most safe
medicine and not unfit to be given children) before all remedies
whatsoever. \authorfootnote{4263}
Paracelsus, in his book of black hellebore, admits this medicine, but
as it is prepared by him. \authorfootnote{4264}It is most certain (saith he) that the
virtue of this herb is great, and admirable in effect, and little
differing from balm itself; and he that knows well how to make use of
it, hath more art than all their books contain, or all the doctors in
Germany can show.

\AE{}lianus Montaltus in his exquisite work \emph{de morb. capitis, cap. 31. de
mel.} sets a special receipt of his own, which, in his practice \blockquote{he
fortunately used}\authorfootnote{4265}; because it is but short I will set it down.

\vspace{-\baselineskip}
\begin{Prescription}[H]
\marginrecipe{Apothecaries` system:}[\baselineskip]
\begin{prescriptionbox}{\marginrecipe{℞: Prescribe}}{\textlatin{mane facta collatura exhibe}}
\item \textlatin{Syrupe de pomis} ℥\marginrecipe{℥: ounce}ij,
\item \textlatin{aqu\ae{} borag ℥iiij}
\item \textlatin{Ellebori nigri per noctem infusi in ligatura 6 vel 8 gr.}
\end{prescriptionbox}
\begin{prescriptionbox}{Prepare}{by morning you should see results}
\item one ounce \emph{fruit syrup},
\item three ounces \emph{borage} infusion,
\item 6 or 8 grains \emph{black hellebore} infused all night
\end{prescriptionbox}
\caption{a recipe 2}
\end{Prescription}

Other receipts of the same to this purpose you shall find in him. Valescus admires pulvis Hali, and Jason Pratensis after him: the confection of which our new London Pharmacopoeia hath lately revived. \authorfootnote{4266}\blockquote{Put case} (saith he) \blockquote{all other medicines fail, by the help of God this alone shall do it, and 'tis a crowned medicine which must be kept
in secret.}

\vspace{-2\baselineskip}
\begin{Prescription}[H]
\marginrecipe{Apothecaries` system:}[\baselineskip]
\begin{prescriptionbox}{}{\textlatin{omnia, et ipsius pulveris scrup. 4. singulis septimanis assumat.}}
\item \textlatin{epithymi semunc. lapidis lazuli},
\item \textlatin{agarici ana ℥ij},
\item \textlatin{Scammnonii ʒj},
\item \textlatin{caryophyllorum numero 20; pulverisentur},
\end{prescriptionbox}
\begin{prescriptionbox}{FIXME translation}{\textlatin{omnia, et ipsius pulveris scrup. 4. singulis septimanis assumat.}}
\item \textlatin{epithymi semunc. lapidis lazuli},
\item \textlatin{agarici ana ℥ij},
\item \textlatin{Scammnonii ʒj},
\item \textlatin{caryophyllorum numero 20; pulverisentur},
\end{prescriptionbox}
\caption{third recipe}
\end{Prescription}

To these I may add Arnoldi vinum Buglossalum, or borage wine before
mentioned, which \authorfootnote{4267}Mizaldus calls vinum mirabile, a wonderful wine,
and Stockerus vouchsafes to repeat verbatim amongst other receipts.

Rubeus his \authorfootnote{4268}compound water out of Savanarola; Pinetus his balm;
Cardan's Pulvis Hyacinthi, with which, in his book de curis admirandis,
he boasts that he had cured many melancholy persons in eight days,
which \authorfootnote{4269}Sckenkius puts amongst his observable medicines; Altomarus
his syrup, with which \authorfootnote{4270}he calls God so solemnly to witness, he
hath in his kind done many excellent cures, and which Sckenkius cent.
7. observ. 80. mentioneth, Daniel Sennertus lib. 1. part. 2. cap. 12.
so much commends; Rulandus' admirable water for melancholy, which cent.
2. cap. 96. he names Spiritum vitae aureum, Panaceam, what not, and his
absolute medicine of 50 eggs, curat. Empir. cent. 1. cur. 5. to be
taken three in a morning, with a powder of his. \authorfootnote{4271}Faventinus prac.
Emper. doubles this number of eggs, and will have 101 to be taken by
three and three in like sort, which Sallust Salvian approves de red.
med. lib. 2. c. 1. with some of the same powder, till all be spent, a
most excellent remedy for all melancholy and mad men.

\begin{Prescription}[H]
\marginrecipe{Apothecaries` system:}[\baselineskip]
\begin{prescriptionbox}{}{\textlatin{misce, fiat pulvis.}}
\item \textlatin{Epithymi, thymi, ana drachmas duas},
\item \textlatin{sacchari albi unciam unam},
\item \textlatin{croci grana tria},
\item \textlatin{Cinamomi drachmam unam}
\end{prescriptionbox}
\begin{prescriptionbox}{FIXME translation}{\textlatin{misce, fiat pulvis.}}
\item \textlatin{Epithymi, thymi, ana drachmas duas},
\item \textlatin{sacchari albi unciam unam},
\item \textlatin{croci grana tria},
\item \textlatin{Cinamomi drachmam unam}
\end{prescriptionbox}
\caption{fourth recipe}
\end{Prescription}

All these yet are nothing to those \authorfootnote{4272}chemical preparatives of Aqua
Chalidonia, quintessence of hellebore, salts, extracts, distillations,
oils, Aurum potabile, \etc{} Dr. Anthony in his book \textlatin{de auro potab. edit.
1600}. is all in all for it. \authorfootnote{4273}And though all the schools of
Galenists, with a wicked and unthankful pride and scorn, detest it in
their practice, yet in more grievous diseases, when their vegetals will
do no good, they are compelled to seek the help of minerals, though
they use them rashly, unprofitably, slackly, and to no purpose.

Rhenanus, a Dutch chemist, in his book \textlatin{de Sale e puteo emergente}, takes
upon him to apologise for Anthony, and sets light by all that speak
against him. But what do I meddle with this great controversy, which is
the subject of many volumes? Let Paracelsus, Quercetan, Crollius, and
the brethren of the rosy cross, defend themselves as they may. Crato,
Erastus, and the Galenists oppugn Paracelsus, he brags on the other
side, he did more famous cures by this means, than all the Galenists in
Europe, and calls himself a monarch; Galen, Hippocrates, infants,
illiterate, \etc{} As Thessalus of old railed against those ancient
Asclepiadean writers, \authorfootnote{4274}he condemns others, insults, triumphs,
overcomes all antiquity (saith Galen as if he spake to him) declares
himself a conqueror, and crowns his own doings. \authorfootnote{4275}One drop of their
chemical preparatives shall do more good than all their fulsome
potions. Erastus, and the rest of the Galenists vilify them on the
other side, as heretics in physic; \authorfootnote{4276}Paracelsus did that in physic,
which Luther in Divinity. \authorfootnote{4277}A drunken rogue he was, a base fellow,
a magician, he had the devil for his master, devils his familiar
companions, and what he did, was done by the help of the devil. Thus
they contend and rail, and every mart write books pro and con, et adhuc
sub judice lis est: let them agree as they will, I proceed.

%SUBSECT. IV.-_Averters_.
\subsection{Averters.}

Averters and purgers must go together, as tending all to the same
purpose, to divert this rebellious humour, and turn it another way. In
this range, clysters and suppositories challenge a chief place, to draw
this humour from the brain and heart, to the more ignoble parts. Some
would have them still used a few days between, and those to be made
with the boiled seeds of anise, fennel, and bastard saffron, hops,
thyme, epithyme, mallows, fumitory, bugloss, polypody, senna, diasene,
hamech, cassia, diacatholicon, hierologodium, oil of violets, sweet
almonds, \etc{} For without question, a clyster opportunely used, cannot
choose in this, as most other maladies, but to do very much good;
Clysteres nutriunt, sometimes clysters nourish, as they may be
prepared, as I was informed not long since by a learned lecture of our
natural philosophy \authorfootnote{4278}reader, which he handled by way of discourse,
out of some other noted physicians. Such things as provoke urine most
commend, but not sweat. Trincavelius consil. 16. cap. 1. in
head-melancholy forbids it. P. Byarus and others approve frictions of
the outward parts, and to bathe them with warm water. Instead of
ordinary frictions, Cardan prescribes rubbing with nettles till they
blister the skin, which likewise \authorfootnote{4279}Basardus Visontinus so much
magnifies.

Sneezing, masticatories, and nasals are generally received. Montaltus
c. 34. Hildesheim spicel. 3. fol. 136 and 238. give several receipts of
all three. Hercules de Saxonia relates of an empiric in Venice
\authorfootnote{4280}that had a strong water to purge by the mouth and nostrils, which
he still used in head-melancholy, and would sell for no gold.

To open months and haemorrhoids is very good physic, \authorfootnote{4281}If they have
been formerly stopped. Faventinus would have them opened with
horseleeches, so would Hercul. de Sax. Julius Alexandrinus consil. 185.

Scoltzii thinks aloes fitter: \authorfootnote{4282}most approve horseleeches in this
case, to be applied to the forehead, \authorfootnote{4283}nostrils, and other places.

Montaltus cap. 29. out of Alexander and others, prescribes \authorfootnote{4284}
cupping-glasses, and issues in the left thigh. Aretus lib. 7. cap. 5.

\authorfootnote{4285}Paulus Regolinus, Sylvius will have them without scarification,
applied to the shoulders and back, thighs and feet: \authorfootnote{4286}Montaltus
cap. 34. bids open an issue in the arm, or hinder part of the head.

\authorfootnote{4287}Piso enjoins ligatures, frictions, suppositories, and
cupping-glasses, still without scarification, and the rest.

Cauteries and hot irons are to be used \authorfootnote{4288}in the suture of the
crown, and the seared or ulcerated place suffered to run a good while.

'Tis not amiss to bore the skull with an instrument, to let out the
fuliginous vapours. Sallus. Salvianus de re medic. lib. 2. cap. 1.
\authorfootnote{4289}because this humour hardly yields to other physic, would have the
leg cauterised, or the left leg, below the knee, \authorfootnote{4290}and the head
bored in two or three places, for that it much avails to the exhalation
of the vapours; \authorfootnote{4291} I saw (saith he) a melancholy man at Rome, that
by no remedies could be healed, but when by chance he was wounded in
the head, and the skull broken, he was excellently cured. Another, to
the admiration of the beholders, \authorfootnote{4292}breaking his head with a fall
from on high, was instantly recovered of his dotage. Gordonius cap. 13.
part. 2. would have these cauteries tried last, when no other physic
will serve. \authorfootnote{4293} The head to be shaved and bored to let out fumes,
which without doubt will do much good. I saw a melancholy man wounded
in the head with a sword, his brainpan broken; so long as the wound was
open he was well, but when his wound was healed, his dotage returned
again. But Alexander Messaria a professor in Padua, lib. 1. pract. med.
cap. 21. de melanchol. will allow no cauteries at all, 'tis too stiff a
humour and too thick as he holds, to be so evaporated.

Guianerius c. 8. Tract. 15. cured a nobleman in Savoy, by boring alone,
\authorfootnote{4294}leaving the hole open a month together, by means of which, after
two years' melancholy and madness, he was delivered. All approve of
this remedy in the suture of the crown; but Arculanus would have the
cautery to be made with gold. In many other parts, these cauteries are
prescribed for melancholy men, as in the thighs, (Mercurialis consil.
86.) arms, legs. Idem consil. 6. \& 19. \& 25. Montanus 86. Rodericus a
Fonseca tom. 2. cousult. 84. pro hypochond. coxa dextra, \etc{}, but most
in the head, if other physic will do no good.

%SUBSECT. V.-_Alteratives and Cordials, corroborating, resolving the Reliques, and mending the Temperament_.
\section[Alternatives and Cordials]{Alteratives and Cordials, corroborating, resolving the Reliques, and mending the Temperament.}

\lettrine{B}{ecause} this humour is so malign of itself, and so hard to be removed,
the reliques are to be cleansed, by alteratives, cordials, and such
means: the temper is to be altered and amended, with such things as
fortify and strengthen the heart and brain, \authorfootnote{4295}which are commonly
both affected in this malady, and do mutually misaffect one another:
which are still to be given every other day, or some few days inserted
after a purge, or like physic, as occasion serves, and are of such
force, that many times they help alone, and as \authorfootnote{4296}Arnoldus holds in
his Aphorisms, are to be preferred before all other medicines, in what
kind soever.

Amongst this number of cordials and alteratives, I do not find a more
present remedy, than a cup of wine or strong drink, if it be soberly
and opportunely used. It makes a man bold, hardy, courageous,
\authorfootnote{4297}whetteth the wit, if moderately taken, (and as Plutarch
\authorfootnote{4298}saith, Symp. 7. quaest. 12.) it makes those which are otherwise
dull, to exhale and evaporate like frankincense, or quicken (Xenophon
adds) \authorfootnote{4299}as oil doth fire. \authorfootnote{4300}A famous cordial Matthiolus in
Dioscoridum calls it, an excellent nutriment to refresh the body, it
makes a good colour, a flourishing age, helps concoction, fortifies the
stomach, takes away obstructions, provokes urine, drives out
excrements, procures sleep, clears the blood, expels wind and cold
poisons, attenuates, concocts, dissipates all thick vapours, and
fuliginous humours. And that which is all in all to my purpose, it
takes away fear and sorrow. \authorfootnote{4301}Curas edaces dissipat Evius. It glads
the heart of man, Psal. \rn{civ.} 15. hilaritatis dulce seminarium. Helena's
bowl, the sole nectar of the gods, or that true nepenthes in
\authorfootnote{4302}Homer, which puts away care and grief, as Oribasius 5. Collect,
cap. 7. and some others will, was nought else but a cup of good wine.

It makes the mind of the king and of the fatherless both one, of the
bond and freeman, poor and rich; it turneth all his thoughts to joy and
mirth, makes him remember no sorrow or debt, but enricheth his heart,
and makes him speak by talents, Esdras \rn{iii.} 19, 20, 21. It gives life
itself, spirits, wit, \etc{} For which cause the ancients called Bacchus,
Liber pater a liberando, and \authorfootnote{4303}sacrificed to Bacchus and Pallas
still upon an altar. \authorfootnote{4304}Wine measurably drunk, and in time, brings
gladness and cheerfulness of mind, it cheereth God and men, Judges \rn{ix.}
13. \li{laetitiae Bacchus dator}, it makes an old wife dance, and such as
are in misery to forget evil, and be \authorfootnote{4305}merry.
%
\begin{latin}
\begin{verse}
Bacchus et afflictis requiem mortalibus affert,\\*
Crura licet duro compede vincta forent.\\!
\end{verse}
\end{latin}
\translationrule
\begin{verse}
Wine makes a troubled soul to rest,\\*
Though feet with fetters be opprest.\\!
\end{verse}%

Demetrius in Plutarch, when he fell into Seleucus's hands, and was
prisoner in Syria, \authorfootnote{4306}spent his time with dice and drink that he
might so ease his discontented mind, and avoid those continual
cogitations of his present condition wherewith he was tormented.

Therefore Solomon, Prov. \rn{xxxi.} 6, bids wine be given to him that is
ready to \authorfootnote{4307}perish, and to him that hath grief of heart, let him
drink that he forget his poverty, and remember his misery no more.

\lit{It easeth a burdened soul}{Sollicitis animis onus eximit}, nothing
speedier, nothing better; which the prophet Zachariah perceived, when
he said, that in the time of Messias, they of Ephraim should be glad,
and their heart should rejoice as through wine. All which makes me very
well approve of that pretty description of a feast in \authorfootnote{4308}
Bartholomeus Anglicus, when grace was said, their hands washed, and the
guests sufficiently exhilarated, with good discourse, sweet music,
dainty fare, \li{exhilarationis gratia, pocula iterum atque iterum
offeruntur}, as a corollary to conclude the feast, and continue their
mirth, a grace cup came in to cheer their hearts, and they drank
healths to one another again and again. Which as I. Fredericus
Matenesius, Crit. Christ. lib. 2. cap. 5, 6, \& 7, was an old custom in
all ages in every commonwealth, so as they be not enforced, bibere per
violentiam, but as in that royal feast of \authorfootnote{4309} Ahasuerus, which
lasted 180 days, without compulsion they drank by order in golden
vessels, when and what they would themselves. This of drink is a most
easy and parable remedy, a common, a cheap, still ready against fear,
sorrow, and such troublesome thoughts, that molest the mind; as
brimstone with fire, the spirits on a sudden are enlightened by it. No
better physic (saith \authorfootnote{4310}Rhasis) for a melancholy man: and he that
can keep company, and carouse, needs no other medicines, 'tis enough.

His countryman \Avicenna{}, 31. doc. 2. cap. 8. proceeds farther yet, and
will have him that is troubled in mind, or melancholy, not to drink
only, but now and then to be drunk: excellent good physic it is for
this and many other diseases. Magninus Reg. san. part. 3. c. 31. will
have them to be so once a month at least, and gives his reasons for it,
\authorfootnote{4311}because it scours the body by vomit, urine, sweat, of all manner
of superfluities, and keeps it clean. Of the same mind is \Seneca{} the
philosopher, in his book de tranquil. lib. 1. c. 15. nonnunquam ut in
aliis morbis ad ebrietatem usque veniendum; Curas deprimit, tristitiae
medetur, it is good sometimes to be drunk, it helps sorrow, depresseth
cares, and so concludes this tract with a cup of wine: Habes, Serene
charissime, quae ad, tranquillitatem animae, pertinent. But these are
epicureal tenets, tending to looseness of life, luxury and atheism,
maintained alone by some heathens, dissolute Arabians, profane
Christians, and are exploded by Rabbi Moses, tract. 4. Guliel,
Placentius, lib. 1. cap. 8. Valescus de Taranta, and most accurately
ventilated by Jo. Sylvaticus, a late writer and physician of Milan,
med. cont. cap. 14. where you shall find this tenet copiously confuted.

Howsoever you say, if this be true, that wine and strong drink have
such virtue to expel fear and sorrow, and to exhilarate the mind, ever
hereafter let's drink and be merry.
%
\authorfootnote{4312}%
\begin{latin}%
\begin{verse}%
Prome reconditum, Lyde strenua, caecubum,\\*
Capaciores puer huc affer Scyphos,\\*
Et Chia vina aut Lesbia.\\!
\end{verse}%
\end{latin}%
\translationrule%
\begin{verse}%
Come, lusty Lyda, fill's a cup of sack,
And, sirrah drawer, bigger pots we lack,
And Scio wines that have so good a smack.
\end{verse}%

I say with him in \authorfootnote{4313}A. Gellius, let us maintain the vigour of our
souls with a moderate cup of wine, \authorfootnote{4314}Natis in usum laetitiae
scyphis, and drink to refresh our mind; if there be any cold sorrow in
it, or torpid bashfulness, let's wash it all away.-Nunc vino pellite
curas; so saith \authorfootnote{4315}Horace, so saith Anacreon,

\textgreek[variant=ancient]{Μεθύοντα γαρ με κεῖσθαι
Πολὺ κρεισσον ἤ θανόντα.}

Let's drive down care with a cup of wine: and so say I too, (though I
drink none myself) for all this may be done, so that it be modestly,
soberly, opportunely used: so that they be not drunk with wine, wherein
is excess, which our \authorfootnote{4316}Apostle forewarns; for as \Chrysostom{} well
comments on that place, ad laetitiam datum est vinum, non ad
ebrietatem, 'tis for mirth wine, but not for madness: and will you know
where, when, and how that is to be understood? \li{Vis discere ubi bonum
sit vinum? Audi quid dicat Scriptura}, hear the Scriptures, Give wine to
them that are in sorrow, or as Paul bid Timothy drink wine for his
stomach's sake, for concoction, health, or some such honest occasion.

Otherwise, as \authorfootnote{4317} \Pliny{} telleth us; if singular moderation be not
had, \authorfootnote{4318}nothing so pernicious, 'tis mere vinegar, blandus daemon,
poison itself. But hear a more fearful doom, Habac. \rn{ii.} 15. and 16. Woe
be to him that makes his neighbour drunk, shameful spewing shall be
upon his glory. Let not good fellows triumph therefore (saith
Matthiolus) that I have so much commended wine, if it be immoderately
taken, instead of making glad, it confounds both body and soul, it
makes a giddy head, a sorrowful heart. And 'twas well said of the poet
of old, Vine causeth mirth and grief, \authorfootnote{4319}nothing so good for some,
so bad for others, especially as \authorfootnote{4320}one observes, qui a causa calida
male habent, that are hot or inflamed. And so of spices, they alone, as
I have showed, cause head-melancholy themselves, they must not use wine
as an \authorfootnote{4321}ordinary drink, or in their diet. But to determine with
Laurentius, c. 8. de melan. wine is bad for madmen, and such as are
troubled with heat in their inner parts or brains; but to melancholy,
which is cold (as most is), wine, soberly used, may be very good.

I may say the same of the decoction of China roots, sassafras,
sarsaparilla, guaiacum: China, saith Manardus, makes a good colour in
the face, takes away melancholy, and all infirmities proceeding from
cold, even so sarsaparilla provokes sweat mightily, guaiacum dries,
Claudinus, consult. 89. \& 46. Montanus, Capivaccius, consult. 188.
Scoltzii, make frequent and good use of guaiacum and China, \authorfootnote{4322}so
that the liver be not incensed, good for such as are cold, as most
melancholy men are, but by no means to be mentioned in hot.

The Turks have a drink called coffee (for they use no wine), so named
of a berry as black as soot, and as bitter, (like that black drink
which was in use amongst the Lacedaemonians, and perhaps the same,)
which they sip still of, and sup as warm as they can suffer; they spend
much time in those coffeehouses, which are somewhat like our alehouses
or taverns, and there they sit chatting and drinking to drive away the
time, and to be merry together, because they find by experience that
kind of drink, so used, helpeth digestion, and procureth alacrity. Some
of them take opium to this purpose.

Borage, balm, saffron, gold, I have spoken of; Montaltus, c. 23.
commends scorzonera roots condite. Garcius ab Horto, plant. hist. lib.
2. cap. 25. makes mention of an herb called datura, \authorfootnote{4323}which, if it
be eaten for twenty-four hours following, takes away all sense of
grief, makes them incline to laughter and mirth: and another called
bauge, like in effect to opium, which puts them for a time into a kind
of ecstasy, and makes them gently to laugh. One of the Roman emperors
had a seed, which he did ordinarily eat to exhilarate himself.

\authorfootnote{4324}Christophorus Ayrerus prefers bezoar stone, and the confection of
alkermes, before other cordials, and amber in some cases.

\authorfootnote{4325}Alkermes comforts the inner parts; and bezoar stone hath an
especial virtue against all melancholy affections, \authorfootnote{4326}it refresheth
the heart, and corroborates the whole body. \authorfootnote{4327}Amber provokes urine,
helps the body, breaks wind, \etc{} After a purge, 3 or 4 grains of bezoar
stone, and 3 grains of ambergris, drunk or taken in borage or bugloss
water, in which gold hot hath been quenched, will do much good, and the
purge shall diminish less (the heart so refreshed) of the strength and
substance of the body.

\begin{Prescription}[H]
\marginrecipe{Apothecaries` system:}[\baselineskip]
\begin{prescriptionbox}{\textlatin{confect}}{\textlatin{cum syrup, de cort. citri; fiat \emph{electuarium}}}
\item alkermes ℥\marginrecipe{℥: ounce}[-1\baselineskip]\marginrecipe{ß: \emph{semis}, half}ß,
\item \textlatin{lap. bezoar.} ℈\marginrecipe{℈: scruple}j,
\item \textlatin{succini albi subtiliss. pulverisat. ℈jj}\marginrecipe{j: unit}
\end{prescriptionbox}
\begin{prescriptionbox}{prepare}{Make into an \marginrecipe{\emph{electuary}: medicine mixed with honey or other sweetener in order to make it more palatable to swallow.}[-4\baselineskip]\emph{electuary} with syrup and citrus stem}
\item half an ounce of \marginrecipe{\emph{alkermes}: Italian liquor}[-3\baselineskip]\emph{alkermes} liquor,
\item one scruple \emph{bezoar} stone,
\item two scruples of fine white amber, pulverised
\end{prescriptionbox}
\caption{another recipe}
\end{Prescription}

To \footnoteA{stone-like mass found in the gastrointestinal organs of animals. \theeditor{}}{bezoar} stone most subscribe, Manardus, and \authorfootnote{4328}many others; it
takes away sadness, and makes him merry that useth it; I have seen some
that have been much diseased with faintness, swooning, and melancholy,
that taking the weight of three grains of this stone, in the water of
oxtongue, have been cured. Garcias ab Horto brags how many desperate
cures he hath done upon melancholy men by this alone, when all
physicians had forsaken them. But alkermes many except against; in some
cases it may help, if it be good and of the best, such as that of
Montpelier in France, which \authorfootnote{4329}Iodocus Sincerus, Itinerario Galliae,
so much magnifies, and would have no traveller omit to see it made. But
it is not so general a medicine as the other. Fernelius, consil. 49,
suspects alkermes, by reason of its heat, \authorfootnote{4330}nothing (saith he)
sooner exasperates this disease, than the use of hot working meats and
medicines, and would have them for that cause warily taken. I conclude,
therefore, of this and all other medicines, as Thucydides of the plague
at Athens, no remedy could be prescribed for it, \li{Nam quod uni profuit,
hoc aliis erat exitio}: there is no Catholic medicine to be had: that
which helps one, is pernicious to another.

\li{Diamargaritum frigidum, diambra, diaboraginatum, electuarium
laetificans Galeni et Rhasis, de gemmis, dianthos, diamoscum dulce et
amarum, electuarium conciliatoris, syrup. Cidoniorum de pomis},
conserves of roses, violets, fumitory, enula campana, satyrion, lemons,
orange-pills, condite, \etc{}, have their good use.

\begin{Prescription}[H]
\marginrecipe{Apothecaries` system:}[\baselineskip]
\begin{prescriptionbox}{\authorfootnote{4331}}{\textlatin{misce cum syrupo de pomis}}
\item \textlatin{Diamoschi dulcis et amari ana ʒjj},
\item \textlatin{Diabuglossati, Diaboraginati, sacchari violacei ana j},
\end{prescriptionbox}
\begin{prescriptionbox}{FIXME translation \authorfootnote{4331}}{\textlatin{misce cum syrupo de pomis}}
\item \textlatin{Diamoschi dulcis et amari ana ʒjj},
\item \textlatin{Diabuglossati, Diaboraginati, sacchari violacei ana j},
\end{prescriptionbox}
\caption{fifth recipe}
\end{Prescription}

Every physician is full of such receipts: one only I will add for the
rareness of it, which I find recorded by many learned authors, as an
approved medicine against dotage, head-melancholy, and such diseases of
the brain. Take a \authorfootnote{4332}ram's head that never meddled with an ewe, cut
off at a blow, and the horns only take away, boil it well, skin and
wool together; after it is well sod, take out the brains, and put these
spices to it, cinnamon, ginger, nutmeg, mace, cloves, ana ℥ß, mingle
the powder of these spices with it, and heat them in a platter upon a
chafing-dish of coals together, stirring them well, that they do not
burn; take heed it be not overmuch dried, or drier than a calf's brains
ready to be eaten. Keep it so prepared, and for three days give it the
patient fasting, so that he fast two hours after it. It may be eaten
with bread in an egg or broth, or any way, so it be taken. For fourteen
days let him use this diet, drink no wine, \etc{} Gesner, hist. animal.
lib. 1. pag. 917. Caricterius, pract. 13. in Nich. de metri. pag. 129.
Iatro: Wittenberg. edit. Tubing. pag. 62, mention this medicine, though
with some variation; he that list may try it, \authorfootnote{4333}and many such.

Odoraments to smell to, of rosewater, violet flowers, balm, rose-cakes,
vinegar, \etc{}, do much recreate the brains and spirits, according to
Solomon. Prov. \rn{xxvii.} 9. They rejoice the heart, and as some say,
nourish; 'tis a question commonly controverted in our schools, an
odores nutriant; let Ficinus, lib. 2. cap. 18. decide it; \authorfootnote{4334}many
arguments he brings to prove it; as of \Democritus{}, that lived by the
smell of bread alone, applied to his nostrils, for some few days, when
for old age he could eat no meat. Ferrerius, lib. 2. meth. speaks of an
excellent confection of his making, of wine, saffron, \etc{}, which he
prescribed to dull, weak, feeble, and dying men to smell to, and by it
to have done very much good, \li{aeque fere profuisse olfactu, et potu}, as
if he had given them drink. Our noble and learned Lord \authorfootnote{4335}Verulam,
in his book \textlatin{de vita et morte}, commends, therefore, all such cold smells
as any way serve to refrigerate the spirits. \textlatin{Montanus, consil. 31},
prescribes a form which he would have his melancholy patient never to
have out of his hands. If you will have them spagirically prepared,
look in \textlatin{Oswaldus Crollius, basil. Chymica}.

Irrigations of the head shaven, \authorfootnote{4336}of the flowers of water lilies,
lettuce, violets, camomile, wild mallows, wether's-head, \etc{}, must be
used many mornings together. Montan. consil. 31, would have the head so
washed once a week. \textlatin{Laelius a Fonte Eugubinus consult. 44}, for an
Italian count, troubled with head-melancholy, repeats many medicines
which he tried, \authorfootnote{4337}but two alone which did the cure; use of whey
made of goat's milk, with the extract of hellebore, and irrigations of
the head with water lilies, lettuce, violets, camomile, \etc{}, upon the
suture of the crown. Piso commends a ram's lungs applied hot to the
fore part of the head, \authorfootnote{4338}or a young lamb divided in the back,
exenterated, \etc{}; all acknowledge the chief cure in moistening
throughout. Some, saith Laurentius, use powders and caps to the brain;
but forasmuch as such aromatical things are hot and dry, they must be
sparingly administered.

Unto the heart we may do well to apply bags, epithems, ointments, of
which Laurentius, c. 9. de melan. gives examples. Bruel prescribes an
epithem for the heart, of bugloss, borage, water-lily, violet waters,
sweet-wine, balm leaves, nutmegs, cloves, \etc{}

For the belly, make a fomentation of oil, \authorfootnote{4339}in which the seeds of
cumin, rue, carrots, dill, have been boiled.

Baths are of wonderful great force in this malady, much admired by
\authorfootnote{4340} Galen, \authorfootnote{4341}Aetius, Rhasis, \etc{}, of sweet water, in which is
boiled the leaves of mallows, roses, violets, water-lilies,
wether's-head, flowers of bugloss, camomile, melilot, \etc{} Guianer, cap.
8. tract. 15, would have them used twice a day, and when they came
forth of the baths, their back bones to be anointed with oil of
almonds, violets, nymphea, fresh capon grease, \etc{}

Amulets and things to be borne about, I find prescribed, taxed by some,
approved by Renodeus, Platerus, (amuleta inquit non negligenda) and
others; look for them in Mizaldus, Porta, Albertus, \etc{} Bassardus
Viscontinus, ant. philos. commends hypericon, or St. John's wort
gathered on a \authorfootnote{4342}Friday in the hour of Jupiter, when it comes to his
effectual operation (that is about the full moon in July); so gathered
and borne, or hung about the neck, it mightily helps this affection,
and drives away all fantastical spirits. \authorfootnote{4343}Philes, a Greek author
that flourished in the time of Michael Paleologus, writes that a sheep
or kid's skin, whom a wolf worried, \authorfootnote{4344}Haedus inhumani raptus ab ore
lupi, ought not at all to be worn about a man, because it causeth
palpitation of the heart, not for any fear, but a secret virtue which
amulets have. A ring made of the hoof of an ass's right fore foot
carried about, \etc{} I say with \authorfootnote{4345}Renodeus, they are not altogether
to be rejected. Paeony doth cure epilepsy; precious stones most
diseases; \authorfootnote{4346}a wolf's dung borne with one helps the colic, \authorfootnote{4347}a
spider an ague, \etc{} Being in the country in the vacation time not many
years since, at Lindley in Leicestershire, my father's house, I first
observed this amulet of a spider in a nut-shell lapped in silk, \etc{}, so
applied for an ague by \authorfootnote{4348}my mother; whom, although I knew to have
excellent skill in chirurgery, sore eyes, aches, \etc{}, and such
experimental medicines, as all the country where she dwelt can witness,
to have done many famous and good cures upon diverse poor folks, that
were otherwise destitute of help: yet among all other experiments, this
methought was most absurd and ridiculous, I could see no warrant for
it. \lit{For what antipathy?}{Quid aranea cum febre?} till at length rambling
amongst authors (as often I do) I found this very medicine in
Dioscorides, approved by Matthiolus, repeated by Alderovandus, \textlatin{cap. de
Aranea, lib. de insectis}, I began to have a better opinion of it, and
to give more credit to amulets, when I saw it in some parties answer to
experience. Some medicines are to be exploded, that consist of words,
characters, spells, and charms, which can do no good at all, but out of
a strong conceit, as Pomponatius proves; or the devil's policy, who is
the first founder and teacher of them.

%SUBSECT. VI.-_Correctors of Accidents to procure Sleep. Against fearful Dreams, Redness, \etc{}_
\section[Correctors for sleep]{Correctors of Accidents to procure Sleep. Against fearful Dreams, Redness, \etc{}}

\lettrine{W}{hen} you have used all good means and helps of alteratives, averters,
diminutives, yet there will be still certain accidents to be corrected
and amended, as waking, fearful dreams, flushing in the face to some
ruddiness, \etc{}

Waking, by reason of their continual cares, fears, sorrows, dry brains,
is a symptom that much crucifies melancholy men, and must therefore be
speedily helped, and sleep by all means procured, which sometimes is a
sufficient \authorfootnote{4349}remedy of itself without any other physic. Sckenkius,
in his observations, hath an example of a woman that was so cured. The
means to procure it, are inward or outward. Inwardly taken, are
simples, or compounds; simples, as poppy, nymphea, violets, roses,
lettuce, mandrake, henbane, nightshade or solanum, saffron, hemp-seed,
nutmegs, willows, with their seeds, juice, decoctions, distilled
waters, \etc{} Compounds are syrups, or opiates, syrup of poppy, violets,
verbasco, which are commonly taken with distilled waters.

\begin{Prescription}[H]
%\marginrecipe{Apothecaries` system:}[\baselineskip]
\begin{prescriptionbox}{}{\textlatin{mista fiat potio ad horam somni sumenda}}
\item \textlatin{diacodii ℥j},
\item \textlatin{diascordii ʒß},
\item \textlatin{aquae lactucae ℥iijß},
\end{prescriptionbox}
\begin{prescriptionbox}{FIXME translation}{\textlatin{mista fiat potio ad horam somni sumenda}}
\item \textlatin{diacodii ℥j},
\item \textlatin{diascordii ʒß},
\item \textlatin{aquae lactucae ℥iijß},
\end{prescriptionbox}
\caption{ recipe}
\end{Prescription}

Requies Nicholai, Philonium Romanum, Triphera magna, pilulae, de
Cynoglossa, Dioscordium, Laudanum Paracelsi, Opium, are in use, \etc{}
Country folks commonly make a posset of hemp-seed, which Fuchsius in
his herbal so much discommends; yet I have seen the good effect, and it
may be used where better medicines are not to be had.

Laudanum Paracelsi is prescribed in two or three grains, with a dram of
Diascordium, which Oswald. Crollius commends. Opium itself is most part
used outwardly, to smell to in a ball, though commonly so taken by the
Turks to the same quantity \authorfootnote{4350}for a cordial, and at Goa in, the
Indies; the dose 40 or 50 grains.

Rulandus calls Requiem Nicholai \lit{the last refuge}{ultimum refugium}; but
of this and the rest look for peculiar receipts in \textlatin{Victorius
Faventinus, cap. de phrensi. Heurnius cap. de mania. Hildesheim spicel.
4. de somno et vigil.} \etc{} Outwardly used, as oil of nutmegs by
extraction, or expression with rosewater to anoint the temples, oils of
poppy, nenuphar, mandrake, purslan, violets, all to the same purpose.

Montan. consil. 24 \& 25. much commends odoraments of opium, vinegar,
and rosewater. Laurentius cap. 9. prescribes pomanders and nodules; see
the receipts in him; Codronchus \authorfootnote{4351}wormwood to smell to.

\emph{Unguentum Alabastritum, populeum} are used to anoint the temples,
nostrils, or if they be too weak, they mix saffron and opium. Take a
grain or two of opium, and dissolve it with three or four drops of
rosewater in a spoon, and after mingle with it as much \emph{Unguentum
populeum} as a nut, use it as before: or else take half a dram of
opium, \emph{Unguentum populeum}, oil of nenuphar, rosewater, rose-vinegar,
of each half an ounce, with as much virgin wax as a nut, anoint your
temples with some of it, \li{ad horam somni}.

Sacks of wormwood, \authorfootnote{4352}mandrake, \authorfootnote{4353}henbane,
roses made like pillows and laid under the patient's head, are mentioned by
\authorfootnote{4354}Cardan and Mizaldus, to anoint the soles of the feet with
the fat of a dormouse, the teeth with ear wax of a dog, swine's gall, hare's
ears: charms, \etc{}

Frontlets are well known to every good wife, rosewater and vinegar,
with a little woman's milk, and nutmegs grated upon a rose-cake applied
to both temples.

For an emplaster, take of castorium a dram and a half, of opium half a
scruple, mixed both together with a little water of life, make two
small plasters thereof, and apply them to the temples.

Rulandus cent. 1. cur. 17. cent. 3. cur. 94. prescribes epithems and
lotions of the head, with the decoction of flowers of nymphea,
violet-leaves, mandrake roots, henbane, white poppy. Herc. de Saxonia,
\emph{stillicidia}, or droppings, \etc{} Lotions of the feet do much avail of
the said herbs: by these means, saith Laurentius, I think you may
procure sleep to the most melancholy man in the world. Some use
horseleeches behind the ears, and apply opium to the place.

Bayerus lib. 2. c. 13. sets down some remedies against fearful
dreams, and such as walk and talk in their sleep.\authorfootnote{4355} Baptista Porta Mag.
nat. l. 2. c. 6. to procure pleasant dreams and quiet rest, would have
you take \margindef{under the tongue. \theeditor{}}hippoglossa, or the herb horsetongue, balm, to use them or
their distilled waters after supper, \etc{} Such men must not eat beans,
peas, garlic, onions, cabbage, venison, hare, use black wines, or any
meat hard of digestion at supper, or lie on their backs, \etc{}
Rusticus pudor, bashfulness, flushing in the face, high colour,
ruddiness, are common grievances, which much torture many melancholy
men, when they meet a man, or come in \authorfootnote{4356}company of their betters,
strangers, after a meal, or if they drink a cup of wine or strong
drink, they are as red and fleet, and sweat as if they had been at a
mayor's feast, \li{praesertim si metus accesserit}, it exceeds, \authorfootnote{4357}they
think every man observes, takes notice of it: and fear alone will
effect it, suspicion without any other cause. Sckenkius observ. med.
lib. 1. speaks of a waiting gentlewoman in the Duke of Savoy's court,
that was so much offended with it, that she kneeled down to him, and
offered Biarus, a physician, all that she had to be cured of it. And
'tis most true, that \authorfootnote{4358}Antony Ludovicus saith in his book de
Pudore, bashfulness either hurts or helps, such men I am sure it hurts.

If it proceed from suspicion or fear, Felix Plater prescribes\authorfootnote{4359} no
other remedy but to reject and contemn it: \li{Id populus curat scilicet},
as a \authorfootnote{4360}worthy physician in our town said to a friend of mine in
like case, complaining without a cause, suppose one look red, what
matter is it, make light of it, who observes it?

If it trouble at or after meals, (as Jobertus observes\authorfootnote{4361} \textlatin{med.
pract. l. 1. c. 7.}) after a little exercise or stirring, for many are
then hot and red in the face, or if they do nothing at all, especially
women; he would have them let blood in both arms, first one, then
another, two or three days between, if blood abound; to use frictions
of the other parts, feet especially, and washing of them, because of
that consent which is between the head and the feet.\authorfootnote{4362}And withal to
refrigerate the face, by washing it often with rose, violet, nenuphar,
lettuce, lovage waters, and the like: but the best of all is that \li{lac
virginale}, or strained liquor of litargy: it is diversely prepared; by
Jobertus thus;
%
\vspace{-\baselineskip}
\begin{Prescription}[H]
\marginrecipe{Apothecaries` system:}[\baselineskip]
\begin{prescriptionbox}{}{}
\item \textlatin{lithar. argent. unc. j},
\item \textlatin{cerussae candidissimae, ʒjjj},
\item \textlatin{caphurae, ℈jj},
\item \textlatin{dissolvantur aquarum solani, lactucae, et nenupharis ana unc. jjj},
\item \textlatin{aceti vini albi. unc. jj},
\item \textlatin{aliquot horas resideat, deinde transmittatur per philt},
\item \textlatin{aqua servetur in vase vitreo, ac ea bis terve facies quotidie irroretur},
\end{prescriptionbox}
\begin{prescriptionbox}{FIXME translation}{}
\item \textlatin{lithar. argent. unc. j},
\item \textlatin{cerussae candidissimae, ʒjjj},
\item \textlatin{caphurae, ℈jj},
\item \textlatin{dissolvantur aquarum solani, lactucae, et nenupharis ana unc. jjj},
\item \textlatin{aceti vini albi. unc. jj},
\item \textlatin{aliquot horas resideat, deinde transmittatur per philt},
\item \textlatin{aqua servetur in vase vitreo, ac ea bis terve facies quotidie irroretur},
\end{prescriptionbox}
\caption{ recipe}
\end{Prescription}%

\authorfootnote{4363}\textlatin{Quercetan spagir. phar. cap. 6.}
commends the water of frog's spawn for ruddiness in the face.

\authorfootnote{4364}Crato \textlatin{consil. 283. Scoltzii} would fain have them use all summer
the condite flowers of succory, strawberry water, roses
(cupping-glasses are good for the time), consil. 285. et 286. and to
defecate impure blood with the infusion of senna, savory, balm water.

Hollerius knew one cured alone with the use of succory boiled,
and drunk for five months, every morning in the summer.\authorfootnote{4365} It is
good overnight to anoint the face with hare's blood,\authorfootnote{4366} and in the morning
to wash it with strawberry and cowslip water, the juice of distilled
lemons, juice of cucumbers, or to use the seeds of melons, or kernels
of peaches beaten small, or the roots of Aron, and mixed with wheat
bran to bake it in an oven, and to crumble it in strawberry water,
\authorfootnote{4367} or to put fresh cheese curds to a red face.

If it trouble them at meal times that flushing, as oft it doth, with
sweating or the like, they must avoid all violent passions and actions,
as laughing, \etc{}, strong drink, and drink very little, \authorfootnote{4368}one
draught, saith Crato, and that about the midst of their meal; avoid at
all times indurate salt, and especially spice and windy meat.

\authorfootnote{4369}Crato prescribes the condite fruit of wild rose, to a nobleman
his patient, to be taken before dinner or supper, to the quantity of a
chestnut. It is made of sugar, as that of quinces. The decoction of the
roots of sowthistle before meat, by the same author is much approved.

To eat of a baked apple some advice, or of a preserved quince,
cuminseed prepared with meat instead of salt, to keep down fumes: not
to study or to be intentive after meals.

\begin{Prescription}[H]
\marginrecipe{Apothecaries` system:}[\baselineskip]
\begin{prescriptionbox}{}{\textlatin{misce, utatur mane}}
\item \textlatin{Nucleorum persic. seminis melonum ana unc. ℈ß},
\item \textlatin{aquae fragrorum l. ij},
\end{prescriptionbox}
\begin{prescriptionbox}{FIXME translation}{\textlatin{misce, utatur mane}}
\item \textlatin{Nucleorum persic. seminis melonum ana unc. ℈ß},
\item \textlatin{aquae fragrorum l. ij},
\end{prescriptionbox}
\caption{ recipe}
\end{Prescription}

\authorfootnote{4370}To apply cupping glasses to the shoulders is very good. For the
other kind of ruddiness which is settled in the face with pimples, \etc{},
because it pertains not to my subject, I will not meddle with it. I
refer you to Crato's counsels, Arnoldus lib. 1. breviar. cap. 39. 1.
Rulande, Peter Forestus de Fuco, lib. 31. obser. 2. To Platerus,
Mercurialis, Ulmus, Rondoletius, Heurnius, Menadous, and others that
have written largely of it.

Those other grievances and symptoms of headache, palpitation of heart,
\li{Vertigo deliquium}, \etc{}, which trouble many melancholy men, because they
are copiously handled apart in every physician, I do voluntarily omit.

%MEMB. II.

\section{Cure of Melancholy over all the Body.}

\lettrine{W}{here} the melancholy blood possesseth the whole body with the brain,
\authorfootnote{4371} it is best to begin with bloodletting. The Greeks prescribe the
\authorfootnote{4372} median or middle vein to be opened, and so much blood to be
taken away as the patient may well spare, and the cut that is made must
be wide enough. The Arabians hold it fittest to be taken from that arm
on which side there is more pain and heaviness in the head: if black
blood issue forth, bleed on; if it be clear and good, let it be
instantly suppressed, \authorfootnote{4373} because the malice of melancholy is much
corrected by the goodness of the blood. If the party's strength will
not admit much evacuation in this kind at once, it must be assayed
again and again: if it may not be conveniently taken from the arm, it
must be taken from the knees and ankles, especially to such men or
women whose haemorrhoids or months have been stopped. \authorfootnote{4374} If the
malady continue, it is not amiss to evacuate in a part in the forehead,
and to virgins in the ankles, who are melancholy for love matters; so
to widows that are much grieved and troubled with sorrow and cares: for
bad blood flows in the heart, and so crucifies the mind. The
haemorrhoids are to be opened with an instrument or horseleeches, \etc{}
See more in Montaltus, cap. 29. \authorfootnote{4375}Sckenkius hath an example of one
that was cured by an accidental wound in his thigh, much bleeding freed
him from melancholy. Diet, diminutives, alteratives, cordials,
correctors as before, intermixed as occasion serves, \authorfootnote{4376}all their
study must be to make a melancholy man fat, and then the cure is ended.

\emph{Diuretics}, or medicines to procure urine, are prescribed by some in
this kind, hot and cold: hot where the heat of the liver doth not
forbid; cold where the heat of the liver is very great: \authorfootnote{4377}amongst
hot are parsley roots, lovage, fennel, \etc{}: cold, melon seeds, \etc{},
with whey of goat's milk, which is the common conveyer.

To purge and \authorfootnote{4378}purify the blood, use sowthistle, succory, senna,
endive, carduus benedictus, dandelion, hop, maidenhair, fumitory,
bugloss, borage, \etc{}, with their juice, decoctions, distilled waters,
syrups, \etc{}

Oswaldus, Crollius, basil Chym. much admires salt of corals in this
case, and Aetius, tetrabib. ser. 2. cap. 114. Hieram Archigenis, which
is an excellent medicine to purify the blood, for all melancholy
affections, falling sickness, none to be compared to it.

%MEMB. III.

%SUBSECT. I.-_Cure of Hypochondriacal Melancholy_.
\section{Cure of Hypochondriacal Melancholy.}

\lettrine{I}{n} this cure, as in the rest, is especially required the rectification
of those six non-natural things above all, as good diet, which
Montanus, consil. 27. enjoins a French nobleman, to have an especial
care of it, without which all other remedies are in vain. Bloodletting
is not to be used, except the patient's body be very full of blood, and
that it be derived from the liver and spleen to the stomach and his
vessels, then \authorfootnote{4379}to draw it back, to cut the inner vein of either
arm, some say the salvatella, and if the malady be continuate, \authorfootnote{4380}to
open a vein in the forehead.

Preparatives and alteratives may be used as before, saving that there
must be respect had as well to the liver, spleen, stomach,
hypochondries, as to the heart and brain. To comfort the \authorfootnote{4381}stomach
and inner parts against wind and obstructions, by Areteus, Galen,
Aetius, Aurelianus, \etc{}, and many latter writers, are still prescribed
the decoctions of wormwood, centaury, pennyroyal, betony sodden in
whey, and daily drunk: many have been cured by this medicine alone.

Prosper Altinus and some others as much magnify the water of Nile against this
malady, an especial good remedy for windy melancholy. For which reason belike
Ptolemeus Philadelphus, when he married his daughter Berenice to the king of
Assyria (as Celsus, lib. 2. records), \lit{to his great charge caused the water
of Nile to be carried with her}{magnis impensis Nili aquam afferri jussit}, and
gave command, that during her life she should use no other drink. I find those
that commend use of apples, in splenetic and this kind of melancholy
(lamb's-wool some call it), which howsoever approved, must certainly be
corrected of cold rawness and wind.

Codronchus in his book \textlatin{de sale absyn.} magnifies the oil and salt of
wormwood above all other remedies, \authorfootnote{4382}which works better and
speedier than any simple whatsoever, and much to be preferred before
all those fulsome decoctions and infusions, which must offend by reason
of their quantity; this alone in a small measure taken, expels wind,
and that most forcibly, moves urine, cleanseth the stomach of all gross
humours, crudities, helps appetite, \etc{} Arnoldus hath a wormwood wine
which he would have used, which every pharmacopoeia speaks of.

Diminutives and purges may \authorfootnote{4383}be taken as before, of hiera, manna,
cassia, which Montanus consil. 230. for an Italian abbot, in this kind
prefers before all other simples, \authorfootnote{4384}And these must be often used,
still abstaining from those which are more violent, lest they do
exasperate the stomach, \etc{}, and the mischief by that means be
increased. Though in some physicians I find very strong purgers,
hellebore itself prescribed in this affection. If it long continue,
vomits may be taken after meat, or otherwise gently procured with warm
water, oxymel, \etc{}, now and then. Fuchsius cap. 33. prescribes
hellebore; but still take heed in this malady, which I have often
warned, of hot medicines, \authorfootnote{4385}because (as Salvianus adds) drought
follows heat, which increaseth the disease: and yet Baptista Sylvaticus
controv. 32. forbids cold medicines, \authorfootnote{4386} because they increase
obstructions and other bad symptoms. But this varies as the parties do,
and 'tis not easy to determine which to use. \authorfootnote{4387}The stomach most
part in this infirmity is cold, the liver hot; scarce therefore (which
Montanus insinuates consil. 229. for the Earl of Manfort) can you help
the one and not hurt the other: much discretion must be used; take no
physic at all he concludes without great need. Laelius Aegubinus
consil. for an hypochondriacal German prince, used many medicines; but
it was after signified to him in \authorfootnote{4388}letters, that the decoction of
China and sassafras, and salt of sassafras wrought him an incredible
good. In his 108 consult, he used as happily the same remedies; this to
a third might have been poison, by overheating his liver and blood.

For the other parts look for remedies in Savanarola, Gordonius,
Massaria, Mercatus, Johnson, \etc{} One for the spleen, amongst many
other, I will not omit, cited by Hildesheim, spicel. 2, prescribed by
Mat. Flaccus, and out of the authority of Benevenius. Antony Benevenius
in a hypochondriacal passion, \authorfootnote{4389}cured an exceeding great swelling
of the spleen with capers alone, a meat befitting that infirmity, and
frequent use of the water of a smith's forge; by this physic he helped
a sick man, whom all other physicians had forsaken, that for seven
years had been splenetic. And of such force is this water, \authorfootnote{4390}that
those creatures as drink of it, have commonly little or no spleen. See
more excellent medicines for the spleen in him and \authorfootnote{4391}Lod. Mercatus,
who is a great magnifier of this medicine. This Chalybs praeparatus, or
steel-drink, is much likewise commended to this disease by Daniel
Sennertus l. 1. part. 2. cap. 12. and admired by J. Caesar Claudinus
Respons. 29. he calls steel the proper \authorfootnote{4392}alexipharmacum of this
malady, and much magnifies it; look for receipts in them. Averters must
be used to the liver and spleen, and to scour the mesaraic veins: and
they are either too open or provoke urine. You can open no place better
than the haemorrhoids, which if by horseleeches they be made to flow,
\authorfootnote{4393}there may be again such an excellent remedy, as Plater holds.

Sallust. Salvian will admit no other phlebotomy but this; and by his
experience in an hospital which he kept, he found all mad and
melancholy men worse for other bloodletting. Laurentius cap. 15. calls
this of horseleeches a sure remedy to empty the spleen and mesaraic
membrane. Only Montanus consil. 241. is against it; \authorfootnote{4394} to other men
(saith he) this opening of the haemorrhoids seems to be a profitable
remedy; for my part I do not approve of it, because it draws away the
thinnest blood, and leaves the thickest behind.

Aetius, Vidus Vidius, Mercurialis, Fuchsius, recommend diuretics, or
such things as provoke urine, as aniseeds, dill, fennel, germander,
ground pine, sodden in water, or drunk in powder: and yet \authorfootnote{4395}P.
Bayerus is against them: and so is Hollerius; All melancholy men (saith
he) must avoid such things as provoke urine, because by them the
subtile or thinnest is evacuated, the thicker matter remains.

Clysters are in good request. Trincavelius lib. 3. cap. 38. for a young
nobleman, esteems of them in the first place, and Hercules de Saxonia
Panth. lib. 1. cap. 16. is a great approver of them. \authorfootnote{4396}I have found
(saith he) by experience, that many hypochondriacal melancholy men have
been cured by the sole use of clysters, receipts are to be had in him.

Besides those fomentations, irrigations, inunctions, odoraments,
prescribed for the head, there must be the like used for the liver,
spleen, stomach, hypochondries, \etc{} \authorfootnote{4397}In crudity (saith Piso) 'tis
good to bind the stomach hard to hinder wind, and to help concoction.

Of inward medicines I need not speak; use the same cordials as before.

In this kind of melancholy, some prescribe \authorfootnote{4398}treacle in winter,
especially before or after purges, or in the spring, as \Avicenna{},
\authorfootnote{4399} \li{Trincavellius mithridate}, \authorfootnote{4400}Montaltus paeony seed, unicorn's
horn; \li{os de corde cervi}, \etc{}

Amongst topics or outward medicines, none are more precious than baths,
but of them I have spoken. Fomentations to the hypochondries are very
good, of wine and water in which are sodden southernwood, melilot,
epithyme, mugwort, senna, polypody, as also \authorfootnote{4401}cerotes,
\authorfootnote{4402}plaisters, liniments, ointments for the spleen, liver, and
hypochondries, of which look for examples in Laurentius, Jobertus lib.
3. c. pra. med. Montanus consil. 231. Montaltus cap. 33. Hercules de
Saxonia, Faventinus. And so of epithems, digestive powders, bags, oils,
Octavius Horatianus lib. 2. c. 5. prescribes calastic cataplasms, or
dry purging medicines; Piso \authorfootnote{4403}dropaces of pitch, and oil of rue,
applied at certain times to the stomach, to the metaphrene, or part of
the back which is over against the heart, Aetius sinapisms; Montaltus
cap. 35. would have the thighs to be \authorfootnote{4404}cauterised, Mercurialis
prescribes beneath the knees; Laelius Aegubinus consil. 77. for a
hypochondriacal Dutchman, will have the cautery made in the right
thigh, and so Montanus consil. 55. The same Montanus consil. 34.
approves of issues in the arms or hinder part of the head. Bernardus
Paternus in Hildesheim spicel 2. would have \authorfootnote{4405} issues made in both
the thighs; \authorfootnote{4406}Lod. Mercatus prescribes them near the spleen, aut
prope ventriculi regimen, or in either of the thighs. Ligatures,
frictions, and cupping-glasses above or about the belly, without
scarification, which \authorfootnote{4407}Felix Platerus so much approves, may be used
as before.

%SUBSECT. II.-_Correctors to expel Wind. Against Costiveness, \etc{}_
\section[To expel Wind. Against Constipation]{Correctors to expel Wind. Against Costiveness, \etc{}}

\lettrine{I}{n} this kind of melancholy one of the most offensive symptoms is wind,
which, as in the other species, so in this, hath great need to be
corrected and expelled.

The medicines to expel it are either inwardly taken, or outwardly.

Inwardly to expel wind, are simples or compounds: simples are herbs,
roots, \etc{}, as galanga, gentian, angelica, enula, calamus aromaticus,
valerian, zeodoti, iris, condite ginger, aristolochy, cicliminus,
China, dittander, pennyroyal, rue, calamint, bay-berries, and
bay-leaves, betony, rosemary, hyssop, sabine, centaury, mint, camomile,
staechas, agnus castus, broom-flowers, origan, orange-pills, \etc{};
spices, as saffron, cinnamon, bezoar stone, myrrh, mace, nutmegs,
pepper, cloves, ginger, seeds of annis, fennel, amni, cari, nettle,
rue, \etc{}, juniper berries, grana paradisi; compounds, dianisum,
diagalanga, diaciminum, diacalaminth, electuarium de baccis lauri,
benedicta laxativa, pulvis ad status. antid. florent. pulvis
carminativus, aromaticum rosatum, treacle, mithridate \etc{} This one
caution of \authorfootnote{4408}Gualter Bruell is to be observed in the administering
of these hot medicines and dry, that whilst they covet to expel wind,
they do not inflame the blood, and increase the disease; sometimes (as
he saith) medicines must more decline to heat, sometimes more to cold,
as the circumstances require, and as the parties are inclined to heat
or cold.

Outwardly taken to expel winds, are oils, as of camomile, rue, bays,
\etc{}; fomentations of the hypochondries, with the decoctions of dill,
pennyroyal, rue, bay leaves, cumin, \etc{}, bags of camomile flowers,
aniseed, cumin, bays, rue, wormwood, ointments of the oil of spikenard,
wormwood, rue, \etc{} \authorfootnote{4409}Areteus prescribes cataplasms of camomile
flowers, fennel, aniseeds, cumin, rosemary, wormwood-leaves, \etc{}
\authorfootnote{4410}Cupping-glasses applied to the hypochondries, without
scarification, do wonderfully resolve wind. Fernelius consil. 43. much
approves of them at the lower end of the belly; \authorfootnote{4411}Lod. Mercatus
calls them a powerful remedy, and testifies moreover out of his own
knowledge, how many he hath seen suddenly eased by them. Julius Caesar
Claudinus respons. med. resp. 33. admires these cupping-glasses, which
he calls out of Galen, \authorfootnote{4412}a kind of enchantment, they cause such
present help.

Empirics have a myriad of medicines, as to swallow a bullet of lead,
\etc{}, which I voluntarily omit. Amatus Lusitanus, cent. 4. curat. 54.
for a hypochondriacal person, that was extremely tormented with wind,
prescribes a strange remedy. Put a pair of bellows end into a clyster
pipe, and applying it into the fundament, open the bowels, so draw
forth the wind, \li{natura non admittit vacuum}. He vaunts he was the first
invented this remedy, and by means of it speedily eased a melancholy
man. Of the cure of this flatuous melancholy, read more in \textlatin{Fienus de
Flatibus, cap. 26. et passim alias.}

Against headache, vertigo, vapours which ascend forth of the stomach to
molest the head, read Hercules de Saxonia, and others.

If costiveness offend in this, or any other of the three species, it is
to be corrected with suppositories, clysters or lenitives, powder of
senna, condite prunes, \etc{}.

\begin{Prescription}[H]
\marginrecipe{Apothecaries` system:}[\baselineskip]
\begin{prescriptionbox}{}{\textlatin{misce}}
\item \textlatin{Elect. lenit, e succo rosar. ana ℥ j},
\end{prescriptionbox}
\begin{prescriptionbox}{FIXME translation}{\textlatin{misce}}
\item \textlatin{Elect. lenit, e succo rosar. ana ℥ j},
\end{prescriptionbox}
\caption{ recipe}
\end{Prescription}

Take as much as a nutmeg at a time, half an hour before dinner
or supper, or pil. mastichin. ℥ j. in six pills, a pill or two at a
time. See more in Montan. consil. 229. Hildesheim spicel. 2. P.
Cnemander, and Montanus commend \authorfootnote{4413}Cyprian turpentine, which they
would have familiarly taken, to the quantity of a small nut, two or
three hours before dinner and supper, twice or thrice a week if need
be; for besides that it keeps the belly soluble, it clears the stomach,
opens obstructions, cleanseth the liver, provokes urine.

These in brief are the ordinary medicines which belong to the cure of
melancholy, which if they be used aright, no doubt may do much good; \li{Si
non levando saltem leniendo valent, peculiaria bene selecta}, saith
Bessardus, a good choice of particular receipts must needs ease, if not
quite cure, not one, but all or most, as occasion serves. \li{Et quae non
prosunt singula, multa juvant.}
