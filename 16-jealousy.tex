\chapter{Jealousy}
{
%SECT. III. MEMB. I.

%SUBSECT. I.-_Jealousy, its Equivocations, Name, Definition, Extent, several kinds; of Princes, Parents, Friends. In Beasts, Men: before marriage, as Co-rivals; or after, as in this place_.
\section[Jealousy, its Equivocations, Name, Definition\ldots{}]{Jealousy, its Equivocations, Name, Definition, Extent, several kinds; of Princes, Parents, Friends. In Beasts, Men: before marriage, as Co-rivals; or after, as in this place.}

\lettrine{V}{alescus} de Taranta cap. de Melanchol. Aelian Montaltus, Felix
Platerus, Guianerius, put jealousy for a cause of melancholy, others
for a symptom; because melancholy persons amongst these passions and
perturbations of the mind, are most obnoxious to it. But methinks for
the latitude it hath, and that prerogative above other ordinary
symptoms, it ought to be treated of as a species apart, being of so
great and eminent note, so furious a passion, and almost of as great
extent as love itself, as Benedetto Varchi holds\authormarginnote{5982}, no love without
a mixture of jealousy, qui non zelat, non amat. For these causes I will
dilate, and treat of it by itself, as a bastard-branch or kind of
love-melancholy, which, as heroical love goeth commonly before
marriage, doth usually follow, torture, and crucify in like sort,
deserves therefore to be rectified alike, requires as much care and
industry, in setting out the several causes of it, prognostics and
cures. Which I have more willingly done, that he that is or hath been
jealous, may see his error as in a glass; he that is not, may learn to
detest, avoid it himself, and dispossess others that are anywise
affected with it.

Jealousy is described and defined to be \authorfootnote{5983}a certain suspicion which
the lover hath of the party he chiefly loveth, lest he or she should be
enamoured of another: or any eager desire to enjoy some beauty alone,
to have it proper to himself only: a fear or doubt, lest any foreigner
should participate or share with him in his love. Or (as \authorfootnote{5984}Scaliger
adds) a fear of losing her favour whom he so earnestly affects. Cardan
calls it a \authorfootnote{5985}zeal for love, and a kind of envy lest any man should
beguile us. \authorfootnote{5986}Ludovicus Vives defines it in the very same words, or
little differing in sense.

There be many other jealousies, but improperly so called all; as that
of parents, tutors, guardians over their children, friends whom they
love, or such as are left to their wardship or protection.
\authorfootnote{5987}Storax non rediit hac nocte a coena Aeschinus,
Neque servulorum quispiam qui adversum ierant?

As the old man in the comedy cried out in a passion, and from a
solicitous fear and care he had of his adopted son; \authorfootnote{5988}not of
beauty, but lest they should miscarry, do amiss, or any way discredit,
disgrace (as Vives notes) or endanger themselves and us. \authorfootnote{5989}Aegeus
was so solicitous for his son Theseus, (when he went to fight with the
Minotaur) of his success, lest he should be foiled, \authorfootnote{5990}Prona est
timori semper in pejus fides. We are still apt to suspect the worst in
such doubtful cases, as many wives in their husband's absence, fond
mothers in their children's, lest if absent they should be misled or
sick, and are continually expecting news from them, how they do fare,
and what is become of them, they cannot endure to have them long out of
their sight: oh my sweet son, O my dear child, \&c. Paul was jealous
over the Church of Corinth, as he confesseth, 2 Cor. xi. 12. With a
godly jealousy, to present them a pure virgin to Christ; and he was
afraid still, lest as the serpent beguiled Eve, through his subtlety,
so their minds should be corrupt from the simplicity that is in Christ.
God himself, in some sense, is said to be jealous, \authorfootnote{5991}I am a jealous
God, and will visit: so Psalm lxxix. 5. Shall thy jealousy burn like
fire for ever? But these are improperly called jealousies, and by a
metaphor, to show the care and solicitude they have of them. Although
some jealousies express all the symptoms of this which we treat of,
fear, sorrow, anguish, anxiety, suspicion, hatred, \&c., the object only
varied. That of some fathers is very eminent, to their sons and heirs;
for though they love them dearly being children, yet now coming towards
man's estate they may not well abide them, the son and heir is commonly
sick of the father, and the father again may not well brook his eldest
son, inde simultates, plerumque contentiones et inimicitiae; but that
of princes is most notorious, as when they fear co-rivals (if I may so
call them) successors, emulators, subjects, or such as they have
offended. \authorfootnote{5992} Omnisque potestas impatiens consortis erit: they are
still suspicious, lest their authority should be diminished, \authorfootnote{5993}as
one observes; and as Comineus hath it, \authorfootnote{5994}it cannot be expressed
what slender causes they have of their grief and suspicion, a secret
disease, that commonly lurks and breeds in princes' families. Sometimes
it is for their honour only, as that of Adrian the emperor, \authorfootnote{5995}that
killed all his emulators. Saul envied David; Domitian Agricola, because
he did excel him, obscure his honour, as he thought, eclipse his fame.
Juno turned Praetus' daughters into kine, for that they contended with
her for beauty; \authorfootnote{5996}Cyparissae, king Eteocles' children, were envied
of the goddesses for their excellent good parts, and dancing amongst
the rest, saith \authorfootnote{5997}Constantine, and for that cause flung headlong
from heaven, and buried in a pit, but the earth took pity of them, and
brought out cypress trees to preserve their memories. \authorfootnote{5998}Niobe,
Arachne, and Marsyas, can testify as much. But it is most grievous when
it is for a kingdom itself, or matters of commodity, it produceth
lamentable effects, especially amongst tyrants, in despotico Imperio,
and such as are more feared than beloved of their subjects, that get
and keep their sovereignty by force and fear. \authorfootnote{5999}Quod civibus tenere
te invitis scias, \&c., as Phalaris, Dionysius, Periander held theirs.
For though fear, cowardice, and jealousy, in Plutarch's opinion, be the
common causes of tyranny, as in Nero, Caligula, Tiberius, yet most take
them to be symptoms. For \authorfootnote{6000}what slave, what hangman (as Bodine well
expresseth this passion, l. 2. c. 5. de rep.) can so cruelly torture a
condemned person, as this fear and suspicion? Fear of death, infamy,
torments, are those furies and vultures that vex and disquiet tyrants,
and torture them day and night, with perpetual terrors and affrights,
envy, suspicion, fear, desire of revenge, and a thousand such
disagreeing perturbations, turn and affright the soul out of the hinges
of health, and more grievously wound and pierce, than those cruel
masters can exasperate and vex their apprentices or servants, with
clubs, whips, chains, and tortures. Many terrible examples we have in
this kind, amongst the Turks especially, many jealous outrages;
\authorfootnote{6001}Selimus killed Kornutus his youngest brother, five of his
nephews, Mustapha Bassa, and diverse others. \authorfootnote{6002}Bajazet the second
Turk, jealous of the valour and greatness of Achmet Bassa, caused him
to be slain. \authorfootnote{6003}Suleiman the Magnificent murdered his own son
Mustapha; and 'tis an ordinary thing amongst them, to make away their
brothers, or any competitors, at the first coming to the crown: 'tis
all the solemnity they use at their fathers' funerals. What mad pranks
in his jealous fury did Herod of old commit in Jewry, when he massacred
all the children of a year old? \authorfootnote{6004}Valens the emperor in
Constantinople, when as he left no man alive of quality in his kingdom
that had his name begun with Theo; Theodoti, Theognosti, Theodosii,
Theoduli, \&c. They went all to their long home, because a wizard told
him that name should succeed in his empire. And what furious designs
hath \authorfootnote{6005}Jo. Basilius, that Muscovian tyrant, practised of late? It
is a wonder to read that strange suspicion, which Suetonius reports of
Claudius Caesar, and of Domitian, they were afraid of every man they
saw: and which Herodian of Antoninus and Geta, those two jealous
brothers, the one could not endure so much as the other's servants, but
made away him, his chiefest followers, and all that belonged to him, or
were his well-wishers. \authorfootnote{6006}Maximinus perceiving himself to be odious
to most men, because he was come to that height of honour out of base
beginnings, and suspecting his mean parentage would be objected to him,
caused all the senators that were nobly descended, to be slain in a
jealous humour, turned all the servants of Alexander his predecessor
out of doors, and slew many of them, because they lamented their
master's death, suspecting them to be traitors, for the love they bare
to him. When Alexander in his fury had made Clitus his dear friend to
be put to death, and saw now (saith \authorfootnote{6007}Curtius) an alienation in his
subjects' hearts, none durst talk with him, he began to be jealous of
himself, lest they should attempt as much on him, and said they lived
like so many wild beasts in a wilderness, one afraid of another. Our
modern stories afford us many notable examples. \authorfootnote{6008}Henry the Third
of France, jealous of Henry of Lorraine, Duke of Guise, \emph{anno} 1588,
caused him to be murdered in his own chamber. \authorfootnote{6009}Louis the Eleventh
was so suspicious, he durst not trust his children, every man about him
he suspected for a traitor; many strange tricks Comineus telleth of
him. How jealous was our Henry the \authorfootnote{6010}Fourth of King Richard the
Second, so long as he lived, after he was deposed? and of his own son
Henry in his latter days? which the prince well perceiving, came to
visit his father in his sickness, in a watchet velvet gown, full of
eyelet holes, and with needles sticking in them (as an emblem of
jealousy), and so pacified his suspicious father, after some speeches
and protestations, which he had used to that purpose. Perpetual
imprisonment, as that of Robert \authorfootnote{6011}Duke of Normandy, in the days of
Henry the First, forbidding of marriage to some persons, with such like
edicts and prohibitions, are ordinary in all states. In a word
(\authorfootnote{6012}as he said) three things cause jealousy, a mighty state, a rich
treasure, a fair wife; or where there is a cracked title, much tyranny,
and exactions. In our state, as being freed from all these fears and
miseries, we may be most secure and happy under the reign of our
fortunate prince:

\begin{verse}
His fortune hath indebted him to none\\*
But to all his people universally;\\*
And not to them but for their love alone,\\*
Which they account as placed worthily.\\*
He is so set, he hath no cause to be\\*
Jealous, or dreadful of disloyalty;\\*
The pedestal whereon his greatness stands.\\*
Is held of all our hearts, and all our hands.
\end{verse}\authormarginnote{6013}

But I \margindef{wander}{rove}, I confess. These equivocations, jealousies, and many such,
which crucify the souls of men, are not here properly meant, or in this
distinction of ours included, but that alone which is for beauty,
tending to love, and wherein they can brook no co-rival, or endure any
participation: and this jealousy belongs as well to brute beasts, as
men. Some creatures, saith \authorfootnote{6014}Vives, swans, doves, cocks, bulls,
\&c., are jealous as well as men, and as much moved, for fear of
communion.

\authorfootnote{6015}Grege pro toto bella juvenci,
Si con jugio timuere suo,
Poscunt timidi praelia cervi,
Et mugitus dant concepti signa furoris.

In Venus' cause what mighty battles make
Your raving bulls, and stirs for their herd's sake:
And harts and bucks that are so timorous,
Will fight and roar, if once they be but jealous.

In bulls, horses, goats, this is most apparently discerned. Bulls
especially, alium in pascuis non admittit, he will not admit another
bull to feed in the same pasture, saith \authorfootnote{6016}Oppin: which Stephanus
Bathorius, late king of Poland, used as an impress, with that motto,
Regnum non capit duos. R. T. in his Blazon of Jealousy, telleth a story
of a swan about Windsor, that finding a strange cock with his mate, did
swim I know not how many miles after to kill him, and when he had so
done, came back and killed his hen; a certain truth, he saith, done
upon Thames, as many watermen, and neighbour gentlemen, can tell. Fidem
suam liberet; for my part, I do believe it may be true; for swans have
ever been branded with that epithet of jealousy.
%
{\gothfont%
\begin{versewithlinenos}{2}{1}{1}%
The jealous swanne against his death that singeth,\\*
And eke the owle that of death bode bringeth.\\!
\end{versewithlinenos}%
}%
\attrib{\getauthornote{6017} [Lines FIXME. \theeditor{}]}
%
\authorfootnote{6018}Some say as much of elephants, that they are more jealous than
any other creatures whatsoever; and those old Egyptians, as
\authorfootnote{6019}Pierius informeth us, express in their hieroglyphics, the passion
of jealousy by a camel; \authorfootnote{6020}because that fearing the worst still
about matters of venery, he loves solitudes, that he may enjoy his
pleasure alone, et in quoscunque obvios insurgit, Zelolypiae stimulis
agitatus, he will quarrel and fight with whatsoever comes next, man or
beast, in his jealous fits. I have read as much of \authorfootnote{6021}crocodiles;
and if Peter Martyr's authority be authentic, legat. Babylonicae lib.
3. you shall have a strange tale to that purpose confidently related.
Another story of the jealousy of dogs, see in Hieron. Fabricius, Tract.
3. cap. 5. de loquela animalium.

But this furious passion is most eminent in men, and is as well amongst
bachelors as married men. If it appear amongst bachelors, we commonly
call them rivals or co-rivals, a metaphor derived from a river,
rivales, a \authorfootnote{6022}rivo; for as a river, saith Acron in Hor. Art. Poet.
and Donat in Ter. Eunuch. divides a common ground between two men, and
both participate of it, so is a woman indifferent between two suitors,
both likely to enjoy her; and thence comes this emulation, which breaks
out many times into tempestuous storms, and produceth lamentable
effects, murder itself, with much cruelty, many single combats. They
cannot endure the least injury done unto them before their mistress,
and in her defence will bite off one another's noses; they are most
impatient of any flout, disgrace, lest emulation or participation in
that kind. \authorfootnote{6023}Lacerat lacerium Largi mordax Memnius. Memnius the
Roman (as Tully tells the story, de oratore, lib. 2.), being co-rival
with Largus Terracina, bit him by the arm, which fact of his was so
famous, that it afterwards grew to a proverb in those parts.

\authorfootnote{6024}Phaedria could not abide his co-rival Thraso; for when Parmeno
demanded, numquid aliud imperas? whether he would command him any more
service: No more (saith he) but to speak in his behalf, and to drive
away his co-rival if he could. Constantine, in the eleventh book of his
husbandry, cap. 11, hath a pleasant tale of the pine-tree; \authorfootnote{6025}she
was once a fair maid, whom Pineus and Boreas, two co-rivals, dearly
sought; but jealous Boreas broke her neck, \&c. And in his eighteenth
chapter he telleth another tale of \authorfootnote{6026}Mars, that in his jealousy
slew Adonis. Petronius calleth this passion amantium furiosum
aemulationem, a furious emulation; and their symptoms are well
expressed by Sir Geoffrey Chaucer in his first Canterbury Tale. It will
make the nearest and dearest friends fall out; they will endure all
other things to be common, goods, lands, moneys, participate of each
pleasure, and take in good part any disgraces, injuries in another
kind; but as Propertius well describes it in an elegy of his, in this
they will suffer nothing, have no co-rivals.

\begin{latin}
\begin{verse}
Tu mihi vel ferro pectus, vel perde veneno,\\*
A domina tantum te modo tolle mea:\\!

Te socium vitae te corporis esse licebit,\\*
Te dominum admitto rebus amice meis.\\!

Lecto te solum, lecto te deprecor uno,\\*
Rivalem possum non ego ferre Jovem.
\end{verse}
\end{latin}
\authorfootnote{6027}

\begin{verse}
Stab me with sword, or poison strong\\*
Give me to work my bane:\\!

So thou court not my lass, so thou\\*
From mistress mine refrain.\\!

Command myself, my body, purse,\\*
As thine own goods take all,\\!

And as my ever dearest friend,\\*
I ever use thee shall.\\!

O spare my love, to have alone\\*
Her to myself I crave,\\!

Nay, Jove himself I'll not endure\\*
My rival for to have.\\!
\end{verse}

This jealousy, which I am to treat of, is that which belongs to married
men, in respect of their own wives; to whose estate, as no sweetness,
pleasure, happiness can be compared in the world, if they live quietly
and lovingly together; so if they disagree or be jealous, those bitter
pills of sorrow and grief, disastrous mischiefs, mischances, tortures,
gripings, discontents, are not to be separated from them. A most
violent passion it is where it taketh place, an unspeakable torment, a
hellish torture, an infernal plague, as Ariosto calls it, a fury, a
continual fever, full of suspicion, fear, and sorrow, a martyrdom, a
mirth-marring monster. The sorrow and grief of heart of one woman
jealous of another, is heavier than death, Ecclus. \rn{xxviii.} 6. as
\authorfootnote{6028}Peninnah did Hannah, vex her and upbraid her sore. 'Tis a main
vexation, a most intolerable burden, a corrosive to all content, a
frenzy, a madness itself; as \authorfootnote{6029}Beneditto Varchi proves out of that
select sonnet of Giovanni de la Casa, that reverend lord, as he styles
him.

%SUBSECT. II.-_Causes of Jealousy. Who are most apt. Idleness, melancholy, impotency, long absence, beauty, wantonness, naught themselves. Allurements, from time, place, persons, bad usage, causes_.
\section[Causes of Jealousy]{Causes of Jealousy. Who are most apt. Idleness, melancholy, impotency, long absence, beauty, wantonness, naught themselves. Allurements, from time, place, persons, bad usage, causes.}

\lettrine{A}{strologers} make the stars a cause or sign of this bitter passion, and
out of every man's horoscope will give a probable conjecture whether he
will be jealous or no, and at what time, by direction of the
significators to their several promissors: their aphorisms are to be
read in Albubater, Pontanus, Schoner, Junctine, \&c. Bodine, cap. 5.
meth. hist. ascribes a great cause to the country or clime, and
discourseth largely there of this subject, saying, that southern men
are more hot, lascivious, and jealous, than such as live in the north;
they can hardly contain themselves in those hotter climes, but are most
subject to prodigious lust. Leo Afer telleth incredible things almost,
of the lust and jealousy of his countrymen of Africa, and especially
such as live about Carthage, and so doth every geographer of them in
\authorfootnote{6030}Asia, Turkey, Spaniards, Italians. Germany hath not so many
drunkards, England tobacconists, France dancers, Holland mariners, as
Italy alone hath jealous husbands. And in \authorfootnote{6031}Italy some account them
of Piacenza more jealous than the rest. In \authorfootnote{6032}Germany, France,
Britain, Scandia, Poland, Muscovy, they are not so troubled with this
feral malady, although Damianus a Goes, which I do much wonder at, in
his topography of Lapland, and Herbastein of Russia, against the stream
of all other geographers, would fasten it upon those northern
inhabitants. Altomarius Poggius, and Munster in his description of
Baden, reports that men and women of all sorts go commonly into the
baths together, without all suspicion, the name of jealousy (saith
Munster) is not so much as once heard of among them. In Friesland the
women kiss him they drink to, and are kissed again of those they
pledge. The virgins in Holland go hand in hand with young men from
home, glide on the ice, such is their harmless liberty, and lodge
together abroad without suspicion, which rash Sansovinus an Italian
makes a great sign of unchastity. In France, upon small acquaintance,
it is usual to court other men's wives, to come to their houses, and
accompany them arm in arm in the streets, without imputation. In the
most northern countries young men and maids familiarly dance together,
men and their wives, \authorfootnote{6033}which, Siena only excepted, Italians may not
abide. The \authorfootnote{6034}Greeks, on the other side, have their private baths
for men and women, where they must not come near, nor so much as see
one another: and as \authorfootnote{6035}Bodine observes lib. 5. de repub. the
Italians could never endure this, or a Spaniard, the very conceit of it
would make him mad: and for that cause they lock up their women, and
will not suffer them to be near men, so much as in the \authorfootnote{6036}church,
but with a partition between. He telleth, moreover, how that when he
was ambassador in England, he heard Mendoza the Spanish legate finding
fault with it, as a filthy custom for men and women to sit
promiscuously in churches together; but Dr. Dale the master of the
requests told him again, that it was indeed a filthy custom in Spain,
where they could not contain themselves from lascivious thoughts in
their holy places, but not with us. Baronius in his Annals, out of
Eusebius, taxeth Licinius the emperor for a decree of his made to this
effect, Jubens ne viri simul cum mulieribus in ecclesia interessent:
for being prodigiously naught himself, aliorum naturam ex sua vitiosa
mente spectavit, he so esteemed others. But we are far from any such
strange conceits, and will permit our wives and daughters to go to the
tavern with a friend, as Aubanus saith, modo absit lascivia, and
suspect nothing, to kiss coming and going, which, as Erasmus writes in
one of his epistles, they cannot endure. England is a paradise for
women, and hell for horses: Italy a paradise for horses, hell for
women, as the diverb goes. Some make a question whether this headstrong
passion rage more in women than men, as Montaigne l. 3. But sure it is
more outrageous in women, as all other melancholy is, by reason of the
weakness of their sex. Scaliger Poet. lib. cap. 13. concludes against
women: \authorfootnote{6037}Besides their inconstancy, treachery, suspicion,
dissimulation, superstition, pride, (for all women are by nature proud)
desire of sovereignty, if they be great women, (he gives instance in
Juno) bitterness and jealousy are the most remarkable affections.

Sed neque fulvus aper media tam fulvus in ira est,
Fulmineo rapidos dum rotat ore canes.
Nec leo, \&c.---

Tiger, boar, bear, viper, lioness,
A woman's fury cannot express.

\authorfootnote{6038}Some say redheaded women, pale-coloured, black-eyed, and of a
shrill voice, are most subject to jealousy.
\authorfootnote{6039}High colour in a woman choler shows,
Naught are they, peevish, proud, malicious;
But worst of all, red, shrill, and jealous.

Comparisons are odious, I neither parallel them with others, nor debase
them any more: men and women are both bad, and too subject to this
pernicious infirmity. It is most part a symptom and cause of
melancholy, as Plater and Valescus teach us: melancholy men are apt to
be jealous, and jealous apt to be melancholy.

\begin{verse}
Pale jealousy, child of insatiate love,\\*
Of heart-sick thoughts which melancholy bred,\\*
A hell-tormenting fear, no faith can move,\\*
By discontent with deadly poison fed;\\*
With heedless youth and error vainly led.\\*
A mortal plague, a virtue-drowning flood,\\*
A hellish fire not quenched but with blood.
\end{verse}

If idleness concur with melancholy, such persons are most apt to be
jealous; 'tis \authorfootnote{6040}Nevisanus' note, an idle woman is presumed to be
lascivious, and often jealous. Mulier cum sola cogitat, male cogitat:
and 'tis not unlikely, for they have no other business to trouble their
heads with.

More particular causes be these which follow. Impotency first, when a
man is not able of himself to perform those dues which he ought unto
his wife: for though he be an honest liver, hurt no man, yet Trebius
the lawyer may make a question, an suum cuique tribuat, whether he give
every one their own; and therefore when he takes notice of his wants,
and perceives her to be more craving, clamorous, insatiable and prone
to lust than is fit, he begins presently to suspect, that wherein he is
defective, she will satisfy herself, she will be pleased by some other
means. Cornelius Gallus hath elegantly expressed this humour in an
epigram to his Lychoris.

\authorfootnote{6041}Jamque alios juvenes aliosque requirit amores,
Me vocat imbellem decrepitumque senem, \&c.

For this cause is most evident in old men, that are cold and dry by
nature, and married, succi plenis, to young wanton wives; with old
doting Janivere in Chaucer, they begin to mistrust all is not well,
%
{\gothfont%
\begin{versewithlinenos}{2}{1}{1}%
---She was young and he was old,\\*
And therefore he feared to be a cuckold.\\!
\end{versewithlinenos}%
}%
\attrib{[Lines FIXME. \theeditor{}]}
%
And how should it otherwise be? old age is a disease of itself,
loathsome, full of suspicion and fear; when it is at best, unable,
unfit for such matters. \authorfootnote{6042}Tam apta nuptiis quam bruma messibus, as
welcome to a young woman as snow in harvest, saith Nevisanus: Et si
capis juvenculam, faciet tibi cornua: marry a lusty maid and she will
surely graft horns on thy head. \authorfootnote{6043}All women are slippery, often
unfaithful to their husbands (as Aeneas Sylvius epist. 38. seconds
him), but to old men most treacherous: they had rather mortem
amplexarier, lie with a corse than such a one: \authorfootnote{6044}Oderunt illum
pueri, contemnunt mulieres. On the other side many men, saith
Hieronymus, are suspicious of their wives, \authorfootnote{6045}if they be lightly
given, but old folks above the rest. Insomuch that she did not complain
without a cause in \authorfootnote{6046}Apuleius, of an old bald bedridden knave she
had to her good man: Poor woman as I am, what shall I do? I have an old
grim sire to my husband, as bald as a coot, as little and as unable as
a child, a bedful of bones, he keeps all the doors barred and locked
upon me, woe is me, what shall I do? He was jealous, and she made him a
cuckold for keeping her up: suspicion without a cause, hard usage is
able of itself to make a woman fly out, that was otherwise honest,
\authorfootnote{6047}---plerasque bonas tractatio pravas
Esse facit,---

bad usage aggravates the matter. Nam quando mulieres cognoscunt maritum
hoc advertere, licentius peccant, \authorfootnote{6048}as Nevisanus holds, when a
woman thinks her husband watcheth her, she will sooner offend;
\authorfootnote{6049}Liberius peccant, et pudor omnis abest, rough handling makes them
worse: as the goodwife of Bath in Chaucer brags,
%
{\gothfont%
\begin{versewithlinenos}{2}{1}{1}%
In his own grease I made him frie\\*
For anger and for every jealousie.\\!
\end{versewithlinenos}%
}%
\attrib{[Lines FIXME. \theeditor{}]}
%
Of two extremes, this of hard usage is the worst. 'Tis a great fault
(for some men are uxorii) to be too fond of their wives, to dote on
them as \authorfootnote{6050}Senior Deliro on his Fallace, to be too effeminate, or as
some do, to be sick for their wives, breed children for them, and like
the \authorfootnote{6051} Tiberini lie in for them, as some birds hatch eggs by turns,
they do all women's offices: Caelius Rhodiginus ant. lect. Lib. 6. cap.
24. makes mention of a fellow out of Seneca, \authorfootnote{6052}that was so besotted
on his wife, he could not endure a moment out of her company, he wore
her scarf when he went abroad next his heart, and would never drink but
in that cup she began first. We have many such fondlings that are their
wives' packhorses and slaves, (nam grave malum uxor superans virum
suum, as the comical poet hath it, there's no greater misery to a man
than to let his wife domineer) to carry her muff, dog, and fan, let her
wear the breeches, lay out, spend, and do what she will, go and come
whither, when she will, they give consent.
Here, take my muff, and, do you hear, good man;
Now give me pearl, and carry you my fan, \&c.

\authorfootnote{6053}---poscit pallam, redimicula, inaures;
Curre, quid hic cessas? vulgo vult illa videri,
Tu pete lecticas---

many brave and worthy men have trespassed in this kind, multos foras
claros domestica haec destruxit infamia, and many noble senators and
soldiers (as \authorfootnote{6054}Pliny notes) have lost their honour, in being
uxorii, so sottishly overruled by their wives; and therefore Cato in
Plutarch made a bitter jest on his fellow-citizens, the Romans, we
govern all the world abroad, and our wives at home rule us. These
offend in one extreme; but too hard and too severe, are far more
offensive on the other. As just a cause may be long absence of either
party, when they must of necessity be much from home, as lawyers,
physicians, mariners, by their professions; or otherwise make
frivolous, impertinent journeys, tarry long abroad to no purpose, lie
out, and are gadding still, upon small occasions, it must needs yield
matter of suspicion, when they use their wives unkindly in the
meantime, and never tarry at home, it cannot use but engender some such
conceit.

\authorfootnote{6055}Uxor si cessas amare te cogitat
Aut tote amari, aut potare, aut animo obsequi,
Ex tibi bene esse soli, quum sibi sit male.

\begin{verse}
If thou be absent long, thy wife then thinks,\\*
Th' art drunk, at ease, or with some pretty minx,\\*
'Tis well with thee, or else beloved of some,\\*
Whilst she poor soul doth fare full ill at home.
\end{verse}

Hippocrates, the physician, had a smack of this disease; for when he
was to go home as far as Abdera, and some other remote cities of
Greece, he writ to his friend Dionysius (if at least those
\authorfootnote{6056}Epistles be his) \authorfootnote{6057} to oversee his wife in his absence, (as
Apollo set a raven to watch his Coronis) although she lived in his
house with her father and mother, who be knew would have a care of her;
yet that would not satisfy his jealousy, he would have his special
friend Dionysius to dwell in his house with her all the time of his
peregrination, and to observe her behaviour, how she carried herself in
her husband's absence, and that she did not lust after other men.

\authorfootnote{6058}For a woman had need to have an overseer to keep her honest; they
are bad by nature, and lightly given all, and if they be not curbed in
time, as an unpruned tree, they will be full of wild branches, and
degenerate of a sudden. Especially in their husband's absence: though
one Lucretia were trusty, and one Penelope, yet Clytemnestra made
Agamemnon cuckold; and no question there be too many of her conditions.

If their husbands tarry too long abroad upon unnecessary business, well
they may suspect: or if they run one way, their wives at home will fly
out another, quid pro quo. Or if present, and give them not that
content which they ought, \authorfootnote{6059}Primum ingratae, mox invisae noctes
quae per somnum transiguntur, they cannot endure to lie alone, or to
fast long. \authorfootnote{6060} Peter Godefridus, in his second book of Love, and
sixth chapter, hath a story out of St. Anthony's life, of a gentleman,
who, by that good man's advice, would not meddle with his wife in the
passion week, but for his pains she set a pair of horns on his head.

Such another he hath out of Abstemius, one persuaded a new married man,
\authorfootnote{6061}to forbear the three first nights, and he should all his lifetime
after be fortunate in cattle, but his impatient wife would not tarry so
long: well he might speed in cattle, but not in children. Such a tale
hath Heinsius of an impotent and slack scholar, a mere student, and a
friend of his, that seeing by chance a fine damsel sing and dance,
would needs marry her, the match was soon made, for he was young and
rich, genis gratus, corpore glabellus, arte multiscius, et fortuna
opulentus, like that Apollo in \authorfootnote{6062}Apuleius. The first night, having
liberally taken his liquor (as in that country they do) my fine scholar
was so fuzzled, that he no sooner was laid in bed, but he fell fast
asleep, never waked till morning, and then much abashed, purpureis
formosa rosis cum Aurora ruberet; when the fair morn with purple hue
'gan shine, he made an excuse, I know not what, out of Hippocrates
Cous, \&c., and for that time it went current: but when as afterward he
did not play the man as he should do, she fell in league with a good
fellow, and whilst he sat up late at his study about those criticisms,
mending some hard places in Festus or Pollux, came cold to bed, and
would tell her still what he had done, she did not much regard what he
said, \&c. \authorfootnote{6063}She would have another matter mended much rather, which
he did not conceive was corrupt: thus he continued at his study late,
she at her sport, alibi enim festivas noctes agitabat, hating all
scholars for his sake, till at length he began to suspect, and turned a
little yellow, as well he might; for it was his own fault; and if men
be jealous in such cases (\authorfootnote{6064}as oft it falls out) the mends is in
their own hands, they must thank themselves. Who will pity them, saith
Neander, or be much offended with such wives, si deceptae prius viros
decipiant, et cornutos reddant, if they deceive those that cozened them
first. A lawyer's wife in \authorfootnote{6065}Aristaenetus, because her husband was
negligent in his business, quando lecto danda opera, threatened to
cornute him: and did not stick to tell Philinna, one of her gossips, as
much, and that aloud for him to hear: If he follow other men's matters
and leave his own, I'll have an orator shall plead my cause, I care not
if he know it.

A fourth eminent cause of jealousy may be this, when he that is
deformed, and as Pindarus of Vulcan, sine gratiis natus, hirsute,
ragged, yet virtuously given, will marry some fair nice piece, or light
housewife, begins to misdoubt (as well he may) she doth not affect him.

\authorfootnote{6066}Lis est cum forma magna pudicitiae, beauty and honesty have ever
been at odds. Abraham was jealous of his wife because she was fair: so
was Vulcan of his Venus, when he made her creaking shoes, saith
\authorfootnote{6067}Philostratus, ne maecharetur, sandalio scilicet deferente, that
he might hear by them when she stirred, which Mars indigne ferre,
\authorfootnote{6068}was not well pleased with. Good cause had Vulcan to do as he did,
for she was no honester than she should be. Your fine faces have
commonly this fault; and it is hard to find, saith Francis Philelphus
in an epistle to Saxola his friend, a rich man honest, a proper woman
not proud or unchaste. Can she be fair and honest too?
\authorfootnote{6069}Saepe etenim oculuit picta sese hydra sub herba,
Sub specie formae, incauto se saepe marito
Nequam animus vendit,---

He that marries a wife that is snowy fair alone, let him look, saith
\authorfootnote{6070} Barbarus, for no better success than Vulcan had with Venus, or
Claudius with Messalina. And 'tis impossible almost in such cases the
wife should contain, or the good man not be jealous: for when he is so
defective, weak, ill-proportioned, unpleasing in those parts which
women most affect, and she most absolutely fair and able on the other
side, if she be not very virtuously given, how can she love him? and
although she be not fair, yet if he admire her and think her so, in his
conceit she is absolute, he holds it impossible for any man living not
to dote as he doth, to look on her and not lust, not to covet, and if
he be in company with her, not to lay siege to her honesty: or else out
of a deep apprehension of his infirmities, deformities, and other men's
good parts, out of his own little worth and desert, he distrusts
himself, (for what is jealousy but distrust?) he suspects she cannot
affect him, or be not so kind and loving as she should, she certainly
loves some other man better than himself.

\authorfootnote{6071}Nevisanus, lib. 4. num. 72, will have barrenness to be a main
cause of jealousy. If her husband cannot play the man, some other
shall, they will leave no remedies unessayed, and thereupon the good
man grows jealous; I could give an instance, but be it as it is.

I find this reason given by some men, because they have been formerly
naught themselves, they think they may be so served by others, they
turned up trump before the cards were shuffled; they shall have
therefore legem talionis, like for like.

\authorfootnote{6072}Ipse miser docui, quo posset ludere pacto
Custodes, eheu nunc premor arte mea.

Wretch as I was, I taught her bad to be,
And now mine own sly tricks are put upon me.

Mala mens, malus animus, as the saying is, ill dispositions cause ill
suspicions.

\authorfootnote{6073}There is none jealous, I durst pawn my life,
But he that hath defiled another's wife,
And for that he himself hath gone astray,
He straightway thinks his wife will tread that way.

To these two above-named causes, or incendiaries of this rage, I may
very well annex those circumstances of time, place, persons, by which
it ebbs and flows, the fuel of this fury, as \authorfootnote{6074}Vives truly
observes; and such like accidents or occasions, proceeding from the
parties themselves, or others, which much aggravate and intend this
suspicious humour. For many men are so lasciviously given, either out
of a depraved nature, or too much liberty, which they do assume unto
themselves, by reason of their greatness, in that they are noble men,
(for licentiae peccandi, et multitudo peccantium are great motives)
though their own wives be never so fair, noble, virtuous, honest, wise,
able, and well given, they must have change.

\authorfootnote{6075}Qui cum legitimi junguntur fccdere lecti,
Virtute egregiis, facieque domoque puellis,
Scorta tamen, foedasque lupas in fornice quaerunt,
Et per adulterium nova carpere gaudia tentant.

Who being match'd to wives most virtuous,
Noble, and fair, fly out lascivious.

Quod licet ingratum est, that which is ordinary, is unpleasant. Nero
(saith Tacitus) abhorred Octavia his own wife, a noble virtuous lady,
and loved Acte, a base quean in respect. \authorfootnote{6076}Cerinthus rejected
Sulpitia, a nobleman's daughter, and courted a poor servant maid.-tanta
est aliena in messe voluptas, for that \authorfootnote{6077}stolen waters be more
pleasant: or as Vitellius the emperor was wont to say, Jucundiores
amores, qui cum periculo habentur, like stolen venison, still the
sweetest is that love which is most difficultly attained: they like
better to hunt by stealth in another man's walk, than to have the
fairest course that may be at game of their own.

\authorfootnote{6078}Aspice ut in coelo modo sol, modo luna ministret,
Sic etiam nobis una pella parum est.

As sun and moon in heaven change their course,
So they change loves, though often to the worse.

Or that some fair object so forcibly moves them, they cannot contain
themselves, be it heard or seen they will be at it. \authorfootnote{6079}Nessus, the
centaur, was by agreement to carry Hercules and his wife over the river
Evenus; no sooner had he set Dejanira on the other side, but he would
have offered violence unto her, leaving Hercules to swim over as he
could: and though her husband was a spectator, yet would he not desist
till Hercules, with a poisoned arrow, shot him to death. \authorfootnote{6080}Neptune
saw by chance that Thessalian Tyro, Eunippius' wife, he forthwith, in
the fury of his lust, counterfeited her husband's habit, and made him
cuckold. Tarquin heard Collatine commend his wife, and was so far
enraged, that in the midst of the night to her he went. \authorfootnote{6081}Theseus
stole Ariadne, vi rapuit that Trazenian Anaxa, Antiope, and now being
old, Helen, a girl not yet ready for a husband. Great men are most part
thus affected all, as a horse they neigh, saith \authorfootnote{6082}Jeremiah, after
their neighbours' wives,-ut visa pullus adhinnit equa: and if they be
in company with other women, though in their own wives' presence, they
must be courting and dallying with them. Juno in Lucian complains of
Jupiter that he was still kissing Ganymede before her face, which did
not a little offend her: and besides he was a counterfeit Amphitryo, a
bull, a swan, a golden shower, and played many such bad pranks, too
long, too shameful to relate.

Or that they care little for their own ladies, and fear no laws, they
dare freely keep whores at their wives' noses. 'Tis too frequent with
noblemen to be dishonest; Pielas, probitas, fides, privata bona sunt,
as \authorfootnote{6083}he said long since, piety, chastity, and such like virtues are
for private men: not to be much looked after in great courts: and which
Suetonius of the good princes of his time, they might be all engraven
in one ring, we may truly hold of chaste potentates of our age. For
great personages will familiarly run out in this kind, and yield
occasion of offence. \authorfootnote{6084} Montaigne, in his Essays, gives instate in
Caesar, Mahomet the Turk, that sacked Constantinople, and Ladislaus,
king of Naples, that besieged Florence: great men, and great soldiers,
are commonly great, \&c., probatum est, they are good doers. Mars and
Venus are equally balanced in their actions,
\authorfootnote{6085}Militis in galea nidum fecere columbae,
Apparet Marti quam sit amica Venus.

A dove within a headpiece made her nest,
'Twixt Mars and Venus see an interest.

Especially if they be bald, for bald men have ever been suspicious
(read more in Aristotle, Sect. 4. prob. 19.) as Galba, Otho, Domitian,
and remarkable Caesar amongst the rest. \authorfootnote{6086}Urbani servate uxores,
maechum calvum adducimus; besides, this bald Caesar, saith Curio in
Sueton, was omnium mulierum vir; he made love to Eunoe, queen of
Mauritania; to Cleopatra; to Posthumia, wife to Sergius Sulpitius; to
Lollia, wife to Gabinius; to Tertulla, of Crassus; to Mutia, Pompey's
wife, and I know not how many besides: and well he might, for, if all
be true that I have read, he had a license to lie with whom he list.

Inter alios honores Caesari decretos (as Sueton, cap. 52. de Julio, and
Dion, lib. 44. relate) jus illi datum, cum quibuscunque faeminis se
jungendi. Every private history will yield such variety of instances:
otherwise good, wise, discreet men, virtuous and valiant, but too
faulty in this. Priamus had fifty sons, but seventeen alone lawfully
begotten. \authorfootnote{6087}Philippus Bonus left fourteen bastards. Lorenzo de
Medici, a good prince and a wise, but, saith Machiavel,
\authorfootnote{6088}prodigiously lascivious. None so valiant as Castruccius
Castrucanus, but, as the said author hath it, \authorfootnote{6089}none so incontinent
as he was. And 'tis not only predominant in grandees this fault: but if
you will take a great man's testimony, 'tis familiar with every base
soldier in France, (and elsewhere, I think). This vice (\authorfootnote{6090} saith
mine author) is so common with us in France, that he is of no account,
a mere coward, not worthy the name of a soldier, that is not a
notorious whoremaster. In Italy he is not a gentleman, that besides his
wife hath not a courtesan and a mistress. 'Tis no marvel, then, if poor
women in such cases be jealous, when they shall see themselves
manifestly neglected, contemned, loathed, unkindly used: their disloyal
husbands to entertain others in their rooms, and many times to court
ladies to their faces: other men's wives to wear their jewels: how
shall a poor woman in such a case moderate her passion? \authorfootnote{6091}Quis tibi
nunc Dido cernenti talia sensus?

How, on the other side, shall a poor man contain himself from this
feral malady, when he shall see so manifest signs of his wife's
inconstancy? when, as Milo's wife, she dotes upon every young man she
sees, or, as \authorfootnote{6092}Martial's Sota,-deserto sequitur Clitum marito,
deserts her husband and follows Clitus. Though her husband be proper
and tall, fair and lovely to behold, able to give contentment to any
one woman, yet she will taste of the forbidden fruit: Juvenal's Iberina
to a hair, she is as well pleased with one eye as one man. If a young
gallant come by chance into her presence, a fastidious brisk, that can
wear his clothes well in fashion, with a lock, jingling spur, a
feather, that can cringe, and withal compliment, court a gentlewoman,
she raves upon him, O what a lovely proper man he was, another Hector,
an Alexander, a goodly man, a demigod, how sweetly he carried himself,
with how comely a grace, sic oculos, sic ille manus, sic ora ferebat,
how neatly he did wear his clothes! \authorfootnote{6093} Quam sese ore ferens, quam
forti pectore et armis, how bravely did he discourse, ride, sing, and
dance, \&c., and then she begins to loathe her husband, repugnans
osculatur, to hate him and his filthy beard, his goatish complexion, as
Doris said of Polyphemus, \authorfootnote{6094}totus qui saniem, totus ut hircus olet,
he is a rammy fulsome fellow, a goblin-faced fellow, he smells, he
stinks, Et caepas simul alliumque ructat \authorfootnote{6095}-si quando ad thalamum,
\&c., how like a dizzard, a fool, an ass, he looks, how like a clown he
behaves himself! \authorfootnote{6096}she will not come near him by her own good will,
but wholly rejects him, as Venus did her fuliginous Vulcan, at last,
Nec Deus hunc mensa, Dea nec dignata cubili est. \authorfootnote{6097}So did Lucretia,
a lady of Senae, after she had but seen Euryalus, in Eurialum tota
ferebatur, domum reversa, \&c., she would not hold her eyes off him in
his presence,- \authorfootnote{6098}tantum egregio decus enitet ore, and in his
absence could think of none but him, odit virum, she loathed her
husband forthwith, might not abide him:
\authorfootnote{6099}Et conjugalis negligens tori, viro
Praesente, acerbo nauseat fastidio;

All against the laws of matrimony,
She did abhor her husband's phis'nomy;

and sought all opportunity to see her sweetheart again. Now when the
good man shall observe his wife so lightly given, to be so free and
familiar with every gallant, her immodesty and wantonness, (as
\authorfootnote{6100}Camerarius notes) it must needs yield matter of suspicion to him,
when she still pranks up herself beyond her means and fortunes, makes
impertinent journeys, unnecessary visitations, stays out so long, with
such and such companions, so frequently goes to plays, masks, feasts,
and all public meetings, shall use such immodest \authorfootnote{6101}gestures, free
speeches, and withal show some distaste of her own husband; how can he
choose, though he were another Socrates, but be suspicious, and
instantly jealous? \authorfootnote{6102} Socraticas tandem faciet transcendere metas;
more especially when he shall take notice of their more secret and sly
tricks, which to cornute their husbands they commonly use (dum ludis,
ludos haec te facit) they pretend love, honour, chastity, and seem to
respect them before all men living, saints in show, so cunningly can
they dissemble, they will not so much as look upon another man in his
presence, \authorfootnote{6103}so chaste, so religious, and so devout, they cannot
endure the name or sight of a quean, a harlot, out upon her! and in
their outward carriage are most loving and officious, will kiss their
husband, and hang about his neck (dear husband, sweet husband), and
with a composed countenance salute him, especially when he comes home;
or if he go from home, weep, sigh, lament, and take upon them to be
sick and swoon (like Jocundo's wife in \authorfootnote{6104}Ariosto, when her husband
was to depart), and yet arrant, \&c. they care not for him,
Aye me, the thought (quoth she) makes me so 'fraid,
That scarce the breath abideth in my breast;
Peace, my sweet love and wife, Jocundo said,
And weeps as fast, and comforts her his best, \&c.
All this might not assuage the woman's pain,
Needs must I die before you come again,
Nor how to keep my life I can devise,
The doleful days and nights I shall sustain,
From meat my mouth, from sleep will keep mine eyes, \&c.
That very night that went before the morrow,
That he had pointed surely to depart,
Jocundo's wife was sick, and swoon'd for sorrow
Amid his arms, so heavy was her heart.

And yet for all these counterfeit tears and protestations, Jocundo
coming back in all haste for a jewel he had forgot,
His chaste and yoke-fellow he found
Yok'd with a knave, all honesty neglected,
The adulterer sleeping very sound,
Yet by his face was easily detected:
A beggar's brat bred by him from his cradle.,
And now was riding on his master's saddle.

Thus can they cunningly counterfeit, as \authorfootnote{6105}Platina describes their
customs, kiss their husbands, whom they had rather see hanging on a
gallows, and swear they love him dearer than their own lives, whose
soul they would not ransom for their little dog's,
---similis si permutatio detur,
Morte viri cupiunt aniniani servare catellae.

Many of them seem to be precise and holy forsooth, and will go to such
a \authorfootnote{6106}church, to hear such a good man by all means, an excellent man,
when 'tis for no other intent (as he follows it) than to see and to be
seen, to observe what fashions are in use, to meet some pander, bawd,
monk, friar, or to entice some good fellow. For they persuade
themselves, as \authorfootnote{6107} Nevisanus shows, That it is neither sin nor shame
to lie with a lord or parish priest, if he be a proper man; \authorfootnote{6108}and
though she kneel often, and pray devoutly, 'tis (saith Platina) not for
her husband's welfare, or children's good, or any friend, but for her
sweetheart's return, her pander's health. If her husband would have her
go, she feigns herself sick, \authorfootnote{6109}Et simulat subito condoluisse caput:
her head aches, and she cannot stir: but if her paramour ask as much,
she is for him in all seasons, at all hours of the night. \authorfootnote{6110}In the
kingdom of Malabar, and about Goa in the East Indies, the women are so
subtile that, with a certain drink they give them to drive away cares
as they say, \authorfootnote{6111}they will make them sleep for twenty-four hours, or
so intoxicate them that they can remember nought of that they saw done,
or heard, and, by washing of their feet, restore them again, and so
make their husbands cuckolds to their faces. Some are ill-disposed at
all times, to all persons they like, others more wary to some few, at
such and such seasons, as Augusta, Livia, non nisi plena navi vectorem
tollebat. But as he said,
\authorfootnote{6112}No pen could write, no tongue attain to tell,
By force of eloquence, or help of art,
Of women's treacheries the hundredth part.

Both, to say truth, are often faulty; men and women give just occasions
in this humour of discontent, aggravate and yield matter of suspicion:
but most part of the chief causes proceed from other adventitious
accidents and circumstances, though the parties be free, and both well
given themselves. The indiscreet carriage of some lascivious gallant
(et e contra of some light woman) by his often frequenting of a house,
bold unseemly gestures, may make a breach, and by his over-familiarity,
if he be inclined to yellowness, colour him quite out. If he be poor,
basely born, saith Beneditto Varchi, and otherwise unhandsome, he
suspects him the less; but if a proper man, such as was Alcibiades in
Greece, and Castruccius Castrucanus in Italy, well descended,
commendable for his good parts, he taketh on the more, and watcheth his
doings. \authorfootnote{6113}Theodosius the emperor gave his wife Eudoxia a golden
apple when he was a suitor to her, which she long after bestowed upon a
young gallant in the court, of her especial acquaintance. The emperor,
espying this apple in his hand, suspected forthwith, more than was, his
wife's dishonesty, banished him the court, and from that day following
forbare to accompany her any more. \authorfootnote{6114}A rich merchant had a fair
wife; according to his custom he went to travel; in his absence a good
fellow tempted his wife; she denied him; yet he, dying a little after,
gave her a legacy for the love he bore her. At his return, her jealous
husband, because she had got more by land than he had done at sea,
turned her away upon suspicion.

Now when those other circumstances of time and place, opportunity and
importunity shall concur, what will they not effect?
Fair opportunity can win the coyest she that is,
So wisely he takes time, as he'll be sure he will not miss:
Then lie that loves her gamesome vein, and tempers toys with art,
Brings love that swimmeth in her eyes to dive into her heart.

As at plays, masks, great feasts and banquets, one singles out his wife
to dance, another courts her in his presence, a third tempts her, a
fourth insinuates with a pleasing compliment, a sweet smile,
ingratiates himself with an amphibological speech, as that merry
companion in the \authorfootnote{6115} Satirist did to his Glycerium, \authorfootnote{6116}adsidens
et interiorem palmam amabiliter concutiens,
Quod meus hortus habet sumat impune licebit,
Si dederis nobis quod tuus hortus habet;

with many such, \&c., and then as he saith,
\authorfootnote{6117}She may no while in chastity abide.
That is assaid on every side.

For after al great feast, \authorfootnote{6118}Vino saepe suum nescit amica virum.
Noah (saith \authorfootnote{6119}Hierome) showed his nakedness in his drunkenness,
which for six hundred years he had covered in soberness. Lot lay with
his daughters in his drink, as Cyneras with Myrrha,-\authorfootnote{6120}quid enim
Venus ebria curat? The most continent may be overcome, or if otherwise
they keep bad company, they that are modest of themselves, and dare not
offend, confirmed by \authorfootnote{6121}others, grow impudent, and confident, and
get an ill habit.
\authorfootnote{6122}Alia quaestus gratia matrimonium corrumpit,
Alia peccans multas vult morbi habere socias.

Or if they dwell in suspected places, as in an infamous inn, near some
stews, near monks, friars, Nevisanus adds, where be many tempters and
solicitors, idle persons that frequent their companies, it may give
just cause of suspicion. Martial of old inveighed against them that
counterfeited a disease to go to the bath; for so, many times,
---relicto
Conjuge Penelope venit, abit Helene.

Aeneas Sylvius puts in a caveat against princes' courts, because there
be tot formosi juvenes qui promittunt, so many brave suitors to tempt,
\&c. \authorfootnote{6123}If you leave her in such a place, you shall likely find her
in company you like not, either they come to her, or she is gone to
them. \authorfootnote{6124}Kornmannus makes a doubting jest in his lascivious country,
Virginis illibata censeatur ne castitas ad quam frequentur accedant
scholares? And Baldus the lawyer scoffs on, quum scholaris, inquit,
loquitur cum puella, non praesumitur ei dicere, Pater noster, when a
scholar talks with a maid, or another man's wife in private, it is
presumed he saith not a pater noster. Or if I shall see a monk or a
friar climb up a ladder at midnight into a virgin's or widow's chamber
window, I shall hardly think he then goes to administer the sacraments,
or to take her confession. These are the ordinary causes of jealousy,
which are intended or remitted as the circumstances vary.

MEMB. II.

\section[Symptoms of Jealousy]{Symptoms of Jealousy, Fear, Sorrow, Suspicion, strange Actions, Gestures, Outrages, Locking up, Oaths, Trials, Laws, \&c.}

\lettrine{O}{f} all passions, as I have already proved, love is most violent, and of
those bitter potions which this love-melancholy affords, this bastard
jealousy is the greatest, as appears by those prodigious symptoms which
it hath, and that it produceth. For besides fear and sorrow, which is
common to all melancholy, anxiety of mind, suspicion, aggravation,
restless thoughts, paleness, meagreness, neglect of business, and the
like, these men are farther yet misaffected, and in a higher strain.

'Tis a more vehement passion, a more furious perturbation, a bitter
pain, a fire, a pernicious curiosity, a gall corrupting the honey of
our life, madness, vertigo, plague, hell, they are more than ordinarily
disquieted, they lose bonum pacis, as \authorfootnote{6125}Chrysostom observes; and
though they be rich, keep sumptuous tables, be nobly allied, yet
miserrimi omnium sunt, they are most miserable, they are more than
ordinarily discontent, more sad, nihil tristius, more than ordinarily
suspicious. Jealousy, saith \authorfootnote{6126}Vives, begets unquietness in the
mind, night and day: he hunts after every word he hears, every whisper,
and amplifies it to himself (as all melancholy men do in other matters)
with a most unjust calumny of others, he misinterprets everything is
said or done, most apt to mistake or misconstrue, he pries into every
corner, follows close, observes to a hair. 'Tis proper to jealousy so
to do,
Pale hag, infernal fury, pleasure's smart,
Envy's observer, prying in every part.

Besides those strange gestures of staring, frowning, grinning, rolling
of eyes, menacing, ghastly looks, broken pace, interrupt, precipitate,
half-turns. He will sometimes sigh, weep, sob for anger. Nempe suos
imbres etiam ista tonitrua fundunt,\authorfootnote{6127}-swear and belie, slander any
man, curse, threaten, brawl, scold, fight; and sometimes again flatter
and speak fair, ask forgiveness, kiss and coll, condemn his rashness
and folly, vow, protest, and swear he will never do so again; and then
eftsoons, impatient as he is, rave, roar, and lay about him like a
madman, thump her sides, drag her about perchance, drive her out of
doors, send her home, he will be divorced forthwith, she is a whore,
\&c., and by-and-by with all submission compliment, entreat her fair,
and bring her in again, he loves her dearly, she is his sweet, most
kind and loving wife, he will not change, nor leave her for a kingdom;
so he continues off and on, as the toy takes him, the object moves him,
but most part brawling, fretting, unquiet he is, accusing and
suspecting not strangers only, but brothers and sisters, father and
mother, nearest and dearest friends. He thinks with those Italians,
Chi non tocca parentado,
Tocca mai e rado.

And through fear conceives unto himself things almost incredible and
impossible to be effected. As a heron when she fishes, still prying on
all sides; or as a cat doth a mouse, his eye is never off hers; he
gloats on him, on her, accurately observing on whom she looks, who
looks at her, what she saith, doth, at dinner, at supper, sitting,
walking, at home, abroad, he is the same, still inquiring, maundering,
gazing, listening, affrighted with every small object; why did she
smile, why did she pity him, commend him? why did she drink twice to
such a man? why did she offer to kiss, to dance? \&c., a whore, a whore,
an arrant whore. All this he confesseth in the poet,
\authorfootnote{6128}Omnia me terrent, timidus sum, ignosce timori.
Et miser in tunica suspicor esse virum.
Me laedit si multa tibi dabit oscula mater,
Me soror, et cum qua dormit amica simul.

Each thing affrights me, I do fear,
Ah pardon me my fear,
I doubt a man is hid within
The clothes that thou dost wear.

Is it not a man in woman's apparel? is not somebody in that great
chest, or behind the door, or hangings, or in some of those barrels?
may not a man steal in at the window with a ladder of ropes, or come
down the chimney, have a false key, or get in when he is asleep? If a
mouse do but stir, or the wind blow, a casement clatter, that's the
villain, there he is: by his goodwill no man shall see her, salute her,
speak with her, she shall not go forth of his sight, so much as to do
her needs. \authorfootnote{6129}Non ita bovem argus, \&c. Argus did not so keep his
cow, that watchful dragon the golden fleece, or Cerberus the coming in
of hell, as he keeps his wife. If a dear friend or near kinsman come as
guest to his house, to visit him, he will never let him be out of his
own sight and company, lest, peradventure, \&c. If the necessity of his
business be such that he must go from home, he doth either lock her up,
or commit her with a deal of injunctions and protestations to some
trusty friends, him and her he sets and bribes to oversee: one servant
is set in his absence to watch another, and all to observe his wife,
and yet all this will not serve, though his business be very urgent, he
will when he is halfway come back in all post haste, rise from supper,
or at midnight, and be gone, and sometimes leave his business undone,
and as a stranger court his own wife in some disguised habit. Though
there be no danger at all, no cause of suspicion, she live in such a
place, where Messalina herself could not be dishonest if she would, yet
he suspects her as much as if she were in a bawdy-house, some prince's
court, or in a common inn, where all comers might have free access. He
calls her on a sudden all to nought, she is a strumpet, a light
housewife, a bitch, an arrant whore. No persuasion, no protestation can
divert this passion, nothing can ease him, secure or give him
satisfaction. It is most strange to report what outrageous acts by men
and women have been committed in this kind, by women especially, that
will run after their husbands into all places and companies, \authorfootnote{6130}as
Jovianus Pontanus's wife did by him, follow him whithersoever he went,
it matters not, or upon what business, raving like Juno in the tragedy,
miscalling, cursing, swearing, and mistrusting every one she sees.

Gomesius in his third book of the Life and Deeds of Francis Ximenius,
sometime archbishop of Toledo, hath a strange story of that incredible
jealousy of Joan queen of Spain, wife to King Philip, mother of
Ferdinand and Charles the Fifth, emperors; when her husband Philip,
either for that he was tired with his wife's jealousy, or had some
great business, went into the Low Countries: she was so impatient and
melancholy upon his departure, that she would scarce eat her meat, or
converse with any man; and though she were with child, the season of
the year very bad, the wind against her, in all haste she would to sea
after him. Neither Isabella her queen mother, the archbishop, or any
other friend could persuade her to the contrary, but she would after
him. When she was now come into the Low Countries, and kindly
entertained by her husband, she could not contain herself, \authorfootnote{6131}but in
a rage ran upon a yellow-haired wench, with whom she suspected her
husband to be naught, cut off her hair, did beat her black and blue,
and so dragged her about. It is an ordinary thing for women in such
cases to scratch the faces, slit the noses of such as they suspect; as
Henry the Second's importune Juno did by Rosamond at Woodstock; for she
complains in a \authorfootnote{6132}modern poet, she scarce spake,
But flies with eager fury to my face,
Offering me most unwomanly disgrace.
Look how a tigress, \&c.
So fell she on me in outrageous wise,
As could disdain and jealousy devise.

Or if it be so they dare not or cannot execute any such tyrannical
injustice, they will miscall, rail and revile, bear them deadly hate
and malice, as \authorfootnote{6133}Tacitus observes, The hatred of a jealous woman is
inseparable against such as she suspects.
\authorfootnote{6134}Nulla vis flammae tumidique venti
Tanta, nec teli metuanda torti.
Quanta cum conjux viduata taedis
Ardet et odit.

Winds, weapons, flames make not such hurly burly,
As raving women turn all topsy-turvy.

So did Agrippina by Lollia, and Calphurnia in the days of Claudius. But
women are sufficiently curbed in such cases, the rage of men is more
eminent, and frequently put in practice. See but with what rigour those
jealous husbands tyrannise over their poor wives. In Greece, Spain,
Italy, Turkey, Africa, Asia, and generally over all those hot
countries, \authorfootnote{6135} Mulieres vestrae terra vestra, arate sicut vultis.

Mahomet in his Alcoran gives this power to men, your wives are as your
land, till them, use them, entreat them fair or foul, as you will
yourselves. \authorfootnote{6136}Mecastor lege dura vivunt mulieres, they lock them
still in their houses, which are so many prisons to them. will suffer
nobody to come at them, or their wives to be seen abroad,-nec campos
liceat lustrare patentes. They must not so much as look out. And if
they be great persons, they have eunuchs to keep them, as the Grand
Signior among the Turks, the Sophies of Persia, those Tartarian Mogors,
and Kings of China. Infantes masculos castrant innumeros ut regi
serviant, saith \authorfootnote{6137}Riccius, they geld innumerable infants to this
purpose; the King of \authorfootnote{6138}China maintains 10,000 eunuchs in his family
to keep his wives. The Xeriffes of Barbary keep their courtesans in
such a strict manner, that if any man come but in sight of them he dies
for it; and if they chance to see a man, and do not instantly cry out,
though from their windows, they must be put to death. The Turks have I
know not how many black, deformed eunuchs (for the white serve for
other ministeries) to this purpose sent commonly from Egypt, deprived
in their childhood of all their privities, and brought up in the
seraglio at Constantinople to keep their wives; which are so penned up
they may not confer with any living man, or converse with younger
women, have a cucumber or carrot sent into them for their diet, but
sliced, for fear, \&c. and so live and are left alone to their unchaste
thoughts all the days of their lives. The vulgar sort of women, if at
any time they come abroad, which is very seldom, to visit one another,
or to go to their baths, are so covered, that no man can see them, as
the matrons were in old Rome, lectica aut sella tecta, vectae, so
\authorfootnote{6139}Dion and Seneca record, Velatae totae incedunt, which
\authorfootnote{6140}Alexander ab Alexandro relates of the Parthians, lib. 5. cap. 24.
which, with Andreas Tiraquellus his commentator, I rather think should
be understood of Persians. I have not yet said all, they do not only
lock them up, sed et pudendis seras adhibent: hear what Bembus relates
lib. 6. of his Venetian history, of those inhabitants that dwell about
Quilon in Africa. Lusitani, inquit, quorundum civitates adierunt: qui
natis statim faeminis naturam consuunt, quoad urinae exitus ne
impediatur, easque quum adoleverint sic consutas in matrimonium
collocant, ut sponsi prima cura sit conglutinatas puellae oras ferro
interscindere. In some parts of Greece at this day, like those old
Jews, they will not believe their wives are honest, nisi pannum
menstruatum prima nocte videant: our countryman \authorfootnote{6141}Sands, in his
peregrination, saith it is severely observed in Zanzynthus, or Zante;
and Leo Afer in his time at Fez, in Africa, non credunt virginem esse
nisi videant sanguineam mappam; si non, ad parentes pudore rejicitur.

Those sheets are publicly shown by their parents, and kept as a sign of
incorrupt virginity. The Jews of old examined their maids ex tenui
membrana, called Hymen, which Laurentius in his anatomy, Columbus lib.
12. cap. 10. Capivaccius lib. 4. cap. 11. de uteri affectibus, Vincent,
Alsarus Genuensis quaesit. med. cent. 4. Hieronymus Mercurialis
consult. Ambros. Pareus, Julius Caesar Claudinus Respons. 4. as that
also de \authorfootnote{6142}ruptura venarum ut sauguis fluat, copiously confute; 'tis
no sufficient trial they contend. And yet others again defend it,
Gaspar Bartholinus Institut. Anat. lib. 1. cap. 31. Pinaeus of Paris,
Albertus Magnus de secret. mulier. cap. 9 \& 10. \&c. and think they
speak too much in favour of women. \authorfootnote{6143} Ludovicus Boncialus lib. 4.
cap. 2. muliebr. naturalem illam uteri labiorum constrictionem, in qua
virginitatem consistere volunt, astringentibus medicinis fieri posse
vendicat, et si defloratae sint, astutae \authorfootnote{6144}mulieres (inquit) nos
fallunt in his. Idem Alsarius Crucius Genuensis iisdem fere verbis.
Idem Avicenna lib. 3. Fen. 20. Tract. 1, cap. 47. \authorfootnote{6145}Rhasis
Continent. lib. 24. Rodericus a Castro de nat. mul. lib. 1. cap. 3. An
old bawdy nurse in \authorfootnote{6146}Aristaenetus, (like that Spanish Caelestina,
\authorfootnote{6147}quae, quinque mille virgines fecit mulieres, totidemque mulieres
arte sua virgines) when a fair maid of her acquaintance wept and made
her moan to her, how she had been deflowered, and now ready to be
married, was afraid it would be perceived, comfortably replied, Noli
vereri filia, \&c. Fear not, daughter, I'll teach thee a trick to help
it. Sed haec extra callem. To what end are all those astrological
questions, an sit virgo, an sit casta, an sit mulier? and such strange
absurd trials in Albertus Magnus, Bap. Porta, Mag. lib. 2. cap. 21. in
Wecker. lib. 5. de secret, by stones, perfumes, to make them piss, and
confess I know not what in their sleep; some jealous brain was the
first founder of them. And to what passion may we ascribe those severe
laws against jealousy, Num. v. 14, Adulterers Deut. cap. 22. v. \rn{xxii.}
as amongst the Hebrews, amongst the Egyptians (read \authorfootnote{6148}Bohemus l. 1.
c. 5. de mor. gen. of the Carthaginians, cap. 6. of Turks, lib. 2. cap.
11.) amongst the Athenians of old, Italians at this day, wherein they
are to be severely punished, cut in pieces, burned, vivi-comburio,
buried alive, with several expurgations, \&c. are they not as so many
symptoms of incredible jealousy? we may say the same of those vestal
virgins that fetched water in a sieve, as Tatia did in Rome, \emph{anno ab.
urb. condita 800.} before the senators; and \authorfootnote{6149}Aemilia, virgo
innocens, that ran over hot irons, as Emma, Edward the Confessor's
mother did, the king himself being a spectator, with the like. We read
in Nicephorus, that Chunegunda the wife of Henricus Bavarus emperor,
suspected of adultery, insimulata adulterii per ignitos vomeres illaesa
transiit, trod upon red hot coulters, and had no harm: such another
story we find in Regino lib. 2. In Aventinus and Sigonius of Charles
the Third and his wife Richarda, \emph{an.} 887, that was so purged with hot
irons. Pausanias saith, that he was once an eyewitness of such a
miracle at Diana's temple, a maid without any harm at all walked upon
burning coals. Pius Secund. in his description of Europe, c. 46.
relates as much, that it was commonly practised at Diana's temple, for
women to go barefoot over hot coals, to try their honesties: Plinius,
Solinus, and many writers, make mention of \authorfootnote{6150}Geronia's temple, and
Dionysius Halicarnassus, lib. 3. of Memnon's statue, which were used to
this purpose. Tatius lib. 6. of Pan his cave, (much like old St.
Wilfrid's needle in Yorkshire) wherein they did use to try, maids,
\authorfootnote{6151}whether they were honest; when Leucippe went in, suavissimus
exaudiri sonus caepit Austin de civ. Dei lib. 10. c. 16. relates many
such examples, all which Lavater de spectr. part. 1. cap. 19 contends
to be done by the illusion of devils; though Thomas quaest. 6. de
polentia, \&c. ascribes it to good angels. Some, saith \authorfootnote{6152}Austin,
compel their wives to swear they be honest, as if perjury were a lesser
sin than adultery; \authorfootnote{6153}some consult oracles, as Phaerus that blind
king of Egypt. Others reward, as those old Romans used to do; if a
woman were contented with one man, Corona pudicitiae donabatur, she had
a crown of chastity bestowed on her. When all this will not serve,
saith Alexander Gaguinus, cap. 5. descript. Muscoviae, the Muscovites,
if they suspect their wives, will beat them till they confess, and if
that will not avail, like those wild Irish, be divorced at their
pleasures, or else knock them on the heads, as the old Gauls have
done in former ages.\authormarginnote{6154} Of this tyranny of jealousy read more in
Parthenius Erot. cap. 10. Camerarius cap. 53. hor. subcis. et cent. 2.
cap. 34. Caelia's epistles, Tho. Chaloner de repub. Aug. lib. 9.
Ariosto lib. 31. stasse 1. Felix Platerus observat. lib. 1. \&c.

MEMB. III.

\section[Prognostics of Jealousy.]{Prognostics of Jealousy. Despair, Madness, to make away themselves and others.}

\lettrine{T}{hose} which are jealous, most part, if they be not otherwise relieved,
\authorfootnote{6155}proceed from suspicion to hatred, from hatred to frenzy, madness,
injury, murder and despair.
\authorfootnote{6156}A plague by whose most damnable effect.
Diverse in deep despair to die have sought,
By which a man to madness near is brought,
As well with causeless as with just suspect.

In their madness many times, saith \authorfootnote{6157}Vives, they make away
themselves and others. Which induceth Cyprian to call it, Foecundam et
multiplicem perniciem, fontem cladium et seminarium delictorum, a
fruitful mischief, the seminary of offences, and fountain of murders.

Tragical examples are too common in this kind, both new and old, in all
ages, as of \authorfootnote{6158} Cephalus and Procris, \authorfootnote{6159}Phaereus of Egypt,
Tereus, Atreus, and Thyestes. \authorfootnote{6160}Alexander Phaereus was murdered of
his wife, ob pellicatus suspitionem, Tully saith. Antoninus Verus was
so made away by Lucilla; Demetrius the son of Antigonus, and Nicanor,
by their wives. Hercules poisoned by Dejanira, \authorfootnote{6161}Caecinna murdered
by Vespasian, Justina, a Roman lady, by her husband. \authorfootnote{6162}Amestris,
Xerxes' wife, because she found her husband's cloak in Masista's house,
cut off Masista, his wife's paps, and gave them to the dogs, flayed her
besides, and cut off her ears, lips, tongue, and slit the nose of
Artaynta her daughter. Our late writers are full of such outrages.

\authorfootnote{6163}Paulus Aemilius, in his history of France, hath a tragical story
of Chilpericus the First his death, made away by Ferdegunde his queen.

In a jealous humour he came from hunting, and stole behind his wife, as
she was dressing and combing her head in the sun, gave her a familiar
touch with his wand, which she mistaking for her lover, said, Ah
Landre, a good knight should strike before, and not behind: but when
she saw herself betrayed by his presence, she instantly took order to
make him away. Hierome Osorius, in his eleventh book of the deeds of
Emanuel King of Portugal, to this effect hath a tragical narration of
one Ferdinandus Chalderia, that wounded Gotherinus, a noble countryman
of his, at Goa in the East Indies, \authorfootnote{6164}and cut off one of his legs,
for that he looked as he thought too familiarly upon his wife, which
was afterwards a cause of many quarrels, and much bloodshed. Guianerius
cap. 36. de aegritud. matr. speaks of a silly jealous fellow, that
seeing his child new-born included in a caul, thought sure a
\authorfootnote{6165}Franciscan that used to come to his house, was the father of it,
it was so like the friar's cowl, and thereupon threatened the friar to
kill him: Fulgosus of a woman in Narbonne, that cut off her husband's
privities in the night, because she thought he played false with her.

The story of Jonuses Bassa, and fair Manto his wife, is well known to
such as have read the Turkish history; and that of Joan of Spain, of
which I treated in my former section. Her jealousy, saith Gomesius, was
the cause of both their deaths: King Philip died for grief a little
after, as \authorfootnote{6166}Martian his physician gave it out, and she for her part
after a melancholy discontented life, misspent in lurking-holes and
corners, made an end of her miseries. Felix Plater, in the first book
of his observations, hath many such instances, of a physician of his
acquaintance, \authorfootnote{6167}that was first mad through jealousy, and afterwards
desperate: of a merchant \authorfootnote{6168}that killed his wife in the same humour,
and after precipitated himself: of a doctor of law that cut off his
man's nose: of a painter's wife in Basil, anno 1600, that was mother of
nine children and had been twenty-seven years married, yet afterwards
jealous, and so impatient that she became desperate, and would neither
eat nor drink in her own house, for fear her husband should poison her.

'Tis a common sign this; for when once the humours are stirred, and the
imagination misaffected, it will vary itself in diverse forms; and many
such absurd symptoms will accompany, even madness itself. Skenkius
observat. lib. 4. cap. de Uter. hath an example of a jealous woman that
by this means had many fits of the mother: and in his first book of
some that through jealousy ran mad: of a baker that gelded himself to
try his wife's honesty, \&c. Such examples are too common.

%MEMB. IV.

%SUBSECT I.-_Cure of Jealousy; by avoiding occasions, not to be idle: of good counsel; to contemn it, not to watch or lock them up: to dissemble it, \&c._
\section[Cure of Jealousy: by avoiding it]{Cure of Jealousy; by avoiding occasions, not to be idle: of good counsel; to contemn it, not to watch or lock them up: to dissemble it, \&c.}

\lettrine{A}{s} of all other melancholy, some doubt whether this malady may be cured
or no, they think 'tis like the \authorfootnote{6169}gout, or Switzers, whom we
commonly call Walloons, those hired soldiers, if once they take
possession of a castle, they can never be got out.
Qui timet ut sua sit, ne quis sibi subtrahat illam,
Ille Machaonia vix ope salvus est.

\authorfootnote{6170}This is the cruel wound against whose smart,
No liquor's force prevails, or any plaister,
No skill of stars, no depth of magic art,
Devised by that great clerk Zoroaster,
A wound that so infects the soul and heart,
As all our sense and reason it doth master;
A wound whose pang and torment is so durable,
As it may rightly called be incurable.

Yet what I have formerly said of other melancholy, I will say again, it
may be cured or mitigated at least by some contrary passion, good
counsel and persuasion, if it be withstood in the beginning, maturely
resisted, and as those ancients hold, \authorfootnote{6171}the nails of it be pared
before they grow too long. No better means to resist or repel it than
by avoiding idleness, to be still seriously busied about some matters
of importance, to drive out those vain fears, foolish fantasies and
irksome suspicions out of his head, and then to be persuaded by his
judicious friends, to give ear to their good counsel and advice, and
wisely to consider, how much he discredits himself, his friends,
dishonours his children, disgraceth his family, publisheth his shame,
and as a trumpeter of his own misery, divulgeth, macerates, grieves
himself and others; what an argument of weakness it is, how absurd a
thing in its own nature, how ridiculous, how brutish a passion, how
sottish, how odious; for as \authorfootnote{6172}Hierome well hath it, Odium sui
facit, et ipse novissime sibi odio est, others hate him, and at last he
hates himself for it; how harebrain a disease, mad and furious. If he
will but hear them speak, no doubt he may be cured. \authorfootnote{6173}Joan, queen
of Spain, of whom I have formerly spoken, under pretence of changing
air was sent to Complutum, or Alcada de las Heneras, where Ximenius the
archbishop of Toledo then lived, that by his good counsel (as for the
present she was) she might be eased. \authorfootnote{6174}For a disease of the soul,
if concealed, tortures and overturns it, and by no physic can sooner be
removed than by a discreet man's comfortable speeches. I will not here
insert any consolatory sentences to this purpose, or forestall any
man's invention, but leave it every one to dilate and amplify as he
shall think fit in his own judgment: let him advise with Siracides cap.
9. 1. Be not jealous over the wife of thy bosom; read that comfortable
and pithy speech to this purpose of Ximenius, in the author himself, as
it is recorded by Gomesius; consult with Chaloner lib. 9. de repub.
Anglor. or Caelia in her epistles, \&c. Only this I will add, that if it
be considered aright, which causeth this jealous passion, be it just or
unjust, whether with or without cause, true or false, it ought not so
heinously to be taken; 'tis no such real or capital matter, that it
should make so deep a wound. 'Tis a blow that hurts not, an insensible
smart, grounded many times upon false suspicion alone, and so fostered
by a sinister conceit. If she be not dishonest, he troubles and
macerates himself without a cause; or put case which is the worst, he
be a cuckold, it cannot be helped, the more he stirs in it, the more he
aggravates his own misery. How much better were it in such a case to
dissemble or contemn it? why should that be feared which cannot be
redressed? multae tandem deposuerunt (saith \authorfootnote{6175}Vives) quum flecti
maritos non posse vident, many women, when they see there is no remedy,
have been pacified; and shall men be more jealous than women? 'Tis some
comfort in such a case to have companions, Solamen miseris socios
habuisse doloris; Who can say he is free? Who can assure himself he is
not one de praeterito, or secure himself de futuro? If it were his case
alone, it were hard; but being as it is almost a common calamity, 'tis
not so grievously to be taken. If a man have a lock, which every man's
key will open, as well as his own, why should he think to keep it
private to himself? In some countries they make nothing of it, ne
nobiles quidem, saith \authorfootnote{6176}Leo Afer, in many parts of Africa (if she
be past fourteen) there's not a nobleman that marries a maid, or that
hath a chaste wife; 'tis so common; as the moon gives horns once a
month to the world, do they to their husbands at least. And 'tis most
part true which that Caledonian lady, \authorfootnote{6177}Argetocovus, a British
prince's wife, told Julia Augusta, when she took her up for dishonesty,
We Britons are naught at least with some few choice men of the better
sort, but you Romans lie with every base knave, you are a company of
common whores. Severus the emperor in his time made laws for the
restraint of this vice; and as \authorfootnote{6178}Dion Nicaeus relates in his life,
tria millia maechorum, three thousand cuckold-makers, or naturae
monetam adulterantes, as Philo calls them, false coiners, and clippers
of nature's money, were summoned into the court at once. And yet, Non
omnem molitor quae fluit undam videt, the miller sees not all the water
that goes by his mill: no doubt, but, as in our days, these were of the
commonalty, all the great ones were not so much as called in question
for it. \authorfootnote{6179}Martial's Epigram I suppose might have been generally
applied in those licentious times, Omnia solus habes, \&c., thy goods,
lands, money, wits are thine own, Uxorem sed habes Candide cum populo;
but neighbour Candidus your wife is common: husband and cuckold in that
age it seems were reciprocal terms; the emperors themselves did wear
Actaeon's badge; how many Caesars might I reckon up together, and what
a catalogue of cornuted kings and princes in every story? Agamemnon,
Menelaus, Philippus of Greece, Ptolomeus of Egypt, Lucullus, Caesar,
Pompeius, Cato, Augustus, Antonius, Antoninus, \&c., that wore fair
plumes of bull's feathers in their crests. The bravest soldiers and
most heroical spirits could not avoid it. They have been active and
passive in this business, they have either given or taken horns.

\authorfootnote{6180}King Arthur, whom we call one of the nine worthies, for all his
great valour, was unworthily served by Mordred, one of his round table
knights: and Guithera, or Helena Alba, his fair wife, as Leland
interprets it, was an arrant honest woman. Parcerem libenter (saith
mine \authorfootnote{6181}author) Heroinarum laesae majestati, si non historiae
veritas aurem vellicaret, I could willingly wink at a fair lady's
faults, but that I am bound by the laws of history to tell the truth:
against his will, God knows, did he write it, and so do I repeat it. I
speak not of our times all this while, we have good, honest, virtuous
men and women, whom fame, zeal, fear of God, religion and superstition
contains: and yet for all that, we have many knights of this order, so
dubbed by their wives, many good women abused by dissolute husbands. In
some places, and such persons you may as soon enjoin them to carry
water in a sieve, as to keep themselves honest. What shall a man do now
in such a case? What remedy is to be had? how shall he be eased? By
suing a divorce? this is hard to be effected: si non caste, tamen caute
they carry the matter so cunningly, that though it be as common as
simony, as clear and as manifest as the nose in a man's face, yet it
cannot be evidently proved, or they likely taken in the fact: they will
have a knave Gallus to watch, or with that Roman \authorfootnote{6182}Sulpitia, all
made fast and sure,
Ne se Cadurcis destitutam fasciis,
Nudam Caleno concumbentem videat.

she will hardly be surprised by her husband, be he never so wary. Much
better then to put it up: the more he strives in it, the more he shall
divulge his own shame: make a virtue of necessity, and conceal it. Yea,
but the world takes notice of it, 'tis in every man's mouth: let them
talk their pleasure, of whom speak they not in this sense? From the
highest to the lowest they are thus censured all: there is no remedy
then but patience. It may be 'tis his own fault, and he hath no reason
to complain, 'tis quid pro quo, she is bad, he is worse: \authorfootnote{6183}Bethink
thyself, hast thou not done as much for some of thy neighbours? why
dost thou require that of thy wife, which thou wilt not perform
thyself? Thou rangest like a town bull, \authorfootnote{6184}why art thou so incensed
if she tread, awry?
\authorfootnote{6185}Be it that some woman break chaste wedlock's laws,
And leaves her husband and becomes unchaste:
Yet commonly it is not without cause,
She sees her man in sin her goods to waste,
She feels that he his love from her withdraws,
And hath on some perhaps less worthy placed.
Who strike with sword, the scabbard them may strike,
And sure love craveth love, like asketh like.

Ea semper studebit, saith \authorfootnote{6186}Nevisanus, pares reddere vices, she
will quit it if she can. And therefore, as well adviseth Siracides,
cap. ix. 1. teach her not an evil lesson against thyself, which as
Jansenius, Lyranus, on his text, and Carthusianus interpret, is no
otherwise to be understood than that she do thee not a mischief. I do
not excuse her in accusing thee; but if both be naught, mend thyself
first; for as the old saying is, a good husband makes a good wife.

Yea but thou repliest, 'tis not the like reason betwixt man and woman,
through her fault my children are bastards, I may not endure it;
\authorfootnote{6187}Sit amarulenta, sit imperiosa prodiga, \&c. Let her scold, brawl,
and spend, I care not, modo sit casta, so she be honest, I could easily
bear it; but this I cannot, I may not, I will not; my faith, my fame,
mine eye must not be touched, as the diverb is, Non patitur tactum
fama, fides, oculus. I say the same of my wife, touch all, use all,
take all but this. I acknowledge that of Seneca to be true, Nullius
boni jucunda possessio sine socio, there is no sweet content in the
possession of any good thing without a companion, this only excepted, I
say, This. And why this? Even this which thou so much abhorrest, it may
be for thy progeny's good, \authorfootnote{6188} better be any man's son than thine,
to be begot of base Irus, poor Seius, or mean Mevius, the town
swineherd's, a shepherd's son: and well is he, that like Hercules he
hath any two fathers; for thou thyself hast peradventure more diseases
than a horse, more infirmities of body and mind, a cankered soul,
crabbed conditions, make the worst of it, as it is vulnus insanabile,
sic vulnus insensibile, as it is incurable, so it is insensible. But
art thou sure it is so? \authorfootnote{6189}res agit ille tuas? doth he so indeed? It
may be thou art over-suspicious, and without a cause as some are: if it
be octimestris partus, born at eight months, or like him, and him, they
fondly suspect he got it; if she speak or laugh familiarly with such or
such men, then presently she is naught with them; such is thy weakness;
whereas charity, or a well-disposed mind, would interpret all unto the
best. St. Francis, by chance seeing a friar familiarly kissing another
man's wife, was so far from misconceiving it, that he presently kneeled
down and thanked God there was so much charity left: but they on the
other side will ascribe nothing to natural causes, indulge nothing to
familiarity, mutual society, friendship: but out of a sinister
suspicion, presently lock them close, watch them, thinking by those
means to prevent all such inconveniences, that's the way to help it;
whereas by such tricks they do aggravate the mischief. 'Tis but in vain
to watch that which will away.

\authorfootnote{6190}Nec custodiri si velit ulla potest;
Nec mentem servare potes, licet omnia serves;
Omnibus exclusis, intus adulter erit.

None can be kept resisting for her part;
Though body be kept close, within her heart
Advoutry lurks, t'exclude it there's no art.

Argus with a hundred eyes cannot keep her, et hunc unus saepe fefellit
amor, as in \authorfootnote{6191}Ariosto,
If all our hearts were eyes, yet sure they said
We husbands of our wives should be betrayed.

Hierome holds, \li{Uxor impudica servari non potest, pudica non debet,
infida custos castitatis est necessitas}, to what end is all your
custody? A dishonest woman cannot be kept, an honest woman ought not to
be kept, necessity is a keeper not to be trusted. \li{Difficile custoditur,
quod plures amant}; that which many covet, can hardly be preserved, as
\authorfootnote{6192} Salisburiensis thinks. I am of Aeneas Sylvius' mind, \authorfootnote{6193}Those
jealous Italians do very ill to lock up their wives; for women are of
such a disposition, they will most covet that which is denied most, and
offend least when they have free liberty to trespass. It is in vain to
lock her up if she be dishonest; \li{et tyrranicum imperium}, as our great
Mr. Aristotle calls it, too tyrannical a task, most unfit: for when she
perceives her husband observes her and suspects, \li{liberius peccat}, saith
\authorfootnote{6194}Nevisanus. \authorfootnote{6195}\li{Toxica Zelotypo dedit uxor moecha marito}, she is
exasperated, seeks by all means to vindicate herself, and will
therefore offend, because she is unjustly suspected. The best course
then is to let them have their own wills, give them free liberty,
without any keeping.

In vain our friends from this do us dehort,
For beauty will be where is most resort.

If she be honest as Lucretia to Collatinus, Laodamia to Protesilaus,
Penelope to her Ulysses, she will so continue her honour, good name,
credit, Penelope conjux semper Ulyssis ero; I shall always be Penelope
the wife of Ulysses. And as Phocias' wife in \authorfootnote{6196}Plutarch, called her
husband her wealth, treasure, world, joy, delight, orb and sphere, she
will hers. The vow she made unto her good man; love, virtue, religion,
zeal, are better keepers than all those locks, eunuchs, prisons; she
will not be moved:
\authorfootnote{6197}At mihi vel tellus optem prius ima dehiscat,
Aut pater omnipotens adigat me fulmine ad umbras,
Pallentes umbras Erebi, noctemque profundam,
Ante pudor quam te violem, aut tua jura resolvam.

First I desire the earth to swallow me.
Before I violate mine honesty,
Or thunder from above drive me to hell,
With those pale ghosts, and ugly nights to dwell.

She is resolved with Dido to be chaste; though her husband be false,
she will be true: and as Octavia writ to her Antony,
\authorfootnote{6198}These walls that here do keep me out of sight,
Shall keep me all unspotted unto thee,
And testify that I will do thee right,
I'll never stain thine house, though thou shame me.

Turn her loose to all those Tarquins and Satyrs, she will not be
tempted. In the time of Valence the Emperor, saith \authorfootnote{6199}St. Austin,
one Archidamus, a Consul of Antioch, offered a hundred pounds of gold
to a fair young wife, and besides to set her husband free, who was then
sub gravissima custodia, a dark prisoner, pro unius noctis concubitu:
but the chaste matron would not accept of it. \authorfootnote{6200}When Ode commended
Theana's fine arm to his fellows, she took him up short, Sir, 'tis not
common: she is wholly reserved to her husband. \authorfootnote{6201}Bilia had an old
man to her spouse, and his breath stunk, so that nobody could abide it
abroad; coming home one day he reprehended his wife, because she did
not tell him of it: she vowed unto him, she had told him, but she
thought every man's breath had been as strong as his. \authorfootnote{6202}Tigranes
and Armena his lady were invited to supper by King Cyrus: when they
came home, Tigranes asked his wife, how she liked Cyrus, and what she
did especially commend in him? she swore she did not observe him; when
he replied again, what then she did observe, whom she looked on? She
made answer, her husband, that said he would die for her sake. Such are
the properties and conditions of good women: and if she be well given,
she will so carry herself; if otherwise she be naught, use all the
means thou canst, she will be naught, Non deest animus sed corruptor,
she hath so many lies, excuses, as a hare hath muses, tricks, panders,
bawds, shifts, to deceive, 'tis to no purpose to keep her up, or to
reclaim her by hard usage. Fair means peradventure may do somewhat.

\authorfootnote{6203} \li{Obsequio vinces aptius ipse tuo.} Men and women are both in a
predicament in this behalf, no sooner won, and better pacified. Duci
volunt, non cogi: though she be as arrant a scold as Xanthippe, as
cruel as Medea, as clamorous as Hecuba, as lustful as Messalina, by
such means (if at all) she may be reformed. Many patient \authorfootnote{6204}Grizels,
by their obsequiousness in this kind, have reclaimed their husbands
from their wandering lusts. In Nova Francia and Turkey (as Leah,
Rachel, and Sarah did to Abraham and Jacob) they bring their fairest
damsels to their husbands' beds; Livia seconded the lustful appetites
of Augustus: Stratonice, wife to King Diotarus, did not only bring
Electra, a fair maid, to her good man's bed, but brought up the
children begot on her, as carefully as if they had been her own.

Tertius Emilius' wife, Cornelia's mother, perceiving her husband's
intemperance, rem dissimulavit, made much of the maid, and would take
no notice of it. A new-married man, when a pickthank friend of his, to
curry favour, had showed him his wife familiar in private with a young
gallant, courting and dallying, \&c. Tush, said he, let him do his
worst, I dare trust my wife, though I dare not trust him. The best
remedy then is by fair means; if that will not take place, to dissemble
it as I say, or turn it off with a jest: hear Guexerra's advice in this
case, vel joco excipies, vel silentio eludes; for if you take
exceptions at everything your wife doth, Solomon's wisdom, Hercules'
valour, Homer's learning, Socrates' patience, Argus' vigilance, will
not serve turn. Therefore Minus malum, \authorfootnote{6205}a less mischief, Nevisanus
holds, dissimulare, to be \authorfootnote{6206}Cunarum emptor, a buyer of cradles, as
the proverb is, than to be too solicitous. \authorfootnote{6207}A good fellow, when
his wife was brought to bed before her time, bought half a dozen of
cradles beforehand for so many children, as if his wife should continue
to bear children every two months. \authorfootnote{6208}Pertinax the Emperor, when one
told him a fiddler was too familiar with his empress, made no reckoning
of it. And when that Macedonian Philip was upbraided with his wife's
dishonesty, \li{cum tot victor regnorum ac populorum esset}, \&c., a
conqueror of kingdoms could not tame his wife (for she thrust him out
of doors), he made a jest of it. Sapientes portant cornua in pectore,
stulti in fronte, saith Nevisanus, wise men bear their horns in their
hearts, fools on their foreheads. Eumenes, king of Pergamus, was at
deadly feud with Perseus of Macedonia, insomuch that Perseus hearing of
a journey he was to take to Delphos, \authorfootnote{6209}set a company of soldiers to
intercept him in his passage; they did it accordingly, and as they
supposed left him stoned to death. The news of this fact was brought
instantly to Pergamus; Attalus, Eumenes' brother, proclaimed himself
king forthwith, took possession of the crown, and married Stratonice
the queen. But by-and-by, when contrary news was brought, that King
Eumenes was alive, and now coming to the city, he laid by his crown,
left his wife, as a private man went to meet him, and congratulate his
return. Eumenes, though he knew all particulars passed, yet dissembling
the matter, kindly embraced his brother, and took his wife into his
favour again, as if on such matter had been heard of or done. Jocundo,
in Ariosto, found his wife in bed with a knave, both asleep, went his
ways, and would not so much as wake them, much less reprove them for
it. \authorfootnote{6210}An honest fellow finding in like sort his wife had played
false at tables, and borne a man too many, drew his dagger, and swore
if he had not been his very friend, he would have killed him. Another
hearing one had done that for him, which no man desires to be done by a
deputy, followed in a rage with his sword drawn, and having overtaken
him, laid adultery to his charge; the offender hotly pursued, confessed
it was true; with which confession he was satisfied, and so left him,
swearing that if he had denied it, he would not have put it up. How
much better is it to do thus, than to macerate himself, impatiently to
rave and rage, to enter an action (as Arnoldus Tilius did in the court
of Toulouse, against Martin Guerre his fellow-soldier, for that he
counterfeited his habit, and was too familiar with his wife), so to
divulge his own shame, and to remain for ever a cuckold on record? how
much better be Cornelius Tacitus than Publius Cornutus, to condemn in
such cases, or take no notice of it? \li{Melius sic errare, quam Zelotypiae
curis}, saith Erasmus, \li{se conficere}, better be a wittol and put it up,
than to trouble himself to no purpose. And though he will not omnibus
dormire, be an ass, as he is an ox, yet to wink at it as many do is not
amiss at some times, in some cases, to some parties, if it be for his
commodity, or some great man's sake, his landlord, patron, benefactor,
(as Calbas the Roman saith \authorfootnote{6211}Plutarch did by Maecenas, and Phayllus
of Argos did by King Philip, when he promised him an office on that
condition he might lie with his wife) and so let it pass:
\authorfootnote{6212}pol me haud poenitet,
Scilicet boni dimidium dividere cum Jove,

it never troubles me (saith Amphitrio) to be cornuted by Jupiter, let
it not molest thee then; be friends with her;
\authorfootnote{6213}Tu cum Alcmena uxore antiquam in gratiam
Redi---

Receive Alcmena to your grace again; let it, I say, make no breach of
love between you. Howsoever the best way is to contemn it, which
\authorfootnote{6214}Henry II. king of France advised a courtier of his, jealous of
his wife, and complaining of her unchasteness, to reject it, and
comfort himself; for he that suspects his wife's incontinency, and
fears the Pope's curse, shall never live a merry hour, or sleep a quiet
night: no remedy but patience. When all is done according to that
counsel of \authorfootnote{6215}Nevisanus, si vitium uxoris corrigi non potest,
ferendum est: if it may not be helped, it must be endured. Date veniam
et sustinete taciti, 'tis Sophocles' advice, keep it to thyself, and
which Chrysostom calls palaestram philosophiae, et domesticum gymnasium
a school of philosophy, put it up. There is no other cure but time to
wear it out, Injuriarum remedium est oblivio, as if they had drunk a
draught of Lethe in Trophonius' den: to conclude, age will bereave her
of it, dies dolorem minuit, time and patience must end it.

\authorfootnote{6216}The mind's affections patience will appease,
It passions kills, and healeth each disease.


%SUBSECT. II.-_By prevention before, or after Marriage, Plato's Community, marry a Courtesan, Philters, Stews, to marry one equal in years, fortunes, of a good family, education, good place, to use them well, \&c._
\section[By prevention before or after Marriage]{By prevention before, or after Marriage, Plato's Community, marry a Courtesan, Philters, Stews, to marry one equal in years, fortunes, of a good family, education, good place, to use them well, \&c.}

\lettrine{O}{f} such medicines as conduce to the cure of this malady, I have
sufficiently treated; there be some good remedies remaining, by way of
prevention, precautions, or admonitions, which if rightly practised,
may do much good. Plato, in his Commonwealth, to prevent this mischief
belike, would have all things, wives and children, all as one: and
which Caesar in his Commentaries observed of those old Britons, that
first inhabited this land, they had ten or twelve wives allotted to
such a family, or promiscuously to be used by so many men; not one to
one, as with us, or four, five, or six to one, as in Turkey. The
\authorfootnote{6217}Nicholaites, a set that sprang, saith Austin, from Nicholas the
deacon, would have women indifferent; and the cause of this filthy
sect, was Nicholas the deacon's jealousy, for which when he was
condemned to purge himself of his offence, he broached his heresy, that
it was lawful to lie with one another's wives, and for any man to lie
with his: like to those \authorfootnote{6218}Anabaptists in Munster, that would
consort with other men's wives as the spirit moved them: or as
\authorfootnote{6219}Mahomet, the seducing prophet, would needs use women as he list
himself, to beget prophets; two hundred and five, their Alcoran saith,
were in love with him, and \authorfootnote{6220}he as able as forty men. Amongst the
old Carthaginians, as \authorfootnote{6221}Bohemus relates out of Sabellicus., the
king of the country lay with the bride the first night, and once in a
year they went promiscuously all together. Munster Cosmog. lib. 3. cap.
497. ascribes the beginning of this brutish custom (unjustly) to one
Picardus, a Frenchman, that invented a new sect of Adamites, to go
naked as Adam did, and to use promiscuous venery at set times. When the
priest repeated that of Genesis, Increase and multiply, out \authorfootnote{6222}went
the candles in the place where they met, and without all respect of
age, persons, conditions, catch that catch may, every man took her that
came next, \&c.; some fasten this on those ancient Bohemians and
Russians: \authorfootnote{6223}others on the inhabitants of Mambrium, in the Lucerne
valley in Piedmont; and, as I read, it was practised in Scotland
amongst Christians themselves, until King Malcolm's time, the king or
the lord of the town had their maidenheads. In some parts of
\authorfootnote{6224}India in our age, and those \authorfootnote{6225}islanders, \authorfootnote{6226}as amongst the
Babylonians of old, they will prostitute their wives and daughters
(which Chalcocondila, a Greek modern writer, for want of better
intelligence, puts upon us Britons) to such travellers or seafaring men
as come amongst them by chance, to show how far they were from this
feral vice of jealousy, and how little they esteemed it. The kings of
Calecut, as \authorfootnote{6227}Lod. Vertomannus relates, will not touch their wives,
till one of their Biarmi or high priests have lain first with them, to
sanctify their wombs. But those Esai and Montanists, two strange sects
of old, were in another extreme, they would not marry at all, or have
any society with women, \authorfootnote{6228}because of their intemperance they held
them all to be naught. Nevisanus the lawyer, lib. 4. num. 33. sylv.
nupt. would have him that is inclined to this malady, to prevent the
worst, marry a quean, Capiens meretricem, hoc habet saltem boni quod
non decipitur, quia scit eam sic esse, quod non contingit aliis. A
fornicator in Seneca constuprated two wenches in a night; for
satisfaction, the one desired to hang him, the other to marry him.

\authorfootnote{6229} Hierome, king of Syracuse in Sicily, espoused himself to Pitho,
keeper of the stews; and Ptolemy took Thaïs a common whore to be his
wife, had two sons, Leontiscus and Lagus by her, and one daughter
Irene: 'tis therefore no such unlikely thing. \authorfootnote{6230}A citizen of
Engubine gelded himself to try his wife's honesty, and to be freed from
jealousy; so did a baker in \authorfootnote{6231} Basil, to the same intent. But of
all other precedents in this kind, that of \authorfootnote{6232}Combalus is most
memorable; who to prevent his master's suspicion, for he was a
beautiful young man, and sent by Seleucus his lord and king, with
Stratonice the queen to conduct her into Syria, fearing the worst,
gelded himself before he went, and left his genitals behind him in a
box sealed up. His mistress by the way fell in love with him, but he
not yielding to her, was accused to Seleucus of incontinency, (as that
Bellerophon was in like case, falsely traduced by Sthenobia, to King
Praetus her husband, cum non posset ad coitum inducere) and that by
her, and was therefore at his corning home cast into prison: the day of
hearing appointed, he was sufficiently cleared and acquitted, by
showing his privities, which to the admiration of the beholders he had
formerly cut off. The Lydians used to geld women whom they suspected,
saith Leonicus var. hist. Tib. 3. cap. 49. as well as men. To this
purpose \authorfootnote{6233}Saint Francis, because he used to confess women in
private, to prevent suspicion, and prove himself a maid, stripped
himself before the Bishop of Assise and others: and Friar Leonard for
the same cause went through Viterbium in Italy, without any garments.

Our pseudo-Catholics, to help these inconveniences which proceed from
jealousy, to keep themselves and their wives honest, make severe laws;
against adultery present death; and withal fornication, a venal sin, as
a sink to convey that furious and swift stream of concupiscence, they
appoint and permit stews, those punks and pleasant sinners, the more to
secure their wives in all populous cities, for they hold them as
necessary as churches; and howsoever unlawful, yet to avoid a greater
mischief, to be tolerated in policy, as usury, for the hardness of
men's hearts; and for this end they have whole colleges of courtesans
in their towns and cities. Of \authorfootnote{6234}Cato's mind belike, that would have
his servants (cum ancillis congredi coitus causa, definito aere, ut
graviora facinora evitarent, caeteris interim interdicens) familiar
with some such feminine creatures, to avoid worse mischiefs in his
house, and made allowance for it. They hold it impossible for idle
persons, young, rich, and lusty, so many servants, monks, friars, to
live honest, too tyrannical a burden to compel them to be chaste, and
most unfit to suffer poor men, younger brothers and soldiers at all to
marry, as those diseased persons, votaries, priests, servants.

Therefore, as well to keep and ease the one as the other, they tolerate
and wink at these kind of brothel-houses and stews. Many probable
arguments they have to prove the lawfulness, the necessity, and a
toleration of them, as of usury; and without question in policy they
are not to be contradicted: but altogether in religion. Others
prescribe filters, spells, charms to keep men and women honest.

\authorfootnote{6235}Mulier ut alienum virum non admittat praeter suum: Accipe fel
hirci, et adipem, et exsicca, calescat in oleo, \&c., et non alium
praeter et amabit. In Alexi. Porta, \&c., plura invenies, et multo his
absurdiora, uti et in Rhasi, ne mulier virum admittat, et maritum solum
diligat, \&c. But these are most part Pagan, impious, irreligious,
absurd, and ridiculous devices.

The best means to avoid these and like inconveniences are, to take away
the causes and occasions. To this purpose \authorfootnote{6236}Varro writ Satyram
Menippeam, but it is lost. \authorfootnote{6237}Patritius prescribes four rules to be
observed in choosing of a wife (which who so will may read); Fonseca,
the Spaniard, in his 45. c. Amphitheat. Amoris, sets down six special
cautions for men, four for women; Sam. Neander out of Shonbernerus,
five for men, five for women; Anthony Guivarra many good lessons;
\authorfootnote{6238}Cleobulus two alone, others otherwise; as first to make a good
choice in marriage, to invite Christ to their wedding, and which
\authorfootnote{6239}St. Ambrose adviseth, Deum conjugii praesidem habere, and to pray
to him for her, A Domino enim datur uxor prudens, Prov. xix. ) not to
be too rash and precipitate in his election, to run upon the first he
meets, or dote on every stout fair piece he sees, but to choose her as
much by his ears as eyes, to be well advised whom he takes, of what
age, \&c., and cautelous in his proceedings. An old man should not marry
a young woman, nor a young woman an old man, \li{Quam male
inaequales veniunt ad arata juvenci}\authorlatintrans{6240.5}!\authormarginnote{6240} such matches must needs minister a
perpetual cause of suspicion, and be distasteful to each other.
\authorfootnote{6241}Noctua ut in tumulis, super atque cadavera bubo,
Talis apud Sophoclem nostra puella sedet.

Night-crows on tombs, owl sits on carcass dead,
So lies a wench with Sophocles in bed.

For Sophocles, as \authorfootnote{6242}Atheneus describes him, was a very old man, as
cold as January, a bedfellow of bones, and doted yet upon Archippe, a
young courtesan, than which nothing can be more odious. \authorfootnote{6243}Senex
maritus uxori juveni ingratus est, an old man is a most unwelcome guest
to a young wench, unable, unfit:
\authorfootnote{6244}Amplexus suos fugiunt puellae,
Omnis horret amor Venusque Hymenque.

And as in like case a good fellow that had but a peck of corn weekly to
grind, yet would needs build a new mill for it, found his error
eftsoons, for either he must let his mill lie waste, pull it quite
down, or let others grind at it. So these men, \&c.

Seneca therefore disallows all such unseasonable matches, habent enim
maledicti locum crebrae nuptiae. And as \authorfootnote{6245}Tully farther inveighs,
'tis unfit for any, but ugly and filthy in old age. Turpe senilis amor,
one of the three things \authorfootnote{6246}God hateth. Plutarch, in his book contra
Coleten, rails downright at such kind of marriages, which are attempted
by old men, qui jam corpore impotenti, et a voluptatibus deserti,
peccant animo, and makes a question whether in some cases it be
tolerable at least for such a man to marry,-qui Venerem affectat sine
viribus, that is now past those venerous exercises, as a gelded man
lies with a virgin and sighs, Ecclus. xxx. 20, and now complains with
him in Petronius, funerata est haec pars jam, quad fuit olim Achillea,
he is quite done,
\authorfootnote{6247}Vixit puellae nuper idoneus,
Et militavit non sine gloria.

But the question is whether he may delight himself as those Priapeian
popes, which, in their decrepit age, lay commonly between two wenches
every night, contactu formosarum, et contrectatione, num adhuc gaudeat;
and as many doting sires do to their own shame, their children's
undoing, and their families' confusion: he abhors it, tanquam ab
agresti et furioso domino fugiendum, it must be avoided as a bedlam
master, and not obeyed.
\authorfootnote{6248}Alecto---
Ipsa faces praefert nubentibus, et malus Hymen
Triste ululat,---

the devil himself makes such matches. \authorfootnote{6249}Levinus Lemnius reckons up
three things which generally disturb the peace of marriage: the first
is when they marry intempestive or unseasonably, as many mortal men
marry precipitately and inconsiderately, when they are effete and old:
the second when they marry unequally for fortunes and birth: the third,
when a sick impotent person weds one that is sound, novae nuptae spes
frustratur: many dislikes instantly follow. Many doting dizzards, it
may not be denied, as Plutarch confesseth, \authorfootnote{6250}recreate themselves
with such obsolete, unseasonable and filthy remedies (so he calls
them), with a remembrance of their former pleasures, against nature
they stir up their dead flesh: but an old lecher is abominable; mulier
tertio nubens, \authorfootnote{6251}Nevisanus holds, praesumitur lubrica, et
inconstans, a woman that marries a third time may be presumed to be no
honester than she should. Of them both, thus Ambrose concludes in his
comment upon Luke, \authorfootnote{6252}they that are coupled together, not to get
children, but to satisfy their lust, are not husbands, but fornicators,
with whom St. Austin consents: matrimony without hope of children, non
matrimonium, sed concubium dici debet, is not a wedding but a jumbling
or coupling together. In a word (except they wed for mutual society,
help and comfort one of another, in which respects, though
\authorfootnote{6253}Tiberius deny it, without question old folks may well marry) for
sometimes a man hath most need of a wife, according to Puccius, when he
hath no need of a wife; otherwise it is most odious, when an old
Acherontic dizzard, that hath one foot in his grave, a silicernium,
shall flicker after a young wench that is blithe and bonny,
\authorfootnote{6254}---salaciorque
Verno passere, et albulis columbis.

What can be more detestable?
\authorfootnote{6255}Tu cano capite amas senex nequissime
Jam plenus aetatis, animaque foetida,
Senex hircosus tu osculare mulierem?
Utine adiens vomitum potius excuties.

Thou old goat, hoary lecher, naughty man,
With stinking breath, art thou in love?
Must thou be slavering? she spews to see
Thy filthy face, it doth so move.

Yet, as some will, it is much more tolerable for an old man to marry a
young woman (our ladies' match they call it) for cras erit mulier, as
he said in Tully. Cato the Roman, Critobulus in \authorfootnote{6256}Xenophon,
\authorfootnote{6257}Tiraquellus of late, Julius Scaliger, \&c., and many famous
precedents we have in that kind; but not e contra: 'tis not held fit
for an ancient woman to match with a young man. For as Varro will, Anus
dum ludit morti delitias facit, 'tis Charon's match between
\authorfootnote{6258}Cascus and Casca, and the devil himself is surely well pleased
with it. And, therefore, as the \authorfootnote{6259}poet inveighs, thou old Vetustina
bedridden quean, that art now skin and bones,
Cui tres capilli, quatuorque sunt dentes,
Pectus cicadae, crusculumque formicae,
Rugosiorem quae geris stola frontem,
Et arenaram cassibus pares mammas.

That hast three hairs, four teeth, a breast
Like grasshopper, an emmet's crest,
A skin more rugged than thy coat,
And drugs like spider's web to boot.

Must thou marry a youth again? And yet ducentas ire nuptum post mortes
amant: howsoever it is, as \authorfootnote{6260}Apuleius gives out of his Meroe,
congressus annosus, pestilens, abhorrendus, a pestilent match,
abominable, and not to be endured. In such case how can they otherwise
choose but be jealous, how should they agree one with another? This
inequality is not in years only, but in birth, fortunes, conditions,
and all good \authorfootnote{6261}qualities, si qua voles apte nubere, nube pari, 'tis
my counsel, saith Anthony Guiverra, to choose such a one. Civis Civem
ducat, Nobilis Nobilem, let a citizen match with a citizen, a gentleman
with a gentlewoman; he that observes not this precept (saith he) non
generum sed malum Genium, non nurum sed Furiam, non vitae Comitem, sed
litis fomitem domi habebit, instead of a fair wife shall have a fury,
for a fit son-in-law a mere fiend, \&c. examples are too frequent.

Another main caution fit to be observed is this, that though they be
equal in years, birth, fortunes, and other conditions, yet they do not
omit virtue and good education, which Musonius and Antipater so much
inculcate in Stobeus:
\authorfootnote{6262}Dos est magna parentum
Virtus, et metuens alterius viri
Certo foedere castitas.

If, as Plutarch adviseth, one must eat modium salis, a bushel of salt
with him, before he choose his friend, what care should be had in
choosing a wife, his second self, how solicitous should he be to know
her qualities and behaviour; and when he is assured of them, not to
prefer birth, fortune, beauty, before bringing up, and good conditions.

\authorfootnote{6263}Coquage god of cuckolds, as one merrily said, accompanies the
goddess Jealousy, both follow the fairest, by Jupiter's appointment,
and they sacrifice to them together: beauty and honesty seldom agree;
straight personages have often crooked manners; fair faces, foul vices;
good complexions, ill conditions. Suspicionis plena res est, et
insidiarum, beauty (saith \authorfootnote{6264}Chrysostom) is full of treachery and
suspicion: he that hath a fair wife, cannot have a worse mischief, and
yet most covet it, as if nothing else in marriage but that and wealth
were to be respected. \authorfootnote{6265}Francis Sforza, Duke of Milan, was so
curious in this behalf, that he would not marry the Duke of Mantua's
daughter, except he might see her naked first: which Lycurgus appointed
in his laws, and Morus in his Utopian Commonwealth approves. \authorfootnote{6266}In
Italy, as a traveller observes, if a man have three or four daughters,
or more, and they prove fair, they are married eftsoons: if deformed,
they change their lovely names of Lucia, Cynthia, Camaena, call them
Dorothy, Ursula, Bridget, and so put them into monasteries, as if none
were fit for marriage, but such as are eminently fair: but these are
erroneous tenets: a modest virgin well conditioned, to such a fair
snout-piece, is much to be preferred. If thou wilt avoid them, take
away all causes of suspicion and jealousy, marry a coarse piece, fetch
her from Cassandra's \authorfootnote{6267}temple, which was wont in Italy to be a
sanctuary of all deformed maids, and so shalt thou be sure that no man
will make thee cuckold, but for spite. A citizen of Bizance in France
had a filthy, dowdy, deformed slut to his wife, and finding her in bed
with another man, cried out as one amazed; O miser! quae te necessitas
huc adegit? O thou wretch, what necessity brought thee hither? as well
he might; for who can affect such a one? But this is warily to be
understood, most offend in another extreme, they prefer wealth before
beauty, and so she be rich, they care not how she look; but these are
all out as faulty as the rest. Attendenda uxoris forma, as
\authorfootnote{6268}Salisburiensis adviseth, ne si alteram aspexeris, mox eam sordere
putes, as the Knight in Chaucer, that was married to an old woman,\phantomsection\label{mention:chaucer-quote-postface}
%
{\gothfont%
\begin{versewithlinenos}{2}{1}{1}%
And all day after hid him as an owl,\\*
So woe was his wife looked so foul.\\!
\end{versewithlinenos}%
}%
\attrib{[Lines FIXME. \theeditor{}]}
%
Have a care of thy wife's complexion, lest whilst thou seest another,
thou loathest her, she prove jealous, thou naught,
\authorfootnote{6269}Si tibi deformis conjux, si serva venusta,
Ne utaris serva,---

I can perhaps give instance. Molestum est possidere, quod nemo habere
dignetur, a misery to possess that which no man likes: on the other
side, Difficile custoditur quod plures amant. And as the bragging
soldier vaunted in the comedy, nimia est miseria pulchrum esse hominem
nimis. Scipio did never so hardly besiege Carthage, as these young
gallants will beset thine house, one with wit or person, another with
wealth, \&c. If she he fair, saith Guazzo, she will be suspected
howsoever. Both extremes are naught, Pulchra cito adamatur, foeda
facile concupiscit, the one is soon beloved, the other loves: one is
hardly kept, because proud and arrogant, the other not worth keeping;
what is to be done in this case? Ennius in Menelippe adviseth thee as a
friend to take statam formam, si vis habere incolumem pudicitiam, one
of a middle size, neither too fair nor too foul, \authorfootnote{6270}Nec formosa
magis quam mihi casta placet, with old Cato, though fit let her beauty
be, neque lectissima, neque illiberalis, between both. This I approve;
but of the other two I resolve with Salisburiensis, caeteris paribus,
both rich alike, endowed alike, majori miseria deformis habetur quam
formosa servatur, I had rather marry a fair one, and put it to the
hazard, than be troubled with a blowze; but do as thou wilt, I speak
only of myself.

Howsoever, \li{quod iterum maneo}, I would advise thee thus much, be she
fair or foul, to choose a wife out of a good kindred, parentage, well
brought up, in an honest place.
\authorfootnote{6271}Primum animo tibi proponas quo sanguine creta.
Qua forma, qua aetate, quibusque ante omnia virgo
Moribus, in junctos veniat nova nupta penates.

He that marries a wife out of a suspected inn or alehouse, buys a horse
in Smithfield, and hires a servant in Paul's, as the diverb is, shall
likely have a jade to his horse, a knave for his man, an arrant honest
woman to his wife. Filia praesumitur, esse matri similis, saith
\authorfootnote{6272}Nevisanus? Such \authorfootnote{6273}a mother, such a daughter; mali corvi malum
ovum., cat to her kind.
\authorfootnote{6274}Scilicet expectas ut tradat mater honestos
Atque alios mores quam quos habet?

If the mother be dishonest, in all likelihood the daughter will
matrizare, take after her in all good qualities,
Creden' Pasiphae non tauripotente futuram
Tauripetam?---

If the dam trot, the foal will not amble. My last caution is, that a
woman do not bestow herself upon a fool, or an apparent melancholy
person; jealousy is a symptom of that disease, and fools have no
moderation. Justina, a Roman lady, was much persecuted, and after made
away by her jealous husband, she caused and enjoined this epitaph, as a
caveat to others, to be engraven on her tomb:
\authorfootnote{6275}Discite ab exemplo Justinae, discite patres,
Ne nubat fatuo filia vestra viro, \&c.

Learn parents all, and by Justina's case,
Your children to no dizzards for to place.

After marriage, I can give no better admonitions than to use their
wives well, and which a friend of mine told me that was a married man,
I will tell you as good cheap, saith Nicostratus in \authorfootnote{6276}Stobeus, to
avoid future strife, and for quietness' sake, when you are in bed, take
heed of your wife's flattering speeches over night, and curtain,
sermons in the morning. Let them do their endeavour likewise to
maintain them to their means, which \authorfootnote{6277}Patricius ingeminates, and
let them have liberty with discretion, as time and place requires: many
women turn queans by compulsion, as \authorfootnote{6278}Nevisanus observes, because
their husbands are so hard, and keep them so short in diet and apparel,
paupertas cogit eas meretricari, poverty and hunger, want of means,
makes them dishonest, or bad usage; their churlish behaviour forceth
them to fly out, or bad examples, they do it to cry quittance. In the
other extreme some are too liberal, as the proverb is, Turdus malum
sibi cacat, they make a rod for their own tails, as Candaules did to
Gyges in \authorfootnote{6279}Herodotus, commend his wife's beauty himself, and
besides would needs have him see her naked. Whilst they give their
wives too much liberty to gad abroad, and bountiful allowance, they are
accessory to their own miseries; animae uxorum pessime olent, as
Plautus jibes, they have deformed souls, and by their painting and
colours procure odium mariti, their husband's hate, especially,-\authorfootnote{6280}
cum misere viscantur labra mariti. Besides, their wives (as \authorfootnote{6281}Basil
notes) Impudenter se exponunt masculorum aspectibus, jactantes tunicas,
et coram tripudiantes, impudently thrust themselves into other men's
companies, and by their indecent wanton carriage provoke and tempt the
spectators. Virtuous women should keep house; and 'twas well performed
and ordered by the Greeks,
\authorfootnote{6282}---mulier ne qua in publicum
Spectandam se sine arbitro praebeat viro:

which made Phidias belike at Elis paint Venus treading on a tortoise, a
symbol of women's silence and housekeeping. For a woman abroad and
alone, is like a deer broke out of a park, quam mille venatores
insequuntur, whom every hunter follows; and besides in such places she
cannot so well vindicate herself, but as that virgin Dinah (Gen.
\rn{xxxiv.}, 2,) going for to see the daughters of the land, lost her
virginity, she may be defiled and overtaken of a sudden: Imbelles damae
quid nisi praeda sumus? \authorfootnote{6283}
And therefore I know not what philosopher he was, that would have women
come but thrice abroad all their time, \authorfootnote{6284}to be baptised, married,
and buried; but he was too strait-laced. Let them have their liberty in
good sort, and go in good sort, modo non annos viginti aetatis suae
domi relinquant, as a good fellow said, so that they look not twenty
years younger abroad than they do at home, they be not spruce, neat,
angels abroad, beasts, dowdies, sluts at home; but seek by all means to
please and give content to their husbands: to be quiet above all
things, obedient, silent and patient; if they be incensed, angry, chid
a little, their wives must not \authorfootnote{6285}cample again, but take it in good
part. An honest woman, I cannot now tell where she dwelt, but by report
an honest woman she was, hearing one of her gossips by chance complain
of her husband's impatience, told her an excellent remedy for it, and
gave her withal a glass of water, which when he brawled she should hold
still in her mouth, and that toties quoties, as often as he chid; she
did so two or three times with good success, and at length seeing her
neighbour, gave her great thanks for it, and would needs know the
ingredients, \authorfootnote{6286}she told her in brief what it was, fair water, and
no more: for it was not the water, but her silence which performed the
cure. Let every froward woman imitate this example, and be quiet within
doors, and (as \authorfootnote{6287}M. Aurelius prescribes) a necessary caution it is
to be observed of all good matrons that love their credits, to come
little abroad, but follow their work at home, look to their household
affairs and private business, oeconomiae incumbentes, be sober,
thrifty, wary, circumspect, modest, and compose themselves to live to
their husbands' means, as a good housewife should do,
\authorfootnote{6288}Quae studiis gavisa coli, partita labores
Fallet opus cantu, formae assimulata coronae
Cura puellaris, circum fusosque rotasque
Cum volvet, \&c.

Howsoever 'tis good to keep them private, not in prison;
\authorfootnote{6289}Quisquis custodit uxorem vectibus et seris,
Etsi sibi sapiens, stultus est, et nihil sapit.

Read more of this subject, Horol. princ. lib. 2. per totum. Arnisaeus,
polit. Cyprian, Tertullian, Bossus de mulier. apparat. Godefridus de
Amor. lib. 2. cap. 4. Levinus Lemnius cap. 54. de institut. Christ.
Barbaras de re uxor. lib. 2. cap. 2. Franciscus Patritius de institut.
Reipub. lib. 4. Tit. 4. et 6. de officio mariti et uxoris, Christ.
Fonseca Amphitheat. Amor. cap. 45. Sam. Neander, \&c.

These cautions concern him; and if by those or his own discretion
otherwise he cannot moderate himself, his friends must not be wanting
by their wisdom, if it be possible, to give the party grieved
satisfaction, to prevent and remove the occasions, objects, if it may
be to secure him. If it be one alone, or many, to consider whom he
suspects or at what times, in what places he is most incensed, in what
companies. \authorfootnote{6290}Nevisanus makes a question whether a young physician
ought to be admitted in cases of sickness, into a new-married man's
house, to administer a julep, a syrup, or some such physic. The
Persians of old would not suffer a young physician to come amongst
women. \authorfootnote{6291}Apollonides Cous made Artaxerxes cuckold, and was after
buried alive for it. A goaler in Aristaenetus had a fine young
gentleman to his prisoner; \authorfootnote{6292}in commiseration of his youth and
person he let him loose, to enjoy the liberty of the prison, but he
unkindly made him a cornuto. Menelaus gave good welcome to Paris a
stranger, his whole house and family were at his command, but he
ungently stole away his best beloved wife. The like measure was offered
to Agis king of Lacedaemon, by \authorfootnote{6293} Alcibiades an exile, for his good
entertainment, he was too familiar with Timea his wife, begetting a
child of her, called Leotichides: and bragging moreover when he came
home to Athens, that he had a son should be king of the Lacedaemonians.

If such objects were removed, no doubt but the parties might easily be
satisfied, or that they could use them gently and entreat them well,
not to revile them, scoff at, hate them, as in such cases commonly they
do, 'tis a human infirmity, a miserable vexation, and they should not
add grief to grief, nor aggravate their misery, but seek to please, and
by all means give them content, by good counsel, removing such
offensive objects, or by mediation of some discreet friends. In old
Rome there was a temple erected by the matrons to that \authorfootnote{6294}Viriplaca
Dea, another to Venus verticorda, quae maritos uxoribus reddebat
benevolos, whither (if any difference happened between man and wife)
they did instantly resort: there they did offer sacrifice, a white
hart, Plutarch records, sine felle, without the gall, (some say the
like of Juno's temple) and make their prayers for conjugal peace;
before some \authorfootnote{6295} indifferent arbitrators and friends, the matter was
heard between man and wife, and commonly composed. In our times we want
no sacred churches, or good men to end such controversies, if use were
made, of them. Some say that precious stone called \authorfootnote{6296}beryllus,
others a diamond, hath excellent virtue, contra hostium injurias, et
conjugatos invicem conciliare, to reconcile men and wives, to maintain
unity and love; you may try this when you will, and as you see cause.

If none of all these means and cautions will take place, I know not
what remedy to prescribe, or whither such persons may go for ease,
except they can get into the same \authorfootnote{6297}Turkey paradise, Where they
shall have as many fair wives as they will themselves, with clear eyes,
and such as look on none but their own husbands, no fear, no danger of
being cuckolds; or else I would have them observe that strict rule of
\authorfootnote{6298}Alphonsus, to marry a deaf and dumb man to a blind woman. If this
will not help, let them, to prevent the worst, consult with an
\authorfootnote{6299}astrologer, and see whether the significators in her horoscope
agree with his, that they be not in signis et partibus odiose
intuentibus aut imperantibus, sed mutuo et amice antisciis et
obedientibus, otherwise (as they hold) there will be intolerable
enmities between them: or else get them sigillum veneris, a
characteristical seal stamped in the day and hour of Venus, when she is
fortunate, with such and such set words and charms, which Villanovanus
and Leo Suavius prescribe, ex sigillis magicis Salomonis, Hermetis,
Raguelis, \&c., with many such, which Alexis, Albertus, and some of our
natural magicians put upon us: ut mulier cum aliquo adulterare non
possit, incide de capillis ejus, \&c., and he shall surely be gracious
in all women's eyes, and never suspect or disagree with his own wife so
long as he wears it. If this course be not approved, and other remedies
may not be had, they must in the last place sue for a divorce; but that
is somewhat difficult to effect, and not all out so fit. For as
Felisacus in his tract de justa uxore urgeth, if that law of
Constantine the Great, or that of Theodosius and Valentinian,
concerning divorce, were in use in our times, innumeras propemodum
viduas haberemus, et coelibes viros, we should have almost no married
couples left. Try therefore those former remedies; or as Tertullian
reports of Democritus, that put out his eyes, \authorfootnote{6300}because he could
not look upon a woman without lust, and was much troubled to see that
which he might not enjoy; let him make himself blind, and so he shall
avoid that care and molestation of watching his wife. One other
sovereign remedy I could repeat, an especial antidote against jealousy,
an excellent cure, but I am not now disposed to tell it, not that like
a covetous empiric I conceal it for any gain, but some other reasons, I
am not willing to publish it: if you be very desirous to know it, when
I meet you next I will peradventure tell you what it is in your ear.

This is the best counsel I can give; which he that hath need of, as
occasion serves, may apply unto himself. In the mean time,-\li{dii talem
terris avertite pestem}\authorlatintrans{6301}, as the proverb is, from heresy, jealousy
and frenzy, good Lord deliver us.
}
