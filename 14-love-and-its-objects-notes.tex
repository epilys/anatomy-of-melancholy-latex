\setmarginnote{4414}{Encom. Moriae leviores esse nugas quam ut Theologum deceant.}
\setmarginnote{4415}{Lib. 8. Eloquent, cap 14. de affectibus mortalium vitio fit qui praeclara quaeque in pravos usus vertunt.}
\setmarginnote{4416}{Quoties de amatoriis mentio facta est, tam vehementer excandui; tam severa tristitia violari aures meas obsceno sermone nolui, ut me tanquam unam ex Philosophis intuerentur.}
\setmarginnote{4417}{Martial. In Brutus' presence Lucretia blushed and laid my book aside; when he retired, she took it up again and read.}
\setmarginnote{4418}{Lib. 4. of civil conversation.}
\setmarginnote{4419}{Si male locata est opera scribendo, ne ipsi locent in legendo.}
\setmarginnote{4420}{Med. epist. l. 1. ep. 14. Cadmus Milesius teste Suida. de hoc Erotico Amore. 14. libros scripsit nec me pigebit in gratiam adolescentum hanc scribere epistolam.}
\setmarginnote{4421}{Comment. in 2. Aeneid.}
\setmarginnote{4422}{Meros amores meram impudicitiam sonare videtur nisi, \&c.}
\setmarginnote{4423}{Ser. 8.}
\setmarginnote{4424}{Quod risum et eorum amores commemoret.}
\setmarginnote{4425}{Quum multa ei objecissent quod Critiam tyrannidem docuisset, quod Platonem juraret loquacem sophistem, \&c. accusationem amoris nullam fecerunt. Ideoque honestus amor, \&c.}
\setmarginnote{4426}{Carpunt alii Platonicam majestatem quod amori nimium indulserit, Dicearchus et alii; sed male. Omnis amor honestus et bonus, et amore digni qui bene dicunt de Amore.}
\setmarginnote{4427}{Med. obser. lib. 2. cap. 7. de admirando amoris affectu dicturus; ingens patet campus ei philosophicus, quo saepe homines ducuntur ad insaniam, libeat modo vagari, \&c. Quae non ornent modo, sed fragrantia et succulentia jucunda plenius alant, \&c.}
\setmarginnote{4428}{Lib. 1. praefat. de amoribus agens relaxandi animi causa laboriosissimis studiis fatigati; quando et Theologi se his juvari et juvare illaesis moribus volunt?}
\setmarginnote{4429}{Hist. lib. 12. cap. 34.}
\setmarginnote{4430}{Praefat. quid quadragenario convenit cum amore? Ego vero agnosco amatorium scriptum mihi non convenire: qui jam meridiem praetergressus in vesperem feror. Aeneas Sylvius praefat.}
\setmarginnote{4431}{Ut severiora studia iis amaenitatibus lector condire possit. Accius.}
\setmarginnote{4432}{Discum quam philosophum audire malunt.}
\setmarginnote{4433}{In Som. Sip. e sacrario suo tum ad cunas nutricum sapientes eliminarunt, solas aurium delitias profitentes.}
\setmarginnote{4434}{Babylonius et Ephesius, qui de Amore scripserunt, uterque amores Myrrhae, Cyrenes, et Adonidis. Suidas.}
\setmarginnote{4435}{Pet. Aretine dial. Ital.}
\setmarginnote{4436}{Hor. He has accomplished every point who has joined the useful to the agreeable.}
\setmarginnote{4437}{Legendi cupidiores, quam ego scribendi, saith Lucian.}
\setmarginnote{4438}{Plus capio voluptatis inde, quam spectandis in theatro ludis.}
\setmarginnote{4439}{Prooemio in Isaim. Multo major pars Milesias fabulas revolventium quam Platonis libros.}
\setmarginnote{4440}{This he took to be his only business, that the plays which he wrote should please the people.}
\setmarginnote{4441}{In vita philosophus, in Epigram, amator, in Epistolis petulanus, in praeceptis severus.}
\setmarginnote{4442}{The poet himself should be chaste and pious, but his verses need not imitate him in these respects; they may therefore contain wit and humour.}
\setmarginnote{4443}{This that I write depends sometimes upon the opinion and authority of others: nor perhaps am I frantic, I only follow madmen: But thus far I may be deranged: we have all been so at some one time, and yourself, I think, art sometimes insane, and this man, and that man, and I also.}
\setmarginnote{4444}{I am mortal, and think no humane action unsuited to me.}
\setmarginnote{4445}{Mart.}
\setmarginnote{4446}{Ovid.}
\setmarginnote{4447}{Isago. ad sac. scrip. cap. 13.}
\setmarginnote{4448}{Barthius notis in Coelestinam, ludum Hisp.}
\setmarginnote{4449}{Ficinus Comment. c. 17. Amore incensi inveniendi amoris, aniorem quaesivimus et invenimus.}
\setmarginnote{4450}{Author Coelestinae Barth. interprete. That, overcome by the solicitations of friends, who requested me to enlarge and improve my volumes, I have devoted my otherwise reluctant mind to the labour; and now for the sixth time have I taken up my pen, and applied myself to literature very foreign indeed to my studies and professional occupations, stealing a few hours from serious pursuits, and devoting them, as it were, to recreation.}
\setmarginnote{4451}{Hor. lib. 1. Ode 34. I am compelled to reverse my sails, and retrace my former course.}
\setmarginnote{4452}{Although I was by no means ignorant that new calumniators would not be wanting to censure my new introductions.}
\setmarginnote{4453}{Haec praedixi ne quis temere nos putaret scripsisse de amorum lenociniis, de praxi, fornicationibus, adulteriis, \&c.}
\setmarginnote{4454}{Taxando et ab his deterrendo humanam lasciviam et insaniam, sed et remedia docendo: non igitur candidus lector nobis succenseat, \&c. Commonitio erit juvenibus haec, hisce ut abstineant magis, et omissa lascivia quae homines reddit insanos, virtutis incumbant studiis (Aeneas Sylv.) et curam amoris si quis nescit hinc poterit scire.}
\setmarginnote{4455}{Martianus Capella lib. 1. de nupt. philol. virginali suffusa rubore oculos peplo obnubens, \&c.}
\setmarginnote{4456}{Catullus. What I tell you, do you tell to the multitude, and make this treatise gossip like an old woman.}
\setmarginnote{4457}{Viros nudos castae feminae nihil a statuis distare.}
\setmarginnote{4458}{Hony soit qui mal y pense.}
\setmarginnote{4459}{Praef. Suid.}
\setmarginnote{4460}{O Arethusa smile on this my last labour.}
\setmarginnote{4461}{Exerc. 301. Campus amoris maximus et spinis obsitus, nec levissimo pede transvolandus.}
\setmarginnote{4462}{Grad. 1. cap. 29. Ex Platone, primae et communissimae perturbationes ex quibus ceterae oriuntur et earum sunt pedissequae.}
\setmarginnote{4463}{Amor est voluntarius affectus et desiderium re bona fruendi.}
\setmarginnote{4464}{Desiderium optantis, amor eorum quibus fruimur; amoris principium, desiderii finis, amatum adest.}
\setmarginnote{4465}{Principio l. de amore. Operae pretium est de amore considerare, utrum Deus, an Daemon, an passio quaedam animae, an partim Deus, partim Daemon, passio partim, \&c. Amor est aetus animi bonum desiderans.}
\setmarginnote{4466}{Magnus Daemon convivio.}
\setmarginnote{4467}{Boni pulchrique fruendi desiderium.}
\setmarginnote{4468}{Godefridus, l. 1. cap. 2 Amor est delectatio cordis, alicujus ad aliquid, propter aliquod desiderium in appertendo, et gaudium perfruendo per desiderium currens, requiescens per gaudium.}
\setmarginnote{4469}{Non est amor desiderium aut appetitus ut ab omnibus hactenus traditim; nam cum potimur amata re, non manet appetitus; est igitur affectus quo cum re amata aut unimur, aut unionem perpetuamus.}
\setmarginnote{4470}{Omnia appetunt bonum.}
\setmarginnote{4471}{Terram non vis malam, malam segetem, sed bonam arborem, equum bonum, \&c.}
\setmarginnote{4472}{Nemo amore capitur nisi qui fuerit ante forma specieque delectatus.}
\setmarginnote{4473}{Amabile objectum amoris et scopus, cujus adeptio est finis, cujus gratia amamus. Animus enim aspirat ut eo fruator, et formam boni habet et praecipue videtur et placet. Picolomineus, grad. 7. cap. 2. et grad. 8. cap. 35.}
\setmarginnote{4474}{Forma est vitalis fulgor ex ipso bono manans per ideas, semina, rationes, umbras effusus, animos excitans ut per bonum in unum redigantur.}
\setmarginnote{4475}{Pulchritudo est perfectio compositi ex congruente ordine, mensura et ratione partium consurgens, et venustas inde prodiens gratia dicitur et res omnes pulchrae gratiosae.}
\setmarginnote{4476}{Gratia et pulchritudo ita suaviter animos demulcent, ita vehementer alluciunt, et admirabiliter connectuntur, ut in inum confundant et distingui non possunt et sunt tanquam radii et splendores divini solis in rebus variis vario modo fulgentes.}
\setmarginnote{4477}{Species pulchrituninis hauriuntur oculis, auribus, aut concipiuntur interna mente.}
\setmarginnote{4478}{Nihil hine magis animos conciliat quam musica, pulchrae, aedes, \&c.}
\setmarginnote{4479}{In reliquis sensibus voluptas, in his pulchritudo et gratia.}
\setmarginnote{4480}{Lib. 4. de divinis. Convivio Platonis.}
\setmarginnote{4481}{Duae Veneres duo amores; quarum una antiquior et sine matre, coelo nata, quam coelestem Venerem nuncupamus; altera vero junior a Jove et Dione prognata, quam vulgarem Venerem vocamus.}
\setmarginnote{4482}{Alter ad superna erigit, alter deprimit ad inferna.}
\setmarginnote{4483}{Alter excitat hominem ad divinam pulchritudinem lustrandam, cujus causa philosophiae studia et justitiae, \&c.}
\setmarginnote{4484}{Omnis creatura cum bona sit, et bene amari potest et male.}
\setmarginnote{4485}{Duas civitates duo faciunt amores; Jerusalem facit amor Dei, Babylonem amor saeculi; unusquisque se quid amet interroget, et inveniet unde sit civis.}
\setmarginnote{4486}{Alter mari ortus, ferox, varius, fluctuans, inanis, juvenum, mare referens, \&c. Alter aurea catena coelo demissa bonum furorem mentibus mittens, \&c.}
\setmarginnote{4487}{Tria sunt, quae amari a nobis bene vel male possunt; Deus, proximus, mundus; Deus supra nos; juxta nos proximus; infra nos mundus. Tria Deus, duo proximus, unum mundus habet, \&c.}
\setmarginnote{4488}{Ne confundam vesanos et foedos amores beatis, sceleratum cum puro divino et vero, \&c.}
\setmarginnote{4489}{Fonseca cap. 1. Amor ex Augustini forsan lib. 11. de Civit. Dei. Amore inconcussus stat mundus, \&c.}
\setmarginnote{4490}{Alciat.}
\setmarginnote{4491}{Porta Vitis laurum non amat, nec ejus odorem; si prope crescat, enecat. Lappus lenti adversatur.}
\setmarginnote{4492}{Sympathia olei et myrti ramorum et radicum se complectentium. Mizaldus secret. cent. l. 47.}
\setmarginnote{4493}{Theocritus. eidyll. 9.}
\setmarginnote{4494}{Mantuan.}
\setmarginnote{4495}{Charitas munifica, qua mercamur de Deo regnum Dei.}
\setmarginnote{4496}{Polanus partit. Zanchius de natura Dei, c. 3. copiose de hoc amore Dei agit.}
\setmarginnote{4497}{Nich. Bellus, discurs. 28. de amatoribus, virtutem provocat, conservat pacem in terra, tranquillitatem in aere, ventis laetitiam, \&c.}
\setmarginnote{4498}{Camerarius Emb. 100. cen. 2.}
\setmarginnote{4499}{Dial. 3.}
\setmarginnote{4500}{Juven.}
\setmarginnote{4501}{Gen. 1.}
\setmarginnote{4502}{Caussinus.}
\setmarginnote{4503}{Theodoret e Plotino.}
\setmarginnote{4504}{Where charity prevails, sweet desire, joy, and love towards God are also present.}
\setmarginnote{4505}{Affectus nunc appetitivae potentiae, nunc rationalis, alter cerebro residet, alter hepate, corde, \&c.}
\setmarginnote{4506}{Cor varie inclinatur, nunc gaudens, nunc moerens; statim ex timore nascitur Zelotypia, furor, spes, desperatio.}
\setmarginnote{4507}{Ad utile sanitas refertur; utilium est ambitio, cupido desiderium potius quam amor excessus avaritia.}
\setmarginnote{4508}{Picolom. grad. 7. cap. 1.}
\setmarginnote{4509}{Lib. de amicit. utile mundanum, carnale jucundum, spirituale honestum.}
\setmarginnote{4510}{Ex. singulis tribus fit charitas et amicitia, quae respicit deum et proximum.}
\setmarginnote{4511}{Benefactores praecipue amamus. Vives 3. de anima.}
\setmarginnote{4512}{Jos. 7.}
\setmarginnote{4513}{Petronius Arbiter.}
\setmarginnote{4514}{Juvenalis.}
\setmarginnote{4515}{Job. Second, lib. sylvarum.}
\setmarginnote{4516}{Lucianus Timon.}
\setmarginnote{4517}{Pers.}
\setmarginnote{4518}{bust of a beautiful woman with the tail of a fish.}
\setmarginnote{4519}{Part. 1. sec. 2. memb. sub. 12.}
\setmarginnote{4520}{1 Tim. i. 8.}
\setmarginnote{4521}{Lips, epist. Camdeno.}
\setmarginnote{4522}{Leland of St. Edmondsbury.}
\setmarginnote{4523}{Coelum serenum, coelum visum foedum. Polid. lib. 1. de Anglia.}
\setmarginnote{4524}{Credo equidem vivos ducent e marmore vultus.}
\setmarginnote{4525}{Max. Tyrius, ser. 9.}
\setmarginnote{4526}{Part 1. sec. 2. memb. 3.}
\setmarginnote{4527}{Mart.}
\setmarginnote{4528}{Omnif. mag. lib. 12. cap. 3.}
\setmarginnote{4529}{De sale geniali, l. 3. c. 15.}
\setmarginnote{4530}{Theod. Prodromus, amor. lib. 3.}
\setmarginnote{4531}{Similitudo morum parit amicitiam.}
\setmarginnote{4532}{Vives 3. de anima.}
\setmarginnote{4533}{Qui simul fecere naufragium, aut una pertulere vincula vel consilii conjurationisve societate junguntur, invicem amant: Brutum et Cassium invicem infensos Caesarianus dominatus conciliavit. Aemilius Lepidus et Julius Flaccus, quum essent inimicissimi, censores renunciati simultates illico deposuere. Scultet. cap. 4. de causa amor.}
\setmarginnote{4534}{Papinius.}
\setmarginnote{4535}{Isocrates demonico praecipit ut quum alicujus amicitiam vellet illum laudet, quod laus initium amoris sit, vituperatio simultatum.}
\setmarginnote{4536}{Suspect, lect. lib. 1. cap. 2.}
\setmarginnote{4537}{The priest of wisdom, perpetual dictator, ornament of literature, wonder of Europe.}
\setmarginnote{4538}{Oh incredible excellence of genius, \&c., more comparable to gods' than man's, in every respect, we venerate your writings on bended knees, as we do the shield that fell from heaven.}
\setmarginnote{4539}{Isa. xlix.}
\setmarginnote{4540}{Rara est concordia fratrum.}
\setmarginnote{4541}{Grad. 1. cap. 22.}
\setmarginnote{4542}{Vives 3. de anima, ut paleam succinum sic formam amor trahit.}
\setmarginnote{4543}{Sect. seq.}
\setmarginnote{4544}{Nihil divinius homine probo.}
\setmarginnote{4545}{James iii. 10.}
\setmarginnote{4546}{Gratior est pulchro veniens e corpore virtus.}
\setmarginnote{4547}{Oral. 18. deformes plerumque philosophi ad id quod in aspectum cadit ea parte elegantes quae oculos fugit.}
\setmarginnote{4548}{43 de consol.}
\setmarginnote{4549}{Causa ei paupertatis, philosophia, sicut plerisque probitas fuit.}
\setmarginnote{4550}{Ablue corpus et cape regis animum, et in eam fortunam qua dignus es continentiam istam profer.}
\setmarginnote{4551}{Vita ejus.}
\setmarginnote{4552}{Qui prae divitiis humana spernunt, nec virtuti locum putant nisi opes affluant. Q. Cincinnatus consensu patrum in dictatorem Romanum electus.}
\setmarginnote{4553}{Curtius.}
\setmarginnote{4554}{Edgar Etheling, England's darling.}
\setmarginnote{4555}{Morum suavitas, obvia comitas, prompta officia mortalium animos demerentur.}
\setmarginnote{4556}{Epist. lib. 8. Semper amavi ut tu scis, M. Brutum propter ejus summum ingenium, suavissimos mores, singularem probitatem et constantiam: nihil est, mihi crede, virtute formosius, nihil amabilius.}
\setmarginnote{4557}{Ardentes amores excitaret, si simulacrum ejus ad oculos penetraret. Plato Phaedone.}
\setmarginnote{4558}{Epist. lib. 4. Validissime diligo virum rectum, disertum, quod apud me potentissimum est.}
\setmarginnote{4559}{Est quaedam pulchritudo justitiae quam videmus oculis cordis, amamus, et exardescimus, ut in martyribus, quum eorum membra bestiae lacerarent, etsi alias deformes, \&c.}
\setmarginnote{4560}{Lipsius manuduc. ad Phys. Stoic. lib. 3. diff. 17, solus sapiens pulcher.}
\setmarginnote{4561}{Fortitudo et prudentia pulchritudinis laudem praecipue merentur.}
\setmarginnote{4562}{Franc. Belforist. in hist. an. 1430.}
\setmarginnote{4563}{Erat autem foede deformis, et ea forma, qua citius pueri terreri possent, quam invitari ad osculum puellae.}
\setmarginnote{4564}{Deformis iste etsi videatur senex, divinum animum habet.}
\setmarginnote{4565}{Fulgebat vultu suo: fulgor et divina majestas homines ad se trahens.}
\setmarginnote{4566}{She excelled all others in beauty.}
\setmarginnote{4567}{Praefat. bib. vulgar.}
\setmarginnote{4568}{Pars inscrip. Tit. Livii statuae Patavii.}
\setmarginnote{4569}{A true love's knot.}
\setmarginnote{4570}{Stobaeus e Graeco.}
\setmarginnote{4571}{Solinus, pulchri nulla est facies.}
\setmarginnote{4572}{O dulcissimi laquei, qui tam feliciter devinciunt, ut etiam a vinctis diligantur, qui a gratiis vincti sunt, cupiunt arctius deligari et in unum redigi.}
\setmarginnote{4573}{Statius.}
\setmarginnote{4574}{He loved him as he loved his own soul, 1 Sam. xv. 1. Beyond the love of women.}
\setmarginnote{4575}{Virg. 9. Aen. Qui super exanimem sese conjecit amicum confessus.}
\setmarginnote{4576}{Amicus animae dimidium, Austin, confess. 4. cap. 6. Quod de Virgilio Horatius, et serves animae dimidium meae.}
\setmarginnote{4577}{Plinius.}
\setmarginnote{4578}{Illum argento et auro, illum ebore, marmore effingit, et nuper ingenti adhibito auditorio ingentem de vita ejus librum recitavit. epist. lib. 4. epist. 68.}
\setmarginnote{4579}{Lib. iv. ep. 61. Prisco suo; Dedit mihi quantum potuit maximum, daturas amplius si potuisset. Tametsi quid homini dari potest majus quum gloria, laus, et aeternitas? At non erunt fortasse quae scripsit. Ille tamen scripsit tanquam essent futura.}
\setmarginnote{4580}{For. genus irritabile vatum.}
\setmarginnote{4581}{Lib. 13 de Legibus. Magnam enim vim habent, \&c.}
\setmarginnote{4582}{Peri tamen studio et pietate conscribendae vitae ejus munus suscepi, et postquam sumptuosa condere pro fortuna non licuit, exiguo sed eo forte liberalis ingenii monumento justa sanctissimo cineri solventur.}
\setmarginnote{4583}{1 Sam. xxv. 3.}
\setmarginnote{4584}{Esther, iii. 2.}
\setmarginnote{4585}{Amm. Marcellinus, l. 14.}
\setmarginnote{4586}{Ut mundus duobus polis sustentatur: ita lex Dei, amore Dei et proximi; duobus his fundamentis vincitur; machina mundi corruit, si una de polis turbatur; lex perit divina si una ex his.}
\setmarginnote{4587}{8 et 9 libro.}
\setmarginnote{4588}{Ter. Adelph. 4, 5.}
\setmarginnote{4589}{De amicit.}
\setmarginnote{4590}{Charitas parentum dilui nisi detestabili scelere non potest, lapidum fornicibus simillima, casura, nisi se invicem sustentaret. Seneca.}
\setmarginnote{4591}{It is sweet to die for one's country.}
\setmarginnote{4592}{Dii immortales, dici non potest quantum charitatis nomen illud habet.}
\setmarginnote{4593}{Ovid. Fast.}
\setmarginnote{4594}{Anno 1347. Jacob Mayer. Annal. Fland. lib. 12.}
\setmarginnote{4595}{Tally.}
\setmarginnote{4596}{Lucianus Toxari. Amicitia ut sol in mundo, \&c.}
\setmarginnote{4597}{Vit. Pompon. Attici.}
\setmarginnote{4598}{Spencer, Faerie Queene, lib. 5. cant. 9. staff. 1, 2.}
\setmarginnote{4599}{Siracides.}
\setmarginnote{4600}{Plutarch, preciosum numisma.}
\setmarginnote{4601}{Xenophon, verus amicus praestantissima possessio.}
\setmarginnote{4602}{Epist. 52.}
\setmarginnote{4603}{Greg. Per amorem Dei, proximi gignitur; et per hunc amorem proximi, Dei nutritur.}
\setmarginnote{4604}{Picolomineus, grad. 7. cap. 27. hoc felici amoris nodo ligantur familiae civitates, \&c.}
\setmarginnote{4605}{Veras absolutas haec parit virtutes, radix omnium virtutum, mens et spiritus.}
\setmarginnote{4606}{Divino calore animos incendit, incensos purgat, purgatos elevat ad Deum, Deum placat, hominem Deo conciliat. Bernard.}
\setmarginnote{4607}{Ille inficit, hic perficit, ille deprimit, hic elevat; hic tranquillitatem ille curas parit: hic vitam recte informat, ille deformat \&c.}
\setmarginnote{4608}{Boethius, lib. 2. met. 8.}
\setmarginnote{4609}{Deliquium patitur charitas, odium ejus loco succedit. Basil. 1. ser. de instit. mon.}
\setmarginnote{4610}{Nodum in scirpo quaerentes.}
\setmarginnote{4611}{Hircanaeque admorunt ubera tigres.}
\setmarginnote{4612}{Heraclitus.}
\setmarginnote{4613}{Si in gehennam abit, pauperem qui non alat: quid de eo fiet qui pauperem denudat? Austin.}
\setmarginnote{4614}{Jovius, vita ejus.}
\setmarginnote{4615}{Immortalitatem beneficio literarum, immortali gloriosa quadam cupiditate concupivit. Quod cives quibus benefecisset perituri, moenia ruitura, etsi regio sumptu aedificata, non libri.}
\setmarginnote{4616}{Plutarch, Pericle.}
\setmarginnote{4617}{Tullius, lib. 1. de legibus.}
\setmarginnote{4618}{Gen. xxxv. 8.}
\setmarginnote{4619}{Hor.}
\setmarginnote{4620}{Durum genus sumus.}
\setmarginnote{4621}{The sister of justice, honour inviolate, and naked truth.}
\setmarginnote{4622}{Tull. pro Rose. Mentiri vis causa mea? ego vero cupide et libenter mentiar tua causa; et si quando me vis perjurare, ut paululum tu compendii facias, paratum fore scito.}
\setmarginnote{4623}{Gallienus in Treb. Pollio lacera, occide, mea mente irascere. Rabie jecur incendente feruntur praecipites, Vopiscus of Aurelian. Tantum fudit sanguinis quantum quis vini potavit.}
\setmarginnote{4624}{Evangelii tubam belli tubam faciunt; in pulpitis pacem, in colloquiis bellum suadent.}
\setmarginnote{4625}{Psal. xiii. 1.}
\setmarginnote{4626}{De bello Judaico, lib. 6. c. 16. Puto si Romani contra hos venire tardassent, aut hiatu terrae devorandam fuisse civitatem, aut diluvio perituram, aut fulmina ac Sodoma cum incendio passuram, ob desperatum populi, \&c.}
\setmarginnote{4627}{Benefacit animae suae vir misericors.}
\setmarginnote{4628}{Concordia magnae res crescunt, discordia maximae dilabuntur.}
\setmarginnote{4629}{Lipsius.}
