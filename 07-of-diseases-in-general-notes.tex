\setmarginnote{820}{Magnum miraculum.}
\setmarginnote{821}{Mundi epitome, naturae deliciae.}
\setmarginnote{822}{Finis rerum omnium, cui sublunaria serviunt. Scalig. exercit. 365. sec. 3. Vales. de sacr. Phil. c. 5.}
\setmarginnote{823}{Ut in numismate Caesaris imago, sic in homine Dei.}
\setmarginnote{824}{Gen. 1.}
\setmarginnote{825}{Imago mundi in corpore, Dei in anima. Exemplumque dei quisque est in imagine parva.}
\setmarginnote{826}{Eph. iv. 24.}
\setmarginnote{827}{Palan terius.}
\setmarginnote{828}{Psal. xlix. 20.}
\setmarginnote{829}{Lascivia superat equum, impudentia canem, astu vulpem, furore leonem. Chrys. 23. Gen.}
\setmarginnote{830}{Gen. iii. 13.}
\setmarginnote{831}{Ecclus. iv. 1, 2, 3, 4, 5, 8.}
\setmarginnote{832}{Gen. iii. 17.}
\setmarginnote{833}{Illa cadens tegmen manibus decussit, et una perniciem immisit miseris mortalibus atram. Hesiod. 1. oper.}
\setmarginnote{834}{Hom. 5. ad pop. Antioch.}
\setmarginnote{835}{Psal. cvii. 17.}
\setmarginnote{836}{Pro. i. 27.}
\setmarginnote{837}{Quod autem crebrius bella concutiant, quod sterilitas et fames solicitudinem cumulent, quod saevientibus morbis valitudo frangitur, quod humanum genus luis populatione vastatur; ob peccatum omnia. Cypr.}
\setmarginnote{838}{Si raro desuper pluvia descendat, si terra situ pulveris squalleat, si vix jejunas et pallidas heibas sterilis gleba producat, si turbo vineam debilitet, \&c. Cypr.}
\setmarginnote{839}{Mat. xiv. 3.}
\setmarginnote{840}{Philostratus, lib. 8. vit. Apollonii. Injustitiam ejus, et sceleratas nuptias, et caeteta quae praeter rationem fecerat, morborum causas dixit.}
\setmarginnote{841}{16.}
\setmarginnote{842}{18.}
\setmarginnote{843}{20.}
\setmarginnote{844}{Verse 17.}
\setmarginnote{845}{28. Deos quos diligit, castigat.}
\setmarginnote{846}{Isa. v. 13. Verse 15.}
\setmarginnote{847}{Nostrae salutis avidus continenter aures vellicat, ac calamitate subinde nos exercet. Levinus Lemn. l. 2. c. 29. de occult, nat. mir.}
\setmarginnote{848}{Vexatio dat Intellectum. Isa. xiviii. 19.}
\setmarginnote{849}{In sickness the mind recollects itself.}
\setmarginnote{850}{Lib. 7. Cum judicio, mores et facta recognoscit et se intuetur. Dum fero languorem, fero religionis amorem. Expers languoris non sum memor hujus amoris.}
\setmarginnote{851}{Summum esse totius philosophiae, ut tales esse perseveremus, quales nos futures esse infirmi profitemur.}
\setmarginnote{852}{Petrarch.}
\setmarginnote{853}{Prov. iii. 12.}
\setmarginnote{854}{Hor. Epis. lib. 1. 4.}
\setmarginnote{855}{Deut. viii. 11. Qui stat videat ne cadat.}
\setmarginnote{856}{Quanto majoribus beneficiis a Deo cumulatur, tanto obligatiorem se debitorem fateri.}
\setmarginnote{857}{Boterus de Inst. urbium.}
\setmarginnote{858}{Lege hist, relationem Lod. Frois de rebus Japonicis ad annum 1596.}
\setmarginnote{859}{Guicciard. descript. Belg. anno 1421.}
\setmarginnote{860}{Giraldus Cambrens.}
\setmarginnote{861}{Janus Dousa, ep. lib. 1. car. 10. And we perceive nothing, except the dead bodies of cities in the open sea.}
\setmarginnote{862}{Munster l. 3. Cos. cap. 462.}
\setmarginnote{863}{Buchanan. Baptist.}
\setmarginnote{864}{Homo homini lupus, homo homini daemon.}
\setmarginnote{865}{Ovid. de Trist. l. 5. Eleg.}
\setmarginnote{866}{Miscent aconita novercae.}
\setmarginnote{867}{Lib. 2 Epist. 2. ad Donatum.}
\setmarginnote{868}{Eze. xviii. 2.}
\setmarginnote{869}{Hor. l. 3. Od. 6.}
\setmarginnote{870}{2 Tim. iii. 2.}
\setmarginnote{871}{Eze. xviii. 31. Thy destruction is from thyself.}
\setmarginnote{872}{21 Macc. iii. 12.}
\setmarginnote{873}{Part. 1. Sec. 2. Memb. 2.}
\setmarginnote{874}{Nequitia est quae te non sinet esse senem.}
\setmarginnote{875}{Homer. Iliad.}
\setmarginnote{876}{Intemperantia, luxus, ingluvies, et infinita hujusmodi flagitia, quae divinas poenas merentur. Crato.}
\setmarginnote{877}{Fern. Path. l. 1. c. 1. Morbus est affectus contra, naturam corpori insides.}
\setmarginnote{878}{Fusch. Instit. l. 3. sect. 1. c. 3. a quo primum vitiatur actio.}
\setmarginnote{879}{Dissolutio foederis in corpore, ut sanitas est consummatio.}
\setmarginnote{880}{Lib. 4. cap. 2. Morbus est habitus contra naturam, qui usum ejus, \&c.}
\setmarginnote{881}{Cap. 11. lib. 7.}
\setmarginnote{882}{Horat. lib. 1. ode 3. Emaciation, and a new cohort of fevers broods over the earth.}
\setmarginnote{883}{Cap. 50. lib. 7. Centum et quinque vixit annos sine ullo incommodo.}
\setmarginnote{884}{Intus mulso, foras oleo.}
\setmarginnote{885}{Exemplis genitur. praefixis Ephemer. cap. de infirmitat.}
\setmarginnote{886}{Qui, quoad pueritae ultimam memoriam recordari potest non meminit se aegrotum decubuisse.}
\setmarginnote{887}{Lib. de vita longa.}
\setmarginnote{888}{Oper. et. dies.}
\setmarginnote{889}{See Fernelius Path. lib. 1. cap. 9, 10, 11, 12. Fuschius Instit. l. 3. sect. 1. c. 7. Wecker. Synt.}
\setmarginnote{890}{Praefat. de morbis capitis. In capite ut variae habitant partes, ita variae querelae ibi eveniunt.}
\setmarginnote{891}{Of which read Heurnius, Montaltus, Hildesheim, Quercetan, Jason Pratensis, \&c.}
\setmarginnote{892}{Cap. 2. de melanchol.}
\setmarginnote{893}{Cap. 2. de Phisiologia sagarum: Quod alii minus recte fortasse dixerint, nos examinare, melius dijudicare, corrigere studeamus.}
\setmarginnote{894}{Cap. 4. de mol.}
\setmarginnote{895}{Art. Med. 7.}
\setmarginnote{896}{Plerique medici uno complexu perstringunt hos duos morbos, quod ex eadem causa oriantur, quodque magnitudine et modo solum distent, et alter gradus ad alterum existat. Jason Pratens.}
\setmarginnote{897}{Lib. Med.}
\setmarginnote{898}{Pars maniae mihi videtur.}
\setmarginnote{899}{Insanus est, qui aetate debita, et tempore debito per se, non momentaneam et fugacem, ut vini, solani, Hyoscyami, sed confirmatam habet impotentiam bene operandi circa intellectum. lib. 2. de intellectione.}
\setmarginnote{900}{Of which read Felix Plater, cap. 3. de mentis alienatione.}
\setmarginnote{901}{Lib. 6. cap. 11.}
\setmarginnote{902}{Lib. 3. cap. 16.}
\setmarginnote{903}{Cap. 9. Art. med.}
\setmarginnote{904}{De praestig. Daemonum, l. 3. cap. 21.}
\setmarginnote{905}{Observat. lib. 10. de morbis cerebri, cap. 15.}
\setmarginnote{906}{Hippocrates lib. de insania.}
\setmarginnote{907}{Lib. 8. cap. 22. Homines interdum lupos feri; et contra.}
\setmarginnote{908}{Met. lib. 1.}
\setmarginnote{909}{Cap. de Man.}
\setmarginnote{910}{Ulcerata crura, sitis ipsis adest immodica, pallidi, lingua sicca.}
\setmarginnote{911}{Cap. 9. art. Hydrophobia.}
\setmarginnote{912}{Lib. 3. cap. 9.}
\setmarginnote{913}{Lib. 7. de Venenis.}
\setmarginnote{914}{Lib. 3. cap. 13. de morbis acutis.}
\setmarginnote{915}{Spicel. 2.}
\setmarginnote{916}{Sckenkius, 7 lib. de Venenis.}
\setmarginnote{917}{Lib. de Hydrophobia.}
\setmarginnote{918}{Observat. lib. 10. 25.}
\setmarginnote{919}{Lascivam Choream. To. 4. de morbis amentium. Tract. 1.}
\setmarginnote{920}{Eventu ut plurimum rem ipsam comprobante.}
\setmarginnote{921}{Lib. 1. cap. de Mania.}
\setmarginnote{922}{Cap. 3. de mentis alienat.}
\setmarginnote{923}{Cap. 4. de mel.}
\setmarginnote{924}{PART. 3.}
\setmarginnote{925}{De quo homine securitas, de quo certum gaudium? quocunque se convertit, in terrenis rebus amaritudinem animi inveniet. Aug. in Psal. viii. 5.}
\setmarginnote{926}{Job. i. 14.}
\setmarginnote{927}{Omni tempore Socratem eodem vultu videri, sive domum rediret, sive domo egrederetur.}
\setmarginnote{928}{Lib. 7. cap. 1. Natus in florentissima totius orbis civitate, nobilissimis parentibus, corpores vires habuit et rarissimas animi dotes, uxorem conapicuam, pudicam, felices liberos, consulare decus, sequentes triumphos, \&c.}
\setmarginnote{929}{Aelian.}
\setmarginnote{930}{Homer. Iliad.}
\setmarginnote{931}{Lipsius, cent. 3. ep. 45, ut coelum, sic nos homines sumus: illud ex intervallo nubibus obducitur et obscuratur. In rosario flores spinis intermixti. Vita similis aeri, udum modo, sudum, tempestas, serenitas: ita vices rerum sunt, praemia gaudiis, et sequaces curae.}
\setmarginnote{932}{Lucretius, l. 4. 1124.}
\setmarginnote{933}{Prov. xiv. 13. Extremum gaudii luctas occupat.}
\setmarginnote{934}{Natalitia inquit celebrantur, nuptiae hic sunt; at ibi quid celebratur quod non dolet, quod non transit?}
\setmarginnote{935}{Apuleius 4. florid. Nihil quicquid homini tam prosperum divinitus datum, quin ei admixtum sit aliquid difficultatis ut etiam amplissima quaqua laetitia, subsit quaepiam vel parva querimonia conjugatione quadam mellis, et fellis.}
\setmarginnote{936}{Caduca nimirum et fragilia, et puerilibus consentanea crepundiis sunt ista quae vires et opes humanae vocantur, affluunt subito, repente delabuntur, nullo in loco, nulla in persona, stabilibus nixa radicibus consistunt, sed incertissimo flatu fortunae quos in sublime extulerunt improviso recursu destitutos in profundo miseriarum valle miserabiliter immergunt. Valerius, lib. 6. cap. 11.}
\setmarginnote{937}{Huic seculo parum aptus es, aut potius omnium nostrorum conditionem ignoras, quibus reciproco quodam nexu, \&c. Lorchanus Gollobelgicus, lib. 3. ad annum 1598.}
\setmarginnote{938}{Horsum omnia studia dirigi debent, ut humana fortiter feramus.}
\setmarginnote{939}{2 Tim. ii. 3.}
\setmarginnote{940}{Epist. 96. lib. 10. Affectus frequentes contemptique morbum faciunt. Distillatio una nec adhuc in morem adaucta, tussim facit, assidua et violenta pthisim.}
\setmarginnote{941}{Calidum ad octo: frigidum ad octo. Una hirundo non facit aestatem.}
\setmarginnote{942}{Lib. 1. c. 6.}
\setmarginnote{943}{Fuschius, l. 3. sec. 1. cap. 7. Hildesheim, fol. 130.}
\setmarginnote{944}{Psal. xxxix. 13.}
\setmarginnote{945}{De Anima. Turpe enim est homini ignorare sui corporis (ut ita dicam) aedificium, praesertim cum ad valetudinem et mores haec cognitio plurimum conducat.}
\setmarginnote{946}{De usu part.}
\setmarginnote{947}{History of man.}
\setmarginnote{948}{D. Crooke.}
\setmarginnote{949}{In Syntaxi.}
\setmarginnote{950}{De Anima.}
\setmarginnote{951}{Istit. lib. 1.}
\setmarginnote{952}{Physiol. l. 1, 2.}
\setmarginnote{953}{Anat. l. 1. c. 18.}
\setmarginnote{954}{In Micro. succos, sine quibus animal sustentari non potest.}
\setmarginnote{955}{Morbosos humores.}
\setmarginnote{956}{Spiritalis anima.}
\setmarginnote{957}{Laurentius, cap. 20, lib. 1. Anat.}
\setmarginnote{958}{In these they observe the beating of the pulse.}
\setmarginnote{959}{Cujus est pars simularis a vi cutifica ut interiora muniat. Capivac. Anat. pag. 252.}
\setmarginnote{960}{Anat. lib. 1. c. 19. Celebris est pervulgata partium divisio principes et ignobiles partes.}
\setmarginnote{961}{D. Crooke out of Galen and others.}
\setmarginnote{962}{Vos vero veluti in templum ac sacrarium quoddam vos duci putetis, \&c. Suavis et utilis cognitio.}
\setmarginnote{963}{Lib. 1. cap. 12. sect. 5.}
\setmarginnote{964}{Haec res est praecipue digna admiratione, quod tanta affectuum varietate cietur cor, quod omnes retristes et laetae statim corda feriunt et movent.}
\setmarginnote{965}{Physio. l. 1. c. 8.}
\setmarginnote{966}{Ut orator regi: sic pulmo vocis instrumentum annectitur cordi, \&c. Melancth.}
\setmarginnote{967}{De anim. c. 1.}
\setmarginnote{968}{Scalig. exerc. 307. Tolet. in lib. de anima. cap. 1. \&c.}
\setmarginnote{969}{l. De anima. cap. 1.}
\setmarginnote{970}{Tuscul. quaest.}
\setmarginnote{971}{Lib. 6. Doct. Va. Gentil. c. 13. pag. 1216.}
\setmarginnote{972}{Aristot.}
\setmarginnote{973}{Anima quaeque intelligimus, et tamen quae sit ipsa intelligere non valemus.}
\setmarginnote{974}{Spiritualem animam a reliquis distinctam tuetur, etiam in cadavere inhaerentem post mortem per aliquot menses.}
\setmarginnote{975}{Lib. 3. cap. 31.}
\setmarginnote{976}{Coelius, lib. 2. c. 31. Plutarch, in Grillo Lips. Cen. 1. ep. 50. Jossius de Risu et Fletu, Averroes, Campanella, \&c.}
\setmarginnote{977}{Phillip. de Anima. ca. 1. Coelius, 20. antiq. cap. 3. Plutarch. de placit. philos.}
\setmarginnote{978}{De vit. et mort. part. 2. c. 3, prop. l. de vit. et mort. 2. c. 22.}
\setmarginnote{979}{Nutritio est alimenti transmutatio, viro naturalis. Scal. exerc. 101, sec. 17.}
\setmarginnote{980}{See more of Attraction in Scal. exer. 343.}
\setmarginnote{981}{Vita consistit in calido et humido.}
\setmarginnote{982}{Too bright an object destroys the organ.}
\setmarginnote{983}{Lumen est actus perspicui. Lumen a luce provenit, lux est in corpore lucido.}
\setmarginnote{984}{In Phaedon. (Notes 984-997 appear in the order 986, 984, 987, 985 in the original-KTH.)}
\setmarginnote{985}{De pract. Philos. 4.}
\setmarginnote{986}{Satur. 7. c. 14.}
\setmarginnote{987}{Lac. cap. 8. de opif. Dei, I.}
\setmarginnote{988}{Lib. 19. cap. 2.}
\setmarginnote{989}{Phis. l. 5. c. 8.}
\setmarginnote{990}{Exercit. 280.}
\setmarginnote{991}{T. W. Jesuite, in his Passions of the Minde.}
\setmarginnote{992}{Velcurio.}
\setmarginnote{993}{Nervi a spiritu moventur, spritus ab anima. Melanct.}
\setmarginnote{994}{Velcurio. Jucundum et anceps subjectum.}
\setmarginnote{995}{Goclenius in Ψυχολ. pag. 302. Bright in Phys. Scrib. l. 1. David Crusius, Melancthon, Hippius Hernius, Levinus Lemnius, \&c.}
\setmarginnote{996}{Lib. an mores sequantur, \&c.}
\setmarginnote{997}{Caesar. 6. com.}
\setmarginnote{998}{Read Aeneas Gazeus dial. of the immortality of the Soul.}
\setmarginnote{999}{Ovid. Met. 15. We, who may take up our abode in wild beasts, or be lodged in the breasts of cattle.}
\setmarginnote{1000}{In Gallo. Idem.}
\setmarginnote{1001}{Nicephorus, hist. lib. 10. c. 35.}
\setmarginnote{1002}{Phaedo.}
\setmarginnote{1003}{Claudian, lib. 1. de rap. Proserp.}
\setmarginnote{1004}{Besides, we observe that the mind is born with the body, grows with it, and decays with it.}
\setmarginnote{1005}{Haec quaestio multos per annos varie, ac mirabiliter impugnata, \&c.}
\setmarginnote{1006}{Colerus, ibid.}
\setmarginnote{1007}{De eccles. dog. cap. 16.}
\setmarginnote{1008}{Ovid. 4. Met. The bloodless shades without either body or bones wanter.}
\setmarginnote{1009}{Bonorum lares, malorum vero larvas et lemures.}
\setmarginnote{1010}{Some say at three days, some six weeks, others otherwise.}
\setmarginnote{1011}{Melancthon.}
\setmarginnote{1012}{Nihil in intellectu, quod non prius fuerat in sensu. Velcurio.}
\setmarginnote{1013}{The pure part of the conscience.}
\setmarginnote{1014}{Quod tibi fieri non vis, alteri ne feceris.}
\setmarginnote{1015}{Res ab intellectu monstratas recipit, vel rejicit; approbat, vel improbat, Philip. Ignoti nulla cupido.}
\setmarginnote{1016}{Melancthon. Operationes plerumque ferae, etsi libera sit illa in essentia sua.}
\setmarginnote{1017}{In civilibus libera, sed non in spiritualibus Osiander.}
\setmarginnote{1018}{Tota voluntas aversa a Deo. Omnis homo mendax.}
\setmarginnote{1019}{Virg. We are neither able to contend against them, nor only to make way.}
\setmarginnote{1020}{Vel propter ignorantium, quod bonis studiis non sit instructa mens ut debuit, aut divinis praeceptis exculta.}
\setmarginnote{1021}{Med. Ovid.}
\setmarginnote{1022}{Ovid.}
\setmarginnote{1023}{Seneca, Hipp.}
\setmarginnote{1024}{Melancholicos vocamus, quos exuperantia vel pravitas Melancholiae ita male habet, ut inde insaniant vel in omnibus, vel in pluribus iisque manifestis sive ad rectam rationem, voluntate pertinent, vel electionem, vel intellectus operationes.}
\setmarginnote{1025}{Pessimum et pertinacissimum morbum qui homines in bruta degenerare cogit.}
\setmarginnote{1026}{Panth. Med.}
\setmarginnote{1027}{Angor animi in una contentione defixus, absque febre.}
\setmarginnote{1028}{Cap. 16. l. 1.}
\setmarginnote{1029}{Eorum definitio morbus quid non sit potius quam quid sit, explicat.}
\setmarginnote{1030}{Animae functiones imminuuntur in fatuitate, tolluntur in mania, depravantur solum in melancholia. Herc. de Sax. cap. 1. tract. de Melanch.}
\setmarginnote{1031}{Cap. 4. de mel.}
\setmarginnote{1032}{Per consensum sive per essentiam.}
\setmarginnote{1033}{Cap. 4. de mel.}
\setmarginnote{1034}{Sec. 7. de mor. vulgar. lib. 6.}
\setmarginnote{1035}{Spicel. de melancholia.}
\setmarginnote{1036}{Cap. 3. de mel. Pars affecta cerebrum sive per consensum, sive per cerebrum contingat, et procerum auctoritate et ratione stabilitur.}
\setmarginnote{1037}{Lib. de mel. Cor vero vicinitatis ratione una afficitur, acceptum transversum ac stomachus cum dorsali spina, \&c.}
\setmarginnote{1038}{Lib. 1. cap. 10. Subjectum est cerebrum interius.}
\setmarginnote{1039}{Raro quisquam tumorem effugit lienis, qui hoc morbo afficitur, Piso. Quis affectus.}
\setmarginnote{1040}{See Donat. ab Altomar.}
\setmarginnote{1041}{Facultas imaginandi, non cogitandi, nec memorandi laesa hic.}
\setmarginnote{1042}{Lib. 3. Fen. 1. Tract. 4. cap. 8.}
\setmarginnote{1043}{Lib. 3. cap. 5.}
\setmarginnote{1044}{Lib. Med. cap. 19. part. 2. Tract. 15. cap. 2.}
\setmarginnote{1045}{Hildesheim, spicel. 2 de Melanc. fol. 207, et fol. 127. Quandoque etiam rationalis si affectus inveteratus sit.}
\setmarginnote{1046}{Lib. posthumo de Melanc. edit. 1620. Deprivatur fides, discursus, opinio, \&c. per vitium Imaginationes, ex Accidenti.}
\setmarginnote{1047}{Qui parvum caput habent, insensati plerique sunt. Arist. in physiognomia.}
\setmarginnote{1048}{Areteus, lib. 3. cap. 5.}
\setmarginnote{1049}{Qui prope statum sunt. Aret. Mediis convenit aetatibus, Piso.}
\setmarginnote{1050}{De quartano.}
\setmarginnote{1051}{Lib. 1. part. 2. cap. 11.}
\setmarginnote{1052}{Primus ad Melancholiam non tam moestus sed et hilares, jocosi, cachinnantes, irrisores, et, qui plerumque praerubri sunt.}
\setmarginnote{1053}{Qui sunt subtilis ingenii, et multae perspicacitatis de facili incidunt in Melancholiam, lib. 1. cont. tract. 9.}
\setmarginnote{1054}{Nunquam sanitate mentis excidit aut dolore capitur. Erasm.}
\setmarginnote{1055}{In laud. calvit.}
\setmarginnote{1056}{Vacant conscientiae carnificina, nec pudefiunt, nec verentur, nec dilacerantur millibus curarum, quibus tota vita obnoxia est.}
\setmarginnote{1057}{Lib. 1. tract. 3. contradic. 18.}
\setmarginnote{1058}{Lib. 1. cont. 21.}
\setmarginnote{1059}{Bright, ca. 16.}
\setmarginnote{1060}{Lib. 1. cap. 6. de sanit. tuenda.}
\setmarginnote{1061}{Quisve aut qualis sit humor aut quae istius differentiae, et quomodo gignantur in corpore, scrutandum, hac enim re multi veterum laboraverunt, nec facile accipere ex Galeno sententiam ob loquendi varietatem. Leon. Jacch. com. in 9. Rhasis, cap. 15. cap. 16. in 9. Rhasis.}
\setmarginnote{1062}{Lib. postum. de Melan. edit. Venetiis, 1620. cap. 7 et 8. Ab intemperie calida, humida, \&c.}
\setmarginnote{1063}{Secundum magis aut minus si in corpore fuerit, ad intemperiem plusquam corpus salubriter ferre poterit: inde corpus morbosum effitur.}
\setmarginnote{1064}{Lib. 1. controvers. cap. 21.}
\setmarginnote{1065}{Lib. 1. sect. 4, cap. 4.}
\setmarginnote{1066}{Concil. 26.}
\setmarginnote{1067}{Lib. 2. contradic. cap. 11.}
\setmarginnote{1068}{De feb. tract. diff. 2. cap. 1. Non est negandum ex hac fieri Melancholicos.}
\setmarginnote{1069}{In Syntax.}
\setmarginnote{1070}{Varie aduritur, et miscetur, unde variae amentium species, Melanct.}
\setmarginnote{1071}{Humor frigidus delirii causa, furoris calidus, \&c.}
\setmarginnote{1072}{Lib. 1. cap. 10. de affect. cap.}
\setmarginnote{1073}{Nigrescit hic humor, aliquando supercalefactus, aliquando super frigefactus, ca. 7.}
\setmarginnote{1074}{Humor hic niger aliquando praeter modum calefactus, et alias refrigeratus evadit: nam recentibus carbonibus ei quid simile accidit, qui durante flamma pellucidissime candent, ea extincta prorsus nigrescunt. Hippocrates.}
\setmarginnote{1075}{Guianerius, diff. 2. cap. 7.}
\setmarginnote{1076}{Non est mania, nisi extensa melancholia.}
\setmarginnote{1077}{Cap. 6. lib. 1.}
\setmarginnote{1078}{2 Ser. 2. cap. 9. Morbus hic est omnifarius.}
\setmarginnote{1079}{Species indefinitae sunt.}
\setmarginnote{1080}{Si aduratur naturalis melancholia, alia fit species, si sanguis, alia, si flavibilis alia, diversa a primis: maxima est inter has differentia, et tot Doctorum sententiae, quot ipsi numero sunt.}
\setmarginnote{1081}{Tract. de mel. cap. 7.}
\setmarginnote{1082}{Quaedam incipiens quaedam consummata.}
\setmarginnote{1083}{Cap. de humor. lib. de anima. Varie aduritur et miscetur ipsa melancholia, unde variae amentium species.}
\setmarginnote{1084}{Cap. 16. in. 9. Rasis.}
\setmarginnote{1085}{Laurentius, cap. 4. de mel.}
\setmarginnote{1086}{Cap. 13.}
\setmarginnote{1087}{480. et 116. consult. consil. 12.}
\setmarginnote{1088}{Hildesheim. spicil. 2. fol. 166.}
\setmarginnote{1089}{Trincavellius, tom. 2. consil. 15 et 16.}
\setmarginnote{1090}{Cap. 13, tract. posth. de melan.}
\setmarginnote{1091}{Guarion. cons. med. 2.}
\setmarginnote{1092}{Laboravit per essentiam et a toto corpore.}
\setmarginnote{1093}{Machiavel, \&c. Smithus de rep. Angl. cap. 8. lib. 1. Buscoldus, discur. polit. discurs. 5. cap. 7. Arist. l. 3. polit. cap. ult. Keckerm. alii, \&c.}
\setmarginnote{1094}{Lib. 6.}
\setmarginnote{1095}{Primo artis curitivae.}
\setmarginnote{1096}{Nostri primum sit propositi affectionum causas indagare; res ipsa hortari videtur, nam alioqui earum curatio, manca et inutilis esset.}
\setmarginnote{1097}{Path. lib. 1. cap. 11. Rerum cognoscere causas, medicis imprimis necessarium, sine qua nec morbum curare, nec praecavere licet.}
\setmarginnote{1098}{Tanta enim morbi varietas ac differentia ut non facile dignoscatur, unde initium morbus sumpserit. Melanelius e Galeno.}
\setmarginnote{1099}{Felix qui potuit rerum cognoscere causas.}
\setmarginnote{1100}{1 Sam. xvi. 14.}
\setmarginnote{1101}{Dan. v. 21.}
\setmarginnote{1102}{Lactant. instit. lib. 2. cap. 8.}
\setmarginnote{1103}{Mente captus, et summo animi moerore consumptus.}
\setmarginnote{1104}{Munster cosmog. lib. 4. cap. 43. De coelo substernebantur, tanquam insani de saxis praecipitati, \&c.}
\setmarginnote{1105}{Livius lib. 38.}
\setmarginnote{1106}{Gaguin. l. 3. c. 4. Quod Dionysii corpus discooperuerat, in insanam incidit.}
\setmarginnote{1107}{Idem lib. 9. sub. Carol. 6. Sacrorum contemptor, templi foribus effractis, dum D. Johannis argenteum simulacrum rapere contendit, simulacrum aversa facie dorsum ei versat, nec mora sacrilegus mentis inops, atque in semet insaniens in proprios artus desaevit.}
\setmarginnote{1108}{Giraldus Cambrensis, lib 1. c. 1. Itinerar. Cambriae.}
\setmarginnote{1109}{Delrio, tom. 3. lib. 6. sect. 3. quaest. 3.}
\setmarginnote{1110}{Psal. xlvi. 1.}
\setmarginnote{1111}{Lib. 8. cap. de Hierar.}
\setmarginnote{1112}{Claudian.}
\setmarginnote{1113}{De Babila Martyre.}
\setmarginnote{1114}{Lib. cap. 5. prog.}
\setmarginnote{1115}{Lib. 1. de Abditis rerum causis.}
\setmarginnote{1116}{Respons. med. 12. resp.}
\setmarginnote{1117}{1 Pet. v. 6.}
\setmarginnote{1118}{Lib. 1. c. 7. de orbis concordia. In nulla re major fuit altercatio, major obscuritas, minor opinionum concordia, quam de daemonibus et substantiis separatis.}
\setmarginnote{1119}{Lib. 3. de Trinit. cap. 1.}
\setmarginnote{1120}{Pererius in Genesin. lib. 4. in cap. 3. v. 23.}
\setmarginnote{1121}{See Strozzius Cicogna omnifariae. Mag. lib. 2. c. 15. Jo. Aubanus, Bredenbachius.}
\setmarginnote{1122}{Angelus per superbiam separatus a Deo, qui in veritate non stetit. Austin.}
\setmarginnote{1123}{Nihil aliud sunt Daemones quam nudae animae quae corpore deposito priorem miserati vitam, cognatis succurrunt commoti misericordia, \&c.}
\setmarginnote{1124}{De Deo Socratis. All those mortals are called Gods, who, the course of life being prudently guided and governed, are honoured by men with temples and sacrifices, as Osiris in Aegypt, \&c.}
\setmarginnote{1125}{He lived 500 years since.}
\setmarginnote{1126}{Apuleius: spiritus animalia sunt animo passibilia, mente rationalia, corpore aeria, tempore sempiterna.}
\setmarginnote{1127}{Nutriuntur, et excrementa habent, quod pulsata doleant solido percussa corpore.}
\setmarginnote{1128}{Whatever occupies space is corporeal:-spirit occupies space, \emph{therefore}, \&c. \&c.}
\setmarginnote{1129}{4 lib. 4. Theol. nat. fol. 535.}
\setmarginnote{1130}{Which has no roughness, angles, fractures, prominences, but is the most perfect amongst perfect bodies.}
\setmarginnote{1131}{Cyprianus in Epist. montes etiam et animalia transferri possunt: as the devil did Christ to the top of the pinnacle; and witches are often translated. See more in Strozzius Cicogna, lib. 3. cap. 4. omnif. mag. Per aera subducere et in sublime corpora ferre possunt, Biarmanus. Percussi dolent et uruntur in conspicuos cineres. Agrippa, lib. 3. cap. de occul. Philos.}
\setmarginnote{1132}{Agrippa, de occult. Philos. lib. 3. cap. 18.}
\setmarginnote{1133}{Part. 3. Sect. 2. Mem. 1. Subs. 1. Love Melancholy.}
\setmarginnote{1134}{By gazing steadfastly on the sun illuminated with his brightest rays.}
\setmarginnote{1135}{Genial. dierum. Ita sibi visum et compertum quum prius an essent ambigeret Fidem suam liberet.}
\setmarginnote{1136}{Lib. 1. de verit. Fidei. Benzo, \&c.}
\setmarginnote{1137}{Lib. de Divinatione et magia.}
\setmarginnote{1138}{Cap. 8. Transportavit in Livoniam cupiditate videndi, \&c.}
\setmarginnote{1139}{Sic Hesiodus de Nymphis vivere dicit. 10. aetates phaenicum vel. 9. 7. 20.}
\setmarginnote{1140}{Custodes hominum et provinciarum, \&c. tanto meliores hominibus, quanto hi brutis animantibus.}
\setmarginnote{1141}{Praesides Pastores, Gubernatores hominum, et illi animalium.}
\setmarginnote{1142}{Coveting nothing more than the admiration of mankind.}
\setmarginnote{1143}{Natura familiares ut canes hominibus multi aversantur et abhorrent.}
\setmarginnote{1144}{Ab nomine plus distant quam homo ab ignobilissimo verne, et tamen quidam ex his ab hominibus superantur ut homines a feris, \&c.}
\setmarginnote{1145}{Cibo et potu uti et venere cum hominibus ac tandem mori, Cicogna. l. part. lib. 2. c. 3.}
\setmarginnote{1146}{Plutarch. de defect. oraculorum.}
\setmarginnote{1147}{Lib. de Zilphis et Pigmeis.}
\setmarginnote{1148}{Dii gentium a Constantio prostigati sunt, \&c.}
\setmarginnote{1149}{Octovian. dial. Judaeorum deum fuisse Romanorum numinibus una cum gente captivum.}
\setmarginnote{1150}{Omnia spiritibus plena, et ex eorum concordia et discordia omnes boni et mali effectus promanant, omnia humana reguntur: paradoxa veterum de quo Cicogna. omnif. mag. l. 2. c. 3.}
\setmarginnote{1151}{Oves quas abacturus erat in quascunque formas vertebat Pausanias, Hyginus.}
\setmarginnote{1152}{Austin in l. 2. de Gen. ad literam cap. 17. Partim quia subtilioris sensus acumine, partim scientia calidiore vigent et experientia propter magnam longitudinem vitae, partim ab Angelis discunt, \&c.}
\setmarginnote{1153}{Lib. 3. omnif. mag. cap. 3.}
\setmarginnote{1154}{L. 18. quest.}
\setmarginnote{1155}{Quum tanti sit et tam profunda spiritum scientia, mirum non est tot tantasque res visu admirabiles ab ipsis patrari, et quidem rerum naturalium ope quas multo melius intelligunt, multoque peritius suis locis et temporibus applicare norunt, quam homo, Cicogna.}
\setmarginnote{1156}{Aventinus, quicquid interdiu exhauriebatur, noctu explebatur. Inde pavefacti cura tores, \&c.}
\setmarginnote{1157}{In lib. 2. de Anima text 29. Homerus discriminatim omnes spiritus daemones vocat.}
\setmarginnote{1158}{A Jove ad inferos pulsi, \&c.}
\setmarginnote{1159}{De Deo Socratis adest mihi divina sorte Daemonium quoddam a prima pueritia me secutum, saepe dissuadet, impellit nonnunquam instar ovis, Plato.}
\setmarginnote{1160}{Agrippa lib. 3. de occul. ph. c. 18. Zancb. Pictorus, Pererius Cicogna. l. 3. cap. 1.}
\setmarginnote{1161}{Vasa irae. c. 13.}
\setmarginnote{1162}{Quibus datum est nocere terrae et mari, \&c.}
\setmarginnote{1163}{Physiol. Stoicorum e Senec. lib. 1. cap. 28.}
\setmarginnote{1164}{Usque ad lunam animas esse aethereas vocarique heroas, lares, genios.}
\setmarginnote{1165}{Mart. Capella.}
\setmarginnote{1166}{Nihil vacuum ab his ubi vel capillum in aere vel aqua jaceas.}
\setmarginnote{1167}{Lib. de Zilp.}
\setmarginnote{1168}{Palingenius.}
\setmarginnote{1169}{Lib. 7. cap. 34 et 5. Syntax. art. mirab.}
\setmarginnote{1170}{Comment in dial. Plat. de amore, cap. 5. Ut sphaera quaelibet super nos, ita praestantiores habent habitatores suae sphaerae consortes, ut habet nostra.}
\setmarginnote{1171}{Lib. de Amica. et daemone med. inter deos et homines, dica ad nos et nostra aequaliter ad deos ferunt.}
\setmarginnote{1172}{Saturninas et Joviales accolas.}
\setmarginnote{1173}{In loca detrusi sunt infra caelestes orbes in aerem scilicet et infra ubi Judicio generali reservantur.}
\setmarginnote{1174}{q. 36. art. 9.}
\setmarginnote{1175}{Virg. 8. Eg.}
\setmarginnote{1176}{Aen. 4.}
\setmarginnote{1177}{Austin: hoc dixi, ne quis existimet habitare ibimala daemonia ubi Solem et Lunam et Stellas Deus ordinavit, et alibi nemo arbitraretur Daemonom coelis habitare cum Angelis suis unde lapsum credimus. Idem. Zanch. l. 4. c. 3. de Angel. mails. Pererius in Gen. cap. 6. lib. 8. in ver. 2.}
\setmarginnote{1178}{Perigram. Hierosol.}
\setmarginnote{1179}{Fire worship, or divination by fire.}
\setmarginnote{1180}{Domus Diruunt, muros dejiciunt, immiscent se turbinibus et procellis et pulverem instar columnae evehunt. Cicogna l. 5. c. 5.}
\setmarginnote{1181}{Quest. in Liv.}
\setmarginnote{1182}{De praestigiis daemonum. c. 16. Convelli culmina videmus, prosterni sata, \&c.}
\setmarginnote{1183}{De bello Neapolitano, lib. 5.}
\setmarginnote{1184}{Suffitibus gaudent. Idem Just. Mart. Apol. pro Christianis.}
\setmarginnote{1185}{In Dei imitationem, saith Eusebius.}
\setmarginnote{1186}{Dii gentium Daemonia, \&c. ego in eorum statuas pellexi.}
\setmarginnote{1187}{Et nunc sub divorum nomine coluntur a Pontificiis.}
\setmarginnote{1188}{Lib. 11. de rerum ver.}
\setmarginnote{1189}{Lib. 3. cap. 3. De magis et veneficis, \&c. Nereides.}
\setmarginnote{1190}{Lib. de Zilphis.}
\setmarginnote{1191}{Lib. 3.}
\setmarginnote{1192}{Pro salute hominum excubare se simulant, sed in eorum perniciem omnia moliuntur. Aust.}
\setmarginnote{1193}{Dryades, Oriades, Hamadryades.}
\setmarginnote{1194}{Elvas Olaus voc. at lib. 3.}
\setmarginnote{1195}{Part 1. cap. 19.}
\setmarginnote{1196}{Lib. 3. cap. 11. Elvarum choreas Olaus lib. 3. vocat saltum adeo profunde in terras imprimunt, ut locus insigni deinceps virore orbicularis sit, et gramen non pereat.}
\setmarginnote{1197}{Sometimes they seduce too simple men into their mountain retreats, where they exhibit wonderful sights to their marvelling eyes, and astonish their ears by the sound of bells, \&c.}
\setmarginnote{1198}{Lib. de Zilph. et Pigmaeus Olaus lib. 3.}
\setmarginnote{1199}{Lib. 7. cap. 14. Qui et in famulitio viris et feminis inserviunt, conclavia scopis purgant, patinas mundant, ligna portant, equos curant, \&c.}
\setmarginnote{1200}{Ad ministeria utuntur.}
\setmarginnote{1201}{Where treasure is hid (as some think) or some murder, or such like villainy committed.}
\setmarginnote{1202}{Lib. 16. de rerum varietat.}
\setmarginnote{1203}{Vel spiritus sunt hujusmodi damnatorum, vel e purgatorio, vel ipsi daemones, c. 4.}
\setmarginnote{1204}{Quidam lemures domesticis instrumentis noctu ludunt: patinas, ollas, cantharas, et alia vasa dejiciunt, et quidam voces emittunt, ejulant, risum emittunt, \&c. ut canes nigri, feles, variis formis, \&c.}
\setmarginnote{1205}{Epist. lib. 7.}
\setmarginnote{1206}{Meridionales Daemones Cicogna calls them, or Alastores, l. 3. cap. 9.}
\setmarginnote{1207}{Sueton. c. 69. in Caligula.}
\setmarginnote{1208}{Strozzius Cicogna. lib. 3. mag. cap. 5.}
\setmarginnote{1209}{Idem. c. 18.}
\setmarginnote{1210}{M. Carew. Survey of Cornwall, lib. 2. folio 140.}
\setmarginnote{1211}{Horto Geniali, folio 137.}
\setmarginnote{1212}{Part 1. c. 19. Abducunt eos a recta via, et viam iter facientibus intercludunt.}
\setmarginnote{1213}{Lib. 1. cap. 44. Daemonum cernuntur et audiuntur ibi frequentes illusiones, unde viatoribus cavendum ne ce dissocient, aut a tergo maneant, voces enim fingunt sociorum, ut a recto itinere abducant, \&c.}
\setmarginnote{1214}{Mons sterilis et nivosus, ubi intempesta nocte umbrae apparent.}
\setmarginnote{1215}{Lib. 2. cap. 21. Offendicula faciunt transeuntibus in via et petulanter ridet cum vel hominem vel jumentum ejus pedes atterere faciant, et maxime si homo maledictus et calcaribus saevint.}
\setmarginnote{1216}{In Cosmogr.}
\setmarginnote{1217}{Vestiti more metallicorum, gestus et opera eorum imitantur.}
\setmarginnote{1218}{Immisso in terrae carceres vento horribiles terrae motus efficiunt, quibus saepe non domus modo et turres, sed civitates integrae et insulae haustae sunt.}
\setmarginnote{1219}{Hierom. in 3. Ephes. Idem Michaelis. c. 4. de spiritibus. Idem Thyreus de locis infestis.}
\setmarginnote{1220}{Lactantius 2. de origins erroris cap. 15. hi maligni spiritus per omnem terram vagantur, et solatium perditionis suae perdendis hominibus operantur.}
\setmarginnote{1221}{Mortalium calamitates epulae sunt malorum daemonum, Synesius.}
\setmarginnote{1222}{Daminus mendacii a seipso deceptus, alios decipere cupit, adversarius humani generis, Inventor mortis, superbiae institutor, radix malitiae, scelerum caput, princeps omnium vitiorum, fuit inde in Dei contumeliam, hominum perniciem: de horum conatibus et operationibus lege Epiphanium. 2. Tom. lib. 2. Dionysium. c. 4. Ambros. Epistol. lib. 10. ep. et 84. August. de civ. Dei lib. 5. c. 9., lib. 8. cap. 22. lib. 9. 18. lib. 10. 21. Theophil. in 12. Mat. Pasil. ep. 141. Leonem Ser. Theodoret. in 11. Cor. ep. 22. Chrys. hom. 53. in 12. Gen. Greg. in 1. c. John. Barthol. de prop. l. 2. c. 20. Zanch. l. 4. de malis angelis. Perer. in Gen. l. 8. in c. 6. 2. Origen. saepe praeliis intersunt, itinera et negotia nostra quaecumque dirigunt, clandestinis subsidiis optatos saepe praebent successus, Pet. Mar. in Sam. \&c. Ruscam de Inferno.}
\setmarginnote{1223}{Et velut mancipia circumfert Psellus.}
\setmarginnote{1224}{Lib. de trans. mut. Malac. ep.}
\setmarginnote{1225}{Custodes sunt hominum, et eorum, ut nos animalium: tum et provinciis praepositi regunt auguriis, somniis, oraculis, pramiis, \&c.}
\setmarginnote{1226}{Lipsius, Physiol. Stoic, lib. 1. cap. 19.}
\setmarginnote{1227}{Leo Suavis. idem et Tritemius.}
\setmarginnote{1228}{They seek nothing more earnestly than the fear and admiration of men.}
\setmarginnote{1229}{It is scarcely possible to describe the impotent ardour with which these malignant spirits aspire to the honour of being divinely worshipped.}
\setmarginnote{1230}{Omnif. mag. lib. 2. cap. 23.}
\setmarginnote{1231}{Ludus deorum sumus.}
\setmarginnote{1232}{Lib. de anima et daemone.}
\setmarginnote{1233}{Quoties sit, ut Principes novitium aulicum divitiis et dignitatibus pene obruant, et multorum annorum ministrum, qui non semel pro hero periculum subiit, ne teruntio donent, \&c. Idem. Quod Philosophi non remunerentur, cum scurra et ineptus ob insulsum jocum saepe praemium reportet, inde fit, \&c.}
\setmarginnote{1234}{Lib de cruelt. Cadaver.}
\setmarginnote{1235}{Boissardus, c. 6 magia.}
\setmarginnote{1236}{Godelmanus, cap. 3. lib. 1 de Magis. idem Zanchius, lib. 4. cap. 10 et 11. de malis angelis.}
\setmarginnote{1237}{Nociva Melancholia furiosos efficit, et quandoque penitus interficit. G. Picolominens Idemque Zanch. cap. 10. lib. 4. si Deus permittat, corpora nostra movere possunt, alterare, quovis morborum et malorum genere afficere, imo et in ipsa penetrare et saevire.}
\setmarginnote{1238}{Inducere potest morbos et sanitates.}
\setmarginnote{1239}{Viscerum actiones potest inhibere latenter, et venenis nobis ignotis corpus inficere.}
\setmarginnote{1240}{Irrepentes corporibus occulto morbos fingunt, mentes terrent, membra distorquent. Lips. Phil. Stoic. l. 1. c. 19.}
\setmarginnote{1241}{De rerum ver. l. 16. c. 93.}
\setmarginnote{1242}{Quum mens immediate decipi nequit, premum movit phantasiam, et ita obfirmat vanis conceptibus aut ut ne quem facultati aestimativae rationi locum relinquat. Spiritus malus invadit animam, turbat sensus, in furorem conjicit. Austin. de vit. Beat.}
\setmarginnote{1243}{Lib. 3. Fen. 1. Tract. 4. c. 18.}
\setmarginnote{1244}{A Daemone maxime proficisci, et saepe solo.}
\setmarginnote{1245}{Lib. de incant.}
\setmarginnote{1246}{Caep. de mania lib. de morbis cerebri; Daemones, quum sint tenues et incomprehensibiles spiritus, se insinuare corporibus humanis possunt, et occulte in viscerribus operti, valetudinem vitiare, somniis animas terrere et mentes furoribus quatere. Insinuant se melancholicorum penetralibus, intus ibique considunt et deliciantur tanquam in regione clarissimorum siderum, coguntque animum furere.}
\setmarginnote{1247}{Lib. 1. cap. 6. occult. Philos. part 1. cap. 1. de spectris.}
\setmarginnote{1248}{Sine cruce et sanctificatione sic \& daemone obsessa. dial.}
\setmarginnote{1249}{Greg. pag. c. 9.}
\setmarginnote{1250}{Penult. de opific. Dei.}
\setmarginnote{1251}{Lib. 28. cap. 26. tom. 9.}
\setmarginnote{1252}{De Lamiis.}
\setmarginnote{1253}{Et quomodo venefici fiant enarrat.}
\setmarginnote{1254}{De quo plura legas in Boissardo, lib. 1. de praestig.}
\setmarginnote{1255}{Rex Jacobus, Daemonol. l. 1. c. 3.}
\setmarginnote{1256}{An university in Spain in old Castile.}
\setmarginnote{1257}{The chief town in Poland.}
\setmarginnote{1258}{Oxford and Paris, see finem P. Lombardi.}
\setmarginnote{1259}{Praefat. de magis et veneficis.}
\setmarginnote{1260}{Rotatum Pileum habebat, quo ventos violentos cieret, aerem turbaret, et in quam partem, \&c.}
\setmarginnote{1261}{Erastus.}
\setmarginnote{1262}{Ministerio hirci nocturni.}
\setmarginnote{1263}{Steriles nuptos et inhabiles, vide Petrum de Pallude, lib. 4. distinct. 34. Paulum Guiclandum.}
\setmarginnote{1264}{Infantes matribus suffurantur, aliis suppositivis in locum verorum conjectis.}
\setmarginnote{1265}{Milles.}
\setmarginnote{1266}{D. Luther, in primum praeceptum, et Leon. Varius, lib. 1. de Fascino.}
\setmarginnote{1267}{Lavat. Cicog.}
\setmarginnote{1268}{Boissardus de Magis.}
\setmarginnote{1269}{Daemon. lib. 3. cap. 3.}
\setmarginnote{1270}{Vide Philostratum, vita ejus; Boissardum de Magis.}
\setmarginnote{1271}{Nubrigenses lege lib. 1. c. 19. Vide Suidam de Paset. De Cruent. Cadaver.}
\setmarginnote{1272}{Erastus. Adolphus Scribanius.}
\setmarginnote{1273}{Virg. Aeneid. 4. Incantatricem describens: Haec se carminibus promittit solvere mentes. Quas velit, ast aliis duras immittere curas.}
\setmarginnote{1274}{Godelmanus, cap. 7. lib. 1. Nutricum mammas praesiccant, solo tactu podagram, Apoplexiam, Paralysin, et alios morbos, quos medicina curare non poterat.}
\setmarginnote{1275}{Factus inde Maniacus, spic. 2. fol. 147.}
\setmarginnote{1276}{Omnia philtra etsi inter se differant, hoc habent commune, quod hominem efficiant melancholicum. epist. 231. Scholtzii.}
\setmarginnote{1277}{De cruent. Cadaver.}
\setmarginnote{1278}{Astra regunt homines, et regit astra Deus.}
\setmarginnote{1279}{Chirom. lib. Quaeris a me quantum operantur astra? dico, in nos nihil astra urgere, sed animos praeclives trahere: qui sic tamen liberi sunt, ut si ducem sequantur rationem, nihil efficiant, sin vero naturam, id agere quod in brutis fere.}
\setmarginnote{1280}{Coelum vehiculum divinae virtutis, cujus mediante motu, lumine et influentia, Deus! elementaria corpora ordinat et disponit Th. de Vio. Cajetanus in Psa. 104.}
\setmarginnote{1281}{Mundus iste quasi lyra ab excellentissimo quodam artifice concinnata, quem qui norit mirabiles eliciet harmonias. J. Dee. Aphorismo 11.}
\setmarginnote{1282}{Medicus sine coeli peritia nihil est, \&c. nisi genesim sciverit, ne tantillum poterit. lib. de podag.}
\setmarginnote{1283}{Constellatio in causa est; et influentia coeli morbum hunc movet, interdum omnibus aliis amotis. Et alibi. Origo ejus a Coelo petenda est. Tr. de morbis amentium.}
\setmarginnote{1284}{Lib. de anima, cap. de humorib. Ea varietas in Melancholia, habet caelestes causas \conjunction{} \saturn{} et \jupiter{} in \libra{} \conjunction{} \mars{} et \leftmoon{} in \scorpio{}.}
\setmarginnote{1285}{Ex atra bile varii generantur morbi perinde ut ipse multum calidi aut frigidi in se habuerit, quum utrique suscipiendo quam aptissima sit, tametsi suapte natura frigida sit. Annon aqua sic afficitur a calore ut ardeat; et a frigore, ut in glaciem concrescat? et haec varietas distinctionum, alii flent, rident, \etc{}.}
\setmarginnote{1286}{Hanc ad intemperantiam gignendam plurimum confert \mars{} et \saturn positus, \etc{}.}
\setmarginnote{1287}{\mercury{} Quoties alicujus genitura in \virgo{} et \pisces{} adverso signo positus, horoscopum partiliter tenueret atque etiam a \mars{} vel \saturn{} \Square radio percussus fuerit, natus ab insania vexabitur.}
\setmarginnote{1288}{Qui \saturn{} et \mars{} habet, alterum in culmine, alterum imo coelo, cum in lucem venerit, melancholicus erit, a qua sanebitur, si \mercury{} illos irradiarit.}
\setmarginnote{1289}{Hac configuratione natus, Aut Lunaticus, aut mente captus.}
\setmarginnote{1290}{Ptolomaeus centiloquio, et quadripartito tribuit omnium melancholicorum symptoma siderum influentis.}
\setmarginnote{1291}{Arte Medica. accedunt ad has causas affectiones siderum. Plurimum incitant et provocant influentiae caelestes. Velcurio, lib. 4. cap. 15.}
\setmarginnote{1292}{Hildesheim, spicel. 2. de mel.}
\setmarginnote{1293}{Joh. de Indag. cap. 9. Montaltus, cap. 22.}
\setmarginnote{1294}{Caput parvum qui habent cerebrum et spiritus plerumque angustos, facile incident in Melancholiam rubicundi. Aetius. Idem Montaltus, c. 21. e Galeno.}
\setmarginnote{1295}{Saturnina a Rascetta per mediam manum decurrens, usque ad radicem montis Saturni, a parvis lineis intersecta, arguit melancholicos. Aphoris. 78.}
\setmarginnote{1296}{Agitantur miseriis, continuis inquietudinibus, neque unquam a solitudine liberi sunt, anxie affiguntur amarissimis intra cogitationibus, semper tristes, suspitiosi, meticulosi: cogitationes sunt, velle agrum colere, stagna amant et paludes, \&c. Jo. de Indagine, lib. 1.}
\setmarginnote{1297}{Caelestis Physiognom. lib. 10.}
\setmarginnote{1298}{Cap. 14. lib. 5. Idem maculae in ungulis nigrae, lites, rixas, melancholiam significant, ab humore in corde tali.}
\setmarginnote{1299}{Lib. 1. Path. cap. 11.}
\setmarginnote{1300}{Venit enim properata malis inopina senectus: et dolor aetatem jussit inesse meam. Boethius, met. 1. de consol. Philos.}
\setmarginnote{1301}{Cap. de humoribus, lib. de Anima.}
\setmarginnote{1302}{Necessarium accidens decrepitis, et inseparabile.}
\setmarginnote{1303}{Psal. xc. 10.}
\setmarginnote{1304}{Meteran. Belg. hist. lib. 1.}
\setmarginnote{1305}{Sunt morosi anxii, et iracundi et difficiles senes, si quaerimus, etiam avari, Tull. de senectute.}
\setmarginnote{1306}{Lib. 2. de Aulico. Senes avari, morosi, jactabundi, philauti, deliri, superstitiosi, auspiciosi, \&c. Lib. 3. de Lamiis, cap. 17. et 18.}
\setmarginnote{1307}{Solarium, opium lupiadeps, lacr. asini, \&c. sanguis infantum, \&c.}
\setmarginnote{1308}{Corrupta est iis ab humore Melancholico phantasia. Nymanus.}
\setmarginnote{1309}{Putant se laedere quando non laedunt.}
\setmarginnote{1310}{Qui haec in imaginationis vim referre conati sunt, atrae bilis, inanem prorsus laborem susceperunt.}
\setmarginnote{1311}{Lib. 3. cap. 4. omnif. mag.}
\setmarginnote{1312}{Lib. 1. cap. 11. path.}
\setmarginnote{1313}{Ut arthritici Epilep. \&c.}
\setmarginnote{1314}{Ut filii non tam possessionum quam morborum baeredes sint.}
\setmarginnote{1315}{Epist. de secretis artis et naturae, c. 7. Nam in hoc quod patres corrupti sunt, generant filios corruptae complexionis, et compositionis, et filii eorum eadem de causa se corrumpunt, et sic derivatur corruptio a patribus ad filios.}
\setmarginnote{1316}{Non tam (inquit Hippocrates) gibbos et cicatrices oris et corporis habitum agnoscis ex iis, sed verum incessum gestus, mores, morbos, \&c.}
\setmarginnote{1317}{Synagog. Jud.}
\setmarginnote{1318}{Affectus parentum in foetus transeunt, et puerorum malicia parentibus imputanda, lib. 4. cap. 3. de occult, nat. mirae.}
\setmarginnote{1319}{Ex pituitosis pituitosi, ex biliosis biliosi, ex lienosis et melancholicis melancholici.}
\setmarginnote{1320}{Epist. 174. in Scoltz. Nascitur nobiscum illa aliturque et una cum parentibus habemus malum hunc assem. Jo. Pelesius, lib. 2. de cura humanorum affectuum.}
\setmarginnote{1321}{Lib. 10. observat.}
\setmarginnote{1322}{Maginus Geog.}
\setmarginnote{1323}{Saepe non eundem, sed similem producit effectum, et illaeso parente transit. in nepotem.}
\setmarginnote{1324}{Dial. praefix. genituris Leovitii.}
\setmarginnote{1325}{Bodin. de rep. cap. de periodis reip.}
\setmarginnote{1326}{Claudius Abaville, Capuchion, in his voyage to Maragnan. 1614. cap. 45. Nemo fere aegrotus, sano omnes et robusto corpore, vivunt annos. 120, 140. sine Medicina. Idem Hector Boethius de insulis Orchad. et Damianus a Goes de Scandia.}
\setmarginnote{1327}{Lib. 4. c. 3. de occult. nat. mir. Tetricos plerumque filios senes progenerant et tristes, rarios exhilaratos.}
\setmarginnote{1328}{Coitus super repletionem pessimus, et filii qui tum gignuntur, aut morbosi sunt, aut stolidi.}
\setmarginnote{1329}{dial, praefix. Leovito.}
\setmarginnote{1330}{L. de ed. liberis.}
\setmarginnote{1331}{De occult. nat. mir. temulentae et stolidae mulieres liberos plerumque producunt sibi similes.}
\setmarginnote{1332}{Lib. 2, c. 8. de occult, nat. mir. Good Master Schoolmaster do not English this.}
\setmarginnote{1333}{De nat. mul. lib. 3. cap. 4.}
\setmarginnote{1334}{Buxdorphius, c. 31. Synag. Jud. Ezek. 18.}
\setmarginnote{1335}{Drusius obs. lib. 3. cap. 20.}
\setmarginnote{1336}{Beda. Eccl. hist. lib. 1. c. 27. respons. 10.}
\setmarginnote{1337}{Nam spiritus cerebri si tum male afficiantur, tales procreant, et quales fuerint affectus, tales filiorum: ex tristibus tristes, ex jucundis jucundi nascuntur, \&c.}
\setmarginnote{1338}{Fol. 129. mer. Socrates' children were fools. Sabel.}
\setmarginnote{1339}{De occul. nat. mir. Pica morbus mulierum.}
\setmarginnote{1340}{Baptista Porta, loco praed. Ex leporum intuitu plerique infantes edunt bifido superiore labello.}
\setmarginnote{1341}{Quasi mox in terram collapsurus, per omne vitam incedebat cum mater gravia ebrium hominem sic incedentem viderat.}
\setmarginnote{1342}{Civem facie cadaverosa, qui dixit, \&c.}
\setmarginnote{1343}{Optimum bene nasci, maxima para felicitatis nostrae bene nasci; quamobrem praeclere humano generi consultam videretur, si solis parentis bene habiti et sani, liberis operam darent.}
\setmarginnote{1344}{Infantes infirmi praecipitio necati. Bohemus, lib. 3. c. 3. Apud Lacones olim. Lipsius, epist. 85. cent. ad Belgas, Dionysio Villerio, si quos aliqua membrorum parte inutiles notaverint, necari jubent.}
\setmarginnote{1345}{Lib. 1. De veterum Scotorum moribus. Morbo comitiali, dementia, mania, lepra, \&c. aut simila labe, quae facile in prolem transmittitur, laborantes inter eos, ingenti facta indagine, inventos, ne gens foeda contagione laederetur, ex iis nata, castraverunt, mulieres hujusmodi procul a virorum consortio abregarunt, quod si harum aliqua concepisse inveniebatur, simul cum foetu nondum edito, defodiebatur viva.}
\setmarginnote{1346}{Euphormio Satyr.}
\setmarginnote{1347}{Fecit omnia delicta quae fieri possunt circa res sex non naturales, et eae fuerunt causae extrinsecae, ex quibus postea ortae sunt obstructiones.}
\setmarginnote{1348}{Path. I. l. c. 2. Maximam in gignendis morbis vim obtinet, pabulum, materiamque morbi suggerens: nam nec ab aere, nec a perturbationibus, vel aliis evidentibus causis morbi sunt, nisi consentiat corporis praeparatio, et humorum constitutio. Ut semel dicam, una gula est omnium morborum mater, etiamsi alius est genitor. Ab hac morbi sponte saepe emanant, nulla alia cogente causa.}
\setmarginnote{1349}{Cogan, Eliot, Vauhan, Vener.}
\setmarginnote{1350}{Frietagius.}
\setmarginnote{1351}{Isaac.}
\setmarginnote{1352}{Non laudatur quia melancholicum praebet alimentum.}
\setmarginnote{1353}{Male alit cervina (inquit Frietagius) crassissimum et atribilarium suppeditat alimentum.}
\setmarginnote{1354}{Lib. de subtiliss. dieta. Equina caro et asinina equinis danda est hominibus et asininis.}
\setmarginnote{1355}{Parum obsunt a natura Leporum. Bruerinus, l. 13. cap. 25. pullorum tenera et optima.}
\setmarginnote{1356}{Illaudabilis succi nauseam provocant.}
\setmarginnote{1357}{Piso. Altomar.}
\setmarginnote{1358}{Curio. Frietagius, Magninus, part. 3. cap. 17. Mercurialis, de affect, lib. I. c. 10. excepts all milk meats in Hypochondriacal Melancholy.}
\setmarginnote{1359}{Wecker, Syntax. theor. p. 2. Isaac, Bruer. lib. 15. cap. 30. et 31.}
\setmarginnote{1360}{Cap. 18. part. 3.}
\setmarginnote{1361}{Omni loco et omni tempore medici detestantur anguillas praesertim circa solstitium. Damnanturtum sanis tum aegris.}
\setmarginnote{1362}{Cap. 6. in his Tract of Melancholy.}
\setmarginnote{1363}{Optime nutrit omnium judicio inter primae notae pisces gustu praestanti.}
\setmarginnote{1364}{Non est dubium, quin pro variorum situ, ac natura, magnas alimentorum sortiantur differentias, alibi suaviores, alibi lutulentiores.}
\setmarginnote{1365}{Observat. 16. lib. 10.}
\setmarginnote{1366}{Pseudolus act. 3. scen. 2.}
\setmarginnote{1367}{Plautus, ibid.}
\setmarginnote{1368}{Quare rectius valedutini suae quisque consulet, qui lapsus priorum parentum memor, eas plane vel omiserit vel parce degustarit. Kersleius, cap. 4, de vero usu med.}
\setmarginnote{1369}{In Mizaldo de Horto, P. Crescent. Herbastein, \&c.}
\setmarginnote{1370}{Cap. 13. part. 3. Bright, in his Tract of Mel.}
\setmarginnote{1371}{Intellectum turbant, producunt insaniam.}
\setmarginnote{1372}{Audivi (inquit Magnin.) quod si quis ex iis per annum continue comedat, in insaniam caderet. cap. 13. Improbi succi sunt. cap. 12.}
\setmarginnote{1373}{De rerum varietat. In Fessa plerumque morbosi, quod fructus comedant ter in die.}
\setmarginnote{1374}{Cap. de Mel.}
\setmarginnote{1375}{Lib. 11. c. 3.}
\setmarginnote{1376}{Bright, c. 6. excepts honey.}
\setmarginnote{1377}{Hor. apud Scoltzium, consil. 186.}
\setmarginnote{1378}{Ne comedas crustam, choleram quia gignit adustam. Schol. Sal.}
\setmarginnote{1379}{Vinum turbidum.}
\setmarginnote{1380}{Ex vini patentis bibitione, duo Alemani in uno mense melancholici facti sunt.}
\setmarginnote{1381}{Hildesheim, spicel. fol. 273.}
\setmarginnote{1382}{Crassum generat sanguinem.}
\setmarginnote{1383}{About Danzig in Spruce, Hamburgh, Leipsig.}
\setmarginnote{1384}{Henricus Abrmcensis.}
\setmarginnote{1385}{Potus tum salubris tum jucundus, l. 1.}
\setmarginnote{1386}{Galen l. 1. de san. tuend. Cavendae sunt aquae quae ex stagnis hauriuntur, et quae turbidae and male olentes, \&c.}
\setmarginnote{1387}{Innoxium reddit et bene olentum.}
\setmarginnote{1388}{Contendit haec vitia coctione non emendari.}
\setmarginnote{1389}{Lib. de bonitate aquae, hydropem auget, febres putridas, splenem, tusses, nocet oculis, malum habitum corporis et colorem.}
\setmarginnote{1390}{Mag. Nigritatem inducit si pecora biberint.}
\setmarginnote{1391}{Aquae nivibus coactae strumosos faciunt.}
\setmarginnote{1392}{Cosmog. l. 3. cap. 36.}
\setmarginnote{1393}{Method, hist. cap. 5. Balbutiunt Labdoni in Aquitania ob aquas, atque hi morbi ab acquis in corpora derivantur.}
\setmarginnote{1394}{Edulia ex sanguine et suffocato parta. Hildesheim.}
\setmarginnote{1395}{Cupedia vero, placentae, bellaria, commentaque alia curiosa pistorum et coquorum, gustui servientium conciliant morbos tum corpori tum animo insanibiles. Philo Judaeus, lib. de victimis. P. Jov. vita ejus.}
\setmarginnote{1396}{As lettuce steeped in wine, birds fed with fennel and sugar, as a Pope's concubine used in Avignon. Stephan.}
\setmarginnote{1397}{Animae negotium illa facessit, et de templo Dii immundum stabulum facit. Peletius, 10. c.}
\setmarginnote{1398}{Lib. 11. c. 52. Homini cibus utilissimus simplex, acervatio cirborum pestifera, et condimenta perniciosa, multos morbos multa fercula ferunt.}
\setmarginnote{1399}{31. Dec. 2. c. Nihil deterius quam si tempus justo longius comedendo protrahatur, et varia ciborum genera conjungantur: inde morborum scaturigo, quae ex repugnantia humorum oritur.}
\setmarginnote{1400}{Path. l. 1. c. 14.}
\setmarginnote{1401}{Juv. Sat. 5.}
\setmarginnote{1402}{Nimia repletio ciborum facit melancholicum.}
\setmarginnote{1403}{Comestio superflua cibi, et potus quantitas nimia.}
\setmarginnote{1404}{Impura corpora quanto magis nutris, tanto magis laedis: putrefacit enim alimentum vitiosus humor.}
\setmarginnote{1405}{Vid. Goclen. de portentosis coenis, \&c. puteani Com.}
\setmarginnote{1406}{Amb. lib. de Jeju. cap. 14. They who invite us to a supper, only conduct us to our tomb.}
\setmarginnote{1407}{Juvenal. The highest-priced dishes afford the greatest gratification.}
\setmarginnote{1408}{Guiccardin.}
\setmarginnote{1409}{Na. quaest. 4. ca. ult. fastidio est lumen gratuitum, dolet quod sole, quod spiritum emere non possimus, quod hic aer non emptus ex facili, \&c. adeo nihil placet, nisi quod carum est.}
\setmarginnote{1410}{Ingeniosi ad Gulam.}
\setmarginnote{1411}{Olim vile mancipium, nunc in omni aestimatione, nunc ars haberi caepta, \&c.}
\setmarginnote{1412}{Epist. 28. l. 7. Quorum in ventre ingenium, in patinis, \&c.}
\setmarginnote{1413}{In lucem coenat. Sertorius.}
\setmarginnote{1414}{Seneca.}
\setmarginnote{1415}{Mancipia gulae, dapes non sapore sed sumptu aestimantes. Seneca, consol. ad Helvidium.}
\setmarginnote{1416}{Saevientia guttura satiare non possunt fluvii et maria, Aeneas Sylvius, de miser. curial.}
\setmarginnote{1417}{Plautus.}
\setmarginnote{1418}{Hor. lib. 1. Sat. 3.}
\setmarginnote{1419}{Diei brevitas conviviis, noctis longitudo stupris conterebratur.}
\setmarginnote{1420}{Et quo plus capiant, irritamenta excogitantur.}
\setmarginnote{1421}{Fores portantur ut ad convivium reportentur, repleri ut exhauriant, et exhauriri ut bibant. Ambros.}
\setmarginnote{1422}{Ingentia vasa velut ad ostentationem, \&c.}
\setmarginnote{1423}{Plautus.}
\setmarginnote{1424}{Lib. 3. Anthol. c. 20.}
\setmarginnote{1425}{Gratiam conciliant potando.}
\setmarginnote{1426}{Notis ad Caesares.}
\setmarginnote{1427}{Lib. de educandis principum liberis.}
\setmarginnote{1428}{Virg. Ae. 1.}
\setmarginnote{1429}{Idem strenui potatoris Episcopi Sacellanus, cum ingentem pateram exhaurit princeps.}
\setmarginnote{1430}{Bohemus in Saxonia. Adeo immoderate et immodeste ab ipsis bibitur, ut in compotationibus suis non cyathis solum et cantharis sat infundere possint, sed impletum mulctrale apponant, et scutella injecta hortantur quemlibet ad libitum potare.}
\setmarginnote{1431}{Dictu incredible, quantum hujusce liquorice immodesta gens capiat, plus potantem amicissimum habent, et cert coronant, inimicissimum e contra qui non vult, et caede et fustibus expiant.}
\setmarginnote{1432}{Qui potare recusat, hostis habetur, et caede nonnunquam res expiatur.}
\setmarginnote{1433}{Qui melius bibit pro salute domini, melior habetur minister.}
\setmarginnote{1434}{Graec. Poeta apud Stobaeum, ser. 18.}
\setmarginnote{1435}{Qui de die jejunant, et nocte vigilant, facile cadunt in melancholiam; et qui naturae modum excedunt, c. 5. tract. 15. c. 2. Longa famis tolerantia, ut iis saepe accidit qui tanto cum fervore Deo servire cupiunt per jejunium, quod maniaci efficiantur, ipse vidi saepe.}
\setmarginnote{1436}{In tenui victu aegri delinquunt, ex quo fit ut majori afficiantur detrimento, majorque fit error tenui quam pleniore victu.}
\setmarginnote{1437}{Quae longo tempore consueta sunt, etiamsi deteriora, minus in assuetis molestare solent.}
\setmarginnote{1438}{Qui medice vivit, misere vivit.}
\setmarginnote{1439}{Consuetudo altera natura.}
\setmarginnote{1440}{Herefordshire, Gloucestershire, Worcestershire.}
\setmarginnote{1441}{Leo Afer. l. 1. solo camelorum lacte contenti, nil praeterea deliciarum ambiunt.}
\setmarginnote{1442}{Flandri vinum butyro dilutum bibunt (nauseo referens) ubique butyrum inter omnia fercula et bellaria locum obtinet. Steph. praefat. Herod.}
\setmarginnote{1443}{Delectantur Graeci piscibus magis quam carnibus.}
\setmarginnote{1444}{Lib. 1. hist. Ang.}
\setmarginnote{1445}{P. Jovius descript. Britonum. They sit, eat and drink all day at dinner in Iceland, Muscovy, and those northern parts.}
\setmarginnote{1446}{Suidas, vict. Herod, nihilo cum eo melius quam si quis Cicutam, Aconitum, \&c.}
\setmarginnote{1447}{Expedit. in Sinas, lib. 1. c. 3. hortensium herbarum et olerum, apud Sinas quam apud nos longe frequentior usus, complures quippe de vulgo reperias nulla alia re vel tenuitatis, vel religionis causa vescentes. Equus, Mulus, Asellus, \&c. aeque fere vescuntur ac pabula omnia, Mat. Riccius, lib. 5. cap. 12.}
\setmarginnote{1448}{Tartari mulis, equis vescuntur et crudis carnibus, et fruges contemnunt, dicentes, hoc jumentorum pabulum et bonum, non hominum.}
\setmarginnote{1449}{Islandiae descriptione victus corum butyro, lacte, caseo consistit: pisces loco panis habent, potus aqua, aut serum, sic vivunt sine medicina multa ad annos 200.}
\setmarginnote{1450}{Laet. occident. Ind. descrip. lib. 11. cap. 10. Aquam marinam bibere sueti absque noxa.}
\setmarginnote{1451}{Davies 2. voyage.}
\setmarginnote{1452}{Patagones.}
\setmarginnote{1453}{Benzo et Fer. Cortesius, lib. novus orbis inscrip.}
\setmarginnote{1454}{Linschoten, c. 56. Palmae instar totius orbis arboribus longe praestantior.}
\setmarginnote{1455}{Lips. epist.}
\setmarginnote{1456}{Teneris assuescere multum.}
\setmarginnote{1457}{Repentinae mutationes noxam pariunt. Hippocrat. Aphorism. 21. Epist. 6. sect. 3.}
\setmarginnote{1458}{Bruerinus, lib. 1. cap. 23.}
\setmarginnote{1459}{Simpl. med. c. 4. l. 1.}
\setmarginnote{1460}{Heurnius, l. 3. c. 19. prax. med.}
\setmarginnote{1461}{Aphoris. 17.}
\setmarginnote{1462}{In dubiis consuetudinem sequatur adolescens, et inceptis perseveret.}
\setmarginnote{1463}{Qui cum voluptate assumuntur cibi, ventriculus avidius complectitur, expeditiusque concoquit, et quae displicent aversatur.}
\setmarginnote{1464}{Nothing against a good stomach, as the saying is.}
\setmarginnote{1465}{Lib. 7. Hist. Scot.}
\setmarginnote{1466}{30. artis.}
\setmarginnote{1467}{Quae excernuntur aut subsistunt.}
\setmarginnote{1468}{Ex ventre suppresso, inflammationes, capitis dolores, caligines crescunt.}
\setmarginnote{1469}{Excrementa retenta mentis agitationem parere solent.}
\setmarginnote{1470}{Cap. de Mel.}
\setmarginnote{1471}{Tam delirus, ut vix se hominem agnosceret.}
\setmarginnote{1472}{Alvus astrictus causa.}
\setmarginnote{1473}{Per octo dies alvum siccum habet, et nihil reddit.}
\setmarginnote{1474}{Sive per nares, sive haemorrhoides.}
\setmarginnote{1475}{Multi intempestive ab haemorrhoidibus curati, melancholia corrupti sunt. Incidit in Scyllam, \&c.}
\setmarginnote{1476}{Lib. 1. de Mania.}
\setmarginnote{1477}{Breviar. l. 7. c. 18.}
\setmarginnote{1478}{Non sine magno incommodo ejus, cui sanguis a naribus promanat, noxii sanguinis vacuatio impediri potest.}
\setmarginnote{1479}{Novi quosdam prae pudore a coitu abstinentes, turpidos, pigrosque factos; nonnullos etiam melancholicos, praeter modum moestos, timidosque.}
\setmarginnote{1480}{Nonnulli nisi coeant assidue capitis gravitate infestantur. Dicit se novisse quosdam tristes et ita factos ex intermissione Veneris.}
\setmarginnote{1481}{Vapores venenatos mittit sperma ad cor et cerebrum. Sperma plus diu retentum, transit in venenum.}
\setmarginnote{1482}{Graves producit corporis et animi aegritudines.}
\setmarginnote{1483}{Ex spermate supra modum retento monachos et viduas melancholicos saepe fieri vidi.}
\setmarginnote{1484}{Melancholia orta a vasis seminariis in utero.}
\setmarginnote{1485}{Nobilis senex Alsatus juvenem uxorem duxit, at ille colico dolore, et multis morbis correptus, non potuit praestare officium mariti, vix inito matrimonio aegrotus. Illa in horrendum furorum incidit, ob Venerem cohibitam ut omnium eam invisentium congressum, voce, vultu, gestu expeteret, et quum non consentirent, molossos Anglicanos magno expetiit clamore.}
\setmarginnote{1486}{Vidi sacerdotem optimum et pium, qui quod nollet uti Venere, in melancholica symptomata incidit.}
\setmarginnote{1487}{Ob abstinentiam a concubitu incidit in melancholiam.}
\setmarginnote{1488}{Quae a coitu exacerbantur.}
\setmarginnote{1489}{Superstuum coitum causam ponunt.}
\setmarginnote{1490}{Exsiccat corpus, spiritus consumit, \&c. caveant ab hoc sicci, velut inimico mortali.}
\setmarginnote{1491}{Ita exsiccatus ut e melancholico statim fuerit insanus, ab humectantibus curatus.}
\setmarginnote{1492}{Ex cauterio et ulcere exsiccato.}
\setmarginnote{1493}{Gord. c. 10. lib. 1. Discommends cold baths as noxious.}
\setmarginnote{1494}{Siccum reddunt corpus.}
\setmarginnote{1495}{Si quis longius moretur in iis, aut nimis frequenter, aut importune utatur, humores putrefacit.}
\setmarginnote{1496}{Ego anno superiore, quendam guttosum vidi adustum, qui ut liberaretur de gutta, ad balnea accessit, et de gutta liberatus, maniacus factus est.}
\setmarginnote{1497}{On Schola Salernitana.}
\setmarginnote{1498}{Calefactio et ebullitio per venae incisionem, magis saepe incitatur et augetur, majore impetu humores per corpus discurrunt.}
\setmarginnote{1499}{Lib. de flatulenta Melancholia. Frequens sanguinis missio corpus extenuat.}
\setmarginnote{1500}{In 9 Rhasis, atram bilem parit, et visum debilitat.}
\setmarginnote{1501}{Multo nigrior spectatur sanguis post dies quosdam, quam fuit ab initio.}
\setmarginnote{1502}{Non laudo eos qui in desipientia docent secandam esse venam frontis, quia spiritus debilitatur inde, et ego longa experientia observavi in proprio Xenodochio, quod desipientes ex phlebotomia magis laeduntur, et magis disipiunt, et melancholici saepe fiunt inde pejores.}
\setmarginnote{1503}{De mentis alienat. cap. 3. etsi multos hoc improbasse sciam, innumeros hac ratione sanatos longa observatione cognovi, qui vigesies, sexagies venas tundendo, \&c.}
\setmarginnote{1504}{Vires debilitat.}
\setmarginnote{1505}{Impurus aer spiritus dejicit, infecto corde gignit morbos.}
\setmarginnote{1506}{Sanguinem densat, et humores, P. 1. c. 13.}
\setmarginnote{1507}{Lib. 3. cap. 3.}
\setmarginnote{1508}{Lib. de quartana. Ex aere ambiente contrahitur humor melancholicus.}
\setmarginnote{1509}{Qualis aer, talis spiritus: et cujusmodi spiritus, humores.}
\setmarginnote{1510}{Aelianus Montaltus, c. 11. calidus et siccus, frigidus et siccus, paludinosus, crassus.}
\setmarginnote{1511}{Multa hic in Xenodochiis fanaticorum millia quae strictissime catenata servantur.}
\setmarginnote{1512}{Lib. med. part. 2. c. 19. Intellige, quod in calidis regionibus, frequenter accidit mania, in frigidis autem tarde.}
\setmarginnote{1513}{Lib. 2.}
\setmarginnote{1514}{Hodopericon, cap. 7.}
\setmarginnote{1515}{Apulia aestivo calore maxime fervet, ita ut ante finem Maii pene exusta sit.}
\setmarginnote{1516}{They perish in clouds of sand. Maginus Pers.}
\setmarginnote{1517}{Pantheo seu Pract. Med. l. 1. cap. 16. Venetae mulieres quae diu sub sole vivunt, aliquando melancholicae evadunt.}
\setmarginnote{1518}{Navig. lib. 2 cap. 4. commercia nocte, hora secunda ob nimios, qui saeviunt interdiu aestus exercent.}
\setmarginnote{1519}{Morbo Gallico laborantes, exponunt ad solem ut morbus exsiccent.}
\setmarginnote{1520}{Sir Richard Hawkins in his Observations, sect. 13.}
\setmarginnote{1521}{Hippocrates, 3. Aphorismorum idem ait.}
\setmarginnote{1522}{Idem Maginus in Persia.}
\setmarginnote{1523}{Descrip. Ter. sanctae.}
\setmarginnote{1524}{Quum ad solis radios in leone longam moram traheret, ut capillos slavos redderet, in maniam incidit.}
\setmarginnote{1525}{Mundus alter et idem, seu Terra Australis incognita.}
\setmarginnote{1526}{Crassus et turpidus aer, tristem efficit animam.}
\setmarginnote{1527}{Commonly called Scandaroon in Asia Minor.}
\setmarginnote{1528}{Atlas geographicus memoria, valent Pisani, quod crassiore fruantur aere.}
\setmarginnote{1529}{Lib. 1. hist. lib. 2. cap. 41. Aura densa ac caliginosa tetrici homines existunt, et substristes, et cap. 3. stante subsolano et Zephyro, maxima in mentibus hominum alacritas existit, mentisque erectio ubi telum solis splendore nitescit. Maxima dejectio maerorque si quando aura caliginosa est.}
\setmarginnote{1530}{Geor.}
\setmarginnote{1531}{Hor.}
\setmarginnote{1532}{Mens quibus vacillat, ab aere cito offenduntur, et multi insani apud Belgas ante tempestates saeviunt, aliter quieti. Spiritus quoque aeris et mali genii aliquando se tempestatibus ingerunt, et menti humanae se latenter insinuant, eamque vexant, exagitant, et ut fluctus marini, humanum corpus ventis agitatur.}
\setmarginnote{1533}{Aer noctu densatur, et cogit moestitiam.}
\setmarginnote{1534}{Lib de Iside et Osyride.}
\setmarginnote{1535}{Multa defatigatio, spiritus, viriumque substantiam exhaurit, et corpus refrigerat. Humores corruptos qui aliter a natura concoqui et domari possint, et demum blande excludi, irritat, et quasi in furorem agit, qui postea mota camerina, tetro vapore corpus varie lacessunt, animumque.}
\setmarginnote{1536}{In Veni mecum: Libro sic inscripto.}
\setmarginnote{1537}{Instit. ad vit. Christ, cap. 44. cibos crudos in venas rapit, qui putrescentes illic spiritus animalis inficiunt.}
\setmarginnote{1538}{Crudi haec humoris copia per venas aggreditur, unde morbi multiplices.}
\setmarginnote{1539}{Immodicum exercitium.}
\setmarginnote{1540}{Hom. 31. in 1 Cor. vi. Nam qua mens hominis quiscere non possit, sed continuo circa varias cogitationes discurrat, nisi honesto aliquo negotio occupetur, ad melancholiam sponte delabitur.}
\setmarginnote{1541}{Crato, consil. 21. Ut immodica corporis exercitatio nocet corporibus, ita vita deses, et otiosa: otium, animal pituitosum reddit, viscerum obstructiones et crebras fluxiones, et morbos concitat.}
\setmarginnote{1542}{Et vide quod una de rebus quae magis generat melancholiam, est otiositas.}
\setmarginnote{1543}{Reponitur otium ab aliis causa, et hoc a nobis observatum eos huic malo magis obnoxios qui plane otiosi sunt, quam eos qui aliquo munere versantur exequendo.}
\setmarginnote{1544}{De Tranquil. animae. Sunt qua ipsum otium in animi conjicit aegritudinem.}
\setmarginnote{1545}{Nihil est quod aeque melancholiam alat ac augeat, ac otium et abstinentia a corporis et animi exercitationibus.}
\setmarginnote{1546}{Nihil magis excaecat intellectum, quam otium. Gordonius de observat. vit. hum. lib. 1.}
\setmarginnote{1547}{Path. lib. 1. cap. 17. exercitationis intermissio, inertem calorem, languidos spiritus, et ignavos, et ad omnes actiones segniores reddit, cruditates, obstructiones, et excrementorum proventus facit.}
\setmarginnote{1548}{Hor. Ser. 1. Sat. 3.}
\setmarginnote{1549}{Seneca.}
\setmarginnote{1550}{Moerorem animi, et maciem, Plutarch calls it.}
\setmarginnote{1551}{Sicut in stagno generantur vermes, sic et otioso malae cogitationes. Sen.}
\setmarginnote{1552}{Now this leg, now that arm, now their head, heart, \&c.}
\setmarginnote{1553}{Exod. v.}
\setmarginnote{1554}{(For they cannot well tell what aileth them, or what they would have themselves) my heart, my head, my husband, my son, \&c.}
\setmarginnote{1555}{Prov. xviii. Pigrum dejiciet timor. Heautontimorumenon.}
\setmarginnote{1556}{Lib. 19. c. 10.}
\setmarginnote{1557}{Plautus, Prol. Mostel.}
\setmarginnote{1558}{Piso, Montaltus, Mercurialis, \&c.}
\setmarginnote{1559}{Aquibus malum, velut a primaria causa, nactum est.}
\setmarginnote{1560}{Jucunda rerum praesentium, praeteritarum, et futurarum meditatio.}
\setmarginnote{1561}{Facilis descensus Averni: Sed revocare gradum, superasque evadere ad auras, Hic labor, hoc opus est. Virg.}
\setmarginnote{1562}{Hieronimus, ep. 72. dixit oppida et urbes videri sibi tetros carceres, solitudinem Paradisum: solum scorpionibus infectum, sacco amictus, humi cubans, aqua et herbis victitans, Romanis praetulit deliciis.}
\setmarginnote{1563}{Offic. 3.}
\setmarginnote{1564}{Eccl 4.}
\setmarginnote{1565}{Natura de te videtur conqueri posse, quod cum ab ea temperatissimum corpus adeptus sis, tam praeclarum a Deo ac utile donum, non contempsisti modo, verum corrupisti, sedasti, prodidisti, optimam temperaturam otio, crapula, et aliis vitae erroribus, \&c.}
\setmarginnote{1566}{Path. lib. cap. 17. Fernel. corpus infrigidat, omnes sensus, mentisque vires torpore debilitat.}
\setmarginnote{1567}{Lib. 2. sect. 2. cap. 4. Magnam excrementorum vim cerebro et aliis partibus conservat.}
\setmarginnote{1568}{Jo. Retzius, lib. de rebus 6 non naturalibus. Praeparat corpus talis somnus ad multas periculosas aegritudines.}
\setmarginnote{1569}{Instit. ad vitam optimam, cap. 26. cerebro siccitatem adfert, phrenesin et delirium, corpus aridum facit, squalidum, strigosum, humores adurit, temperamentum cerebri corrumpit, maciem inducit: exsiccat corpus, bilem accendit, profundos reddit oculos, calorem augit.}
\setmarginnote{1570}{Naturalem calorem dissipat, laesa concoctione cruditates facit. Attenuant juvenum vigilatae corpora noctes.}
\setmarginnote{1571}{Vita Alexan.}
\setmarginnote{1572}{Grad. 1. c. 14.}
\setmarginnote{1573}{Hor. The body oppressed by yesterday's vices weighs down the spirit also.}
\setmarginnote{1574}{Perturbationes clavi sunt, quibus corpori animus seu patibulo affigitur. Jamb. de mist.}
\setmarginnote{1575}{Lib. de sanitat. tuend.}
\setmarginnote{1576}{Prolog. de virtute Christi; Quae utitur corpore, ut faber malleo.}
\setmarginnote{1577}{Vita Apollonij, lib. 1.}
\setmarginnote{1578}{Lib. de anim. ab inconsiderantia, et ignorantia omnes animi motus.}
\setmarginnote{1579}{De Physiol. Stoic.}
\setmarginnote{1580}{Grad. 1. c. 32.}
\setmarginnote{1581}{Epist. 104.}
\setmarginnote{1582}{Aelianus.}
\setmarginnote{1583}{Lib. 1. cap. 6. si quis ense percusserit eos, tantum respiciunt.}
\setmarginnote{1584}{Terror in sapiente esse non debet.}
\setmarginnote{1585}{De occult nat. mir. l. 1. c. 16. Nemo mortalium qui affectibus non ducatur: qui non movetur, aut saxum, aut Deus est.}
\setmarginnote{1586}{Instit. l. 2. de humanorum affect. morborumque curat.}
\setmarginnote{1587}{Epist. 105.}
\setmarginnote{1588}{Granatensis.}
\setmarginnote{1589}{Virg.}
\setmarginnote{1590}{De civit. Dei. l. 14. c. 9. qualis in oculis hominum qui inversis pedibus ambulat, talis in oculis sapientum, cui passiones dominantur.}
\setmarginnote{1591}{Lib. de Decal. passiones maxime corpus offendunt et animam, et frequentissimae causae melancholiae, dimoventes ab ingenio et sanitate pristina, l. 3. de anima.}
\setmarginnote{1592}{Fraenaet stimuli animi, velut in mari quaedam aurae leves, quaedam placidae, quaedam turbulentae: sic in corpore quaedam affectiones excitant tantum, quaedam ita movent, ut de statu judicii depellant.}
\setmarginnote{1593}{Ut gutta lapidem, sic paulatim hae penetrant animum.}
\setmarginnote{1594}{Usu valentes recte morbi animi vocantur.}
\setmarginnote{1595}{Imaginatio movet corpus, ad cujus motum excitantur humores, et spiritus vitales, quibus alteratur.}
\setmarginnote{1596}{Eccles., xiii. 26. The heart alters the countenance to good or evil, and distraction of the mind causeth distemperature of the body.}
\setmarginnote{1597}{Spiritus et sanguis a laesa Imaginatione contaminantur, humores enim mutati actiones animi immutant, Piso.}
\setmarginnote{1598}{Montani, consil. 22. Hae vero quomodo causent melancholiam, clarum; et quod concoctionem impediant, et membra principalia debilitent.}
\setmarginnote{1599}{Breviar. l. 1. cap. 18.}
\setmarginnote{1600}{Solent hujusmodi egressiones favorabiliter oblectare, et lectorem lassum jucunde refovere, stomachumque nauseantem, quodam quasi condimento reficere, et ego libenter excurro.}
\setmarginnote{1601}{Ab imaginatione oriuntur affectiones, quibus anima componitur, aut turbata deturbatur, Jo. Sarisbur. Metolog. lib. 4. c. 10.}
\setmarginnote{1602}{Scalig. exercit.}
\setmarginnote{1603}{Qui quotis volebat, mortuo similis jacebat auferens se a sensibus, et quum pungeretur dolorem non sensit.}
\setmarginnote{1604}{Idem Nymannus orat. de Imaginat.}
\setmarginnote{1605}{Verbis et unctionibus se consecrant daemoni pessimae mulieres qui iis ad opus suum utitur, et earum phantasiam regit, ducitque ad loca ab ipsis desiderata, corpora vero earum sine sensu permanent, quae umbra cooperit diabolus, ut nulli sine conspicua, et post, umbra sublata, propriis corporibus eas restitut, l. 3. c. 11. Wier.}
\setmarginnote{1606}{Denario medico.}
\setmarginnote{1607}{Solet timor, prae omnibus affectibus, fortes imaginationes gignere, post amor, \&c. l. 3. c. 8.}
\setmarginnote{1608}{Ex viso urso, talem peperit.}
\setmarginnote{1609}{Lib. 1. cap. 4. de occult. nat. mir. si inter amplexus et suavia cogitet de uno, aut alio absente, ejus effigies solet in faetu elucere.}
\setmarginnote{1610}{Quid non faetui adhuc matri unito, subita spirituum vibratione per nervos, quibus matrix cerebro conjuncta est, imprimit impregnatae imaginatio? ut si imaginetur matum granatum, illius notas secum proferet faetus: Si leporem, infans editur supremo labello bifido, et dissecto: Vehemens cogitatio movet rerum species. Wier. lib. 3. cap. 8.}
\setmarginnote{1611}{Ne dum uterum gestent, admittant absurdas cogitationes, sed et visu, audituque foeda et horrenda devitent.}
\setmarginnote{1612}{Occult. Philos. lib. 1. cap. 64.}
\setmarginnote{1613}{Lib. 3. de Lamiis, cap. 10.}
\setmarginnote{1614}{Agrippa, lib. 1. cap. 64.}
\setmarginnote{1615}{Sect. 3. memb. 1. subsect. 3.}
\setmarginnote{1616}{Malleus malefic. fol. 77. corpus mutari potest in diversas aegritudines, ex forti apprehensione.}
\setmarginnote{1617}{Fr. Vales. l. 5. cont. 6. nonnunquam etiam morbi diuturni consequuntur, quandoque curantur.}
\setmarginnote{1618}{Expedit. in Sinas, l. 1. c. 9. tantum porro multi praedictoribus hisce tribuunt ut ipse metus fidem faciat: nam si praedictum iis fuerit tali die eos morbo corripiendos, ii ubi dies advenerit, in morbum incidunt, et vi metus afflicti, cum aegritudine, aliquando etiam cum morte colluctantur.}
\setmarginnote{1619}{Subtil. 18.}
\setmarginnote{1620}{Lib. 3. de anima, cap. de mel.}
\setmarginnote{1621}{Lib. de Peste.}
\setmarginnote{1622}{Lib. 1. cap. 63. Ex alto despicientes aliqui prae timore contremiscunt, caligant, infirmantur; sic singultus, febres, morbi comitiales quandoque sequuntur, quandoque recedunt.}
\setmarginnote{1623}{Lib. de Incantatione, Imaginatio subitum humorum, et spirituum motum infert, undo vario affectu rapitur sanguis, ac una morbificas causas partibus affectis eripit.}
\setmarginnote{1624}{Lib. 3. c. 18. de praestig. Ut impia credulitate quis laeditur, sic et levari eundem credibile est, usuque observatum.}
\setmarginnote{1625}{Aegri persuasio et fiducia, omni arti et consilio et medicinae praeferenda. Avicen.}
\setmarginnote{1626}{Plures sanat in quem plures confidunt. lib. de sapientia.}
\setmarginnote{1627}{Marcelius Ficinus, l. 13. c. 18. de theolog. Platonica. Imaginatio est tanquam Proteus vel Chamaeleon, corpus proprium et alienum nonnunquam afficiens.}
\setmarginnote{1628}{Cur oscitantes oscitent, Wierus.}
\setmarginnote{1629}{Thomas Wright Jesuit.}
\setmarginnote{1630}{3. de Anima.}
\setmarginnote{1631}{Ser. 35. Hae quatuor passiones sunt tanquam rotae in curru, quibus vehimur hoc mundo.}
\setmarginnote{1632}{Harum quippe immoderatione, spiritus marcescunt. Fernel. l. 1. Path. c. 18.}
\setmarginnote{1633}{Mala consuetudine depravatur ingenium ne bene faciat. Prosper Calenus, l. de atra bile. Plura faciunt homines e consuetudine quam e ratione. A teneris assuescere multum est. Video meliora proboque deteriora sequor. Ovid.}
\setmarginnote{1634}{Nemo laeditur nisi a seipso.}
\setmarginnote{1635}{Multi se in inquietudinem praecipitant ambitione et cupiditatibus excaecati, non intelligunt se illud a diis petere, quod sibi ipsis si velint praestare possint, si curis et perturbationibus, quibus assidue se macerant, imperare vellent.}
\setmarginnote{1636}{Tanto studio miseriarum causas, et alimenta dolorum quaerimus, vitamque secus felicissimam, tristem et miserabilem efficimus. Petrarch. praefat. de Remediis, \&c.}
\setmarginnote{1637}{Timor et moestitia, si diu perseverent, causa et soboles atri humoris sunt, et in circulum se procreant. Hip. Aphoris. 23. l. 6. Idem Montaltus, cap. 19. Victorius Faventinus, pract. imag.}
\setmarginnote{1638}{Multi ex maerore et metu huc delapsi sunt. Lemn., lib. 1. cap. 16.}
\setmarginnote{1639}{Multa cura et tristitia faciunt accedere melancholiam (cap. 3. de mentis alien.) si altas radices agat, in veram fixamque degenerat melancholiam et in desperationem desinit.}
\setmarginnote{1640}{Ille luctus, ejus vero soror desperatio simul ponitur.}
\setmarginnote{1641}{Animarum crudele tormentum, dolor inexplicabilis, tinea non solum ossa, sed corda pertingens, perpetuus carnifex, vires animae consumens, jugis nox, et tenebrae profundae, tempestas et turbo et febris non apparens, omni igne validius incendens; longior, et pugnae finem non habens-Crucem circumfert dolor, faciemque omni tyranno crudeliorem prae se fert.}
\setmarginnote{1642}{Nat. Comes Mythol. l. 4. c. 6.}
\setmarginnote{1643}{Tully 3. Tusc. omnis perturbatio miseria et carnificina est dolor.}
\setmarginnote{1644}{M. Drayton in his Her. ep.}
\setmarginnote{1645}{Crato consil. 21. lib. 2. moestitia universum infrigidat corpus, calorem innatum extinguit, appetitum destruit.}
\setmarginnote{1646}{Cor refrigerat tristitia, spiritus exsiccat, innatumque calorem obruit, vigilias inducit, concoctionem labefactat, sanguinem incrassat, exageratque melancholicum succum.}
\setmarginnote{1647}{Spiritus et sanguis hoc contaminatur. Piso.}
\setmarginnote{1648}{Marc. vi. 16. 11.}
\setmarginnote{1649}{Maerore maceror, marcesco et consenesco miser, ossa atque pellis sum misera macritudice. Plaut.}
\setmarginnote{1650}{Malum inceptum et actum a tristitia sola.}
\setmarginnote{1651}{Hildesheim, spicel. 2. de melancholia, maerore animi postea accedente, in priora symptomata incidit.}
\setmarginnote{1652}{Vives, 3. de anima, c. de maerore. Sabin. in Ovid.}
\setmarginnote{1653}{Herodian. l. 3. maerore magis quem morbo consumptus est.}
\setmarginnote{1654}{Bothwallius atribilarius obiit Brizarrus Genuensis hist. \&c.}
\setmarginnote{1655}{So great is the fierceness and madness of melancholy.}
\setmarginnote{1656}{Moestitia cor quasi percussum constringitur, tremit et languescit cum acri sensu doloris. In tristitia cor fugiens attrahit ex Splene lentum humorem melancholicum, qui effusus sub costis in sinistro latere hypocondriacos flatus facit, quod saepe accidit iis qui diuturna cura et moestitia conflictantur. Melancthon.}
\setmarginnote{1657}{Lib. 3. Aen. 4.}
\setmarginnote{1658}{Et metum ideo deam sacrarunt ut bonam mentem concederet. Varro, Lactantius, Aug.}
\setmarginnote{1659}{Lilius Girald. Syntag. l. de diis miscellaniis.}
\setmarginnote{1660}{Calendis Jan. feriae sunt divae Angeronae, cui pontifices in sacello Volupiae sacra faciunt, quod angores et animi solicitudines propitiata propellat.}
\setmarginnote{1661}{Timor inducit frigus, cordis palpitationem, vocis defectum atque pallorem. Agrippa, lib. 1. cap. 63. Timidi semper spiritus habent frigidos. Mont.}
\setmarginnote{1662}{Effusas cernens fugientes agmine turmas; quis mea nunc inflat cornua Faunus ait? Alciat.}
\setmarginnote{1663}{Metus non solum memoriam consternat, sed et institutum animi omne et laudabilem conatum impedit. Thucidides.}
\setmarginnote{1664}{Lib. de fortitudine et virtute Alexandri, ubi prope res adfuit terribilis.}
\setmarginnote{1665}{Sect. 2. Mem. 3. Subs. 2.}
\setmarginnote{1666}{Sect. 2. Memb. 4. Subs. 3.}
\setmarginnote{1667}{Subtil. 18. lib. timor attrahit ad se Daemonas, timor et error multum in hominibus possunt.}
\setmarginnote{1668}{Lib. 2. Spectris ca. 3. fortes raro spectra vident, quia minus timent.}
\setmarginnote{1669}{Vita ejus.}
\setmarginnote{1670}{Sect. 2. Memb. 4. Subs. 7.}
\setmarginnote{1671}{De virt. et vitiis.}
\setmarginnote{1672}{Com. in Arist. de Anima.}
\setmarginnote{1673}{Qui mentem subjecit timoria dominationi, cupiditatis, doloris, ambitionis, pudoris, felix non est, sed omnino miser, assiduis laborius torquetur et miseria.}
\setmarginnote{1674}{Multi contemnunt mundi strepitum, reputant pro nihilo gloriam, sed timent infamiam, offensionem, repulsam. Voluptatem severissime contemnunt, in dolore sunt molliores, gloriam negligunt, franguntur infamia.}
\setmarginnote{1675}{Gravius contumeliam ferimus quam detrimentum, ni abjecto nimis animo sinius. Plut. in Timol.}
\setmarginnote{1676}{Quod piscatoris aenigma solvere non posset.}
\setmarginnote{1677}{Ob Tragoediam explosam, mortem sibi gladio concivit.}
\setmarginnote{1678}{Cum vidit in triumphum se servari, causa ejus ignominiae vitandae mortem sibi concivit. Plut.}
\setmarginnote{1679}{Bello victus, per tres dies sedit in prora navis, abstinens ab omni consortio, etiam Cleopatiae, postea se interfecit.}
\setmarginnote{1680}{Cum male recitasset Argonautica, ob pudorem exulavit.}
\setmarginnote{1681}{Quidam prae verecundia simul et dolore in insaniam incidunt, eo quod a literatorum gradu in examine excluduntur.}
\setmarginnote{1682}{Hostratus cucullatus adeo graviter ob Reuclini librum, qui inscribitur, Epistolae obscurorum virorum, dolore simul et pudore sauciatus, ut seipsum interfecerit.}
\setmarginnote{1683}{Propter ruborem confusus, statim cepit delirare, \&c. ob suspicionem, quod vili illum crimine accusarent.}
\setmarginnote{1684}{Horat.}
\setmarginnote{1685}{Ps. Impudice. B. Ita est. Ps. sceleste. B. dicis vera Ps. Verbero. B. quippeni Ps. furcifer. B. factum optime. Ps. soci fraude. B. sunt mea istaec Ps. parricida B. perge tu Ps. sacrilege. B. fateor. Ps. perjure B. vera dicis. Ps. pernities adolescentum B. acerrime. Ps. fur. B. babe. Ps. fugitive. B. bombax. Ps. fraus populi. B. Planissime. Ps. impure leno, coenum. B. cantores probos. Pseudolus, act. 1. Scen. 3.}
\setmarginnote{1686}{Melicerta exclaims, all shame has vanished from human transactions. Persius. Sat. V.}
\setmarginnote{1687}{Cent. 7. e Plinio.}
\setmarginnote{1688}{Multos vide mus propter invidiam et odium in melancholiam incidisse: et illos potissimum quorum corpora ad hanc apta sunt.}
\setmarginnote{1689}{Invidia affligit homines adeo et corrodit, ut hi melancholici penitus fiant.}
\setmarginnote{1690}{Hor.}
\setmarginnote{1691}{His vultus minax, torvus aspectus, pallor in facie, in labiis tremor, stridor in dentibus, \&c.}
\setmarginnote{1692}{Ut tinea corrodit vestimentum sic, invidiae eum qui zelatur consumit.}
\setmarginnote{1693}{Pallor in ore sedet, macies in corpore toto. Nusquam recta acies, livent rubigine dentes.}
\setmarginnote{1694}{Diaboli expressa Imago, toxicum charitatis, venenum amicitiae, abyssus mentis, non est eo monstrosius monstrum, damnosius damnum, urit, torret, discruciat macie et squalore conficit. Austin. Domin. primi. Advent.}
\setmarginnote{1695}{Ovid. He pines away at the sight of another's success--it is his special torture.}
\setmarginnote{1696}{Declam. 13. linivit flores maleficis succis in venenum mella convertens.}
\setmarginnote{1697}{Statuis cereis Basilius eos comparat, qui liquefiunt ad praesentiam solis, qua alii gaudent et ornantur. Muscis alii, quae ulceribus gaudent, amaena praetereunt sistunt in faetidis.}
\setmarginnote{1698}{Misericordia etiam quae tristitia quaedam est, saepe miserantis corpus male afficit Agrippa. l. 1. cap. 63.}
\setmarginnote{1699}{Insitum mortalibus a natura recentem aliorem felicitatem aegris oculis intueri, hist. l. 2. Tacit.}
\setmarginnote{1700}{Legi Chaldaeos, Graecos, Hebraeos, consului sapientes pro remedio invidiae, hoc enim inveni, renunciare felicitati, et perpetuo miser esse.}
\setmarginnote{1701}{Omne peccatum aut excusationem secum habet, aut voluptatem, sola invidia utraque caret, reliqua vitia finem habent, ira defervescit, gula satiatur, odium finem habet, invidia nunquam quiescit.}
\setmarginnote{1702}{Urebat me aemulatio propter stultos.}
\setmarginnote{1703}{Hier. 12.1.}
\setmarginnote{1704}{Hab. 1.}
\setmarginnote{1705}{Invidit privati nomen supra principis attolli.}
\setmarginnote{1706}{Tacit. Hist. lib. 2. part. 6.}
\setmarginnote{1707}{Periturae dolore et invidia, si quem viderint ornatiorem se in publicum prodiisse. Platina dial. amorum.}
\setmarginnote{1708}{Ant. Guianerius, lib. 2. cap. 8. vim. M. Aurelii faemina vicinam elegantius se vestitam videns, leaenae instar in virum insurgit, \&c.}
\setmarginnote{1709}{Quod insigni equo et ostro veheretur, quanquam nullius cum injuria, ornatum illum tanquam laesae gravabantur.}
\setmarginnote{1710}{Quod pulchritudine omnes excelleret, puellae indignatae occiderunt.}
\setmarginnote{1711}{Late patet invidiae foecundae pernities, et livor radix omnium malorum, fons cladium, inde odium surgit emulatio Cyprian, ser. 2. de Livore.}
\setmarginnote{1712}{Valerius, l. 3. cap. 9.}
\setmarginnote{1713}{Qualis est animi tinea, quae tabes pectoris zelare in altero vel aliorum felicitatem suam facere miseriam, et velut quosdam pectori suo admovere carnifices, cogitationibus et sensibus suis adhibere tortores, qui se intestinis cruciatibus lacerent. Non cibus talibus laetus, non potus potest esse jucundus; suspiratur semper et gemitur, et doletur dies et noctes, pectus sine intermissione laceratur.}
\setmarginnote{1714}{Quisquis est ille quem aemularis, cui invides is te subterfugere potest, at tu non te ubicunque fugeris adversarius tuus tecum est, hostis tuus semper in pectore tuo est, pernicies intus inclusa, ligatus es, victus, zelo dominante captivus: nec solatia tibi ulla subveniunt; hinc diabolus inter initia statim mundi, et periit primus, et perdidit, Cyprian, ser. 2. de zelo et livore.}
\setmarginnote{1715}{Hesiod op dies.}
\setmarginnote{1716}{Rama cupida aequandi bovem, se distendebat, \&c.}
\setmarginnote{1717}{alit ingenia: Paterculus poster. Vol.}
\setmarginnote{1718}{Grotius Epig. lib. 1. Ambition always is a foolish confidence, never a slothful arrogance.}
\setmarginnote{1719}{Anno 1519. between Ardes and Quine.}
\setmarginnote{1720}{Spartian.}
\setmarginnote{1721}{Plutarch.}
\setmarginnote{1722}{Johannes Heraldus, l. 2. c. 12. de bello sac.}
\setmarginnote{1723}{Nulla dies tantum poterit lenire furorem. Aeterna bella pace sublata gerunt. Jurat odium, nec ante invisum esse desinit, quam esse desiit. Paterculus, vol. 1.}
\setmarginnote{1724}{Ita saevit haec stygia ministra ut urbes subvertat aliquando, deleat populos, provincias alioqui florentes redigat in solitudines, mortales vero miseros in profunda miseriarum valle miserabiliter immergat.}
\setmarginnote{1725}{Carthago aemula Romani imperii funditus interiit. Salust. Catil.}
\setmarginnote{1726}{Paul 3. Col.}
\setmarginnote{1727}{Rom. 12.}
\setmarginnote{1728}{Grad. I. c. 54.}
\setmarginnote{1729}{Ira et in moeror et ingens animi consternatio melancholicos facit. Areteus. Ira Immodica gignit insaniam.}
\setmarginnote{1730}{Reg. sanit. parte 2. c. 8. in apertam insaniam mox duciter iratus.}
\setmarginnote{1731}{Gilberto Cognato interprete. Multis, et praesertim senibus ira impotens insaniam fecit, et importuna calumnia, haec initio perturbat animum, paulatim vergit ad insaniam. Porro mulierum corpora multa infestant, et in hunc morbum adducunt, praecipue si que oderint aut invideant, \&c. haec paulatim in insaniam tandem evadunt.}
\setmarginnote{1732}{Saeva animi tempestas tantos excitans, fluctus ut statim ardescant oculi os tremat, lingua titubet, dentes concrepant, \&c.}
\setmarginnote{1733}{Ovid.}
\setmarginnote{1734}{Terence.}
\setmarginnote{1735}{Infensus Britanniae Duci, et in ultionem versus, nec cibum cepit, nec quietem, ad Calendas Julias 1392. comites occidit.}
\setmarginnote{1736}{Indignatione nimia furens, animique impotens, exiliit de lecto, furentem non capiebat aula, \&c.}
\setmarginnote{1737}{An ira possit hominem interimere.}
\setmarginnote{1738}{Abernethy.}
\setmarginnote{1739}{As Troy, saevae memorem Hunonis ob iram.}
\setmarginnote{1740}{Stultorum regum et populorum continet astus.}
\setmarginnote{1741}{Lib. 2. Invidia est dolor et ambitio est dolor, \&c.}
\setmarginnote{1742}{Insomnes Claudianus. Tristes, Virg. Mordaces, Luc. Edaces, Hor. moestae, amarae, Ovid damnosae, inquietae, Mart. Urentes, Rodentes. Mant. \&c.}
\setmarginnote{1743}{Galen, l. 3. c. 7. de locis affectis, homines sunt maxime melancholici, quando vigiliis multis, et solicitudinibus, et laboribus, et curis fuerint circumventi.}
\setmarginnote{1744}{Lucian. Podag.}
\setmarginnote{1745}{Omnia imperfecta, confusa, et perturbatione plena, Cardan.}
\setmarginnote{1746}{Lib. 7. nat. hist, cap. 1. hominem nudum, et ad vagitum edit, natura. Flens ab initio, devinctus jacet, \&c.}
\setmarginnote{1747}{\textgreek{Δακρυχέων γενόμην, καί δακρύσας αποθνήσκω, ὦ γένος ἀνθρώπων πολυδάκρυτον, ἀσθενείς οίκτρον} Lachrymans natus sum, et lachrymans morior, \&c.}
\setmarginnote{1748}{Ad Marinum.}
\setmarginnote{1749}{Boethius.}
\setmarginnote{1750}{Initium caecitas progressum labor, exitum dolor, error omnia: quem tranquillum quaeso, quem non laboriosum aut anxium diem egimus? Petrarch.}
\setmarginnote{1751}{Ubique periculum, ubique dolor, ubique naufragium, in hoc ambitu quocunque me vertam. Lipsius.}
\setmarginnote{1752}{Hom. 10. Si in forum iveris, ibi rixae, et pugnae; si in curiam, ibi fraus, adulatio: si in domum privatam, \&c.}
\setmarginnote{1753}{Homer.}
\setmarginnote{1754}{Multis repletur homo miseriis, corporis miseriis, animi miseriis, dum dormit, dum vigilat, quocunque se vertit. Lususque rerum, temporumque nascimur.}
\setmarginnote{1755}{In blandiente fortuna intolerandi, in calamitatibus lugubres, semper stulti et miseri, Cardan.}
\setmarginnote{1756}{Prospera in adversis desidero, et adversa prosperis timeo, quis inter haec medius locus, ubi non fit humanae vitae tentatio?}
\setmarginnote{1757}{Cardan. consol. Sapientiae Labor annexus, gloriae invidia, divitiis curae, soboli solicitudo, voluptati morbi, quieti paupertas, ut quasi fruendoriun scelerum causa nasci hominem possis cum Platonistis agnoscere.}
\setmarginnote{1758}{Lib. 7. cap. 1. Non satis aestimare, an melior parens natura homini, an tristior noverca fuerit: Nulli fragilior vita, pavor, confusio, rabies major, uni animantium ambitio data, luctus, avaritia, uni superstitio.}
\setmarginnote{1759}{Euripides. I perceive such an ocean of troubles before me, that no means of escape remain.}
\setmarginnote{1760}{De consol. l. 2. Nemo facile cum conditione sua concordat, inest singulis quod imperiti petant, experti horreant.}
\setmarginnote{1761}{Esse in honore juvat, mox displicet.}
\setmarginnote{1762}{Hor.}
\setmarginnote{1763}{Borrheus in 6. Job. Urbes et oppida nihil aliud sunt quam humanarum aerumnarum domicilia quibus luctus et moeror, et mortalium varii infinitique labores, et omnis generis vitia, quasi septis includuntur.}
\setmarginnote{1764}{Nat. Chytreus de lit. Europae. Laetus nunc, mox tristis; nunc sperans, paulo post diffidens; patiens hodie, cras ejuians; nunc pallens, rubens, currens, sedens, claudicans; tremens, \&c.}
\setmarginnote{1765}{Sua cuique calamitas praecipua.}
\setmarginnote{1766}{Cn. Graecinus.}
\setmarginnote{1767}{Epist. 9. l. 7. Miser est qui se beatissimum non judicat, licet imperet mundo non est beatus, qui se non putat: quid enim refert qualis status tuus sit, si tibi videtur malus.}
\setmarginnote{1768}{Hor. ep. 1. l. 4.}
\setmarginnote{1769}{Hor. Ser. 1. Sat. 1.}
\setmarginnote{1770}{Lib. de curat. graec. affect. cap. 6. de provident. Multis nihil placet atque adeo et divitias damnant, et paupertatem, de morbis expostulant, bene valentes graviter ferunt, atque ut semel dicam, nihil eos delectat, \&c.}
\setmarginnote{1771}{Vix ultius gentis, aetatis, ordinis, hominem invenies cujus felicitatem fortunae Metelli compares, Vol. 1.}
\setmarginnote{1772}{P. Crassus Mutianus, quinque habuisse dicitur rerum bonarum maxima, quod esset ditissimus, quod esset nobilissimus, eloquentissimus, Jurisconsultissimus, Pontifex maximus.}
\setmarginnote{1773}{Lib. 7. Regis filia, Regis uxor, Regis mater.}
\setmarginnote{1774}{Qui nihil unquam mali aut dixit, aut fecit, aut sensit, qui bene semper fecit, quod aliter facere non potuit.}
\setmarginnote{1775}{Solomon. Eccles. 1. 14.}
\setmarginnote{1776}{Hor. Art. Poet.}
\setmarginnote{1777}{Jovius, vita ejus.}
\setmarginnote{1778}{2 Sam. xii. 31.}
\setmarginnote{1779}{Boethius, lib. 1. Met. Met. 1.}
\setmarginnote{1780}{Omnes hic aut captantur, aut captant: aut cadavera quae lacerantur, aut corvi qui lacterant. Petron.}
\setmarginnote{1781}{Homo omne monstrum est, ille nam susperat feras, luposque et ursos pectore obscuro tegit. Hens.}
\setmarginnote{1782}{Quod Paterculus de populo Romano durante bello Punico per annos 115, aut bellum inter eos, aut belli praeparatio, aut infida pax, idem ego de mundi accolis.}
\setmarginnote{1783}{Theocritus Edyll. 15.}
\setmarginnote{1784}{Qui sedet in mensa, non meminit sibi otioso ministrare negotiosos, edenti esurientes, bibenti sitientes, \&c.}
\setmarginnote{1785}{Quando in adolescentia sua ipsi vixerint, lautius et liberius voluptates suas expleverint, illi gnatis impenunt duriores continentiae leges.}
\setmarginnote{1786}{Lugubris Ate luctuque fero Regum tumidas obsidet arces. Res est inquieta felicitas.}
\setmarginnote{1787}{Plus aloes quam mellis habet. Non humi jacentem tolleres. Valer. l. 7. c. 3.}
\setmarginnote{1788}{Non diadema aspicias, sed vitam afflictione refertam, non catervas satellitum, sed curarum multitudinem.}
\setmarginnote{1789}{As Plutarch relateth.}
\setmarginnote{1790}{Sect. 2. memb. 4. subsect. 6.}
\setmarginnote{1791}{Stercus et urina, medicorum fercula prima.}
\setmarginnote{1792}{Nihil lucrantur, nisi admodum mentiendo. Tull. Offic.}
\setmarginnote{1793}{Hor. l. 2. od. 1.}
\setmarginnote{1794}{Rarus felix idemque senex. Seneca in Her. aeteo.}
\setmarginnote{1795}{Omitto aegros, exules, mendicos, quos nemo audet felices dicere. Card. lib. 8. c. 46. de rer. var.}
\setmarginnote{1796}{Spretaeque injuria formae.}
\setmarginnote{1797}{Hor.}
\setmarginnote{1798}{Attenuant vigiles corpus miserabile curae.}
\setmarginnote{1799}{Plautus.}
\setmarginnote{1800}{Haec quae crines evellit, aerumna.}
\setmarginnote{1801}{Optimum non nasci, aut cito mori.}
\setmarginnote{1802}{Bonae si rectam rationem sequuntur, malae si exorbitant.}
\setmarginnote{1803}{Tho. Buovie. Prob. 18.}
\setmarginnote{1804}{Molam asinariam.}
\setmarginnote{1805}{Tract. de Inter. c. 92.}
\setmarginnote{1806}{Circa quamlibet rem mundi haec passio fieri potest, quae superflue diligatur. Tract. 15. c. 17.}
\setmarginnote{1807}{Ferventius desiderium.}
\setmarginnote{1808}{Imprimis vero Appetitus, \&c. 3. de alien. ment.}
\setmarginnote{1809}{Conf. l. c. 29.}
\setmarginnote{1810}{Per diversa loca vagor, nullo temporis momento quiesco, talis et talis esse cupio, illud atque illud habere desidero.}
\setmarginnote{1811}{Ambros. l. 3. super Lucam. aerugo animae.}
\setmarginnote{1812}{Nihil animum cruciat, nihil molestius inquietat, secretum virus, pestis occulta, \&c. epist. 126.}
\setmarginnote{1813}{Ep. 88.}
\setmarginnote{1814}{Nihil infelicius his, quantus iis timor, quanta dubitatio, quantus conatus, quanta solicitudo, nulla illis a molestiis vacua hora.}
\setmarginnote{1815}{Semper attonitus, semper pavidus quid dicat, faciatve: ne displiceat humilitatem simulat, honestatem mentitur.}
\setmarginnote{1816}{Cypr. Prolog. ad ser. To. 2. cunctos honorat, universis inclinat, subsequitur, obsequitur, frequentat curias, visitat, optimates amplexatur, applaudit, adulatur: per fas et nefas e latebris, in omnem gradum ubi aditus patet se integrit, discurrit.}
\setmarginnote{1817}{Turbae cogit ambitio regem inservire, ut Homerus Agamemnonmem querentem inducit.}
\setmarginnote{1818}{Plutarchus. Quin convivemur, et in otio nos oblectemur, quoniam in promptu id nobis sit, \&c.}
\setmarginnote{1819}{Jovius hist. l. 1. vir singulari prudentia, sed profunda ambitione, ad exitium Italae natus.}
\setmarginnote{1820}{Ut hedera arbori adhaeret, sic ambitio, \&c.}
\setmarginnote{1821}{Lib. 3. de contemptu rerum fortuitarum. Magno conatu et impetu moventur, super eodem centro rotati, non proficiunt, nec ad finem perveniunt.}
\setmarginnote{1822}{Vita Pyrrhi.}
\setmarginnote{1823}{Ambitio in insaniam facile delabitur, si excedat. Patritius, l. 4. tit. 20. de regis instit.}
\setmarginnote{1824}{Lib. 5. de rep. cap. 1.}
\setmarginnote{1825}{Imprimis vero appetitus, seu concupiscentia nimia rei alicujus, honestae vel inhonestae, phantasiam laedunt; unde multi ambitiosi, philauti, irati, avari, insani, \&c. Felix Plater, l. 3. de mentis alien.}
\setmarginnote{1826}{Aulica vita colluvies ambitionis, cupiditatis, simulationis, imposturae, fraudis, invidiae, superbiae Titannicae diversorium aula, et commune conventiculum assentandi artificum, \&c. Budaeus de asse. lib. 5.}
\setmarginnote{1827}{In his Aphor.}
\setmarginnote{1828}{Plautus Curcul. Act. 4. Sce. 1.}
\setmarginnote{1829}{Tom. 2. Si examines, omnes miseriae causas vel a furioso contendendi studio, vel ab injusta cupiditate, origine traxisse scies. Idem fere Chrysostomus com. in c. 6. ad Roman. ser. 11.}
\setmarginnote{1830}{Cap. 4. 1.}
\setmarginnote{1831}{Ut sit iniquus in deum, in proximum, in seipsum.}
\setmarginnote{1832}{Si vero, Crateva, inter caeteras herbarum radices, avaritiae radicem secare posses amaram, ut nullae reliquiae essent, probe scito, \&c.}
\setmarginnote{1833}{Cap. 6. Dietae salutis: avaritia est amor immoderatus pecuniae vel acquirendae, vel retinendae.}
\setmarginnote{1834}{Ferum profecto dirumque ulcus animi, remediis non cedens medendo exasperatur.}
\setmarginnote{1835}{Malus est morbus maleque afficit avaritia siquidem censeo, \&c. avaritia difficilius curatur quam insania: quoniam hac omnes fere medici laborant. Hib. ep. Abderit.}
\setmarginnote{1836}{Qua re non es lassus? lucrum faciendo: quid maxime delectabile? lucrari.}
\setmarginnote{1837}{Extremos currit mercator ad Indos. Hor.}
\setmarginnote{1838}{Hom. 2. aliud avarus aliud dives.}
\setmarginnote{1839}{Divitiae ut spinae animum hominis timoribus, solicitudinibus, angoribus mirifice pungunt, vexant, cruciant. Greg. in hom.}
\setmarginnote{1840}{Epist. ad Donat. cap. 2.}
\setmarginnote{1841}{Lib. 9. ep. 30.}
\setmarginnote{1842}{Lib. 9. cap. 4. insulae rex titulo, sed animopecuniae miserabile mancipium.}
\setmarginnote{1843}{Hor. 10. lib. 1.}
\setmarginnote{1844}{Danda est hellebori multo pars maxima avaris.}
\setmarginnote{1845}{Luke. xii. 20. Stulte, hac nocte eripiam animam tuam.}
\setmarginnote{1846}{Opes quidem mortalibus sunt dementia Theog.}
\setmarginnote{1847}{Ed. 2. lib. 2. Exonerare cum se possit et relevare ponderibus pergit magis fortunis augentibus pertinaciter incubare.}
\setmarginnote{1848}{Non amicis, non liberis, non ipsi sibi quidquam impertit, possidet ad hoc tantum, ne possidere alteri liceat, \&c. Hieron. ad Paulin. tam deest quod habet quam quod non habet.}
\setmarginnote{1849}{Epist. 2. lib. 2. Suspirat in convivio, bibat licet gemmis et toro molliore marcidum corpus condiderit, vigilat in pluma.}
\setmarginnote{1850}{Angustatur ex abundantia, contristatur ex opulentia, infelix praesentibus bonis, infelicior in futuris.}
\setmarginnote{1851}{Illorum cogitatio nunquam cessat qui pecunias supplere diligunt. Guianer. tract. 15. c. 17.}
\setmarginnote{1852}{Hor. 3. Od. 24. Quo plus sunt potae, plus sitiunter aquae.}
\setmarginnote{1853}{Hor. l. 2. Sat. 6. O si angulus ille proximus accedat, qui nunc deformat agellum.}
\setmarginnote{1854}{Lib. 3. de lib. arbit. Immoritur studiis, et amore senescit habendi.}
\setmarginnote{1855}{Avarus vir inferno est similis, \&c. modum non habet, hoc egentior quo plura habet.}
\setmarginnote{1856}{Erasm. Adag. chil. 3. cent. 7. pro. 72 Nulli fidentes omnium formidant opes, ideo pavidum malum vocat Euripides: metuunt tempestates ob frumentum, amicos ne rogent, inimicos ne laedant, fures ne rapiant, bellum timent, pacem timent, summos, medios, infinos.}
\setmarginnote{1857}{Hall Char.}
\setmarginnote{1858}{Agellius, lib. 3. cap. 1. interdum eo sceleris perveniunt ob lucrum, ut vitam propriam commutent.}
\setmarginnote{1859}{Lib. 7. cap. 6.}
\setmarginnote{1860}{Omnes perpetuo morbo agitantur, suspicatur omnes timidus sibique ob aurum insidiari putat, nunquam quiescens, Plin. Prooem. lib. 14.}
\setmarginnote{1861}{Cap. 18. in lecto jacens interrogat uxorem an arcam probe clausit, an capsula, \&c. E lecto surgens nudus et absque calceis, accensa lucerna omnia obiens et lustrans, et vix somno indulgens.}
\setmarginnote{1862}{Curis extenuatus, vigilans et secum supputans.}
\setmarginnote{1863}{Cave quenquam alienum in aedes intromiseris. Ignem extinqui volo, ne causae quidquam sit quod te quisquam quaeritet. Si bona fortuna veniat ne intromiseris; Occlude sis fores ambobus pessulis. Discrutior animi quia domo abeundum est mihi: Nimis hercule invitus abeo, nec quid agam scio.}
\setmarginnote{1864}{Ploras aquam profundere, \&c. periit dum fumus de tigillo exit foras.}
\setmarginnote{1865}{Juv. Sat. 14.}
\setmarginnote{1866}{Ventrocosus, nudus, pallidus, laeva pudorem occultans, dextra siepsum strangulans, occurit autem exeunti poenitentia his miserum conficiens, \&c.}
\setmarginnote{1867}{Luke XV.}
\setmarginnote{1868}{Boethius.}
\setmarginnote{1869}{In Oeconom. Quid si nunc ostendam eos qui magna vi argenti domus inutiles aedificant, inquit Socrates.}
\setmarginnote{1870}{Sarisburiensis Polycrat. l. 1. c. 14. venatores omnes adhuc institutionem redolent centaurorum. Raro invenitur quisquam eorum modestus et gravis, raro continens, et ut credo sobrius unquam.}
\setmarginnote{1871}{Pancirol. Tit. 23. avolant opes cum accipitre.}
\setmarginnote{1872}{Insignis venatorum stultitia, et supervacania cura eorum, qui dum nimium venationi insistunt, ipsi abjecta omni humanitate in feras degenerant, ut Acteon, \&c.}
\setmarginnote{1873}{Sabin. in Ovid. Metamor.}
\setmarginnote{1874}{Agrippa de vanit. scient. Insanum venandi studium, dum a novalibus arcentur agricolae subtrahunt praedia rusticis, agricolonis praecluduntur sylvae et prata pastoribus ut augeantur pascua feris.-Majestatis reus agricola si gustarit.}
\setmarginnote{1875}{A novalibus suis arcentur agricolae, dum ferae habeant vagandi libertatem: istis, ut pascua augeantur, praedia subtrahuntur, \&c. Sarisburiensis.}
\setmarginnote{1876}{Feris quam hominibus aequiores. Cambd. de Guil. Conq. qui 36 Ecclesias matrices depopulatus est ad forestam novam. Mat. Paris.}
\setmarginnote{1877}{Tom. 2. de vitis illustrium, l. 4. de vit. Leon. 10.}
\setmarginnote{1878}{Venationibus adeo perdite studebat et aucupiis.}
\setmarginnote{1879}{Aut infeliciter venatus tam impatiens inde, ut summos saepe viros acerbissimis contumeliis oneraret, et incredibile est quali vultus animique habitu dolorem iracundiamque praeferret, \&c.}
\setmarginnote{1880}{Unicuique autem hoc a natura insitum est, ut doleat sicubi erraverit aut deceptus sit.}
\setmarginnote{1881}{Juven. Sat. 8. Nec enim loculis comitan tibus itur, ad casum tabulae, posita sed luditur arca Leinnius instit. ca. 44. mendaciorum quidem, et perjuriorum et paupertatis mater est alea, nullam habens patrimonii reverentiam, quum illud effuderit, sensim in furta delabitur et rapinas. Saris, polycrat. l. 1. c. 5.}
\setmarginnote{1882}{Damhoderus.}
\setmarginnote{1883}{Dan. Souter.}
\setmarginnote{1884}{Petrar. dial. 27.}
\setmarginnote{1885}{Salust.}
\setmarginnote{1886}{Tom. 3 Ser. de Allea.}
\setmarginnote{1887}{Plutus in Aristop. calls all such gamesters madmen. Si in insanum hominem contigero. Spontaneum ad se trahunt furorem, et os, et nares et oculos rivos faciunt furoris et diversoria, Chrys. hom. 17.}
\setmarginnote{1888}{Pascasius Justus l. 1. de alea.}
\setmarginnote{1889}{Seneca.}
\setmarginnote{1890}{Hall.}
\setmarginnote{1891}{In Sat. 11. Sed deficiente crumena: et crescente gula, quis te manet exitus-rebus in ventrem mersis.}
\setmarginnote{1892}{Spartian. Adriano.}
\setmarginnote{1893}{Alex. ab. Alex. lib. 6. c. 10. Idem Gerbelius, lib. 5. Grae. disc.}
\setmarginnote{1894}{Fines Moris.}
\setmarginnote{1895}{Justinian in Digestis.}
\setmarginnote{1896}{Persius Sat. 5. One indulges in wine, another the die consumes, a third is decomposed by venery.}
\setmarginnote{1897}{Poculum quasi sinus in quo saepe naufragium faciunt, jactura tum pecuniae tum mentis Erasm. in Prov. calicum remiges. chil. 4. cent. 7. Pro. 41.}
\setmarginnote{1898}{Ser. 33. ad frat. in Eremo.}
\setmarginnote{1899}{Liberae unius horae insaniam aeterno temporis taedio pensant.}
\setmarginnote{1900}{Menander.}
\setmarginnote{1901}{Prov. 5.}
\setmarginnote{1902}{Merlin, cocc. That momentary pleasure blots out the eternal glory of a heavenly life..}
\setmarginnote{1903}{Hor.}
\setmarginnote{1904}{Sagitta quae animam penetrat, leviter penetrat, sed non leve infligit vulnus sup. cant.}
\setmarginnote{1905}{Qui omnem pecuniarum contemptum habent, et nulli imaginationis totius munsi se immiscuerint, et tyrannicas corporis concupiscentias sustinuerint hi multoties capti a vana gloria omnia perdiderunt.}
\setmarginnote{1906}{Hac correpti non cogitant de medela.}
\setmarginnote{1907}{Dii talem a terris avertite pestem.}
\setmarginnote{1908}{Ep ad Eustochium, de custod. virgin.}
\setmarginnote{1909}{Lyps. Ep. ad Bonciarium.}
\setmarginnote{1910}{Ep. lib. 9. Omnia tua scripta pulcherrima existimo, maxime tamen illa, quae de nobis.}
\setmarginnote{1911}{Exprimere non possum quam sit jucundum, \&c.}
\setmarginnote{1912}{Hierom. et licet nos indignos dicimus et calidus rubor ora perfundat, attamen ad laudem suam intrinsecus animae laetantur.}
\setmarginnote{1913}{Thesaur. Theo.}
\setmarginnote{1914}{Nec enim mihi cornea fibra est. Per.}
\setmarginnote{1915}{E manibus illis, Nascentur violae. Pers. 1. Sat.}
\setmarginnote{1916}{Omnia enim nostra, supra modum placent.}
\setmarginnote{1917}{Fab. l. 10. c. 3. Ridentur mala componunt carmina, verum gaudent scribentes, et se venerantur, et ultra. Si taceas laudant, quicquid scripsere beati. Hor. ep. 2. l. 2.}
\setmarginnote{1918}{Luke xviii. 10.}
\setmarginnote{1919}{De meliore luto finxit praecordia Titan.}
\setmarginnote{1920}{Auson. sap. Chil. 3. cent. 10. pro. 97.}
\setmarginnote{1921}{Qui se crederet neminem ulla u re praestantiorem.}
\setmarginnote{1922}{Tanto fastu scripsit, ut Alexandri gesta inferiora scriptis suis existimaret, Io. Vossius lib. 1. cap. 9. de hist.}
\setmarginnote{1923}{Plutarch. vie. Catonis.}
\setmarginnote{1924}{Nemo unquam Poeta aut Orator, qui quenquam se meliorem arbitraretur.}
\setmarginnote{1925}{Consol. ad Pammachium mundi Philosophus, gloriae animal, et popularis aurae et rumorum venale mancipium.}
\setmarginnote{1926}{Epist. 5. Capitoni suo Diebus ac noctibus, hoc solum cogito si qua me possum levare humo. Id voto meo sufficit, \&c.}
\setmarginnote{1927}{Tullius.}
\setmarginnote{1928}{Ut nomen meum scriptis, tuis illustretur. Inquies animus studio aeternitatis, noctes et dies angebatur. Hensius forat. uneb. de Scal.}
\setmarginnote{1929}{Hor. art. Poet.}
\setmarginnote{1930}{Od. Vit. l. 3. Jamque opus exegi. Vade liber felix Palingen. lib. 18.}
\setmarginnote{1931}{In lib. 8.}
\setmarginnote{1932}{De ponte dejicere.}
\setmarginnote{1933}{Sueton. lib. degram.}
\setmarginnote{1934}{Nihil libenter audiunt, nisi laudes suas.}
\setmarginnote{1935}{Epis. 56. Nihil aliud dies noctesque cogitant nisi ut in studiis suis laudentur ab hominibus.}
\setmarginnote{1936}{Quae major dementia aut dici, aut excogitari potest, quam sic ob gloriam cruciari? Insaniam istam domine longe fac a me. Austin. cons. lib. 10. cap. 37.}
\setmarginnote{1937}{As Camelus in the novel, who lost his ears while he was looking for a pair of horns.}
\setmarginnote{1938}{Mart. l. 5. 51.}
\setmarginnote{1939}{Hor. Sat. 1. l. 2.}
\setmarginnote{1940}{Lib. cont. Philos. cap. 1.}
\setmarginnote{1941}{Tul. som. Scip.}
\setmarginnote{1942}{Boethius.}
\setmarginnote{1943}{Putean. Cisalp. hist. lib. 1.}
\setmarginnote{1944}{Plutarch. Lycurgo.}
\setmarginnote{1945}{Epist. 13. Illud te admoneo, ne eorum more facias, qui non proficere, sed conspici cupiunt, quae in habitu tuo, aut genere vitae notabilia sunt. Asperum cultum et vitiosum caput, negligentiorem barbam, indictum argento odium, cubile humi positum, et quicquid ad laudem perversa via sequitur evita.}
\setmarginnote{1946}{Per.}
\setmarginnote{1947}{Quis vero tam bene modulo suo metiri se novit, ut eum assiduae et immodicae laudationes non moveant? Hen. Steph.}
\setmarginnote{1948}{Mart.}
\setmarginnote{1949}{Stroza. If you will accept divine honours, we will willingly erect and consecrate altars to you.}
\setmarginnote{1950}{Justin.}
\setmarginnote{1951}{Livius. Gloria tantum elatus, non ira, in medios hostes irruere, quod completis muris conspici se pugnantem, a muro spectantibus, egregium ducebat.}
\setmarginnote{1952}{Applauded virtue grows apace, and glory includes within it an immense impulse.}
\setmarginnote{1953}{I demens, et suevas curre per Alpes, Aude Aliquid, \&c. ut pueris placeas, et declamatio fias. Juv. Sat. 10.}
\setmarginnote{1954}{In moriae Encom.}
\setmarginnote{1955}{Juvenal. Sat. 4.}
\setmarginnote{1956}{There is nothing which overlauded power will not presume to imagine of itself.}
\setmarginnote{1957}{Sueton. c. 12. in Domitiano.}
\setmarginnote{1958}{Brisonius.}
\setmarginnote{1959}{Antonius ab assentatoribus evectus Librum se patrem apellari jussit, et pro deo se venditavit redimitus hedera, et corona velatus aurea, et thyrsum tenens, cothurnisque succinctus curru velut Liber pater vectus est Alexandriae. Pater. vol. post.}
\setmarginnote{1960}{Minervae nuptias ambit, tanto furore percitus, ut satellites mitteret ad videndum num dea in thalamis venisset, \&c.}
\setmarginnote{1961}{Aelian. li. 12.}
\setmarginnote{1962}{De mentis alienat. cap. 3.}
\setmarginnote{1963}{Sequiturque superbia formam. Livius li. 11. Oraculum est, vivida saepe ingenia, luxuriare hac et evanescere multosque sensum penitus amisisse. Homines intuentur, ac si ipsi non essent homines.}
\setmarginnote{1964}{Galeus de rubeis, civis noster faber ferrarius, ob inventionem instrumenti Cocleae olim Archimedis dicti, prae laetitia insanivit.}
\setmarginnote{1965}{Insania postmodum correptus, ob nimiam inde arrogantiam.}
\setmarginnote{1966}{Bene ferre magnam disce fortunam Hor. Fortunam reverenter habe, quicunque repente Dives ab exili progrediere loco. Ausonius.}
\setmarginnote{1967}{Processit squalidus et submissus, ut hesterni Diei gaudium intemperans hodie castigaret.}
\setmarginnote{1968}{Uxor Hen. 8.}
\setmarginnote{1969}{Neutrius se fortunae extremum libenter experturam dixit: sed si necessitas alterius subinde imponeretur, optare se difficilem et adversam: quod in hac nulli unquam defuit solatium, in altera multis consilium, \&c. Lod. Vives.}
\setmarginnote{1970}{Peculiaris furor, qui ex literis fit.}
\setmarginnote{1971}{Nihil magis auget, ac assidua studia, et profundae cogitationes.}
\setmarginnote{1972}{Non desunt, qui ex jugi studio, et intempestiva lucubratione, huc devenerunt, hi prae caeteris enim plerunque melancholia solent infestari.}
\setmarginnote{1973}{Study is a continual and earnest meditation, applied to something with great desire. Tully.}
\setmarginnote{1974}{Et illi qui sunt subtilis ingenii, et multae praemeditationis, de facili incidunt in melancholiam.}
\setmarginnote{1975}{Ob studiorum solicitudinem lib. 5. Tit. 5.}
\setmarginnote{1976}{Gaspar Ens Thesaur Polit. Apoteles. 31. Graecis hanc pestem relinquite quae dubium non est, quin brevi omnem iis vigorem ereptura Martiosque spiritus exhaustura sit; Ut ad arma tractanda plane inhabiles futuri sint.}
\setmarginnote{1977}{Knoles Turk. Hist.}
\setmarginnote{1978}{Acts, xxvi. 24.}
\setmarginnote{1979}{Nimiis studiis melancholicus evasit, dicens se Biblium in capite habere.}
\setmarginnote{1980}{Cur melancholia assidua, crebrisque deliramentis vexentur eorum animi ut desipere cogantur.}
\setmarginnote{1981}{Solers quilibet artifex instrumenta sua diligentissime curat, penicellos pictor; malleos incudesque faber ferrarius; miles equos, arma venator, auceps aves, et canes, Cytharam Cytharaedus, \&c. soli musarum mystae tam negligentes sunt, ut instrumentum illud quo mundum universum metiri solent, spiritum scilicet, penitus negligere videantur.}
\setmarginnote{1982}{Arcus et arma tibi non sunt imitanda Dianae. Si nunquam cesses tendere mollis erit. Ovid.}
\setmarginnote{1983}{Ephemer.}
\setmarginnote{1984}{Contemplatio cerebrum exsiccat et extinguit calorem naturalem, unde cerebrum frigidum et siccum evadit quod est melancholicum. Accedit ad hoc, quod natura in contemplatione, cerebro prorsus cordique intenta, stomachum heparque destituit, unde ex alimentis male coctis, sanguis crassus et niger efficitur, dum nimio otio membrorum superflui vapores non exhalant.}
\setmarginnote{1985}{Cerebrum exsiccatur, corpora sensim gracilescunt.}
\setmarginnote{1986}{Studiosi sunt Cacectici et nunquam bene colorati, propter debilitatem digestivae facultatis, multiplicantur in iis superfluitates. Jo. Voschius parte 2. cap. 5. de peste.}
\setmarginnote{1987}{Nullus mihi per otium dies exit, partem noctis studiis dedico, non vero somno, sed oculos vigilia fatigatos cadentesque, in operam detineo.}
\setmarginnote{1988}{Johannes Hanuschias Bohemus. nat. 1516. eruditus vir, nimiis studiis in Phrenesin incidit. Montanus instances in a Frenchman of Tolosa.}
\setmarginnote{1989}{Cardinalis Caecius; ob laborem, vigiliam, et diuturna studia factus Melancholicus.}
\setmarginnote{1990}{Perls. Sat. 3. They cannot fiddle; but, as Themistocles said, he could make a small town become a great city.}
\setmarginnote{1991}{Perls. Sat.}
\setmarginnote{1992}{Ingenium sibi quod vanas desumpsit Athenas et septem studiis annos dedit, insenuitque. Libris et curis statua taciturnius exit, Plerunque et risu populum quatit, Hor. ep. 1. lib. 2.}
\setmarginnote{1993}{Translated by M. B. Holiday.}
\setmarginnote{1994}{Thomas rubore confusus dixit se de argumento cogitasse.}
\setmarginnote{1995}{Plutarch. vita Marcelli, Nec sensit urbem captam, nec milites in domum irruentes, adeo intentus studiis, \&c.}
\setmarginnote{1996}{Sub Furiae larva circumivit urbem, dictitans se exploratorem ab inferis venisse, delaturum daemonibus mortalium pecata.}
\setmarginnote{1997}{Petronius. Ego arbitror in scholis stultissimos fieri, quia nihil eorum quae in usu habemus aut audiunt aut vident.}
\setmarginnote{1998}{Novi meis diebus, plerosque studiis literarum deditos, qui disciplinis admodum abundabant, sed si nihil civilitatis habent, nec rem publ. nec domesticam regere norant. Stupuit Paglarensis et furti vilicum accusavit, qui suem foetam undecim pocellos, asinam unum duntaxat pullam enixam retulerat.}
\setmarginnote{1999}{Lib. 1. Epist. 3. Adhuc scholasticus tantum est; quo genere hominum, nihil aut est simplicius, aut sincerius aut melius.}
\setmarginnote{2000}{Jure privilegiandi, qui ob commune bonum abbreviant sibi vitam.}
\setmarginnote{2001}{Virg. 6. Aen.}
\setmarginnote{2002}{Plutarch, vita ejus. Certum agricolationis lucrum, \&c.}
\setmarginnote{2003}{Quotannis fiunt consules et proconsules. Rex et Poeta quotannis non nascitur.}
\setmarginnote{2004}{Mat. 21.}
\setmarginnote{2005}{Hor. epis. 20. l. 1.}
\setmarginnote{2006}{Lib 1. de contem. amor.}
\setmarginnote{2007}{Satyricon.}
\setmarginnote{2008}{Juv, Sat. 5.}
\setmarginnote{2009}{Ars colit astra.}
\setmarginnote{2010}{Aldrovandus de Avibus. l. 12. Gesner, \&c.}
\setmarginnote{2011}{Literas habent queis sibi et fortunae suae maledicant. Sat. Menip.}
\setmarginnote{2012}{Lib. de libris Propriis fol. 24.}
\setmarginnote{2013}{Praefat translat. Plutarch.}
\setmarginnote{2014}{Polit. disput. laudibus extollunt eos ac si virtutibus pollerent quos ob infinita scelera potius vituperare oporteret.}
\setmarginnote{2015}{Or as horses know not their strength, they consider not their own worth.}
\setmarginnote{2016}{Plura ex Simonidis familiaritate Hieron consequutus est, quam ex Hieronis Simonides.}
\setmarginnote{2017}{Hor. lib. 4. od. 9.}
\setmarginnote{2018}{Inter inertes et Plebeios fere jacet, ultimum locum habens, nisi tot artis virtutisque insignia, turpiter, obnoxie, supparisitando fascibus subjecerit protervae insolentisque potentiae, Lib. I. de contempt. rerum fortuitarum.}
\setmarginnote{2019}{Buchanan. eleg. lib.}
\setmarginnote{2020}{In Satyricon. intrat senex, sed culta non ita speciosus, ut facile appararet eum hac nota literatum esse, quos divites odisse solent. Ego inquit Poeta sum: Quare ergo tam male vestitus es? Propter hoc ipsum; amor ingenii neminem unquam divitem fecit.}
\setmarginnote{2021}{Petronius Arbiter.}
\setmarginnote{2022}{Oppressus paupertate animus nihil eximium, aut sublime cogitare potest, amoenitates literarum, aut elegantiam, quoniam nihil praesidii in his ad vitae commodum videt, primo negligere, mox odisse incipit. Hens.}
\setmarginnote{2023}{Epistol. quaest. lib. 4. Ep. 21.}
\setmarginnote{2024}{Ciceron. dial. lib. 2.}
\setmarginnote{2025}{Epist. lib. 2.}
\setmarginnote{2026}{Ja. Dousa Epodon. lib. 2. car. 2.}
\setmarginnote{2027}{Plautus.}
\setmarginnote{2028}{Barc. Argenis lib. 3.}
\setmarginnote{2029}{Joh. Howson 4 Novembris 1597. the sermon was printed by Arnold Hartfield.}
\setmarginnote{2030}{Pers. Sat. 3.}
\setmarginnote{2031}{E lecto exsilientes, ad subitum tintinnabuli plausum quasi fulmine territi. I.}
\setmarginnote{2032}{Mart.}
\setmarginnote{2033}{Mart.}
\setmarginnote{2034}{Sat. Menip.}
\setmarginnote{2035}{Lib. 3. de cons.}
\setmarginnote{2036}{I had no money, I wanted impudence, I could not scramble, temporise, dissemble: non pranderet olus, \&c. vis dicam, ad palpandum et adulandum penitus insulsus, recudi non possum, jam senior ut sim talis, et fingi nolo, utcunque male cedat in rem meam et obscurus inde delitescam.}
\setmarginnote{2037}{Vit. Crassi. nec facile judicare potest utrum pauperior cum primo ad Crassum, \&c.}
\setmarginnote{2038}{Deum habent iratum, sibique mortem aeternam acquirunt, aliis miserabilem ruinam. Serrarius in Josuam, 7. Euripides.}
\setmarginnote{2039}{Nicephorus lib. 10. cap. 5.}
\setmarginnote{2040}{Lord Cook, in his Reports, second part, fol. 44.}
\setmarginnote{2041}{Euripides.}
\setmarginnote{2042}{Sir Henry Spelman, de non temerandis Ecclesiis.}
\setmarginnote{2043}{1 Tim. 42.}
\setmarginnote{2044}{Hor.}
\setmarginnote{2045}{Primum locum apud omnes gentes habet patritius deorum cultus, et geniorum, nam hunc diutissime custodiunt, tam Graeci quam Barbari, \&c.}
\setmarginnote{2046}{Tom. 1. de steril. trium annorum sub Elia sermone.}
\setmarginnote{2047}{Ovid. Fast.}
\setmarginnote{2048}{De male quaesitis vix gaudet tertius haeres.}
\setmarginnote{2049}{Strabo. lib. 4. Geog.}
\setmarginnote{2050}{Nihil facilius opes evertet, quam avaritia et fraude parta. Et si enim seram addas tali arcae et exteriore janua et vecte eam communias, intus tamen fraudem et avaritiam, \&c. In 5. Corinth.}
\setmarginnote{2051}{Acad. cap. 7.}
\setmarginnote{2052}{Ars neminem habet inimicum praeter ignorantem.}
\setmarginnote{2053}{He that cannot dissemble cannot live.}
\setmarginnote{2054}{Epist. quest. lib. 4. epist. 21. Lipsius.}
\setmarginnote{2055}{Dr. King, in his last lecture on Jonah, sometime right reverend lord bishop of London.}
\setmarginnote{2056}{Quibus opes et otium, hi barbaro fastu literas contemnunt.}
\setmarginnote{2057}{Lucan. lib. 8.}
\setmarginnote{2058}{Spartian. Soliciti de rebus minis.}
\setmarginnote{2059}{Nicet. 1. Anal. Fumis lucubrationum sordebant.}
\setmarginnote{2060}{Grammaticis olim et dialecticis Jurisque Professoribus, qui specimen eruditionis dedissent eadem dignitatis insignia decreverunt Imperatores, quibus ornabant heroas. Erasm. ep. Jo. Fabio epis. Vien.}
\setmarginnote{2061}{Probus vir et Philosophus magis praestat inter alios homines, quam rex inclitus inter plebeios.}
\setmarginnote{2062}{Heinsius praefat. Poematum.}
\setmarginnote{2063}{Servile nomen Scholaris jam.}
\setmarginnote{2064}{Seneca.}
\setmarginnote{2065}{Haud facile emergunt, \&c.}
\setmarginnote{2066}{Media quod noctis ab hora sedisti qua nemo faber, qua nemo sedebat, qui docet obliquo lanam deducere ferro: rara tamen merces. Juv. Sat. 7.}
\setmarginnote{2067}{Chil. 4. Cent. 1. adag. J.}
\setmarginnote{2068}{Had I done as others did, put myself forward, I might have haply been as great a man as many of my equals.}
\setmarginnote{2069}{Catullus, Juven.}
\setmarginnote{2070}{All our hopes and inducements to study are centred in Caesar alone.}
\setmarginnote{2071}{Nemo est quem non Phaebus hic noster, solo intuitu lubentiorem reddat.}
\setmarginnote{2072}{Panegyr.}
\setmarginnote{2073}{Virgil.}
\setmarginnote{2074}{Rarus enim ferme sensus communis in illa Fortuna. Juv. Sat. 8.}
\setmarginnote{2075}{Quis enim generosum dixerit hunc que Indignus genere, et praeclaro nomine tantum, Insignis. Juve. Sat. 8.}
\setmarginnote{2076}{I have often met with myself, and conferred with divers worthy gentlemen in the country, no whit inferior, if not to be preferred for divers kinds of learning to many of our academics.}
\setmarginnote{2077}{Ipse licet Musis venias comitatus Homere, Nil tamen attuleris, ibis Homere foras.}
\setmarginnote{2078}{Et legat historicos auctores, noverit omnes Tanquam ungues digitosque suos. Juv. Sat. 7.}
\setmarginnote{2079}{Juvenal.}
\setmarginnote{2080}{Tu vero licet Orpheus sis, saxa sono testudinis emolliens, nisi plumbea eorum corda, auri vel argenti malleo emollias, \&c. Salisburiensis Policrat. lib. 5. c. 10.}
\setmarginnote{2081}{Juven. Sat. 7.}
\setmarginnote{2082}{Euge bene, no need, Dousa epod. lib. 2.-dos ipsa scientia sibique congiarium est.}
\setmarginnote{2083}{Quatuor ad portas Ecclesias itus ad omnes; sanguinis aut Simonis, praesulis atque Dei. Holcot.}
\setmarginnote{2084}{Lib. contra Gentiles de Babila martyre.}
\setmarginnote{2085}{Praescribunt, imperant, in ordinem cogunt, ingenium nostrum prout ipsis vicebitur, astriugunt et relaxant ut papilionem pueri aut bruchum filo demitturit, aut attrahunt, nos a libidine sua pendere aequum censentes. Heinsins.}
\setmarginnote{2086}{Joh. 5.}
\setmarginnote{2087}{Epist. lib. 2. Jam suffectus in locum demortui, protinus exortus est adversarius, \&c. post multos labores, sumptus, \&c.}
\setmarginnote{2088}{Jun. Acad. cap. 6.}
\setmarginnote{2089}{Accipiamus pecuniam, demittamus asinum ut apud Patavinos, Italos.}
\setmarginnote{2090}{Hos non ita pridem perstrinxi, in Philosophastro Commaedia latina, in Aede Christi Oxon, publice habita, Anno 1617. Feb. 16.}
\setmarginnote{2091}{Sat. Menip.}
\setmarginnote{2092}{2 Cor. vii. 17.}
\setmarginnote{2093}{Comment. in Gal.}
\setmarginnote{2094}{Heinsius.}
\setmarginnote{2095}{Ecclesiast.}
\setmarginnote{2096}{Luth. in Gal.}
\setmarginnote{2097}{Pers. Sat. 2.}
\setmarginnote{2098}{Sallust.}
\setmarginnote{2099}{Sat. Menip.}
\setmarginnote{2100}{Budaeus de Asse, lib. 5.}
\setmarginnote{2101}{Lib. de rep. Gallorum.}
\setmarginnote{2102}{Campian.}
\setmarginnote{2103}{}
\setmarginnote{2104}{Proem lib. 2. Nulla ars constitui poset.}
\setmarginnote{2105}{Lib. 1. c. 19. de morborum causis. Quas declinare licet aut nulla necessitate utimur.}
\setmarginnote{2106}{Quo semel est imbuta recens servabit odorem Testa diu. Hor.}
\setmarginnote{2107}{Sicut valet ad fingendas corporis atque animi similitudines vis et natura seminis, sic quoque lactis proprietas. Neque id in hominibus solum, sed in pecudibus animadversum. Nam si ovium lacte hoedi, aut caprarum agni alerentur, constat fieri in his lanam duriorem, in illis capillum gigni severiorem.}
\setmarginnote{2108}{Adulta in ferarum persequatione ad miraculum usque sagax.}
\setmarginnote{2109}{Tam animal quodlibet quam homo, ab illa cujus lacte nutritur, naturam contrahit.}
\setmarginnote{2110}{Improba, informis, impudica, temulenta, nutrix, \&c. quoniam in moribus efformandis magnam saepe partem igenium altricis et natura lactis tenet.}
\setmarginnote{2111}{Hircanaeque admorunt ubera Tigres, Virg.}
\setmarginnote{2112}{Lib. 2. de Caesaribus.}
\setmarginnote{2113}{Beda c. 27. l. 1 Eccles. hist.}
\setmarginnote{2114}{Ne insitivo lactis alimento degeneret corpus, et animus corrumpatur.}
\setmarginnote{2115}{Lib. 3. de civ. convers.}
\setmarginnote{2116}{Stephanus.}
\setmarginnote{2117}{To. 2. Nutrices non quasvis, sed maxime probas deligamus.}
\setmarginnote{2118}{Nutrix non sit lasciva aut temulenta. Hier.}
\setmarginnote{2119}{Prohibendum ne stolida lactet.}
\setmarginnote{2120}{Pers.}
\setmarginnote{2121}{Nutrices interdum matribus sunt meliores.}
\setmarginnote{2122}{Lib. de morbis capitis, cap. de mania; Haud postrema causa supputatur educatio, inter has mentis abalienationis causas. Injusta noverca.}
\setmarginnote{2123}{Lib. 2. cap. 4.}
\setmarginnote{2124}{Idem. Et quod maxime nocet, dum in teneris ita timent nihil conantur.}
\setmarginnote{2125}{The pupil's faculties are perverted by the indiscretion of the master.}
\setmarginnote{2126}{Praefat. ad Testam.}
\setmarginnote{2127}{Plus mentis paedagogico supercilio abstulit, quam unquam praeceptis suis sapientiae instillavit.}
\setmarginnote{2128}{Ter. Adel. 3. 4.}
\setmarginnote{2129}{Idem. Ac. 1. sc. 2. Let him feast, drink, perfume himself at my expense: If he be in love, I shall supply him with money. Has he broken in the gates? they shall be repaired. Has he torn his garments? they shall be replaced. Let him do what he pleases, take, spend, waste, I am resolved to submit.}
\setmarginnote{2130}{Camerarius em. 77. cent. 2. hath elegantly expressed it an emblem, perdit amando, \&c.}
\setmarginnote{2131}{Prov. xiii. 24. He that spareth the rod hates his son.}
\setmarginnote{2132}{Lib. de consol. Tam Stulte pueros diligimus ut odisse potius videamur, illos non ad virtutem sed ad injuriam, non ad eruditionem sed ad luxum, non ad virtutem sed voluptatem educantes.}
\setmarginnote{2133}{Lib. 1. c. 3. Educatio altera natura, alterat animos et voluntatem, atque utinam (inquit) liberorum nostrorum mores non ipsi perderemus, quum infantiam statim deliciis solvimus: mollior ista educatio, quam indulgentiam vocamus, nervos omnes, et mentis et corporis frangit; fit ex his consuetudo, inde natura.}
\setmarginnote{2134}{Perinde agit ac siquis de calceo sit sollicitus, pedem nihil curet. Juven. Nil patri minus est quam filius.}
\setmarginnote{2135}{Lib. 3. de sapient: qui avaris paedagogis pueros alendos dant, vel clausos in coenobiis jejunare simul et sapere, nihil aliud agunt, nisi ut sint vel non sine stultitia eruditi, vel non integra vita sapientes.}
\setmarginnote{2136}{Terror et metus maxime ex improviso accedentes ita animum commovent, ut spiritus nunquam recuperent, gravioremque melancholiam terror facit, quam quae ab interna causa fit. Impressio tam fortis in spiritibus humoribusque cerebri, ut extracta tota sanguinea massa, aegre exprimatur, et haec horrenda species melancholiae frequenter oblata mihi, omnes exercens, viros, juvenes, senes.}
\setmarginnote{2137}{Tract. de melan. cap. 7. et 8. non ab intemperie, sed agitatione, dilatatione, contractione, motu spirituum.}
\setmarginnote{2138}{Lib. de fort. et virtut. Alex. praesertim ineunte periculo, ubi res prope adsunt terribiles.}
\setmarginnote{2139}{Fit a visione horrenda, revera apparente, vel per insomnia, Platerus.}
\setmarginnote{2140}{A painter's wife in Basil, 1600. Somniavit filium bello mortuum, inde Melancholica consolari noluit.}
\setmarginnote{2141}{Senec. Herc. Oet.}
\setmarginnote{2142}{Quarta pars comment. de Statu religionis in Gallia sub Carolo. 9. 1572.}
\setmarginnote{2143}{Ex occursu daemonum aliqui furore corripiuntur, et experientia notum est.}
\setmarginnote{2144}{Lib. 8. in Arcad.}
\setmarginnote{2145}{Lucret.}
\setmarginnote{2146}{Puellae extra urbem in prato concurrentes, \&c. maesta et melancholica domum rediit per dies aliquot vexata, dum mortua est. Plater.}
\setmarginnote{2147}{Altera trans-Rhenana ingressa sepulchrum recens apertum, vidit cadaver, et domum subito reversa putavit eam vocare, post paucos dies obiit, proximo sepulchre collocata. Altera patibulum sero praeteriens, metuebat ne urbe exclusa illic pernoctaret, unde melancholica facta, per multos annos laboravit. Platerus.}
\setmarginnote{2148}{Subitus occursus, inopinata lectio.}
\setmarginnote{2149}{Lib. de auditione.}
\setmarginnote{2150}{Theod. Prodromus lib. 7. Amorum.}
\setmarginnote{2151}{Effuso cernens fugientes agmine turmas, Quis mea nunc inflat cornua Faunus ait. Alciat. embl. 122.}
\setmarginnote{2152}{Jud. 6. 19.}
\setmarginnote{2153}{Plutarchus vita ejus.}
\setmarginnote{2154}{In furorem cum sociis versus.}
\setmarginnote{2155}{Subitarius terrae motus.}
\setmarginnote{2156}{Caepit inde desipere cum dispendio sanitatis, inde adeo dementans, ut sibi ipsi mortem inferret.}
\setmarginnote{2157}{Historica relatio de rebus Japonicis Tract. 2. de legat, regis Chinensis, a Lodovico Frois Jesuita. A. 1596. Fuscini derepente tanta acris caligo et terraemotus, ut multi capite dolerent, plurimus cor moerore et melancholia obrueretur. Tantum fremitum edebat, ut tonitru fragorem imitari videretur, tantamque, \&c. In urbe Sacai tam horrificus fuit, ut homines vix sui compotes essent a sensibus abalienati, moerore oppressi tam horrendo spectaculo, \&c.}
\setmarginnote{2158}{Quum subit illius tristissima noctis Imago.}
\setmarginnote{2159}{Qui solo aspectu medicinae movebatur ad purgandum.}
\setmarginnote{2160}{Sicut viatores si ad saxum impegerint, aut nautae, memores sui casus, non ista modo quae offendunt, sed et similia horrent perpetuo et tremunt.}
\setmarginnote{2161}{Leviter volant graviter vulnerant. Bernardus.}
\setmarginnote{2162}{Ensis sauciat corpus, mentem sermo.}
\setmarginnote{2163}{Sciatis eum esse qui a nemine fere aevi sui magnate, non illustre stipendium habuit, ne mores ipsorum Satyris suis notaret. Gasp. Barthius praefat. parnodid.}
\setmarginnote{2164}{Jovius in vita ejus, gravissime tulit famosis libellis nomen suum ad Pasquilli statuam fuisse laceratum, decrevitque ideo statuam demoliri, \&c.}
\setmarginnote{2165}{Plato, lib. 13. de legibus. Qui existimationem curant, poetas vereantur, quia magnam vim habent ad laudandum et vituperandum.}
\setmarginnote{2166}{Petulanti splene cachinno.}
\setmarginnote{2167}{Curial. lib. 2. Ea quorundam est inscitia, ut quoties loqui, toties mordere licere sibi putent.}
\setmarginnote{2168}{Ter. Eunuch.}
\setmarginnote{2169}{Hor. ser. lib. 2. Sat. 4. Provided he can only excite laughter, he spares not his best friend.}
\setmarginnote{2170}{Lib. 2.}
\setmarginnote{2171}{De orat.}
\setmarginnote{2172}{Laudando, et mira iis persuadendo.}
\setmarginnote{2173}{Et vana inflatus opinione, incredibilia ac ridenda quaedam Musices praecepta commentaretur, \&c.}
\setmarginnote{2174}{Ut voces nudis parietibus illisae, suavius ac acutius resilirent.}
\setmarginnote{2175}{Immortalitati et gloriae suae prorsus invidentes.}
\setmarginnote{2176}{2. 2 dae quaest 75. Irrisio mortale peccatum.}
\setmarginnote{2177}{Psal. xv. 3.}
\setmarginnote{2178}{Balthazar Castilio lib. 2. de aulico.}
\setmarginnote{2179}{De sermone lib. 4. cap. 3.}
\setmarginnote{2180}{Fol. 55. Galateus.}
\setmarginnote{2181}{Tully Tusc. quaest.}
\setmarginnote{2182}{Every reproach uttered against one already condemned is mean-spirited.}
\setmarginnote{2183}{Mart. lib. 1. epig. 35.}
\setmarginnote{2184}{Tales joci ab injuriis non possint discerni. Galateus fo. 55.}
\setmarginnote{2185}{Pybrac in his Quadraint 37.}
\setmarginnote{2186}{Ego hujus misera fatuitate et dementia conflictor. Tull. ad Attic li. 11.}
\setmarginnote{2187}{Miserum est aliena vivere quadra. Juv.}
\setmarginnote{2188}{Crambae bis coctae. Vitae me redde priori.}
\setmarginnote{2189}{Hor.}
\setmarginnote{2190}{De tranquil animae.}
\setmarginnote{2191}{Lib. 8.}
\setmarginnote{2192}{Tullius Lepido Fam. 10. 27.}
\setmarginnote{2193}{Boterus l. 1. polit. cap. 4.}
\setmarginnote{2194}{Laet. descrip. Americae.}
\setmarginnote{2195}{If there be any inhabitants.}
\setmarginnote{2196}{In Taxari. Interdiu quidem collum vinctum est, et manus constricta, noctuvero totum corpus vincitur, ad has miserias accidit corporis faetor, strepitus ejulantium, somni brevitas, haec omnia plane molesta et intolerabilia.}
\setmarginnote{2197}{In 9 Rhasis.}
\setmarginnote{2198}{William the Conqueror's eldest son.}
\setmarginnote{2199}{Salust. Romam triumpho ductus tandemque in carcerem conjectus, animi dolore periit.}
\setmarginnote{2200}{Camden in Wiltsh. miserum senem ita fame et calamitatibus in carcere fregit, inter mortis metum, et vitae tormenta, \&c.}
\setmarginnote{2201}{Vies hodie.}
\setmarginnote{2202}{Seneca.}
\setmarginnote{2203}{Com. ad Hebraeos.}
\setmarginnote{2204}{Part. 2. Sect. 3. Memb. 3.}
\setmarginnote{2205}{Quem ut difficilem morbum pueris tradere formidamus. Plut.}
\setmarginnote{2206}{Lucan. l. 1.}
\setmarginnote{2207}{As in the silver mines at Friburgh in Germany. Fines Morison.}
\setmarginnote{2208}{Euripides.}
\setmarginnote{2209}{Tom. 4. dial. minore periculo Solem quam hunc defixis oculis licet intueri.}
\setmarginnote{2210}{Omnis enim res, virtus, fama, decus, divina, humanaque pulchris Divitiis parent. Hor. Ser. l. 2. Sat. 3. Clarus eris, fortis justus, sapiens, etiam rex. Et quicquid volet. Hor.}
\setmarginnote{2211}{Et genus, et formam, regina pecunia donat. Money adds spirits, courage, \&c.}
\setmarginnote{2212}{Epist. ult. ad Atticum.}
\setmarginnote{2213}{Our young master, a fine towardly gentleman, God bless him, and hopeful; why? he is heir apparent to the right worshipful, to the right honourable, \&c.}
\setmarginnote{2214}{O nummi, nummi: vobis hunc praestat honorem.}
\setmarginnote{2215}{Exinde sapere eum omnes dicimus, ac quisque fortunam habet. Plaut. Pseud.}
\setmarginnote{2216}{Aurea fortuna, principum cubiculis reponi solita. Julius Capitolinus vita Antonini.}
\setmarginnote{2217}{Petronius.}
\setmarginnote{2218}{Theologi opulentis adhaerent, Jurisperiti pecuniosis, literati nummosis, liberalibus artifices.}
\setmarginnote{2219}{Multi illum juvenes, multae petiere puellae.}
\setmarginnote{2220}{He may have Danae to wife.}
\setmarginnote{2221}{Dummodo sit dives barbarus, ille placet.}
\setmarginnote{2222}{Plut. in Lucullo, a rich chamber so called.}
\setmarginnote{2223}{Panis pane melior.}
\setmarginnote{2224}{Juv. Sat. 5.}
\setmarginnote{2225}{Hor. Sat. 5. lib. 2.}
\setmarginnote{2226}{Bohemus de Turcis et Bredenbach.}
\setmarginnote{2227}{Euphormio.}
\setmarginnote{2228}{Qui pecuniam habens, elati sunt animis, lofty spirits, brave men at arms; all rich men are generous, courageous, \&c.}
\setmarginnote{2229}{Nummus ait pro me nubat Cornubia Romae.}
\setmarginnote{2230}{A diadem is purchased with gold; silver opens the way to heaven; philosophy may be hired for a penny; money controls justice; one obolus satisfies a man of letters; precious metal procures health; wealth attaches friends.}
\setmarginnote{2231}{Non fuit apud mortales ullum excellentius certamen, non inter celeres celerrimo, non inter robustos robustissimo, \&c.}
\setmarginnote{2232}{Quicquid libet licet.}
\setmarginnote{2233}{Hor. Sat. 5. lib. 2.}
\setmarginnote{2234}{Cum moritur dives concurrunt undique cives: Pauperis ad funus vix est ex millibus unus.}
\setmarginnote{2235}{Et modo quid fuit ignoscat mihi genius tuus, noluisses de manu ejus nummos accipere.}
\setmarginnote{2236}{that wears silk, satin, velvet, and gold lace, must needs be a gentleman.}
\setmarginnote{2237}{Est sanguis utque spiritus pecunia mortalibus.}
\setmarginnote{2238}{Euripides.}
\setmarginnote{2239}{Xenophon. Cyropaed. l. 8.}
\setmarginnote{2240}{In tenui rara est facundia panno. Juv.}
\setmarginnote{2241}{Hor. more worthless than rejected weeds.}
\setmarginnote{2242}{Egere est offendere, et indigere scelestum esse. Sat. Menip.}
\setmarginnote{2243}{Plaut. act. 4.}
\setmarginnote{2244}{Nullum tam barbarum, tam vile munus est, quod non lubentissime obire velit gens vilissima.}
\setmarginnote{2245}{Lausius orat. in Hispaniam.}
\setmarginnote{2246}{Laet. descrip. Americiae.}
\setmarginnote{2247}{Who daily faint beneath the burdens they are compelled to carry from place to place: for they carry and draw the loads which oxen and asses formerly used, \&c.}
\setmarginnote{2248}{Plautus.}
\setmarginnote{2249}{Leo. Afer. ca. ult. l. 1. edunt non ut bene vivant, sed ut fortiter laborent. Heinsius.}
\setmarginnote{2250}{Munster de rusticis Germaniae, Cosmog. cap. 27. lib. 3.}
\setmarginnote{2251}{Ter. Eunuch.}
\setmarginnote{2252}{Pauper paries factus, quem caniculae commingant.}
\setmarginnote{2253}{Lib. 1. cap ult.}
\setmarginnote{2254}{Deos omnes illis infensos diceres: tam pannosi, famefracti, tot assidue malis afficiuntur, tanquam pecora quibus splendor rationis emortuus.}
\setmarginnote{2255}{Peregrin. Hieros.}
\setmarginnote{2256}{Nihil omnino meliorem vitam degunt, quam ferae in silvis, jumenta in terris. Leo Afer.}
\setmarginnote{2257}{Bartholomeus a Casa.}
\setmarginnote{2258}{Ortelius in Helvetia. Qui habitant in Caesia valle ut plurimum latomi, in Oscella valle cultrorum fabri fumarii, in Vigetia sordidum genus hominum, quod repurgandis caminis victum parat.}
\setmarginnote{2259}{I write not this any ways to upbraid, or scoff at, or misuse poor men, but rather to condole and pity them by expressing, \&c.}
\setmarginnote{2260}{Chremilus, act. 4. Plaut.}
\setmarginnote{2261}{Paupertas durum onus miseris mortalibus.}
\setmarginnote{2262}{Vexat censura columbas.}
\setmarginnote{2263}{Deux ace non possunt, et sixeinque solvere nolunt; Omnibus est notum quater tre solvere totum.}
\setmarginnote{2264}{Scandia, Africa, Lithuania.}
\setmarginnote{2265}{Montaigne, in his Essays, speaks of certain Indians in France, that being asked how they liked the country, wondered how a few rich men could keep so many poor men in subjection, that they did not cut their throats.}
\setmarginnote{2266}{Augustas animas animoso in pectore versans.}
\setmarginnote{2267}{A narrow breast conceals a narrow soul.}
\setmarginnote{2268}{Donatus vit. ejus.}
\setmarginnote{2269}{Publius Scipio, Laelius and Furius, three of the most distinguished noblemen at that day in Rome, were of so little service to him, that he could scarcely procure a lodging through their patronage.}
\setmarginnote{2270}{Prov. xix. 7. Though he be instant, yet they will not.}
\setmarginnote{2271}{Petronius.}
\setmarginnote{2272}{Non est qui doleat vicem, ut Petrus Christum, jurant se hominem non novisse.}
\setmarginnote{2273}{Ovid, in Trist.}
\setmarginnote{2274}{Horat.}
\setmarginnote{2275}{Ter. Eunuchus, act. 2.}
\setmarginnote{2276}{Quid quod materiam praebet causamque jocandi: Si toca sordida sit, Juv. Sat. 2.}
\setmarginnote{2277}{Hor.}
\setmarginnote{2278}{In Phaenis.}
\setmarginnote{2279}{Odyss. 17.}
\setmarginnote{2280}{Idem.}
\setmarginnote{2281}{Mantuan.}
\setmarginnote{2282}{Since cruel fortune has made Sinon poor, she has made him vain and mendacious.}
\setmarginnote{2283}{De Africa Lib. 1. cap. ult.}
\setmarginnote{2284}{4. de legibus. furacissima paupertas, sacrilega, turbis, flagitiosa, omnium malorum opifex.}
\setmarginnote{2285}{Theognis.}
\setmarginnote{2286}{Dipnosophist lib. 12. Millies potius moriturum (si quis sibi mente constaret) quam tam vilis et aerumnosi victus communionem habere.}
\setmarginnote{2287}{Gasper Vilela Jesuita epist. Japon. lib.}
\setmarginnote{2288}{Mat. Riccius expedit. in Sinas lib. 1. c. 3.}
\setmarginnote{2289}{Vos Romani procreatos filios feris et canibus exponitis, nunc strangulatis vel in saxum eliditis, \&c.}
\setmarginnote{2290}{Cosmog. 4. lib. cap. 22. vendunt liberos victu carentes tanquam pecora interdum et seipsos; ut apud divites saturentur cibis.}
\setmarginnote{2291}{Vel honorum desperatione vel malorum perpessione fracti el fatigati, plures violentas manus sibi inferunt.}
\setmarginnote{2292}{Hor.}
\setmarginnote{2293}{Ingenio poteram superas volitare per arces: Ut me pluma levat, sic grave mergit onus.}
\setmarginnote{2294}{Terent.}
\setmarginnote{2295}{Hor. Sat. 3. lib. 1.}
\setmarginnote{2296}{They cannot easily rise in the world who are pinched by poverty at home.}
\setmarginnote{2297}{Paschalius.}
\setmarginnote{2298}{Petronius.}
\setmarginnote{2299}{Herodotus vita ejus. Scaliger in poet. Potentiorum aedes ostratim adiens, aliquid accipiebat, canens carmina sua, concomitante eum puerorum choro.}
\setmarginnote{2300}{Plautus Ampl.}
\setmarginnote{2301}{Ter. Act. 4. Scen. 3. Adelph. Hegio.}
\setmarginnote{2302}{Donat. vita ejus.}
\setmarginnote{2303}{Reduced to the greatest necessity, he withdrew from the gaze of the public to the most remote village in Greece.}
\setmarginnote{2304}{Euripides.}
\setmarginnote{2305}{Plutarch, vita ejus.}
\setmarginnote{2306}{Vita Ter.}
\setmarginnote{2307}{Gomesius lib. 3. c. 21. de sale.}
\setmarginnote{2308}{Ter. Eunuch. Act. 2. Scen. 2.}
\setmarginnote{2309}{Liv. dec. 9. l. 2.}
\setmarginnote{2310}{Comineus.}
\setmarginnote{2311}{He that hath 5\emph{l}. per annum coming in more than others, scorns him that has less, and is a better man.}
\setmarginnote{2312}{Prov. xxx. 8.}
\setmarginnote{2313}{De anima, cap. de maerore.}
\setmarginnote{2314}{Lib. 12. epist.}
\setmarginnote{2315}{Oh sweet offspring; oh my very blood; oh tender flower, \&c.}
\setmarginnote{2316}{Vir. 4. Aen.}
\setmarginnote{2317}{Patres mortuos coram astantes et filios, \&c. Marcellus Donatus.}
\setmarginnote{2318}{Epist. lib. 2. Virginium video audio defunctum cogito, alloquor.}
\setmarginnote{2319}{Calphurnius Graecus. Without thee, ah! wretched me, the lillies lose their whiteness, the roses become pallid, the hyacinth forgets to blush neither the myrtle nor the laurel retains its odours.}
\setmarginnote{2320}{Chaucer.}
\setmarginnote{2321}{Praefat. lib. 6.}
\setmarginnote{2322}{Lib. de obitu Satyri fratris.}
\setmarginnote{2323}{Ovid. Met.}
\setmarginnote{2324}{Plut. vita ejus.}
\setmarginnote{2325}{Nobilis matrona melancholica ob mortem mariti.}
\setmarginnote{2326}{Ex matris obitu in desperationem incidit.}
\setmarginnote{2327}{Mathias a Michou. Boter. Amphitheat.}
\setmarginnote{2328}{Lo. Vertoman. M. Polus Venetus lib. 1. cap. 54. perimunt eos quos in via obvios habent, dicentes, Ite, et domino nostro regi servile in alia vita. Nec tam in homines insaniunt sed in equos, \&c.}
\setmarginnote{2329}{Vita ejus.}
\setmarginnote{2330}{Lib. 4. vitae ejus, auream aetatem condiderat ad humani generis salutem quum nos statim ab optimi principis excessu. vere ferream, pateremur, famem, pestem, \&c.}
\setmarginnote{2331}{Lib. 5. de asse.}
\setmarginnote{2332}{Maph. They became fallen in feelings, as the great forest laments its fallen leaves.}
\setmarginnote{2333}{Ortelius Itinerario: ob annum integrum a cantu, tripudiis et saltationibus tota civitas abstinere jubetur.}
\setmarginnote{2334}{Virg.}
\setmarginnote{2335}{See Barletius de vita et ob. Scanderbeg. lib. 13. hist.}
\setmarginnote{2336}{Mat. Paris.}
\setmarginnote{2337}{Juvenalis.}
\setmarginnote{2338}{Multi qui res amatas perdiderant, ut filios, opes, non sperantes recuperare, propter assiduam talium considerationem melancholici fiunt, ut ipse vidi.}
\setmarginnote{2339}{Stanihurstus Hib. Hist.}
\setmarginnote{2340}{Cap. 3. Melancholia semper venit ab jacturam pecuniae, victoriae, repulsam, mortem liberorum, quibus longo post tempore animus torquetur, et a dispositione sit habitus.}
\setmarginnote{2341}{Consil. 26.}
\setmarginnote{2342}{Nubrigensis.}
\setmarginnote{2343}{Epig. 22.}
\setmarginnote{2344}{Lib. 8. Venet. hist.}
\setmarginnote{2345}{Templa ornamentis nudata, spoliata, in stabula equorum et asinorum versa, \&c. Insulae humi conculcatae, peditae, \&c.}
\setmarginnote{2346}{In oculis maritorum dilectissimae conjuges ab Hispanorum lixis constupratae sunt. Filiae magnatum thoris destinatae, \&c.}
\setmarginnote{2347}{Ita fastu ante unum mensem turgida civitas, et cacuminibos coelum pulsare visa, ad inferos usque paucis diebus dejecta.}
\setmarginnote{2348}{Sect. 2. Memb. 4. Subs. 3. fear from ominous accidents, destinies foretold.}
\setmarginnote{2349}{Accersunt sibi malum.}
\setmarginnote{2350}{Si non observemus, nihil valent. Polidor.}
\setmarginnote{2351}{Consil. 26. l. 2.}
\setmarginnote{2352}{Harm watch harm catch.}
\setmarginnote{2353}{Geor. Bucha.}
\setmarginnote{2354}{Juvenis solicitus de futuris frustra, factus melancholicus.}
\setmarginnote{2355}{Pausanius in Achaicis lib. 7. Ubi omnium eventus dignoscuntur. Speculum tenui suspensum funiculo demittunt: et ad Cyaneas petras ad Lycicae fontes, \&c.}
\setmarginnote{2356}{Expedit. in Sinas, lib. 1. c. 3.}
\setmarginnote{2357}{Timendo praeoccupat, quod vitat, ultro provocatque quod fugit, gaudetque moerens et lubens miser fuit. Heinsius Austriac.}
\setmarginnote{2358}{Must I be deprived of this life,-of those possessions?}
\setmarginnote{2359}{Tom. 4. dial. 8 Cataplo. Auri puri mille talenta, me hodie tibi daturum promitto, \&c.}
\setmarginnote{2360}{Ibidem. Hei mihi quae relinquenda praedia? quam fertiles agri! \&c.}
\setmarginnote{2361}{Adrian.}
\setmarginnote{2362}{Industria superflua circa res inutiles.}
\setmarginnote{2363}{Flavae secreta Minervae ut viderat Aglauros. Ov. Met. 2.}
\setmarginnote{2364}{Contra Philos. cap. 61.}
\setmarginnote{2365}{Mat. Paris.}
\setmarginnote{2366}{Seneca.}
\setmarginnote{2367}{Jos. Scaliger in Gnomit. To profess a disinclination for that knowledge which is beyond our reach, is pedantic ignorance.}
\setmarginnote{2368}{A virtuous woman is the crown of her husband. Prov. xii. 4. but she, \&c. \&c.}
\setmarginnote{2369}{Lib. 17. epist. 105.}
\setmarginnote{2370}{Titionatur, candelabratur, \&c.}
\setmarginnote{2371}{Daniel in Rosamund.}
\setmarginnote{2372}{Chalinorus lib. 9. de repub. Angl.}
\setmarginnote{2373}{Elegans virgo invita cuidam e nostratibus nupsit, \&c.}
\setmarginnote{2374}{Prov.}
\setmarginnote{2375}{De increm. urb. lib. 3. c. 3. tanquam diro mucrone confossi, his nulla requies, nulla delectatio, solicitudine, gemitu, furore, desperatione, timore, tanquam ad perpetuam aerumnam infeliciter rapti.}
\setmarginnote{2376}{Humfredus Llwyd epist. ad Abrahamum Ortelium. M. Vaughan in his Golden Fleece. Litibus et controversiis usque ad omnium bonorum consumptionem contendunt.}
\setmarginnote{2377}{Spretaeque injuria formae.}
\setmarginnote{2378}{Quaeque repulsa gravis.}
\setmarginnote{2379}{Lib. 36. c. 5.}
\setmarginnote{2380}{Nihil aeque amarum, quam diu pendere: quidam aequiore animo ferunt praecidi spem suam quam trahi. Seneca cap. 3. lib. 2. de Den. Virg. Plater observat. lib. 1.}
\setmarginnote{2381}{Turpe relinqui est, Hor.}
\setmarginnote{2382}{Scimus enim generosas naturas, nulla re citius moveri, aut gravius affici quam contemptu ac despicientia.}
\setmarginnote{2383}{At Atticum epist. lib. 12.}
\setmarginnote{2384}{Epist. ad Brutum.}
\setmarginnote{2385}{In Phaeniss.}
\setmarginnote{2386}{In laudem calvit.}
\setmarginnote{2387}{Ovid.}
\setmarginnote{2388}{E Cret.}
\setmarginnote{2389}{Hor. Car. Lib. 3. Ode. 27.}
\setmarginnote{2390}{Hist. lib. 6.}
\setmarginnote{2391}{Non mihi si centum linguae sint, oraque centum. Omnia causarum percurrere nomina possem.}
\setmarginnote{2392}{Celius l. 17. cap. 2.}
\setmarginnote{2393}{Ita mente exagitati sunt, ut in triremi se constitutos putarent, marique vadabundo tempestate jactatos, proinde naufragium veriti, egestis undique rebus vasa omnia in viam e fenestris, seu in mare praecipitarunt: postridie, \&c.}
\setmarginnote{2394}{Aram vobis servatoribus diis erigemus.}
\setmarginnote{2395}{Lib. de gemmis.}
\setmarginnote{2396}{Quae gestatae infelicem et tristem reddunt, curas augent, corpus siccant, somnum minuunt.}
\setmarginnote{2397}{Ad unum die mente alienatus.}
\setmarginnote{2398}{Part. 1. Sect. 2. Subsect. 3.}
\setmarginnote{2399}{Juven. Sat. 3.}
\setmarginnote{2400}{Intus bestiae minutae multae necant. Numquid minutissima sunt grana arenae? sed si arena amplius in navem mittatur, mergit illam; quam minutae guttae, pluviae? et tamen implent flumina, domus ejiciunt, timenda ergo ruina multiuidinis, si non magnitudinis.}
\setmarginnote{2401}{Mores sequuntur temperaturam corporis.}
\setmarginnote{2402}{Scintillae latent in corporibus.}
\setmarginnote{2403}{Gal. 5.}
\setmarginnote{2404}{Sicut ex animi afflictionibus corpus languescit: sic ex corporis vitiis, et morborum plerisque cruciatibus animum videmus hebetari, Galenus.}
\setmarginnote{2405}{Lib. 1. c. 16.}
\setmarginnote{2406}{Corporis itidem morbi animam per consensum, a lege consortii afficiunt, et quanquam objecta multos motus turbulentos in homine concitet, praecipua tamen causa in corde et humoribus spiritibusque consistit, \&c.}
\setmarginnote{2407}{Hor. Vide ante.}
\setmarginnote{2408}{Humores pravi mentum obnubilant.}
\setmarginnote{2409}{Hic humor vel a partis intemperie generatur vel relinquitur post inflammationes, vel crassior in venis conclusus vel torpidus malignam qualitatem contrabit.}
\setmarginnote{2410}{Saepe constat in febre hominem Melancholicum vel post febrem reddi, aut alium morbum. Calida intemperies innata, vel a febre contracta.}
\setmarginnote{2411}{Raro quis diuturno morbo laborat, qui non sit melancholicus, Mercurialis de affect. capitis lib. 1 c. 10 de Melanc.}
\setmarginnote{2412}{Ad nonum lib. Rhasis ad Almansor. c. 16. Universaliter a quacunque parte potest fieri melancholicus. Vel quia aduritur, vel quia non expellit superfluitatem excrementi.}
\setmarginnote{2413}{A Liene, juvidore, utero, et aliis partibus oritur.}
\setmarginnote{2414}{Materia Melancholiae aliquando in corde, in stomacho, hepate, ab hypocondriis, myruche, splene, cum ibi romanet humor melancholicus.}
\setmarginnote{2415}{Ex sanguine adusto, intra vel extra caput.}
\setmarginnote{2416}{Qui calidum cor habent, cerebrum humidum, facile melancholici.}
\setmarginnote{2417}{Sequitur melancholia malam intemperiem frigidam et siccam ipsius cerebri.}
\setmarginnote{2418}{Saepe fit ex calidiore cerebro, aut corpore colligente melancholiam. Piso.}
\setmarginnote{2419}{Vel per propriam affectionem, vel per consensum, cum vapores exhalant in cerebrum. Montalt. cap. 14.}
\setmarginnote{2420}{Aut ibi gignitur, melancholicus fumus, aut aliunde vehitur, alterando animales facultates.}
\setmarginnote{2421}{Ab intemperie cordis, modo calidiore, molo frigidiore.}
\setmarginnote{2422}{Epist. 209. Scoltzii.}
\setmarginnote{2423}{Officina humorum hepar concurrit, \&c.}
\setmarginnote{2424}{Ventriculus et venae meseraicae concurrunt, quod hae partes obstructae sunt, \&c.}
\setmarginnote{2425}{Per se sanguinem adurentes.}
\setmarginnote{2426}{Lien frigidus et siccus c. 13.}
\setmarginnote{2427}{Splen obstructus.}
\setmarginnote{2428}{De arte med. lib. 3. cap. 24.}
\setmarginnote{2429}{A sanguinis putredine in vasis seminariis et utero, et quandoque a spermate diu retento, vel sanguine menstruo in melancholiam verso per putrefactionem, vel adustionem.}
\setmarginnote{2430}{Magirus.}
\setmarginnote{2431}{Ergo efficiens causa melancholiae est calida et sicca intemperies, non frigida et sicca, quod multi opinati sunt, oritur enim a calore celebri assante sanguinem, \&c. tum quod aromata sanguinem incendunt, solitudo, vigiliae, febris praecedens, meditatio, studium, et haec omnia calefaciunt, ergo ratum sit, \&c.}
\setmarginnote{2432}{Lib. 1. cap. 13. de Melanch.}
\setmarginnote{2433}{Lib. 3. Tract. posthum. de melan.}
\setmarginnote{2434}{A fatuitate inseparabilis cerebri frigiditas.}
\setmarginnote{2435}{Ab interno calore assatur.}
\setmarginnote{2436}{Intemperies innata exurens. flavam bilem ac sanguinem in melancholiam convertens.}
\setmarginnote{2437}{Si cerebrum sit calidius, fiet spiritus animales calidior, et dilirium maniacum; si frigidior, fie fatuitas.}
\setmarginnote{2438}{Melancholia capitis accedit post phrenesim aut longam moram sub sole, aut percussionem in capite, cap. 13. lib. 1.}
\setmarginnote{2439}{Qui bibunt vina potentia, et saepe sunt sub sole.}
\setmarginnote{2440}{Curae validae, largioris vini et aromatum usus.}
\setmarginnote{2441}{A cauterio et ulcere exsiccato.}
\setmarginnote{2442}{Ab ulcere curato incidit in insaniam, aperto vulnere curatur.}
\setmarginnote{2443}{A galea nimis calefacta.}
\setmarginnote{2444}{Exuritur sanguis et venae obstruuntur, quibus obstructis prohibetur transitus Chili ad jecur, corrumpitur et in rugitus et flatus vertitur.}
\setmarginnote{2445}{Stomacho laeso robur corporis imminuitur, et reliqua membra alimento orbata, \&c.}
\setmarginnote{2446}{Hildesheim.}
\setmarginnote{2447}{Habuit saeva animi symptomata quae impediunt concoctionem, \&c.}
\setmarginnote{2448}{Usitatissimus morbus cum sit, utile est hujus visceris accidentia considerare, nec leve periculum hujus causas morbi ignorantibus.}
\setmarginnote{2449}{Jecur aptum ad generandum talem humorem, splen natura imbecillior. Piso, Altomarus, Guianerius.}
